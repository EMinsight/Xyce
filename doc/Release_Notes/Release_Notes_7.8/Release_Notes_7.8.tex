% Sandia National Laboratories is a multimission laboratory managed and
% operated by National Technology & Engineering Solutions of Sandia, LLC, a
% wholly owned subsidiary of Honeywell International Inc., for the U.S.
% Department of Energy’s National Nuclear Security Administration under
% contract DE-NA0003525.

% Copyright 2002-2023 National Technology & Engineering Solutions of Sandia,
% LLC (NTESS).

\documentclass[letterpaper]{scrartcl}
\usepackage[hyperindex=true, colorlinks=false]{hyperref}
\usepackage{ltxtable, multirow}
\usepackage{Xyce}
\usepackage{geometry}
\usepackage{eso-pic}

\pdfcatalog {/PageMode /UseNone}
\renewcommand{\arraystretch}{1.2}

% Sets the page margins to be the same as the Guides (SAND reports)
\geometry{pdftex, inner=1in, textwidth=6.5in, textheight=9in}

% Gets rid of Section numbers
\setcounter{secnumdepth}{0}

% Set this here once, and use \XyceVersionVar{} in the document
\XyceVersion{7.8}

% ---------------------------------------------------------------------------- %
%
% Set the title, author, and date
%
\title{\XyceTitle{} Parallel Electronic Simulator\\
Version \XyceVersionVar{} Release Notes}

\author{ Sandia National Laboratories}

\date{\today}

% ---------------------------------------------------------------------------- %
% Start the document

\begin{document}
\maketitle

\AddToShipoutPictureBG*{%
  \AtPageUpperLeft{%
    \hspace{\dimexpr\paperwidth-1in\relax}%
    \raisebox{-0.5in}{%
         \makebox[0pt][r]{\textsf{SAND2023-13932R}}%
}}}

The \XyceTM{} Parallel Electronic Simulator has been written to support the
simulation needs of Sandia National Laboratories' electrical designers.
\XyceTM{} is a SPICE-compatible simulator with the ability to solve extremely
large circuit problems on large-scale parallel computing platforms, but also
includes support for most popular parallel and serial computers.

For up-to-date information not available at the time these notes were produced,
please visit the \XyceTM{} web page at
{\color{XyceDeepRed}\url{http://xyce.sandia.gov}}.

\tableofcontents
\vspace*{\fill}
\parbox{\textwidth}
{
  \raisebox{0.13in}{\includegraphics[height=0.5in]{snllineblubrd}}
  \hfill
  \includegraphics[width=1.5in]{xyce_flat_white}
}


\newpage
\section{New Features and Enhancements}

\subsubsection*{XDM}
\begin{XyceItemize}
\item XDM did not change in this release, so new installers were not generated.
\end{XyceItemize}

\subsubsection*{New Devices and Device Model Improvements}
\begin{XyceItemize}
\item The PSP103 model has been modified so that the DTA parameter can be
     specified on the instance line.  Previously it was exclusively a model
     parameter.
\end{XyceItemize}

\subsubsection*{Enhanced Solver Stability, Performance and Features}
\begin{XyceItemize}
\item  \Xyce{} can now handle the sources that involve transient specifications
     during source stepping. Source stepping is one of the solver techniques
     that \Xyce{} automatically attempts when trying to solve the DC operating
     point (DCOP). Previously, source stepping would only apply to DC sources
     that had no transient specifications. It now applies to sources that
     involve transient specifications, with or without DC specifications. When
     a source has inconsistent DC and transient specifications, the value used
     in the source stepping depends on the analysis type. See the \Xyce{} Users
     Guide for details. This leads to improved robustness and convergence of
     source stepping in circuits using these sources for the DCOP calculation.

\item \Xyce{} now provides a sequential source stepping algorithm that will
     ramp up sources with non-zero DC value in sequence, which can be more
     robust than the default simultaneous source stepping algorithm.  This new
     source stepping approach can be called directly using \texttt{.options
     nonlin continuation=35} or \texttt{.options nonlin
     continuation=sourcestep2}.  It will not replace simultaneous source
     stepping in the default DCOP strategy at this time.

\end{XyceItemize}

\subsubsection*{Interface Improvements}
\begin{XyceItemize}
\item \Xyce{} can now handle multiple \texttt{.OPTIONS} statements for the same
     keyword.   So, for example, if the netlist contains more than one
     \texttt{.OPTIONS TIMEINT} command, \Xyce{} will combine all the statements
     into a single, combined \texttt{.OPTIONS TIMEINT} statement.  Previously,
     Xyce's behavior in this respect was undefined.  For example, if the
     netlist contained multiple \texttt{.options timeint} statements, \Xyce{}
     would only use one of them and silently ignore the others.  \Xyce{} now
     treats multiple statements as a combined single statement, consistent with
     other simulators, and issues warnings for duplicate parameters.

\item The \texttt{.GLOBAL} statement now supports multiple global nodes on the
     same line.  Previously each specified node needed its own unique
     \texttt{.GLOBAL} statement.

\end{XyceItemize}

\subsubsection*{Important Announcements}
\begin{XyceItemize}
\item The model interpolation technique previously described in the \Xyce{}
  Reference Guide has been removed from Xyce.
\end{XyceItemize}

\newpage
\section{Defects Fixed in this Release}
% Sandia National Laboratories is a multimission laboratory managed and
% operated by National Technology & Engineering Solutions of Sandia, LLC, a
% wholly owned subsidiary of Honeywell International Inc., for the U.S.
% Department of Energy's National Nuclear Security Administration under
% contract DE-NA0003525.

% Copyright 2002-2024 National Technology & Engineering Solutions of Sandia,
% LLC (NTESS).

%%
%% Fixed Defects.
%%
{
\small

\begin{longtable}[h] {>{\raggedright\small}m{2in}|>{\raggedright\let\\\tabularnewline\small}m{3.5in}}
    \caption{Fixed Defects.  The Xyce team has multiple issue
     trackers, and the table below indicates fixed issues by
     indentifying both the tracker and the issue number.  Further,
     some issues are reported by open source users on GitHub and these
     issues may be tracked using multiple issue numbers.} \\ \hline
     \rowcolor{XyceDarkBlue} \color{white}\textbf{Defect} & \color{white}\textbf{Description} \\ \hline
     \endfirsthead
     \caption[]{Fixed Defects.  Note that we have two multiple issue tracking systems for Sandia Users.
     SON and SRN refer to our legacy open- and restricted-network Bugzilla system, and Gitlab refers to issues in our gitlab repositories.  } \\ \hline
     \rowcolor{XyceDarkBlue} \color{white}\textbf{Defect} & \color{white}\textbf{Description} \\ \hline
     \endhead

  \textbf{Xyce Project Backlog/xxx}: Description
  &  Details
  \\\hline

\textbf{Xyce Project Backlog/613}: 
  Noise  comma-separated values (CSV) output format should put quotes around variable/column names
  & Comma-separated values (CSV) output for noise had a bug, in that the comma character was used as a separator between variables, but for noise analysis variable names often include commas.  This has been fixed in two ways.  One is to add quotes around variable names in the CSV noise output file header.  The other is to expand the \texttt{DELIMITER} parameter on the \texttt{PRINT} line to allow other options, including \texttt{SEMICOLON} and \texttt{COLON}.  Also, this parameter previously applied only to \texttt{STD} format files.  It now also applies to \texttt{CSV} format files.  \\ \hline

\textbf{Xyce Project Backlog/694}: 
Error in MOSFET numerical sensitivities, when using \texttt{.option scale}
  & When using sensitivity analysis, one aspect of the calculation is device-level derivatives.  For some device models, analytic derivatives are available.  When they are not available, numerical device-level derivatives must be used.  This is the case with several older MOSFET models, such as the BSIM4.  The code for computing these device-level derivatives did not handle length scalars (set by the netlist command \texttt{.option scale} properly, so this produced at best nonsense results and at worst caused the code to exit with a fatal error.  This has been fixed. \\ \hline

\textbf{Xyce Project Backlog/729}: 
Expression-based objective functions for .SENS conflate I(VA) with V(VA) (nodes and devices w/ same names)
& Sensitivity analysis had a bug related to the naming of circuit variables in the objective function.
If a circuit had a voltage source and a voltage node of the same name, the expression capability could not differentiate between the two, and could choose the wrong one.  This has been fixed. \\ \hline

\textbf{Xyce Project Backlog/740}: Adjoint, output, and user-provided breakpoints present a scaling issue for
extremely long lists & A user can request output using \texttt{.options output}
in the form of a list of time points.  If this list was very long (10k+ values),
then processing the list would result in a substantial slowdown in the setup time of the 
simulation.  This has been fixed. \\ \hline

\textbf{Xyce Project Backlog/767}: Location of device model repositories 
should be optional & When building Xyce from source with CMake, the 
location of optional device model packages can now be specifed as 
configuration parameters. The new parameter options and their 
associated default values are given below as \texttt{parameter name} 
\texttt{(default value)}. \newline
\texttt{Xyce\_ADMS\_MODELS\_DIR} \texttt{(\$\{PROJECT\_SOURCE\_DIR\}/src/DeviceModelPKG/ADMS)} \newline
\texttt{Xyce\_NEURON\_MODELS\_DIR} \texttt{(\$\{PROJECT\_SOURCE\_DIR\}/src/DeviceModelPKG/NeuronModels)} \newline
\texttt{Xyce\_NONFREE\_MODELS\_DIR} \texttt{(\$\{PROJECT\_SOURCE\_DIR\}/src/DeviceModelPKG/Xyce\_NonFree)} \newline
\texttt{Xyce\_RAD\_MODELS\_DIR} \texttt{(\$\{PROJECT\_SOURCE\_DIR\}/src/DeviceModelPKG/SandiaModels)} \newline
\\ \hline

\textbf{Xyce Project Backlog/752}: Enable Hysteresis in Switch Devices.
&  Hysteresis in the on and off behavior of voltage, current and general 
expression based switches has been added to the switch device.  See the 
reference guide for further details. 
\\\hline


\end{longtable}
}


\newpage
\section{Supported Platforms}
\subsection*{Certified Support}
The following platforms have been subject to certification testing for the
\Xyce{} version 7.8 release.
\begin{XyceItemize}
  \item Red Hat Enterprise Linux${}^{\mbox{\textregistered}}$ 7, x86-64 (serial and parallel)
  \item Microsoft Windows 11${}^{\mbox{\textregistered}}$, x86-64 (serial)
  \item Apple${}^{\mbox{\textregistered}}$ macOS, x86-64 (serial and parallel)
\end{XyceItemize}


\subsection*{Build Support}
Though not certified platforms, \Xyce{} has been known to run on the following
systems.
\begin{XyceItemize}
  \item FreeBSD 12.X on Intel x86-64 and AMD64 architectures (serial
    and parallel)
  \item Distributions of Linux other than Red Hat Enterprise Linux 6
  \item Microsoft Windows under Cygwin and MinGW
\end{XyceItemize}


\section{\Xyce{} Release \XyceVersionVar{} Documentation}
The following \Xyce{} documentation is available on the \Xyce{} website in pdf
form.
\begin{XyceItemize}
  \item \Xyce{} Version \XyceVersionVar{} Release Notes (this document)
  \item \Xyce{} Users' Guide, Version \XyceVersionVar{}
  \item \Xyce{} Reference Guide, Version \XyceVersionVar{}
  \item \Xyce{} Mathematical Formulation
  \item Power Grid Modeling with \Xyce{}
  \item Application Note: Coupled Simulation with the \Xyce{} General
    External Interface
  \item Application Note: Mixed Signal Simulation with \Xyce{} 7.2
\end{XyceItemize}
Also included at the \Xyce{} website as web pages are the following.
\begin{XyceItemize}
  \item Frequently Asked Questions
  \item Building Guide (instructions for building \Xyce{} from the source code)
  \item Running the \Xyce{} Regression Test Suite
  \item \Xyce{}/ADMS Users' Guide
  \item Tutorial:  Adding a new compact model to \Xyce{}
\end{XyceItemize}


\section{External User Resources}
\begin{itemize}
  \item Website: {\color{XyceDeepRed}\url{http://xyce.sandia.gov}}
  \item Google Groups discussion forum:
    {\color{XyceDeepRed}\url{https://groups.google.com/forum/#!forum/xyce-users}}
  \item Email support:
    {\color{XyceDeepRed}\href{mailto:xyce@sandia.gov}{xyce@sandia.gov}}
  \item Address:
    \begin{quote}
            Electrical Models and Simulation Dept.\\
            Sandia National Laboratories\\
            P.O. Box 5800, M.S. 1177\\
            Albuquerque, NM 87185-1177 \\
    \end{quote}
\end{itemize}

\vspace*{\fill}
\noindent
Sandia National Laboratories is a multimission laboratory managed and
operated by National Technology and Engineering Solutions of Sandia,
LLC, a wholly owned subsidiary of Honeywell International, Inc., for
the U.S. Department of Energy's National Nuclear Security
Administration under contract DE-NA0003525.

\end{document}

