% Sandia National Laboratories is a multimission laboratory managed and
% operated by National Technology & Engineering Solutions of Sandia, LLC, a
% wholly owned subsidiary of Honeywell International Inc., for the U.S.
% Department of Energy's National Nuclear Security Administration under
% contract DE-NA0003525.

% Copyright 2002-2023 National Technology & Engineering Solutions of Sandia,
% LLC (NTESS).

% Sandia National Laboratories is a multimission laboratory managed and
% operated by National Technology & Engineering Solutions of Sandia, LLC, a
% wholly owned subsidiary of Honeywell International Inc., for the U.S.
% Department of Energy's National Nuclear Security Administration under
% contract DE-NA0003525.

% Copyright 2002-2023 National Technology & Engineering Solutions of Sandia,
% LLC (NTESS).


%%
%% Fixed Defects.
%%
{
\small

\begin{longtable}[h] {>{\raggedright\small}m{2in}|>{\raggedright\let\\\tabularnewline\small}m{3.5in}}
     \caption{Fixed Defects.  The Xyce team has multiple issue
     trackers, and the table below indicates fixed issues by
     indentifying both the tracker and the issue number.  Further,
     some issues are reported by open source users on GitHub and these
     issues may be tracked using multiple issue numbers.} \\ \hline
     \rowcolor{XyceDarkBlue} \color{white}\textbf{Defect} & \color{white}\textbf{Description} \\ \hline
     \endfirsthead
     \caption[]{Fixed Defects.  Note that we have two multiple issue tracking systems for Sandia Users.
     SON and SRN refer to our legacy open- and restricted-network Bugzilla system, and Gitlab refers to issues in our gitlab repositories.  } \\ \hline
     \rowcolor{XyceDarkBlue} \color{white}\textbf{Defect} & \color{white}\textbf{Description} \\ \hline
     \endhead

  \textbf{Xyce Project Backlog/xxx}: Desciption
  &  Details
  \\\hline

  \textbf{Xyce Project Backlog/571}: Modify \Xyce{} parser so that it can handle multiple \texttt{.options} statements for the same component (for example, support \texttt{.options timeint} statements) 
  &  Most SPICE-style simulators use the \texttt{.option} (non-plural) command, and allow the netlist to have multiple \texttt{.option} comamnds.   \Xyce{} historically was different in two respects.  One difference is that \Xyce{} uses a plural \texttt{.options} command, and another difference is a required keyword to identify the part of the code to which the block of options is applied.  Also, finally, \Xyce{} was designed to only have a single \texttt{.options} statement of each type.  So, for example, at most one \texttt{.options device} statement, at most one \texttt{.options timeint} statement, etc.  Unforutunately, recent versions of \Xyce{} silently accepted multiple \texttt{.options} statements, and behaved differently depending on the keyword.  For example, if the netlist contained multiple \texttt{.options timeint} statements, \Xyce{} would only use one of them and silently ignore the others.  In contrast, if the netlist contained multiple \texttt{.options device} statements, it would use them all, but would ignore duplicate parameters.  This was confusing for users of other simulators, and has been fixed.  \Xyce{} now treats multiple statements as a combined single statement, and issues warnings for duplicate parameters.
  \\\hline



\end{longtable}
}
