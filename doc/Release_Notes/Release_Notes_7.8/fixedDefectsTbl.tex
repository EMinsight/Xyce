% Sandia National Laboratories is a multimission laboratory managed and
% operated by National Technology & Engineering Solutions of Sandia, LLC, a
% wholly owned subsidiary of Honeywell International Inc., for the U.S.
% Department of Energy's National Nuclear Security Administration under
% contract DE-NA0003525.

% Copyright 2002-2023 National Technology & Engineering Solutions of Sandia,
% LLC (NTESS).

% Sandia National Laboratories is a multimission laboratory managed and
% operated by National Technology & Engineering Solutions of Sandia, LLC, a
% wholly owned subsidiary of Honeywell International Inc., for the U.S.
% Department of Energy's National Nuclear Security Administration under
% contract DE-NA0003525.

% Copyright 2002-2023 National Technology & Engineering Solutions of Sandia,
% LLC (NTESS).


%%
%% Fixed Defects.
%%
{
\small

\begin{longtable}[h] {>{\raggedright\small}m{2in}|>{\raggedright\let\\\tabularnewline\small}m{3.5in}}
     \caption{Fixed Defects.  The Xyce team has multiple issue
     trackers, and the table below indicates fixed issues by
     indentifying both the tracker and the issue number.  Further,
     some issues are reported by open source users on GitHub and these
     issues may be tracked using multiple issue numbers.} \\ \hline
     \rowcolor{XyceDarkBlue} \color{white}\textbf{Defect} & \color{white}\textbf{Description} \\ \hline
     \endfirsthead
     \caption[]{Fixed Defects.  Note that we have two multiple issue tracking systems for Sandia Users.
     SON and SRN refer to our legacy open- and restricted-network Bugzilla system, and Gitlab refers to issues in our gitlab repositories.  } \\ \hline
     \rowcolor{XyceDarkBlue} \color{white}\textbf{Defect} & \color{white}\textbf{Description} \\ \hline
     \endhead

  \textbf{Xyce Project Backlog/xxx}: Desciption
  &  Details
  \\\hline

\textbf{Xyce Backlog Bugs/64}: LTRA device does not properly initialize history vector when UIC/NOOP is used &
  The lossy transmission line device (LTRA) was not bein intialized correctly,
  for transient simulations that skipped the DCOP calculation.  As a result,
  transient simulations using this device would crash if the netlist contained
  a \texttt{UIC} or \texttt{NOOP} keyword on the \texttt{.TRAN} line.  This has
  been fixed.  \\ \hline

  \textbf{Xyce Project Backlog/571}: Modify \Xyce{} parser so that it can
  handle multiple \texttt{.options} statements for the same component (for
  example, support \texttt{.options timeint} statements) &
  Most SPICE-style simulators use the \texttt{.option} (non-plural) command,
  and allow the netlist to have multiple \texttt{.option} comamnds.   \Xyce{}
  historically was different in two respects.  One difference is that \Xyce{}
  uses a plural \texttt{.options} command, and another difference is a required
  keyword to identify the part of the code to which the block of options is
  applied.  Also, finally, \Xyce{} was designed to only have a single
  \texttt{.options} statement of each type.  So, for example, at most one
  \texttt{.options device} statement, at most one \texttt{.options timeint}
  statement, etc.  Unforutunately, recent versions of \Xyce{} silently accepted
  multiple \texttt{.options} statements, and behaved differently depending on
  the keyword.  For example, if the netlist contained multiple \texttt{.options
  timeint} statements, \Xyce{} would only use one of them and silently ignore
  the others.  In contrast, if the netlist contained multiple \texttt{.options
  device} statements, it would use them all, but would ignore duplicate
  parameters.  This was confusing for users of other simulators, and has been
  fixed.  \Xyce{} now treats multiple statements as a combined single
  statement, and issues warnings for duplicate parameters.
  \\\hline
 
 \textbf{Xyce Project Backlog/609}: 
  Global parameters that are applied to subcircuit instance parameters don't work in parallel & 
  If a subcircuit instance parameter (on the \texttt{X} line) depended on a
  \texttt{.param} or \texttt{.global\_param}, and that parameter was a
  variable, this dependency worked in serial, but not parallel.  In this case,
  ``variable'' means parameters that are allowed to change as part of a
  \texttt{.STEP} sweep, or as part of a sampling method.  This was related to
  the order of operations in the parser.  This has been fixed.  \\ \hline

 \textbf{Xyce Project Backlog/631}: 
  Expression-based local variation doesn't always work with subcircuit arguments & 
  When expression-based local variation was applied to a subcircuit argument, it 
  was handled incorrectly.  This meant that subcircuit arguments that should have 
  had a random distribution were fixed to the mean value during sampling.  
  This has been fixed.  \\ \hline

  \textbf{Xyce Project Backlog/637}: Support shared library build of \Xyce{} under Windows.
  &  Modify the cmake build system to support building a shared library version
  of xyce, \texttt{xyce-share.dll}, so that Xyce functionality can be accessed
  as a library from other applications.  \\\hline


\end{longtable}
}
