% Sandia National Laboratories is a multimission laboratory managed and
% operated by National Technology & Engineering Solutions of Sandia, LLC, a
% wholly owned subsidiary of Honeywell International Inc., for the U.S.
% Department of Energy's National Nuclear Security Administration under
% contract DE-NA0003525.

% Copyright 2002-2022 National Technology & Engineering Solutions of Sandia,
% LLC (NTESS).

% Sandia National Laboratories is a multimission laboratory managed and
% operated by National Technology & Engineering Solutions of Sandia, LLC, a
% wholly owned subsidiary of Honeywell International Inc., for the U.S.
% Department of Energy's National Nuclear Security Administration under
% contract DE-NA0003525.

% Copyright 2002-2022 National Technology & Engineering Solutions of Sandia,
% LLC (NTESS).


%%
%% Fixed Defects.
%%
{
\small

\begin{longtable}[h] {>{\raggedright\small}m{2in}|>{\raggedright\let\\\tabularnewline\small}m{3.5in}}
     \caption{Fixed Defects.  The Xyce team has multiple issue
     trackers, and the table below indicates fixed issues by
     indentifying both the tracker and the issue number.  Further,
     some issues are reported by open source users on GitHub and these
     issues may be tracked using multiple issue numbers.} \\ \hline
     \rowcolor{XyceDarkBlue} \color{white}\textbf{Defect} & \color{white}\textbf{Description} \\ \hline
     \endfirsthead
     \caption[]{Fixed Defects.  Note that we have two multiple issue tracking systems for Sandia Users.
     SON and SRN refer to our legacy open- and restricted-network Bugzilla system, and Gitlab refers to issues in our gitlab repositories.  } \\ \hline
     \rowcolor{XyceDarkBlue} \color{white}\textbf{Defect} & \color{white}\textbf{Description} \\ \hline
     \endhead

  \textbf{Xyce Project Backlog/106}: Xyce/ADMS generates incorrect
  derivative code for integer variables & Xyce/ADMS would emit code
  that would not compile if a Verilog-A module tried to assign a
  probe-dependent expression into an integer variable.  \\\hline

  \textbf{Xyce Project Backlog/157}: Xyce diode does not support
  sidewall effects & The Xyce diode now supports an instance parameter
  for the junction perimeter (\texttt{PJ}) and a number of model
  parameters
  (\texttt{JSW}, \texttt{NS}, \texttt{CJSW}, \texttt{PHP}, \texttt{MJSW}, \texttt{FCS})
  supporting the effects.  Default values are such that the feature is
  disabled.  This brings the Xyce level 1 and 2 diodes into agreement
  with other simulators when these parameters are
  specified.  \\ \hline
  
  \textbf{Xyce Project Backlog/492}: Subcircuit multiliplier
  parameters (\texttt{M}) need to be handled as an AST (enabling .STEP
  to be applied to subcircuit \texttt{M}) & This task was related to
  backlog isssue 494.  In the initial implementation of subcircuit
  multipliers, they were constrained to have fixed values.  However,
  in some cases a user may wish to change multiplier values during a
  simulation.  To enable this feature, the subcircuit multiplier
  parameters had to be part of the same abstract syntax tree (AST) as
  other parameters from the netlist.  This has been fixed.  \\ \hline

  \textbf{Xyce Project Backlog/494}: Remove parser error trap for
  using \texttt{.global\_param} on the implicit subcircuit multiplier
  parameter (M) & The subcircuit multiplier parameter, \texttt{M}, is
  treated as a special case in the \Xyce{} parser.  Due to its special
  treatment, the use case of setting this parameter via a mutable
  parameter (i.e.  \texttt{.global\_param} ) did not work correctly.
  This has been fixed. This issue did not apply to device multipliers,
  just to subcircuit multipliers.  \\ \hline

  \textbf{Xyce Project Backlog/527}: Xyce/ADMS aborts on analog
  functions that have arguments of a type different from their return
  types & When Xyce/ADMS was refactored to remove use of Sacado, a bit
  of templated code was not rewritten because it appeared to be good
  enough for the new implementation.  But that version had a
  restriction that all arguments of an analog function must have the
  same type as the return value.  This code has now been
  rewritten. Analog functions can now return either reals or integers,
  and may have arguments of either type irrespective of their return
  type.  \\ \hline

  \textbf{Xyce Project Bugs/36}: Xyce/ADMS aborts when an analog
  function calls \$strobe, \$bound\_step, or other ``callfunction'' &
  Due to the way Xyce/ADMS implements \$strobe, \$bound\_step, and
  other statements that look like function calls and how it implements
  analog functions themselves, these ``callfunctions'' cannot be used
  inside analog functions.  Until this release, trying to
  call \$strobe from inside an analog function would cause ADMS to
  abort with a fatal error about a missing template.  Now it will emit
  a warning that the usage is not implemented, but will simply ignore
  the callfunction and generate otherwise functional code. \\ \hline

  \textbf{Xyce Project Bugs/35}: Xyce/ADMS does not allow ceil and
  floor in analog functions & A mistake in the code for generating
  analog function derivatives caused any analog function that used
  ceil or floor to abort processing of an entire Verilog-A model.
  This mistake has been fixed and now ceil and floor can be used
  inside analog functions \\ \hline

  \textbf{Xyce Project Bugs/27}: Transient adjoint sensitivity doesn't
  correctly handle scaling & Sensitivity analysis in \Xyce{} can
  optionally scale sensitivity values by \texttt{p/100},
  where \texttt{p} is the original value of the sensitivity parameter.
  This particular capability was never implemented for transient
  adjoints, which was an oversight.  This has been fixed.  \\ \hline

  \textbf{Xyce Project Backlog/534}: Xyce needs to remove deprecated
  features in C++17, like std::binary\_function() and std::unary\_function(). &
  These features have been replaced with C++17 compliant features. \\ \hline

  \textbf{Xyce Project Backlog/565}: Fix VPWL/IPWL independent sources to work 
  with \texttt{.STEP} & Piecewise-linear sources were not instrumented to be 
  updated during \texttt{.STEP} loops.  This has been fixed.  \\ \hline

  \textbf{Xyce Project Backlog/566}:  Expression library breakpoint handling has 
  a problem with \texttt{.STEP} & The breakpoint functions in the expression 
  library were not properly reset at the beginning of each \texttt{.STEP} 
  iteration.  This was causing breakpoints to be overlooked early in the transient.
  This has been fixed.  \\ \hline
\end{longtable}
}
