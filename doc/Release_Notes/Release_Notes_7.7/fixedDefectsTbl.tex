% Sandia National Laboratories is a multimission laboratory managed and
% operated by National Technology & Engineering Solutions of Sandia, LLC, a
% wholly owned subsidiary of Honeywell International Inc., for the U.S.
% Department of Energy's National Nuclear Security Administration under
% contract DE-NA0003525.

% Copyright 2002-2023 National Technology & Engineering Solutions of Sandia,
% LLC (NTESS).

% Sandia National Laboratories is a multimission laboratory managed and
% operated by National Technology & Engineering Solutions of Sandia, LLC, a
% wholly owned subsidiary of Honeywell International Inc., for the U.S.
% Department of Energy's National Nuclear Security Administration under
% contract DE-NA0003525.

% Copyright 2002-2023 National Technology & Engineering Solutions of Sandia,
% LLC (NTESS).


%%
%% Fixed Defects.
%%
{
\small

\begin{longtable}[h] {>{\raggedright\small}m{2in}|>{\raggedright\let\\\tabularnewline\small}m{3.5in}}
     \caption{Fixed Defects.  The Xyce team has multiple issue
     trackers, and the table below indicates fixed issues by
     indentifying both the tracker and the issue number.  Further,
     some issues are reported by open source users on GitHub and these
     issues may be tracked using multiple issue numbers.} \\ \hline
     \rowcolor{XyceDarkBlue} \color{white}\textbf{Defect} & \color{white}\textbf{Description} \\ \hline
     \endfirsthead
     \caption[]{Fixed Defects.  Note that we have two multiple issue tracking systems for Sandia Users.
     SON and SRN refer to our legacy open- and restricted-network Bugzilla system, and Gitlab refers to issues in our gitlab repositories.  } \\ \hline
     \rowcolor{XyceDarkBlue} \color{white}\textbf{Defect} & \color{white}\textbf{Description} \\ \hline
     \endhead

  \textbf{Xyce Project Backlog/106}: Xyce/ADMS generates incorrect
  derivative code for integer variables & Xyce/ADMS would emit code
  that would not compile if a Verilog-A module tried to assign a
  probe-dependent expression into an integer variable.  \\\hline

  \textbf{Xyce Project Backlog/157}: Xyce diode does not support
  sidewall effects & The Xyce diode now supports an instance parameter
  for the junction perimeter (\texttt{PJ}) and a number of model
  parameters
  (\texttt{JSW}, \texttt{NS}, \texttt{CJSW}, \texttt{PHP}, \texttt{MJSW}, \texttt{FCS})
  supporting the effects.  Default values are such that the feature is
  disabled.  This brings the Xyce level 1 and 2 diodes into agreement
  with other simulators when these parameters are
  specified.  \\ \hline
  
  \textbf{Xyce Project Backlog/492}: Subcircuit multiliplier
  parameters (\texttt{M}) need to be handled as an AST (enabling .STEP
  to be applied to subcircuit \texttt{M}) & This task was related to
  backlog isssue 494.  In the initial implementation of subcircuit
  multipliers, they were constrained to have fixed values.  However,
  in some cases a user may wish to change multiplier values during a
  simulation.  To enable this feature, the subcircuit multiplier
  parameters had to be part of the same abstract syntax tree (AST) as
  other parameters from the netlist.  This has been fixed.  \\ \hline

  \textbf{Xyce Project Backlog/494}: Remove parser error trap for
  using \texttt{.global\_param} on the implicit subcircuit multiplier
  parameter (M) & The subcircuit multiplier parameter, \texttt{M}, is
  treated as a special case in the \Xyce{} parser.  Due to its special
  treatment, the use case of setting this parameter via a mutable
  parameter (i.e.  \texttt{.global\_param} ) did not work correctly.
  This has been fixed. This issue did not apply to device multipliers,
  just to subcircuit multipliers.  \\ \hline

  \textbf{Xyce Project Backlog/527}: Xyce/ADMS aborts on analog
  functions that have arguments of a type different from their return
  types & When Xyce/ADMS was refactored to remove use of Sacado, a bit
  of templated code was not rewritten because it appeared to be good
  enough for the new implementation.  But that version had a
  restriction that all arguments of an analog function must have the
  same type as the return value.  This code has now been
  rewritten. Analog functions can now return either reals or integers,
  and may have arguments of either type irrespective of their return
  type.  \\ \hline

  \textbf{Xyce Project Bugs/36}: Xyce/ADMS aborts when an analog
  function calls \$strobe, \$bound\_step, or other ``callfunction'' &
  Due to the way Xyce/ADMS implements \$strobe, \$bound\_step, and
  other statements that look like function calls and how it implements
  analog functions themselves, these ``callfunctions'' cannot be used
  inside analog functions.  Until this release, trying to
  call \$strobe from inside an analog function would cause ADMS to
  abort with a fatal error about a missing template.  Now it will emit
  a warning that the usage is not implemented, but will simply ignore
  the callfunction and generate otherwise functional code. \\ \hline

  \textbf{Xyce Project Bugs/35}: Xyce/ADMS does not allow ceil and
  floor in analog functions & A mistake in the code for generating
  analog function derivatives caused any analog function that used
  ceil or floor to abort processing of an entire Verilog-A model.
  This mistake has been fixed and now ceil and floor can be used
  inside analog functions \\ \hline

  \textbf{Xyce Project Bugs/27}: Transient adjoint sensitivity doesn't
  correctly handle scaling & Sensitivity analysis in \Xyce{} can
  optionally scale sensitivity values by \texttt{p/100},
  where \texttt{p} is the original value of the sensitivity parameter.
  This particular capability was never implemented for transient
  adjoints, which was an oversight.  This has been fixed.  \\ \hline

  \textbf{Xyce Project Backlog/534}: Xyce needs to remove deprecated
  features in C++17, like std::binary\_function() and std::unary\_function(). &
  These features have been replaced with C++17 compliant features. \\ \hline

  \textbf{Xyce Project Backlog/565}: Fix VPWL/IPWL independent sources to work 
  with \texttt{.STEP} & Piecewise-linear sources were not instrumented to be 
  updated during \texttt{.STEP} loops.  This has been fixed.  \\ \hline

  \textbf{Xyce Project Backlog/566}:  Expression library breakpoint handling has 
  a problem with \texttt{.STEP} & The breakpoint functions in the expression 
  library were not properly reset at the beginning of each \texttt{.STEP} 
  iteration.  This was causing breakpoints to be overlooked early in the transient.
  This has been fixed.  \\ \hline


  \textbf{Xyce Project Backlog/575}: 
   Expression tables are inefficient when they are large and contain expressions & 
   This issue was inspired by backlog issues 565 and 566.  If a PWL source is 
   specified using a Bsrc and the \texttt{TABLE} operator (instead of a PWL 
   source), then it was very 
   inefficient if the table was large and the entries of that table were 
   based on expressions rather than pure numbers.  This was because the subordinate 
   expressions in the table were being updated too frequently.
  This has been fixed.  \\ \hline

  \textbf{Xyce Project Bugs/41}: Expression library inconsistently handles 
  subcircuit nodes for derivatives &  When invoked from a Bsrc, the expression 
  library has to provide derivatives as well as function evaluations.  In the 
  Bsrc, the needed derivatives are with respect to solution variables, which 
  are usually voltage nodes.  When a Bsrc is inside of a subcircuit, it is 
  possible that the original expression contains voltage node dependencies 
  that will need to be replaced with the subcircuit instance nodes.  When 
  this happens, the new list of nodes might include a duplicate or ground node, 
  even if the original version did not.  The expression library wasn't 
  handling this use case (new duplicates) correctly, and this caused a 
  memory error.  This has been fixed.  \\ \hline

  \textbf{Xyce Project Bugs/54}: subcircuit parameters incorrectly 
  deleted when multiple subcircuit instances (X lines) refer to same 
  subcircuit definition &   There was a bug in the parser in the function
  that resolved subcircuit parameters.  When subcircuit instance
  parameters (from the X line) matched a parameter in the list 
  of unresolved parameters in the subcircuit definition, that parameter
  was incorrectly being erased from the definition container.
  This has been fixed.  \\ \hline

  \textbf{Xyce Project Backlog/147}: The handling of the first failed step out 
  of a DC op was incorrect & When Xyce cannot take the first transient step out
  of a DC op, it incorrectly reported that Xyce reaches the maximum local error
  test failures. Xyce does not check local truncation error for the first step.
  This has been fixed. Xyce now handles this case correctly and it also reports
  the time when Xyce reaches the maximum number of failures without
  convergence. \\ \hline   

 \textbf{Xyce Project Backlog/595}: 
   Expression library does not recognize device names that include square
    brackets and a bunch of other unusual (but supported) characters & 
   The expression library could not tokenize a device name such as \texttt{V\_B[0]}. 
   The use of brackets and other unusual characters was allowed in other tokens 
   such as voltage nodes.  For this bug only device name tokens, used in the expression 
   library, were affected.  This was an oversight in the expression library lexer code. 
   This has been fixed.  \\ \hline

  \textbf{Xyce Project Bugs/55}: Null pointer in expression &
  In the resolution of parameters in subcircuits, expressions
  could be deleted before they were stored in parameter objects
  resulting in a null pointer for the expression.  This issue
  has been fixed. \\ \hline
  
  
\textbf{Xyce Project Bugs/59}: Mixed-Signal Segmentation Fault &
  The Python interface to Xyce was incorrectly using an ADC 
  base name when trying to get DAC devices.  This has been fixed. \\ \hline

\textbf{Xyce Project Bugs/60}: Error building with Xyce\_GRAPH\_DEBUG=ON &
  The bracket operator used in a block of debugging code was incorrect
  for use on the std::unordered\_map object.  This code was fixed to use
  the correct accessor function for the map. \\ \hline

 \textbf{Xyce Project Backlog/611}: 
   The devConMap is not implemented in any of the BSIM CMG devices & 
   The devConMap is a data structure provided by each device model to enable 
   the topology package to perform the "no DC path to ground" diagnostic.
   The BSIMCMG devices did not have this set up, and so circuits using 
   these devices would sometimes produce spurious warning messages.  
   This has been fixed.  \\ \hline

 \textbf{Xyce Project Backlog/631}: 
  Expression-based local variation doesn't always work with subcircuit arguments & 
  When expression-based local variation was applied to a subcircuit argument, it 
  was handled incorrectly.  This meant that subcircuit arguments that should have 
  had a random distribution were fixed to the mean value during sampling.  
  This has been fixed.  \\ \hline

   \textbf{Xyce Project Bugs/61}: Circuit causes Segfault &
   \Xyce{} would get a segmentation fault when the netlist specified a Bsrc that depended on the current thru another Bsrc, that happened to be a current-source style Bsrc.  Behavioral sources can either be current or voltage sources, depending on the netlist specification.  When specified as a voltage source, they have an internal current variable, which is accessible by other behavioral sources.   When specified as a current source, they do not.  \Xyce{} was not catching this mistake early enough and this was resulting in a memory error.  This has been fixed. \\ \hline



\end{longtable}
}
