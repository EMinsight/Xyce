% Sandia National Laboratories is a multimission laboratory managed and
% operated by National Technology & Engineering Solutions of Sandia, LLC, a
% wholly owned subsidiary of Honeywell International Inc., for the U.S.
% Department of Energy's National Nuclear Security Administration under
% contract DE-NA0003525.

% Copyright 2002-2022 National Technology & Engineering Solutions of Sandia,
% LLC (NTESS).

% Sandia National Laboratories is a multimission laboratory managed and
% operated by National Technology & Engineering Solutions of Sandia, LLC, a
% wholly owned subsidiary of Honeywell International Inc., for the U.S.
% Department of Energy's National Nuclear Security Administration under
% contract DE-NA0003525.

% Copyright 2002-2022 National Technology & Engineering Solutions of Sandia,
% LLC (NTESS).


%%
%% Fixed Defects.
%%
{
\small

\begin{longtable}[h] {>{\raggedright\small}m{2in}|>{\raggedright\let\\\tabularnewline\small}m{3.5in}}
     \caption{Fixed Defects.  Note that we have multiple issue
     tracking systems for Sandia users.  SON, which bugzilla on the
     open network, and SRN, which is bugzilla on the restricted
     network.  We are also transitioning from bugzilla to gitlab issue
     tracking.  Further, some issues are reported by open source users
     on GitHub and these issues may be tracked using multiple issue
     numbers.} \\ \hline
     \rowcolor{XyceDarkBlue} \color{white}\textbf{Defect} & \color{white}\textbf{Description} \\ \hline
     \endfirsthead
     \caption[]{Fixed Defects.  Note that we have two multiple issue tracking systems for Sandia Users.
     SON and SRN refer to our legacy open- and restricted-network Bugzilla system, and Gitlab refers to issues in our gitlab repositories.  } \\ \hline
     \rowcolor{XyceDarkBlue} \color{white}\textbf{Defect} & \color{white}\textbf{Description} \\ \hline
     \endhead

\textbf{placeholder issue number}: placeholder description &
     placeholder explanation \\ \hline

\textbf{Gitlab-ex issue 220}:  .MEASURE trig/targ doesn't match ngspice
when there is negative setup time. & Previous \Xyce{} versions would
not return a negative value from a \texttt{TRIG-TARG} measure. The
\texttt{TARG} clause would only be evaluated if the \texttt{TRIG} clause
was satisfied.  A measure line such as this will now properly return
M4 = -0.5 with targ = 0.25 and trig = 0.75, instead of
failing to find the targ time.

\begin{verbatim}
VPWL2 2 0 pwl(0 0 0.5 1 1 0)
.MEASURE TRAN M4 TRIG V(2)=0.5 CROSS=2
+ TARG V(2)=0.5 CROSS=1
\end{verbatim}
 \\ \hline

 \textbf{Gitlab-ex issue 270}: Inconsistency in column headers for .PRINT SENS for TRAN/DC and AC &
  The format of the column headers in the sensitivity output files is different
  between AC and DC/TRAN analysis. An example of a sensitivity data column name
  in a DC analysis output file might look like this:\begin{verbatim}
  d{V(B)}/d(R1:R)_Dir\end{verbatim}
  while in an AC analysis output file the title of the sensitivity data might be
  \begin{verbatim}
  d_VR(B)/d_R1:R_dir\end{verbatim}
  Additionally, the complex and polar notation in the AC output implies that the
  objective is a voltage (using VR, VI, VM, VP). This change tries to make all
  the sensitivity column headers follow the same convention:
  \begin{verbatim}d_{expression}/d_parameter_dir\end{verbatim}
  (the suffix may be either {\_}dir or {\_}adj). The component notaion on the AC
  output has been updated to use Re({exp}), Im({exp}), Mag({exp}) and Ph({exp}).
  In working this issue, the processing logic for AC .SENS lines was also
  refactored. The objvars and acobjfunc components of a .SENS line are now
  handled by the same processing functions. With this change, the restriction
  against using both objective function specifiers on the same .SENS line was
  removed. \\ \hline

\textbf{Gitlab-ex issue 289} Improvements to TRIG-TARG measure &
Separate \texttt{TD} (time delay) and/or \texttt{AT} qualifiers
are now allowed for both the \texttt{TRIG} and \texttt{TARG}
clauses.  Expression support has also been improved.  Previously,
expressions did not work correctly in the \texttt{TARG} clause.
Examples are as follows:

\begin{verbatim}
.TRAN 0 1
.MEASURE TRAN M1 TRIG V(1)=0.5 CROSS=1 TD=0.3
+ TARG V(1)=0.5 CROSS=1 TD=0.8
.MEASURE TRAN M2 TRIG AT=0.05
+ TARG V(1)=0.5 CROSS=1
.MEASURE TRAN M3 TRIG V(1)=0.5
+ CROSS=1 TARG AT=0.8
\end{verbatim}

Xyce will now properly report M1 = 0.5 with targ = 0.875 and
trig = 0.375, M2 = 0.075 with targ = 0.125 and trig = 0.05, and
M3 = 0.675 with targ = 0.8 and trig = 0.125.
\\ \hline

\textbf{Gitlab-ex issue 304}: Change \Xyce{} behavior for failed measures &
\Xyce{} was using ``-1'' as the output value when any measure (from a
\texttt{.MEASURE} statement) failed.  This was inappropriate, because -1 is a
legal value for many measures, making it difficult to tell whether the
measure worked or not.  In \Xyce{} 7.4 an option \texttt{.options
measure MEASFAIL=1} that allowed a user to request that Xyce output
``FAILED'' for failed measures instead.  In \Xyce{} 7.5 this style of
output has been made the default.  The old behavior can be forced by
using \texttt{.options measure MEASFAIL=0 DEFAULT\_VAL=-1} if
desired. \\ \hline

\textbf{Gitlab-ex issue 306}:  Using expression random operators with sampling is too slow for large numbers of parameters &
For complicated netlists and PDKs, \Xyce{} was inefficient in
processing large numbers of uncertain parameters, when specified using
expression-based random operators such as \texttt{AGAUSS}.  There were
several compounding issues that caused this to happen, and they have
been corrected.  \\ \hline

\textbf{Gitlab-ex issue 318}:  Fix .PREPROCESS REPLACEGROUND to avoid
unnecessary replacement of ground synonyms & When using ``.PREPROCESS
REPLACEGROUND TRUE'' to replace ground synonyms, character strings
were being replaced on lines where it was not necessary, like
subcircuit definition lines.  This has been corrected. \\ \hline

\textbf{Gitlab-ex issue 319}:  Improve error checking for invalid
.MEASURE lines & The error checking for invalid \texttt{DERIV-AT},
\texttt{DERIV-WHEN}, \texttt{FIND-AT}, \texttt{FIND-WHEN} and
\texttt{WHEN} measures has been improved.  Some invalid measure
lines that would previously be reported as ``FAILED'' are now
correctly reported as parsing errors. \\ \hline

\textbf{Gitlab-ex issue 321}:  Fix Inf/NaN trap in DampedNewton nonlinear solver &
The convergence check in the DampedNewton nonlinear solver was not
correctly trapping NaNs in the residual vector.  Thus, the nonlinear
solver failure logic would not be correct, allowing the nonlinear
solver to continue when it should stop.  This has been
corrected. \\ \hline

\textbf{Gitlab-ex issue 328}:  Support C for temperature units &
  When specifying temperature, \Xyce{} would inappropriately exit with
  error if the units of Celsius were explicitly used in the netlist.
  This has been corrected, and now a line such as \texttt{.options
  device temp=25C} will work correctly.  The units for temperature
  in \Xyce{} have always been Celsius, so this was only a parsing
  issue. \\ \hline

\textbf{Gitlab-ex issue 338}: Added the function \texttt{Simulator:: getTimeVoltagePairsSz} &
The function \texttt{Simulator::getTimeVoltagePairsSz} supplies the
maximum number of time, voltage or state values for an ADC in a
subsequent call to \texttt{Simulator::getTimeVoltagePairs}
or \texttt{Simulator::getTimeStatePairs}.  Also,
calling \texttt{Simulator::getTimeStatePairs} no longer clears the
voltage history of an ADC, so a calling program that needs both state
and voltage history can first
call \texttt{Simulator::getTimeStatePairs} and then call
\texttt{Simulator::getTimeVoltagePairs} to get both.  Calling
\texttt{Simulator::getTimeVoltagePairs} still clears the ADC history.
See the Application note, Mixed Signal Simulation with \Xyce{} 7.5 for further
details. \\ \hline

\textbf{Gitlab-ex issue 340}: Bug in breakpointing, when an expression combines a ternary and a table &
  Under rare circumstances, if an expression combined a ternary
  operator and a time-dependent table, an error in breakpointing could
  cause the time integrator to get stuck in a near-infinite loop.
  This has been fixed.  \\ \hline

\textbf{Gitlab-ex issue 348}:  Added the function \texttt{xyce\_getTimeVoltagePairs\-ADCLimitData()}  &
  A new function has been added to the \texttt{XyceCInterface}
  called \texttt{xyce\_getTimeVoltagePairsADCLimitData()} which limits
  the data copied to the caller allocated space to whatever maximum
  allocation length is provided.  This avoids potential memory
  overwriting that could occur with the general access
  function \texttt{xyce\_getTimeVoltagePairsADC()}.  See the
  Application note, Mixed Signal Simulation with \Xyce{} 7.5 for
  further details.  \\ \hline

\textbf{Gitlab-ex issue 351}: Support Python 3 in Xyce &
  The Python interface to Xyce in \texttt{xyce\_interface.py}
  has been updated to support Python 3.x.  See the Application note,
  Mixed Signal Simulation with \Xyce{} 7.5 for further details.  \\ \hline

\textbf{Gitlab-ex issue 355}:  Make it possible to turn off initial junction voltages globally in device package w/o turning off voltlim &
Most semiconductor device models apply a initial junction voltages
during the initial iterations of a DCOP solve.  However, when applying
source stepping to obtain the solution, it doesn't make sense to apply
these initial junction voltages, as the intent of source stepping is
to initially start with all sources set to zero.  This has been fixed,
and is automatically applied when Xyce performs source stepping.  It
is also possible to manually disable initial junction voltages from
then netlist via \texttt{.options device all\_off=true}.  \\ \hline

\textbf{Gitlab-ex issue 358/GitHub issue 49}: Ternary operator broken in
analog function context & There were some use cases in which Xyce/ADMS
would emit bad code for ternary operator expressions in analog
function context.  Correct code is now emitted in all
cases.  \\ \hline

\textbf{Gitlab-ex issue 364}:  AC sensitivities for nonlinear parameters are very slow &
 The AC sensitivity calculation had an inefficiency in it which made
 it really slow for large numbers of parameters, particularly for
 parameters from nonlinear devices, which had to rely on a finite
 differenced matrix derivative.  Most of the work required could be
 performed a single time, rather than every time through the main
 loop.  \\ \hline

\textbf{Gitlab-ex issue 365}: IE() does not work for BSIM-SOI 4.x
& The BSIM-SOI 3 (level 10) and BSIM-SOI 4.x (levels 70 and 70450)
devices all support a fourth node named ``E'', but until now the
``IE()'' print operator only worked for printing lead currents through
this node for the BSIM-SOI 3.  Prior versions of Xyce required use of
``I4()'' to output this lead current.

Now, BSIM-SOI 4 and all BSIM-CMG models support ``IE()'' lead current
operators.  Wildcards for IE will now print lead currents for all
devices that have an ``E'' node. \\ \hline

\textbf{Gitlab-ex issue 366}: IB() does not work for MOSFETs derived from Verilog-A
& Most MOSFET devices have ``B'' nodes (either bulk or body, depending
on which model is under consideration).  None of the devices generated
from Verilog-A supported printing of the lead current associated with
this node via the ``IB()'' print accessor, and required instead that
one use ``In'' (where ``n'' is the position of the node on the
instance line).  Now they are correctly accessible using ``IB()''. \\
\hline

\textbf{Gitlab-ex issue 369}:  Out-of-bounds data storage in dx2 capability of expression library  &
The expression library, when used to support a behavioral model,
computes an array of one or more derivatives when it is evaluated.
The expression library was incorrectly written to assume that the size
of this array would never change.  However, this was incorrect for the
use case of \texttt{.func}, which might be called multiple times, with
arguments containing differing numbers differentiable variables.  This
was a use case that didn't come up very often, but when it did caused
a memory error.\\ \hline

\textbf{Gitlab-ex issue 372}:  AC sensitivities are incorrect when using .param for sensitivity parameters &
When performing AC sensitivities with respect to .param parameters,
there was an arithmetic mistake in the setup of the right-hand-side
vector.  This has been corrected.  \\ \hline

\textbf{Gitlab-ex issue 377}:  The limit random operator (2-argument version) is not correct in the expression library &
\Xyce{} has supported two versions of limit, one with two arguments and one
with three.  The version with two arguments was not correct, in
multiple ways.  When used without UQ (or sampling) it was supposed to
return the mean, but instead returned the sum of the two arguments.
When used with sampling, it incorrectly behaved the same
as \texttt{rand()}.  Both problems have been corrected.  \\ \hline

\textbf{Gitlab-ex issue 396}: Incorrect code generated for noise contributions that reference analog functions &
Xyce/ADMS was generating incorrect C++ code when a noise contribution
referenced analog functions on its RHS.  \\ \hline

\textbf{Gitlab-ex issue 398}: Incorrect code generated for single-line case items &
Xyce/ADMS was generating incorrect C++ code when a case item consisted
of a single probe-dependent assignment, and in analog functions when
the case item was a single line assignment that depends on the
function arguments.  There was no error if the item was enclosed in a
begin/end pair.  \\ \hline


\end{longtable}
}
