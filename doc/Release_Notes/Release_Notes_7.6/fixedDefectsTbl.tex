% Sandia National Laboratories is a multimission laboratory managed and
% operated by National Technology & Engineering Solutions of Sandia, LLC, a
% wholly owned subsidiary of Honeywell International Inc., for the U.S.
% Department of Energy's National Nuclear Security Administration under
% contract DE-NA0003525.

% Copyright 2002-2022 National Technology & Engineering Solutions of Sandia,
% LLC (NTESS).

% Sandia National Laboratories is a multimission laboratory managed and
% operated by National Technology & Engineering Solutions of Sandia, LLC, a
% wholly owned subsidiary of Honeywell International Inc., for the U.S.
% Department of Energy's National Nuclear Security Administration under
% contract DE-NA0003525.

% Copyright 2002-2022 National Technology & Engineering Solutions of Sandia,
% LLC (NTESS).


%%
%% Fixed Defects.
%%
{
\small

\begin{longtable}[h] {>{\raggedright\small}m{2in}|>{\raggedright\let\\\tabularnewline\small}m{3.5in}}
     \caption{Fixed Defects.  The Xyce team has multiple issue
     trackers, and the table below indicates fixed issues by
     indentifying both the tracker and the issue number.  Further,
     some issues are reported by open source users on GitHub and these
     issues may be tracked using multiple issue numbers.} \\ \hline
     \rowcolor{XyceDarkBlue} \color{white}\textbf{Defect} & \color{white}\textbf{Description} \\ \hline
     \endfirsthead
     \caption[]{Fixed Defects.  Note that we have two multiple issue tracking systems for Sandia Users.
     SON and SRN refer to our legacy open- and restricted-network Bugzilla system, and Gitlab refers to issues in our gitlab repositories.  } \\ \hline
     \rowcolor{XyceDarkBlue} \color{white}\textbf{Defect} & \color{white}\textbf{Description} \\ \hline
     \endhead

\textbf{Xyce Backlog Bugs/1}: Mutual inductor jacobian error &
A jacobian term in the nonlinear mutual inductor was wrong when the bias was changing 
across the primary inductor.  This jacobian error could lead to slower nonlinear 
convergence in transient simulations.  It has been fixed. \\ \hline

\textbf{Xyce Backlog Bugs/2}: Mutual inductors have a bad device connectivity map &
>>>>>>> 38619b588 (Document fix of the CMake version reporting)
The device connectivity map is used to determine the path to ground for error checking.  
The linear and nonlinear mutual inductors were not correctly setting up 
the device connectivity map and this resulted in false warnings that some circuit 
nodes did not have a path to ground.  The warning was invalid and this issue did 
not effect Xyce's calculations.  But the warning was incorrect and could cause 
confusion.  The code has been fixed and this warning should no longer appear under
false conditions. \\ \hline

\textbf{Xyce Project Backlog/260}: Fix version string in CMake development builds &
Xyce reports its version using \texttt{Xyce -v}.  With development versions of
Xyce, the reported version should include the Git SHA against which the code
was compiled along with the time of compilation. Prior to this change, CMake
would include the SHA and time at \emph{configure} time, not compilation time.
That is now fixed. \\ \hline

\textbf{Xyce Project Backlog/285}: CMake support for XyceCInterface &
The C-interface to the C++ object N\_CIR\_Xyce did not have the needed
files for building under CMake.  This issue has been resolved and the 
C-interface is now built and installed as part of Xyce. \\ \hline

\textbf{Xyce Project Backlog/302}: Xyce legacy MOSFETs do not recognize VT0 &
Xyce did not properly recognize that VT0 should be an alias for VTO in
the levels 1 through 6 legacy MOSFET devices.  These have been
recognized by every SPICE derivative since SPICE3 (though were not
recognized by SPICE2), and are now recognized by Xyce, too. \\ \hline

\textbf{repo/issue number}: Short description placeholder &
Description placeholder \\ \hline

\end{longtable}
}
