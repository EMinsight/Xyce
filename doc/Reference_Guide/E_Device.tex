% Sandia National Laboratories is a multimission laboratory managed and
% operated by National Technology & Engineering Solutions of Sandia, LLC, a
% wholly owned subsidiary of Honeywell International Inc., for the U.S.
% Department of Energy’s National Nuclear Security Administration under
% contract DE-NA0003525.

% Copyright 2002-2023 National Technology & Engineering Solutions of Sandia,
% LLC (NTESS).


\begin{Device}\label{E_DEVICE}

\symbol
{\includegraphics{vcvsSymbol}}

\device
\begin{alltt}
E<name> <(+) node> <(-) node> <(+) controlling node>
+ <(-) controlling node> <gain>
E<name> <(+) node> <(-) node> VALUE = \{ <expression> \}  
+ [device parameters]
E<name> <(+) node> <(-) node> TABLE \{ <expression> \} = 
+ < <input value>,<output value> >*
E<name> <(+) node> <(-) node> POLY(<value>) 
+ [<+ control node> <- control node>]*
+ [<polynomial coefficient value>]*
\end{alltt}

\examples
\begin{alltt}
EBUFFER 1 2 10 11 5.0
ESQROOT   5   0 VALUE = \{5V*SQRT(V(3,2))\}
ET2 2 0 TABLE \{V(ANODE,CATHODE)\} = (0,0) (30,1)
EP1 5 1 POLY(2) 3 0 4 0 0 .5 .5
\end{alltt}

\parameters

\begin{Parameters}

\param{\vbox{\hbox{(+) node\hfil}\hbox{(-) node}}}

Output nodes. Positive current flows from the \texttt{(+)} node through
the source to the \texttt{(-)} node.

\param{\vbox{\hbox{(+) controlling node\hfil}\hbox{(-) controlling node}}}

Node pairs that define a set of controlling voltages. A given node may
appear multiple times and the output and controlling nodes may be the
same.


\param{device parameters} 

The second form supports two instance parameters \texttt{smoothbsrc} and
\texttt{rcconst}. Parameters may be provided as space separated
\texttt{<parameter>=<value>} specifications as needed. The default value for
\texttt{smoothbsrc} is 0 and the default for \texttt{rcconst} is 1e-9.

\end{Parameters}

\comments 

In the first form, a specified voltage drop between controlling nodes is
multiplied by the gain to determine the voltage drop across the output nodes. 

The second through fourth forms allow nonlinear controlled sources using the
\texttt{VALUE}, \texttt{TABLE}, or \texttt{POLY} keywords, respectively, and
are used in analog behavioral modeling.  They are provided primarily for
netlist compatibility with other simulators.  These three forms are
automatically converted within \Xyce{} to its principal ABM device, the
\texttt{B} nonlinear dependent source device. See the B-source section
(\ref{B_Source_Device}) and the \Xyce{} User's Guide for more guidance on
analog behavioral modeling.  For details concerning the use of the
\texttt{POLY} format, see section~\ref{PspicePoly}.

For HSPICE compatibility, \texttt{VOL} is an allowed synonym for
\texttt{VALUE} for the E-source.

The power supplied or dissipated by this source device is calculated 
with $I \cdot \Delta V$ where the voltage drop is calculated as $(V_+ - V_-)$ 
and positive current flows from $V_+$ to $V_-$.  Dissipated power has a
positive sign, while supplied power has a negative sign.

{\bf NOTE:} The expression given on the left hand side of the equals
sign in E source TABLE expressions may be enclosed in braces, but is
not required to be.  Further, if braces are present there must be
exactly one pair of braces and it must enclose the entire expression.
It is not legal to use additional pairs of braces as parentheses
inside these expressions.  So
\begin{alltt}
ET2 2 0 TABLE \{V(ANODE,CATHODE)+5\} = (0,0) (30,1)
ET3 2 0 TABLE V(ANODE,CATHODE)+5 = (0,0) (30,1)
\end{alltt}
are legal, but 
\begin{alltt}
ET2 2 0 TABLE \{V(ANODE,CATHODE)+\{5\}\} = (0,0) (30,1)
\end{alltt}
is not.  This last will result in a parsing error about missing braces.

E-sources were originally developed primarily to support DC and transient analysis.  
As such, their support for frequency domain analysis (AC and HB) has some limitations.  
The main limitation to be aware of is that time-dependent sources will not work with AC or HB analysis.  
These are sources in which the variable \texttt{TIME} is used in the \texttt{VALUE=} expression. 
However, this time-dependent usage is not common.  The most 
common use case is one in which the E-source is purely dependent (depends only 
on other solution variables), and this use case will work with AC and HB.  

\end{Device}
