% Sandia National Laboratories is a multimission laboratory managed and
% operated by National Technology & Engineering Solutions of Sandia, LLC, a
% wholly owned subsidiary of Honeywell International Inc., for the U.S.
% Department of Energy’s National Nuclear Security Administration under
% contract DE-NA0003525.

% Copyright 2002-2024 National Technology & Engineering Solutions of Sandia,
% LLC (NTESS).

User-defined global parameter that can be used in expressions throughout the netlist.
\begin{Command}

\format
.GLOBAL\_PARAM [<name>=<value>]*

\examples
.GLOBAL\_PARAM T=\{27+100*time\}

\begin{Arguments}
\argument{name}

Global parameter name.  Global parameters may be redefined.  
If the same name is used on multiple parameters, \Xyce{} by default will use the last parameter of that name.  
  By default, no warning will be emitted.
To change this behavior, one can use the \texttt{-redefined\_param} command line option, described in section~\ref{cmd_line_arg_list}.

\argument{value}

The value may be a number or an expression.

\end{Arguments}

\comments

A \texttt{.PARAM} 
defined in the top level netlist is equivalent to 
a \texttt{.GLOBAL\_PARAM}, and they can be combined as needed.
Thus, you may use parameters defined by \texttt{.PARAM} in expressions used to
define global parameters, and you may also use global parameters in
\texttt{.PARAM} definitions.    However, a \texttt{.GLOBAL\_PARAM} 
  can only depend on \texttt{.PARAM} parameter from the top level circuit scope.

Like \texttt{.PARAM} parameters, \texttt{.GLOBAL\_PARAM} may
depend on time dependent quantities in the circuit.  They may also
be frequency dependent.  They cannot, however, be 
dependent on solution variaables such as voltage nodes.

To load an external data file with time voltage pairs of data on each 
line into a global parameter, use this syntax:

\texttt{.GLOBAL\_PARAM extdata = \{tablefile("filename")\}}

or

\texttt{.GLOBAL\_PARAM extdata = \{table("filename")\}}

where \texttt{filename} would be the name of the file to load.  
Other interpolators that can read in a data table from a file 
include \texttt{fasttable},\texttt{spline}, \texttt{akima}, \texttt{cubic}, 
\texttt{wodicka} and \texttt{bli}.  See \ref{ExpressionDocumentation} 
for further information. 

There are several reserved words that may not be used as names for parameters.  These reserved words are:
\begin{XyceItemize}
\item \verb+Time+
\item \verb+Freq+ 
\item \verb+Hertz+ 
\item \verb+Vt+
\item \verb+Temp+
\item \verb+Temper+
\item \verb+GMIN+
\end{XyceItemize}

Global parameters are accessible, and have the same value, throughout all
levels of the netlist hierarchy.  It is not legal to redefine global parameters
in different levels of the netlist hierarchy.  Also, global parameters can only 
  be defined in the top level circuit scope.   Parameters defined inside of 
  subcircuits must be of the \texttt{.PARAM} type.

\end{Command}
