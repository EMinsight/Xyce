% Sandia National Laboratories is a multimission laboratory managed and
% operated by National Technology & Engineering Solutions of Sandia, LLC, a
% wholly owned subsidiary of Honeywell International Inc., for the U.S.
% Department of Energy’s National Nuclear Security Administration under
% contract DE-NA0003525.

% Copyright 2002-2023 National Technology & Engineering Solutions of Sandia,
% LLC (NTESS).

The \texttt{.MEASURE FFT} statement allows calculation or reporting of simulation
metrics, from data associated with .FFT analyses, to an external file as well as to
the standard output and/or a log file, So, it is only supported for \texttt{.TRAN},
analyses.  It can be used with {\tt .STEP}.  For HSPICE compatibility, \texttt{.MEAS}
is an allowed synonym for \texttt{.MEASURE}.

The syntaxes for the \texttt{.MEASURE FFT} statements are shown below.

\begin{Command}
\format
\begin{alltt}
.MEASURE FFT <result name> ENOB <variable> [BINSIZ=<value>]
+ [DEFAULT_VAL=<value>] [PRECISION=<value>] [PRINT=<value>]

.MEASURE FFT <result name> EQN <variable>
+ [DEFAULT_VAL=<value>] [PRECISION=<value>] [PRINT=<value>]

.MEASURE FFT <result name> FIND <variable> AT=<value>
+ [PRECISION=<value>] [PRINT=<value>]

.MEASURE FFT <result name> SFDR <variable>
+ [MINFREQ=<value>] [MAXFREQ=<value>]  [BINSIZ=<value>]
+ [DEFAULT_VAL=<value>] [PRECISION=<value>] [PRINT=<value>]

.MEASURE FFT <result name> SNDR <variable>  [BINSIZ=<value>]
+ [DEFAULT_VAL=<value>] [PRECISION=<value>] [PRINT=<value>]

.MEASURE FFT <result name> SNR <variable> [MAXFREQ=<value>]
+ [DEFAULT_VAL=<value>] [PRECISION=<value>] [PRINT=<value>]

.MEASURE FFT <result name> THD <variable>
+ [NBHARM=<value>] [MAXFREQ=<value>]
+ [DEFAULT_VAL=<value>] [PRECISION=<value>] [PRINT=<value>]

\end{alltt}
\index{\texttt{.MEASURE FFT}}
\index{results!measure fft}

\examples
\begin{alltt}
.FFT V(1) NP=16
.MEASURE FFT ENOBVAL ENOB V(1)
.MEASURE FFT EQNVAL EQN VR1AT2
.MEASURE FFT VR1AT2 VR(1) AT=2
.MEASURE FFT SFDRVAL SFDR V(1)
.MEASURE FFT SNDRVAL SNDR V(1)
.MEASURE FFT SNRVAL SNR V(1)
.MEASURE FFT THDVAL THD V(1)
\end{alltt}

\arguments

\begin{Arguments}
\argument{result name}

Measured results are reported to the output and log file.
Additionally, the results are stored in files called
\texttt{circuitFileName.mt\#}, where the suffixed number
(\texttt{\#}) starts at \texttt{0} and increases for multiple
iterations (\texttt{.STEP} iterations) of a given simulation. Each
line of this file will contain the measurement name, \texttt{<result
name>}, followed by its value for that run.  The \texttt{<result
name>} must be a legal \Xyce{} character string.

If multiple measures are defined with the same \texttt{<result name>} then
\Xyce{} uses the last such definition, and issues warning messages about
(and discards) any previous measure definitions with the same
\texttt{<result name>}.

\argument{measure type}

\texttt{ENOB, EQN, FIND, SFDR, SNDR, SNR, THD}

The third argument specifies the type of measurement or calculation to
be done. By default, the measurement is performed over the time window defined
by the {\tt START} and {\tt STOP} parameters on the associated {\tt .FFT}
line.  So, the  {\tt FROM}, {\tt TO} and {\tt TD} qualifiers have no
effect on FFT-based measures.

The supported measure types are:

\begin{description}
  \item[\tt ENOB] Calculates the ``Effective Number of Bits'', where that metric
    is defined in Section \ref{FFT_metrics} which covers the \texttt{.FFT} command.

  \item[\tt EQN] Calculates the value of {\tt <variable>} during the simulation.
    That variable can use the results of other measure statements. {\tt PARAM}
    is an allowed synonym for {\tt EQN} as a measure type.  For {\tt FFT} measure
    mode, an {\tt EQN} measure will be reported as ``failed'' until the associated
    FFT has been calculated.

  \item[\tt FIND] Returns the requested FFT cofficient at the requested
    frequency.  Examples of the mapping of \Xyce{} operators (e.g., {\tt VM} and {\tt IM})
    to FFT cofficients is given in the ``Additional Examples'' subsection below.
    {\tt FIND} measures can be also used in conjunction with {\tt EQN} measures
    to generate fairly arbitrary FFT-based measures.  The {\tt FIND} measure for
    {\tt FFT} measure mode does not support expressions, or the {\tt P} and
    {\tt W} operators.  It also does not support multi-terminal lead current
    operators, such as {\tt IC()}.

  \item[\tt SFDR] Calculates the ``Spurious Free Dynamic Range'', where that metric
    is defined in Section \ref{FFT_metrics}.

  \item[\tt SNDR] Calculates the ``Signal to Noise-plus-Distortion Ratio'', where that metric
    is defined in Section \ref{FFT_metrics}.

  \item[\tt SNR] Calculates the ``Signal to Noise Ratio'', where that metric
    is defined in Section \ref{FFT_metrics}.

  \item[\tt THD] Calculates the ``Total Harmonic Distortion'', where that metric
    is defined further below and also in in Section \ref{FFT_metrics}.
\end{description}

\argument{variable}

The \texttt{<variable>} is a simulation quantity, such as a
voltage or current.  Additionally, the \texttt{<variable>} may be
a \Xyce{} expression delimited by \{ \} brackets.  The only constraint
is that the \texttt{variable} on the \texttt{.MEASURE FFT} line must
be an exact match for the \texttt{ov} on at least one \texttt{.FFT} line
in the netlist.  If there are multiple \texttt{.FFT} lines in the netlist
with the same \texttt{ov} then the corresponding \texttt{.MEASURE FFT}
statements will use the first such one.

\argument{AT=value}
A frequency {\em at which} the measurement calculation will occur.  This is
used by the {\tt FIND} measure only.  The entered {\tt AT} value will be
rounded to the nearest harmonic frequency, as defined by the {\tt FREQ},
{\tt START} and {\tt STOP} parameters on the associated {\tt .FFT} line.
An {\tt AT} value that rounds to a harmonic frequency of less than zero,
or to more than {\tt NP/2}, will produce a failed measure in \Xyce{}, where
{\tt NP} is the number of points specified on the associated {\tt .FFT} line..
The behavior of these ''failed'' cases may differ from commercial simulators.

\argument{BINSIZ=value}
This parameter is implemented in \Xyce{} for the \texttt{ENOB}, \texttt{SFDR}
and \texttt{SNDR} measure types. It can be used to account for any ``broadening''
of the spectral energy in the first harmonic of the signal, as discussed below.
\texttt{BINSIZ} has a default value of 0.

\argument{DEFAULT\_VAL=value}

If the conditions specified for finding a given value are not found
during the simulation then the measure will return the default value
of {\tt -1} in the \texttt{circuitFileName.mt\#} file.
The measure value in the standard output or log file will be
FAILED.  The default return value for the \texttt{circuitFileName.mt\#}
file is settable by the user for each measure by adding the qualifier
{\tt DEFAULT\_VAL=<retval>} on that measure line.  If either
\texttt{.OPTIONS MEASURE MEASFAIL=<val>} or
\texttt{.OPTIONS MEASURE DEFAULT\_VAL=<val>} are given in the
netlist then those values override the \texttt{DEFAULT\_VAL}
parameters given on individual \texttt{.MEASURE FFT} lines.
See Section \ref{Options_Reference} for more details.

\argument{MAXFREQ=value}
The maximum frequency over which to perform a {\tt SFDR}, {\tt SNR}
or {\tt THD} measure. The entered {\tt MAXFREQ} value will be rounded
to the nearest harmonic frequency, as defined by the {\tt FREQ},
{\tt START} and {\tt STOP} parameters on the associated {\tt .FFT}
line.  The default value is {\tt NP/2}, where {\tt NP} is the number
of points specified on the associated {\tt .FFT} line.

\argument{MINFREQ=value}
The minimum frequency over which to perform a {\tt SFDR} or {\tt THD} measure.
The entered {\tt MINFREQ} value will be rounded to the nearest harmonic
frequency, as defined by the {\tt FREQ}, {\tt START} and {\tt STOP}
parameters on the associated {\tt .FFT} line.  The default value is 1.

\argument{NBHARM=value}
The maximum (integer) number of harmonics over which to perform a THD measure.
The default value is {\tt NP/2}.  The {\tt NBHARM} qualifier has precedence
over the {\tt MAXFREQ} qualifier if both are given on a {\tt .MEASURE} line.

\argument{PRECISION=value}

The default precision for {\tt .MEASURE} output is 6 digits after the
decimal point.  This argument provides a user configurable precision
for a given {\tt .MEASURE} statement that applies to both the
\texttt{.mt\#} file and standard output.
If \texttt{.OPTIONS MEASURE MEASDGT=<val>} is given in the netlist
then that value overrides the \texttt{PRECISION} parameters given on
individual \texttt{.MEASURE} lines.

\argument{PRINT=value}

This parameter controls where the {\tt .MEASURE} output appears.  The
default is {\tt ALL}, which produces measure output in both the
\texttt{.mt\#} and the standard output.  A value of {\tt STDOUT}
only produces measure output to standard output, while a value of
{\tt NONE} suppresses the measure output to both the \texttt{.mt\#}
file and standard output.

\end{Arguments}

\end {Command}

\subsubsection{Measure Definitions}
The \texttt{ENOB}, \texttt{SNDR}, \texttt{SNR} and \texttt{SFDR} measure
types use the same definitions as the metrics produced by \texttt{.FFT} lines.
Section \ref{FFT_metrics} provides more details on those definitions.  (Note: The
\texttt{MAXFREQ} and \texttt{MINFREQ} qualifiers from the \texttt{.MEASURE FFT}
lines are mapped into the \texttt{FMAX} and \texttt{FMIN} parameters used in those
equations.)  There are two exceptions.

The first exception is the \texttt{THD} measure.  If the optional \texttt{NBHARM}
qualifier is not used then the definition given in Section \ref{FFT_metrics} is
used.  If the \texttt{NBHARM} qualifier is used then it takes precedence over the
\texttt{MAXFREQ} qualifier.  The \texttt{THD} measure definition is then as follow.
Let $f_{0}$ be the integer index of the ``first harmonic'' ($f_{0}$), as defined in
Section \ref{FFT_metrics}, from the associated \texttt{.FFT} line.  Then the
effective value of the upper frequency limit ($f_{2}$) in the THD calculation
is \texttt{NBHARM}$\cdot f_{0}$, with the caveat that all of the harmomics will
be used if \texttt{NBHARM} < 0 or \texttt{NBHARM} > \texttt{NP/2}.

The second exception is the \texttt{BINSIZ} qualifier for the \texttt{ENOB},
\texttt{SFDR} and \texttt{SNDR} measures.  For a non-zero value of
\texttt{BINSIZ}, the ``signal power'' is considered to reside in the harmonic
indexes between ($f_{0} \pm$ \texttt{BINSIZ}), where the DC value is still
excluded from the measure calculations. (Note: This definition for 
\texttt{BINSIZ} may differ from HSPICE.)

\subsubsection{Re-Measure}
\label{Measure_FFT_ReMeasure}
\index{measure fft!re-measure}
\Xyce{} can re-calculate (or re-measure) the values for {\tt .MEASURE FFT}
statements using existing \Xyce{} output files.  Section~\ref{Measure_ReMeasure}
discusses this topic in more detail for both {\tt .MEASURE} and {\tt .FFT}
statements.

\subsubsection{Additional Examples}
\label{Measure_FFT_Additional_Examples}
\index{measure fft!additional examples}
This section provides a simple example how to use the {\tt FIND} measure,
along with the {\tt V()}, {\tt VR()}, {\tt VI()}, {\tt VM()}, {\tt VP()}
and {\tt VDB()} operators, to obtain the real and imaginary parts of the
FFT coefficients, along with the magnitude and phase of those coefficients,
at a specified frequency.  Those coefficient values are unnormalized.

\begin{alltt}
* Example of obtaining FFT coefficients
.TRAN 0 1
.PRINT TRAN V(1)
.OPTIONS FFT FFT_ACCURATE=1 FFTOUT=1
V1 1 0 1
R1 1 0 1
.FFT V(1) NP=8 WINDOW=HANN

* Unnormalized one-sided FFT cofficients for V(1) at F=1.0
* Magnitude
.MEASURE FFT V1AT1 FIND V(1) AT=1.0
* Real part
.MEASURE FFT VR1AT1 FIND VR(1) AT=1.0
* Imaginary part
.MEASURE FFT VI1AT1 FIND VI(1) AT=1.0
* Magnitude
.MEASURE FFT VM1AT1 FIND VM(1) AT=1.0
* Phase
.MEASURE FFT VP1AT1 FIND VP(1) AT=1.0
* Magnitude in dB
.MEASURE FFT VDB1AT1 FIND VDB(1) AT=1.0

.END
\end{alltt}

The {\tt .MEASURE} output is then:
\begin{alltt}
FINDV1AT1 = 5.200051e-01
FINDVR1AT1 = -4.804221e-01
FINDVI1AT1 = -1.989973e-01
FINDVM1AT1 = 5.200051e-01
FINDVP1AT1 = -1.575000e+02
FINDVDB1AT1 = -5.679847e+00
\end{alltt}


