% Sandia National Laboratories is a multimission laboratory managed and
% operated by National Technology & Engineering Solutions of Sandia, LLC, a
% wholly owned subsidiary of Honeywell International Inc., for the U.S.
% Department of Energy’s National Nuclear Security Administration under
% contract DE-NA0003525.

% Copyright 2002-2023 National Technology & Engineering Solutions of Sandia,
% LLC (NTESS).


\begin{Device}

\device
S<name> <(+) switch node> <(-) switch node> <model name> [ON] [OFF] <control = { expression }>

\model
\begin{alltt}
.MODEL <model name> VSWITCH [model parameters]
.MODEL <model name> ISWITCH [model parameters]
.MODEL <model name> SWITCH [model parameters]
\end{alltt}

\examples
\begin{alltt}
S1 1 2 SWI OFF CONTROL=\{I(VMON)\}
SW2 1 2 SWV OFF CONTROL=\{V(3)-V(4)\}
S3 1 2 SW OFF CONTROL=\{if(time>0.001,1,0)\}
\end{alltt}

\comments

The generic switch is similar to the voltage- or current-controlled
switch except that the control variable is anything that can be writen
as an expression.  The examples show how a voltage- or
current-controlled switch can be implemented with the generic switch.
Also shown is a relay that turns on when a certain time is reached.
Model parameters are given in Table ~\ref{S_1_Device_Model_Params}.

The voltage- and current-controlled switch syntaxes are converted at
parse time to their equivalent generic device, and so all three
variants in fact use the same code internally.

The power dissipated in the generic switch is calculated with $I \cdot \Delta V$ 
where the voltage drop is calculated as $(V_+ - V_-)$ and positive current 
flows from $V_+$ to $V_-$.  This will essentially be the power dissipated
in either \texttt{RON} or \texttt{ROFF}, since the generic switch is a particular 
type of controlled resistor.

\end{Device}
