% Sandia National Laboratories is a multimission laboratory managed and
% operated by National Technology & Engineering Solutions of Sandia, LLC, a
% wholly owned subsidiary of Honeywell International Inc., for the U.S.
% Department of Energy’s National Nuclear Security Administration under
% contract DE-NA0003525.

% Copyright 2002-2024 National Technology & Engineering Solutions of Sandia,
% LLC (NTESS).


\chapter{Netlist Reference}
\label{Netlist_Reference}
\index{netlist!reference}

\chapteroverview{Chapter Overview}
{
This chapter contains reference material directed towards working with circuit
analyses in \Xyce{} using the netlist interface.  Included are
detailed command descriptions, start-up option definitions and a list of
devices supported by the \Xyce{} netlist interface.
}

\section{Netlist Commands}
\label{Netlist_Commands}
\index{netlist!commands}

This section outlines the netlist commands that can be used with \Xyce{} to
setup and control circuit analysis.

%%%%%%%%%%%%%%%%%%%%%%%%%%%%%%%%%%%%%%%%%%%%%%%%%%%%%%%%%%%%%%%%%%%%%%%%%%%%%%%%
%%\newpage
\subsection{\texttt{.AC} (AC Analysis) }
\index{\texttt{.AC}}
\index{analysis!AC} \index{AC analysis}
% Sandia National Laboratories is a multimission laboratory managed and
% operated by National Technology & Engineering Solutions of Sandia, LLC, a
% wholly owned subsidiary of Honeywell International Inc., for the U.S.
% Department of Energy’s National Nuclear Security Administration under
% contract DE-NA0003525.

% Copyright 2002-2022 National Technology & Engineering Solutions of Sandia,
% LLC (NTESS).


Calculates the frequency response of a circuit over a range of frequencies.

The .AC command can specify a linear sweep, decade logarithmic sweep,
octave logarithmic sweep, or a data table of multivariate values.

\begin{Command}

\format
\begin{alltt}
.AC <sweep type> <points value>
+ <start frequency value> <end frequency value>
\end{alltt}

\examples
\begin{alltt}
.AC LIN 101 100Hz 200Hz
.AC OCT 10 1kHz 16kHz
.AC DEC 20 1MEG 100MEG
.AC DATA=<table name>

.param points=101, start=100Hz, end=200Hz
.AC LIN \{points\} \{start\}  \{end\}
\end{alltt}

\arguments

\begin{Arguments}

\argument{sweep type}
Must be \texttt{LIN}, \texttt{OCT}, \texttt{DEC}, or \texttt{DATA} as described below.
\begin{description}

\item[\tt LIN] Linear sweep\\
The sweep variable is swept linearly from the starting to the ending value.

\item[\tt OCT] Sweep by octaves\\
The sweep variable is swept logarithmically by octaves.

\item[\tt DEC] Sweep by decades\\
The sweep variable is swept logarithmically by decades.

\item[\tt DATA] Sweep values from a table\\
Sweep variables are applied based on the rows of a data table.  This format allows magnitude and phase to be swept in addition to frequency.    If using this format, no other arguments are needed on the \texttt{.AC} line.

\end{description}

\argument{points value}
Specifies the number of points in the sweep, using an integer greater than or equal to 1.

\argument{\vbox{\hbox{start frequency value\hfil}\hbox{end frequency value}}}

The end frequency value must not be less than the start frequency value,
and both must be greater than zero. The whole sweep must include at
least one point.

\end{Arguments}

\comments

AC analysis is a linear analysis. The simulator calculates the frequency
response by linearizing the circuit around the DCOP bias point.

If specifying the sweep points using the \texttt{DATA} type, one can
also sweep the magnitude and phase of an AC source, as well as the
values of linear model parameters.  However, unlike the use of
\texttt{DATA} for \texttt{.STEP} and \texttt{.DC}, it is not possible
to sweep nonlinear device parameters.  This is because changing other
nonlinear device parameters would alter the correct DCOP solution, and
the AC sweep happens after the DCOP calculation in the analysis flow.
To sweep a nonlinear device parameter on an AC problem, add a
\texttt{.STEP} command to the netlist to provide an outer parametric
sweep around the analysis.

\index{\texttt{.PRINT}}\index{results!print}\index{\texttt{.PRINT}!\texttt{AC}}
A \texttt{.PRINT AC} must be used to get the results of the AC sweep
analysis.  See Section \ref{.PRINT}.

Some devices that may be expected to work in AC analysis do not at
this time.  This includes, but is not limited to, the lossy
transmission line (LTRA) and lossless transmission line (TRA).  The
LTRA and TRA models will need to be replaced with lumped transmission
line models (YTRANSLINE).

Power calculations (\texttt{P(<device>)} and \texttt{W(<device>}) are
not supported for any devices for AC analysis.  Current variables
(e.g., \texttt{I(<device>)}) are only supported for devices that have
``branch currents'' that are part of the solution vector. This
includes the V, E, H and L devices.  It also includes the voltage-form
of the B device.

\end{Command}


%%%%%%%%%%%%%%%%%%%%%%%%%%%%%%%%%%%%%%%%%%%%%%%%%%%%%%%%%%%%%%%%%%%%%%%%%%%%%%%%
\newpage
\subsection{\texttt{.DATA} (Data Table for sweeps) }
\index{\texttt{DATA}}
% Sandia National Laboratories is a multimission laboratory managed and
% operated by National Technology & Engineering Solutions of Sandia, LLC, a
% wholly owned subsidiary of Honeywell International Inc., for the U.S.
% Department of Energy’s National Nuclear Security Administration under
% contract DE-NA0003525.

% Copyright 2002-2023 National Technology & Engineering Solutions of Sandia,
% LLC (NTESS).


User-defined data table, which can be used to specify sweep points for \texttt{.AC}, \texttt{.DC}, \texttt{.NOISE} or \texttt{.STEP}

\begin{Command}

\format
.DATA [<name>] \\
+ <parameter name> [parameter name]* \\
+ <parameter value> [parameter value]* \\
.ENDDATA

\examples
.data test \\
+ r1   r2 \\
+ 8.0000e+00  4.0000e+00 \\
+ 9.0000e+00  4.0000e+00 \\
.enddata

\arguments

\begin{Arguments}
\argument{name}
Name of the data table.

\argument{parameter name}
Name of sweep parameter.  This can be a device instance parameter, 
  a device model parameter or a user-defined parameter specified using \texttt{.GLOBAL\_PARAM} or a globally scoped \texttt{.PARAM} statement.

\argument{parameter value}
Value of sweep parameter for the given sweep point.  This must be a double precision number.  Each row of the table corresponds to a different sweep step, so multiple parameters can be changed simultaneously.

\end{Arguments}

\comments

Each column of a data table corresponds to a different parameter, and each row corresponds to a different sweep point.

If using \texttt{.DATA} with \texttt{.DC} or \texttt{.STEP}, then any instance parameter, model parameter, 
or user-defined parameter in the global scope.

However, if using \texttt{.DATA} with \texttt{.AC} or \texttt{.NOISE}, then one can sweep the magnitude and phase of an AC source, and linear model parameters (such as resistance and capacitance) in addition to the traditional AC sweep variable, frequency.  Parameters associated with nonlinear models (like transistors) are not allowed.  This is because AC analysis is a linear analysis, performed after the DCOP calculation.  Changing nonlinear device model parameters would result in a different DCOP solution, so changing them during the AC (or NOISE) analysis phase is not valid.

Another caveat, for both \texttt{.AC} and \texttt{.NOISE}, is that all of the frequency
values in the data table must be positive.  If \texttt{.DATA} is used with \texttt{.NOISE}
then the integrals for the total input noise and total output noise will only be calculated,
and sent to stdout, if the frequencies in the data table are monotonically increasing.

\end{Command}


%%%%%%%%%%%%%%%%%%%%%%%%%%%%%%%%%%%%%%%%%%%%%%%%%%%%%%%%%%%%%%%%%%%%%%%%%%%%%%%%
\newpage
\subsection{\texttt{.DC} (DC Sweep Analysis) }
\index{\texttt{DC}}
\index{analysis!DC} \index{DC analysis}
% Sandia National Laboratories is a multimission laboratory managed and
% operated by National Technology & Engineering Solutions of Sandia, LLC, a
% wholly owned subsidiary of Honeywell International Inc., for the U.S.
% Department of Energy’s National Nuclear Security Administration under
% contract DE-NA0003525.

% Copyright 2002-2023 National Technology & Engineering Solutions of Sandia,
% LLC (NTESS).


Calculates the operating point\index{Operating Point} for the circuit for a
range of values.  Primarily, this capability is applied to independent
voltage sources, but it can also be applied to most device parameters.
Note that this may be repeated for multiple sources in the
same \texttt{.DC} line.

The .DC command can specify a linear sweep, decade logarithmic sweep,
octave logarithmic sweep, a list of values, or a data table of multivariate values.

\subsubsection{Linear Sweeps}
\index{analysis!DC!Linear sweeps} \index{DC analysis!Linear sweeps}

\begin{Command}
\format
\begin{alltt}
.DC [LIN] <sweep variable name> <start> <stop> <step>
+ [<sweep variable name> <start> <stop> <step>]*
\end{alltt}

\examples
\begin{alltt}
.DC LIN V1 5 25 5
.DC VIN -10 15 1
.DC R1 0 3.5 0.05 C1 0 3.5 0.5

.param start=5, stop=25, points=5
.DC \{start\} \{stop\} \{points\} 
\end{alltt}

\comments
A \texttt{.PRINT DC} must be used to get the results of the DC sweep
analysis.  See Section \ref{.PRINT}.
\index{\texttt{.PRINT}}\index{results!print}\index{\texttt{.PRINT}!\texttt{DC}}

A \texttt{.OP} comand will result in a linear DC analysis if there is no .DC specified.

If the stop value is smaller than the start value, the step value
should be negative.  If a positive step value is given in this case,
only a single point (at the start value) will be performed, and a
warning will be emitted.

\end{Command}

\subsubsection{Decade Sweeps}
\index{analysis!DC!Decade sweeps} \index{DC analysis!Decade sweeps}

\begin{Command}
\format
\begin{alltt}
.DC DEC <sweep variable name> <start> <stop> <points>
+ [DEC <sweep variable name> <start> <stop> <points>]*
\end{alltt}

\examples
\begin{alltt}
.DC DEC VIN 1 100 2 
.DC DEC R1 100 10000 3 DEC VGS 0.001 1.0 2

.param start=1, stop=100, points=2
.DC DEC VIN \{start\} \{stop\} \{points\}
\end{alltt}

\comments
The stop value should be larger than the start value.  If a stop value
smaller than the start value is given, only a single point at the
start value will be performed, and a warning will be emitted.  The
points value must be an integer.

\end{Command}

\subsubsection{Octave Sweeps}
\index{analysis!DC!Octave sweeps} \index{DC analysis!Octave sweeps}

\begin{Command}
\format
\begin{alltt}
.DC OCT <sweep variable name> <start> <stop> <points>
+ [OCT <sweep variable name><start> <stop> <points>]\ldots 
\end{alltt}

\examples
\begin{alltt}
.DC OCT VIN 0.125 64 2 
.DC OCT R1 0.015625 512 3 OCT C1 512 4096 1

.param start=0.125, stop=64, points=2
.DC OCT VIN \{start\} \{stop\} \{points\}
\end{alltt}

\comments
The stop value should be larger than the start value.  If a stop value
smaller than the start value is given, only a single point at the
start value will be performed, and a warning will be emitted.  The
points value must be an integer.

\end{Command}

\subsubsection{List Sweeps}
\index{analysis!DC!List Sweeps} \index{DC analysis!List sweeps}

\begin{Command}
\format
\begin{alltt}
.DC <sweep variable name> LIST <val> <val> <val>*
+ [ <sweep variable name> LIST <val> <val>* ]*
\end{alltt}

\examples
\begin{alltt}
.DC VIN LIST 1.0 2.0 5.0 6.0 10.0 
.DC VDS LIST 0 3.5 0.05 VGS LIST 0 3.5 0.5
.DC TEMP LIST 10.0 15.0 18.0 27.0 33.0

.param val1=0, val2=3.5, val3=0.5
.DC VDS LIST \{val1\} \{val2\} \{val3\} 
\end{alltt}

\end{Command}


\subsubsection{Data Sweeps}
\index{analysis!DC!Data Sweeps} \index{DC analysis!Data sweeps}

\begin{Command}
\format
\begin{alltt}
.DC DATA=<data table name> 
\end{alltt}

\examples
.DC data=resistorValues

.data resistorValues \\
+ r1   r2 \\
+ 8.0000e+00  4.0000e+00 \\
+ 9.0000e+00  4.0000e+00 \\
.enddata

\end{Command}


%%%%%%%%%%%%%%%%%%%%%%%%%%%%%%%%%%%%%%%%%%%%%%%%%%%%%%%%%%%%%%%%%%%%%%%%%%%%%%%%
\newpage
\subsection{\texttt{.DCVOLT} (Initial Condition, Bias point)}
\index{\texttt{.DCVOLT}}
\index{initial condition!DCVOLT}
\index{initial condition}

The \texttt{.DCVOLT} sets initial conditions for an operating point calculation.
It is identical in function to the \texttt{.IC} command.  See
section~\ref{IC_section} for detailed guidance.

%%%%%%%%%%%%%%%%%%%%%%%%%%%%%%%%%%%%%%%%%%%%%%%%%%%%%%%%%%%%%%%%%%%%%%%%%%%%%%%%
\newpage
\subsection{\texttt{.EMBEDDEDSAMPLING} (Embedded Sampling)}\label{.EMBEDDEDSAMPLING}
% Sandia National Laboratories is a multimission laboratory managed and
% operated by National Technology & Engineering Solutions of Sandia, LLC, a
% wholly owned subsidiary of Honeywell International Inc., for the U.S.
% Department of Energy’s National Nuclear Security Administration under
% contract DE-NA0003525.

% Copyright 2002-2023 National Technology & Engineering Solutions of Sandia,
% LLC (NTESS).

\label{EMBEDDEDSAMPLING_section}
\index{\texttt{.EMBEDDEDSAMPLING}}
\index{analysis!sampling} \index{embedded sampling analysis}
Calculates a full analysis (for \verb|.DC| or \verb|.TRAN| only) over
a distribution of parameter values.  Embedded sampling operates
similarly to \verb|.STEP|, except that the parameter values are
generated from random distributions rather than sweeps, and that the 
loop over parameters happens at the inner-most part of the calculation, 
so all samples are propagated simultaneously.
If used in conjunction 
with projection-based PCE methods, then the sample points are not based on random samples.  
Instead they are based on the quadrature points.

\index{analysis!SAMPLING}
\index{analysis!EMBEDDEDSAMPLING}
\index{EMBEDDEDSAMPLING analysis}
\index{analysis!MC}
\index{MC analysis}
\index{analysis!PCE}
\index{PCE analysis}
\index{analysis!Monte Carlo}
\index{Monte Carlo analysis}
\index{analysis!LHS}
\index{LHS analysis}
\index{Latin Hypercube Sampling analysis}

\index{analysis!EMBEDDEDSAMPLING}
\index{EMBEDDEDSAMPLING analysis}

\begin{Command}
\format
.EMBEDDEDSAMPLING  \\
+ param=<parameter name>,[parameter name]*  \\
+ type=<parameter type>,[parameter type]*  \\
+ means=<mean>,[mean]*  \\
+ std\_deviations=<standard deviation>,[standard deviation]* \\
+ lower\_bounds=<lower bound>,[lower bound]*  \\
+ upper\_bounds=<upper bound>,[upper bound]* \\
+ alpha=<alpha>,[alpha]*  \\
+ beta=<beta>,[beta]*

\examples
\begin{alltt}
.EMBEDDEDSAMPLING
+ param=R1
+ type=normal
+ means=3K
+ std\_deviations=1K

.EMBEDDEDSAMPLING
+ param=R1,R2
+ type=uniform,uniform
+ lower\_bounds=1K,2K
+ upper\_bounds=5K,6K

.EMBEDDEDSAMPLING
+ useExpr=true

.options EMBEDDEDSAMPLES numsamples=10000

.options EMBEDDEDSAMPLES numsamples=25000
+ OUTPUTS=\{R1:R\},\{V(1)\}
+ SAMPLE\_TYPE=MC

.options EMBEDDEDSAMPLES numsamples=1000
+ MEASURES=maxSine
+ SAMPLE\_TYPE=LHS

.options embeddedsamples numsamples=30
+ covmatrix=1e6,1.0e-3,1.0e-3,4e-14
+ OUTPUTS=\{V(1)\},\{R1:R\},\{C1:C\}
\end{alltt}

\arguments

\begin{Arguments}

\argument{param}
Names of the parameters to be sampled.  This may be any of the parameters
that are valid for \verb|.STEP|, including device instance, device model,
or global parameters.  If more than one parameter, then specify as a
comma-separated list.

\argument{type}
Distribution type for each parameter.  This may be uniform, normal or gamma.
If more than one parameter, then specify as a comma-separated list.

\argument{means}
If using normal distributions, the mean for each parameter must be specified.
If more than one parameter, then specify as a comma-separated list.

\argument{std\_deviations}
If using normal distributions, the standard deviation for each parameter
must be specified.  If more than one parameter, then specify as a
comma-separated list.

\argument{lower\_bounds}
If using uniform distributions, the lower bound must be specified.
This is optional for normal distributions.  If used with normal
distributions, may alter the mean and standard deviation.
If more than one parameter, then specify as a comma-separated list.

\argument{upper\_bounds}
If using uniform distributions, the upper bound must be specified.
This is optional for normal distributions.  If used with normal
distributions, may alter the mean and standard deviation.
If more than one parameter, then specify as a comma-separated list.

\argument{alpha}
If using gamma distributions, the alpha value for each parameter
must be specified.  If more than one parameter, then specify as a
comma-separated list.

\argument{beta}
If using gamma distributions, the beta value for each parameter
must be specified.  If more than one parameter, then specify as a
comma-separated list.

\argument{useExpr}
If this argument is set to true, then the sampling algorithm will set up random 
  inputs from expression operators such as \verb|AGAUSS| and \verb|AUNIF|.  In 
  this case it will also ignore the list of parameters on the \verb|.EMBEDDEDSAMPLING| command line.
  For a complete description of expression-based random operators, see the expression
  documentation in section~\ref{ExpressionDocumentation}.

\end{Arguments}

\comments

In addition to the \verb|.EMBEDDEDSAMPLING| command, this analysis
requires a \verb|.options EMBEDDEDSAMPLES| command as well.  The
\verb|.EMBEDDEDSAMPLING| command specifies parameters and their
attributes, either using the \verb|useExpr| option, or with 
comma-separated lists.  The \verb|.options EMBEDDEDSAMPLES| command specifies
analysis options, including the number of samples, the type of
sampling (LHS or MC) 
and the outputs and/or measures for which to compute statistics.  
This line also allows one to specify a non-intrusive Polynomial Chaos 
Expansion (PCE) method (either regression or projection PCE).
To see the details of the \verb|.options EMBEDDEDSAMPLES| command , see table~\ref{EmbeddedSamplesPKG}.

On the \verb|.EMBEDDEDSAMPLING| command line, if not using \verb|useExpr|, 
parameters and their
attributes must be specified using comma-separated lists. The
comma-separated lists must all be the same length.

The \texttt{.PRINT ES} command provides output based on the contents
of those print-lines, and also the \texttt{NUMSAMPLES} and \texttt{OUTPUT}
arguments on the \texttt{.OPTIONS EMBEDDEDSAMPLES} line. If the
\texttt{OUTPUT\_SAMPLE\_STATS} argument on a \texttt{.PRINT ES} line is
set to ``true'' then the statistics for the \texttt{MEAN}, \texttt{MEANPLUS},
\texttt{MEANMINUS}, \texttt{STDDEV} and \texttt{VARIANCE} will be output for each
variable in the \texttt{OUTPUT} argument.  If the \texttt{OUTPUT\_ALL\_SAMPLES}
argument on a \texttt{.PRINT ES} line is set to ``true'' then the values
of all \texttt{NUMSAMPLES} samples, for each variable requested
in the \texttt{OUTPUTS} argument, will be output.

\end{Command}



%%%%%%%%%%%%%%%%%%%%%%%%%%%%%%%%%%%%%%%%%%%%%%%%%%%%%%%%%%%%%%%%%%%%%%%%%%%%%%%%
\newpage
\subsection{\texttt{.END} (End of Circuit)}
Marks the end of netlist file.

%%%%%%%%%%%%%%%%%%%%%%%%%%%%%%%%%%%%%%%%%%%%%%%%%%%%%%%%%%%%%%%%%%%%%%%%%%%%%%%%
\newpage
\subsection{\texttt{.ENDS} (End of Subcircuit)}
Marks the end of a subcircuit definition.

%%%%%%%%%%%%%%%%%%%%%%%%%%%%%%%%%%%%%%%%%%%%%%%%%%%%%%%%%%%%%%%%%%%%%%%%%%%%%%%%
\newpage
\subsection{\texttt{.FFT} (FFT Analysis)}
\index{\texttt{.FFT}}
% Sandia National Laboratories is a multimission laboratory managed and
% operated by National Technology & Engineering Solutions of Sandia, LLC, a
% wholly owned subsidiary of Honeywell International Inc., for the U.S.
% Department of Energy’s National Nuclear Security Administration under
% contract DE-NA0003525.

% Copyright 2002-2023 National Technology & Engineering Solutions of Sandia,
% LLC (NTESS).

\label{FFT}
\index{fft}\index{results!fft}

Performs Fast Fourier Transform analysis of transient analysis output.

\begin{Command}

\format
\begin{alltt}
.FFT <ov> [NP=<value>] [WINDOW=<value>] [ALFA=<value>]
+ [FORMAT=<value>] [START=<value>] [STOP=<value>]
+ [FREQ=value] [FMIN=value] [FMAX=value]
\end{alltt}

\examples
\begin{alltt}
.FFT v(1)
.FFT v(1,2) NP=512 WINDOW=HANN
.FFT \{v(3)-v(2)\} START=1 STOP=2
\end{alltt}

\arguments

\begin {Arguments}

\argument{ov}
The desired solution output to be analyzed. Only one
output variable can be specified on a {\tt .FFT} line. However,
multiple {\tt .FFT} lines with the same output variable but, for
example, different windowing functions may be used in a netlist.
The available outputs are:

\begin{itemize}
  \item \texttt{V(<circuit node>)} the voltage at \texttt{<circuit node>}
  \item \texttt{V(<circuit node>,<circuit node>)} to output the voltage
    difference between the first \texttt{<circuit node>} and second
    \texttt{<circuit node>}
  \item \texttt{I(<device>)} the current through a two terminal device
  \item \texttt{I<lead abbreviation>(<device>)} the current into a particular lead
    of a three or more terminal device (see the Comments, below, for details)
  \item \texttt{N(<device parameter>)} a specific device parameter (see the
    individual devices in Section~\ref{Analog_Devices} for syntax)
\end{itemize}

\argument{NP}
The number of points in the FFT.  This value must be a power of 2.  If
the user-entered value is not a power of two then it will be rounded to
the nearest power of 2.  The minimum allowed value of {\tt NP} is 4.
The default value is 1024.

\argument{WINDOW}
The windowing function that will be applied to the sampled waveform
values.  The allowed values are as follows, where table~\ref{FFT_Window_Funcs}
gives their exact definitions:

\begin{itemize}
  \item \texttt{RECT} (or \texttt{RECTANGULAR}) = rectangular window (default)
  \item \texttt{BART} (or \texttt{BARTLETT}) = Bartlett (triangular) window
  \item \texttt{BARTLETTHANN} = Bartlett-Hann window
  \item \texttt{BLACK} = Blackman window
  \item \texttt{BLACKMAN} = ``Conventional Blackman'' window
  \item \texttt{HAMM} (or \texttt{HAMMING}) = Hamming window
  \item \texttt{HANN} (or \texttt{HANNING}) = Hanning window
  \item \texttt{HARRIS} (or \texttt{BLACKMANHARRIS}) = Blackman-Harris window
  \item \texttt{NUTTALL} = Nuttall window
  \item \texttt{COSINE2} = Power-of-cosine window, with exponent 2.
  \item \texttt{COSINE4} = Power-of-cosine window, with exponent 4.
  \item \texttt{HALFCYCLESINE} = Half-cycle sine window
  \item \texttt{HALFCYCLESINE3} = Half-cycle sine window, with exponent 3.
  \item \texttt{HALFCYCLESINE6} = Half-cycle sine window, with exponent 6.
\end{itemize}

\argument{ALFA}
This parameter is supported for HSPICE compatibility. It currently
has no effect though, since the {\tt GAUSS} and {\tt KAISER} windows
are not supported.

\argument{FORMAT}
The allowed values are {\tt NORM} and {\tt UNORM}.  If {\tt NORM} is
selected then the magnitude values will be normalized to the largest
magnitude.  If {\tt UNORM} is selected then the actual magnitude values
will be output instead.  The default value for this parameter depends
on the \texttt{.OPTIONS FFT FFT\_MODE} setting.  If \texttt{FFT\_MODE=0},
which is the default options setting, then the default value is {\tt NORM}.
If \texttt{FFT\_MODE=1} then the default value is {\tt UNORM}.

\argument{START}
The start time for the FFT analysis.  The default value is the
start time for the transient analysis. {\tt FROM} is an allowed
synonym for {\tt START}.

\argument{STOP}
The end time for the FFT analysis.  The default value is the
end time for the transient analysis. {\tt TO} is an allowed
synonym for {\tt STOP}.

\argument{FREQ}
The ``first harmonic'' of the frequencies provided in the output
file.  The default value for {\tt FREQ} is {\tt 1/(STOP - START)}.
If {\tt FREQ} is given then it is rounded to the nearest integer
multiple of the default value.  The \Xyce{} Users' Guide\UsersGuide
provides an example.

\argument{FMIN}
This parameter can use to adjust the harmonics included in the
``additional metrics'' defined in Section \ref{FFT_metrics}.  It
has a default value of 1.

\argument{FMAX}
This parameter can use to adjust the harmonics included in the
``additional metrics'' defined in Section \ref{FFT_metrics}.  It
has a default value of \texttt{NP/2}.

\end{Arguments}

\comments
Multiple \texttt{.FFT} lines may be used in a netlist.  All results from FFT analyses
will be returned to the user in a file with the same name as the netlist file suffixed
with \texttt{.fftX} where {\tt X} is the step number.

\texttt{<lead abbreviation>} is a single character designator for individual
leads on a device with three or more leads.  For bipolar transistors these are:
c (collector), b (base), e (emitter), and s (substrate).  For mosfets, lead
abbreviations are: d (drain), g (gate), s (source), and b (bulk).  SOI
transistors have: d, g, s, e (bulk), and b (body).  For PDE devices, the nodes
are numbered according to the order they appear, so lead currents are
referenced like I1(\texttt{<device>}), I2(\texttt{<device>}), etc.

For this analysis, the phase data is always output in degrees.

In \Xyce{}, \texttt{WINDOW=TRIANGULAR} is not an allowed synonym for
\texttt{WINDOW=BART}.  This is to avoid confusion with other analysis
packages such as SciPy and Matlab.

\end{Command}

\subsubsection{.OPTIONS FFT FFT\_MODE}
\label{FFT_MODE}
\index{fft!options}
The setting for \texttt{.OPTIONS FFT FFT\_MODE} is used to control whether the
\Xyce{} FFT processing and output are more compatible with HSPICE (0) or
Spectre (1).  This setting affects the format of the window functions, the
conversion from two-sided to one-sided results, and whether the default output
for the magnitude values is normalized, or not.  The default setting for
\texttt{FFT\_MODE} is 0.

If \texttt{FFT\_MODE=0} then symmetric window functions are used.  If
\texttt{FFT\_MODE=1} then periodic window functions are used.  The next
subsection provides more details on that difference.

If \texttt{FFT\_MODE=0} then the two-sided to one-sided conversion doubles
the magnitudes of the 1,2,...,NP harmonics. If \texttt{FFT\_MODE=1} then that
conversion only doubles the magnitudes of the 1,2,...,(NP-1) harmonics.

The default value for the \texttt{FORMAT} parameter depends on the
\texttt{.OPTIONS FFT FFT\_MODE} setting.  If \texttt{FFT\_MODE=0} then
the default value for that parameter is {\tt NORM}. If \texttt{FFT\_MODE=1}
then the default value is {\tt UNORM}.

\subsubsection{Window Functions}
\index{fft!window functions}
Table~\ref{FFT_Window_Funcs} gives the definitions of the window functions
implemented in \Xyce{}.  For HSPICE compatibility, the {\tt BLACK} window type
is actually the ``-67 dB Three-Term Blackman-Harris window''~\cite{Doerry2017}
rather than the ``Conventional Blackman Window'' used by Spectre, Matlab and SciPy.
The Convential Blackman Window can be selected with the {\tt BlACKMAN} window
type instead.  The definition of the {\tt BART} window type ~\cite{oppenheimSchafer}
was chosen to match Spectre, Matlab and SciPy.  The \Xyce{} definition may differ
from HSPICE.

As mentioned in in the previous subsection, the choice of symmetric vs. periodic
window functions can be selected via the use of \texttt{.OPTIONS FFT FFT\_MODE=<0|1>}.
If symmetric windows are used then $L = N-1$ in the formulas in
table~\ref{FFT_Window_Funcs}, where $N$ is the number of points in the FFT.
If periodic windows are used then $L=N$.

\begin{longtable}[h] {>{\raggedright\small}m{1.15in}|>{\raggedright\let\\\tabularnewline\small}m{1.35in}
  |>{\raggedright\let\\\tabularnewline\small}m{3.5in}}
  \caption{.FFT Window Function Definitions (N = Number of Points).} \\ \hline
  \rowcolor{XyceDarkBlue}
  \color{white}\bf Value &
  \color{white}\bf Description &
  \color{white}\bf Definition \\ \hline \endfirsthead
  \rowcolor{XyceDarkBlue}
  \color{white}\bf Value &
  \color{white}\bf Description &
  \color{white}\bf Definition \\ \hline \endhead
  \label{FFT_Window_Funcs}

  RECT & Rectangular &
    $w(i)=1, \hspace*{0.1in}\textrm{for}~0 \leq i < N$ \\ \hline
  BART & Barlett~\cite{oppenheimSchafer} &
    $w(i) = \left\{\begin{array}{ll}\frac{2 i}{L},\hspace*{0.1in}\textrm{if}~i < 0.5 \cdot (N-1) \\
                                     2-\frac{2 i}{L},\hspace*{0.1in}\textrm{otherwise}\end{array}\right.$\\ \hline
  BARTLETTHANN & Bartlett-Hann~\cite{Doerry2017}&
    $w(i) = 0.62 - 0.48 \cdot |\frac{i}{L} -0.5| + 0.38 \cdot cos(2 \pi \cdot (\frac{i}{L} -0.5)), \hspace*{0.1in}\textrm{for}~0 \leq i < N $ \\ \hline
  HANN & Hanning~\cite{oppenheimSchafer} &
    $w(i) = sin^2(\frac{\pi i}{L}), \hspace*{0.1in}\textrm{for}~0 \leq i < N$ \\ \hline
  HAMM & Hamming~\cite{oppenheimSchafer} &
    $w(i) = 0.54 - 0.46 \cdot cos(\frac{2 \pi i}{L}), \hspace*{0.1in}\textrm{for}~0 \leq i < N $ \\ \hline
  BLACKMAN & ``Conventional Blackman window''~\cite{Doerry2017} &
    $w(i) = 0.42-0.5 \cdot cos(\frac{2 \pi i}{L})+0.08 \cdot cos(\frac{4 \pi i}{L}), \hspace*{0.1in}\textrm{for}~0 \leq i < N$  \\ \hline
  BLACK & -67 dB Three-Term Blackman-Harris window~\cite{Doerry2017} &
    $w(i) = 0.42323-0.49755 \cdot cos(\frac{2 \pi i}{L})+0.07922 \cdot cos(\frac{4 \pi i}{L}), \hspace*{0.1in}\textrm{for}~0 \leq i < N$  \\ \hline
  HARRIS & -92 dB Four-Term Blackman-Harris window~\cite{Doerry2017} &
    $w(i) = 0.35875 - 0.48829 \cdot cos(\frac{2 \pi i}{L}) + 0.14128 \cdot cos(\frac{4 \pi i}{L}) - 0.01168 \cdot cos(\frac{6 \pi i}{L}), \hspace*{0.1in}\textrm{for}~0 \leq i < N$  \\ \hline
  NUTTALL & Four-Term Nuttall, Minimum Sidelobe (Blackman-Nuttall)~\cite{Doerry2017} &
    $w(i) = 0.3635819 - 0.4891775 \cdot cos(\frac{2 \pi i}{L}) + 0.1365995 \cdot cos(\frac{4 \pi i}{L}) - 0.0106411 \cdot cos(\frac{6 \pi i}{L}), \hspace*{0.1in}\textrm{for}~0 \leq i < N$  \\ \hline
   COSINE2 & Power-of-cosine window, with exponent 2 &
   $w(i) = 0.5 - 0.5 \cdot cos(\frac{2 \pi i}{L}), \hspace*{0.1in}\textrm{for}~0 \leq i < N $ \\ \hline
   COSINE2 & Power-of-cosine window, with exponent 4 &
   $w(i) = 0.375-0.5 \cdot cos(\frac{2 \pi i}{L})+0.125 \cdot cos(\frac{4 \pi i}{L}), \hspace*{0.1in}\textrm{for}~0 \leq i < N$ \\ \hline
   HALFCYCLESINE & Half-cycle sine window & $w(i) = sin(\frac{\pi i}{L}) $ \\ \hline
   HALFCYCLESINE3 & Half-cycle sine window, with exponent 3 & $w(i) = sin^{3}(\frac{\pi i}{L}) $ \\ \hline
   HALFCYCLESINE6 & Half-cycle sine window, with exponent 6 & $w(i) = sin^{6}(\frac{\pi i}{L}) $ \\ \hline
\end{longtable}

\subsubsection{Additional FFT Metrics}
\index{fft!additional metrics}
\label{FFT_metrics}
The following additional metrics will be sent to stdout if \texttt{.OPTIONS FFT FFTOUT=1}
is used in the netlist.  These definitions are also used by the corresponding measure
types on \texttt{.MEASURE FFT} lines.

Define the integer index of the ``first harmonic'' ($f_{0}$) as follows (where it has a
default value of 1 and the $ROUND()$ function rounds to the nearest integer):

$f_{0}=\left\{\begin{array}{ll}ROUND(\frac{FREQ}{STOP-START}),\hspace*{0.1in}\textrm{if FREQ given} \\
 1,\hspace*{1.4in}\textrm{otherwise}\end{array}\right.$

Finally, define $mag[i]$ as the magnitude of the FFT coefficient at index $i$, and $N$ as
the number of points in the FFT.

The Signal to Noise-plus-Distortion Ratio (SNDR) is calculated as follows, where the summation
in the denominator includes all of the frequencies except for the first harmonic frequency $f_{0}$:

$SNDR = 20 \cdot log10(\frac{mag[f_{0}]}{sqrt(\sum mag[i]*mag[i])}),\hspace*{0.1in}\textrm{for}~1 \leq i \leq 0.5 \cdot N ~\textrm{, and}~i \neq f_{0}$

The Effective Number of Bits (ENOB) is calculated as follows, where the units is ``bits'':

$ENOB = \frac{SNDR - 1.76}{6.02}$

For the Signal to Noise Ratio (SNR) metric define the upper frequency limit for the SNR metric as follows,
with the caveat that if $f_{2}$ is less than $f_{0}$ then its value is set to $f_{0}$:

$f_{2}=\left\{\begin{array}{ll}ROUND(\frac{FMAX}{STOP-START}),\hspace*{0.1in}\textrm{if FMAX given}\\
   0.5 \cdot N,\hspace*{1.1in}\textrm{otherwise}\end{array}\right.$

The SNR is then calculated as follows, where the summation in the denominator only uses indexes that
are either not an integer-multiple of $f_{0}$ or that are greater than the upper frequency limit of $f_{2}$.
(Note: for the default case of $f_{0}$ = 1 and \texttt{FMAX} not given there are no ``noise frequencies''.)

$SNR = 20 \cdot log10(\frac{mag[f_{0}]}{sqrt(\sum mag[i]*mag[i])}),\hspace*{0.1in}\textrm{for}~i > f_{2} ~\textrm{ or }~i\%f_{0} \ne 0$

For the Total Harmonic Distortion (THD) metric define the upper frequency limit ($f_{2}$) in
the summation in the numerator as:

$f_{2}=\left\{\begin{array}{ll}ROUND(\frac{FMAX}{STOP-START}),\hspace*{0.1in}\textrm{if FMAX given} \\ 0.5 \cdot N,\hspace*{1.1in}\textrm{otherwise}\end{array}\right.$

The THD is then calculated using only the indexes that are multiples of $f_{0}$.  So, if $f_{0}$=2
then the summation in the numerator would only include indexes 4,6,8,etc.

$THD = 20 \cdot log10(\frac{sqrt(\sum mag[i]*mag[i])}{mag[f_{0}]}),\hspace*{0.1in}\textrm{for}~2 \cdot f_{0} \leq i \leq f_{2} ~\textrm{, and}~i\%f_{0}=0$

For Spurious Free Distortion Ratio (SFDR) metric define the upper and lower frequency
limits ($f_{2}$ and $f_{1}$) considered in the denominator of the calculation as:

$f_{2}=\left\{\begin{array}{ll}ROUND(\frac{FMAX}{STOP-START}),\hspace*{0.1in}\textrm{if FMAX given} \\
 0.5 \cdot N,\hspace*{1.1in}\textrm{otherwise}\end{array}\right.$

and:

$f_{1}=\left\{\begin{array}{ll}ROUND(\frac{FMIN}{STOP-START}),\hspace*{0.1in}\textrm{if FMIN given} \\
 f_{0},\hspace*{1.35in}\textrm{if FMIN not given, and}~f_{2} \geq f_{0}\\
 1,\hspace*{1.4in}\textrm{if FMIN not given, and}~f_{2} < f_{0}\end{array}\right.$

The SFDR is then calculated as:

$SFDR = 20 \cdot log10(\frac{mag[f_{0}]}{MAX(mag[i])}), \hspace*{0.1in}\textrm{for}~f_{1} \leq i \leq f_{2}~\textrm{, and}~i \neq f_{0}$

\subsubsection{Re-Measure}
\label{FFT_ReMeasure}
\index{fft!remeasure}
\Xyce{} can re-calculate (or re-measure) the values for {\tt .FFT} statements
using existing \Xyce{} output files.  Section~\ref{Measure_ReMeasure} discusses
this topic in more detail for both {\tt .MEASURE} and {\tt .FFT} statements. One
additional caveat is that \texttt{FFT\_ACCURATE} is set to 0 during the re-measure
operation.  This should have no impact on the accuracy of the re-measured results
if the output file was previously generated with \texttt{FFT\_ACCURATE} set to 1.

\subsubsection{Compatibility Between OUTPUT and FFT Options}
\index{fft!options}
Some \texttt{.OPTIONS OUTPUT} settings are incompatible with using the default
setting of \texttt{.OPTIONS FFT FFT\_ACCURATE=1}.  The use of
\texttt{.OPTIONS OUTPUT OUTPUTTIMEPOINTS} is compatible.  However, the use of
\texttt{.OPTIONS OUTPUT INITIAL\_INTERVAL} is not.  In that latter case,
\texttt{.OPTIONS FFT FFT\_ACCURATE} will be automatically set to 0.


%%%%%%%%%%%%%%%%%%%%%%%%%%%%%%%%%%%%%%%%%%%%%%%%%%%%%%%%%%%%%%%%%%%%%%%%%%%%%%%%
\newpage
\subsection{\texttt{.FOUR} (Fourier Analysis)}
\index{\texttt{.FOUR}}
% Sandia National Laboratories is a multimission laboratory managed and
% operated by National Technology & Engineering Solutions of Sandia, LLC, a
% wholly owned subsidiary of Honeywell International Inc., for the U.S.
% Department of Energy’s National Nuclear Security Administration under
% contract DE-NA0003525.

% Copyright 2002-2023 National Technology & Engineering Solutions of Sandia,
% LLC (NTESS).



Performs Fourier analysis of transient analysis output.

\begin{Command}

\format
.FOUR <freq> <ov> [ov]*

\examples
\begin{alltt}
.FOUR 100K v(5)
.FOUR 1MEG v(5,3) v(3)
.FOUR 20MEG  SENS
.FOUR 40MEG  \{v(3)-v(2)\}
\end{alltt}

\arguments

\begin {Arguments}

\argument{freq}
The fundamental frequency used for Fourier analysis.
Fourier analysis is performed over the last period (\texttt{1/freq}) of the transient simulation.
The DC component and the first nine harmonics are calculated.  

\argument{ov}
The desired solution output, or outputs, to be analyzed. Fourier analysis can be performed
on several outputs for each fundamental frequency, \texttt{freq}. At least one
output must be specified in the {\tt .FOUR} line.  The available outputs are:

\begin{itemize}
\item \texttt{V(<circuit node>)} the voltage at \texttt{<circuit node>}
\item \texttt{V(<circuit node>,<circuit node>)} to output the voltage difference between the first \texttt{<circuit node>} and second \texttt{<circuit node>}  
\item \texttt{I(<device>)} the current through a two terminal device
\item \texttt{I<lead abbreviation>(<device>)} the current into a particular lead of a three or more terminal device (see the Comments, below, for details)
\item \texttt{N(<device parameter>)} a specific device parameter (see the individual devices in Section~\ref{Analog_Devices} for syntax)
\item \texttt{SENS} transient direct sensitivities (see Section~\ref{SensitivityAnalysis} for more details about setting up the \texttt{.SENS} command)
\end{itemize}

\end{Arguments}

\comments
Multiple \texttt{.FOUR} lines may be used in a netlist.

All results from Fourier analysis will be returned to the user in a file with the
same name as the netlist file suffixed with \texttt{.four\#}, where the suffixed
number (\texttt{\#}) starts at \texttt{0} and increases for multiple
iterations (\texttt{.STEP} iterations) of a given simulation.

\texttt{<lead abbreviation>} is a single character designator for individual
leads on a device with three or more leads.  For bipolar transistors these are:
c (collector), b (base), e (emitter), and s (substrate).  For mosfets, lead
abbreviations are: d (drain), g (gate), s (source), and b (bulk).  SOI
transistors have: d, g, s, e (bulk), and b (body).  For PDE devices, the nodes
are numbered according to the order they appear, so lead currents are
referenced like I1(\texttt{<device>}), I2(\texttt{<device>}), etc.

For this analysis, the phase data is always output in degrees.

\end{Command}



%%%%%%%%%%%%%%%%%%%%%%%%%%%%%%%%%%%%%%%%%%%%%%%%%%%%%%%%%%%%%%%%%%%%%%%%%%%%%%%%
\newpage
\subsection{\texttt{.FUNC} (Function)}
\index{\texttt{.FUNC}}
\index{\texttt{.PARAM}}
% Sandia National Laboratories is a multimission laboratory managed and
% operated by National Technology & Engineering Solutions of Sandia, LLC, a
% wholly owned subsidiary of Honeywell International Inc., for the U.S.
% Department of Energy’s National Nuclear Security Administration under
% contract DE-NA0003525.

% Copyright 2002-2024 National Technology & Engineering Solutions of Sandia,
% LLC (NTESS).


User defined functions that can be used in expressions appearing later 
in the same scope as the \texttt{.FUNC} statement (or \texttt{.PARAM} statement)

\begin{Command}
\format
.FUNC <name>([arg]*) [=] \{<body>\} \\
.PARAM <name>([arg]*) = \{<body>\}

\examples
\begin{alltt}
.FUNC E(x) \{exp(x)\}
.FUNC DECAY(CNST) \{E(-CNST*TIME)\}
.FUNC TRIWAV(x) \{ACOS(COS(x))/3.14159\}
.FUNC MIN3(A,B,C) \{MIN(A,MIN(B,C))\}
.FUNC SUM(A,B,C)=\{A+B+C\}
.PARAM SUM(A,B,C)=\{A+B+C\}
\end{alltt}

\arguments

\begin{Arguments}
\argument{name}

Function name.  Functions cannot be redefined and the function name must 
not be the same as any of the
predefined functions (e.g., \texttt{SIN} and \texttt{SQRT}).  

\argument{arg}

The arguments to the function.  \texttt{.FUNC} arguments cannot be node names.
The number of arguments in the use of a function must agree with the number 
in the definition. Parameters, TIME, FREQ, and other functions are allowed in 
the body of function definitions.  \index{constants (\texttt{EXP},\texttt{PI})}
Two constants \texttt{EXP} and \texttt{PI} cannot
be used a argument names.  These constants are equal to $e$ and $\pi$, respectively,
and cannot be redefined.

\argument{body}

May refer to other (previously defined) functions; the second example, 
DECAY, uses the first example, E. 

\end{Arguments}

\comments

The \texttt{<body>} of a defined function is handled in the same way as any 
math expression; it must be enclosed in curly braces (\{\}) or single quotes (').

The original keyword for user-defined functions is \texttt{.FUNC}.  However, 
for simulator compatibility functions can also be specified using \texttt{.PARAM}.
If using \texttt{.FUNC}, each function must get their own netlist line and the equals sign is optional.  
If using \texttt{.PARAM}, multiple functions can be specified on the same line, and the equals sign is required.

\index{\texttt{.FUNC}!subcircuit scoping}The scoping rules for functions are:
\begin{XyceItemize}
\item If a \texttt{.FUNC}, statement is included in the main circuit 
netlist, then it is accessible from the main circuit and all subcircuits. 
\item \texttt{.FUNC} statements defined within a subcircuit are scoped 
to that subciruit definition.  So, their functions are only accessible within 
that subcircuit definition, as well as within ``nested subcircuits'' also 
defined within that subcircuit definition.
\end{XyceItemize}

Additional illustative examples of scoping are given in the
``Working with Subcircuits and Models'' section of the \Xyce{} Users'
Guide\UsersGuide.

\index{\texttt{.FUNC}!allowed names}Rules for function names are as follows:
\begin{XyceItemize}
\item They should start with a letter or the underscore (\verb|_|) character,
for maximal compatibilty with other Spice-like simulators.  The hash (\verb|#|)
at (\verb|@|) and backtick (\verb|`|) symbols also work, but they are not
reserved characters.
\item These arithmetic operators \verb|%| \verb|^| \verb|&| \verb|~|
\verb|*| \verb|-| \verb|+| \verb|<| \verb|>| \verb|/| \verb+|+ should not
be used anywhere in function names, as they cause problems with expression
parsing.
\item Parentheses (``('' or ``)''), braces (``\{'' or ``\}''), commas,
semi-colons, double quotes and single quotes are also not allowed.
\end{XyceItemize}

\end{Command}



%%%%%%%%%%%%%%%%%%%%%%%%%%%%%%%%%%%%%%%%%%%%%%%%%%%%%%%%%%%%%%%%%%%%%%%%%%%%%%%%
\newpage
\subsection{\texttt{.GLOBAL} (Global Node)}
\index{\texttt{.GLOBAL}}
% Sandia National Laboratories is a multimission laboratory managed and
% operated by National Technology & Engineering Solutions of Sandia, LLC, a
% wholly owned subsidiary of Honeywell International Inc., for the U.S.
% Department of Energy’s National Nuclear Security Administration under
% contract DE-NA0003525.

% Copyright 2002-2023 National Technology & Engineering Solutions of Sandia,
% LLC (NTESS).


\label{GLOBAL_section}

The \texttt{.GLOBAL} command provides another way to designate certain 
nodes as global nodes, besides starting their node name with the two 
characters ``\$G'' as discussed in section \ref{Voltage_Nodes}.  A typical 
usage of such global nodes is to define a VDD or VSS signal that all
subcircuits need to be able to access, but without having to provide 
VSS and VDD input nodes to every subcircuit. 

\begin{Command}

\format
\begin{alltt}
.GLOBAL <node>
\end{alltt}

\examples
\begin{alltt}
.GLOBAL g1
.subckt rsub  a  b
Rab  a  b  2
* since node G1 is global, it may be used here without
* being listed on the .subckt line
Rbg  G1  b  3  
.ends
\end{alltt}

\comments
The name of the global node can be any legal node name, per 
section \ref{legalCharacters}.

\end{Command}



%%%%%%%%%%%%%%%%%%%%%%%%%%%%%%%%%%%%%%%%%%%%%%%%%%%%%%%%%%%%%%%%%%%%%%%%%%%%%%%%
\newpage
\subsection{\texttt{.GLOBAL\_PARAM} (Global parameter)}
\index{\texttt{.GLOBAL\_PARAM}}
% Sandia National Laboratories is a multimission laboratory managed and
% operated by National Technology & Engineering Solutions of Sandia, LLC, a
% wholly owned subsidiary of Honeywell International Inc., for the U.S.
% Department of Energy’s National Nuclear Security Administration under
% contract DE-NA0003525.

% Copyright 2002-2024 National Technology & Engineering Solutions of Sandia,
% LLC (NTESS).

User-defined global parameter that can be used in expressions throughout the netlist.
\begin{Command}

\format
.GLOBAL\_PARAM [<name>=<value>]*

\examples
.GLOBAL\_PARAM T=\{27+100*time\}

\begin{Arguments}
\argument{name}

Global parameter name.  Global parameters may be redefined.  
If the same name is used on multiple parameters, \Xyce{} by default will use the last parameter of that name.  
  By default, no warning will be emitted.
To change this behavior, one can use the \texttt{-redefined\_param} command line option, described in section~\ref{cmd_line_arg_list}.

\argument{value}

The value may be a number or an expression.

\end{Arguments}

\comments

A \texttt{.PARAM} 
defined in the top level netlist is equivalent to 
a \texttt{.GLOBAL\_PARAM}, and they can be combined as needed.
Thus, you may use parameters defined by \texttt{.PARAM} in expressions used to
define global parameters, and you may also use global parameters in
\texttt{.PARAM} definitions.    However, a \texttt{.GLOBAL\_PARAM} 
  can only depend on \texttt{.PARAM} parameter from the top level circuit scope.

Like \texttt{.PARAM} parameters, \texttt{.GLOBAL\_PARAM} may
depend on time dependent quantities in the circuit.  They may also
be frequency dependent.  They cannot, however, be 
dependent on solution variaables such as voltage nodes.

To load an external data file with time voltage pairs of data on each 
line into a global parameter, use this syntax:

\texttt{.GLOBAL\_PARAM extdata = \{tablefile("filename")\}}

or

\texttt{.GLOBAL\_PARAM extdata = \{table("filename")\}}

where \texttt{filename} would be the name of the file to load.  
Other interpolators that can read in a data table from a file 
include \texttt{fasttable},\texttt{spline}, \texttt{akima}, \texttt{cubic}, 
\texttt{wodicka} and \texttt{bli}.  See \ref{ExpressionDocumentation} 
for further information. 

There are several reserved words that may not be used as names for parameters.  These reserved words are:
\begin{XyceItemize}
\item \verb+Time+
\item \verb+Freq+ 
\item \verb+Hertz+ 
\item \verb+Vt+
\item \verb+Temp+
\item \verb+Temper+
\item \verb+GMIN+
\end{XyceItemize}

Global parameters are accessible, and have the same value, throughout all
levels of the netlist hierarchy.  It is not legal to redefine global parameters
in different levels of the netlist hierarchy.  Also, global parameters can only 
  be defined in the top level circuit scope.   Parameters defined inside of 
  subcircuits must be of the \texttt{.PARAM} type.

\end{Command}


%%%%%%%%%%%%%%%%%%%%%%%%%%%%%%%%%%%%%%%%%%%%%%%%%%%%%%%%%%%%%%%%%%%%%%%%%%%%%%%%
\newpage
\subsection{\texttt{.HB} (Harmonic Balance Analysis)}
% Sandia National Laboratories is a multimission laboratory managed and
% operated by National Technology & Engineering Solutions of Sandia, LLC, a
% wholly owned subsidiary of Honeywell International Inc., for the U.S.
% Department of Energy’s National Nuclear Security Administration under
% contract DE-NA0003525.

% Copyright 2002-2024 National Technology & Engineering Solutions of Sandia,
% LLC (NTESS).


\index{\texttt{.HB}}
\index{analysis!HB} \index{harmonic balance analysis}
Calculates steady states of nonlinear circuits in the frequency domain.

\begin{Command}

\format
.HB <fundamental frequencies>

\examples
.HB 1e4

.hb 1e4 2e2
\arguments

\begin{Arguments}
\argument{fundamental frequencies}
Sets the fundamental frequencies for the analysis.

\end{Arguments}

\comments

Harmonic balance analysis calculates the magnitude and phase of voltages
and currents in a nonlinear circuit. Use a \texttt{.OPTIONS HBINT}
statement to set additional harmonic balance analysis options.

The \texttt{.PRINT HB}\index{\texttt{.PRINT}}\index{results!print}\index{\texttt{.PRINT}!\texttt{HB}}
statement must be used to get the results of the harmonic balance analysis. 
See section \ref{.PRINT}.

Some devices that may be expected to work in HB analysis do not at this time.  
  This includes some use cases of B sources (but not all).  A time-dependent B 
  source will not work with HB.  However, a B source that is purely dependent 
  (such as a nonlinear resistor) will work.    This same guidance applies to the 
  E,F,G, and H dependent sources.

\end{Command}


%%%%%%%%%%%%%%%%%%%%%%%%%%%%%%%%%%%%%%%%%%%%%%%%%%%%%%%%%%%%%%%%%%%%%%%%%%%%%%%%
\newpage
\subsection{\texttt{.IC} (Initial Condition, Bias point)}
\index{\texttt{.IC}}
% Sandia National Laboratories is a multimission laboratory managed and
% operated by National Technology & Engineering Solutions of Sandia, LLC, a
% wholly owned subsidiary of Honeywell International Inc., for the U.S.
% Department of Energy’s National Nuclear Security Administration under
% contract DE-NA0003525.

% Copyright 2002-2023 National Technology & Engineering Solutions of Sandia,
% LLC (NTESS).


\label{IC_section}
\index{\texttt{.IC}}
\index{\texttt{.DCVOLT}}
\index{initial condition!IC}
\index{initial condition!DCVOLT}
\index{initial condition}

The \texttt{.IC/.DCVOLT} command sets initial conditions for operating point calculations.
These operating point conditions will be enforced the entire way through the
nonlinear solve.  Initial conditions can be given for some or all of the
circuit nodes.

As the conditions are enforced for the entire solve, only the nodes not
specified with \texttt{.IC} statements will change over the course of the
operating point calculation.

Note that it is possible to specify conditions that are not solvable.
Consult the \Xyce{} Users' Guide~\UsersGuide{} for more guidance.

\begin{Command}
\format
\begin{alltt}
.IC V(<node>)=<value>
.IC <node> <value>
.DCVOLT V(<node>)=<value>
.DCVOLT <node> <value>
\end{alltt}

\examples
\begin{alltt}
.IC V(2)=3.1
.IC 2 3.1
.DCVOLT V(2)=3.1
.DCVOLT 2 3.1
\end{alltt}

\comments
The \texttt{.IC} capability can only set voltage values, not current values.

The \texttt{.IC} capability can not be used within subcircuits to set
voltage values on global nodes.

\end{Command}



%%%%%%%%%%%%%%%%%%%%%%%%%%%%%%%%%%%%%%%%%%%%%%%%%%%%%%%%%%%%%%%%%%%%%%%%%%%%%%%%
\newpage
\subsection{\texttt{.INC or .INCLUDE or .INCL} (Include file)}
\index{\texttt{.INC}}
% Sandia National Laboratories is a multimission laboratory managed and
% operated by National Technology & Engineering Solutions of Sandia, LLC, a
% wholly owned subsidiary of Honeywell International Inc., for the U.S.
% Department of Energy’s National Nuclear Security Administration under
% contract DE-NA0003525.

% Copyright 2002-2024 National Technology & Engineering Solutions of Sandia,
% LLC (NTESS).


Include specified file in netlist.

The file name can be surrounded by single or double quotes, 'filename'
or "filename", but this is not necessary.  The directory for the
include file is assumed to be the execution directory unless a full or
relative path is given as a part of the file name.

\begin{Command}
\format
\begin{alltt}
.INC <include file name>
.INCLUDE <include file name>
.INCL <include file name>
\end{alltt}

\examples
\begin{alltt}
.INC models.lib
.INC 'models.lib'
.INC "models.lib"
.INCLUDE models.lib
.INCLUDE 'models.lib'
.INCLUDE "path\_to\_library/models.lib"
\end{alltt}

\comments
If {\texttt <include file name>} uses an absolute path then that path
is used.  Otherwise, the search-path order for {\texttt <include file name>} is:
\begin{XyceItemize}
  \item Relative to the directory that contains {\texttt <include file name>}.
  \item Relative to the directory that contains the file with the top-level
netlist.
  \item Relative to the \Xyce{} execution directory.
\end{XyceItemize}

\end{Command}


%%%%%%%%%%%%%%%%%%%%%%%%%%%%%%%%%%%%%%%%%%%%%%%%%%%%%%%%%%%%%%%%%%%%%%%%%%%%%%%%
\newpage
\subsection{\texttt{.LIB} (Library file)}
\index{\texttt{.LIB}}
% Sandia National Laboratories is a multimission laboratory managed and
% operated by National Technology & Engineering Solutions of Sandia, LLC, a
% wholly owned subsidiary of Honeywell International Inc., for the U.S.
% Department of Energy’s National Nuclear Security Administration under
% contract DE-NA0003525.

% Copyright 2002-2023 National Technology & Engineering Solutions of Sandia,
% LLC (NTESS).


The \texttt{.LIB} command is similar to \texttt{.INCLUDE}, in that it
brings in an external file.  However, it is
designed to only bring in specific parts of a library file, as designated
by an entry name.
Note that the \Xyce{} version of \texttt{.LIB} has been designed to be compatible
with HSPICE~\cite{Hspice}, not PSpice~\cite{Pspice}.

There are two forms of the \texttt{.LIB} statement, the call and the
definition.  The call statement reads in a specified subset of a
library file, and the definition statement defines the subsets.

\subsubsection{.LIB call statement}
\begin{Command}
\format
.LIB <file name> <entry name>

\examples
\begin{alltt}
.LIB models.lib nom
.LIB 'models.lib'  low
.LIB "models.lib"  low
.LIB "path/models.lib"  high
\end{alltt}

\arguments

\begin{Arguments}

\argument{file name}
Name of file containing netlist data.  Single or double quotes
(\texttt{"} or \texttt{'}) may be used around the file name.

\argument{entry name}
Entry name, which determines the section of the file to be included.
These sections are defined in the included file using the definition
form of the \texttt{.LIB} statement.

\end{Arguments}
\end{Command}

The library file name can be surrounded by quotes (single or double),
as in "path/filename" but this is not necessary.  The directory for
the library file is assumed to be the execution directory unless a
full or relative path is given as a part of the file name.  The
section name denotes the section or sections of the library file to
include.

If {\texttt <file name>} uses an absolute path then that path is used.
Otherwise, the search-path order for {\texttt <file name>} is:
\begin{XyceItemize}
  \item Relative to the directory that contains {\texttt <file name>}.
  \item Relative to the directory that contains the file with the top-level
netlist.
  \item Relative to the \Xyce{} execution directory.
\end{XyceItemize}

\subsubsection{.LIB definition statement}
The format given above is when the \texttt{.LIB} command is used to reference
a library file; however, it is also used as part of the syntax in a library
file. 

\begin{Command}
\format
\begin{alltt}
.LIB <entry name>
<netlist lines>*
.endl <entry name>
\end{alltt}

\examples
\begin{alltt}

* Library file res.lib
.lib low
.param rval=2
r3  2  0  9
.endl low

.lib nom
.param rval=3
r3  2  0  8
.endl nom
\end{alltt}

\arguments
\begin{Arguments}
\argument{entry name}
The name to be used to identify this library component.  When used on a \texttt{.LIB} call line, these segments of the library file will be included in the calling file.
\end{Arguments}
\end{Command}

Note that for each entry name, there is a matched \texttt{.lib}
and \texttt{.endl}.  Any valid netlist commands can be placed inside
the \texttt{.lib} and \texttt{.endl} statements.  The following is an
example calling netlist, which refers to the library in the examples above:

\begin{centering}
\shadowbox{
\begin{minipage}{0.8\textwidth}
\begin{vquote}
\color{blue}* Netlist file res.cir\color{black}
V1  1   0   1
R   1   2   \{rval\}
.lib res.lib nom
.tran 1 ps 1ns
.end
\end{vquote}
\end{minipage}
}
\end{centering}

In this example, only the netlist commands that are inside of the
``nom'' library will be parsed, while the commands inside of the
``low'' library will be discarded.  As a result, the value for
resistor r3 is 8, and the value for rval is 3.


%%%%%%%%%%%%%%%%%%%%%%%%%%%%%%%%%%%%%%%%%%%%%%%%%%%%%%%%%%%%%%%%%%%%%%%%%%%%%%%%
\newpage
\subsection{\texttt{.LIN} (Linear Analysis)}
\index{\texttt{.LIN}}
% Sandia National Laboratories is a multimission laboratory managed and
% operated by National Technology & Engineering Solutions of Sandia, LLC, a
% wholly owned subsidiary of Honeywell International Inc., for the U.S.
% Department of Energy’s National Nuclear Security Administration under
% contract DE-NA0003525.

% Copyright 2002-2023 National Technology & Engineering Solutions of Sandia,
% LLC (NTESS).


Extracts linear transfer parameters (S-, Y- and Z-parameters) for a general
multiport network.  Those parameters can be output in either
Touchstone format \cite{touchstone2_std_2009}.

\begin{Command}

\format
\begin{alltt}
.LIN [SPARCALC=<1|0>] [FORMAT=<TOUCHSTONE2|TOUCHSTONE>]
+ [LINTYPE=<S|Y|Z>] [DATAFORMAT=<RI|MA|DB>]
+ [FILE=<output filename>] [WIDTH=<print field width>]
+ [PRECISION=<floating point output precision>]
\end{alltt}

\examples
\begin{alltt}
.LIN
.LIN FORMAT=TOUCHSTONE DATAFORMAT=MA FILE=foo
\end{alltt}

\arguments

\begin{Arguments}

\argument{SPARCALC=<1|0>}
If this is set to 1 then the \texttt{LIN} analysis is done
at the frequency values specified on the \texttt{.AC} line.
The default value is 1.

\argument{FORMAT=<TOUCHSTONE2|TOUCHSTONE>} Output file format
\begin{description}
\item[\tt TOUCHSTONE] Output file is in Touchstone 1 format

\item[\tt TOUCHSTONE2] Output file is in Touchstone 2 format. The default
is \texttt{TOUCHSTONE2}.
\end{description}

\argument{LINTYPE=<S|Y|Z>} The type of parameter data (S, Y or Z) in the
output file.  The default is S.

\argument{DATAFORMAT=<RI|MA|DB>} Format for the S-, Y- or Z-parameter data

\begin{description}
\item[\tt RI] Real-imaginary format \\
The data is output as the real and imaginary parts for each
extracted S-, Y- or Z-parameter.  This is the default.

\item[\tt MA] Magnitude-angle format \\
The data is output as the magnitude and the phase angle of each
extracted S-, Y- or Z-parameter.  For compatibility with Touchstone formats,
the angle values are in degrees.

\item[\tt DB] Magnitude(dB)-angle format \\
The data is output as the magnitude (in dB) and the phase angle of
each extracted S-, Y- or Z-parameter.  For compatibility with Touchstone
formats, the angle values are in degrees.
\end{description}

\argument{FILE=<output filename>}
Specifies the name of the file to which the output will be written.
For HSPICE compatibility \texttt{FILENAME=} is an allowed synonym
for \texttt{FILE=} on \texttt{.LIN} lines.

\argument{WIDTH=<print field width>}

Controls the output width used in formatting the output.

\argument{PRECISION=<floating point precision>}

Number of floating point digits past the decimal for output data.

\end{Arguments}


\comments

The \texttt{.LIN} command line functions like a \texttt{.PRINT} line for
the extracted S-, Y- or Z-parameter data.  So, a netlist can have multiple
\texttt{.LIN} lines with different values for the \texttt{LINTYPE},
\texttt{DATAFORMAT} and \texttt{FILE} arguments on each line.  If there are
multiple \texttt{.LIN} lines in the netlist, then a linear analysis will
be performed if \texttt{SPARCALC=1} on any of those \texttt{.LIN} lines.

The default filename for both Touchstone formats is \texttt{<netlistName>.sNp}
where N is the number of ``ports'' (\texttt{P} devices) specified in the netlist.

The \Xyce{} Touchstone output is based on the Touchstone standard
\cite{touchstone2_std_2009}. So, it differs slightly from the
corresponding HSPICE output.  In particular, the full matrix of
S-, Y- or Z-parameters is always output.

The HSPICE \texttt{SPARDIGIT} and \texttt{FREQDIGIT} arguments are
not supported.  Instead, the \texttt{PRECISION} argument is used for
all of the output values.

The output of individual S-parameters via the \texttt{.PRINT AC} line is
supported.

If the \texttt{-r <raw-file-name>} and \texttt{-a} command line options
are used with \texttt{.LIN} with \texttt{SPARCALC=1} then \Xyce{} will
exit with a parsing error.

The \texttt{-o} command line option can be used with \texttt{.LIN}.
In that case, the output defaults to Touchstone 2 format and any
\texttt{FILE=<filename>} argument on the \texttt{.LIN} line is
ignored.

\end{Command}


%%%%%%%%%%%%%%%%%%%%%%%%%%%%%%%%%%%%%%%%%%%%%%%%%%%%%%%%%%%%%%%%%%%%%%%%%%%%%%%%
\newpage
\subsection{\texttt{.MEASURE or .MEAS} (Measure output)}
\index{output!control}
\index{results!output control}
\index{\texttt{.MEASURE}}
% Sandia National Laboratories is a multimission laboratory managed and
% operated by National Technology & Engineering Solutions of Sandia, LLC, a
% wholly owned subsidiary of Honeywell International Inc., for the U.S.
% Department of Energy’s National Nuclear Security Administration under
% contract DE-NA0003525.

% Copyright 2002-2023 National Technology & Engineering Solutions of Sandia,
% LLC (NTESS).

\label{Measure_section}

The \texttt{.MEASURE} statement allows calculation or reporting of simulation 
metrics to an external file, as well as to the standard output and/or a log file.  One can 
measure when simulated signals reach designated values, or when they are equal
to other simulation values.  The \texttt{.MEASURE} statement is supported for 
\texttt{.TRAN}, \texttt{.DC}, \texttt{.AC} and \texttt{.NOISE} analyses.  It can
be used with {\tt .STEP}  in all four cases.  For HSPICE compatibility,
\texttt{.MEAS} is an allowed synonym for \texttt{.MEASURE}.

The syntaxes for the \texttt{.MEASURE} statements are shown below.  The \texttt{AVG},
\texttt{DERIV}, \texttt{EQN},   \texttt{ERR},  \texttt{ERR1}, \texttt{ERR2}, \texttt{FIND-AT},
\texttt{FIND-WHEN}, \texttt{INTEG}, \texttt{MIN}, \texttt{MAX}, \texttt{PP}, \texttt{RMS},
\texttt{WHEN} and \texttt{TRIG-TARG} measures are supported for all four ``measure modes''
(\texttt{TRAN}, \texttt{AC}, \texttt{DC} and \texttt{NOISE}).  Note that each measure
type (e.g., \texttt{MAX}) may be listed
twice. This is because only a subset of the allowed ``qualifiers'' (e.g., \texttt{FROM} and
\texttt{TO}) may be supported for the \texttt{AC}, \texttt{DC} and \texttt{NOISE} measure modes.

The \texttt{ERROR} measure is \Xyce{}-specific, and is supported for \texttt{TRAN}, \texttt{AC},
\texttt{DC} and \texttt{NOISE} measure modes.  The \texttt{DUTY}, \texttt{FREQ}, \texttt{FOUR},
\texttt{OFF\_TIME} and \texttt{ON\_TIME} measures are also \Xyce{}-specific, and are  only
supported for \texttt{TRAN} measure mode.

\begin{Command}
\format
\begin{alltt}
.MEASURE TRAN <result name> AVG <variable>
+ [MIN_THRESH=<value>] [MAX_THRESH=<value>]
+ [FROM=<time>] [TO=<time>] [TD=<time>]
+ [DEFAULT_VAL=<value>] [PRECISION=<value>] [PRINT=<value>]

.MEASURE TRAN <result name> DERIV <variable> AT=<value>
+ [MINVAL=<value>] [DEFAULT_VAL=<value>] 
+ [PRECISION=<value>] [PRINT=<value>]

.MEASURE TRAN <result name> DERIV <variable>
+ WHEN <variable>=<variable\(\sb{2}\)>|<value>
+ [MINVAL=<value>] [FROM=<value>] [TO=<value>] [TD=<value>] 
+ [RISE=r|LAST] [FALL=f|LAST] [CROSS=c|LAST]
+ [DEFAULT_VAL=<value>] [PRECISION=<value>] [PRINT=<value>]

.MEASURE TRAN <result name> DUTY <variable>
+ [ON=<value>] [OFF=<value>] [MINVAL=<value>]
+ [FROM=<value>] [TO=<value>] [TD=<value>]
+ [DEFAULT_VAL=<value>] [PRECISION=<value>] [PRINT=<value>]

.MEASURE TRAN <result name> EQN <variable>
+ [FROM=<value>] [TO=<value>] [TD=<value>]
+ [DEFAULT_VAL=<value>] [PRECISION=<value>] [PRINT=<value>]

.MEASURE TRAN <result name> <ERR|ERR1|ERR2>
+ <variable\(\sb{1}\)> <variable\(\sb{2}\)> [FROM=<value>] [TO=<value>]
+ [MINVAL=<value>] [IGNOR|YMIN=<value>] [YMAX=<value>]
+ [DEFAULT_VAL=<value>] [PRECISION=<value>] [PRINT=<value>]

.MEASURE TRAN <result name> ERROR <variable> FILE=<value>
+ INDEPVARCOL=<value> DEPVARCOL=<value> [COMP_FUNCTION=<value>]
+ [DEFAULT_VAL=<value>] [PRECISION=<value>] [PRINT=<value>]

.MEASURE TRAN <result name> FIND <variable> AT=<value>
+ [MINVAL=<value>] [DEFAULT_VAL=<value>]
+ [PRECISION=<value>] [PRINT=<value>]

.MEASURE TRAN <result name> FIND <variable>
+ WHEN <variable>=<variable\(\sb{2}\)>|<value>
+ [FROM=<value>] [TO=<value>] [TD=<value>] 
+ [RISE=r|LAST] [FALL=f|LAST] [CROSS=c|LAST]
+ [MINVAL=<value>] [DEFAULT_VAL=<value>] 
+ [PRECISION=<value>] [PRINT=<value>]

.MEASURE TRAN <result name> FOUR <variable> AT=freq
+ [NUMFREQ=<value>] [GRIDSIZE=<value>]
+ [FROM=<value>] [TO=<value>] [TD=<value>] 
+ [DEFAULT_VAL=<value>] [PRECISION=<value>] [PRINT=<value>]

.MEASURE TRAN <result name> FREQ <variable>
+ [ON=<value>] [OFF=<value>] [MINVAL=<value>]
+ [FROM=<value>] [TO=<value>] [TD=<value>] 
+ [DEFAULT_VAL=<value>] [PRECISION=<value>] [PRINT=<value>]

.MEASURE TRAN <result name> INTEG <variable>
+ [FROM=<value>] [TO=<value>] [TD=<value>]
+ [DEFAULT_VAL=<value>] [PRECISION=<value>] [PRINT=<value>]

.MEASURE TRAN <result name> MAX <variable>
+ [FROM=<value>] [TO=<value>] [TD=<value>]
+ [RISE=r|LAST] [FALL=f|LAST] [CROSS=c|LAST] [RFC_LEVEL=<value>]
+ [DEFAULT_VAL=<value>] [PRECISION=<value>] 
+ [PRINT=<value>] [OUTPUT=<value>]

.MEASURE TRAN <result name> MIN <variable>
+ [FROM=<value>] [TO=<value>] [TD=<value>]
+ [RISE=r|LAST] [FALL=f|LAST] [CROSS=c|LAST] [RFC_LEVEL=<value>]
+ [DEFAULT_VAL=<value>] [PRECISION=<value>] 
+ [PRINT=<value>] [OUTPUT=<value>]

.MEASURE TRAN <result name> OFF_TIME <variable>
+ [OFF=<value>] [MINVAL=<value>]
+ [FROM=<value>] [TO=<value>] [TD=<value>]
+ [DEFAULT_VAL=<value>] [PRECISION=<value>] [PRINT=<value>]

.MEASURE TRAN <result name> ON_TIME <variable>
+ [ON=<value>] [MINVAL=<value>]
+ [FROM=<value>] [TO=<value>] [TD=<value>]
+ [DEFAULT_VAL=<value>] [PRECISION=<value>] [PRINT=<value>]

.MEASURE TRAN <result name> PP <variable>
+ [FROM=<value>] [TO=<value>] [TD=<value>]
+ [RISE=r|LAST] [FALL=f|LAST] [CROSS=c|LAST] [RFC_LEVEL=<value>]
+ [DEFAULT_VAL=<value>] [PRECISION=<value>] [PRINT=<value>]

.MEASURE TRAN <result name> RMS <variable>
+ [FROM=<value>] [TO=<value>] [TD=<value>]
+ [DEFAULT_VAL=<value>] [PRECISION=<value>] [PRINT=<value>]

.MEASURE TRAN <result name> WHEN <variable>=<variable\(\sb{2}\)>|<value>
+ [FROM=<value>] [TO=<value>] [TD=<value>] 
+ [RISE=r|LAST] [FALL=f|LAST] [CROSS=c|LAST]
+ [MINVAL=<value>] [DEFAULT_VAL=<value>] 
+ [PRECISION=<value>] [PRINT=<value>]

.MEASURE <AC|DC|NOISE|TRAN> <result name>
+ TRIG <variable\(\sb{1}\)>=<variable\(\sb{2}\)>|<value> 
+ [TD=<val>] [RISE=r] [FALL=f] [CROSS=c]
+ TARG <variable\(\sb{3}\)>=<variable\(\sb{4}\)>|<value> 
+ [TD=<val>] [RISE=r] [FALL=f] [CROSS=c]
+ [MINVAL=<value>] [DEFAULT_VAL=<value>]
+ [PRECISION=<value>] [PRINT=<value>]

.MEASURE <AC|DC|NOISE|TRAN> <result name>
+ TRIG AT=<value> TARG AT=<value> 
+ [MINVAL=<value>] [DEFAULT_VAL=<value>]
+ [PRECISION=<value>] [PRINT=<value>]

.MEASURE TRAN <result name>
+ TRIG <variable\(\sb{1}\)> FRAC\_MAX=<value> 
+ [RISE=r] [FALL=f] [CROSS=c]
+ [FROM=<value>] [TO=<value>] [TD=<value>] 
+ TARG <variable\(\sb{2}\)> FRAC\_MAX=<value> 
+ [RISE=r] [FALL=f] [CROSS=c]
+ [FROM=<value>] [TO=<value>] [TD=<value>] 
+ [MINVAL=<value>] [DEFAULT_VAL=<value>]
+ [PRECISION=<value>] [PRINT=<value>]

.MEASURE <AC|DC|NOISE> <result name> AVG <variable>
+ [FROM=<value>] [TO=<value>]
+ [DEFAULT_VAL=<value>] [PRECISION=<value>] [PRINT=<value>]

.MEASURE <AC|DC|NOISE> <result name> DERIV <variable> AT=<value>
+ [MINVAL=<value>] [DEFAULT_VAL=<value>]
+ [PRECISION=<value>] [PRINT=<value>]

.MEASURE <AC|DC|NOISE> <result name> DERIV <variable>
+ WHEN <variable>=<variable\(\sb{2}\)>|<value>
+ [MINVAL=<value>] [FROM=<value>] [TO=<value>]
+ [RISE=r|LAST] [FALL=f|LAST] [CROSS=c|LAST]
+ [DEFAULT_VAL=<value>] [PRECISION=<value>] [PRINT=<value>]

.MEASURE <AC|DC|NOISE> <result name> EQN <variable>
+ [FROM=<value>] [TO=<value>] 
+ [DEFAULT_VAL=<value>] [PRECISION=<value>] [PRINT=<value>]

.MEASURE <AC|DC|NOISE> <result name> <ERR|ERR1|ERR2>
+ <variable\(\sb{1}\)> <variable\(\sb{2}\)> [FROM=<value>] [TO=<value>]
+ [MINVAL=<value>] [IGNOR|YMIN=<value>] [YMAX=<value>]
+ [DEFAULT_VAL=<value>] [PRECISION=<value>] [PRINT=<value>]

.MEASURE <AC|DC|NOISE> <result name> ERROR <variable>
+ FILE=<value> [DEPVARCOL=<value>] [COMP_FUNCTION=<value>]
+ [DEFAULT_VAL=<value>] [PRECISION=<value>] [PRINT=<value>]

.MEASURE <AC|DC|NOISE> <result name> FIND <variable> AT=<value>
+ [MINVAL=<value>] [DEFAULT_VAL=<value>]
+ [PRECISION=<value>] [PRINT=<value>]

.MEASURE <AC|DC|NOISE> <result name> FIND <variable>
+ WHEN <variable>=<variable\(\sb{2}\)>|<value>
+ [FROM=<value>] [TO=<value>]
+ [RISE=r|LAST] [FALL=f|LAST] [CROSS=c|LAST]
+ [MINVAL=<value>] [DEFAULT_VAL=<value>]
+ [PRECISION=<value>] [PRINT=<value>]

.MEASURE <AC|DC|NOISE> <result name> INTEG <variable>
+ [FROM=<value>] [TO=<value>]
+ [DEFAULT_VAL=<value>] [PRECISION=<value>] [PRINT=<value>]

.MEASURE <AC|DC|NOISE> <result name> MAX <variable>
+ [FROM=<value>] [TO=<value>] 
+ [DEFAULT_VAL=<value>] [PRECISION=<value>] 
+ [PRINT=<value>] [OUTPUT=<value>]

.MEASURE <AC|DC|NOISE> <result name> MIN <variable>
+ [FROM=<value>] [TO=<value>] 
+ [DEFAULT_VAL=<value>] [PRECISION=<value>]
+ [PRINT=<value>] [OUTPUT=<value>]

.MEASURE <AC|DC|NOISE> <result name> PP <variable>
+ [FROM=<value>] [TO=<value>] 
+ [DEFAULT_VAL=<value>] [PRECISION=<value>] [PRINT=<value>]

.MEASURE <AC|DC|NOISE> <result name> RMS <variable>
+ [FROM=<value>] [TO=<value>]
+ [DEFAULT_VAL=<value>] [PRECISION=<value>] [PRINT=<value>]

.MEASURE <AC|DC|NOISE> <result name>
+ WHEN <variable>=<variable\(\sb{2}\)>|<value>
+ [FROM=<value>] [TO=<value>]
+ [RISE=r|LAST] [FALL=f|LAST] [CROSS=c|LAST]
+ [MINVAL=<value>] [DEFAULT_VAL=<value>]
+ [PRECISION=<value>] [PRINT=<value>]
\end{alltt}
\index{\texttt{.MEASURE}}
\index{results!measure}

\examples
\begin{alltt}
.MEASURE TRAN hit1_75 WHEN V(1)=0.75 MINVAL=0.02
.MEASURE TRAN hit2_75 WHEN V(1)=0.75 MINVAL=0.08 RISE=2
.MEASURE TRAN avgAll AVG V(1)
.MEASURE TRAN dutyAll DUTY V(1) ON=0.75 OFF=0.25
.MEASURE DC maxV1 MAX V(1)
.MEAS DC minV2 MIN V(2)
.MEASURE AC maxV1R MAX VR(1)
.MEASURE NOISE maxonoise MAX ONOISE
\end{alltt}

\arguments

\begin{Arguments}
\argument{result name}

Measured results are reported to the output and log file.
Additionally, for \texttt{TRAN} measures, the results are stored in
files called \texttt{circuitFileName.mt\#}, where the suffixed number
(\texttt{\#}) starts at \texttt{0} and increases for multiple
iterations (\texttt{.STEP} iterations) of a given simulation. Each
line of this file will contain the measurement name, \texttt{<result
name>}, followed by its value for that run.  The \texttt{<result
name>} must be a legal \Xyce{} character string.  For \texttt{DC} measures,
the results are stored in the files \texttt{circuitFileName.ms\#},
while \texttt{AC} and \texttt{NOISE} measures use the files
\texttt{circuitFileName.ma\#}.

If multiple measures are defined with the same \texttt{<result name>} then
\Xyce{} uses the last such definition, and issues warning messages about
(and discards) any previous measure definitions with the same
\texttt{<result name>}.

\argument{measure type}

\texttt{AVG, DERIV, DUTY, EQN, ERR, ERR1, ERR2, ERROR, FIND, FREQ, FOUR, INTEG, MAX, MIN, OFF\_TIME, ON\_TIME, PP, RMS, WHEN, TRIG, TARG}

The third argument specifies the type of measurement or calculation to
be done. The only exception is the {\tt TARG} clause which comes later
in the argument list, after the {\tt TRIG} clause has been specified.

By default, the measurement is performed over the entire simulation.
The calculations can be limited to a specific measurement window by
using the qualifiers {\tt FROM}, {\tt TO}, {\tt TD}, {\tt RISE}, {\tt
FALL}, {\tt CROSS} and {\tt MINVAL}, which are explained below.

The supported measure types and their definitions are:

\begin{description}
  \item[\tt AVG] Computes the arithmetic mean of {\tt <variable>} for
    the simulation, or within the extent of the measurement window.
    The measurement window can be limited with the qualifiers {\tt FROM},
    {\tt TO} and {\tt TD} for {\tt TRAN} measures, and with {\tt FROM}
    and {\tt TO} for {\tt AC}, {\tt DC} and {\tt NOISE} measures.

  \item[\tt DERIV] Computes the derivative of {\tt <variable>} at a
    user-specified time (by using the {\tt AT} qualifier) or when a
    user-specified condition occurs (by using the {\tt WHEN}
    qualifier). If the {\tt WHEN} qualifier is used then the
    measurement window can be limited with the qualifiers {\tt FROM},
    {\tt TO}, {\tt RISE}, {\tt FALL} and {\tt CROSS} for all measure
    modes.  In addition, the {\tt TD} qualifier is supported for
    {\tt TRAN} measures. The {\tt MINVAL} qualifier is used as a
    comparison tolerance for both {\tt AT} and {\tt WHEN}.  For HSPICE
    compatibility, {\tt DERIVATIVE} is an allowed synonym for {\tt
    DERIV}.

  \item[\tt DUTY] Fraction of time that {\tt <variable>} is greater than
   {\tt ON} and does not fall below {\tt OFF} either for the entire
    simulation, or the measurement window. The qualifier {\tt MINVAL}
    is used as a tolerance on the {\tt ON } and {\tt OFF} values, so
    that the thresholds become ({\tt ON} $-$ {\tt MINVAL}) and ({\tt
    OFF} $-$ {\tt MINVAL}).  The measurement window can be limited
    with the qualifiers {\tt FROM}, {\tt TO}, and {\tt TD} for
   {\tt TRAN} measures.

\item[\tt EQN] Calculates the value of {\tt <variable>} during the simulation.
    The measurement window can be limited with the qualifiers {\tt FROM},
    {\tt TO} and {\tt TD} for {\tt TRAN} measures, and with {\tt FROM}
    and {\tt TO} for {\tt AC}, {\tt DC} and {\tt NOISE} measures.  As noted in the
    ``Additional Examples'' subsection, the variable can use the
    results of other measure statements.

\item[\tt ERRx] Calculates the error between two simulation variables, where
    the {\tt ERR1} and {\tt ERR2} functions (and the use of the {\tt MINVAL},
    {\tt YMIN} and {\tt YMAX} qualifiers in those functions) are defined further
    in the ``Error Functions (ERR1 and ERR2)'' subsection.  The {\tt ERR} measure
    type is a synonym for  the {\tt ERR1} measure type.  The measurement
    window can be limited  with the qualifiers {\tt FROM} and {\tt TO}.

  \item[\tt ERROR] Calculates the norm between the measured waveform and a
   ``comparison waveform'' specified in a file.  The supported norms are
    L1, L2 and INFNORM.  The default norm is the L2 norm.

  \item[\tt FIND-AT] Returns the value of {\tt <variable>} at the
    time when the {\tt AT} clause is satisfied.  The {\tt AT}
    clause is described in more detail later in this list.

  \item[\tt FIND-WHEN] Returns the value of {\tt <variable>} at the
    time when the {\tt WHEN} clause is satisfied.  The {\tt WHEN}
    clause is described in more detail later in this list.

  \item[\tt FOUR] Calculates the fourier transform of the transient
    waveform for {\tt <variable>}, given the fundamental frequency
    {\tt AT}.  All frequencies output by the measure will be multiples
    of that fundamental frequency, and will always start at that
    fundamental frequency. The values of the DC component and the
    first {\tt NUMFREQ-1} harmonics are determined using an
    interpolation of {\tt GRIDSIZE} points.  The default values for
    {\tt NUMFREQ} and {\tt GRIDSIZE} are 10 and 200, respectively.
    The measurement window can be limited with the qualifiers {\tt
    FROM}, {\tt TO} and {\tt TD} for {\tt TRAN} measures.  For this
    measure, the phase data is always output in degrees.

  \item[\tt FREQ] An estimate of the frequency of {\tt <variable>},
    found by cycle counting during the simulation.  Cycles are defined
    through the values of {\tt ON} and {\tt OFF} with {\tt MINVAL}
    being used as a tolerance so that the thresholds becomes ({\tt ON}
    $-$ {\tt MINVAL}) and ({\tt OFF} $+$ {\tt MINVAL}). The
    measurement window can be limited with the qualifiers {\tt FROM},
    {\tt TO} and {\tt TD} for {\tt TRAN} measures.

  \item[\tt INTEG] Calculates the integral of {\tt outVal} through
    second order numerical integration.  The integration window can be
    limited with the qualifiers {\tt FROM}, {\tt TO} and {\tt TD} for
    {\tt TRAN} measures, and with {\tt FROM} and {\tt TO} for {\tt AC},
    {\tt DC} and {\tt NOISE} measures.  For HSPICE compatibility,
    {\tt INTEGRAL} is an allowed synonym for {\tt INTEG}.

  \item[\tt MAX] Returns the maximum value of {\tt <variable>} during
    the simulation.  The measurement window can be limited with the
    qualifiers {\tt FROM}, {\tt TO}, {\tt TD}, {\tt RISE}, {\tt FALL}
    and {\tt CROSS} for {\tt TRAN} measures, and with {\tt FROM} and
    {\tt TO} for {\tt AC}, {\tt DC} and {\tt NOISE} measures.

  \item[\tt MIN] Returns the minimum value of {\tt <variable>} during
    the simulation.  The measurement window can be limited with the
    qualifiers {\tt FROM}, {\tt TO}, {\tt TD}, {\tt RISE}, {\tt FALL}
    and {\tt CROSS} for {\tt TRAN} measures, and with {\tt FROM} and
    {\tt TO} for {\tt AC}, {\tt DC} and {\tt NOISE} measures.

  \item[\tt OFF\_TIME] Returns the time that {\tt <variable>} is below
    {\tt OFF} during the simulation or measurement window, normalized
    by the number of cycles of the waveform during the simulation or
    measurement window.  {\tt OFF} uses {\tt MINVAL} as a tolerance,
    and the threshold becomes ({\tt OFF} $+$ {\tt MINVAL}).  The
    measurement window can be limited with the qualifiers {\tt FROM},
    {\tt TO} and {\tt TD} for {\tt TRAN} measures.

  \item[\tt ON\_TIME] Returns the time that {\tt <variable>} is above
    {\tt ON} during the simulation or measurement window, normalized
    by the number of cycles of the waveform during the simulation or
    measurement window.  {\tt ON} uses {\tt MINVAL} as a tolerance,
    and the threshold becomes ({\tt ON} $-$ {\tt MINVAL}).  The
    measurement window can be limited with the qualifiers {\tt FROM},
    {\tt TO} and {\tt TD} for {\tt TRAN} measures.

  \item[\tt PP] Returns the difference between the maximum value and
    the minimum value {\tt <variable>} during the simulation. The
    measurement window can be limited with the qualifiers {\tt FROM},
    {\tt TO}, {\tt TD}, {\tt RISE}, {\tt FALL} and {\tt CROSS} for
    {\tt TRAN} measures, and with {\tt FROM} and {\tt TO} for {\tt AC},
    {\tt DC} and {\tt NOISE} measures.

  \item[\tt RMS] Computes the root-mean-squared value of {\tt
    <variable>} during the simulation, which is defined as ''the
    square root of the area under the {\tt <variable>} curve, divided by
    the period of interest''.  The measurement window can be
    limited with the qualifiers {\tt FROM}, {\tt TO} and {\tt TD} for
    {\tt TRAN} measures, and with {\tt FROM} and {\tt TO} for {\tt AC},
    {\tt DC} and {\tt NOISE} measures.

  \item[\vbox{\hbox{\tt TRIG\hfil}\hbox{\tt TARG\hfil}}] Measures the
    time between a trigger event and a target event.  The trigger is
    specified with {\tt TRIG <variable\(\sb{1}\)>=<variable\(\sb{2}\)>} or {\tt
    TRIG <variable\(\sb{1}\)>=<value>} or {\tt TRIG AT=<value>}.  The
    target is then specified as {\tt TARG
    <variable\(\sb{3}\)>=<variable\(\sb{4}\)>} or {\tt TARG <variable\(\sb{3}\)>=<value>}.
    or {\tt TARG AT=<value>}.  It is also possible to use this
    measure to find a rise time for variable when the rise time is
    defined as the time to go from some small fraction of the maxima
    to some other fraction of the maxima.  For example, the syntax for
    finding a rise time from 10\% to 90\% of the maxima
    is:\\ \texttt{TRIG V(node) FRAC\_MAX=0.1 TARG V(node)
    FRAC\_MAX=0.9}

  \item[\tt WHEN] Returns the time (or frequency or DC sweep value) when
    {\tt <variable>} reaches {\tt <variable\(\sb{2}\)>} or the constant
    value, {\tt value}.  The measurement window can be limited with the
    qualifiers {\tt FROM}, {\tt TO}, {\tt RISE}, {\tt FALL} and {\tt CROSS}
    for all measure modes.  In addition, the {\tt TD} qualifier is supported
    for {\tt TRAN} measures. The qualifier {\tt MINVAL} acts as a
    tolerance for the comparison.  For example when {\tt <variable\(\sb{2}\)>}
    is specified, the comparison used is when {\tt <variable>} $=$
    {\tt <variable\(\sb{2}\)>} $\pm$ {\tt MINVAL} or when a constant,
    {\tt value} is given: {\tt <variable>} $=$ {\tt value} $\pm$ {\tt
    MINVAL}.  If the conditions specified for finding a given value
    were not found during the simulation then the measure will return
    the default value of {\tt 0}.  The user may change this default
    value with the {\tt DEFAULT\_VAL} qualifier described below.
    Note: The use of {\tt FIND} and {\tt WHEN} in one measure
    statement is also supported.
\end{description}

\argument{\vbox{\hbox{variable\hfil}\hbox{variable\(\sb{n}\)\hfil}\hbox{value}}}

These quantities represents the test for the stated
measurement.  \texttt{<variable>} is a simulation quantity, such as a
voltage or current.  One can compare it to another simulation variable
or a fixed quantity.  Additionally, the \texttt{<variable>} may be
a \Xyce{} expression delimited by \{ \} brackets.  As noted above, an
example is {\tt V(1)=0.75}

\argument{AT=value}
A time {\em at which} the measurement calculation will occur.  This is
used by the {\tt DERIV} and {\tt FIND} measures and the {\tt TRIG} clause.  Note that
ill-considered use of the {\tt FROM}, {\tt TO}, {\tt TD} and {\tt AT}
qualifiers in the same {\tt TRIG-TARG} measure statement can cause an
empty measurement window, and thus a failed measure.  Finally, the {\tt FROM}
and {\tt TO} qualifiers take precedence over the {\tt AT} qualifier for
{\tt DERIV} and {\tt FIND} measures.

\argument{FROM=value}

A time (or frequency or DC sweep value) {\em after which} the
measurement calculation will start.  For {\tt DC} measures, this
qualifier uses the first variable on the {\tt .DC} line.

\argument{TO=value}

A time (or frequency or DC sweep value) {\em at which} the measurement
calculation will stop.  For {\tt DC} measures, this qualifier uses the
first variable on the {\tt .DC} line.

\argument{TD=value}

A time delay before which this measurement should be taken or checked.
Note that ill-considered use of both {\tt FROM} and {\tt TO}
qualifiers and a {\tt TD} qualifier in the same measure statement can
cause an empty measurement window, and thus a failed measure.

\argument{MIN\_THRESH=value}

A minimum threshold value above which the measurement calculation will
be done and below which it will not be done.  This is only used by the
{\tt AVG} measure.

\argument{MAX\_THRESH=value}

A maximum threshold value above which the measurement calculation will
not be done and below which it will be done.  This is only used by the
{\tt AVG} measure.

\argument{RISE=r|LAST}

The number of rises after which the measurement should be checked.  If
\texttt{LAST} is specified, then the last rise found in the simulation
will be used.  It is recommended that only one of the qualifiers {\tt RISE}, 
{\tt FALL} or {\tt CROSS} be used in a given measure statement.  The exception 
is {\tt TRIG-TARG} measures.  In that case, different {\tt RISE}, {\tt FALL}
and {\tt CROSS} criteria can be specified for {\tt TRIG} and {\tt TARG}.

\argument{FALL=f|LAST}

The number of falls after which the measurement should be checked.  If
\texttt{LAST} is specified, then the last fall found in the simulation
will be used.

\argument{CROSS=c|LAST}

The number of zero crossings after which the measurement should be
checked.  If \texttt{LAST} is specified, then the last zero crossing
found in the simulation will be used.

\argument{RFC\_LEVEL=value}

The level used to calculate rises, falls and crosses when the
``level-crossing'' mode is used by measure types that do not support
the {\tt WHEN} qualifier.  So, {\tt RFC\_LEVEL} is used by the
{\tt MAX}, {\tt MIN} and {\tt PP} measures.  Its usage is discussed
further in the subsection on ``Rise, Fall and Cross Qualifiers''.

\argument{MINVAL=value}

For the {\tt DERIV}, {\tt DUTY}, {\tt FIND}, {\tt FREQ}, {\tt OFF\_TIME},
{\tt ON\_TIME} and {\tt WHEN} measures, this is allowed difference between
\texttt{outVal} and the variable to  which it is being compared.  This has
a default value of 1.0e-12.  One may need to specify a larger value to avoid
missing the test condition in a transient run.  The descriptions of those
seven measures detail how {\tt MINVAL} is used by each measure.  For the
{\tt ERR1} and {\tt ERR2} measures, if the absolute value of
{\tt <variable\(\sb{1}\)>} is less than {\tt MINVAL}, then {\tt MINVAL} replaces
the value of the denominator of the {\tt ERR1} or {\tt ERR2} expression. For
all measure types, that support the {\tt FROM}, {\tt TO} and/or {\tt TD}
qualifiers, {\tt MINVAL} also functions as a relative tolerance for the
comparison of the simulation time (or sweep value) to the bounds of the
measurement window. This allows for numerical-roundoff errors if the
{\tt FROM}, {\tt TO} and/or {\tt TD} qualifiers are expressions.

\argument{YMIN=value}

If the absolute value of {\tt <variable\(\sb{1}\)>} in {\tt ERR1} or
{\tt ERR2} measure is less than the {\tt YMIN} value then the {\tt ERR1} or
{\tt ERR2} calculation does not consider that point. The default is 1.0e-15.
{\tt IGNOR} and {\tt IGNORE} are synonyms for {\tt YMIN}.

\argument{YMAX=value}

If the absolute value of {\tt <variable\(\sb{1}\)>} in {\tt ERR1} or
{\tt ERR2} measure is greater than {\tt YMAX} value then the {\tt ERR1} or
{\tt ERR2} calculation does not consider that point. The default is 1.0e15.

\argument{FRAC\_MAX=value}

A fractional value of the maximum value of \texttt{<variable>}.  This
is useful for ensemble runs where the maximum value of a waveform is
not known in advance.  {\tt FRAC\_MAX} is used by the {\tt TRIG} and
{\tt TARG} measures for {\tt TRAN} measure mode, only.

\argument{ON=value}

The value at which a signal is considered to be ``on'' for {\tt FREQ},
{\tt DUTY} and {\tt ON\_TIME} measure calculations.  This has a
default value of 0.

\argument{OFF=value}

The value at which a signal is considered to be ``off'' for {\tt
FREQ}, {\tt DUTY} and {\tt ON\_TIME} measure calculations.  This has a
default value of 0.

\argument{DEFAULT\_VAL=value}

If the conditions specified for finding a given measure's value are not found
during the simulation then the measure will return a default value of {\tt 0}.
As examples, a measure will fail if the condition specified by a {\tt WHEN} or
{\tt AT} qualifier is not found.  It will also fail if the user specifies a
set of {\tt FROM}, {\tt TO} and {\tt TD} values for a given measure that 
yields an empty measurement interval. The default value for a given measure
is settable by the user by adding the qualifier {\tt DEFAULT\_VAL=<retval>}
on that measure line.  The \texttt{.OPTIONS MEASURE DEFAULT\_VAL=<value>}
setting can be used to set the default value of all of the measures in the
netlist. The measure value in the standard output or log file will always
be FAILED. The measure value in the \texttt{circuitFileName.mt\#}
(or \texttt{circuitFileName.ms\#} or \texttt{circuitFileName.ma\#}) files will
also be FAILED by default (\texttt{.OPTIONS MEASURE MEASFAIL=1}).  If 
\texttt{.OPTIONS MEASURE MEASFAIL=0} is used in the netlist then the
measure value in the output file will be the default value. See Section
\ref{Options_Reference} for more details on the \texttt{.OPTIONS} settings.
 As a final note, the \texttt{FOUR} measure is a special case since it produces
multiline output.  Failed \texttt{FOUR} measures will be reported as FAILED in the
\texttt{circuitFileName.mt\#} ( or \texttt{circuitFileName.ms\#} or
\texttt{circuitFileName.ma\#}) files, irrespective of the various
\texttt{MEASFAIL} and \texttt{DEFAULT\_VAL} settings.

\argument{PRECISION=value}

The default precision for {\tt .MEASURE} output is 6 digits after the
decimal point.  This argument provides a user configurable precision
for a given {\tt .MEASURE} statement that applies to both
the \texttt{.mt\#} ( or \texttt{.ms\#} or \texttt{.ma\#}) files and
standard output.  If \texttt{.OPTIONS MEASURE MEASDGT=<val>} is given
in the netlist then that value overrides the \texttt{PRECISION}
parameters given on individual \texttt{.MEASURE} lines.

\argument{PRINT=value}

This parameter controls where the {\tt .MEASURE} output appears.  The
default is {\tt ALL}, which produces measure output in both
the \texttt{.mt\#} (or \texttt{.ms\#} or
\texttt{.ma\#}) file and to the standard output.  A value of
{\tt STDOUT} only produces measure output to standard output, while a
value of {\tt NONE} suppresses the measure output to both
the \texttt{.mt\#} (or \texttt{.ms\#} or \texttt{.ma\#}) file and
standard output.  The subsection on ``Suppresing Measure Output''
gives examples and also discuss the interactions of this parameter
with \texttt{.OPTIONS MEASURE MEASPRINT=<val>}.

\argument{OUTPUT=value}

This parameter is only supported for the {\tt MAX} and {\tt MIN}
measures.  The default is {\tt VALUE}.  For {\tt TRAN} measures, a
value of {\tt VALUE} will print the maximum (or minimum) value to
the \texttt {.mt\#} file.  A value of {\tt TIME} will print the time
of the maximum (or minimum) value to the \texttt{.mt\#} file. For {\tt
DC} measures, a value of {\tt SV} will output the value of the first
variable on the {\tt .DC} line to the \texttt{.ms\#} file.  For {\tt AC}
and {\tt NOISE} measures, a value of {\tt FREQ} will print the frequency
at which the maximum (or minimum) value occurs to the \texttt{.ma\#} file.
This parameter does not affect the descriptive output that is printed
to the standard output.  The ``Additional Examples'' subsection gives
an example for the {\tt MAX} measure.

\argument{VAL=value}
This parameter is only implemented for the {\tt TRIG} and {\tt TARG}
measures.  It is not the preferred \Xyce{} syntax. It is only
supported for HSPICE compatibility (see that subsection, below, for
details).

\argument{GOAL=value}

This parameter is not implemented in \Xyce{}, but is included for compatibility
with HSPICE netlists.

\argument{WEIGHT=value}

This parameter is not implemented in \Xyce{}, but is included for compatibility
with HSPICE netlists.

\argument{FILE=value}
The filename for the ``comparison file'' used for the {\tt ERROR}
measure.  This qualifier is required for the {\tt ERROR} measure.

\argument{INDEPVARCOL=value}
The column index, in the ``comparison file'', of the independent
variable (e.g, the simulation time or frequency) used in an {\tt ERROR}
measure.  This qualifier is required for the {\tt TRAN}, {\tt AC} and {\tt NOISE}
measure modes.  For those modes, the {\tt INDEPVARCOL} and {\tt DEPVARCOL}
qualifiers must have different values.  The {\tt INDEPVARCOL}
qualifier is not used for {\tt DC} mode {\tt ERROR} measures, and will
be ``silently ignored'' in that case.  Finally, note that the column
indices in Xyce output files start with 0.

\argument{DEPVARCOL=value}
The column index, in the ``comparison file'', of the dependent
variable used in an {\tt ERROR} measure.  This qualifier is required
for the {\tt ERROR} measure for all four measure modes ({\tt TRAN},
{\tt AC}, {\tt DC} and {\tt NOISE}).  For the {\tt TRAN}, {\tt AC} and
{\tt NOISE} measure modes, the {\tt DEPVARCOL} and {\tt INDEPVARCOL}
qualifiers must have different values.  Finally, note
that the column indices in Xyce output files start with 0.

\argument{COMP\_FUNCTION=value} 
This is the norm used by the {\tt ERROR} measure to compare the
simulation values for the measured variable with the corresponding
values in the ``comparison file'' specified with the {\tt FILE}
qualifier.  The allowed values are {\tt L1NORM}, {\tt L2NORM} and {\tt
INFNORM}.  Any other values will default to {\tt L2NORM}.  This
qualifier is optional for the {\tt ERROR} measure, and has a default
value of {\tt L2NORM}.  The descriptive output for each {\tt ERROR}
measure, that is printed to standard output, will explicitly state
which norm was used for each {\tt ERROR} measure.

\end{Arguments}

\end {Command}

\subsubsection{Measure Output}
\label{Measure_Measurement_Output}
\index{measure!measurement output}
As previously mentioned, measured results are reported to the output
and log file.  Additionally, for \texttt{TRAN} measures, the results
are stored in files called
\texttt{circuitFileName.mt\#}, where the suffixed number (\texttt{\#}) starts
at \texttt{0} and increases for multiple iterations (\texttt{.STEP}
iterations) of a given simulation. For \texttt{DC} measures, the results are
stored in the files \texttt{circuitFileName.ms\#}, while \texttt{AC} and
\texttt{NOISE} measures use the files \texttt{circuitFileName.ma\#}.

A user-defined measure can also be output at each time-step via
inclusion in a .PRINT command.  For example, this netlist excerpt
outputs the integral of {\tt V(1)} at each time step.  The measure
value {\tt TINTV1} is then also output at the end of the simulation to
both the standard output and the \texttt{.mt\#} (or \texttt{.ms\#}
or \texttt{.ma\#}) files.
\begin{alltt}
.MEASURE TRAN TINTV1 INTEG V(1)
.PRINT TRAN FORMAT=NOINDEX V(1) TINTV1
\end{alltt}

The output for successful and failed measures to the standard output
(and log files) provides more information than just the measure's
calculated values.  As an example, for a successful and failed {\tt
MAX} measure the standard output would be:

\begin{alltt}
MAXVAL = 0.999758 at time = 0.000249037
Measure Start Time= 0	Measure End Time= 0.001

Netlist warning: MAXFAIL failed. TO value < FROM value
MAXFAIL = FAILED at time = 0
Measure Start Time= 1	Measure End Time= 0.001
\end{alltt}

In general, information on the measurement window, the time(s) that
the measure's value(s) were calculated and a possible cause for a
failed measure are output to standard output for all measures except
for {\tt FOUR}. This information is similar, but not identical, to
HSPICE's verbose output.  For a failed {\tt FOUR} measure, the
standard output will have ``FAILED'', but there may be less
information provided as to why the {\tt FOUR} measure failed.

In this example, the \texttt{circuitFileName.mt\#} file would have the following output:
\begin{alltt}
MAXVAL = 0.999758
MAXFAIL = -1
\end{alltt}

\subsubsection{Measurement Windows}
\label{Measure_Measurement_Windows}
\index{measure!measurement windows}
There is an implicit precedence when multiple qualifiers are specified
to limit the measurement window for a given {\tt .MEASURE} statement
for {\tt TRAN} measures.  In general, \Xyce{} first considers the
time-window criteria of the {\tt FROM}, {\tt TO} and {\tt TD}
qualifiers.  If the simulation time is within that user-specified
time-window then the {\tt RISE}, {\tt FALL}, {\tt CROSS} are
qualifiers are counted and/or the {\tt TRIG}, {\tt TARG} and {\tt
WHEN} qualifiers are evaluated.

The following netlist excerpt shows simple examples where the {\tt
.MEASURE} statement may return the default value because the measure
``failed''.  For {\tt riseSine}, this may occur because {\tt V(1)}
never has an output value of 1.0 because of the time steps chosen
by \Xyce{}. So, careful selection of the threshold values in WHEN,
TRIG and TARG clauses may be needed in some cases.  For {\tt
fallPulseFracMax}, the simulation interval is too short and the {\tt
TARG} value of 0.3 for {\tt V(2)} is not reached within the specified
one-second simulation time.  For {\tt maxSine}, the {\tt FROM}, {\tt
TO} and {\tt TD} values yield an empty time interval, which is
typically an error in netlist entry.
\begin{alltt}
VS  1  0  SIN(0 1.0 0.5 0 0)
VP  2  0  PULSE( 0 10 0.2 0.2 0.2 0.5 2)
R1  1  0  100K
R2  2  0  100K
.TRAN 0  1
.PRINT TRAN FORMAT=NOINDEX V(1) V(2)
.MEASURE TRAN riseSine TRIG V(1)=0 TARG V(1)=1.0
.MEASURE TRAN fallPulseFracMax TRIG V(2) FRAC_MAX=0.97
+ TARG V(2) FRAC_MAX=0.03
.MEASURE TRAN maxSine MAX V(1) FROM=0.2 TO=0.25 TD=0.5
\end{alltt}
The intent in \Xyce{} is for the measurement window to be the
intersection of the {\tt FROM-TO} and {\tt TD} windows, if both are
specified.  As noted above, the use of both {\tt FROM-TO} and {\tt TD}
windows can lead to an empty measurement window.  So, that usage is
not recommended.

\subsubsection{Expression Support}
\index{measure!expression support}
These measure ``qualifiers'' (\texttt{TO}, \texttt{FROM}, \texttt{TD},
\texttt{RISE}, \texttt{FALL}, \texttt{CROSS}, \texttt{AT}, \texttt{OFF},
\texttt{ON}, \texttt{DEFAULT\_VAL} and \texttt{VAL}) support expressions.
The caveat is that the expression must evaluate to a constant at the time
that each measure object is made.  So, that expression can not depend
on solution variables or lead currents.  This limitation matches HSPICE.
It also can not depend on a global parameter.  Finally, it can not depend
on another measure's value, which is an allowed syntax in HSPICE.

Simple examples of allowed syntaxes for qualifiers are as follows, where
all three measures will get the same answer:
\begin{alltt}
.PARAM t1=0.2
.PARAM t2=0.3
.MEASURE TRAN M1 PP V(1) FROM=`0.1+0.2'
.MEASURE TRAN M2 PP V(1) FROM=\{0.1+t1\}
.MEASURE TRAN M3 PP V(1) TO=\{t2\}
\end{alltt}
Expressions should also work in FIND-WHEN, WHEN and TRIG-TARG measures. The
preferred Xyce syntax with curly braces and the three legal HSPICE syntaxes
for expressions should all work.  However, note that the two HSPICE
expression syntaxes shown below are only legal in \Xyce{} \texttt{.MEASURE}
statements.
\begin{alltt}
.PARAM a1=0.1
.PARAM a2=0.7
.MEASURE TRAN M4 FIND V(2) WHEN V(1)=\{a1\}
.MEASURE TRAN M4PAR FIND V(2) WHEN V(1)=PAR(`a1') ; HSPICE exp. syntax
.MEASURE TRAN M4PAREN FIND V(2) WHEN V(1)=(`a1')  ; HSPICE exp. syntax
.MEASURE TRAN M5 WHEN V(1)=\{a1\}
.MEASURE TRAN M6 TRIG \{v(1)-0.1\} VAL=\{a1\} TARG \{v(1)-0.5\} VAL=\{a2\}
\end{alltt}

\subsubsection{Re-Measure}
\label{Measure_ReMeasure}
\index{measure!re-measure}
\Xyce{} can re-calculate (or re-measure) the values for {\tt .MEASURE} and/or
{\tt .FFT} statements using existing \Xyce{} output files.  This is useful for
tuning {\tt .MEASURE} and/or {\tt .FFT} statements to better capture response
metrics for a circuit when the underlying simulation runtime is long.  To use this
functionality, add the command line argument {\tt -remeasure <file>},
where {\tt <file>} is a \Xyce{}-generated {\tt .prn}, {\tt .csv} or
{\tt .csd} output file.

There are several important limitations with {\tt -remeasure}:
\begin{XyceItemize}

  \item The data required by the {\tt .MEASURE} and/or {\tt .FFT} statements
   must have been output in the simulation output file.  When using
  {\tt -remeasure}, \Xyce{} does not recalculate the full solution, but
  uses the data supplied in the output file instead.  Thus, everything
  a {\tt .MEASURE} and/or {\tt .FFT} statement needs to calculate its results
  must be in the output file.  So, the nodal voltages (e.g., node A), lead
  currents (e.g, for device R1) and branch currrents requested by the
  {\tt .MEASURE} statements must have been used, at least once, on the
  {\tt .PRINT} statement in the form of {\tt V(A)}, {\tt N(a)} or {\tt
  I(R1)}.  They can not only appear on the {\tt .PRINT} line within an
  expression or a voltage-difference operator.

  \item Only voltage node values, lead currents and branch currents
  can be used in {\tt .MEASURE} statements while using {\tt
  -remeasure}.  Power values will not be interpreted correctly during
  a re-measure operation.  A work-around for that limitation is
  illustrated below.

  \item {\tt -remeasure} only works with {\tt .tran} or {\tt .dc}
  analyses.  However, it can be used with {\tt .STEP} in both
  cases. It is not currently supported for {\tt .ac} analyses.

  \item For {\tt .tran} analyses, {\tt -remeasure} works with {\tt
  .prn}, {\tt .csv} and {\tt .csd} formatted output data.  However, it
  might only work with {\tt .csv} and {\tt .csd} files generated
  by \Xyce{}.  \item For {\tt .dc} analyses, {\tt -remeasure} works
  with {\tt .prn} and {\tt .csd} formatted output data.  However, it
  might only work with {\tt .csd} files generated by \Xyce{}.

  \item {\tt -remeasure} will fail if the netlist has a {\tt .op}
  statement that precedes the {\tt .tran} or {\tt .dc} statement.
  This can be fixed by either moving the {\tt .op} statement or by
  temporarily commenting the {\tt .op} statement out during {\tt
  -remeasure}.
\end{XyceItemize}

As an example in using {\tt -remeasure}, consider a netlist called
{\tt myCircuit.cir} which had previously been run in \Xyce{} and
produced the output file {\tt myCircuit.cir.prn}.  One could run {\tt
-remeasure} with the following command:
\begin{alltt}
Xyce -remeasure myCircuit.cir.prn myCircuit.cir
\end{alltt}

A work-around for re-measuring power values (e.g., for device R1) is
to use this combination of {\tt .PRINT} and {\tt .MEASURE} lines in
the netlist.  As noted above, expressions will work with re-measure if
all of the quantities used in the expression also appear outside of an
expression on the {\tt .PRINT} line.
\begin{alltt}
R1 a b 1
.PRINT TRAN V(a) V(b) I(R1)
.MEASURE TRAN PR1B MAX \{(V(a)-V(b))*I(R1)\}
\end{alltt}

\subsubsection{RISE, FALL and CROSS Qualifiers}
The \texttt{RISE}, \texttt{FALL} and \texttt{CROSS} qualifiers are
supported for more measures types, and in more ways, in \Xyce{} than
in HSPICE for \texttt{TRAN} meaures.  This sections explains those
differences and supplies some examples.  One key difference is that
\Xyce{} supports two different ``modes'' for these qualifiers for
\texttt{TRAN} meaures.

The first mode is ``level-crossing'', where
the \texttt{RISE}, \texttt{FALL} and \texttt{CROSS} counts are
incremented each time the measured signal (e.g,
\texttt{ V(a)}) crosses the user-specified level (termed
\texttt{crossVal} here).  This mode should work identically to HSPICE
for the \texttt{DERIV-WHEN}, \texttt{FIND-WHEN}, \texttt{WHEN} and
\texttt{TRIG-TARG} measures.

If we define \texttt{currentVal} and \texttt{lastVal} as the current and previous
values of \texttt{V(a)}, and \texttt{riseCount},
\texttt{fallCount} and \texttt{crossCount} as the number of rises, falls and
crosses that have occurred, then the pseudo-code for the ``level-crossing'' mode is:
\begin{alltt}
if ( (currentVal-crossVal >= 0.0) AND (lastVal-crossVal < 0.0) )
\{
  riseCount++;
  crossCount++;
\}
else if( (currentVal-crossVal) <= 0.0) AND (lastVal-crossVal > 0.0) )
\{
  fallCount++;
  crossCount++;
\}
\end{alltt}

For \texttt{DERIV-WHEN}, \texttt{FIND-WHEN}, \texttt{WHEN},
measures, the cross value is set by the value (or second variable) in the
\texttt{WHEN} clause. For \texttt{TRIG-TARG} measures, the cross values are
set separately by the the values (or second variables) in the \texttt{TRIG}
and \texttt{TARG} clauses.

The second mode is termed ``absolute''. In this mode, \Xyce{} attempts
to auto-detect whether the measured waveform has started a new rise or
fall. However, the \texttt{crossCount} is still evaluated against a
fixed \texttt{crossVal} of 0.  This mode may be useful for pulse
waveforms with sharp rises and falls, where the waveform's maximum (or
minimum) level is not exactly known in advance.  It may not work well
with noisy waveforms.  

If we define two Boolean variables \texttt{isRising} and \texttt{isFalling} then
the pseudo-code for the ``absolute'' mode is:
\begin{alltt}
if( (currentVal > lastVal) AND !isRising )
\{
  isRising= true;
  isFalling = false;
  riseCount++;
\}
else if( (currentVal < lastVal) AND !isFalling )
\{
  isRising = false;
  isFalling = true;
  fallCount++;
\}
if ( ( (currentVal >= 0.0) AND (lastVal < 0.0) ) OR
     ( (currentVal <= 0.0) AND (lastVal > 0.0) ) )
\{
  crossCount++;
\}
\end{alltt}
The following table shows which of these two modes are supported for which
\Xyce{} measure types.
% Sandia National Laboratories is a multimission laboratory managed and
% operated by National Technology & Engineering Solutions of Sandia, LLC, a
% wholly owned subsidiary of Honeywell International Inc., for the U.S.
% Department of Energy’s National Nuclear Security Administration under
% contract DE-NA0003525.

% Copyright 2002-2024 National Technology & Engineering Solutions of Sandia,
% LLC (NTESS).


%%
%% Which Measures support which modes for RISE, FALL and CROSS
%%
\begin{longtable}[h] {>{\raggedright\small}m{1.0in}|>{\raggedright\let\\\tabularnewline\small}m{2.5in}
  |>{\raggedright\let\\\tabularnewline\small}m{2.5in}}
  \caption{RISE, FALL and CROSS Support in .MEASURE.} \\ \hline
  \rowcolor{XyceDarkBlue}
  \color{white}\bf Measure &
  \color{white}\bf Level-Crossing &
  \color{white}\bf Absolute \\ \hline \endfirsthead
  \rowcolor{XyceDarkBlue}
  \color{white}\bf Measure &
  \color{white}\bf Level-Crossing &
  \color{white}\bf Absolute \\ \hline \endhead
  \label{RISE_FALL_CROSS}

  \texttt{DERIV-WHEN} & The  \texttt{crossVal} is set by the value of the
                 \texttt{WHEN} clause & No \\ \hline
  \texttt{FIND-WHEN} and \texttt{WHEN} & The  \texttt{crossVal} is set by the
         value of the \texttt{WHEN} clause & No \\ \hline
  \texttt{MAX} & A fixed \texttt{crossVal} can be set with \texttt{RFC\_LEVEL} 
               & Default, if \texttt{RFC\_LEVEL} is not set \\ \hline
  \texttt{MIN} & A fixed \texttt{crossVal} can be set with \texttt{RFC\_LEVEL} 
               & Default, if \texttt{RFC\_LEVEL} is not set \\ \hline
  \texttt{PP} & A fixed \texttt{crossVal} can be set with \texttt{RFC\_LEVEL} 
              & Default, if \texttt{RFC\_LEVEL} is not set \\ \hline
  \texttt{TRIG} and \texttt{TARG} & The levels are set separately by the values 
                in the \texttt{TRIG} and \texttt{TARG} clauses & Only if 
                \texttt{FRAC\_MAX} is used\\ \hline
\end{longtable}


As simple examples of these two modes for the \texttt{MAX} measure, consider
the following netlist:
\begin{alltt}
*examples of RFC modes
VPWL1  1  0 PWL(0 0 0.2 0.5 0.4 0 0.6 0.75 0.8 0 1.0 0.75 1.2 0.0)
R1  1  0  100
.TRAN 0 1.2s
.MEASURE TRAN MAX1 MAX V(1) RISE=1
.MEASURE TRAN MAX2 MAX V(1) RISE=1 RFC_LEVEL=0.6
.MEASURE TRAN MAX3 MAX V(1) FALL=1 RFC_LEVEL=0.5
.PRINT TRAN V(1) MAX1 MAX2 MAX3
.END
\end{alltt}
The descriptive output to standard output would then be:
\begin{alltt}
MAX1 = 5.000000e-01 at time = 2.000000e-01
Measure Start Time= 0.000000e+00        Measure End Time= 1.200000e+00
Rise 1: Start Time= 1.000000e-10        End Time= 4.000000e-01

MAX2 = 7.500000e-01 at time = 6.000000e-01
Measure Start Time= 0.000000e+00        Measure End Time= 1.200000e+00
Rise 1: Start Time= 5.600000e-01        End Time= 9.500000e-01

MAX3 = 7.500000e-01 at time = 1.000000e+00
Measure Start Time= 0.000000e+00        Measure End Time= 1.200000e+00
Fall 1: Start Time= 6.700000e-01        End Time= 1.060000e+00
\end{alltt}
The \texttt{MAX1} measure uses the ``absolute'' mode, so the first
rise begins with the very first time-step.  The maximum value in that
first rise interval for measure \texttt{MAX1} then occurs at
time=0.2s.  The \texttt{MAX2} measure uses the ``level-crossing'' mode
with a user-specified \texttt{RFC\_LEVEL} of 0.6V.  So, the first rise
interval for the \texttt{MAX2} measure begins at time=0.56s, and the
maximum value in that first rise interval occurs at time=0.6s.  The
\texttt{MAX3} measure illustrates an important point.  A ``fall'' is not
recorded for the \texttt{MAX3} measure at t=0.2 seconds, but a
``rise'' (and ``cross'') would be recorded, since the value
of \texttt{V(1)} is exactly equal to the
user-specified \texttt{RFC\_LEVEL}. So, the first fall interval for
measure
\texttt{MAX3} begins at time=0.67s, when V(1) first passes through the
user-specified \texttt{RFC\_LEVEL} of 0.5V.

\subsubsection{Additional Examples}
\label{Measure_Additional_Examples}
\index{measure!additional examples}
Pulse width measurements in \Xyce{} can be done as follows, based on
this netlist excerpt.  This may be useful for ensemble runs, where the
maximum value of a one-shot pulse is not known in advance.  The first
syntax uses three measure statements to measure the 50\% pulse width,
and works with noisy waveforms.  The second syntax uses only one
measure statement, but may not always work with noisy waveforms.

\begin{alltt}
* pulse-width measurement example 1
.measure tran rise50FracMax trig v(1) frac_max=0.5 targ v(1) frac_max=1
.measure tran fall50FracMax trig v(1) frac_max=1 targ v(1) frac_max=0.5
.measure tran 50width EQN\{rise50FracMax + fall50FracMax\}

* pulse-width measurement example 2
.measure tran 50widthFracMax trig v(1) frac_max=0.50
+ targ v(1) frac_max=0.50 FALL=1
\end{alltt}

In some cases, the user may wish to print out both the measure value
and measure time (or the value of the first variable on the {\tt .DC}
line) of a
\texttt{MAX} or \texttt{MIN} measure to the \texttt{.mt0} file.  For a TRAN measure,
this can be done for these two measures with the \texttt{OUTPUT}
keyword as follows:
\begin{alltt}
* printing maximum value and time of maximum value to .mt0 file
.TRAN 0 1
V1 1 0 PWL 0 0 0.5 1 1 0
R1 1 0 1
.MEASURE TRAN MAXVAL MAX V(1)
.MEASURE TRAN TIMEOFMAXVAL MAX V(1) OUTPUT=TIME
\end{alltt}
The output to the \texttt{.mt0} file would be:
\begin{alltt}
MAXVAL = 1.000000e+00
TIMEOFMAXVAL = 5.000000e-01
\end{alltt}
The descriptive output to standard output would be the same for both
measures though.  The measure value and measure time are not
re-ordered in the descriptive output when \texttt{OUTPUT=VALUE} is
used for the \texttt{MAX} or
\texttt{MIN} measures.
\begin{alltt}
MAXVAL = 1.000000e+00 at time = 5.000000e-01
Measure Start Time= 0.000000e+00        Measure End Time= 1.000000e+00

TIMEOFMAXVAL = 1.000000e+00 at time = 5.000000e-01
Measure Start Time= 0.000000e+00        Measure End Time= 1.000000e+00
\end{alltt}

For a DC measure, one would use \texttt{OUTPUT=SV} instead
of \texttt{OUTPUT=TIME}.  In that case, the ``sweep value'' (SV) is
the value of the first variable on the
\texttt{.DC} line. For an AC or NOISE measure, one would use \texttt{OUTPUT=FREQ}.

\subsubsection{Suppresssing Measure Output}
\label{Measure_Suppressing_Measure_Output}
\index{measure!suppressing measure output}
If the \Xyce{} output is post-processed with other programs, such as
Dakota, it may be desirable to only print a subset of the measure
values to the \texttt{.mt\#} (or \texttt{.ms\#} or \texttt{.ma\#})
files, but to print all of the measure output to standard output.  As
an example, these {\tt .MEASURE} statements:
\begin{alltt}
.TRAN 0  2ms
.measure tran minSineOne  min V(1) print=none
.measure tran minSinTwo   min V(2) print=stdout
.measure tran minSinThree min V(3) print=all
.measure tran sinSinFive  min V(4)
\end{alltt}
would produce the following measure output in the {\tt .mt0} file:
\begin{alltt}
MINSINTHREE = -3.851422e-01
MINSINFOUR = -1.998548e+00
\end{alltt}
and the following measure output in standard output:
\begin{alltt}
MINSINTWO = -1.188589e+00 at time = 7.400000e-04
Measure Start Time= 0.000000e+00	Measure End Time= 2.000000e-03

MINSINTHREE = -3.851422e-01 at time = 2.400000e-04
Measure Start Time= 0.000000e+00	Measure End Time= 2.000000e-03

MINSINFOUR = -1.998548e+00 at time = 7.500000e-04
Measure Start Time= 0.000000e+00	Measure End Time= 2.000000e-03
\end{alltt}

\texttt{.OPTIONS MEASURE MEASPRINT=<val>} also provides the option to accomplish
these same effects, but for all of the measure statements in the
netlist.  The interactions between these two features are as follows.
If \texttt{MEASPRINT=ALL} is used, which is the default setting, then
the \texttt{PRINT} qualifier on a given
\texttt{.MEASURE} line will override that setting.  However,
\texttt{MEASPRINT=NONE}
and \texttt{MEASPRINT=STDOUT} will take precedence over
the \texttt{PRINT} qualifiers on individual \texttt{.MEASURE} lines.
Finally, the \texttt{MEASPRINT} option will be ignored during
remeasure, but the \texttt{PRINT} qualifiers on individual measure
lines will be used.

\texttt{.OPTIONS MEASURE MEASOUT=<val>} provides another way to suppress the output
of the \texttt{.mt\#} (or \texttt{.ms\#} or \texttt{.ma\#}) files.
See Section \ref{Options_Reference} for more details.  If given, this
option takes precedence over the \texttt{MEASPRINT} option setting.
However, it is also ignored during remeasure.

\subsubsection{ERROR Functions (ERR1 and ERR2)}
\label{Measure_Error_Functions}
\index{measure! err1 measure}
\index{measure! err2 measure}
This subsection defines the calculation functions for the \texttt{ERR1} and
\texttt{ERR2} measure types.  For the \texttt{ERR1} measure, the measure
value is calculated as follows, where $M_i$ and $C_i$ are the first and
second variables on the measure line and N is the number of time, frequency
or DC sweep values included in the measure calculation:
\begin{equation}
ERR1 = \sqrt{ \frac{1}{N}\sum_{i=1}^{N} \left(\frac{M_i - C_i}{max(MINVAL,\mid M_i \mid)}\right)^2}
\end{equation}

For the \texttt{ERR2} measure, the value is:
\begin{equation}
ERR2 = \frac{1}{N}\sum_{i=1}^{N} \left|\frac{M_i - C_i}{max(MINVAL,\mid M_i \mid)}\right|
\end{equation}

For both measures, if the absolute value of $M_i$ is less than the
\texttt{YMIN} value or greater than the \texttt{YMAX} value then the
{\tt ERR1} or {\tt ERR2} calculation does not consider that point. The
default for \texttt{YMIN} is 1.0-e15.  The default for \texttt{YMAX}
is 1.0e15.

\subsubsection{ERROR Measure}
\label{Measure_Error_Measure}
\index{measure! error measure}
The \Xyce{} ERROR measure is not the functional equivalent of the \texttt{ERR1}
or \texttt{ERR2} measures.  It is intended to solve a different problem, namely
the comparison of data in multiple simulation runs to an assumed ``gold standard''
read in from a file.  It also uses different comparision functions then the
\texttt{ERR1} and \texttt{ERR2} measures..  This  subsection lists some important
caveats with the use of the \texttt{ERROR} measure.
\begin{XyceItemize}

  \item The comparison file, specified with the {\tt FILE} qualifier,
    can be {\tt .prn}, {\tt .csv} and {\tt .csd} formatted output
    data.  However, the {\tt ERROR} measure might only work with {\tt
    .csv} and {\tt .csd} files generated by \Xyce{}.  \item The data
    in the comparison file is assumed to be ``non-step data'', from
    one simulation iteration.  The simulated data can use {\tt .STEP}
    though and the {\tt ERROR} measure values will be re-evaluated for
    each step.  

  \item For {\tt TRAN} (or {\tt AC} or {\tt NOISE}) measures, the values
    of the measured waveform are interpolated to the simulation times (or
    frequencies) in the comparison waveform.  So, the norm calculation is 
    inherently windowed to the time (or frequency) interval of the comparison
    waveform.  For the best interpolation results for {\tt AC} or {\tt NOISE}
    measures, it is recommended that the frequency extent of the comparison
    waveform be greater than or equal to the frequency extent of the measured
    waveforms.

  \item For {\tt DC} measures, interpolation is not used.  So, the values
    of the simulated and comparion waveforms are compared at the
    values specified by the {\tt DEPVARCOL} qualifier.  Any value for
    the {\tt INDEPVARCOL} qualifier specified on a {\tt DC} measure
    line will be ``silently ignored''.

  \item The time and frequency window constraints ({\tt TO}, {\tt FROM} and {\tt TD}
    qualifiers) are not supported for the {\tt ERROR} measure.  So, as
    noted above, the effective window for the norm calculation is set
    by the extent of the comparison waveform.

  \item The values in the column in the comparison file specified with
    the {\tt INDEPVARCOL} qualifier must be monotonically increasing
    for a {\tt TRAN}, {\tt AC} or {\tt NOISE} measure.  Otherwise,
    \Xyce{} will not run the simulation.

  \item The {\tt ERROR} measure currently supports the L1, L2 and
    INFNORM, with the default being the L2 norm.  If anything other
    than L1, L2 or INFNORM is specified, \Xyce{} will default to the
    L2 norm.  The descriptive output for each {\tt ERROR} measure,
    that is printed to standard output, will explicitly state which
    norm was used for each {\tt ERROR} measure.  (Note: The norm value
    is selected with the {\tt COMP\_FUNCTION} qualifier, and the
    allowed values are {\tt L1NORM}, {\tt L2NORM} and {\tt INFNORM}.)

\end{XyceItemize}

As a final note, the \texttt{ERROR} measure can enable the use
of \Xyce{} simulation output in optimization problems, like device
calibration.  However, for internal Sandia users, there may be better
approaches that leverage the combined capabilities of Sandia's Dakota
and \Xyce{} software packages.

\subsubsection{Operator Support for AC Mode Measures}
\label{Measure_AC_Op_Support}
\index{measure!supported operators for AC measures}
All of the operators supported on \texttt{.PRINT AC} lines are supported
for AC measure mode.  The linear parameter operators (e.g., \texttt{SR(1,1)})
are only supported when a \texttt{.LIN} analysis is done, but their values
can be used in \texttt{.MEASURE AC} statements in that case.

One caveat is that AC mode measures that use \texttt{V(a)} will actually
measure \texttt{VR(a)}. The same caveat applies to the use of \texttt{S(1,1)}.
An AC mode measure would measure \texttt{SR(1,1)} instead.

\subsubsection{Operator Support for NOISE Mode Measures}
\label{Measure_Noise_Op_Support}
\index{measure!supported operators for Noise measures}
All of the operators supported on \texttt{.PRINT NOISE} lines are supported
for NOISE measure mode.  One caveat is that NOISE mode measures that use
\texttt{V(a)} will actually measure \texttt{VR(a)}.

\subsubsection{Behavior for Unsupported Modes and Types}
\label{Measure_Unsupported_Types}
\index{measure!unsupported types}
The \texttt{.MEASURE} statement is supported
for \texttt{.TRAN}, \texttt{.AC}, \texttt{.DC} and \texttt{.NOISE}
analyses. It can be used with {\tt .STEP} in all four cases.
So, \Xyce{} does not support \texttt{HB} measure mode.  If that mode
is included in the netlist then \Xyce{}
parsing will fail and emit error messages.  Similarly, \Xyce{} parsing
will fail if the requested measure type is not supported for a given
measure mode (e.g., \texttt{OFF\_TIME} for a \texttt{AC}, \texttt{DC}
or \texttt{NOISE} measure).

\subsubsection{Compatibility with .DATA}
The \texttt{.DATA} command can be used to specify table-based \texttt{.AC},
\texttt{.DC} or \texttt{.NOISE} sweeps for those three analysis types. For
\texttt{AC} and \texttt{NOISE} measures, the ``swept variable'' then uses
the frequency values in the table specified on the \texttt{.AC} or
\texttt{.NOISE} line.

For \texttt{DC} measures, the swept variable uses the row index in the
table specified on the \texttt{.DC} line.  An example is as follows:
\begin{alltt}
* example of .DATA with DC measures
V1 1 0 1
R1 1 2 1
R2 2 0 1

.data test
+ r1           r2
+ 1.0e+00  4.0e+00
+ 4.0e+00  6.0e+00
+ 6.0e+00  4.0e+00
.enddata

.DC data=test
.print DC V(1) V(2)

.OPTIONS MEASURE MEASDGT=1
.MEASURE DC MAXV2TO MAX V(2) TO=2
.MEASURE DC MAXV2FROM MAX V(2) FROM=2

.END
\end{alltt}

The measure results reported in stdout will be as follows where the respective
maximum values occur for the R1 and R2 values given in the first and second rows
of the \texttt{test} table:
\begin{alltt}
MAXV2TO = 8.0e-01 at Table Row value = 1.0e+00
Measure Start Table Row Value= 1.0e+00  Measure End Table Row Value= 2.0e+00

MAXV2FROM = 6.0e-01 at Table Row value = 2.0e+00
Measure Start Table Row Value= 2.0e+00  Measure End Table Row Value= 3.0e+00
\end{alltt}

All valid measure types will return an answer when a data-based sweep is used
on the \texttt{.AC}, \texttt{.DC} or \texttt{.NOISE} line.  However, the results
for \texttt{AVG}, \texttt{DERIV}, \texttt{FIND}, \texttt{INTEG}, \texttt{RMS}
and \texttt{WHEN} measures may be ``non-physical'' if the frequency values in the
data table are not monotonically increasing.  In addition, for \texttt{DC} measures
the effective step size between table rows is equal to one.

\subsubsection{HSPICE Compatibility}
\label{Measure_HSpice_Compatibility}
\index{measure!HSPICE compatibility}
There are known incompatibilities between the \Xyce{} and HSPICE implementation
of {\tt .MEASURE}.  They include the following:

\begin{XyceItemize}
  \item Since \texttt{.AC} and \texttt{.NOISE} are separate analysis types in
        \Xyce{}, there are separate \texttt{AC} and \texttt{NOISE} measure modes.

  \item Several of the \Xyce{} measure types ({\tt DUTY}, {\tt EQN},
         {\tt FREQ}, {\tt FOUR}, {\tt ON\_TIME}, and {\tt OFF\_TIME})
         and qualifers (e.g., {\tt FRAC\_MAX}) are not found in
         HSPICE.  Several HSPICE measure types are not supported
         in \Xyce{}.

  \item The default, in both HSPICE and \Xyce{}. for {\tt DERIV-WHEN}, {\tt FIND-WHEN},
        {\tt WHEN} and {\tt TRIG-TARG} measures is to use {\tt CROSS=0} if a {\tt RISE},
        {\tt FALL} or {\tt CROSS} value is not explicitly given in the {\tt WHEN}, {\tt TRIG}
        or {\tt TARG} clause,   However, the \Xyce{} and HSPICE results may differ in those
        cases if the measured signal is either a constant value or meets the measure criteria
        at the first simulation step (e.g., t=0).

  \item The HSPICE qualifers of {\tt REVERSE} and {\tt PREVIOUS} are
  not supported in \Xyce{}.

  \item The HSPICE {\tt .POWER} statement, which prints out a table
        with the AVG, RMS, MIN and MAX measures for each specified
        signal, is not supported in \Xyce{}.

  \item \Xyce{} generally supports more qualifiers ({\tt FROM}, {\tt
        TO}, {\tt TD}, {\tt RISE}, {\tt FALL} and {\tt CROSS}) for the
        measurement windows for a given measure-type.  So, some
        legal \Xyce{} syntaxes may not be legal in HSPICE.

  \item The Xyce {\tt EQN} measure can calculate an expression based
        on other measure values.  So, one of its usages is similar to
        the HSPICE {\tt PARAM} measure.  However, their syntaxes are
        different.

  \item A mismatch between the measure mode and the analysis mode
        (e.g., a {\tt DC} measure in a netlist that uses a {\tt .TRAN}
        analysis statement) will cause a \Xyce{} netlist parsing
        error.  That same mismatch might be silently ignored by
        HSPICE.

  \item How \Xyce{} and HSPICE handle ``steps'' may be different.
        In \Xyce{}, the ``steps'' in the measured data (e.g., the
        generation of new \texttt{.mt\#} or \texttt{.ms\#}
        or \texttt{.ma\#} files) are triggered by the variable(s) on
        the {\tt .STEP} line, but not by the variable(s) on the {\tt
        .DC} line.

  \item Expressions on \texttt{.MEASURE} lines are supported in fewer
        contexts then in HSPICE.  See the ``Expression Support''
        subsection for more details.

  \item The settings for the \texttt{MEASFAIL} and \texttt{MEASOUT}
        options are only used if those options are explicitly given in
        the netlist.  Otherwise, the \Xyce{} defaults will be used.
\end{XyceItemize}

The following HSPICE syntax ({\tt VAL=0.9}) is supported in \Xyce{}
for {\tt TRIG} and {\tt TARG} measures. However, the preferred \Xyce{}
syntax would use {\tt targ v(1)=0.9} instead.
\begin{alltt}
.measure tran riseSine trig v(1) AT=0.0001 targ v(1) VAL=0.9 RISE=1
\end{alltt}

The remainder of this subsection discusses alternate syntaxes
for \Xyce{} measure lines that are supported for improved HSPICE
compatibility.  The definitions of the measure syntaxes given at the
beginning of this \texttt{.MEASURE} section give the preferred
\Xyce{} syntaxes.  However, \texttt{PARAM} (and the equivalent \texttt{EQN})
measure lines are allowed with, or without, the equal sign after
the \texttt{PARAM} keyword.  So, these two \Xyce{} measure statements
are equivalent:
\begin{alltt}
.measure tran noEqualSgn PARAM \{v(1)+1.0\}
.measure tran equalSgn PARAM=\{v(1)+1.0\}
\end{alltt}

There are multiple expression syntaxes that are allowed in various
contexts on HSPICE measure lines.  So, all of these example syntaxes
are allowed in expression contexts on \Xyce{} measure lines.  (Note:
Only the first single-quote-delimited expression format is supported
in all \Xyce{} expression contexts, in addition to the \Xyce{}
curly-braces format.)
\begin{alltt}
.measure tran curlyBraces MAX \{V(1)+1\}
.measure tran singleQuote MAX `V(1)+1'
.measure tran parenSingleQuote MAX (`V(1)+1')
.measure tran parSyntax MAX PAR(`V(1)+1')
\end{alltt}

Undelimited expressions are allowed in some contexts in HSPICE.
However, the syntax for the notLegalInXyce measure shown below
is not allowed in \Xyce{}, since it uses an undelimited expression.
\begin{alltt}
.measure tran PLUS PP PAR(`V(1)+V(2)')
.measure tran notLegalInXyce PARAM PLUS+2.0 ; not legal
\end{alltt}

\subsubsection{Legacy Trig-Targ Mode (LTTM)}
For the \Xyce{} 7.5 release, the code for the \texttt{TRIG-TARG} measure
was extensively re-written to provide better HSPICE (and ngspice)
compatibility.  However, mostly for backwards compatibility for
internal Sandia users, the previous behavior can be recovered by
using \texttt{.OPTION MEASURE USE\_LTTM=1} in the netlist.  It is
anticipated that this feature will be removed in a future release.
However, support for the \texttt{FRAC\_MAX} qualifier, which is not
in either HSPICE or ngspice, will likely be continued.

The allowed syntaxes for this mode are shown below.  Note that this
mode lacks several features.  In particular:

\begin{XyceItemize}
  \item The \texttt{AT} qualifer is only allowed in the \texttt{TRIG} clause.
  \item The \texttt{FROM}, \texttt{TO} and \texttt{TD} qualifiers apply to
both the \texttt{TRIG} and \texttt{TARG} clauses.
  \item Expression support, especially in the \texttt{TARG} clause, was less
available.
\end{XyceItemize}

\begin{Command}
\format
\begin{alltt}
.MEASURE TRAN <result name> TRIG <variable>=<variable\(\sb{2}\)>|<value>
+ [RISE=r1|LAST] [FALL=f1|LAST] [CROSS=c1|LAST]
+ TARG <variable\(\sb{3}\)>=<variable\(\sb{4}\)>|<value> 
+ [RISE=r2|LAST] [FALL=f2|LAST] [CROSS=c2|LAST]
+ [FROM=<value>] [TO=<value>] [TD=<value>] 
+ [DEFAULT_VAL=<value>] [PRECISION=<value>] [PRINT=<value>]

.MEASURE TRAN <result name> TRIG AT=<value>
+ TARG <variable\(\sb{2}\)>=<variable\(\sb{3}\)>|<value> 
+ [RISE=r2|LAST] [FALL=f2|LAST] [CROSS=c2|LAST]
+ [FROM=<value>] [TO=<value>] [TD=<value>] 
+ [DEFAULT_VAL=<value>] [PRECISION=<value>] [PRINT=<value>]
\end{alltt}
\end{Command}




%%%%%%%%%%%%%%%%%%%%%%%%%%%%%%%%%%%%%%%%%%%%%%%%%%%%%%%%%%%%%%%%%%%%%%%%%%%%%%%%
\newpage
\subsection{\texttt{.MEASURE } (Continuous results)}
\index{output!control}
\index{results!output control}
\index{\texttt{.MEASURE continuous}}
% Sandia National Laboratories is a multimission laboratory managed and
% operated by National Technology & Engineering Solutions of Sandia, LLC, a
% wholly owned subsidiary of Honeywell International Inc., for the U.S.
% Department of Energy’s National Nuclear Security Administration under
% contract DE-NA0003525.

% Copyright 2002-2024 National Technology & Engineering Solutions of Sandia,
% LLC (NTESS).

\label{Measure_CONT_section}

``Continuous'' measure results are supported for \texttt{DERIV-AT}, \texttt{DERIV-WHEN},
\texttt{FIND-AT}, \texttt{FIND-WHEN}, \texttt{WHEN} and \texttt{TRIG-TARG} measures
for \texttt{.TRAN}, \texttt{.DC}, \texttt{.AC} and \texttt{.NOISE}
analyses.  They are identical to the ``non-continuous'' versions, except that
they can return more than one measured value in some cases.

\begin{Command}
\format
\begin{alltt}
.MEASURE <AC_CONT|DC_CONT|NOISE_CONT|TRAN_CONT> <result name>
+ DERIV <variable> AT=<value>
+ [MINVAL=<value>] [DEFAULT_VAL=<value>]
+ [PRECISION=<value>] [PRINT=<value>]

.MEASURE TRAN_CONT <result name>
+ DERIV <variable> WHEN <variable>=<variable\(\sb{2}\)>|<value>
+ [MINVAL=<value>] [FROM=<value>] [TO=<value>] [TD=<value>]
+ [RISE=r|LAST] [FALL=f|LAST] [CROSS=c|LAST]
+ [DEFAULT_VAL=<value>] [PRECISION=<value>] [PRINT=<value>]

.MEASURE <AC_CONT|DC_CONT|NOISE_CONT> <result name>
+ DERIV <variable> WHEN <variable>=<variable\(\sb{2}\)>|<value>
+ [MINVAL=<value>] [FROM=<value>] [TO=<value>]
+ [RISE=r|LAST] [FALL=f|LAST] [CROSS=c|LAST]
+ [DEFAULT_VAL=<value>] [PRECISION=<value>] [PRINT=<value>]

.MEASURE <AC_CONT|DC_CONT|NOISE_CONT|TRAN_CONT> <result name>
+ FIND <variable> AT=<value>
+ [MINVAL=<value>] [DEFAULT_VAL=<value>]
+ [PRECISION=<value>] [PRINT=<value>]

.MEASURE TRAN_CONT <result name>
+ FIND <variable> WHEN <variable>=<variable\(\sb{2}\)>|<value>
+ [FROM=<value>] [TO=<value>] [TD=<value>]
+ [RISE=r|LAST] [FALL=f|LAST] [CROSS=c|LAST]
+ [MINVAL=<value>] [DEFAULT_VAL=<value>]
+ [PRECISION=<value>] [PRINT=<value>]

.MEASURE <AC_CONT|DC_CONT|NOISE_CONT> <result name>
+ FIND <variable> WHEN <variable>=<variable\(\sb{2}\)>|<value>
+ [FROM=<value>] [TO=<value>]
+ [RISE=r|LAST] [FALL=f|LAST] [CROSS=c|LAST]
+ [MINVAL=<value>] [DEFAULT_VAL=<value>]
+ [PRECISION=<value>] [PRINT=<value>]

.MEASURE TRAN_CONT <result name>
+ WHEN <variable>=<variable\(\sb{2}\)>|<value>
+ [FROM=<value>] [TO=<value>] [TD=<value>]
+ [RISE=r|LAST] [FALL=f|LAST] [CROSS=c|LAST]
+ [MINVAL=<value>] [DEFAULT_VAL=<value>]
+ [PRECISION=<value>] [PRINT=<value>]

.MEASURE <AC_CONT|DC_CONT|NOISE_CONT> <result name>
+ WHEN <variable>=<variable\(\sb{2}\)>|<value>
+ [FROM=<value>] [TO=<value>]
+ [RISE=r|LAST] [FALL=f|LAST] [CROSS=c|LAST]
+ [MINVAL=<value>] [DEFAULT_VAL=<value>]
+ [PRECISION=<value>] [PRINT=<value>]

.MEASURE <AC_CONT|DC_CONT|NOISE_CONT|TRAN_CONT> <result name>
+ TRIG <variable\(\sb{1}\)>=<variable\(\sb{2}\)>|<value> 
+ [TD=<val>] [RISE=r] [FALL=f] [CROSS=c]
+ TARG <variable\(\sb{3}\)>=<variable\(\sb{4}\)>|<value> 
+ [TD=<val>] [RISE=r] [FALL=f] [CROSS=c]
+ [MINVAL=<value>] [DEFAULT_VAL=<value>]
+ [PRECISION=<value>] [PRINT=<value>]

.MEASURE <AC_CONT|DC_CONT|NOISE_CONT|TRAN_CONT> <result name>
+ TRIG AT=<value> TARG AT=<value> 
+ [MINVAL=<value>] [DEFAULT_VAL=<value>]
+ [PRECISION=<value>] [PRINT=<value>]

\end{alltt}
\index{results!measure continuous}

\examples
\begin{alltt}
.MEASURE TRAN_CONT DERIV1At5 DERIV V(1) AT=5
.MEASURE DC_CONT deriv2 DERIV WHEN V(2)=0.75
.MEASURE AC_CONT find1at5 FIND V(1) AT=5
.MEASURE NOISE_CONT find1 FIND V(1) WHEN V(2)=1 RISE=2
.MEASURE TRAN_CONT whenv1 WHEN V(1)=5
.MEASURE TRAN_CONT TrigTargAT TRIG AT=2ms TARG AT=8ms
.MEASURE TRAN_CONT TrigTargAT1 TRIG V(1)=0.2 CROSS=1
+ TARG AT=8ms
.MEASURE TRAN_CONT TrigTargAT2 TRIG AT=2ms
+ TARG V(1)=0.2 CROSS=1
.MEASURE TRAN_CONT TrigTarg TRIG V(1)=0.2 CROSS=1
+ TARG V(1)=0.3 CROSS=1 TD=8ms
\end{alltt}

\arguments

\begin{Arguments}
\argument{result name}

Measured results are reported to the log file and (possibly) multiple
output files. Section~\ref{Measure_CONT_Measurement_Output} below gives more
information on the output files produced by continuous mode measures.

The \texttt{<result name>} must be a legal \Xyce{} character string.
If multiple measures are defined with the same \texttt{<result name>} then
\Xyce{} uses the last such definition, and issues warning messages about
(and discards) any previous measure definitions with the same
\texttt{<result name>}.

\argument{measure type}

\texttt{DERIV, FIND, WHEN, TRIG, TARG}

The third argument specifies the type of measurement or calculation to
be done.

By default, the measurement is performed over the entire simulation.
The calculations can be limited to a specific measurement window by
using the qualifiers {\tt FROM}, {\tt TO}, {\tt TD}, {\tt RISE}, {\tt
FALL}, {\tt CROSS} and {\tt MINVAL}, which are explained below and in
section~\ref{Measure_section}..

The supported ``continuous'' measure types and their definitions are:

\begin{description}
 \item[\tt DERIV] Computes the derivative of {\tt <variable>} at a
    user-specified time (by using the {\tt AT} qualifier) or when a
    user-specified condition occurs (by using the {\tt WHEN}
    qualifier). If the {\tt WHEN} qualifier is used then the
    measurement window can be limited with the qualifiers {\tt FROM},
    {\tt TO}, {\tt RISE}, {\tt FALL} and {\tt CROSS} for all measure
    modes.  In addition, the {\tt TD} qualifier is supported for
    {\tt TRAN\_CONT} measures. The {\tt MINVAL} qualifier is used as a
    comparison tolerance for both {\tt AT} and {\tt WHEN}.  For HSPICE
    compatibility, {\tt DERIVATIVE} is an allowed synonym for {\tt
    DERIV}.

   \item[\tt FIND-AT] Returns the value of {\tt <variable>} at the
    time when the {\tt AT} clause is satisfied.  The {\tt AT}
    clause is described in more detail later in this list.

  \item[\tt FIND-WHEN] Returns the value of {\tt <variable>} at the
    time when the {\tt WHEN} clause is satisfied.  The {\tt WHEN}
    clause is described in more detail later in this list.

  \item[\tt WHEN] Returns the time (or frequency or DC sweep value) when
    {\tt <variable>} reaches {\tt <variable\(\sb{2}\)>} or the constant
    value, {\tt value}.  The measurement window can be limited with the
    qualifiers {\tt FROM}, {\tt TO}, {\tt RISE}, {\tt FALL} and {\tt CROSS}
    for all measure modes.  In addition, the {\tt TD} qualifier is supported
    for {\tt TRAN\_CONT} measures. The qualifier {\tt MINVAL} acts as a
    tolerance for the comparison.  For example when {\tt <variable\(\sb{2}\)>}
    is specified, the comparison used is when {\tt <variable>} $=$
    {\tt <variable\(\sb{2}\)>} $\pm$ {\tt MINVAL} or when a constant,
    {\tt value} is given: {\tt <variable>} $=$ {\tt value} $\pm$ {\tt
    MINVAL}.  If the conditions specified for finding a given value
    were not found during the simulation then the measure will return
    the default value of {\tt -1}.  The user may change this default
    value with the {\tt DEFAULT\_VAL} qualifier.  Note: The use of
    {\tt FIND} and {\tt WHEN} in one measure statement is also supported.

  \item[\vbox{\hbox{\tt TRIG\hfil}\hbox{\tt TARG\hfil}}] Measures the
    time between a trigger event and a target event.  The trigger is
    specified with {\tt TRIG <variable\(\sb{1}\)>=<variable\(\sb{2}\)>} or {\tt
    TRIG <variable\(\sb{1}\)>=<value>} or {\tt TRIG AT=<value>}.  The target
     is specified as {\tt TARG <variable\(\sb{3}\)>=<variable\(\sb{4}\)>}
    or {\tt TARG <variable\(\sb{3}\)>=<value>} or {\tt TARG AT=<value>}.  The
    measurement window can be limited with the qualifiers {\tt TD}, {\tt RISE},
    {\tt FALL} and {\tt CROSS} for all measure modes.  The qualifier {\tt MINVAL}
     acts as a tolerance for the comparison.  For example when {\tt <variable\(\sb{2}\)>}
    is specified, the comparison used is when {\tt <variable\(\sb{1}\)>} $=$
    {\tt <variable\(\sb{2}\)>} $\pm$ {\tt MINVAL} or when a constant,
    {\tt value} is given: {\tt <variable\(\sb{1}\)>} $=$ {\tt value} $\pm$ {\tt
    MINVAL}.  If the conditions specified for finding a given value
    were not found during the simulation then the measure will return
    the default value of {\tt -1}.  The user may change this default
    value with the {\tt DEFAULT\_VAL} qualifier.  
\end{description}

\argument{\vbox{\hbox{variable\hfil}\hbox{variable\(\sb{n}\)\hfil}\hbox{value}}}

These quantities represents the test for the stated
measurement.  \texttt{<variable>} is a simulation quantity, such as a
voltage or current.  One can compare it to another simulation variable
or a fixed quantity.  Additionally, the \texttt{<variable>} may be
a \Xyce{} expression delimited by \{ \} brackets.  As noted above, an
example is {\tt V(2)=0.75}
\end{Arguments}

Additional information on the \texttt{TO}, \texttt{FROM}, \texttt{TD},
\texttt{RISE}, \texttt{FALL}, \texttt{CROSS}, \texttt{MINVAL},
\texttt{DEFAUAL\_VAL}, \texttt{PRECISION} and \texttt{PRINT} qualifiers
is given in section~\ref{Measure_section}.

\end {Command}

\subsubsection{Measure Output}
\label{Measure_CONT_Measurement_Output}
\index{measure continuous!measurement output}
As discussed in section~\ref{Measure_Measurement_Output}, measured results
for \texttt{AC}, \texttt{DC}, \texttt{NOISE} and \texttt{TRAN} mode measures
are reported to the log file.  Additionally, for \texttt{TRAN} measures, the
results are stored in files called \texttt{circuitFileName.mt\#}, where the
suffixed number (\texttt{\#}) starts at \texttt{0} and increases for multiple
iterations (\texttt{.STEP} iterations) of a given simulation. For \texttt{DC}
measures, the results are stored in the files \texttt{circuitFileName.ms\#},
while \texttt{AC} and \texttt{NOISE} measures use the files
\texttt{circuitFileName.ma\#}.

For \texttt{AC\_CONT}, \texttt{DC\_CONT}, \texttt{NOISE\_CONT} and
\texttt{TRAN\_CONT} mode measures, the output for successful and failed
measures is sent to the standard output (and log files), as described in
section ~\ref{Measure_Measurement_Output}.  There are two options for the
output files though. The default is for each continuous mode measure to generate its
own output file where, for example for a non-step transient analysis, the file name
would be \texttt{circuitFileName\_resultname.mt0} where the result (measure) name is
always output in lower-case. This default matches HSPICE. The second option uses
\texttt{.OPTIONS MEASURE USE\_CONT\_FILES=0}. In that case, the results for all
of the continuous mode measures are sent to the \texttt{circuitFileName.mt\#} file.

An example is as follows.

\begin{alltt}
VPWL1 1 0 pwl(0 0 2.5m 1 5m 0 7.5m 1 10m 0)
R1 1 0 1

.TRAN 0 10ms
.PRINT TRAN V(1)

.MEASURE TRAN MAXV1 MAX V(1)
.MEASURE TRAN_CONT FindV1 WHEN V(1)=0.5
.MEASURE TRAN_CONT FindV1AT FIND V(1) AT=0.6ms

.END
\end{alltt}

The result for measure \texttt{MAXV1} is sent to \texttt{<netlistName>.mt0}. The
results for measures \texttt{FindV1} and \texttt{FindV1AT} are then sent to individual
files, named \texttt{<netlistName>\_findv1.mt0} and \texttt{<netlistName>\_findv1at.mt0}.
Note that the measure names have been lower-cased in the output file names.  The contents
of those files are then as follows.

\begin{alltt}
FINDV1 = 1.250000e-03
FINDV1 = 3.750000e-03
FINDV1 = 6.250000e-03
FINDV1 = 8.750000e-03
\end{alltt}

and:

\begin{alltt}
FINDV1AT = 2.400000e-01
\end{alltt}

Note that \texttt{FIND-AT} measures will still only return one measure value, even for
\texttt{TRAN\_CONT} measure mode.  However, in this simple example, the specified
\texttt{FIND-WHEN} measure returns all four times where \texttt{V(1)} equals 0.5.
The next subsection will describe how the \texttt{RISE}, \texttt{FALL} and \texttt{CROSS}
qualifiers can used to return only a subset of those four crossings.

\subsubsection{RISE, FALL and CROSS Qualifiers}
\label{Measure_CONT_RFC}
\index{measure continuous!rise, fall and cross qualifiers}
\Xyce{} supports non-negative values for the \texttt{RISE}, \texttt{FALL}
and \texttt{CROSS} qualifiers for all continuous measure types.  It supports
negative values for the \texttt{RISE}, \texttt{FALL} and \texttt{CROSS} qualifiers
for the \texttt{DERIV-WHEN}, \texttt{FIND-WHEN} and \texttt{WHEN} measure types.
However, their interpretation is slightly different for
\texttt{TRAN} and \texttt{TRAN\_CONT} measure modes, as illustrated by the following
netlist for the \texttt{TRAN\_CONT} measure mode and \texttt{WHEN} measure. The rules
are then the same for the other continuous measures modes and the \texttt{RISE} and
\texttt{FALL} qualifiers.

\begin{alltt}
VPWL1 1 0 pwl(0 0 2.5m 1 5m 0 7.5m 1 10m 0)
R1 1 0 1

.TRAN 0 10ms
.PRINT TRAN V(1)

.MEASURE TRAN FindV1_CROSS3 WHEN V(1)=0.5 CROSS=3
.MEASURE TRAN_CONT FindV1_CONT_CROSS3 WHEN V(1)=0.5 CROSS=3

.MEASURE TRAN FindV1_CROSS_NEG3 WHEN V(1)=0.5 CROSS=-3
.MEASURE TRAN_CONT FindV1_CONT_CROSS_NEG3 WHEN V(1)=0.5 CROSS=-3

.END
\end{alltt}

The \texttt{<netlistName>.mt0} file will contain the results for both
\texttt{TRAN} mode measures.  The result for the \texttt{FindV1\_CROSS3}
is the time of the third crossing.  The result for the \texttt{FindV1\_CROSS\_NEG3}
is the time of the second crossing, which is also the ``third to last'' (or
negative third) crossing in this case.

\begin{alltt}
FINDV1_CROSS3 = 6.250000e-03
FINDV1_CROSS_NEG3 = 3.750000e-03
\end{alltt}

The \texttt{<netlistName>\_findv1\_cont\_cross3.mt0} output file will have two
values.  For non-negative values of \texttt{CROSS}, a \texttt{TRAN\_CONT} measure
will return all crossings, starting with the specified value.  This is the third
and fourth crossings in this case.

\begin{alltt}
FINDV1_CONT_CROSS3 = 6.250000e-03
FINDV1_CONT_CROSS3 = 8.750000e-03
\end{alltt}

For negative values of \texttt{CROSS}, a \texttt{TRAN\_CONT} measure will only
return one value.  That is the third-to-last crossing in this case.  So, the
\texttt{<netlistName>\_findv1\_cont\_cross3\_neg3.mt0} file only has one
value in it.  As a final note, a \texttt{CROSS} value of either 5 or -5 would
produce failed measures in this example.

\begin{alltt}
FINDV1_CONT_CROSS_NEG3 = 3.750000e-03
\end{alltt}

\subsubsection{AT and TD Qualifiers for TRIG-TARG}
\index{measure continuous!AT and TD qualifiers for TRIG-TARG}
The following rules apply to the \texttt{AT} and \texttt{TD} qualifiers
for \texttt{TRIG-TARG} measures:

\begin{XyceItemize}
  \item Separate \texttt{AT} values can be given for the \texttt{TRIG} and
\texttt{TARG} clauses.
  \item Separate \texttt{TD} values can be given for the \texttt{TRIG} and
\texttt{TARG} clauses.
  \item The \texttt{AT} value takes precedence over the \texttt{TD} qualifier if
both are given in a \texttt{TRIG} or \texttt{TARG} clause.
  \item If the \texttt{TD} value is only given for the \texttt{TRIG} clause then
that value will be used for both the \texttt{TRIG} and \texttt{TARG} clauses.
  \item An \texttt{AT} value that is outside of the simulation window, or a 
\texttt{TD} value that is greater than the end simulation time or the largest
\texttt{AC}, \texttt{DC} or \texttt{NOISE} sweep value, will produce a failed
measure.
  \item A \texttt{TD} value that is less than 0,  or the smallest \texttt{AC},
\texttt{DC} or \texttt{NOISE} sweep value, is essentially ignored.
\end{XyceItemize}

\subsubsection{HSPICE Compatibility}
\label{Measure_CONT__HSpice_Compatibility}
\index{measure continuous!HSPICE compatibility}
There are known incompatibilities between the \Xyce{} and HSPICE implementation
of continuous measures.  They include the following:

\begin{XyceItemize}
  \item \Xyce{} will not return a trig or targ value that is outside of the simulation
bounds.  In some case, HSPICE will return a trig or targ value that is earlier than
the start of the simulation window.
  \item \Xyce{} does not support negative values for the \texttt{RISE}, \texttt{FALL} or
\texttt{CROSS} qualifiers for the continuous version of the \texttt{TRIG-TARG} measure.

\end{XyceItemize}



%%%%%%%%%%%%%%%%%%%%%%%%%%%%%%%%%%%%%%%%%%%%%%%%%%%%%%%%%%%%%%%%%%%%%%%%%%%%%%%%
\newpage
\subsection{\texttt{.MEASURE FFT} (Measure output for .FFT)}
\index{output!control}
\index{results!output control}
\index{\texttt{.MEASURE FFT}}
% Sandia National Laboratories is a multimission laboratory managed and
% operated by National Technology & Engineering Solutions of Sandia, LLC, a
% wholly owned subsidiary of Honeywell International Inc., for the U.S.
% Department of Energy’s National Nuclear Security Administration under
% contract DE-NA0003525.

% Copyright 2002-2024 National Technology & Engineering Solutions of Sandia,
% LLC (NTESS).

The \texttt{.MEASURE FFT} statement allows calculation or reporting of simulation
metrics, from data associated with .FFT analyses, to an external file as well as to
the standard output and/or a log file, So, it is only supported for \texttt{.TRAN},
analyses.  It can be used with {\tt .STEP}.  For HSPICE compatibility, \texttt{.MEAS}
is an allowed synonym for \texttt{.MEASURE}.

The syntaxes for the \texttt{.MEASURE FFT} statements are shown below.

\begin{Command}
\format
\begin{alltt}
.MEASURE FFT <result name> ENOB <variable> [BINSIZ=<value>]
+ [DEFAULT_VAL=<value>] [PRECISION=<value>] [PRINT=<value>]

.MEASURE FFT <result name> EQN <variable>
+ [DEFAULT_VAL=<value>] [PRECISION=<value>] [PRINT=<value>]

.MEASURE FFT <result name> FIND <variable> AT=<value>
+ [PRECISION=<value>] [PRINT=<value>]

.MEASURE FFT <result name> SFDR <variable>
+ [MINFREQ=<value>] [MAXFREQ=<value>]  [BINSIZ=<value>]
+ [DEFAULT_VAL=<value>] [PRECISION=<value>] [PRINT=<value>]

.MEASURE FFT <result name> SNDR <variable>  [BINSIZ=<value>]
+ [DEFAULT_VAL=<value>] [PRECISION=<value>] [PRINT=<value>]

.MEASURE FFT <result name> SNR <variable> [MAXFREQ=<value>]
+ [DEFAULT_VAL=<value>] [PRECISION=<value>] [PRINT=<value>]

.MEASURE FFT <result name> THD <variable>
+ [NBHARM=<value>] [MAXFREQ=<value>]
+ [DEFAULT_VAL=<value>] [PRECISION=<value>] [PRINT=<value>]

\end{alltt}
\index{\texttt{.MEASURE FFT}}
\index{results!measure fft}

\examples
\begin{alltt}
.FFT V(1) NP=16
.MEASURE FFT ENOBVAL ENOB V(1)
.MEASURE FFT EQNVAL EQN VR1AT2
.MEASURE FFT VR1AT2 VR(1) AT=2
.MEASURE FFT SFDRVAL SFDR V(1)
.MEASURE FFT SNDRVAL SNDR V(1)
.MEASURE FFT SNRVAL SNR V(1)
.MEASURE FFT THDVAL THD V(1)
\end{alltt}

\arguments

\begin{Arguments}
\argument{result name}

Measured results are reported to the output and log file.
Additionally, the results are stored in files called
\texttt{circuitFileName.mt\#}, where the suffixed number
(\texttt{\#}) starts at \texttt{0} and increases for multiple
iterations (\texttt{.STEP} iterations) of a given simulation. Each
line of this file will contain the measurement name, \texttt{<result
name>}, followed by its value for that run.  The \texttt{<result
name>} must be a legal \Xyce{} character string.

If multiple measures are defined with the same \texttt{<result name>} then
\Xyce{} uses the last such definition, and issues warning messages about
(and discards) any previous measure definitions with the same
\texttt{<result name>}.

\argument{measure type}

\texttt{ENOB, EQN, FIND, SFDR, SNDR, SNR, THD}

The third argument specifies the type of measurement or calculation to
be done. By default, the measurement is performed over the time window defined
by the {\tt START} and {\tt STOP} parameters on the associated {\tt .FFT}
line.  So, the  {\tt FROM}, {\tt TO} and {\tt TD} qualifiers have no
effect on FFT-based measures.

The supported measure types are:

\begin{description}
  \item[\tt ENOB] Calculates the ``Effective Number of Bits'', where that metric
    is defined in Section \ref{FFT_metrics} which covers the \texttt{.FFT} command.

  \item[\tt EQN] Calculates the value of {\tt <variable>} during the simulation.
    That variable can use the results of other measure statements. {\tt PARAM}
    is an allowed synonym for {\tt EQN} as a measure type.  For {\tt FFT} measure
    mode, an {\tt EQN} measure will be reported as ``failed'' until the associated
    FFT has been calculated.

  \item[\tt FIND] Returns the requested FFT cofficient at the requested
    frequency.  Examples of the mapping of \Xyce{} operators (e.g., {\tt VM} and {\tt IM})
    to FFT cofficients is given in the ``Additional Examples'' subsection below.
    {\tt FIND} measures can be also used in conjunction with {\tt EQN} measures
    to generate fairly arbitrary FFT-based measures.  The {\tt FIND} measure for
    {\tt FFT} measure mode does not support expressions, or the {\tt P} and
    {\tt W} operators.  It also does not support multi-terminal lead current
    operators, such as {\tt IC()}.

  \item[\tt SFDR] Calculates the ``Spurious Free Dynamic Range'', where that metric
    is defined in Section \ref{FFT_metrics}.

  \item[\tt SNDR] Calculates the ``Signal to Noise-plus-Distortion Ratio'', where that metric
    is defined in Section \ref{FFT_metrics}.

  \item[\tt SNR] Calculates the ``Signal to Noise Ratio'', where that metric
    is defined in Section \ref{FFT_metrics}.

  \item[\tt THD] Calculates the ``Total Harmonic Distortion'', where that metric
    is defined further below and also in in Section \ref{FFT_metrics}.
\end{description}

\argument{variable}

The \texttt{<variable>} is a simulation quantity, such as a
voltage or current.  Additionally, the \texttt{<variable>} may be
a \Xyce{} expression delimited by \{ \} brackets.  The only constraint
is that the \texttt{variable} on the \texttt{.MEASURE FFT} line must
be an exact match for the \texttt{ov} on at least one \texttt{.FFT} line
in the netlist.  If there are multiple \texttt{.FFT} lines in the netlist
with the same \texttt{ov} then the corresponding \texttt{.MEASURE FFT}
statements will use the first such one.

\argument{AT=value}
A frequency {\em at which} the measurement calculation will occur.  This is
used by the {\tt FIND} measure only.  The entered {\tt AT} value will be
rounded to the nearest harmonic frequency, as defined by the {\tt FREQ},
{\tt START} and {\tt STOP} parameters on the associated {\tt .FFT} line.
An {\tt AT} value that rounds to a harmonic frequency of less than zero,
or to more than {\tt NP/2}, will produce a failed measure in \Xyce{}, where
{\tt NP} is the number of points specified on the associated {\tt .FFT} line..
The behavior of these ''failed'' cases may differ from commercial simulators.

\argument{BINSIZ=value}
This parameter is implemented in \Xyce{} for the \texttt{ENOB}, \texttt{SFDR}
and \texttt{SNDR} measure types. It can be used to account for any ``broadening''
of the spectral energy in the first harmonic of the signal, as discussed below.
\texttt{BINSIZ} has a default value of 0.

\argument{DEFAULT\_VAL=value}

If the conditions specified for finding a given value are not found
during the simulation then the measure will return the default value
of {\tt -1} in the \texttt{circuitFileName.mt\#} file.
The measure value in the standard output or log file will be
FAILED.  The default return value for the \texttt{circuitFileName.mt\#}
file is settable by the user for each measure by adding the qualifier
{\tt DEFAULT\_VAL=<retval>} on that measure line.  If either
\texttt{.OPTIONS MEASURE MEASFAIL=<val>} or
\texttt{.OPTIONS MEASURE DEFAULT\_VAL=<val>} are given in the
netlist then those values override the \texttt{DEFAULT\_VAL}
parameters given on individual \texttt{.MEASURE FFT} lines.
See Section \ref{Options_Reference} for more details.

\argument{MAXFREQ=value}
The maximum frequency over which to perform a {\tt SFDR}, {\tt SNR}
or {\tt THD} measure. The entered {\tt MAXFREQ} value will be rounded
to the nearest harmonic frequency, as defined by the {\tt FREQ},
{\tt START} and {\tt STOP} parameters on the associated {\tt .FFT}
line.  The default value is {\tt NP/2}, where {\tt NP} is the number
of points specified on the associated {\tt .FFT} line.

\argument{MINFREQ=value}
The minimum frequency over which to perform a {\tt SFDR} or {\tt THD} measure.
The entered {\tt MINFREQ} value will be rounded to the nearest harmonic
frequency, as defined by the {\tt FREQ}, {\tt START} and {\tt STOP}
parameters on the associated {\tt .FFT} line.  The default value is 1.

\argument{NBHARM=value}
The maximum (integer) number of harmonics over which to perform a THD measure.
The default value is {\tt NP/2}.  The {\tt NBHARM} qualifier has precedence
over the {\tt MAXFREQ} qualifier if both are given on a {\tt .MEASURE} line.

\argument{PRECISION=value}

The default precision for {\tt .MEASURE} output is 6 digits after the
decimal point.  This argument provides a user configurable precision
for a given {\tt .MEASURE} statement that applies to both the
\texttt{.mt\#} file and standard output.
If \texttt{.OPTIONS MEASURE MEASDGT=<val>} is given in the netlist
then that value overrides the \texttt{PRECISION} parameters given on
individual \texttt{.MEASURE} lines.

\argument{PRINT=value}

This parameter controls where the {\tt .MEASURE} output appears.  The
default is {\tt ALL}, which produces measure output in both the
\texttt{.mt\#} and the standard output.  A value of {\tt STDOUT}
only produces measure output to standard output, while a value of
{\tt NONE} suppresses the measure output to both the \texttt{.mt\#}
file and standard output.

\end{Arguments}

\end {Command}

\subsubsection{Measure Definitions}
The \texttt{ENOB}, \texttt{SNDR}, \texttt{SNR} and \texttt{SFDR} measure
types use the same definitions as the metrics produced by \texttt{.FFT} lines.
Section \ref{FFT_metrics} provides more details on those definitions.  (Note: The
\texttt{MAXFREQ} and \texttt{MINFREQ} qualifiers from the \texttt{.MEASURE FFT}
lines are mapped into the \texttt{FMAX} and \texttt{FMIN} parameters used in those
equations.)  There are two exceptions.

The first exception is the \texttt{THD} measure.  If the optional \texttt{NBHARM}
qualifier is not used then the definition given in Section \ref{FFT_metrics} is
used.  If the \texttt{NBHARM} qualifier is used then it takes precedence over the
\texttt{MAXFREQ} qualifier.  The \texttt{THD} measure definition is then as follow.
Let $f_{0}$ be the integer index of the ``first harmonic'' ($f_{0}$), as defined in
Section \ref{FFT_metrics}, from the associated \texttt{.FFT} line.  Then the
effective value of the upper frequency limit ($f_{2}$) in the THD calculation
is \texttt{NBHARM}$\cdot f_{0}$, with the caveat that all of the harmomics will
be used if \texttt{NBHARM} < 0 or \texttt{NBHARM} > \texttt{NP/2}.

The second exception is the \texttt{BINSIZ} qualifier for the \texttt{ENOB},
\texttt{SFDR} and \texttt{SNDR} measures.  For a non-zero value of
\texttt{BINSIZ}, the ``signal power'' is considered to reside in the harmonic
indexes between ($f_{0} \pm$ \texttt{BINSIZ}), where the DC value is still
excluded from the measure calculations. (Note: This definition for 
\texttt{BINSIZ} may differ from HSPICE.)

\subsubsection{Re-Measure}
\label{Measure_FFT_ReMeasure}
\index{measure fft!re-measure}
\Xyce{} can re-calculate (or re-measure) the values for {\tt .MEASURE FFT}
statements using existing \Xyce{} output files.  Section~\ref{Measure_ReMeasure}
discusses this topic in more detail for both {\tt .MEASURE} and {\tt .FFT}
statements.

\subsubsection{Additional Examples}
\label{Measure_FFT_Additional_Examples}
\index{measure fft!additional examples}
This section provides a simple example how to use the {\tt FIND} measure,
along with the {\tt V()}, {\tt VR()}, {\tt VI()}, {\tt VM()}, {\tt VP()}
and {\tt VDB()} operators, to obtain the real and imaginary parts of the
FFT coefficients, along with the magnitude and phase of those coefficients,
at a specified frequency.  Those coefficient values are unnormalized.

\begin{alltt}
* Example of obtaining FFT coefficients
.TRAN 0 1
.PRINT TRAN V(1)
.OPTIONS FFT FFT_ACCURATE=1 FFTOUT=1
V1 1 0 1
R1 1 0 1
.FFT V(1) NP=8 WINDOW=HANN

* Unnormalized one-sided FFT cofficients for V(1) at F=1.0
* Magnitude
.MEASURE FFT V1AT1 FIND V(1) AT=1.0
* Real part
.MEASURE FFT VR1AT1 FIND VR(1) AT=1.0
* Imaginary part
.MEASURE FFT VI1AT1 FIND VI(1) AT=1.0
* Magnitude
.MEASURE FFT VM1AT1 FIND VM(1) AT=1.0
* Phase
.MEASURE FFT VP1AT1 FIND VP(1) AT=1.0
* Magnitude in dB
.MEASURE FFT VDB1AT1 FIND VDB(1) AT=1.0

.END
\end{alltt}

The {\tt .MEASURE} output is then:
\begin{alltt}
FINDV1AT1 = 5.200051e-01
FINDVR1AT1 = -4.804221e-01
FINDVI1AT1 = -1.989973e-01
FINDVM1AT1 = 5.200051e-01
FINDVP1AT1 = -1.575000e+02
FINDVDB1AT1 = -5.679847e+00
\end{alltt}




%%%%%%%%%%%%%%%%%%%%%%%%%%%%%%%%%%%%%%%%%%%%%%%%%%%%%%%%%%%%%%%%%%%%%%%%%%%%%%%%
\newpage
\subsection{\texttt{.MODEL} (Model Definition)}
\index{model!definition}
% Sandia National Laboratories is a multimission laboratory managed and
% operated by National Technology & Engineering Solutions of Sandia, LLC, a
% wholly owned subsidiary of Honeywell International Inc., for the U.S.
% Department of Energy’s National Nuclear Security Administration under
% contract DE-NA0003525.

% Copyright 2002-2023 National Technology & Engineering Solutions of Sandia,
% LLC (NTESS).

\label{modelCommand}
The \texttt{.MODEL} command provides a set of device parameters to be
referenced  by device instances in the circuit.

\begin{Command}

\format
.MODEL <model name> <model type> (<name>=<value>)*

\examples
\begin{alltt}
.MODEL RMOD R (RSH=1)
.MODEL MOD1 NPN BF=50 VAF=50 IS=1.E-12 RB=100 CJC=.5PF TF=.6NS
.MODEL NFET NMOS(LEVEL=1 KP=0.5M VTO=2V)
\end{alltt}

\arguments

\begin{Arguments}

\argument{model name}
The model name used to reference the model.

\argument{model type}

The model type used to define the model.  This determines if the model
is (for example) a resistor, or a MOSFET, or a diode, etc.  For
transistors, there will usually be more than one type possible, such as
NPN and PNP for BJTs, and NMOS and PMOS for MOSFETs.

\argument{\vbox{\hbox{name\hfil}\hbox{value}}}

The name of a parameter and its value.  Most models will have a list of
parameters available for specification.  Those which are not set will
receive default values.  Most will be floating point numbers, but some
can be integers and some can be strings, depending on the definition of
the model.

\end{Arguments}

\comments
\index{\texttt{.MODEL}!subcircuit scoping}The scoping rules for models are:
\begin{XyceItemize}
\item If a \texttt{.MODEL}, statement is included in the main circuit 
netlist, then it is accessible from the main circuit and all subcircuits. 
\item \texttt{.MODEL} statements defined within a subcircuit are scoped 
to that subciruit definition.  So, their models are only accessible within 
that subcircuit definition, as well as within ``nested subcircuits'' also 
defined within that subcircuit definition.
\end{XyceItemize}

Additional illustative examples of scoping are given in the
``Working with Subcircuits and Models'' section of the \Xyce{} Users' 
Guide\UsersGuide. 

A model name can be the same as a device name in \Xyce{}.  However, that
usage will generate a warning message during netlist parsing.  
The reason is that it can lead to ambiguous \texttt{.PRINT} lines 
when a model parameter and instance parameter, for a given device, have
the same name but a different meaning.  For example, \texttt{R1} 
could be used as both a resistor device-name, and as a resistor model-name.
However, \texttt{.PRINT TRAN R1:R} would then be ambiguous.  In addition,
the use of duplicate model and device names is not recommended if those
names will be used within a \Xyce{} expression since that can result in 
an ambiguous expression.

\end{Command}

\subsubsection{\textrmb{LEVEL} Parameter}
\index{model!level parameter}
\index{level parameter}

A common parameter is the \textrmb{LEVEL} parameter, which is set to an
integer value.  This parameter will define exactly which model of the
given type is to be used.  For example, there are many different
available MOSFET models.  All of them will be specified using the same
possible names and types.  The way to differentiate (for example)
between the BSIM3 model and the PSP model is by setting the appropriate
\textrmb{LEVEL}.

\subsubsection{Model Binning}
\index{model!binning}
\index{model binning}

Model binning is supported for MOSFET models.  For model binning, the netlist 
contains a set of similar \texttt{.MODEL} cards which correspond to different 
sizing information (length and width).  They are similar in that they are for the same
model (and same \texttt{LEVEL} number), and have the same prefix.  They are different in that 
they have different \texttt{lmin,lmax,wmin,wmax} parameters, and the name suffix will be
the bin number.  For a MOSFET device instance, \Xyce{} will automatically select the
appropriate binned model, based on the \texttt{L} and \texttt{W} parameters of that 
instance.   It will only seach the models with matching name prefixes, and can only work if
all the binned models have specified all the \texttt{lmin,lmax,wmin,wmax} parameters.

Model binning is enabled by default.  To disable it, specify \texttt{.options parser model\_binning=false}.

\begin{figure}[htbp]
  \begin{centering}
    \shadowbox{
      \begin{minipage}{0.9\textwidth}
        \begin{vquote}
\color{blue}* Model binning example adapted from the BSIM4 tests\color{black}
m1 2 1 0 b nch L=0.11u W=10.1u NF=5 rgeomod=1 geomod=0
vgs 1 0 1.2
vds 2 0 1.2
Vb b 0 0.0

.dc vds 0.0 1.21 0.02 vgs 0.2 1.21 0.2

.print dc v(2) v(1) i(vds)

* model binning
.model nch.1 nmos(level=14 
+ lmin=0.1u lmax=20u 
+ wmin=0.1u wmax=10u)
.model nch.2 nmos(level=14 
+ lmin=0.1u lmax=20u 
+ wmin=10u  wmax=100u)

.end
\end{vquote}
\end{minipage}
}
\caption[Model Binning Example]
{Model Binning Example\label{binningExample} }
\end{centering}
\end{figure}

\subsubsection{Length Scaling}
\index{model!scale}
\index{scale}

It can be convenient to specify the lengths and widths for MOSFET instances in scaled units.  
To enable this, the netlist should include \texttt{.options parser scale} or \texttt{.option scale}.
This feature is only supported in MOSFET compact models.  An example usage is given in figure~\ref{scaleExample}.
In this example, the scaled length and width for transistor \texttt{mn1} is \texttt{l=5.0e-6} and \texttt{w=175e-6}.
\begin{figure}[htbp]
  \begin{centering}
    \shadowbox{
      \begin{minipage}{0.9\textwidth}
        \begin{vquote}
\color{blue}mos level 1 model cmos inverter using scale]\color{black}
.tran 20ns 6us
.print tran v(vout) v(in) v(1)
vdddev 	vdd	0	5v
rin	in	1	1k
vin1  1	0  5v pulse (5v 0v 1.5us 5ns 5ns 1.5us 3us)
r1    vout  0  10k  
c2    vout  0  0.1p 
mn1   vout  in 0 0 cd4012\_nmos l=5 w=175 
mp1   vout in vdd vdd cd4012\_pmos l=5 w=270 

.options parser scale=1.0e-6

\color{blue}* also valid:
*.option scale=1.0e-6\color{black}

.model cd4012\_pmos pmos (level=1 uo=310  vto=-1.6 tox=6e-08)
.model cd4012\_nmos nmos (level=1 uo=190 vto=1.679 tox=6e-08)
.end
\end{vquote}
\end{minipage}
}
\caption[Scale Example]
{Scale Example\label{scaleExample} }
\end{centering}
\end{figure}



%% %%%%%%%%%%%%%%%%%%%%%%%%%%%%%%%%%%%%%%%%%%%%%%%%%%%%%%%%%%%%%%%%%%%%%%%%%%%%%%%%
%% \newpage
%% \subsection{\texttt{.MPDE} (Multi-Time Partial Differential Equation Analysis)}
%% % Sandia National Laboratories is a multimission laboratory managed and
% operated by National Technology & Engineering Solutions of Sandia, LLC, a
% wholly owned subsidiary of Honeywell International Inc., for the U.S.
% Department of Energy’s National Nuclear Security Administration under
% contract DE-NA0003525.

% Copyright 2002-2023 National Technology & Engineering Solutions of Sandia,
% LLC (NTESS).


Calculates the time-domain response of a circuit with two disparate time scales.
\index{\texttt{.MPDE}}
\index{analysis!MPDE} \index{multi-time partial differential equation analysis}

\begin{Command}}

\format
.MPDE <print step value> <final time value>

\examples
.MPDE 1us 100ms

\arguments

\begin{Arguments}

\argument{print step value}
Used to calculate the initial time step (see .tran).

\arguments{final time value}
Sets the end time (duration) for the analysis.

\end{Arguments}

\end{Command}


%%%%%%%%%%%%%%%%%%%%%%%%%%%%%%%%%%%%%%%%%%%%%%%%%%%%%%%%%%%%%%%%%%%%%%%%%%%%%%%%
\newpage
\subsection{\texttt{.NODESET} (Approximate Initial Condition, Bias point)}
% Sandia National Laboratories is a multimission laboratory managed and
% operated by National Technology & Engineering Solutions of Sandia, LLC, a
% wholly owned subsidiary of Honeywell International Inc., for the U.S.
% Department of Energy’s National Nuclear Security Administration under
% contract DE-NA0003525.

% Copyright 2002-2023 National Technology & Engineering Solutions of Sandia,
% LLC (NTESS).


\index{\texttt{.NODESET}}
\index{initial condition!NODESET}
\index{initial condition}
\label{NODESET_section}

The \texttt{.NODESET} command sets initial conditions for operating point
calculations.  It is similar to \texttt{.IC} (Section~\ref{IC_section}), except 
it is applied as an initial guess, rather than as a firmly enforced
condition. Like \texttt{.IC}, \texttt{.NODESET} initial conditions can be specified
for some or all of the circuit nodes.

Consult the \Xyce{} Users' Guide~\UsersGuide{} for more guidance.

\begin{Command}

\format
\begin{alltt}
.NODESET < V(<node>)=<value>
.NODESET <node> <value>
\end{alltt}

\examples
\begin{alltt}
.NODESET V(2)=3.1
.NODESET 2 3.1
\end{alltt}

\comments
The \Xyce{} \texttt{.NODESET} command uses a different strategy than either
SPICE or HSPICE.  When \texttt{.NODESET} is specified, \Xyce{} does two solves
for the DC operating point.  One with the \texttt{.NODESET} values held as 
initial conditions (i.e., the same as if it was an .IC solve). The second solve
is then done without any conditions imposed, but with the first solution as an
initial guess. 

The \texttt{.NODESET} capability can only set voltage values, not current values.

The \texttt{.NODESET} capability can not be used, within subcircuits, to set voltage values on global nodes.

\end{Command}



%%%%%%%%%%%%%%%%%%%%%%%%%%%%%%%%%%%%%%%%%%%%%%%%%%%%%%%%%%%%%%%%%%%%%%%%%%%%%%%%
\newpage
\subsection{\texttt{.NOISE} (Noise Analysis) }
\index{\texttt{.NOISE}}
\index{analysis!Noise} \index{Noise analysis}
% Sandia National Laboratories is a multimission laboratory managed and
% operated by National Technology & Engineering Solutions of Sandia, LLC, a
% wholly owned subsidiary of Honeywell International Inc., for the U.S.
% Department of Energy’s National Nuclear Security Administration under
% contract DE-NA0003525.

% Copyright 2002-2023 National Technology & Engineering Solutions of Sandia,
% LLC (NTESS).


%%%
%%% Transient Analysis Table
%%%

Calculates the the small signal noise response of a circuit over a range of frequencies.
The .NOISE command can specify a linear sweep, decade logarithmic sweep, octave
logarithmic sweep, or a data table of multivariate values.

\begin{Command}

\format
\begin{alltt}
  .NOISE V(OUTPUT <, REF>) SRC <sweep type> <points value>
+ <start frequency value> <end frequency value>
\end{alltt}

\examples
\begin{alltt}
  .NOISE V(5) VIN LIN 101 100Hz 200Hz
  .NOISE V(5,3) V1 OCT 10 1kHz 16kHz
  .NOISE V(4) V2 DEC 20 1MEG 100MEG
  .NOISE V(4) V2 DATA=<table name>
\end{alltt}

\arguments

\begin{Arguments}

\argument{V(OUTPUT <,REF>)}
   The node at which the total output noise is desired. If REF is 
   specified, then the noise voltage V(OUTPUT) - V(REF) is calculated. By default, 
   REF is assumed to be ground. 

\argument{SRC}
   The name of an independent source to which input noise is referred. 

\argument{sweep type}

Must be \texttt{LIN}, \texttt{OCT}, or \texttt{DEC}, as described below.
\begin{description}

\item[\tt LIN] Linear sweep\\
The sweep variable is swept linearly from the starting to the ending value.

\item[\tt OCT] Sweep by octaves\\
The sweep variable is swept logarithmically by octaves.

\item[\tt DEC] Sweep by decades\\
The sweep variable is swept logarithmically by decades.

\item[\tt DATA] Sweep values from a table\\
Sweep variables are applied based on the rows of a data table.  This format
allows magnitude and phase to be swept in addition to frequency.  If using
this format, then the \texttt{V(OUTPUT <,REF>)} and \texttt{SRC} arguments
are still needed on the \texttt{.NOISE} line.

\end{description}

\argument{points value}
Specifies the number of points in the sweep, using an integer greater than or equal to 1.

\argument{\vbox{\hbox{start frequency value\hfil}\hbox{end frequency value}}}

The end frequency value must not be less than the start frequency value,
and both must be greater than zero. The whole sweep must include at
least one point.

\end{Arguments}

\comments

Noise analysis is a linear analysis. The simulator calculates the noise
response by linearizing the circuit around the bias point.

If specifying the sweep points using the \texttt{DATA} type, one can also
sweep the magnitude and phase of an AC source, as well as the values of
linear model parameters.  However, unlike the use of \texttt{DATA} for
\texttt{.STEP} and \texttt{.DC}, it is not possible to sweep nonlinear device
parameters.  This is because changing other nonlinear device parameters would
alter the correct DCOP solution, and the NOISE sweep happens after the DCOP
calculation in the analysis flow.  To sweep a nonlinear device parameter on
a NOISE problem, add a \texttt{.STEP} command to the netlist to provide an
outer parametric sweep around the analysis.

If \texttt{.DATA} is used with \texttt{.NOISE} then the integrals for the total
input noise and total output noise will only be calculated, and sent to stdout,
if the frequencies in the data table are monotonically increasing.

\index{\texttt{.PRINT}}\index{results!print}\index{\texttt{.PRINT}!\texttt{NOISE}}
A \texttt{.PRINT NOISE} must be used to get the results of the NOISE sweep
analysis.  See Section \ref{.PRINT}.

Noise analysis is a relatively new feature to \Xyce{}, so not all noise models
have been supported.

Power calculations (\texttt{P(<device>} and \texttt{W(<device>}) are not supported for any
devices for noise analysis.  Current variables (e.g., \texttt{I(<device>)} are only supported
for devices that have ``branch currents'' that are part of the solution vector. This includes
the V, E, H and L devices.  It also includes the voltage-form of the B device.

\end{Command}



%%%%%%%%%%%%%%%%%%%%%%%%%%%%%%%%%%%%%%%%%%%%%%%%%%%%%%%%%%%%%%%%%%%%%%%%%%%%%%%%
\newpage
\subsection{\texttt{.OP} (Bias Point Analysis)}
% Sandia National Laboratories is a multimission laboratory managed and
% operated by National Technology & Engineering Solutions of Sandia, LLC, a
% wholly owned subsidiary of Honeywell International Inc., for the U.S.
% Department of Energy’s National Nuclear Security Administration under
% contract DE-NA0003525.

% Copyright 2002-2023 National Technology & Engineering Solutions of Sandia,
% LLC (NTESS).


\index{\texttt{.OP}}
\index{analysis!op} \index{operating point analysis}
The .OP command causes detailed information about the bias point to be printed.

\begin{Command}\label{OP_COMMAND}

\format
.OP 

\comments

This type of analysis can be specified by itself, in which case \Xyce{}
will run a nominal operating point.  However, if specified with another
analysis type, no additional operating point will be calculated, as most
analyses require a DC operating point for initialization.

\texttt{.OP} outputs the parameters for all the device models
and all the device instances present in the circuit.  For large
circuits, this can be a very large amount of output, so use with
caution.

If no analysis command is provided, \texttt{.OP} will run a DC Operating
Point calculation (i.e., a DC analysis) with all the voltage sources 
left at their nominal (instance line) values.

The \Xyce{} \texttt{.OP} statement may provide less, or different, output than other simulators.  
For some of the missing quantities, a \Xyce{} \texttt{.PRINT} line can give similar information.  
Nodal voltages are always available  on a \texttt{.PRINT} line.  Device currents for many 
devices are available on a \texttt{.PRINT} line using the lead current notation 
\texttt{(I(devicename))}. Similarly, device power is available on a \texttt{.PRINT} line
via \texttt{P(devicename)} or \texttt{W(devicename)}.  However, these capabilities are not 
supported in all devices.  Table \ref{deviceFeatureSupportTable} shows which devices 
support these lead current and power notations.  Currently, there is no way to print 
out internal capacitances. 

\end{Command}



%%%%%%%%%%%%%%%%%%%%%%%%%%%%%%%%%%%%%%%%%%%%%%%%%%%%%%%%%%%%%%%%%%%%%%%%%%%%%%%%
\newpage
\subsection{\texttt{.OPTIONS} Statements}
% Sandia National Laboratories is a multimission laboratory managed and
% operated by National Technology & Engineering Solutions of Sandia, LLC, a
% wholly owned subsidiary of Honeywell International Inc., for the U.S.
% Department of Energy’s National Nuclear Security Administration under
% contract DE-NA0003525.

% Copyright 2002-2022 National Technology & Engineering Solutions of Sandia,
% LLC (NTESS).


\label{Options_Reference}

\index{\texttt{.OPTIONS}} \index{analysis!options}  \index{analysis!control parameters} \index{solvers!control parameters}

Set various simulation limits, analysis control parameters
and output parameters.  In general, they use the following format:

\begin{Command}
\format
.OPTIONS <pkg> [<name>=<value>]*

\examples
.OPTIONS TIMEINT ABSTOL=1E-8

\arguments

\begin{Arguments}

\argument{pkg}

\begin{basedescript}{
    \desclabelstyle{\multilinelabel}
    \desclabelwidth{1.5in}
    \renewcommand{\makelabel}[1]{\tt #1\hfill}}
  \item[\tt DEVICE]       Device Model
  \item[\tt DIAGNOSTIC]   Diagnostic Simulation Output
  \item[\tt TIMEINT]      Time Integration
  \item[\tt NONLIN]       Nonlinear Solver
  \item[\tt NONLIN-TRAN]  Transient Nonlinear Solver
  \item[\tt NONLIN-HB]    HB Nonlinear Solver
  \item[\tt LOCA]         Continuation/Bifurcation Tracking
  \item[\tt LINSOL]       Linear Solver
  \item[\tt LINSOL-HB]    HB Linear Solver
  \item[\tt LINSOL-AC]    AC Linear Solver
  \item[\tt OUTPUT]       Output
  \item[\tt RESTART]      Restart
  \item[\tt SAMPLES]      Sampling analysis and non-intrusive Polynomial Chaos (PCE)
  \item[\tt EMBEDDEDSAMPLES]  EmbeddedSampling and non-intrusive Polynomial Chaos (PCE) 
  \item[\tt PCES]         Fully intrusive Polynomial Chaos (PCE)
  \item[\tt SENSITIVITY]  Direct and Adjoint sensitivities
  \item[\tt HBINT]        Harmonic Balance (HB)
  \item[\tt DIST]         Distribution
  \item[\tt MEASURE]      Measure
  \item[\tt PARSER]       Parsing
\end{basedescript}

\argument{\vbox{\hbox{name\hfil}\hbox{value}}}
The name of the parameter and the value it will be assigned.

\end{Arguments}

\comments

Exceptions to this format are the \texttt{OUTPUT} and \texttt{RESTART}
options, which use their own format. They are defined under their
respective descriptions.

The designator \texttt{pkg} refers loosely to a {\em module} in the
code.  Thus, the term is used here as identifying a specific module to
be controlled via {\em options} set in the netlist input file.

\end{Command}

\subsubsection{\texttt{.OPTIONS DEVICE} (Device Package Options)}
\index{device!package options}\index{\texttt{.OPTIONS}!\texttt{DEVICE}}

The device package parameters listed in Table~\ref{DevicePKG} outline the options
available for specifying device specific parameters.  Some of these (\texttt{DEFAS,
DEFAD, TNOM} etc.) have the same meaning as they do for
the \texttt{.OPTION} line from Berkeley SPICE (3f5).  Parameters which
apply globally to all device models will be specified here.
Parameters specific to a particular device instance or model are
specified in section~\ref{Analog_Devices}.

% Sandia National Laboratories is a multimission laboratory managed and
% operated by National Technology & Engineering Solutions of Sandia, LLC, a
% wholly owned subsidiary of Honeywell International Inc., for the U.S.
% Department of Energy’s National Nuclear Security Administration under
% contract DE-NA0003525.

% Copyright 2002-2024 National Technology & Engineering Solutions of Sandia,
% LLC (NTESS).


%%
%% Device Selections Table
%%

\begin{OptionTable}{Options for Device Package}
\label{DevicePKG}%
DEFAD & MOS Drain Diffusion Area & 0.0 \\ \hline
DEFAS & MOS Source Diffusion Area & 0.0 \\ \hline
DEFL & MOS Default Channel Length & 1.0E-4 \\ \hline
DEFW & MOS Default Channel Width & 1.0E-4 \\ \hline
DIGINITSTATE & This option controls the behavior of the Digital Latch (DLTCH), D Flip-Flop (DFF),
JK Flip-Flop (JKFF) and T Flip-Flop (TFF) behavioral digital devices during the DC Operating Point (DCOP)
calculations. See ~\ref{U_DEVICE} for more details.  & 3 \\ \hline 
GMIN & Minimum Conductance & 1.0E-12 \\ \hline
MINRES & This is a minimum resistance to be used in place of the default zero value of semiconductor device internal resistances.  It is only used when model specifications (\texttt{.MODEL} cards) leave the parameter at its default value of zero, and is not used if the model explicitly sets the resistance to zero.   & 0.0 \\ \hline
MINCAP & This is a minimum capacitance to be used in place of the default zero value of semiconductor device internal capacitances.  It is only used when model specifications (\texttt{.MODEL} cards) leave the parameter at its default value of zero, and is not used if the model explicitly sets the capacitance to zero.   & 0.0 \\ \hline
TEMP & Temperature & 27.0 $^\circ$C (300.15K) \\ \hline
TNOM & Nominal Temperature & 27.0 $^\circ$C (300.15K) \\ \hline
% TVR: I can't find any use of this parameter in Xyce!
%\debug{\texttt{SCALESRC}} & \debug{Scaling factor for source scaling} & \debug{0.0} \\ \hline
\debug{NUMJAC} & \debug{Numerical Jacobian flag (only use for small problems)}
& \debug{0 (FALSE)} \\ \hline
VOLTLIM & Voltage limiting & 1 (TRUE) \\ \hline
B3SOIVOLTLIM & BSIMSOI3 Voltage limiting.  This flag is similar to VOLTLIM, except that it only 
  applies to the BSIMSOI version 3  (the newer versions of the BSIM SOI do not have voltage limiting).  
  Turning this off will often improve numerical robustness.  Unlike VOLTLIM, turning this off 
  does not disable the initial condition code in the BSIMSOI model.  & 1 (TRUE) \\ \hline
B3SOIGMINSCALING & This flag enables scaling of gmin in the BSIMSOI version 3 device model. When
it is enabled, a scaling factor of 1e-6 is applied to the gmin value. & 1 (TRUE) \\ \hline

\debug{icFac} 
& \debug{This is a multiplicative factor which is applied to right-hand side vector loads of 
.IC initial conditions during the DCOP phase. }
& \debug{10000.0} \\ \hline

MAXTIMESTEP & Maximum time step size & 1.0E+99 \\ \hline

SMOOTHBSRC & This flag enables smooth transitions by adding a RC network to the output of ABM devices    &    0  \\ \hline


RCCONST & This option controls the smoothness of the transitions if the
SMOOTHBSRC flag is enabled. This is done by specifying the RC constant of the
RC network & 1e-9 \\ \hline

\\ \hline
\multicolumn{3}{|c|}{\color{XyceDarkBlue}\em\bfseries MOSFET Homtopy parameters} \\ \hline

VDSSCALEMIN & Scaling factor for Vds    & 0.3      \\ \hline
VGSTCONST   & Initial value for Vgst    & 4.5 Volt \\ \hline
LENGTH0     & Initial value for length  & 5.0e-6   \\ \hline
WIDTH0      & Initial value for width   & 200.0e-6 \\ \hline
TOX0        & Initial value for oxide thickness & 6.0e-8  \\ \hline

\\ \hline
\multicolumn{3}{|c|}{\color{XyceDarkBlue}\em\bfseries Debug output parameters} \\ \hline

\debug{DEBUGLEVEL} & \debug{The higher this number, the more info is output} & \debug{1} \\ \hline
\debug{DEBUGMINTIMESTEP} & \debug{First time-step debug information is output} & \debug{0} \\ \hline
\debug{DEBUGMAXTIMESTEP} & \debug{Last time-step of debug output} & \debug{65536} \\ \hline
\debug{DEBUGMINTIME} & \debug{Same as \texttt{DEBUGMINTIMESTEP} except controlled by time (sec.) instead of step number} & \debug{0.0} \\ \hline
\debug{DEBUGMAXTIME} & \debug{Same as \texttt{DEBUGMAXTIMESTEP} except controlled by time (sec.) instead of step number} & \debug{100.0} \\ \hline
\end{OptionTable}


\subsubsection{\texttt{.OPTIONS DIAGNOSTIC} (Diagnostic Simulation Output)}
\index{diagnostic!options}\index{\texttt{.OPTIONS}!\texttt{DIAGNOSTIC}}

This option enables the output of diagnostic data during a simulation to aid
in debugging circuit problems.  Currently the only diagnostic information 
available is the absolute value of the largest unknown in the solution vector.  
One can use the diagnostic option, \texttt{EXTREMALIMIT}, to set a limit above
which output will be sent to the diagnostic file set by \texttt{DIAGFILENAME}.  
Both the extrema and the circuit node associated with the extrema are 
written to the diagnostic file along with step information (time step in 
transient simulation and step number otherwise). A user can examine the 
extrema found in the solution vector to infer if the circuit simulation 
is performing as expected.

% Sandia National Laboratories is a multimission laboratory managed and
% operated by National Technology & Engineering Solutions of Sandia, LLC, a
% wholly owned subsidiary of Honeywell International Inc., for the U.S.
% Department of Energy’s National Nuclear Security Administration under
% contract DE-NA0003525.

% Copyright 2002-2023 National Technology & Engineering Solutions of Sandia,
% LLC (NTESS).


%%
%% Diagnostic Selections Table
%%

\begin{OptionTable}{Options for Diagnostic Output}
\label{DiagnosticOptions}%
EXTREMA & Output extreme values occurring in the solution vector & true \\ \hline
EXTREMALIMIT & Output the absolute value of extrema that exceed EXTREMALIMIT & 0.0 \\ \hline 
VOLTAGELIMIT & Output voltages that exceed $\pm$VOLTAGELIMIT & 0.0 \\ \hline 
CURRENTLIMIT & Output branch and lead currents that  that exceed $\pm$CURRENTLIMIT & 0.0 \\ \hline 
DISCLIMIT & Output solution variables where the absolute value of the difference 
  between the solution and the transient predictor exceed $\pm$DISCLIMIT & 0.0 \\ \hline 
DIAGFILENAME & Save diagnostic data to a file named by DIAFILENAME & \texttt{[input netlist].dia} \\ \hline

\end{OptionTable}



\subsubsection{\texttt{.OPTIONS TIMEINT} (Time Integration Options)}
\index{solvers!time integration!options}
\index{\texttt{.OPTIONS}!\texttt{TIMEINT}}

The time integration parameters listed in Table~\ref{TimeIntPKG} give the
available options for helping control the time integration algorithms for
transient analysis.

Time integration options are set using the \texttt{.OPTIONS TIMEINT} command.

% Sandia National Laboratories is a multimission laboratory managed and
% operated by National Technology & Engineering Solutions of Sandia, LLC, a
% wholly owned subsidiary of Honeywell International Inc., for the U.S.
% Department of Energy’s National Nuclear Security Administration under
% contract DE-NA0003525.

% Copyright 2002-2023 National Technology & Engineering Solutions of Sandia,
% LLC (NTESS).


%%
%% Time Integration Package Table
%%

\begin{OptionTable}{Options for Time Integration Package.}
\label{TimeIntPKG}
METHOD & Time integration method.  This parameter is only
relevant when running \Xyce{} in transient mode.  Supported methods:
\begin{XyceItemize}
\item trap or 7 (variable order Trapezoid)
\item gear or 8 (Gear method) 
\end{XyceItemize} &
trap or 7 (variable order Trapezoid) \\ \hline
RELTOL\index{\texttt{RELTOL}}  & Relative error tolerance & 
1.0E-03 \\ \hline
ABSTOL\index{\texttt{ABSTOL}}  & Absolute error tolerance & 1.0E-06 \\
\hline
RESTARTSTEPSCALE\index{\texttt{RESTARTSTEPSCALE}}  & 
This parameter is a scalar which determines how small the initial
time step out of a breakpoint should be.  In the current version of the
time integrator, the first step after a breakpoint isn't subjected to
much error analysis, so for very stiff circuits, this step can be 
problematic. & 0.005 \\ \hline
NLNEARCONV\index{\texttt{NLNEARCONV}}  & 
This flag sets if ``soft'' 
failures of the nonlinear solver, when the convergence criteria are almost, 
but not quite, met, should result in a "success" code being returned from
the nonlinear solver to the time integrator.
If this is enabled, it is expected that the error analysis performed by the 
time integrator will be the sole determination of whether or not the time 
step is considered a ``pass'' or a ``fail''.  This is on by default, but
occasionally circuits need tighter convergence criteria.  
& 0 (FALSE)  \\ \hline
NLSMALLUPDATE\index{\texttt{NLSMALLUPDATE}} &   
This flag is another ``soft'' nonlinear solver failure flag.  In
this case, if the flag is set, time steps in which the nonlinear solver 
stalls, and is using updates that are numerically tiny, can be considered
to have converged by the nonlinear solver.  If this flag is set, 
the time integrator is responsible for determining
if a step should be accepted or not.
& 1 (TRUE) \\ \hline

RESETTRANNLS & The nonlinear solver resets its settings for the
transient part of the run to something more efficient (basically a simpler set
of options with smaller numbers for things like max Newton step).  If this is
set to false, this resetting is turned off. Normally should be left as
default. & 1 (TRUE) \\ \hline

MAXORD & This parameter determines the maximum order of integration
that time integrators will attempt.  Setting this option
does not guarantee that the integrator will integrate at this order, it just
sets the maximum order the integrator will attempt.  In order to guarantee a
particular order is used, see the option \texttt{MINORD} below.  & 
2 for variable order Trapezoid and Gear \\ \hline

MINORD & This parameter determines the minimum order of integration
that  time integrators will attempt to maintain.  The integrator will
start at Backward Euler and move up in order as quickly as possible to achive
\texttt{MINORD} and then it will keep the order above this.  If \texttt{MINORD}
is set at 2 and \texttt{MAXORD} is set at 2, then the integrator will move to
second order as quickly as possible and stay there.  & 1 \\ \hline

NEWLTE & This parameter determines the reference value for relative
convergence criterion in the local truncation error based time step control.
The  supported choices
\begin{XyceItemize}
\item 0. The reference value is the current value on each node.
\item 1. The reference value is the maximum of all the signals at the current time.
\item 2. The reference value is the maximum of all the signals over all past time.
\item 3. The reference value is the maximum value on each signal over all past time.
\end{XyceItemize}   & 1 \\ \hline

NEWBPSTEPPING & This flag sets a new time stepping method after a break point. 
Previously, \Xyce{} treats each breakpoint identically to the DCOP point, in which
the intitial time step out of the DCOP is made to be very very small, because
the LTE calculation is unreliable.  As a result, \Xyce{} takes an incredibly small
step out of each breakpoint and then tries to grow the stepsize from there. 
When \texttt{NEWBPSTEPPING} is set, \Xyce{} can take a reasonable
large step out of every non-DCOP breakpoint, and then just relies on the step
control to ensure that the step is small enough.  

Note that the new time stepping method after a break point does not work
well with the old LTE calculation since the old LTE calculation is
conservative and it tends to reject the first time step out of a break
point. We recommend to use newlte if you choose to use the new time
stepping method out of a break point. & 1 (TRUE) \\ \hline

MASKIVARS & This parameter masks out current variables in the local truncation error (LTE) based time step
control. & 0 (FALSE) \\  \hline

ERROPTION & This parameter determines if Local Truncation Error (LTE)
control is turned on or not.  If \texttt{ERROPTION} is  on, then step-size
selection is based on the number of Newton iterations nonlinear solve.  
For Trapezoid and Gear, if the number of nonlinear
iterations is below \texttt{NLMIN} then the step is doubled.  If the number
of nonlinear iterations is above \texttt{NLMAX} then the step is cut by one
eighth.  In between, the step-size is left alone.  Because this option can
lead to very large time-steps, it is very important   to specify an appropriate
\texttt{DELMAX} option.  If the circuit has breakpoints, then the option
\texttt{MINTIMESTEPSBP} can also help to adjust the maximum time-step by
specifying the minimum number of time points between breakpoints. & 0 (Local Truncation Error is used)  \\ \hline

NLMIN &  This parameter determines the lower bound for the desired
number of nonlinear iterations during a Trapezoid time or Gear integration solve with
\texttt{ERROPTION}=1.
& 3  \\ \hline

NLMAX & This parameter determines the upper bound for the desired
number of nonlinear iterations during a Trapezoid time or Gear integration solve with
\texttt{ERROPTION}=1.
& 8  \\ \hline

DELMAX & This parameter determines the maximum time step-size used
with \texttt{ERROPTION}=1.  If a maximum time-step is also specified on the
\texttt{.TRAN} line, then the minimum of that value and \texttt{DELMAX} is
used.
& 1e99 \\ \hline

MINTIMESTEPSBP & This parameter determines the minimum number of
time-steps to use between breakpoints.  This enforces a maximum time-step
between breakpoints equal to the distance between the last breakpoint and the
next breakpoint divided by \texttt{MINTIMESTEPSBP}.
& 10  \\ \hline

TIMESTEPSREVERSAL & This parameter determines whether time-steps are
rejected based upon the step-size selection strategy in \texttt{ERROPTION}=1.
If it is set to 0, then a step will be accepted with successful nonlinear
solves independent of whether the number of nonlinear iterations is between
\texttt{NLMIN} and \texttt{NLMAX}.  If it is set to 1, then when the number of
nonlinear iterations is above \texttt{NLMAX}, the step will be rejected and the
step-size cut by one eighth and retried.  If \texttt{ERROPTION}=0 (use LTE) then
\texttt{TIMESTEPSREVERSAL}=1 (reject steps) is set.  
& 0 (do not reject steps) \\ \hline

DOUBLEDCOPSTEP \index{PDE Devices!time integration parameters} \index{TCAD Devices!time integration parameters} & 
TCAD devices by default will solve an extra "setup" problem to mitigate
some of the convergence problems that TCAD devices often exhibit.
This extra setup problem solves a nonlinear Poisson equation first to establish 
an initial guess for the full drift-diffusion(DD) problem.  The name of this 
parameter refers to the fact that the code is solving two DC operating point 
steps instead of one.  To solve only the nonlinear Poisson problem, then set 
\texttt{DOUBLEDCOP=nl\_poisson}.  To solve only the drift-diffusion problem 
(skipping the nonlinear Poisson), set \texttt{DOUBLEDCOP=drift\_diffusion}.
To explicitly set the default behavior, then set \texttt{DOUBLEDCOP=nl\_poisson, drift\_diffusion}.
& 
Default value, for TCAD circuits, is a combination: nl\_poisson, drift\_diffusion.
Default value, for non-TCAD circuits is a moot point.  If no TCAD devices are present
in the circuit, then there will not be an extra DCOP solve.
\\ \hline

BREAKPOINTS & This parameter specifies a comma-separated list of timepoints that 
should be used as breakpoints.  They do not replace the existing breakpoints 
already being set internally by Xyce, but instead will add to them.
& N/A  \\ \hline

\debug{BPENABLE}\index{\texttt{BPENABLE}} & \debug{Flag for
  turning on/off breakpoints (1 = ON, 0 = OFF).  It is unlikely anyone would
  ever set this to FALSE, except to help debug the breakpoint capability.}
& \debug{1 (TRUE)} \\ \hline

\debug{EXITTIME}\index{\texttt{EXITTIME}} & \debug{If this is set
  to nonzero, the code will check the simulation time at the end of each step.
  If the total time exceeds the exittime, the code will ungracefully exit.
  This is a debugging option, the point of which is the have the code stop at a
  certain time during a run without affecting the step size control.  If not
  set by the user, it isn't activated.}& \debug{-} \\ \hline

\debug{EXITSTEP}\index{\texttt{EXITSTEP}} &
\debug{Same as \texttt{EXITTIME}, only applied to step number.
The code will exit at the specified step.  If not set by the user,
it isn't activated.} &
\debug{-} \\ \hline

\index{solvers!time integration!options}
\end{OptionTable}



\subsubsection{\texttt{.OPTIONS NONLIN} (Nonlinear Solver Options)}
\index{solvers!nonlinear!options}\index{\texttt{.OPTIONS}!\texttt{NONLIN}}\index{\texttt{.OPTIONS}!\texttt{NONLIN-TRAN}}

The nonlinear solver parameters listed in Table~\ref{NonlinPKG} provide
methods for controlling the nonlinear solver for \index{DC analysis}
\index{analysis!DC} DC, \index{transient analysis}
\index{analysis!transient} transient and harmonic balance. Note that the
nonlinear solver options for DCOP, transient and harmonic balance are
specified in separate options statements, using \texttt{.OPTIONS
 NONLIN}, \texttt{.OPTIONS NONLIN-TRAN} and \texttt{.OPTIONS
 NONLIN-HB}, respectively. The defaults for  \texttt{.OPTIONS
 NONLIN} and \texttt{.OPTIONS NONLIN-TRAN} are specified in the
third and fourth columns of Table~\ref{NonlinPKG}.  The defaults for 
\texttt{.OPTIONS NONLIN-HB} are the same as the default settings given for
\texttt{NONLIN-TRAN} with two exceptions. For \texttt{NONLIN-HB}, the default
for \texttt{ABSTOL} is 1e-9 and the default for \texttt{RHSTOL} is 1e-4.

% Sandia National Laboratories is a multimission laboratory managed and
% operated by National Technology & Engineering Solutions of Sandia, LLC, a
% wholly owned subsidiary of Honeywell International Inc., for the U.S.
% Department of Energy’s National Nuclear Security Administration under
% contract DE-NA0003525.

% Copyright 2002-2023 National Technology & Engineering Solutions of Sandia,
% LLC (NTESS).


%%
%% Nonlinear Solver Options Table
%%
\index{solvers!nonlinear!options}
\begin{OptionTable4}{Options for Nonlinear Solver Package.}
\label{NonlinPKG}
NOX & Use NOX nonlinear solver. & 1 (TRUE) & 0 (FALSE) \\ \hline

NLSTRATEGY & Nonlinear solution strategy.  Supported Strategies:
\begin{XyceItemize}
\item 0 (Newton)
\item 1 (Gradient)
\item 2 (Trust Region)
\end{XyceItemize} &
0 (Newton) &
0 (Newton) \\ \hline

SEARCHMETHOD &
Line-search method used by the nonlinear solver.  Supported
line-search methods:
\begin{XyceItemize}
\item 0 (Full Newton - no line search)
\item 1 (Interval Halving)
\item 2 (Quadratic Interpolation)
\item 3 (Cubic Interpolation)
\item 4 (More'-Thuente)
\end{XyceItemize} &
0 (Full Newton) & 
0 (Full Newton) \\ \hline

CONTINUATION & Enables the use of Homotopy/Continuation algorithms for the nonlinear solve.  Options are:
\begin{XyceItemize}
\item 0 (Standard nonlinear solve)
\item 1 (Natural parameter homotopy.  See LOCA options list)
\item 2/mos (Specialized dual parameter homotopy for MOSFET circuits)
\item 3/gmin (GMIN stepping, similar to that of SPICE)
\item 34/sourcestep (Simultaneous source stepping)
\item 35/sourcestep2 (Sequential source stepping)
\end{XyceItemize} & 
0 (Standard nonlinear solve) &
0 (Standard nonlinear solve) \\ \hline

ABSTOL\index{\texttt{ABSTOL}} & Absolute residual vector tolerance &
1.0E-12 & 1.0E-06 \\ \hline

RELTOL\index{\texttt{RELTOL}} & Relative residual vector tolerance &
1.0E-03 & 1.0E-02 \\ \hline

DELTAXTOL & Weighted nonlinear-solution update norm convergence
tolerance & 1.0 & 0.33 \\ \hline

RHSTOL & Residual convergence tolerance (unweighted 2-norm) &
1.0E-06 & 1.0E-02 \\ \hline

SMALLUPDATETOL & Minimum acceptable norm for weighted nonlinear-solution update & 
1.0E-06 & 1.0E-06 \\ \hline

MAXSTEP\index{\texttt{MAXSTEP}} & Maximum number of Newton steps & 200 & 20 \\ \hline

MAXSEARCHSTEP & Maximum number of line-search steps & 2 & 2 \\ \hline

IN\_FORCING & Inexact Newton-Krylov forcing flag & 
0 (FALSE) &
0 (FALSE) \\ \hline

AZ\_TOL &  Sets the minimum allowed linear solver tolerance. Valid only if \texttt{IN\_FORCING}=1.  & 
1.0E-12 &
1.0E-12 \\ \hline

RECOVERYSTEPTYPE &  If using a line search, this option determines the type of step to take if the line search fails. Supported strategies:
\begin{XyceItemize}
\item 0 (Take the last computed step size in the line search algorithm)
\item 1 (Take a constant step size set by \texttt{RECOVERYSTEP})
\end{XyceItemize} & 
0 &
0 \\ \hline

RECOVERYSTEP & Value of the recovery step if a constant step length is selected & 
1.0 &
1.0 \\ \hline

\debug{DEBUGLEVEL} & \debug{The higher this number, the more info is output} & 
\debug{1} &
\debug{1} 
\\ \hline

\debug{DEBUGMINTIMESTEP} & \debug{First time-step debug information is output} & 
\debug{0} &
\debug{0}
\\ \hline

\debug{DEBUGMAXTIMESTEP} & \debug{Last time-step of debug output} &
\debug{99999999} &
\debug{99999999} \\
\hline

\debug{DEBUGMINTIME} & \debug{Same as \texttt{DEBUGMINTIMESTEP} except controlled by
time (sec.) instead of step number} & 
\debug{0.0}  &
\debug{0.0} 
\\ \hline

\debug{DEBUGMAXTIME} & \debug{Same as \texttt{DEBUGMAXTIMESTEP} except controlled by
time (sec.) instead of step number} & 
\debug{1.0E+99} &
\debug{1.0E+99} 
\\ \hline

\end{OptionTable4}


\index{solvers!nonlinear-transient!options}

\subsubsection{\texttt{.OPTIONS LOCA} (Continuation and Bifurcation Tracking Package Options)}
\index{solvers!continuation!options}
\index{solvers!homotopy!options}

The continuation selections listed in Table~\ref{ContinuationPKG}
provide methods for controlling continuation and bifurcation analysis.
These override the defaults and any that were set simply in the
continuation package.  This option block is only used if the nonlinear
solver or transient nonlinear solver enable continuation through the
\texttt{CONTINUATION} flag.

There are two specialized homotopy methods that can be set in the nonlinear
solver options line.  One is MOSFET-based homotopy, which is specific to MOSFET
circuits.  This is specified using \texttt{continuation=2} or
\texttt{continuation=mos}.  The other is GMIN stepping, which is specified
using \texttt{continuation=3} or \texttt{continuation=gmin}.  For either of
these methods, while it is possible to modify their default LOCA options, it is
generally not necessary to do so. Note that \Xyce{} automatically attempts GMIN
stepping if the inital attempt to find the DC operating point fails. If GMIN
stepping is specified in the netlist, \Xyce{} will not attempt to find a DC
operating point without GMIN stepping. 

LOCA options are set using the \texttt{.OPTIONS LOCA} command.\index{\texttt{.OPTIONS}!\texttt{LOCA}}

% Sandia National Laboratories is a multimission laboratory managed and
% operated by National Technology & Engineering Solutions of Sandia, LLC, a
% wholly owned subsidiary of Honeywell International Inc., for the U.S.
% Department of Energy’s National Nuclear Security Administration under
% contract DE-NA0003525.

% Copyright 2002-2024 National Technology & Engineering Solutions of Sandia,
% LLC (NTESS).


%%
%% LOCA Solver Options Table
%%
\index{solvers!continuation!options}
\begin{OptionTable}{Options for Continuation and Bifurcation Tracking Package.}
\label{ContinuationPKG}
STEPPER & Stepping algorithm to use:
\begin{XyceItemize}
\item 0 (Natural or Zero order continuation)
\item 1 (Arc-length continuation)
\end{XyceItemize} &
0 (Natural) \\ \hline

PREDICTOR &
Predictor algorithm to use:
\begin{XyceItemize}
\item 0 (Tangent)
\item 1 (Secant)
\item 2 (Random)
\item 3 (Constant)
\end{XyceItemize} &
0 (Tangent) \\ \hline

STEPCONTROL &
Algorithm used to adjust the step size between continuation steps:
\begin{XyceItemize}
\item 0 (Constant)
\item 1 (Adaptive)
\end{XyceItemize} &
0 (Constant) \\ \hline

CONPARAM &
Parameter in which to step during a continuation run &
VA:V0 \\ \hline

INITIALVALUE & Starting value of the continuation parameter &
0.0 \\ \hline

MINVALUE & Minimum value of the continuation parameter &
-1.0E20 \\ \hline

MAXVALUE & Maximum value of the continuation parameter &
1.0E20 \\ \hline

BIFPARAM & Parameter to compute during bifurcation tracking runs &
VA:V0 \\ \hline

MAXSTEPS & Maximum number of continuation steps (includes failed steps) & 20 \\ \hline

MAXNLITERS & Maximum number of nonlinear iterations allowed (set this parameter equal to the \texttt{MAXSTEP} parameter in the  \texttt{NONLIN} option block & 20 \\ \hline

INITIALSTEPSIZE & Starting value of the step size & 1.0 \\ \hline

MINSTEPSIZE & Minimum value of the step size & 1.0E20 \\ \hline

MAXSTEPSIZE & Maximum value of the step size & 1.0E-4 \\ \hline

AGGRESSIVENESS & Value between 0.0 and 1.0 that determines how aggressive the step size control algorithm should be when increasing the step size.  0.0 is a constant step size while 1.0 is the most aggressive. & 0.0 \\ \hline

RESIDUALCONDUCTANCE & If set to a nonzero (small) number, this parameter will
force the GMIN stepping algorithm  to stop and declare victory once the
artificial resistors have a conductance that is smaller  than this number.
This should only be used in transient simulations, and \emph{ONLY} if it is
absolutely necessary to get past the DC operating point calculation. It is
almost always better to fix the circuit so that residual conductance is not
necessary. & 0.0 \\ \hline

\end{OptionTable}


\subsubsection{\texttt{.OPTIONS LINSOL} (Linear Solver Options)}
\index{solvers!linear!options}

\Xyce{} uses both sparse direct
solvers\index{solvers!linear!sparse-direct} as well as Krylov iterative
methods\index{solvers!linear!iterative (preconditioned Krylov methods)}
for the solution of the linear equations generated by Newton's method.
For the advanced users, there are a variety of options that can be set
to help improve these solvers.  Transformations of the linear system
have a ``\verb+TR_+'' prefix on the flag.  Many of the options for the
Krylov solvers are simply passed through to the underlying
Trilinos/AztecOO\index{Trilinos}\index{solvers!linear!Trilinos}\index{Aztec}\index{solvers!linear!Aztec}
solution settings and thus have an ``\verb+AZ_+'' prefix on the flag.

Linear solver options are set using the \texttt{.OPTIONS LINSOL}
command.\index{\texttt{.OPTIONS}!\texttt{LINSOL}}

% Sandia National Laboratories is a multimission laboratory managed and
% operated by National Technology & Engineering Solutions of Sandia, LLC, a
% wholly owned subsidiary of Honeywell International Inc., for the U.S.
% Department of Energy’s National Nuclear Security Administration under
% contract DE-NA0003525.

% Copyright 2002-2023 National Technology & Engineering Solutions of Sandia,
% LLC (NTESS).


%%
%% Linear Solver Options Table
%%
\index{solvers!linear!options}
\begin{OptionTable}{Options for Linear Solver Package.}  
\label{LinSolPKG}
type & Determines which linear solver will be used.
\begin{XyceItemize}
\item KLU
\item KSparse
\item SuperLU (optional) 
\item AztecOO
\item Belos
\item ShyLU (optional) 
\end{XyceItemize}
Note that while KLU, KSparse, and SuperLU (optional) are available for parallel execution they will solve the linear system in serial.  Therefore they will be useful for moderate problem sizes but will not scale in memory or performance for large problems  &
KLU (Serial, Parallel $< 10^4$ unknowns) AztecOO, (Parallel, $\geq 10^4$ unknowns) \\ \hline

prec\_type & Determines which preconditioner will be used with an iterative linear solver
\begin{XyceItemize}
\item Ifpack
\end{XyceItemize}
A preconditioner will not be used if a direct solver (KLU, KSparse, SuperLU) is specified.
& Ifpack (Ifpack\_IlukGraph)\\ \hline

use\_aztec\_precond & Triggers use of native AztecOO preconditioners for the iterative linear solves & 0 (FALSE) \\ \hline

use\_ifpack\_factory & Use Ifpack factory to create preconditioner instead of using Ifpack\_IlukGraph & 0 (FALSE) \\ \hline

ifpack\_type & If using the Ifpack factory, {\tt use\_ifpack\_factory=1}, which preconditioner is created
\begin{XyceItemize}
\item Amesos (Additive Schwarz w/ KLU subdomain solve)
\item ILU
\item ILUT
\end{XyceItemize}
& Amesos \\ \hline

ShyLU\_rthresh & Relative dropping threshold for Schur complement preconditioner (ShyLU only) & 1.0E-03 \\ \hline

\multicolumn{3}{|c|}{\color{XyceDarkBlue}\em\bfseries Transformation parameters} \\ \hline\hline

TR\_partition & Perform load-balance partitioning on the linear system & 0 (NONE, Serial) \hspace{3em} 1 (Isorropia, Parallel) \\ \hline

TR\_partition\_type & Type of load-balance partitioning on the linear system & HYPERGRAPH \\ \hline

TR\_singleton\_filter & Triggers use of singleton filter for linear system & 0 (FALSE, Serial) 1 (TRUE, Parallel) \\ \hline

TR\_amd & Triggers use of approximate minimum-degree (AMD) ordering for linear system & 0 (FALSE, Serial) \hspace{3em} 1 (TRUE, Parallel) \\ \hline

TR\_global\_btf & Triggers use of block triangular form (BTF) ordering for linear system, requires \texttt{TR\_amd}=0 and \texttt{TR\_partition}=0 & 0 (FALSE) \\ \hline

TR\_reindex & Reindexes linear system parallel global indices in lexigraphical order, recommended with singleton filter  & 1 (TRUE) \\ \hline

TR\_solvermap & Triggers remapping of column indices for parallel runs, recommended with singleton filter & 1 (TRUE) \\ \hline

\multicolumn{3}{|c|}{\color{XyceDarkBlue}\em\bfseries Iterative linear solver parameters} \\ \hline\hline

adaptive\_solve & Triggers use of AztecOO adaptive solve algorithm for preconditioning of iterative linear solves & 0 (FALSE) \\ \hline

AZ\_max\_iter & Maximum number of iterative solver iterations & 200 \\ \hline

AZ\_precond & AztecOO iterative solver preconditioner flag (used only when \texttt{use\_aztec\_precond}=1) &
\texttt{AZ\_dom\_decomp (14)}
\\ \hline

AZ\_solver & Iterative solver type & \texttt{AZ\_gmres (1)} \\ \hline

AZ\_conv & Convergence type & \texttt{AZ\_r0 (0)} \\ \hline

AZ\_pre\_calc & Type of precalculation & \texttt{AZ\_recalc (1)} \\ \hline

AZ\_keep\_info & Retain calculation info & \texttt{AZ\_true (1)} \\ \hline

AZ\_orthog & Type of orthogonalization & \texttt{AZ\_modified (1)} \\ \hline

AZ\_subdomain\_solve & Subdomain solution for domain decomposition preconditioners & \texttt{AZ\_ilut (9)} \\ \hline

AZ\_ilut\_fill & Approximate allowed fill-in factor for the ILUT preconditioner & 2.0 \\ \hline

AZ\_drop & Specifies drop tolerance used in conjunction with LU or ILUT preconditioners & 1.0E-03 \\ \hline

AZ\_reorder & Reordering type & \texttt{AZ\_none (0)} \\ \hline

AZ\_scaling & Type of scaling & \texttt{AZ\_none (0)} \\ \hline

AZ\_kspace & Maximum size of Krylov subspace & 50 \\ \hline

AZ\_tol & Convergence tolerance & 1.0E-9 \\ \hline

AZ\_output & Output level & AZ\_none (0) \\
& & 50 (if verbose build) \\ \hline

AZ\_diagnostics & Diagnostic information level & AZ\_none (0) \\ \hline

AZ\_overlap & Schwarz overlap level for ILU preconditioners & 0 \\ \hline

AZ\_rthresh & Diagonal shifting relative threshold for ILU preconditioners & 1.0001 \\ \hline

AZ\_athresh & Diagonal shifting absolute threshold for ILU preconditioners & 1.0E-04 \\ \hline

output\_ls & Write out linear system matrix and right-hand-side vector, post-transformation, to Matrix Market file every \# solves
& 0 (no output) \\ \hline

output\_base\_ls & Write out linear system matrix and right-hand-side vector, pre-transformation, to Matrix Market file every \# solves
& 0 (no output) \\ \hline

output\_failed\_ls & Write out linear system matrix and right-hand-side vector to Matrix Market file every \# solves when linear solver fails (only available for direct solvers)
& 0 (no output) \\ \hline
\end{OptionTable}

%%% Local Variables:
%%% mode: latex
%%% End:


\subsubsection{\texttt{.OPTIONS LINSOL-HB} (Linear Solver Options)}

For harmonic balance (HB) analysis, \Xyce{} provides both iterative 
and direct methods for the solution of the steady state.  Only matrix-free techniques
are available for preconditioning the HB Jacobian with an iterative linear solver.
The direct linear solver explicitly forms the HB Jacobian and solves the complex-valued
linear system with the requested solver.
For HB analysis, a reduced number of linear solver options are available,  
and are set using the \texttt{.OPTIONS LINSOL-HB} command.\index{\texttt{.OPTIONS}!\texttt{LINSOL-HB}}

% Sandia National Laboratories is a multimission laboratory managed and
% operated by National Technology & Engineering Solutions of Sandia, LLC, a
% wholly owned subsidiary of Honeywell International Inc., for the U.S.
% Department of Energy’s National Nuclear Security Administration under
% contract DE-NA0003525.

% Copyright 2002-2024 National Technology & Engineering Solutions of Sandia,
% LLC (NTESS).


%%
%% Linear Solver Options Table
%%
\index{solvers!linear!options}
\label{HBLinSolPKG}
\begin{OptionTable}{Options for Linear Solver Package for HB.}  \\ \hline

type & Determines which linear solver will be used
\begin{XyceItemize}
\item AztecOO 
\item Belos
\item Direct
\end{XyceItemize}
&
AztecOO\\ \hline

prec\_type & Determines which preconditioner will be used with an iterative linear solver 
\begin{XyceItemize}
\item block\_jacobi
\end{XyceItemize}
A preconditioner will not be used if {\tt type=Direct} is specified
& block\_jacobi\\ \hline

block\_jacobi\_corrected & Enable one-step correction to the {\tt block\_jacobi} preconditioner.
& 0 (FALSE)\\ \hline
 
direct\_solver & Determines which direct linear solver will be used if {\tt type=Direct}
is specified
\begin{XyceItemize}
\item LAPACK
\end{XyceItemize}
&
LAPACK\\ \hline

AZ\_kspace & Maximum size of Krylov subspace & 50 \\ \hline

AZ\_max\_iter & Maximum number of iterative solver iterations & 200 \\ \hline

AZ\_tol & Convergence tolerance & 1.0E-9 \\ \hline

output\_ls & Write out linear system matrix and right-hand-side vector to Matrix Market file every \# solves
& 0 (no output) \\ \hline 
\end{OptionTable}

%%% Local Variables:
%%% mode: latex
%%% End:


\subsubsection{\texttt{.OPTIONS LINSOL-AC} (Linear Solver Options)}

For AC analysis, \Xyce{} provides both iterative and direct methods for the
solution of the linear equations. For the advanced users, there are a variety 
of options that can be set to help improve these solvers.  Transformations 
of the linear system have a ``\verb+TR_+'' prefix on the flag.  Many of the 
options for the Krylov solvers are simply passed through to the underlying
Trilinos/AztecOO\index{Trilinos}\index{solvers!linear!Trilinos}\index{Aztec}\index{solvers!linear!Aztec}
solution settings and thus have an ``\verb+AZ_+'' prefix on the flag.

Linear solver options are set using the \texttt{.OPTIONS LINSOL-AC}
command.\index{\texttt{.OPTIONS}!\texttt{LINSOL-AC}}  The available options
are the same as those for \texttt{.OPTIONS LINSOL}.


\subsubsection{\texttt{.OPTIONS OUTPUT} (Output Options)}

The \index{results!output options} \index{\texttt{.OPTIONS}!\texttt{OUTPUT}} \verb+.OPTIONS OUTPUT+
command can be used to allow control of the output frequency of data to files specified
by \index{\texttt{.TRAN}} \verb+.PRINT TRAN+ commands.  

One method is to specify output intervals.  The format for this method is:
\begin{vquote}
.OPTIONS OUTPUT INITIAL_INTERVAL=<interval> [<t0> <i0> [<t1> <i1>]* ]
\end{vquote}
where \verb+INITIAL_INTERVAL=<interval>+ specifies the starting interval time
for output and \verb+<tx> <ix>+ specifies later simulation times \verb+<tx>+
where the output interval will change to \verb+<ix>+. The solution is output at the
exact intervals requested; this is done by interpolating the solution
to the requested time points.

Another useful method for controlling the output frequency is to specify discrete output
points.  
\begin{vquote}
.OPTIONS OUTPUT OUTPUTTIMEPOINTS=<t0>,<t1>,* 
\end{vquote}
If this option is used, then only the specified time points will appear in the output file.
No other points will be output, so files using this method can be very sparse.  For this type
of output, the output values are not interpolated.  Instead, the specified output points are 
set as breakpoints in the time integrator, so the output values are computed directly.

In addition to controlling the frequency of output, it is also possible to use
output options to suppress the header from standard format output files, and the footer
from both standard and tecplot format output files.
\begin{vquote}
.OPTIONS OUTPUT PRINTHEADER=<boolean> PRINTFOOTER=<boolean>
\end{vquote}
where setting the \texttt{PRINTHEADER} variable to ``false'' will suppress the header and 
\texttt{PRINTFOOTER} variable to ``false'' will suppress the footer.  The \texttt{PRINTHEADER}
option is only applicable to \texttt{.PRINT <analysis> FORMAT=<STD|GNUPLOT|SPLOT>} files.
The \texttt{PRINTFOOTER} option is only applicable to
\texttt{.PRINT <analysis> FORMAT=<STD|GNUPLOT|SPLOT|TECPLOT>} files.

It is possible to add a \texttt{STEPNUM} column as the first column in the output file.
\begin{vquote}
.OPTIONS OUTPUT ADD\_STEPNUM\_COL=<boolean>
\end{vquote}
where setting the \texttt{ADD\_STEPNUM\_COL} variable to ``true'' will add the
\texttt{STEPNUM} column.  The default is ``false''. This option is applicable to
\texttt{FORMAT=<STD|NOINDEX|GNUPLOT|SPLOT>} for any \texttt{.PRINT} line that
supports \texttt{FORMAT=STD} output.

The default \Xyce{} output for phase operators, such as \texttt{VP()}, \texttt{IP()},
\texttt{SP()}, \texttt{YP()} and \texttt{ZP()}, is in degrees.  For compatibility with
other simulators like Spice3f5 and ngspice, it is possible to change that operator
output to use radians instead:
\begin{vquote}
.OPTIONS OUTPUT PHASE\_OUTPUT\_RADIANS=<boolean>
\end{vquote}
The default value for this option is FALSE.  If set to TRUE then the phase output
will be in radians instead of degrees.  This option also applies to the format
for AC sensitivity output.  It does not affect the output from a \texttt{.FOUR}
analysis or a texttt{.FOUR} measure though.  Those two outputs are always in degrees.

\subsubsection{\texttt{.OPTIONS RESTART} (Checkpointing Options)}

The \index{restart} \index{\texttt{.OPTIONS}!\texttt{RESTART}} \verb+.OPTIONS RESTART+ command is
used to control all \index{checkpoint} checkpoint output and restarting.

The checkpointing form of the \texttt{.OPTIONS RESTART} command takes the following format:
\Format{\par\tt .OPTIONS RESTART [PACK=<0|1>] JOB=<job prefix> \linebreak
+ [INITIAL\_INTERVAL=<initial interval time> [<t0> <i0> [<t1> <i1>]* ]]}

\texttt{PACK=<0|1>} indicates whether the restart data will be byte packed
or not.  Parallel restarts must always be packed while Windows/MingW
runs are always not packed.  Otherwise, data will be packed by default unless
explicitly specified.
\texttt{JOB=<job prefix>} identifies the prefix for restart files.  The
actual restart files will be the job name with the current simulation time
appended (e.g. \texttt{name1e-05} for \texttt{JOB=name} and simulation time
1e-05 seconds).  Furthermore, \texttt{INITIAL\_INTERVAL=<initial interval
  time>} identifies the initial interval time used for restart output.  The
\texttt{<tx> <ix>} intervals identify times \texttt{<tx>} at which the output
interval \texttt{(ix)} should change.  This functionality is identical to
that described for the \texttt{.OPTIONS OUTPUT} command.

\paragraph{Examples}

To generate checkpoints at every time step (default):

\Example{\texttt{.OPTIONS RESTART JOB=checkpt}}

To generate checkpoints every 0.1 $\mu s$:

\Example{\texttt{.OPTIONS RESTART JOB=checkpt INITIAL\_INTERVAL=0.1us}}

To generate unpacked checkpoints every 0.1 $\mu s$:

\Example{\texttt{.OPTIONS RESTART PACK=0 JOB=checkpt INITIAL\_INTERVAL=0.1us}}

To specify an initial interval of 0.1 $\mu s$, at 1 $\mu s$ change to interval
of 0.5 $\mu s$, and at 10 $\mu s$ change to interval of 0.1 $\mu s$:

\Example{\par\texttt{.OPTIONS RESTART JOB=checkpt INITIAL\_INTERVAL=0.1us 1.0us\linebreak + 0.5us 10us 0.1us}}

\subsubsection{\texttt{.OPTIONS RESTART} (Restarting Options)}

To restart from an existing restart file\index{restart!file}, specify the file
by either \texttt{FILE=<restart file name>} to explicitly use a restart file or
by \texttt{JOB=<job name> START\_TIME=<specified name>} to specify a file
prefix and a specified time.  The time must exactly match an output file time
for the simulator to correctly identify the correct file.  To continue
generating restart output files, \texttt{INITIAL\_INTERVAL=<interval>} and
following intervals can be appended to the command in the same format as
described above.  New restart files will be packed according to the previous
restart file read in.  

The restarting form of the \texttt{.OPTIONS RESTART} command takes the following format:

\Format{\par\tt .OPTIONS RESTART FILE=<restart file name>|JOB=<job name> START\_TIME=<time> \linebreak
    + [ INITIAL\_INTERVAL=<interval> [<t0> <i0> [<t1> <i1>]* ]]}

\paragraph{Examples}

Example restarting from checkpoint file at 0.133 $\mu s$:
\Example{\texttt{.OPTIONS RESTART JOB=checkpt START\_TIME=0.133us}}

To restart from checkpoint file at 0.133 $\mu s$:
\Example{\texttt{.OPTIONS RESTART FILE=checkpt0.000000133}}

Restarting from 0.133 $\mu s$ and continue checkpointing at 0.1 $\mu s$
intervals:
\Example{\par\texttt{
    .OPTIONS RESTART FILE=checkpt0.000000133 JOB=checkpt\_again\linebreak
    + INITIAL\_INTERVAL=0.1us
  }}

\subsubsection{\texttt{.OPTIONS RESTART}: special notes for use with two-level-Newton}
\index{restart!two-level}

Large parallel problems which involve power supply parasitics often
require a two-level solve, in which different parts of the problem
are handled separately.  In most respects, restarting a two-level
simulation is similar to restarting a conventional simulation.
However, there are a few differences:

\begin{XyceItemize}
\item When running with a two-level algorithm, \Xyce{} requires (at least) two
different input files.  In order to do a restart of a two-level \Xyce{}
simulation, it is necessary to have an \texttt{.OPTIONS RESTART} statement
in each file.

\item It is necessary for the statements to be consistent.  For example,
the output times must be exactly the same, meaning the initial intervals
must be exactly the same.

\item \Xyce{} will \emph{not} check to make sure that the restart
options used in different files match, so it is up to the user to ensure
matching options.

\item Finally, as each netlist that is part of a two-level solve will have its
own \texttt{.OPTIONS RESTART} statement, that means that each netlist
will generate and/or use its own set of restart files.  As a result,
the restart file name used by each netlist must be unique.
\end{XyceItemize}

\subsubsection{\texttt{.OPTIONS SAMPLES} (Sampling options)}
\index{solvers!sampling!options}

The sampling selections listed in Table~\ref{SamplesPKG}
provide methods for controlling Monte Carlo and Latin Hypercube Sampling methods.

SAMPLES options are set using the \texttt{.OPTIONS SAMPLES} command.\index{\texttt{.OPTIONS}!\texttt{SAMPLES}} 
They are only used if the netlist also includes a \texttt{.SAMPLING} statement. 

% Sandia National Laboratories is a multimission laboratory managed and
% operated by National Technology & Engineering Solutions of Sandia, LLC, a
% wholly owned subsidiary of Honeywell International Inc., for the U.S.
% Department of Energy’s National Nuclear Security Administration under
% contract DE-NA0003525.

% Copyright 2002-2023 National Technology & Engineering Solutions of Sandia,
% LLC (NTESS).


%%
%% Sampling Options Table
%%
\index{solvers!sampling!options}
\begin{OptionTable}{Options for Sampling Package.} \label{SamplesPKG}
NUMSAMPLES   & Total number of samples & 0 \\ \hline
SAMPLE\_TYPE & Sampling type (MC or LHS) & MC \\ \hline
OUTPUTS      & Comma separated list of outputs (anything that would be a valid \texttt{.RESULT} output command) & -- \\ \hline
MEASURES     & Comma separated list of measure names (must refer to \texttt{.MEASURE} commands in the netlist) & -- \\ \hline
COVMATRIX    & Covariance matrix specified in row major form as comma-separated double precision numbers. & -- \\ \hline
SEED         & Random seed & See footnote.\footnotemark[1] \\ \hline
OUTPUT\_SAMPLE\_STATS   &  Compute and outputs statistics for specified outputs and/or measures. & -- \\ \hline
REGRESSION\_PCE    & Enable regression based PCE.  If this is enabled, the randomly sampled points will be used to produce a PCE approximation using regression methodss.     & -- \\ \hline
PROJECTION\_PCE    & Enable projection based PCE (quadrature).  If this is enabled, a PCE approximation will be created using quadrature methods.  The NUMSAMPLES parameter will be ignored, and the samples will be the quadrature points used by projection PCE.     & -- \\ \hline
RESAMPLE    & Once the PCE coefficients are obtained, perform sampling on the PCE approximation & -- \\ \hline
OUTPUT\_PCE\_COEFFS    & Output the PCE coefficients & -- \\ \hline
SPARSE\_GRID    &  Use sparse grid methods if using projection PCE.    & -- \\ \hline
STDOUTPUT    &  Send sampling and PCE output to the terminal   & -- \\ \hline
\end{OptionTable}

\footnotetext[1]{The seed can also be set using command line option, -randseed.   The command line seed will override the netlist seed value.  If the seed is not set in either the netlist or on the command line, then Xyce generates a seed internally.  In all cases, \Xyce{} will output text to the console indicating what seed is being used.}  



\subsubsection{\texttt{.OPTIONS EMBEDDEDSAMPLES} (Embedded Sampling options)}
\index{solvers!embeddedsampling!options}

The sampling selections listed in Table~\ref{EmbeddedSamplesPKG}
provide methods for controlling Embedded Sampling methods.

EMBEDDEDSAMPLES options are set using the \texttt{.OPTIONS EMBEDDEDSAMPLES}
command.\index{\texttt{.OPTIONS}!\texttt{EMBEDDEDSAMPLES}}  They are only used if the
netlist also includes a \texttt{.EMBEDDEDSAMPLING} statement.

% Sandia National Laboratories is a multimission laboratory managed and
% operated by National Technology & Engineering Solutions of Sandia, LLC, a
% wholly owned subsidiary of Honeywell International Inc., for the U.S.
% Department of Energy’s National Nuclear Security Administration under
% contract DE-NA0003525.

% Copyright 2002-2023 National Technology & Engineering Solutions of Sandia,
% LLC (NTESS).

%%
%% Embedded Sampling Table
%%
\index{solvers!embeddedsampling!options}
\begin{OptionTable}{Options for Embedded Sampling Package.} \label{EmbeddedSamplesPKG}
NUMSAMPLES   & Total number of samples & 0 \\ \hline
SAMPLE\_TYPE & Sampling type (MC or LHS) & MC \\ \hline
OUTPUTS      & Comma separated list of outputs (anything that would be a valid \texttt{.PRINT} output variable) & -- \\ \hline
COVMATRIX    & Covariance matrix specified in row major form as comma-separated double precision numbers. & -- \\ \hline
SEED         & Random seed & See footnote.\footnotemark[1] \\ \hline
OUTPUT\_SAMPLE\_STATS   &  Compute and outputs statistics for specified outputs. & -- \\ \hline
REGRESSION\_PCE    & Enable regression based PCE.  If this is enabled, the randomly sampled points will be used to produce a PCE approximation using regression methods.     & -- \\ \hline
PROJECTION\_PCE    & Enable projection based PCE (quadrature).  If this is enabled, a PCE approximation will be created using quadrature methods.  The NUMSAMPLES parameter will be ignored, and the samples will be the quadrature points used by projection PCE.     & -- \\ \hline
RESAMPLE    & Once the PCE coefficients are obtained, perform sampling on the PCE approximation & -- \\ \hline
OUTPUT\_PCE\_COEFFS    & Output the PCE coefficients & -- \\ \hline
SPARSE\_GRID    &  Use sparse grid methods if using projection PCE.    & -- \\ \hline
STDOUTPUT    &  Send sampling and PCE output to the terminal   & -- \\ \hline
\end{OptionTable}

\footnotetext[1]{The seed can also be set using command line option, -randseed.   The command line seed will override the netlist seed value.  If the seed is not set in either the netlist or on the command line, then Xyce generates a seed internally.  In all cases, \Xyce{} will output text to the console indicating what seed is being used.}



\subsubsection{\texttt{.OPTIONS PCES} (Fully intrusive PCE options)}
\index{solvers!pce!options}
\index{solvers!PCE!options}

The sampling selections listed in Table~\ref{PCEPKG}
provide methods for controlling Embedded Sampling methods.

PCES options are set using the \texttt{.OPTIONS PCES}
command.\index{\texttt{.OPTIONS}!\texttt{PCES}}  They are only used if the
netlist also includes a \texttt{.PCE} statement.

% Sandia National Laboratories is a multimission laboratory managed and
% operated by National Technology & Engineering Solutions of Sandia, LLC, a
% wholly owned subsidiary of Honeywell International Inc., for the U.S.
% Department of Energy’s National Nuclear Security Administration under
% contract DE-NA0003525.

% Copyright 2002-2024 National Technology & Engineering Solutions of Sandia,
% LLC (NTESS).

%%
%% Fully intrusive PCE Table
%%
\index{solvers!pce!options}
\begin{OptionTable}{Options for PCE Package.} \label{PCEPKG}
OUTPUTS      & Comma separated list of outputs (anything that would be a valid \texttt{.PRINT} output variable) & -- \\ \hline
COVMATRIX    & Covariance matrix specified in row major form as comma-separated double precision numbers. & -- \\ \hline
SAMPLE\_TYPE & Sampling type (MC or LHS). This is only used if resampling is enabled. & MC \\ \hline
SEED         & Random seed. This is only used if resampling is enabled. & See footnote.\footnotemark[1] \\ \hline
OUTPUT\_SAMPLE\_STATS   &  Compute and outputs statistics for specified outputs. & -- \\ \hline
RESAMPLE    & Once the PCE coefficients are obtained, perform sampling on the PCE approximation & -- \\ \hline
OUTPUT\_PCE\_COEFFS    & Output the PCE coefficients & -- \\ \hline
SPARSE\_GRID    &  Use sparse grid methods if using projection PCE.    & -- \\ \hline
STDOUTPUT    &  Send sampling and PCE output to the terminal   & -- \\ \hline
\end{OptionTable}

\footnotetext[1]{The seed can also be set using command line option, -randseed.   The command line seed will override the netlist seed value.  If the seed is not set in either the netlist or on the command line, then Xyce generates a seed internally.  In all cases, \Xyce{} will output text to the console indicating what seed is being used.}



\subsubsection{\texttt{.OPTIONS SENSITIVITY} (Direct and Adjoint Sensitivity Options)}
\index{solvers!sensitivty!options}

The sensitivity selections listed in Table~\ref{SensitivityPKG}
provide methods for controlling direct and adjoint sensitivity analysis.

SENSITIVITY options are set using the \texttt{.OPTIONS SENSITIVITY} command.\index{\texttt{.OPTIONS}!\texttt{SENSITIVITY}} 
They are only used if the netlist also includes a \texttt{.SENS} statement. 

% Sandia National Laboratories is a multimission laboratory managed and
% operated by National Technology & Engineering Solutions of Sandia, LLC, a
% wholly owned subsidiary of Honeywell International Inc., for the U.S.
% Department of Energy’s National Nuclear Security Administration under
% contract DE-NA0003525.

% Copyright 2002-2023 National Technology & Engineering Solutions of Sandia,
% LLC (NTESS).


%%
%% Sensitivity Solver Options Table
%%
\index{solvers!sensitivity!options}
\begin{OptionTable}{Options for Sensitivity Package.} \label{SensitivityPKG}
ADJOINT & Flag to enable adjoint sensitivity calculation & false \\ \hline
DIRECT & Flag to enable direct sensitivity calculation & false \\ \hline
OUTPUTSCALED & Flag to enable output of scaled sensitivities & false \\ \hline
OUTPUTUNSCALED & Flag to enable output of unscaled sensitivities & true \\ \hline
STDOUTPUT & Flag to enable output of sensitivies to std output & false \\ \hline
ADJOINTBEGINTIME & Start time for set of time steps over which to compute transient adjoints. & 0.0 \\ \hline
ADJOINTFINALTIME & End time for set of time steps over which to compute transient adjoints. & 1.0e+99 \\ \hline
ADJOINTTIMEPOINTS & List of comma-separated time points at which to compute transient adjoints. & -- \\ \hline
\end{OptionTable}



\subsubsection{\texttt{.OPTIONS HBINT} (Harmonic Balance Options)}
\index{solvers!hb!options}\index{\texttt{.OPTIONS}!\texttt{HBINT}}

The Harmonic Balance parameters listed in Table~\ref{hbPKG} give the available
options for helping control the algorithm for harmonic balance analysis.

Harmonic Balance options are set using the \texttt{.OPTIONS HBINT} command.

% Sandia National Laboratories is a multimission laboratory managed and
% operated by National Technology & Engineering Solutions of Sandia, LLC, a
% wholly owned subsidiary of Honeywell International Inc., for the U.S.
% Department of Energy’s National Nuclear Security Administration under
% contract DE-NA0003525.

% Copyright 2002-2023 National Technology & Engineering Solutions of Sandia,
% LLC (NTESS).


\index{solvers!hb!options}
\begin{OptionTable}{Options for HB.}
\label{hbPKG}

NUMFREQ & Number of harmonics to be calculated for each tone. It must have the same number of entries as .HB
statement & 10\\ \hline

STARTUPPERIODS & Number of periods to integrate through before calculating the initial conditions.  This option is only used when TAHB=1.& 0\\ \hline

SAVEICDATA & Write out the initial conditions to a file. & 0\\ \hline

TAHB &  This flag sets transient assisted HB. When TAHB=0, transient analysis is not performed to get an initial guess. When TAHB=1, it uses transient analysis to get an initial guess. For multi-tone HB simulation, the initial guess is generated by a single tone transient simulation. The first tone following \verb|.HB| is used to determine the period for the transient simulation.
For multi-tone HB simulation, it should be set to the frequency that produces the most nonlinear response 
by the circuit. When tahb = 2, the DC op is used as an initial guess & 1  \\ \hline

VOLTLIM &  This flag sets voltage limiting for HB. During the initial guess calculation, which normally uses transient simulation, the voltage limiting flag is determined by .options device voltlim. During the HB phase, the voltage limiting flag is determined by .options hbint voltlim. & 1 \\ \hline

INTMODMAX & The maximum intermodulation product order used in the spectrum. & 
the largest value in the NUMFREQ list. \\ \hline

NUMTPTS & Number of time points in the output & The total number of frequencies (positive, negative and DC). \\ \hline 

SELECTHARMS & The truncation method used in multi-tone HB to select harmonics. Box, diamond and hybrid truncation methods are     supported &  hybrid  \\ \hline
\end{OptionTable}

%%% Local Variables:
%%% mode: latex
%%% End:


\subsubsection{\texttt{.OPTIONS DIST} (Parallel Distribution Options)}
\index{dist!options}\index{\texttt{.OPTIONS}!\texttt{DIST}}

The parameters listed in Table~\ref{distPKG} give the available
options for controlling the parallel distribution used in \Xyce{}.
There are three choices for distribution strategy.

The default distribution strategy is ``first-come, first-served''
(\texttt{STRATEGY=0}), which divides the devices found in the netlist
into equal sized groups (in the order they are parsed) and distributes
a group to each processor.  This does not take into account the
connectivity of the circuit or balance device model computation, and
therefore can exhibit parallel imbalance for post-layout circuits that
have a substantial portion of parasitic devices.

The ``flat round-robin'' strategy (\texttt{STRATEGY=1}) will generate
the same distribution as the default strategy, but every parallel
processor will participate in reading its portion of the netlist.
This strategy provides a more scalable setup than the default
strategy, but can only be applied to flattened (non-hierarchical)
netlists.

The ``device balanced'' strategy (\texttt{STRATEGY=2}) will evenly
divide each of the device types over the number of parallel
processors, so each processor will have a balanced number of each
model type.  This allieviates the parallel imbalance in the device
model computation that can be experienced with post-layout circuits.
However, it does not take into account the circuit connectivity, so
the communication will not be minimized by this strategy.

% Sandia National Laboratories is a multimission laboratory managed and
% operated by National Technology & Engineering Solutions of Sandia, LLC, a
% wholly owned subsidiary of Honeywell International Inc., for the U.S.
% Department of Energy’s National Nuclear Security Administration under
% contract DE-NA0003525.

% Copyright 2002-2024 National Technology & Engineering Solutions of Sandia,
% LLC (NTESS).


\index{dist!options}
\begin{OptionTable}{Options for Parallel Distribution.}
\label{distPKG}

STRATEGY & Parallel device distribution strategy
\begin{XyceItemize}
\item 0 (First-Come, First-Served)
\item 1 (Flat Round-Robin)
\item 2 (Device Balanced)
\end{XyceItemize}
& 0 \\ \hline
\end{OptionTable}

%%% Local Variables:
%%% mode: latex
%%% End:


\subsubsection{\texttt{.OPTIONS FFT} (FFT Options)}
\index{fft!options}\index{\texttt{.OPTIONS}!\texttt{FFT}}
The parameters listed in Table~\ref{fftPKG} give the available
options for controlling all of the \texttt{.FFT} statements in
a given \Xyce{} netlist.

If \texttt{FFT\_ACCURATE} is set to 1 (true), which is the default, then
\Xyce{} will insert breakpoints at the sample times requested by the
collection of \texttt{.FFT} lines in the netlist.  This has been found
to improve the accuracy of the \texttt{.FFT} analyses, at the possible
expense of simulation speed.  If \texttt{FFT\_ACCURATE} is set to
0 (false), then interpolation is used to determine the output variable
values at the specified sample times.  If the \texttt{-remeasure} command
line option is used to recalculate the \texttt{.MEASURE FFT} and/or
\texttt{.FFT} statements for a \texttt{.TRAN} analysis, then
\texttt{FFT\_ACCURATE} is set to 0 during the re-measure operation.  Finally,
if \texttt{.OPTIONS OUTPUT INITIAL\_INTERVAL} is used in the netlist
then \texttt{.OPTIONS FFT FFT\_ACCURATE} will also be set to 0.

If \texttt{FFTOUT} is set to 1 then additional metrics are output to both stdout
and the \verb+<netlistName>.fft0+ file for each \texttt{.FFT} line.  In
addition a sorted list of the 30 largest harmonics is output to stdout.
Those additional metrics are as follows, where Section \ref{FFT_metrics}
provides detailed definitions for these metrics:

\begin{itemize}
  \item Effective Number of Bits (ENOB)
  \item Spurious Free Dynamic Range (SFDR)
  \item Signal to Noise Ratio (SNR)
  \item Signal to Noise-and-Distortion Ratio (SNDR)
  \item Total Harmonic Distorion (THD)
\end{itemize}

The setting for \texttt{FFT\_MODE} is used to control whether the \Xyce{} FFT
processing and output are more compatible with HSPICE (0) or Spectre (1).
This setting affects the format of the window functions, the conversion from
two-sided to one-sided results, and whether the default output for the
magnitude values is normalized, or not.  Section~\ref{FFT_MODE} gives
more details.

% Sandia National Laboratories is a multimission laboratory managed and
% operated by National Technology & Engineering Solutions of Sandia, LLC, a
% wholly owned subsidiary of Honeywell International Inc., for the U.S.
% Department of Energy’s National Nuclear Security Administration under
% contract DE-NA0003525.

% Copyright 2002-2023 National Technology & Engineering Solutions of Sandia,
% LLC (NTESS).

\index{fft!options}
\begin{OptionTable}{Options for FFT.}
\label{fftPKG}

FFT\_ACCURATE & Insert breakpoints at the sample times requested by the
collection of .FFT lines in the netlist & 1 (true) \\ \hline

FFTOUT & Output additional metrics (ENOB, SFDR, SNR, SNDR and THD) and
a sorted list of the 30 largest harmonics for each .FFT line & 0 (false) \\ \hline

FFT\_MODE & Controls whether the FFT calculations and output format are
more compatible with HSPICE (0) or Spectre (1) & 0 \\ \hline
\end{OptionTable}

%%% Local Variables:
%%% mode: latex
%%% End:


\subsubsection{\texttt{.OPTIONS MEASURE} (Measure Options)}
\index{measure!options}\index{\texttt{.OPTIONS}!\texttt{MEASURE}}
The parameters listed in Table~\ref{measurePKG} give the available
options for controlling all of the \texttt{.MEASURE} statements in
a given \Xyce{} netlist.  The \texttt{MEASDGT}, \texttt{MEASFAIL}
and \texttt{MEASOUT} options are included for HSPICE compatibility.

If given in the netlist, the setting for \texttt{MEASOUT} controls whether 
the \texttt{.mt\#} (or \texttt{.ms\#} or \texttt{.ma\#}) files are made (1) or not (0). 
The \texttt{MEASOUT} setting takes precedence over the \texttt{MEASPRINT} setting 
(which is a Xyce-specific option) if both are given in the netlist.
See Section \ref{Measure_Suppressing_Measure_Output} for more details then on 
how the \texttt{MEASPRINT} option interacts with the individual 
\texttt {.MEASURE} statements and the \texttt{-remeasure} command 
line option.

If given in the netlist, the setting for the \texttt{MEASDGT} overrides the 
\texttt{PRECISION} qualifiers given on individual \texttt{.MEASURE} lines. 
The default value for the \texttt{MEASDGT} option is different from in HSPICE.

The \Xyce{} behavior for failed measures can be controlled via the \texttt{MEASFAIL}
and \texttt{DEFAULT\_VAL} options, as well as with the \texttt{DEFAULT\_VAL} 
qualifiers on individual \texttt{.MEASURE} lines.  The order of precedence is
the \texttt{DEFAULT\_VAL} option and then the \texttt{DEFAULT\_VAL} qualifier on
individual \texttt{.MEASURE} lines.  If \texttt{MEASFAIL=0} then \Xyce{} outputs
the default value in the \texttt{.mt\#} ( or \texttt{.ms\#} or \texttt{.ma\#})
files for a failed measure.  If \texttt{MEASFAIL=1} (or any other non-zero 
value) then \Xyce{} outputs ``FAILED'' in the \texttt{.mt\#} ( or \texttt{.ms\#} or \texttt{.ma\#}) 
files for a failed measure.  If given in the netlist, the setting for the 
\texttt{DEFAULT\_VAL} option overrides the \texttt{DEFAULT\_VAL} qualifiers given 
on individual \texttt{.MEASURE} lines.  The \texttt{DEFAULT\_VAL} option and the 
\texttt{DEFAULT\_VAL} qualifiers can be set to any real number.  For all of these 
cases, \Xyce{} will print ``FAILED'' to the standard output for a failed measure.
As a final note, the \texttt{FOUR} measure is a special case since it produces multiline
output.  Failed \texttt{FOUR} measures will be reported as ``FAILED'' in the
\texttt{.mt\#} ( or \texttt{.ms\#} or \texttt{.ma\#}) files, irrespective of the various
\texttt{MEASFAIL} and \texttt{DEFAULT\_VAL} settings.

The \texttt{USE\_CONT\_FILES} option controls whether each \texttt{AC\_CONT},
\texttt{DC\_CONT}, \texttt{NOISE\_CONT} or \texttt{TRAN\_CONT} mode measure uses
a separate output file for its results, or not. Section~\ref{Measure_CONT_Measurement_Output}
provides more details and an example netlist for this options setting.

For backwards compatibility with previous \Xyce{} versions, \texttt{USE\_LTTM} has
been added.  This option defaults to 0, which uses the new version of the \texttt{TRIG-TARG}
measure; while setting it to 1 will use the old version of the \texttt{TRIG-TARG} measure
for all \texttt{TRIG-TARG} measures in the netlist.  If the \texttt{FRAC\_MAX} qualifier
is used on a \texttt{TRIG-TARG} line then \Xyce{} will automatically default to
\texttt{USE\_LTTM=1} for that particular measure line.  It is anticipated that this
option setting will be removed at some point.

% Sandia National Laboratories is a multimission laboratory managed and
% operated by National Technology & Engineering Solutions of Sandia, LLC, a
% wholly owned subsidiary of Honeywell International Inc., for the U.S.
% Department of Energy’s National Nuclear Security Administration under
% contract DE-NA0003525.

% Copyright 2002-2023 National Technology & Engineering Solutions of Sandia,
% LLC (NTESS).


\index{measure!options}
\begin{OptionTable}{Options for MEASURE.}
\label{measurePKG}

DEFAULT\_VAL & Default value for ``failed measures'' in the .mt\# 
( or .ms\# or .ma\#) files. & -1 \\ \hline
MEASDGT & Precision for all .MEASURE statements.  This value applies to the
output to both the .mt\# ( or .ms\# or .ma\#) files and the standard output. & 6 \\ \hline
MEASFAIL & Specify output format for failed measures & 1 \\ \hline
MEASOUT & Control whether the .mt0 file is made or not & 1 \\ \hline
MEASPRINT & Measure Output
\begin{XyceItemize}
\item ALL (Output measure information to both file(s) and stdout)
\item STDOUT (Output measure information to stdout only)
\item NONE (Suppress all measure output)
\end{XyceItemize}
& ALL \\ \hline
USE\_CONT\_FILES & Specifies whether ``continuous'' mode measures use
separate output files for each such measure & 1 (TRUE) \\ \hline
USE\_LTTM & Use the ``Legacy Trig-Trag Mode``.  This option is included
for backwards compatibility with previous \Xyce{} versions.  It may be
removed in the future & 0 (FALSE).
\end{OptionTable}

%%% Local Variables:
%%% mode: latex
%%% End:


\subsubsection{\texttt{.OPTIONS PARSER} (Parser Options)}
\index{parser!options}\index{\texttt{.OPTIONS}!\texttt{PARSER}}
The parameter listed in Table~\ref{parserPKG} gives the available
option for netlist parsing.

% Sandia National Laboratories is a multimission laboratory managed and
% operated by National Technology & Engineering Solutions of Sandia, LLC, a
% wholly owned subsidiary of Honeywell International Inc., for the U.S.
% Department of Energy’s National Nuclear Security Administration under
% contract DE-NA0003525.

% Copyright 2002-2024 National Technology & Engineering Solutions of Sandia,
% LLC (NTESS).


\index{parser!options}
\begin{OptionTable}{Options for Parsing.}
\label{parserPKG}
\index{model binning}
\index{scale}
MODEL\_BINNING & Enable model binning during netlist parsing.  See 
Section \ref{modelCommand} for more details on how model binning 
works in \Xyce{}. & TRUE \\ \hline
  SCALE & Scale factor for geometric parameters such as MOSFET length and width.  This can also be specified as \texttt{.option scale} (singular \texttt{.OPTION} and omitting the keyword \texttt{PARSER}) for compatibility with other simulators. See section~\ref{modelCommand} for an example usage.  & 1.0 \\ \hline
\end{OptionTable}

%%% Local Variables:
%%% mode: latex
%%% End:





%%%%%%%%%%%%%%%%%%%%%%%%%%%%%%%%%%%%%%%%%%%%%%%%%%%%%%%%%%%%%%%%%%%%%%%%%%%%%%%%
\newpage
\subsection{\texttt{.PARAM} (Parameter)}
\index{\texttt{.PARAM}}
% Sandia National Laboratories is a multimission laboratory managed and
% operated by National Technology & Engineering Solutions of Sandia, LLC, a
% wholly owned subsidiary of Honeywell International Inc., for the U.S.
% Department of Energy’s National Nuclear Security Administration under
% contract DE-NA0003525.

% Copyright 2002-2024 National Technology & Engineering Solutions of Sandia,
% LLC (NTESS).


%%
%% Function Table
%%

User defined parameter or function that can be used in expressions throughout the netlist.

\begin{Command}

\format
.PARAM <name>[(arg*)]=<value> [[,]<name>[(arg*)]=<value>]* 

\examples
\begin{alltt}
.PARAM A_Param=1K
.PARAM B_Param=\{A_Param*PI\}
.PARAM FTEST(X)='2*X' B=12.0 FTEST2(X)='3*X'
.PARAM SUM(A,B,C)=\{A+B+C\}
.PARAM A=1, B=2
.PARAM C=AGAUSS(1,0.1,0.1)
\end{alltt}

\arguments

\begin{Arguments}

\argument{name}

Parameter name.  Parameters may be redefined.  
If the same parameter name is used on multiple parameters, \Xyce{} by default will use the last parameter of that name.  By default, no warning will be emitted.
To change this behavior, one can use the \texttt{-redefined\_param} command line option, described in section~\ref{cmd_line_arg_list}.

\argument{arg}

Optional arguments to parameter that is a function.  \texttt{.PARAM} arguments cannot be node names.
The number of arguments in the use of a function must agree with the number 
in the definition. Parameters, TIME, FREQ, and other functions are allowed in 
the body of function definitions.  \index{constants (\texttt{EXP},\texttt{PI})}
Two constants \texttt{EXP} and \texttt{PI} cannot
be used a argument names.  These constants are equal to $e$ and $\pi$, respectively,
and cannot be redefined.

\argument{value}

The value may be a number or an expression.  If it is an expression, it can be surrounded 
by curly braces (\{ \}), single quotes ('), or without delimiters.  Curly braces were originally 
required by the \Xyce{} parser, so if one encounters parsing difficulties, consider 
surrouding expression values with them.

%\medskip
\end{Arguments}

\comments
Parameters defined using \texttt{.PARAM} only have a few restrictions on their 
  usage.  In earlier versions of \Xyce{} they were handled as constants that 
  were evaluated during parsing.  This is no longer the case, and parameters 
  can now have their values change throughout the calculation.   A \texttt{.PARAM} 
  defined in the top level netlist is equivalent to 
  a \texttt{.GLOBAL\_PARAM}, and they can be combined as needed.
Thus, you may use parameters defined by \texttt{.PARAM} in expressions used to
define global parameters, and you may also use global parameters in
\texttt{.PARAM} definitions.  

It is legal for parameters to depend on special variables such as 
TIME, FREQ, TEMP and VT variables.  However, it is not legal for parameters 
to depend on solution variables such as voltage nodes or independent source currents.

Parameters defined using \texttt{.PARAM} can be modified directly by various analyses, such as \texttt{.DC},   
\texttt{.STEP}, \texttt{.SAMPLING} and \texttt{.EMBEDDEDSAMPLING}, subject to scoping rules.  
This will not work, however, for parameters defined as functions.

Multiple parameters (conventional or function) can be specified on the same netlist line.  
Parameters on the same line can optionally be separated using commas.  If \Xyce{} has difficulty 
parsing a multi-parameter \texttt{.PARAM} line, consider adding commas if they are not already present.


To load an external data file with time voltage pairs of data on each 
line into a global parameter, use this syntax:

\texttt{.GLOBAL\_PARAM extdata = \{tablefile("filename")\}}

or

\texttt{.GLOBAL\_PARAM extdata = \{table("filename")\}}

where \texttt{filename} would be the name of the file to load.  
Other interpolators that can read in a data table from a file 
include \texttt{fasttable},\texttt{spline}, \texttt{akima}, \texttt{cubic}, 
\texttt{wodicka} and \texttt{bli}.  See \ref{ExpressionDocumentation} 
for further information.  

There are several reserved words that may not be used as names for parameters.  These reserved words are:
\begin{XyceItemize}
\item \verb+Time+
\item \verb+Freq+ 
\item \verb+Hertz+ 
\item \verb+Vt+
\item \verb+Temp+
\item \verb+Temper+
\item \verb+GMIN+
\end{XyceItemize}

\index{\texttt{.PARAM}!subcircuit scoping}The scoping rules for parameters are:
\begin{XyceItemize}
\item If a \texttt{.PARAM}, statement is included in the main circuit 
netlist, then it is accessible from the main circuit and all subcircuits. 
\item \texttt{.PARAM} statements defined within a subcircuit are scoped 
to that subciruit definition.  So, their parameters are only accessible within 
that subcircuit definition, as well as within ``nested subcircuits'' also 
defined within that subcircuit definition.
\item Parameters defined via \texttt{.PARAM} statements can be modified by the 
  various UQ analysis techniques (\texttt{.STEP}, \texttt{.SAMPLING}, etc) but
  this only works for \texttt{.PARAM} that have been defined in the top level netlist.
  Parameters defined inside of subcircuits cannot be modified directly 
  by these analyes, but they can be modified indirectly via dependence on other 
  globally scoped parameter.
\end{XyceItemize}

Additional illustative examples of scoping are given in the
``Working with Subcircuits and Models'' section of the \Xyce{} Users' 
Guide\UsersGuide. 

\end{Command}



%%%%%%%%%%%%%%%%%%%%%%%%%%%%%%%%%%%%%%%%%%%%%%%%%%%%%%%%%%%%%%%%%%%%%%%%%%%%%%%%
\newpage
\subsection{\texttt{.PCE} (Fully Intrusive Polynomial Chaos Expansion (PCE) Analysis)}\label{.PCE}
% Sandia National Laboratories is a multimission laboratory managed and
% operated by National Technology & Engineering Solutions of Sandia, LLC, a
% wholly owned subsidiary of Honeywell International Inc., for the U.S.
% Department of Energy’s National Nuclear Security Administration under
% contract DE-NA0003525.

% Copyright 2002-2024 National Technology & Engineering Solutions of Sandia,
% LLC (NTESS).

\label{PCE_section}
\index{\texttt{.PCE}}
\index{analysis!pce} 
Calculates a fully intrusive Polynommial Chaos Expansion (PCE) analysis 
(for \verb|.DC| or \verb|.TRAN| only) 
to propagate uncertainty from a set of uncertain inputs to uncertain outputs.
This involves evaluating the circuit at a set of parameter values corresponding to
quadrature points used by the PCE algorithm.
The loop over parameter values happens at the inner-most part of the calculation, 
so all samples are propagated simultaneously.

This fully-intrusive form of PCE is an experimental analysis method.  Non-intrusive 
methods of PCE are also available in \Xyce{} and will usually be a better choice.  
The non-intrusive methods are used in combination with \texttt{.SAMPLING} 
and/or \texttt{.EMBEDDEDSAMPLING}.  Of those other methods, the behavior of \texttt{.PCE} 
most closely resembles that of \texttt{.EMBEDDEDSAMPLING} with \texttt{projection\_pce=true}.
\index{analysis!PCE}
\index{PCE analysis}

\index{analysis!PCE}
\index{PCE analysis}

\begin{Command}
\format
.PCE  \\
+ param=<parameter name>,[parameter name]*  \\
+ type=<parameter type>,[parameter type]*  \\
+ means=<mean>,[mean]*  \\
+ std\_deviations=<standard deviation>,[standard deviation]* \\
+ lower\_bounds=<lower bound>,[lower bound]*  \\
+ upper\_bounds=<upper bound>,[upper bound]* \\
+ alpha=<alpha>,[alpha]*  \\
+ beta=<beta>,[beta]*

\examples
\begin{alltt}
.PCE
+ param=R1
+ type=normal
+ means=3K
+ std\_deviations=1K

.PCE
+ param=R1,R2
+ type=uniform,uniform
+ lower\_bounds=1K,2K
+ upper\_bounds=5K,6K

.PCE useExpr=true

.options PCES 
+ OUTPUTS=\{R1:R\},\{V(1)\}
\end{alltt}

\arguments

\begin{Arguments}

\argument{param}
Names of the parameters to be sampled.  This may be any of the parameters
that are valid for \verb|.STEP|, including device instance, device model,
or global parameters.  If more than one parameter, then specify as a
comma-separated list.

\argument{type}
Distribution type for each parameter.  This may be uniform, normal or gamma.
If more than one parameter, then specify as a comma-separated list.

\argument{means}
If using normal distributions, the mean for each parameter must be specified.
If more than one parameter, then specify as a comma-separated list.

\argument{std\_deviations}
If using normal distributions, the standard deviation for each parameter
must be specified.  If more than one parameter, then specify as a
comma-separated list.

\argument{lower\_bounds}
If using uniform distributions, the lower bound must be specified.
This is optional for normal distributions.  If used with normal
distributions, may alter the mean and standard deviation.
If more than one parameter, then specify as a comma-separated list.

\argument{upper\_bounds}
If using uniform distributions, the upper bound must be specified.
This is optional for normal distributions.  If used with normal
distributions, may alter the mean and standard deviation.
If more than one parameter, then specify as a comma-separated list.

\argument{alpha}
If using gamma distributions, the alpha value for each parameter
must be specified.  If more than one parameter, then specify as a
comma-separated list.

\argument{beta}
If using gamma distributions, the beta value for each parameter
must be specified.  If more than one parameter, then specify as a
comma-separated list.

\argument{useExpr}
If this argument is set to true, then the sampling algorithm will set up random 
  inputs from expression operators such as \verb|AGAUSS| and \verb|AUNIF|.  In 
  this case it will also ignore the list of parameters on the \verb|.PCE| command line.
  For a complete description of expression-based random operators, see the expression
  documentation in section~\ref{ExpressionDocumentation}.

\end{Arguments}

\comments

In addition to the \verb|.PCE| command, this analysis
requires a \verb|.options PCES| command as well.  The
\verb|.PCE| command specifies parameters and their
attributes, either using the \verb|useExpr| option, or with 
comma-separated lists.  The \verb|.options PCES| command specifies
the outputs for which to compute statistics.
To see the details of the \verb|.options PCES| command , see table~\ref{PCEPKG}.

On the \verb|.PCE| command line, if not using \verb|useExpr|, 
parameters and their
attributes must be specified using comma-separated lists. The
comma-separated lists must all be the same length.

The \texttt{.PRINT PCE} command provides output based on the contents
of those print-lines, and also the \texttt{OUTPUT}
arguments on the \texttt{.OPTIONS PCES} line. 

If the \texttt{OUTPUT\_SAMPLE\_STATS} argument on a \texttt{.PRINT PCE} line is
set to ``true'' then the statistics for the \texttt{MEAN}, \texttt{MEANPLUS},
\texttt{MEANMINUS}, \texttt{STDDEV} and \texttt{VARIANCE} will be output for each
variable in the \texttt{OUTPUT} argument.  
If the \texttt{OUTPUT\_ALL\_SAMPLES}
argument on a \texttt{.PRINT PCE} line is set to ``true'' then the values
of all quadrature points, for each variable requested
in the \texttt{OUTPUTS} argument, will be output.

\end{Command}



%%%%%%%%%%%%%%%%%%%%%%%%%%%%%%%%%%%%%%%%%%%%%%%%%%%%%%%%%%%%%%%%%%%%%%%%%%%%%%%%
% Sandia National Laboratories is a multimission laboratory managed and
% operated by National Technology & Engineering Solutions of Sandia, LLC, a
% wholly owned subsidiary of Honeywell International Inc., for the U.S.
% Department of Energy’s National Nuclear Security Administration under
% contract DE-NA0003525.

% Copyright 2002-2023 National Technology & Engineering Solutions of Sandia,
% LLC (NTESS).

\newpage
\subsection{\texttt {.PREPROCESS REPLACEGROUND} (Ground Synonym)}

The purpose of ground synonym replacement is to treat nodes with the names
{\tt GND}, {\tt GND!}, {\tt GROUND} or any capital/lowercase variant thereof
as synonyms for node {\tt 0}.  The general invocation is
\index{\texttt{.PREPROCESS}!{\tt REPLACEGROUND}}

\begin{Command}

\format
.PREPROCESS REPLACEGROUND <bool>

\arguments

\begin{Arguments}

\argument{bool} 

If {\tt TRUE}, \Xyce~will treat all instances of {\tt GND}, {\tt GND!}, {\tt GROUND},
or any capital/lowercase variant thereof, as synonyms for node {\tt 0}.  If
{\tt FALSE}, \Xyce~will consider each term as a separate node.  Only one
{\tt .PREPROCESS REPLACEGROUND} statement is permissible in a given netlist file.

\end{Arguments}
\end{Command}

\newpage
\subsection{\texttt {.PREPROCESS REMOVEUNUSED} (Removal of Unused Components)}
\index{\texttt{.PREPROCESS}!{\tt REMOVEUNUSED}}
If a given netlist file contains devices whose terminals are all connected to
the same node (\emph{e.g.},~\texttt{R2~1~1~1M}), it may be desirable to remove such
components from the netlist before simulation begins.  This is the purpose of
the command

\begin{Command}

\format
.PREPROCESS REMOVEUNUSED [<value>]

\arguments

\begin{Arguments}

\argument{value}

is a list of components separated by commas.  The allowed values are 
\begin{basedescript}{
    \desclabelstyle{\multilinelabel}
    \desclabelwidth{1in}
    \renewcommand{\makelabel}[1]{\tt #1\hfill}}
\item[\tt C] Capacitor
\item[\tt D] Diode
\item[\tt I] Independent Current Source
\item[\tt L] Inductor
\item[\tt M] MOSFET
\item[\tt Q] BJT
\item[\tt R] Resistor
\item[\tt V] Independent Voltage Source
\end{basedescript}

\end{Arguments}

\examples
.PREPROCESS REMOVEUNUSED R,C

\normalfont
\texttt{.PREPROCESS} will attempt to search for all resistors and capacitors in
a given netlist file whose individual device terminals are connected to the
same node and remove these components from the netlist before simulation
ensues.  A list of components that are supported for removal is given above.
Note that for MOSFETS and BJTs, three terminals on each device (the gate,
source, and drain in the case of a MOSFET and the collector, base, and emitter
in the case of a BJT) must be the same for the device to be removed from the
netlist.  As before, only one \texttt{.PREPROCESS REMOVEUNUSED} line is allowed
in a given netlist file.

\end{Command}

\newpage
\subsection{\texttt {.PREPROCESS ADDRESISTORS} (Adding Resistors to Dangling Nodes)}
\index{\texttt{.PREPROCESS}!{\tt ADDRESISTORS}}

We refer to a \emph{dangling node} as a circuit node in one of the following
two scenarios:  either the node is connected to only one device terminal,
and/or the node has no DC path to ground.  If several such nodes exist in a
given netlist file, it may be desirable to automatically append a resistor of a
specified value between the dangling node and ground.  To add resistors to
nodes which are connected to only one device terminal, one may use the command

\begin{Command}

\format
.PREPROCESS ADDRESISTORS ONETERMINAL <value>

\arguments

\begin{Arguments}

\argument{value}

is the value of the resistor to be placed between
nodes with only one device terminal connection and ground.  For instance,
the command

\end{Arguments}

\examples
.PREPROCESS ADDRESISTORS ONETERMINAL 1G

\normalfont
will add resistors of value 1G between ground and nodes with only one
device terminal connection and ground.  The command

\examples

.PREPROCESS ADDRESISTORS NODCPATH <value>

\normalfont

acts similarly, adding resistors of value {\tt <VALUE>} between ground and
all nodes which have no DC path to ground.

The {\tt .PREPROCESS ADDRESISTORS} command is functionally different from
either of the prior {\tt .PREPROCESS} commands in the following way:  while
the other commands augment the netlist file for the current simulation, a
{\tt .PREPROCESS ADDRESISTORS} statement creates an auxiliary netlist file
which explicitly contains a set of resistors that connect dangling nodes to
ground.  If the original netlist file containing a {\tt .PREPROCESS
ADDRESISTORS} statement is called {\tt filename}, invoking \Xyce~on this file
will produce a file {\tt filename\_xyce.cir} which contains the resistors that
connect dangling nodes to ground.  One can then run \Xyce~on this file to
run a simulation in which the dangling nodes are tied to ground.  Note that,
in the original run on the file {\tt filename}, \Xyce~will continue to run a
simulation as usual after producing the file {\tt filename\_xyce.cir}, but this
simulation will {\em not} include the effects of adding resistors between the
dangling nodes and ground.  Refer to the \Xyce{} Users' Guide~\UsersGuide{} for more detailed
examples on the use of {\tt .PREPROCESS ADDRESISTOR} statements.

Note that it is possible for a node to have one device terminal connection
and, simultaneously, have no DC path to ground.  In this case, if both a
{\tt ONETERMINAL} and {\tt NODCPATH} command are invoked, only the resistor
for the {\tt ONETERMINAL} connection is added to the netlist; the
{\tt NODCPATH} connection is omitted.

As before, each netlist file is allowed to contain only one {\tt .PREPROCESS
ADDRESISTORS ONETERMINAL} and one {\tt .PREPROCESS ADDRESISTORS NODCPATH} line
each, or else \Xyce~will exit in error.

\end{Command}


%%%%%%%%%%%%%%%%%%%%%%%%%%%%%%%%%%%%%%%%%%%%%%%%%%%%%%%%%%%%%%%%%%%%%%%%%%%%%%%%
\newpage
\subsection{\texttt{.PRINT} (Print output)}\label{.PRINT}
% Sandia National Laboratories is a multimission laboratory managed and
% operated by National Technology & Engineering Solutions of Sandia, LLC, a
% wholly owned subsidiary of Honeywell International Inc., for the U.S.
% Department of Energy’s National Nuclear Security Administration under
% contract DE-NA0003525.

% Copyright 2002-2024 National Technology & Engineering Solutions of Sandia,
% LLC (NTESS).


Send analysis results to an output file.
\index{\texttt{.PRINT}}
\index{results!print}

\Xyce{} allows multiple output files to be created during the run and
supports several options for each.

\newenvironment{PrintCommandTable}[1]
               {\renewcommand{\arraystretch}{1.2}
                 \newcommand{\category}[1]{\multicolumn{3}{c}{\smallskip\color{XyceDarkBlue}\em\bfseries ##1}}
                 \begin{longtable}{>{\ttfamily\small}m{3in}<{\normalfont}>{\raggedright\small}m{1.75in}>{\raggedright\let\\\tabularnewline\small}m{1.75in}}
                   \caption{#1} \\ \hline
                   \rowcolor{XyceDarkBlue}
                   \color{white}\normalfont\bf Trigger &
                   \color{white}\bf Files &
                   \color{white}\bf Columns/Description \endfirsthead
                   \caption[]{#1} \\ \hline
                   \rowcolor{XyceDarkBlue}
                   \color{white}\normalfont\bf Trigger &
                   \color{white}\bf Files &
                   \color{white}\bf Columns/Description \endhead}
               {\end{longtable}}

\begin{Command}

\format
\begin{alltt}
.PRINT <print type> [FILE=<output filename>]
+ [FORMAT=<STD|NOINDEX|PROBE|TECPLOT|RAW|CSV|GNUPLOT|SPLOT>]
+ [WIDTH=<print field width>]
+ [PRECISION=<floating point output precision>]
+ [FILTER=<absolute value below which a number outputs as 0.0>]
+ [DELIMITER=<TAB|COMMA|SEMICOLON|COLON|"string">] 
+ [TIMESCALEFACTOR=<real scale factor>]
+ [OUTPUT\_SAMPLE\_STATS=<boolean>] [OUTPUT\_ALL\_SAMPLES=<boolean>]
+ <output variable> [output variable]*
\end{alltt}

\examples
\begin{alltt}
.print tran format=tecplot V(1) I(Vsrc) \{V(1)*(I(Vsrc)**2.0)\}

.PRINT TRAN FORMAT=PROBE FILE=foobar.csd V(1) \{abs(V(1))-5.0\}

.PRINT DC FILE=foobar.txt WIDTH=19 PRECISION=15 FILTER=1.0e-10
+ I(VSOURCE5) I(VSOURCE6)

.print tran FORMAT=RAW V(1) I(Vsrc)

R1 1 0 100
X1 1 2 3 MySubcircuit
V1 3 0 1V
.SUBCKT MYSUBCIRCUIT 1 2 3
R1 1 2  100K
R2 2 4  50K
R3 4 3  1K
.ENDS

.PRINT DC V(X1:4) V(2) I(V1)
\end{alltt}


\arguments

\begin{Arguments}

\argument{print type}

A print type is the name of an analysis, one of the analysis specific
print subtypes, or a specialized output command.

% Sandia National Laboratories is a multimission laboratory managed and
% operated by National Technology & Engineering Solutions of Sandia, LLC, a
% wholly owned subsidiary of Honeywell International Inc., for the U.S.
% Department of Energy’s National Nuclear Security Administration under
% contract DE-NA0003525.

% Copyright 2002-2023 National Technology & Engineering Solutions of Sandia,
% LLC (NTESS).

{
\renewcommand{\arraystretch}{1.2}
\begin{tabular}{>{\ttfamily\small}m{1.5in}<{\normalfont}>{\ttfamily\small}m{0.75in}<{\normalfont}m{2.25in}@{}}
  \rowcolor{XyceDarkBlue}
  \color{white}\normalfont\bf Analysis &
  \color{white}\bf Print Type &
  \color{white}\bf Description \\
.AC & AC & Sets default variable list and formats for print subtypes \\ \hline
.AC & AC\_IC & Overrides variable list and format for AC initial conditions \\ \hline
.DC & DC &  \\ \hline
.EMBDEDDEDSAMPLING & ES & \\ \hline
.HB & HB & \\ \hline
.HB & HB\_FD & Overrides variable list and format for HB frequency domain \\ \hline
.HB & HB\_IC & Overrides variable list and format for HB initial conditions \\ \hline
.HB & HB\_STARTUP & Overrides variable list and format for HB start up \\ \hline
.HB & HB\_TD & Overrides variable list and format for HB time domain \\ \hline
.NOISE & Noise & Outputs Noise spectral density curves\\ \hline
.TRAN & TRAN &  \\ \hline
\multicolumn{3}{c}{\smallskip\color{XyceDarkBlue}\em\bfseries Specialized Output Commands} \\
\emph{Homotopy} & HOMOTOPY & Sets variable list and format for homotopy \\ \hline
.SENS & SENS & Sets variable list and format for sensitivity \\ \hline
\end{tabular}
}


A netlist may contain many \texttt{.PRINT} commands, but only commands
with analysis types which are appropriate for the analysis being
performed are processed.  This feature allows you to generate multiple
formats and variable sets in a single analysis run.

For analysis types that generate multiple output files, the print
subtype allows you to specify variables and output parameters for each
of those output files.  If there is no \texttt{.PRINT <subtype>} provided in the
net list, the variables and parameters from the analysis type will be
used.

\argument{FORMAT=<STD|NOINDEX|PROBE|TECPLOT|RAW|CSV|GNUPLOT|SPLOT>}

The output format may be specified using the \texttt{FORMAT} option.
The \texttt{STD} format outputs the data divided up into data columns.
The \texttt{NOINDEX} format is the same as the \texttt{STD} format
except that the index column is omitted. The \texttt{PROBE} format
specifies that the output should be formatted to be compatible with the
\index{PSpice!Probe} PSpice Probe plotting utility.  The
\texttt{TECPLOT} format specifies that the output should be formatted to
be compatible with the Tecplot plotting program.  The \texttt{RAW}
format specifies that the output should comply with the SPICE binary
rawfile format.  The {\bf -a} command line option, in conjunction with
\texttt{FORMAT=RAW} on the \texttt{.PRINT} line, can then be used to output an
ASCII rawfile.  The \texttt{CSV} format specifies that the output file
should be a comma-separated value file with a header indicating the
variables printed in the file.  It is similar to, but not identical to
using \texttt{DELIMITER=COMMA}; the latter will also print a footer that
is not compatible with most software that requires CSV format.  The \texttt{GNUPLOT}
(or \texttt{SPLOT}) format is the same as the \texttt{STD} format except
that if \texttt{.STEP} is used then two (or one) blank lines are inserted
before the data for steps 1,2,3,... where the first step is step 0. The
\texttt{SPLOT} format is useful for when the ``splot'' command in gnuplot
is used to produce 3D perspective plots.

\argument{FILE=<output filename>}

Specifies the name of the file to which the output will be written.
See the ``Results Output and Evaluation Options'' section of the 
\Xyce{} Users' Guide\UsersGuide for more information on how this
feature works for analysis types (e.g., AC and HB) that can produce 
multiple output files. 

\argument{WIDTH=<print field width>}

Controls the output width used in formatting the output.

\argument{PRECISION=<floating point precision>}

Number of floating point digits past the decimal for output data.

\argument{FILTER=<filter floor value>}

Used to specify the absolute value below which output variables will be
printed as \texttt{0.0}.

\argument{DELIMITER=<TAB|COMMA|SEMICOLON|COLON|"string">}

Used to specify an alternate delimiter in the STD, NOINDEX, GNUPLOT or CSV format
output.  In addition to \texttt{TAB}, \texttt{COMMA}, \texttt{SEMICOLON} and 
\texttt{COLON}, the delimiter can also be specified using a quoted 
character string.  When this option is used, that string is the delimiter.

\argument{TIMESCALEFACTOR=<real scale factor>}

Specify a constant scaling factor for time.  Time is normally printed in
units of seconds, but if one would like the units to be milliseconds,
then set TIMESCALEFACTOR=1000.

\argument{OUTPUT\_SAMPLE\_STATS=<boolean>}
Output the sample statistics for an \texttt{EMBEDDEDSAMPLING}
analysis.This argument is only supported for \texttt{.PRINT ES}.
Its default value is true.  Section \ref{EMBEDDEDSAMPLING_section}
has more details.

\argument{OUTPUT\_ALL\_SAMPLES=<boolean>}
Output all of the sample values for an \texttt{EMBEDDEDSAMPLING}
analysis.  This argument is only supported for \texttt{.PRINT ES}.
Its default value is false.  Section \ref{EMBEDDEDSAMPLING_section}
has more details.

\argument{<output variable>}

Following the analysis type and other options is a list of output
variables. There is no upper bound on the number of output variables.
The output is divided up into data columns and output according to any
specified options (see options given above).  Output variables can be
specified as:

\begin{XyceItemize}
\item \texttt{V(<circuit node>)} to output the voltage at \texttt{<circuit node>}
\item \texttt{V(<circuit node>,<circuit node>)} to output the voltage difference between the first \texttt{<circuit node>} and second \texttt{<circuit node>}
\item \texttt{I(<device>)} to output current through a two terminal device
\item \texttt{I<lead abbreviation>(<device>)} to output current into a particular lead of a three or more terminal device (see the Comments, below, for details) \index{lead currents}
\item \texttt{P(<device>)} or \texttt{W(<device>)} to output the power dissipated/generated in a device. 
At this time, not all devices support power calculations. In addition, the results for semiconductor devices
(D, J, M, Q and Z devices) and the lossless transmission device (T device) may differ from other simulators.  
Consult the Features Supported by Xyce Device Models table
in section \ref{Analog_Devices} and the individual sections on each device for more details.  Finally,
power calculations are not supported for any devices for \texttt{.AC} and \texttt{.NOISE} analyses.
\item \texttt{N(<device internal variable>)} to output a specific device's internal variable. (The comments section
below has more detail on this syntax.)
\item \texttt{\{expression\}} to output an expression
\item \texttt{<device>:<parameter>} to output a device parameter
\item \texttt{<model>:<parameter>} to output a model parameter
\end{XyceItemize}
When the analysis type is AC, HB or Noise, additional output variable formats are available:
\begin{XyceItemize}
\item \texttt{VR(<circuit node>)} to output the real component of voltage response at a point in the circuit
\item \texttt{VI(<circuit node>)} to output the imaginary component of voltage response at a point in the circuit
\item \texttt{VM(<circuit node>)} to output the magnitude of voltage response
\item \texttt{VP(<circuit node>)} to output the phase of voltage response in degrees
\item \texttt{VDB(<circuit node>)} to output the magnitude of voltage response in decibels.
\item \texttt{VR(<circuit node>,<circuit node>)} to output the real component of voltage response between two nodes in the circuit
\item \texttt{VI(<circuit node>,<circuit node>)} to output the imaginary component of voltage response between two nodes in the circuit
\item \texttt{VM(<circuit node>,<circuit node>)} to output the magnitude of voltage response between two nodes in the circuit
\item \texttt{VP(<circuit node>,<circuit node>)} to output the phase of voltage response between two nodes in the circuit in degrees
\item \texttt{VDB(<circuit node>,<circuit node>)} to output the magnitude of voltage response between two nodes in the circuit, in decibels
\item \texttt{IR(<device>)} to output the real component of the current through a two terminal device
\item \texttt{II(<device>)} to output the imaginary component of the current through a two terminal device
\item \texttt{IM(<device>)} to output the magnitude of the current through a two terminal device
\item \texttt{IP(<device>)} to output the phase of the current through a two terminal device in degrees
\item \texttt{IDB(<device>)} to output the magnitude of the current through a two terminal device in decibels.
\end{XyceItemize}

In AC and Noise analyses, outputting a voltage node without any of these
optional designators results in output of the real and imaginary parts of the signal.
Note that under AC and Noise analyses, current variables are only supported for
devices that have ``branch currents''that are part of the solution vector. This includes
the V, E, H and L devices.  It also includes the voltage-form of the B device.

Note that when using the variable list for time domain output, usage of
frequency domain functions like \texttt{VDB} can result in -Inf output being
written to the output file.  This is easily solved by defining the time
domain equivalent command to specify the correct output for time domain
data.

Further explanation of the current specifications is given in comments section below.

When a \texttt{.LIN} analysis is done then additional output variable formats are available
via the \texttt{.PRINT AC} line, where \texttt{<index1>} and \texttt{<index2>} must both
be greater than 0 and also both less than or equal to the number of ports in the netlist:
\begin{XyceItemize}
\item \texttt{SR(<index1>,<index2>)} to output the real component of an S-parameter
\item \texttt{SI(<index1>,<index2>)} to output the imaginary component of an S-parameter
\item \texttt{SM(<index1>,<index2>)} to output the magnitude of an S-parameter
\item \texttt{SP(<index1>,<index2>)} to output the phase of an S-parameter in degrees
\item \texttt{SDB(<index1>,<index2>)} to output the magnitude of an S-parameter in decibels.
\item \texttt{YR(<index1>,<index2>)} to output the real component of a Y-parameter
\item \texttt{YI(<index1>,<index2>)} to output the imaginary component of a Y-parameter
\item \texttt{YM(<index1>,<index2>)} to output the magnitude of a Y-parameter
\item \texttt{YP(<index1>,<index2>)} to output the phase of a Y-parameter in degrees
\item \texttt{YDB(<index1>,<index2>)} to output the magnitude of a Y-parameter in decibels.
\item \texttt{ZR(<index1>,<index2>)} to output the real component of a Z-parameter
\item \texttt{ZI(<index1>,<index2>)} to output the imaginary component of a Z-parameter
\item \texttt{ZM(<index1>,<index2>)} to output the magnitude of a Z-parameter
\item \texttt{ZP(<index1>,<index2>)} to output the phase of a Z-parameter in degrees
\item \texttt{ZDB(<index1>,<index2>)} to output the magnitude of a Z-parameter in decibels.
\end{XyceItemize}

When the analysis type is Noise, additional output variable formats are available
via the \texttt{.PRINT NOISE} line for devices that support stationary noise.
\begin{XyceItemize}
\item \texttt{INOISE} to output the input noise contributions
\item \texttt{ONOISE} to output the output noise contributions
\item \texttt{DNI(<deviceName>)} to output the input noise contribution from device <deviceName>
\item \texttt{DNI(<deviceName>,<noiseSource>)} to output the input noise contribution from
                source <noiseSource> for device <deviceName>
\item \texttt{DNO(<deviceName>)} to output the output noise contribution from device <deviceName>
\item \texttt{DNO(<deviceName>,<noiseSource>)} to output the output noise contribution from
                source <noiseSource> for device <deviceName>
\end{XyceItemize}

\end{Arguments}

\comments

\begin{XyceItemize}
\item Currents are positive flowing from node 1 to node 2 for two node
  devices, and currents are positive flowing into a particular lead for
  multi-terminal devices.

\item \texttt{<circuit node>} is simply the name of any node in your
  top-level circuit, or \texttt{<subcircuit name>:<node>} to reference
  nodes that are internal to a subcircuit.

\item \texttt{<device>} is the name of any device in your top-level
  circuit, or \texttt{<subcircuit name>:<device>} to reference devices
  that are internal to a subcircuit.

\item \texttt{<lead abbreviation>} is a single character designator for
  individual leads on a device with three or more leads.  For bipolar
  transistors these are: c (collector), b (base), e (emitter), and s
  (substrate).  For MOSFETs, JFETs, and MESFETs, lead abbreviations
  are: d (drain), g (gate), s (source), and for MOSFETS and JFETs,
  b (bulk).  In addition to these standard leads, SOI
  and CMG MOSFETs have e (bulk) nodes and SOI transistors have
  optional b (body) nodes whose lead currents may also be printed in
  this manner.  For PDE devices, the nodes are numbered according to
  the order they appear, so lead currents are referenced like
  I1(\texttt{<device>}), I2(\texttt{<device>}), etc.  In \Xyce{},
  a \texttt{.PRINT} line request like \texttt{I(Q1)} is a parsing
  error for a multi-terminal device.  Instead, an explicit lead
  current designator like \texttt{IC(Q1)} must be used.

\item The "lead current" method of printing from devices in Xyce is done
  at a low level with special code added to each device; the method is
  therefore only supported in specific devices that have this extra
  code.  So, if \texttt{.PRINT I(Y)} does not work, for a device called
  Y, then you will need to attach an ammeter (zero-volt voltage source)
  in series with that device and print the ammeter's current instead.

\item Lead currents of subcircuit ports are not supported.  However,
  access is provided via specific node names (e.g.,
  \texttt{X1:internalNodeName}) or specific devices (e.g.,
  \texttt{X1:V3}) inside the subcircuit.

\item For STD formatted output, the values of the output variables are
  output as a series of columns (one for each output variable).

\item When the command line option \texttt{-r <raw-file-name>} is used,
  all of the output is diverted to the \emph{raw-file-name} file as a
  concatenation of the plots, and each plot includes all of the
  variables of the circuit instead of the variable list(s) given on the
  \texttt{.PRINT} lines in the netlist.  Using the \texttt{-a} options in
  conjunction with the \texttt{-r} option results in a raw file that is
  output all in ASCII characters.

\item Any output going to the same file from one simulation of \Xyce{}
  results in the concatenation of output.  However, if a simulation is
  re-run then the original output will we over-written.

\item During analysis a number of output files may be generated.  The
  selection of which files are created depends on a variety of factors,
  most obvious of which is the \texttt{.PRINT} command. See
  section~\ref{Netlist_Commands} for more details.

\item Frequency domain values are output as complex values for Raw,
  TecPlot and Probe formats when a complex variable is printed.  For STD
  and CSV formats, the output appears in two columns, the real part
  followed by the imaginary part.  The print variables 
  \texttt{VR, VI, VM, VDB} and \texttt{VP} print the scalar values 
  for the real part, imaginary part,
  magnitude, magnitude in decibels, and phase, respectively.

\item When outputting a device or model parameter, it is usually
  necessary to specify both the device name and the parameter name,
  separated by a colon.  For example, the saturation current of a
  diode model \texttt{DMOD} would be requested
  as \texttt{DMOD:IS}. Section~\ref{Print_Device_Info} on ``Device
  Parameters and Internal Variables'' below gives more details and
  provides an example.

\item The \texttt{N()} syntax is used to access internal solution variables 
that are not normally visible from the netlist, such as voltages on
internal nodes and/or branch currents within a given device.  The
internal solution variables for each \Xyce{} device are not given in
the Reference Guide sections on those devices.  However, if the user
runs \texttt{Xyce -namesfile <filename> <netlist>} then \Xyce{} will
output into the first filename a list of all solution variables
generated by that netlist.  Section~\ref{Print_Device_Info} on
``Device Parameters and Internal Variables'' below gives more details
and provides an example.

\item The \texttt{DNI()} and \texttt{DNO()} syntax is used to print out the
  individual input and output noise contributions for each noise source within
  a device. The user can get a listing of the noise source names for each device in
  a netlist by running \texttt{Xyce -noise\_names\_file <filename> <netlist>}.
  The \Xyce{} Users' Guide\UsersGuide provides an example.

\item If multiple \texttt{.PRINT} lines are given for the same analysis type,
  the same output file name, and the same format, the variable lists
  of all matching \texttt{.PRINT} lines are merged together in the order
  found, and the resulting output is the same as if all the print line
  variable lists had been specified on a single \texttt{.PRINT} line.

\item Attempting to specify multiple \texttt{.PRINT} lines for the same
  analysis type to the same file with different specifications
  of \texttt{FORMAT} is an error.

\item \Xyce{} should emit a warning or error message, similar to 
  ``Could not open filename'' if: 1) the name of the output file is 
  actually a directory name; or 2) the output file is in a 
  subdirectory that does not already exist.  \Xyce{} 
  will not create new subdirectories.

\item The output filename specified with the -r command line option,
  to produce raw file output, should take precedence over
  a {\tt FILE=} parameter specified on a {\tt .PRINT} line.

\item The print statements for some analysis types could result in multiple
  output files.  For example, \texttt{.PRINT HB} will produce both
  frequency- and time-domain output, and place these in different files.
  The default name of these files is the name of the netlist followed by
  a data type suffix, followed by a format-specific extension.

  In \Xyce{}, if a \texttt{FILE} option is given to such a print
  statement, only the ``primary'' data for that analysis type is sent to
  the named file.  The secondary data is still sent to the default file
  name.  This behavior may be subject to change in future releases.

  For analysis types that can produce multiple files,
  special \texttt{.PRINT} lines have been provided to allow the user to
  control the handling of the additional files.  These additional print
  line specifiers are enumerated in the analysis-specific sections
  below.

  If one desires that all outputs for a given analysis type be given
  user-defined file names, it is necessary to use additional print lines
  with additional \texttt{FILE} options.  For example, if one uses
  a \texttt{FILE} option to a \texttt{.PRINT HB} line, only
  frequency-domain data will be sent to the named file.  To redirect the
  time-domain data to a file with a user-defined name, add
  a \texttt{.PRINT HB\_TD} line.  See the individual analysis types
  below for details of what additional print statements are available.
\end{XyceItemize}

\end{Command}

\subsubsection{Print AC Analysis}
\index{\texttt{.PRINT}!AC Analysis}
\index{\texttt{.PRINT}!\texttt{AC}}
% Sandia National Laboratories is a multimission laboratory managed and
% operated by National Technology & Engineering Solutions of Sandia, LLC, a
% wholly owned subsidiary of Honeywell International Inc., for the U.S.
% Department of Energy’s National Nuclear Security Administration under
% contract DE-NA0003525.

% Copyright 2002-2023 National Technology & Engineering Solutions of Sandia,
% LLC (NTESS).

AC Analysis generates two output files, the primary output is in the
frequency domain and the initial conditions output is in the time domain.

Note that when using the \texttt{.PRINT AC} to create the variable list
for DC type output, usage of frequency domain functions like \texttt{VDB} can result
in -Inf output being written to the output file.  This is easily solved
by defining a \texttt{.PRINT AC\_IC} command to specify the correct
output for initial condition data.

Homotopy output can also be generated.

{
\begin{PrintCommandTable}{Print AC Analysis Type}
.PRINT AC & \emph{circuit-file}.FD.prn & INDEX FREQ \\ \hline
.PRINT AC FORMAT=GNUPLOT & \emph{circuit-file}.FD.prn & INDEX FREQ \\ \hline
.PRINT AC FORMAT=SPLOT & \emph{circuit-file}.FD.prn & INDEX FREQ \\ \hline
.PRINT AC FORMAT=NOINDEX & \emph{circuit-file}.FD.prn & FREQ \\ \hline
.PRINT AC FORMAT=CSV & \emph{circuit-file}.FD.csv & FREQ \\ \hline
.PRINT AC FORMAT=RAW & \emph{circuit-file}.raw & FREQ \\ \hline
Xyce -a \newline .PRINT AC FORMAT=RAW & \emph{circuit-file}.raw & FREQ \\ \hline
.PRINT AC FORMAT=TECPLOT & \emph{circuit-file}.FD.dat & FREQ \\ \hline
.PRINT AC FORMAT=PROBE & \emph{circuit-file}.csd & -- \\ \hline


\multicolumn{3}{c}{\smallskip\color{XyceDarkBlue}\em\bfseries Add \texttt{.OP} To Netlist To Enable AC\_IC Output} \\ \hline
.PRINT AC\_IC & \emph{circuit-file}.TD.prn & INDEX TIME \\ \hline
.PRINT AC\_IC FORMAT=GNUPLOT & \emph{circuit-file}.TD.prn & INDEX TIME \\ \hline
.PRINT AC\_IC FORMAT=SPLOT & \emph{circuit-file}.TD.prn & INDEX TIME \\ \hline
.PRINT AC\_IC FORMAT=NOINDEX & \emph{circuit-file}.TD.prn & TIME \\ \hline
.PRINT AC\_IC FORMAT=CSV & \emph{circuit-file}.TD.csv & TIME \\ \hline
.PRINT AC\_IC FORMAT=RAW & \emph{circuit-file}.raw & TIME \\ \hline
Xyce -a \newline .PRINT AC\_IC FORMAT=RAW & \emph{circuit-file}.raw & TIME \\ \hline
.PRINT AC\_IC FORMAT=TECPLOT & \emph{circuit-file}.TD.dat & TIME \\ \hline
.PRINT AC\_IC FORMAT=PROBE & \emph{circuit-file}.TD.csd & -- \\ \hline

\multicolumn{3}{c}{\smallskip\color{XyceDarkBlue}\em\bfseries Command Line Raw Override Output} \\ \hline
Xyce -r raw-file-name & \emph{raw-file-name} & All circuit variables printed \\ \hline
Xyce -r raw-file-name -a & \emph{raw-file-name} & All circuit variables printed \\ \hline

\multicolumn{3}{c}{\smallskip\color{XyceDarkBlue}\em\bfseries Additional Output Available} \\ \hline
.OP & \emph{log file} & Operating point data \\ \hline
.SENS \newline .PRINT SENS & \multicolumn{2}{c}{see~\nameref{Print_Sensitivity}} \\ \hline
.OPTIONS NONLIN CONTINUATION=<method> \newline .PRINT HOMOTOPY & \multicolumn{2}{c}{see~\nameref{Print_Homotopy}} \\ \hline
\end{PrintCommandTable}
}



\subsubsection{Print DC Analysis}
\index{\texttt{.PRINT}!DC Analysis}
\index{\texttt{.PRINT}!\texttt{DC}}
% Sandia National Laboratories is a multimission laboratory managed and
% operated by National Technology & Engineering Solutions of Sandia, LLC, a
% wholly owned subsidiary of Honeywell International Inc., for the U.S.
% Department of Energy’s National Nuclear Security Administration under
% contract DE-NA0003525.

% Copyright 2002-2023 National Technology & Engineering Solutions of Sandia,
% LLC (NTESS).

DC Analysis generates output based on the format specified by the \texttt{.PRINT} command.

Homotopy and sensitivity output can also be generated.

{
\begin{PrintCommandTable}{Print DC Analysis Type}
.PRINT DC & \emph{circuit-file}.prn & INDEX \\ \hline
.PRINT DC FORMAT=GNUPLOT & \emph{circuit-file}.prn & INDEX \\ \hline
.PRINT DC FORMAT=SPLOT & \emph{circuit-file}.prn & INDEX \\ \hline
.PRINT DC FORMAT=NOINDEX & \emph{circuit-file}.prn & -- \\ \hline
.PRINT DC FORMAT=CSV & \emph{circuit-file}.csv & -- \\ \hline
.PRINT DC FORMAT=RAW & \emph{circuit-file}.raw & -- \\ \hline
Xyce -a \newline .PRINT DC FORMAT=RAW & \emph{circuit-file}.raw & -- \\ \hline
.PRINT DC FORMAT=TECPLOT & \emph{circuit-file}.dat & -- \\ \hline
.PRINT DC FORMAT=PROBE & \emph{circuit-file}.csd & -- \\ \hline
\category{Command Line Raw Override Output} \\
Xyce -r raw-file-name & \emph{raw-file-name} & All circuit variables \\ \hline
Xyce -r raw-file-name -a & \emph{raw-file-name} & All circuit variables \\ \hline
\category{Additional Output Available} \\
.OP & \emph{log file} & Operating point data \\ \hline
.SENS \newline .PRINT SENS & \multicolumn{2}{c}{see~\nameref{Print_Sensitivity}} \\ \hline
.OPTIONS NONLIN CONTINUATION=<method> \newline .PRINT HOMOTOPY & \multicolumn{2}{c}{see~\nameref{Print_Homotopy}} \\ \hline
\end{PrintCommandTable}
}


\subsubsection{Print Harmonic Balance Analysis}
\index{\texttt{.PRINT}!Harmonic Balance Analysis}
\index{\texttt{.PRINT}!\texttt{HB}}
% Sandia National Laboratories is a multimission laboratory managed and
% operated by National Technology & Engineering Solutions of Sandia, LLC, a
% wholly owned subsidiary of Honeywell International Inc., for the U.S.
% Department of Energy’s National Nuclear Security Administration under
% contract DE-NA0003525.

% Copyright 2002-2023 National Technology & Engineering Solutions of Sandia,
% LLC (NTESS).

HB Analysis generates one output file in the frequency domain and one in the
time domain based on the format specified by the \texttt{.PRINT}
command.  Additional startup and initial conditions output can be
generated based on \texttt{.OPTIONS} commands.

Note that when using the \texttt{.PRINT HB} to create the variable list
for time domain output, usage of frequency domain functions like \texttt{VDB} can
result in -Inf output being written to the output file.  This is easily
solved by defining a \texttt{.PRINT HB\_TD}, \texttt{.PRINT HB\_IC} and
\texttt{.PRINT HB\_STARTUP} commands to specify the correct output for
the time domain data.

If \texttt{.STEP} is used with HB then the Initial Condition (IC) data will initially
be output to a ``tmp file'' (e.g., \texttt{<netlist-name>.hb\_ic.prn.tmp}).
If that IC data meets the required tolerance then it will be copied to the end
of the \texttt{<netlist-name>.hb\_ic.prn} file, and the tmp file will be deleted.

Homotopy output can also be generated.

{
\begin{PrintCommandTable}{Print HB Analysis Type}
.PRINT HB & \emph{circuit-file}.HB.TD.prn \newline \emph{circuit-file}.HB.FD.prn  \newline \emph{circuit-file}.hb\_ic.prn & INDEX TIME \newline INDEX FREQ \newline INDEX TIME \\ \hline
.PRINT HB FORMAT=GNUPLOT & \emph{circuit-file}.HB.TD.prn \newline \emph{circuit-file}.HB.FD.prn  \newline \emph{circuit-file}.hb\_ic.prn & INDEX TIME \newline INDEX FREQ \newline INDEX TIME \\ \hline
.PRINT HB FORMAT=SPLOT & \emph{circuit-file}.HB.TD.prn \newline \emph{circuit-file}.HB.FD.prn  \newline \emph{circuit-file}.hb\_ic.prn & INDEX TIME \newline INDEX FREQ \newline INDEX TIME \\ \hline
.PRINT HB FORMAT=NOINDEX & \emph{circuit-file}.HB.TD.prn \newline \emph{circuit-file}.HB.FD.prn  \newline \emph{circuit-file}.hb\_ic.prn & TIME \newline FREQ \newline TIME \\ \hline
.PRINT HB FORMAT=CSV & \emph{circuit-file}.HB.TD.csv \newline \emph{circuit-file}.HB.FD.csv  \newline \emph{circuit-file}.hb\_ic.csv &  TIME \newline FREQ \newline TIME \\ \hline
.PRINT HB FORMAT=TECPLOT & \emph{circuit-file}.HB.TD.dat \newline \emph{circuit-file}.HB.FD.dat  \newline \emph{circuit-file}.hb\_ic.dat &  TIME \newline FREQ \newline TIME \\ \hline

.PRINT HB\_FD & \emph{circuit-file}.HB.FD.prn & INDEX FREQ \\ \hline
.PRINT HB\_FD FORMAT=GNUPLOT& \emph{circuit-file}.HB.FD.prn & INDEX FREQ \\ \hline
.PRINT HB\_FD FORMAT=SPLOT& \emph{circuit-file}.HB.FD.prn & INDEX FREQ \\ \hline
.PRINT HB\_FD FORMAT=NOINDEX & \emph{circuit-file}.HB.FD.prn & FREQ \\ \hline
.PRINT HB\_FD FORMAT=CSV & \emph{circuit-file}.HB.FD.csv &  FREQ \\ \hline
.PRINT HB\_FD FORMAT=TECPLOT & \emph{circuit-file}.HB.FD.dat & FREQ \\ \hline

.PRINT HB\_TD & \emph{circuit-file}.HB.TD.prn & INDEX TIME \\ \hline
.PRINT HB\_TD FORMAT=GNUPLOT & \emph{circuit-file}.HB.TD.prn & INDEX TIME \\ \hline
.PRINT HB\_TD FORMAT=SPLOT & \emph{circuit-file}.HB.TD.prn & INDEX TIME \\ \hline
.PRINT HB\_TD FORMAT=NOINDEX & \emph{circuit-file}.HB.TD.prn & TIME \\ \hline
.PRINT HB\_TD FORMAT=CSV & \emph{circuit-file}.HB.TD.csv & TIME \\ \hline
.PRINT HB\_TD FORMAT=TECPLOT & \emph{circuit-file}.HB.TD.dat & TIME \\ \hline

\multicolumn{3}{c}{\smallskip\color{XyceDarkBlue}\em\bfseries  Startup Period} \\ \hline
.OPTIONS HBINT STARTUPPERIODS=<n> \newline .PRINT HB\_STARTUP & \emph{circuit-file}.startup.prn & INDEX TIME \\ \hline
.OPTIONS HBINT STARTUPPERIODS=<n> \newline .PRINT HB\_STARTUP FORMAT=GNUPLOT & \emph{circuit-file}.startup.prn & INDEX TIME \\ \hline
.OPTIONS HBINT STARTUPPERIODS=<n> \newline .PRINT HB\_STARTUP FORMAT=SPLOT & \emph{circuit-file}.startup.prn & INDEX TIME \\ \hline
.OPTIONS HBINT STARTUPPERIODS=<n> \newline .PRINT HB\_STARTUP FORMAT=NOINDEX & \emph{circuit-file}.startup.prn & TIME \\ \hline
.OPTIONS HBINT STARTUPPERIODS=<n> \newline .PRINT HB\_STARTUP FORMAT=CSV \newline .OPTIONS HBINT STARTUPPERIODS=<n> & \emph{circuit-file}.startup.csv & TIME \\ \hline
.OPTIONS HBINT STARTUPPERIODS=<n> \newline .PRINT HB\_STARTUP FORMAT=TECPLOT \newline .OPTIONS HBINT STARTUPPERIODS=<n> & \emph{circuit-file}.startup.dat & TIME \\ \hline

\multicolumn{3}{c}{\smallskip\color{XyceDarkBlue}\em\bfseries  Initial Conditions} \\ \hline
.OPTIONS HBINT SAVEICDATA=1 \newline .PRINT HB\_IC & \emph{circuit-file}.hb\_ic.prn  & INDEX TIME \\ \hline
.OPTIONS HBINT SAVEICDATA=1 \newline .PRINT HB\_IC FORMAT=GNUPLOT & \emph{circuit-file}.hb\_ic.prn  & INDEX TIME \\ \hline
.OPTIONS HBINT SAVEICDATA=1 \newline .PRINT HB\_IC FORMAT=SPLOT & \emph{circuit-file}.hb\_ic.prn  & INDEX TIME \\ \hline
.OPTIONS HBINT SAVEICDATA=1 \newline .PRINT HB\_IC FORMAT=NOINDEX & \emph{circuit-file}.hb\_ic.prn  & TIME \\ \hline
.OPTIONS HBINT SAVEICDATA=1 \newline .PRINT HB\_IC FORMAT=CSV  & \emph{circuit-file}.hb\_ic.csv & TIME \\ \hline
.OPTIONS HBINT SAVEICDATA=1 \newline .PRINT HB\_IC FORMAT=TECPLOT & \emph{circuit-file}.hb\_ic.dat & TIME \\ \hline

\multicolumn{3}{c}{\smallskip\color{XyceDarkBlue}\em\bfseries Additional Output Available} \\ \hline
.OP & \emph{log file} & Operating point data \\ \hline
.SENS \newline .PRINT SENS & \multicolumn{2}{c}{see~\nameref{Print_Sensitivity}} \\ \hline
.OPTIONS NONLIN CONTINUATION=<method> \newline .PRINT HOMOTOPY & \multicolumn{2}{c}{see~\nameref{Print_Homotopy}} \\ \hline
\end{PrintCommandTable}
}


\subsubsection{Print Noise Analysis}
\index{\texttt{.PRINT}!Noise Analysis}
\index{\texttt{.PRINT}!\texttt{NOISE}}
% Sandia National Laboratories is a multimission laboratory managed and
% operated by National Technology & Engineering Solutions of Sandia, LLC, a
% wholly owned subsidiary of Honeywell International Inc., for the U.S.
% Department of Energy’s National Nuclear Security Administration under
% contract DE-NA0003525.

% Copyright 2002-2023 National Technology & Engineering Solutions of Sandia,
% LLC (NTESS).

NOISE Analysis generates two output files, the primary output is in the
frequency domain and the initial conditions output is in the time domain.

{
\begin{PrintCommandTable}{Print NOISE Analysis Type}
.PRINT NOISE & \emph{circuit-file}.NOISE.prn & INDEX FREQ \\ \hline
.PRINT NOISE FORMAT=GNUPLOT & \emph{circuit-file}.NOISE.prn & INDEX FREQ \\ \hline
.PRINT NOISE FORMAT=SPLOT & \emph{circuit-file}.NOISE.prn & INDEX FREQ \\ \hline
.PRINT NOISE FORMAT=NOINDEX & \emph{circuit-file}.NOISE.prn & FREQ \\ \hline
.PRINT NOISE FORMAT=CSV & \emph{circuit-file}.NOISE.csv & FREQ \\ \hline
.PRINT NOISE FORMAT=TECPLOT & \emph{circuit-file}.NOISE.dat & FREQ \\ \hline
%.PRINT NOISE FORMAT=PROBE & \emph{circuit-file}.FD.csd & -- \\ \hline

\multicolumn{3}{c}{\smallskip\color{XyceDarkBlue}\em\bfseries Additional Output Available} \\ \hline
.OP & \emph{log file} & Operating point data \\ \hline
.OPTIONS NONLIN CONTINUATION=<method> \newline .PRINT HOMOTOPY & \multicolumn{2}{c}{see~\nameref{Print_Homotopy}} \\ \hline
\end{PrintCommandTable}
}



\subsubsection{Print Transient Analysis}
\index{\texttt{.PRINT}!Transient Analysis}
\index{\texttt{.PRINT}!\texttt{TRAN}}
% Sandia National Laboratories is a multimission laboratory managed and
% operated by National Technology & Engineering Solutions of Sandia, LLC, a
% wholly owned subsidiary of Honeywell International Inc., for the U.S.
% Department of Energy’s National Nuclear Security Administration under
% contract DE-NA0003525.

% Copyright 2002-2023 National Technology & Engineering Solutions of Sandia,
% LLC (NTESS).


Transient Analysis generates time domain output based on the format specified by the \texttt{.PRINT} command.

Homotopy and sensitivty output can also be generated.
{
\begin{PrintCommandTable}{Print Transient Analysis Type}
.PRINT TRAN & \emph{circuit-file}.prn & INDEX TIME \\ \hline
.PRINT TRAN FORMAT=GNUPLOT & \emph{circuit-file}.prn & INDEX TIME \\ \hline
.PRINT TRAN FORMAT=SPLOT & \emph{circuit-file}.prn & INDEX TIME \\ \hline
.PRINT TRAN FORMAT=NOINDEX & \emph{circuit-file}.prn & TIME \\ \hline
.PRINT TRAN FORMAT=CSV & \emph{circuit-file}.csv & TIME \\ \hline
.PRINT TRAN FORMAT=RAW & \emph{circuit-file}.raw & TIME \\ \hline
Xyce -a \newline .PRINT TRAN FORMAT=RAW & \emph{circuit-file}.raw & TIME \\ \hline
.PRINT TRAN FORMAT=TECPLOT & \emph{circuit-file}.dat & TIME \\ \hline
.PRINT TRAN FORMAT=PROBE & \emph{circuit-file}.csd & -- \\ \hline
\multicolumn{3}{c}{\smallskip\color{XyceDarkBlue}\em\bfseries Command Line Raw Override Output} \\ \hline
Xyce -r raw-file-name & \emph{raw-file-name} & All circuit variables printed \\ \hline
Xyce -r raw-file-name -a & \emph{raw-file-name} & All circuit variables printed \\ \hline
\multicolumn{3}{c}{\smallskip\color{XyceDarkBlue}\em\bfseries Additional Output Available} \\ \hline
.OP & \emph{log file} & Operating point data \\ \hline
.SENS \newline .PRINT SENS & \multicolumn{2}{c}{see~\nameref{Print_Sensitivity}} \\ \hline
.OPTIONS NONLIN CONTINUATION=<method> \newline .PRINT HOMOTOPY & \multicolumn{2}{c}{see~\nameref{Print_Homotopy}} \\ \hline
\end{PrintCommandTable}
}


\subsubsection{Print Homotopy}
\label{Print_Homotopy}
\index{\texttt{.PRINT}!Homotopy Analysis}
% Sandia National Laboratories is a multimission laboratory managed and
% operated by National Technology & Engineering Solutions of Sandia, LLC, a
% wholly owned subsidiary of Honeywell International Inc., for the U.S.
% Department of Energy’s National Nuclear Security Administration under
% contract DE-NA0003525.

% Copyright 2002-2023 National Technology & Engineering Solutions of Sandia,
% LLC (NTESS).

Homotopy output is generated by the inclusion of the \newline
\texttt{.OPTIONS NONLIN CONTINUATION=<method>} command.

{
\begin{PrintCommandTable}{Print Homotopy}
.OPTIONS NONLIN CONTINUATION=<method> \newline .PRINT HOMOTOPY & 
\emph{circuit-file}.HOMOTOPY.prn & INDEX TIME \\ \hline
.OPTIONS NONLIN CONTINUATION=<method> \newline .PRINT HOMOTOPY FORMAT=GNUPLOT & 
\emph{circuit-file}.HOMOTOPY.prn & INDEX TIME \\ \hline
.OPTIONS NONLIN CONTINUATION=<method> \newline .PRINT HOMOTOPY FORMAT=SPLOT &
\emph{circuit-file}.HOMOTOPY.prn & INDEX TIME \\ \hline
.OPTIONS NONLIN CONTINUATION=<method> \newline .PRINT HOMOTOPY FORMAT=NOINDEX &
\emph{circuit-file}.HOMOTOPY.prn & TIME \\ \hline
.OPTIONS NONLIN CONTINUATION=<method> \newline .PRINT HOMOTOPY FORMAT=CSV & 
\emph{circuit-file}.HOMOTOPY.csv & TIME \\ \hline
.OPTIONS NONLIN CONTINUATION=<method> \newline .PRINT HOMOTOPY FORMAT=TECPLOT & 
\emph{circuit-file}.HOMOTOPY.dat & TIME \\ \hline
\end{PrintCommandTable}
}


\subsubsection{Print Sensitivity}
\label{Print_Sensitivity}
\index{\texttt{.PRINT}!Homotopy Analysis}
% Sandia National Laboratories is a multimission laboratory managed and
% operated by National Technology & Engineering Solutions of Sandia, LLC, a
% wholly owned subsidiary of Honeywell International Inc., for the U.S.
% Department of Energy’s National Nuclear Security Administration under
% contract DE-NA0003525.

% Copyright 2002-2024 National Technology & Engineering Solutions of Sandia,
% LLC (NTESS).

Sensitivity is enabled by inclusion of the \newline .SENS command.

Steady-state sensitivities (adjoint or direct) and transient direct sensitivities
will be handled by the .PRINT SENS command. Transient adjoint, on the other hand,
is handled by the .PRINT TRANADJOINT command.

For transient sensitivity output, a \texttt{TIME} column will be included for the
STD, GNUPLOT, SPLOT, NOINDEX and CSV formats.  For AC sensitivity output, a \texttt{FREQ}
column will be included for the STD, GNUPLOT, SPLOT, NOINDEX and CSV formats.

{
\begin{PrintCommandTable}{Print Sensitivities for .TRAN and .DC}
.SENS objfunc={<$obj$>} param=[$p_1$1][,$p_n$]* \newline
.PRINT SENS & \emph{circuit-file}.SENS.prn & \{$obj$\}, d{\_}\{$obj$\}/d{\_}$p_1${\_}[dir\textbar adj], d{\_}\{$obj$\}/d{\_}$p_n${\_}[dir\textbar adj] \newline \\ \hline
.SENS objfunc={<$obj$>} param=[$p_1$1][,$p_n$]* \newline
.PRINT SENS FORMAT=GNUPLOT & \emph{circuit-file}.SENS.prn & \{$obj$\}, d{\_}\{$obj$\}/d{\_}$p_1${\_}[dir\textbar adj], d{\_}\{$obj$\}/d{\_}$p_n${\_}[dir\textbar adj] \newline \\ \hline
.SENS objfunc={<$obj$>} param=[$p_1$1][,$p_n$]* \newline
.PRINT SENS FORMAT=SPLOT & \emph{circuit-file}.SENS.prn & \{$obj$\}, d{\_}\{$obj$\}/d{\_}$p_1${\_}[dir\textbar adj], d{\_}\{$obj$\}/d{\_}$p_n${\_}[dir\textbar adj] \newline \\ \hline
.SENS objfunc={<$obj$>} param=[$p_1$1][,$p_n$]* \newline
.PRINT SENS FORMAT=NOINDEX & \emph{circuit-file}.SENS.prn & \{$obj$\}, d{\_}\{$obj$\}/d{\_}$p_1${\_}[dir\textbar adj], d{\_}\{$obj$\}/d{\_}$p_n${\_}[dir\textbar adj] \newline \\ \hline
.SENS objfunc={<$obj$>} param=[$p_1$1][,$p_n$]* \newline
.PRINT SENS FORMAT=CSV & \emph{circuit-file}.SENS.csv & \{$obj$\}, d{\_}\{$obj$\}/d{\_}$p_1${\_}[dir\textbar adj], d{\_}\{$obj$\}/d{\_}$p_n${\_}[dir\textbar adj] \newline \\ \hline
.SENS objfunc={<$obj$>} param=[$p_1$][,$p_n$]* \newline
.PRINT SENS FORMAT=TECPLOT & \emph{circuit-file}.SENS.dat & \{$obj$\}, d{\_}\{$obj$\}/d{\_}$p_1${\_}[dir\textbar adj], d{\_}\{$obj$\}/d{\_}$p_n${\_}[dir\textbar adj] \newline \\ \hline

\end{PrintCommandTable}
}

{
\begin{PrintCommandTable}{Print Sensitivities for .AC}
.SENS [objvars\textbar acobjfunc]={<$obj$>} \newline + param=[$p_1$1][,$p_n$]* \newline
.PRINT SENS & \emph{circuit-file}.FD.SENS.prn & \{$obj$\}, d{\_}\{$obj$\}/d{\_}$p_1${\_}[dir\textbar adj], d{\_}\{$obj$\}/d{\_}$p_n${\_}[dir\textbar adj] \newline \\ \hline
.SENS [objvars\textbar acobjfunc]={<$obj$>} \newline + param=[$p_1$1][,$p_n$]* \newline
.PRINT SENS FORMAT=GNUPLOT & \emph{circuit-file}.FD.SENS.prn & \{$obj$\}, d{\_}\{$obj$\}/d{\_}$p_1${\_}[dir\textbar adj], d{\_}\{$obj$\}/d{\_}$p_n${\_}[dir\textbar adj] \newline \\ \hline
.SENS [objvars\textbar acobjfunc]={<$obj$>} \newline + param=[$p_1$1][,$p_n$]* \newline
.PRINT SENS FORMAT=SPLOT & \emph{circuit-file}.FD.SENS.prn & \{$obj$\}, d{\_}\{$obj$\}/d{\_}$p_1${\_}[dir\textbar adj], d{\_}\{$obj$\}/d{\_}$p_n${\_}[dir\textbar adj] \newline \\ \hline
.SENS [objvars\textbar acobjfunc]={<$obj$>} \newline + param=[$p_1$1][,$p_n$]* \newline
.PRINT SENS FORMAT=NOINDEX & \emph{circuit-file}.FD.SENS.prn & \{$obj$\}, d{\_}\{$obj$\}/d{\_}$p_1${\_}[dir\textbar adj], d{\_}\{$obj$\}/d{\_}$p_n${\_}[dir\textbar adj] \newline \\ \hline
.SENS [objvars\textbar acobjfunc]={<$obj$>} \newline + param=[$p_1$1][,$p_n$]* \newline
.PRINT SENS FORMAT=CSV & \emph{circuit-file}.FD.SENS.csv & \{$obj$\}, d{\_}\{$obj$\}/d{\_}$p_1${\_}[dir\textbar adj], d{\_}\{$obj$\}/d{\_}$p_n${\_}[dir\textbar adj] \newline \\ \hline
.SENS [objvars\textbar acobjfunc]={<$obj$>} \newline + param=[$p_1$][,$p_n$]* \newline
.PRINT SENS FORMAT=TECPLOT & \emph{circuit-file}.FD.SENS.dat & \{$obj$\}, d{\_}\{$obj$\}/d{\_}$p_1${\_}[dir\textbar adj], d{\_}\{$obj$\}/d{\_}$p_n${\_}[dir\textbar adj] \newline \\ \hline

\end{PrintCommandTable}
}

{
\begin{PrintCommandTable}{Print Transient Adjoint Sensitivities}
.SENS objfunc={<$obj$>} param=[$p_1$1][,$p_n$]* \newline
.PRINT TRANADJOINT & \emph{circuit-file}.TRADJ.prn & \{$obj$\}, d{\_}\{$obj$\}/d{\_}$p_1${\_}adj, d{\_}\{$obj$\}/d{\_}$p_n${\_}adj \newline \\ \hline
.SENS objfunc={<$obj$>} param=[$p_1$1][,$p_n$]* \newline
.PRINT TRANADJOINT FORMAT=NOINDEX & \emph{circuit-file}.TRADJ.prn & \{$obj$\}, d{\_}\{$obj$\}/d{\_}$p_1${\_}adj, d{\_}\{$obj$\}/d{\_}$p_n${\_}adj \newline \\ \hline
.SENS objfunc={<$obj$>} param=[$p_1$1][,$p_n$]* \newline
.PRINT TRANADJOINT FORMAT=CSV & \emph{circuit-file}.TRADJ.csv & \{$obj$\}, d{\_}\{$obj$\}/d{\_}$p_1${\_}adj, d{\_}\{$obj$\}/d{\_}$p_n${\_}adj \newline \\ \hline
.SENS objfunc={<$obj$>} param=[$p_1$][,$p_n$]* \newline
.PRINT TRANADJOINT FORMAT=TECPLOT & \emph{circuit-file}.TRADJ.dat & \{$obj$\}, d{\_}\{$obj$\}/d{\_}$p_1${\_}adj, d{\_}\{$obj$\}/d{\_}$p_n${\_}adj \newline \\ \hline

\end{PrintCommandTable}
}


\subsubsection{Print Embedded Sampling Analysis}
\index{\texttt{.PRINT}!EMBEDDEDSAMPLING Analysis}
\index{\texttt{.PRINT}!\texttt{ES}}
% Sandia National Laboratories is a multimission laboratory managed and
% operated by National Technology & Engineering Solutions of Sandia, LLC, a
% wholly owned subsidiary of Honeywell International Inc., for the U.S.
% Department of Energy’s National Nuclear Security Administration under
% contract DE-NA0003525.

% Copyright 2002-2023 National Technology & Engineering Solutions of Sandia,
% LLC (NTESS).

EMBEDDEDSAMPLING Analysis generates one output file, with the information for
each output variable grouped in a set of contiguous columns, based on the format
specified by the \texttt{.PRINT} command.  The arguments \texttt{OUTPUT\_SAMPLE\_STATS}
and \texttt{OUTPUT\_ALL\_SAMPLES} are specific to \texttt{.PRINT ES} lines. Section
\ref{EMBEDDEDSAMPLING_section} has more details on their usage.

For a transient analysis, a TIME column will also be included for the STD,
GNUPLOT, SPLOT, NOINDEX and CSV formats.  The TIME variable will also be
included in the TECPLOT format for that case.

{
\begin{PrintCommandTable}{Print EMBEDDEDSAMPLING Analysis Type}
.PRINT ES & \emph{circuit-file}.ES.prn & INDEX \\ \hline
.PRINT ES FORMAT=GNUPLOT & \emph{circuit-file}.ES.prn & INDEX \\ \hline
.PRINT ES FORMAT=SPLOT & \emph{circuit-file}.ES.prn & INDEX \\ \hline
.PRINT ES FORMAT=NOINDEX & \emph{circuit-file}.ES.prn & -- \\ \hline
.PRINT ES FORMAT=CSV & \emph{circuit-file}.ES.csv & -- \\ \hline
.PRINT ES FORMAT=TECPLOT & \emph{circuit-file}.ES.dat & -- \\ \hline
\end{PrintCommandTable}
}



\subsubsection{Print Intrusive PCE Analysis}
\index{\texttt{.PRINT}!PCE Analysis}
\index{\texttt{.PRINT}!\texttt{PCE}}
% Sandia National Laboratories is a multimission laboratory managed and
% operated by National Technology & Engineering Solutions of Sandia, LLC, a
% wholly owned subsidiary of Honeywell International Inc., for the U.S.
% Department of Energy’s National Nuclear Security Administration under
% contract DE-NA0003525.

% Copyright 2002-2023 National Technology & Engineering Solutions of Sandia,
% LLC (NTESS).

PCE Analysis generates one output file, with the information for
each output variable grouped in a set of contiguous columns, based on the format
specified by the \texttt{.PRINT} command.  The arguments \texttt{OUTPUT\_SAMPLE\_STATS}
and \texttt{OUTPUT\_ALL\_SAMPLES} are specific to \texttt{.PRINT PCE} lines. Section
\ref{PCE_section} has more details on their usage.

For a transient analysis, a TIME column will also be included for the STD,
GNUPLOT, SPLOT, NOINDEX and CSV formats.  The TIME variable will also be
included in the TECPLOT format for that case.

{
\begin{PrintCommandTable}{Print PCE Analysis Type}
.PRINT PCE & \emph{circuit-file}.PCE.prn & INDEX \\ \hline
.PRINT PCE FORMAT=GNUPLOT & \emph{circuit-file}.PCE.prn & INDEX \\ \hline
.PRINT PCE FORMAT=SPLOT & \emph{circuit-file}.PCE.prn & INDEX \\ \hline
.PRINT PCE FORMAT=NOINDEX & \emph{circuit-file}.PCE.prn & -- \\ \hline
.PRINT PCE FORMAT=CSV & \emph{circuit-file}.PCE.csv & -- \\ \hline
.PRINT PCE FORMAT=TECPLOT & \emph{circuit-file}.PCE.dat & -- \\ \hline
\end{PrintCommandTable}
}



\subsubsection{Parameter Stepping}
\label{Print_Step}
\index{\texttt{.PRINT}!Parameter Stepping}
% Sandia National Laboratories is a multimission laboratory managed and
% operated by National Technology & Engineering Solutions of Sandia, LLC, a
% wholly owned subsidiary of Honeywell International Inc., for the U.S.
% Department of Energy’s National Nuclear Security Administration under
% contract DE-NA0003525.

% Copyright 2002-2023 National Technology & Engineering Solutions of Sandia,
% LLC (NTESS).

During parameter stepping, enabled with the \texttt{.STEP} command, the
output generated by each of analysis types varies.  Generally the FORMAT
indicates this variation, however some combinations of analysis and
format can result in additional variation.

The following table lists how the output differs for each analysis type
and format.

{
\renewcommand{\arraystretch}{1.2}
\begin{longtable}{>{\ttfamily\small}m{1in}<{\normalfont\small}>{\ttfamily\small}m{1.5in}<{\normalfont\footnotesize}m{2.25in}@{}}
  \rowcolor{XyceDarkBlue}
  \color{white}\normalfont\bf Print Type &
  \color{white}\bf Format &
  \color{white}\bf Description \endfirsthead
  \rowcolor{XyceDarkBlue}
  \color{white}\normalfont\bf Print Type &
  \color{white}\bf Format &
  \color{white}\bf Description \endhead
AC & STD & 1, 3, 4, 11, 12, 13 \\ \hline
AC & GNUPLOT & 1, 3, 4, 11, 12, 13, 20 \\ \hline
AC & SPLOT & 1, 3, 4, 11, 12, 13, 21 \\ \hline
AC & CSV & 4, 11 \\ \hline
AC & PROBE & 16 \\ \hline
AC & TECPLOT & 4, 12, 13, 18 \\ \hline
AC & RAW & 19 \\ \hline
AC & RAW (Xyce -a) & 19 \\ \hline
AC\_IC & STD & 1, 4, 11, 12, 13 \\ \hline
AC\_IC & GNUPLOT & 1, 4, 11, 12, 13, 20 \\ \hline
AC\_IC & SPLOT & 1, 4, 11, 12, 13, 21 \\ \hline
AC\_IC & CSV & 4, 11 \\ \hline
AC\_IC & PROBE & 16 \\ \hline
AC\_IC & TECPLOT & 12, 13, 18 \\ \hline
AC\_IC & RAW & 19 \\ \hline
AC\_IC & RAW (Xyce -a) & 19 \\ \hline
DC & STD & 1, 11, 12 \\ \hline
DC & GNUPLOT & 1, 11, 12, 20 \\ \hline
DC & SPLOT & 1, 11, 12, 21 \\ \hline
DC & CSV & 11 \\ \hline
DC & PROBE & 17 \\ \hline
DC & TECPLOT & 4, 12, 13, 18 \\ \hline
DC & RAW & 19 \\ \hline
DC & RAW (Xyce -a) & 19 \\ \hline
HB\_TD & STD & 1, 2, 4, 11, 12, 13 \\ \hline
HB\_TD & GNUPLOT & 1, 2, 4, 11, 12, 13, 20 \\ \hline
HB\_TD & SPLOT & 1, 2, 4, 11, 12, 13, 21 \\ \hline
HB\_TD & CSV & 11 \\ \hline
HB\_TD & TECPLOT & 12, 13, 18 \\ \hline
HB\_FD & STD & 1, 3, 4, 11, 12, 13 \\ \hline
HB\_FD & GNUPLOT & 1, 3, 4, 11, 12, 13, 20 \\ \hline
HB\_FD & SPLOT & 1, 3, 4, 11, 12, 13, 21 \\ \hline
HB\_FD & CSV & 4, 11 \\ \hline
HB\_FD & TECPLOT & 4, 12, 13, 18 \\ \hline
HB\_IC & STD & 1, 2, 4, 11, 12, 13 \\ \hline
HB\_IC & GNUPLOT & 1, 2, 4, 11, 12, 13, 20 \\ \hline
HB\_IC & SPLOT & 1, 2, 4, 11, 12, 13, 21 \\ \hline
HB\_IC & CSV & 11 \\ \hline
HB\_IC & TECPLOT & 12, 13, 18 \\ \hline
HB\_STARTUP & STD & 1, 2, 4, 11, 12, 13 \\ \hline
HB\_STARTUP & GNUPLOT & 1, 2, 4, 11, 12, 13, 20 \\ \hline
HB\_STARTUP & SPLOT & 1, 2, 4, 11, 12, 13, 21 \\ \hline
HB\_STARTUP & CSV & 11 \\ \hline
HB\_STARTUP & TECPLOT & 12, 13, 18 \\ \hline
TRAN & STD & 1, 2, 11, 12 \\ \hline
TRAN & GNUPLOT & 1, 2, 11, 12, 20 \\ \hline
TRAN & SPLOT & 1, 2, 11, 12, 21 \\ \hline
TRAN & CSV & 2, 11 \\ \hline
TRAN & PROBE & 17 \\ \hline
TRAN & TECPLOT & 2, 4, 12, 13, 18 \\ \hline
TRAN & RAW & 2, 19 \\ \hline
TRAN & RAW (Xyce -a) & 2, 19 \\ \hline
\multicolumn{3}{c}{\smallskip\color{XyceDarkBlue}\em\bfseries Specialized Output Commands} \\
HOMOTOPY & STD & 1, 2, 4, 11, 15 \\ \hline
HOMOTOPY & GNUPLOT & 1, 2, 4, 11, 15, 20 \\ \hline
HOMOTOPY & SPLOT & 1, 2, 4, 11, 15, 21 \\ \hline
HOMOTOPY & CSV & 2, 11 \\ \hline
HOMOTOPY & PROBE & 17 \\ \hline
HOMOTOPY & TECPLOT & 2, 4, 15, 18 \\ \hline
SENSITIVITY & STD & 1, 2, 11, 14 \\ \hline
SENSITIVITY & GNUPLOT & 1, 2, 11, 14, 20 \\ \hline
SENSITIVITY & SPLOT & 1, 2, 11, 14, 21 \\ \hline
SENSITIVITY & CSV & 2, 11 \\ \hline
SENSITIVITY & TECPLOT & 2, 14, 18 \\ \hline
\end{longtable}
}

{
\renewcommand{\arraystretch}{1.2}
\begin{longtable}{>{\ttfamily\small}m{1in}<{\normalfont\small}>{\ttfamily\small}m{3.75in}@{}}
  \rowcolor{XyceDarkBlue}
  \color{white}\normalfont\bf Description & 
  \endfirsthead
  \rowcolor{XyceDarkBlue}
  \color{white}\normalfont\bf Description & 
  \endhead
1 & INDEX column added to output variable list \\
2 & TIME column added to output variable list \\
3 & FREQ column added to output variable list \\
4 & Frequency domain data written as Re($var$) and Im($var$) \\
11 & INDEX resets to zero at start of each .STEP \\
12 & Prints 'End of Xyce(TM) Parameter Sweep' at end of .STEP simulation \\
13 & Prints 'End of Xyce(TM) Simulation' at end of non-.STEP simulation \\
14 & Prints 'End of Xyce(TM) Sensitivity Simulation' at end of simulation \\
15 & Prints 'End of Xyce(TM) Homotopy Simulation' at end of simulation \\
16 & Two '\#;' at the end of each .STEP (BUG) \\
17 & One '\#;' at end of each .STEP \\
18 & New ZONE for each .STEP, and AUXDATA for each .STEP parameter \\
19 & Prints 'Plotname: Step Analysis: Step $s$ of $n$ params' at the start of each .STEP \\
20 & Inserts two blank lines before the data for steps 1,2,3,... where 
     the first step is step 0 \\
21 & Inserts one blank line before the data for steps 1,2,3,... where
     the first step is step 0 \\
\end{longtable}
}



\subsubsection{Print Wildcards}
\label{Print_Wildcards}
\index{\texttt{.PRINT}!Wildcards}
Wildcards are supported on \texttt{.PRINT} lines, as described below.  In particular, 
\texttt{V(*)} will print all of the node voltages in the circuit for all
analysis modes.  The \texttt{P(*)}  and \texttt{W(*)} wildcards are supported
for analysis modes (TRAN and DC) that support power calculations.

For TRAN and DC analysis modes, \texttt{I(*)} will print all of the
currents.  This includes both solution variables, which generally means those
associated  with voltage sources and inductors that are not coupled through
a mutual inductance device, and the lead currents associated with most
other devices.  For TRAN and DC, the \texttt{I(*)} wildcard also supports
lead currents for the multi-terminal J, M and Z devices via
\texttt{IB(*)}, \texttt{ID(*)}, \texttt{IG(*)} and \texttt{IS(*)}, and 
for the multi-terminal Q device via \texttt{IB(*)}, \texttt{IC(*)}, \texttt{IE(*)}
and \texttt{IS(*)}.  The \texttt{IE(*)} wildcard is also supported for SOI
and CMG devices.  A  request for \texttt{I(*)} will not return any of the lead currents
for  J, M, Q or Z devices.  Wildcards of the form \texttt{I1(*)},
that use numerical designators, are only supported for the T and YGENEXT devices.
Finally, as an example, a request for IC(*) in a netlist that does not contain
any Q devices will be silently ignored. 

For AC and NOISE analysis modes, the \texttt{I(*)} operator will only output
the branch currents, since lead currents are not supported for those two
analysis modes.  The \texttt{VR(*)}, \texttt{VI(*)}, \texttt{VP(*)},
\texttt{VM(*)}, \texttt{VDB(*)}, \texttt{IR(*)}, \texttt{II(*)},
\texttt{IP(*)}, \texttt{IM(*)} and \texttt{IDB(*)} wildcards are also
supported for these two analysis modes.

There is also support for the * character (meaning ``zero or more
characters'') and the ? character (meaning ``any one character'') in more
complex wildcards, where the * and/or ? characters can be in
any positions in the wildcard specification.  For example, \texttt{V(X1*)}
will output the voltage at all nodes in subcircuit X1 for all analysis
modes.  As another example, \texttt{V(1?)} will output the voltage at all
nodes that have two-character names that start with the character 1.  These
more complex wildcards should work for all supported voltage operators.

Similarly, \texttt{P(X1*)} or \texttt{W(X1*)} will output the
power for all devices, that support power calculations, in subcircuit X1.
Devices that don't support power calculations
will be silently omitted.  Alternately, \texttt{P(R?)} or \texttt{W(R?)}
will output the power for all resistors that have two-character names.

More complex wildcards are also supported for all valid current operators.
The caveats are that for DC and TRAN analyses, the wildcard will include
both branch and lead currents.  For AC and NOISE analyses, the wildcard will
only include branch currents. 

\subsubsection{Device Parameters and Internal Variables}
\label{Print_Device_Info}
\index{\texttt{.PRINT}!Device Parameters and Internal Variables}
This subsection describes how to print out device parameters and device 
internal variables, via a simple V-R circuit example. In particular, 
the example given below gives illustrative examples of how to print out the voltage
at a node ({\tt V(1)}), the current through a device ({\tt I(V1)}), the current through
a device using using an internal solution variable ({\tt N(V1\_branch)}), a device 
parameter ({\tt R1:R}) and the power dissipated by a device ({\tt P(R1)}).  It also shows
how device parameters and internal variables can be used in a \Xyce{} expression.
\begin{alltt}
* filename is example.cir
.DC V1 1 2 1
V1 1 0 1
R1 1 0 2
.PRINT DC FORMAT=NOINDEX PRECISION=2 WIDTH=8 
+ V(1) I(V1) N(V1_branch) R1:R P(R1) \{R1:R*N(V1_branch)*I(V1)\}
.END
\end{alltt}
The \Xyce{} output would then be (where the {\tt NOFORMAT}, {\tt WIDTH} and {\tt PRECISION}
arguments were used mainly to format the example output for this guide):
\begin{alltt}
   V(1)        I(V1)    N(V1\_BRANCH)    R1:R        P(R1)    \{R1:R*N(V1\_BRANCH)*I(V1)\}
   1.00e+00   -5.00e-01   -5.00e-01    2.00e+00    5.00e-01    5.00e-01
   2.00e+00   -1.00e+00   -1.00e+00    2.00e+00    2.00e+00    2.00e+00
\end{alltt}
The internal solution variables for each \Xyce{} device are typically not given in the Reference 
Guide sections on those devices.  However, if for the example given above, the user runs \texttt{Xyce 
-namesfile example\_names example.cir} then the file \texttt{example\_names} would contain a list
of the two solution variables that are accessible with the {\tt N()} syntax on a {\tt .PRINT} line.
In this simple example, they are the voltage at Node 1 and the branch current through the voltage
source {\tt V1}.  If {\tt V1} was in a subcircuit then the \texttt{example\_names} file would have shown
the ``fully-qualified'' device name, including the subcircuit names.
\begin{alltt}
HEADER
	0	   v1_branch
	1	           1
\end{alltt}
Additional (and more useful) examples for using the N() syntax to print out:
\begin{XyceItemize}
\item The $M$, $R$, $B$ and $H$ internal variables for mutual inductors are given in Section
   \ref{K_DEVICE}.  This includes an example where the mutual inductor is in a sub-circuit.
\item The $g_{m}$ (tranconductance), $V_{th}$, $V_{ds}$, $V_{gs}$, $V_{bs}$, and $V_{dsat}$ 
   internal variables for the BSIM3 and BSIM4 models for the MOSFET are given in 
   Section \ref{M_DEVICE}.
\end{XyceItemize}
In these two cases, only the $M$ and $R$ variables for the mutual inductors are
actually solution variables.  However, the {\tt -namesfile} approach can still be
used to determine the fully-qualified \Xyce{} device names required to use the {\tt N()}
syntax.



%%%%%%%%%%%%%%%%%%%%%%%%%%%%%%%%%%%%%%%%%%%%%%%%%%%%%%%%%%%%%%%%%%%%%%%%%%%%%%%%
\newpage
\subsection{\texttt{.RESULT} (Print results)}\label{.RESULT}
% Sandia National Laboratories is a multimission laboratory managed and
% operated by National Technology & Engineering Solutions of Sandia, LLC, a
% wholly owned subsidiary of Honeywell International Inc., for the U.S.
% Department of Energy’s National Nuclear Security Administration under
% contract DE-NA0003525.

% Copyright 2002-2024 National Technology & Engineering Solutions of Sandia,
% LLC (NTESS).


\index{\texttt{.RESULT}}
Outputs the value of user-specified expressions at the end of a simulation.

\begin{Command}

\format
\begin{alltt}
.RESULT \{output variable\}
\end{alltt}

\examples
\begin{alltt}
.RESULT \{V(a)\}
.RESULT \{V(a)+V(b)\}
\end{alltt}

\comments
The \texttt{.RESULT} line must use an expression.  The line \texttt{.RESULT V(a)} will result in a parse
error.

Each \texttt{.RESULT} line must have only one expression.  Multiple \texttt{.RESULT} lines can be
used though to output multiple columns in the output \texttt{.res} file.

\Xyce{} will not produce output for \texttt{.RESULT} statements if there are 
no \texttt{.STEP} statements in the netlist.   

\end{Command}

\subsubsection{Example Netlist}
\texttt{.RESULT} lines can be combined with \texttt{.STEP} lines to output the ending values of
multiple simulation runs in one \texttt{.res} file, as shown in the following usage example. The 
resultant \texttt{.res} file will have four lines that give the final values of the expressions
\texttt{\{v(b)\}} and \texttt{\{v(b)*v(b)/2\}} at time=0.75 seconds for all four requested 
combinations of \texttt{R2} and \texttt{v\_amplitude}. 
\begin{alltt}
Simple Example of .RESULT capability with .STEP
R1 a b 10.0
R2 b 0 2.0

.GLOBAL_PARAM  v_amplitude=2.0
Va a 0 sin (5.0 \{v_amplitude\} 1.0 0.0 0.0)

.PRINT TRAN v(b) \{v(b)*v(b)/2\}
.TRAN 0 0.75

.STEP R2 1.0 2.0 1.0 
.STEP v_amplitude 1.0 2.0 1.0

.RESULT \{v(b)\} 
.RESULT \{v(b)*v(b)/2\}

.END
\end{alltt}


%%{%%%%%%%%%%%%%%%%%%%%%%%%%%%%%%%%%%%%%%%%%%%%%%%%%%%%%%%%%%%%%%%%%%%%%%%%%%%%%%
\newpage
\subsection{\texttt{.SAMPLING} (Sampling UQ Analysis)}\label{.SAMPLING}
% Sandia National Laboratories is a multimission laboratory managed and
% operated by National Technology & Engineering Solutions of Sandia, LLC, a
% wholly owned subsidiary of Honeywell International Inc., for the U.S.
% Department of Energy’s National Nuclear Security Administration under
% contract DE-NA0003525.

% Copyright 2002-2024 National Technology & Engineering Solutions of Sandia,
% LLC (NTESS).


\index{\texttt{.SAMPLING}}
\index{analysis!sampling} \index{sampling analysis}
Calculates a full analysis (\verb|.DC|, \verb|.TRAN|, \verb|.AC|, etc.) over a distribution of
parameter values.  Sampling operates similarly to \verb|.STEP|, except that the parameter
values are generated from random distributions rather than sweeps.  
If used in conjunction 
with projection-based PCE methods, then the sample points are not based on random samples.  
Instead they are based on the quadrature points.

\index{analysis!SAMPLING} 
\index{SAMPLING analysis}
\index{analysis!MC} 
\index{MC analysis}
\index{analysis!PCE}
\index{PCE analysis}
\index{analysis!Monte Carlo} 
\index{Monte Carlo analysis}
\index{analysis!LHS} 
\index{LHS analysis}
\index{Latin Hypercube Sampling analysis}

\begin{Command}
\format
.SAMPLING  \\
+ param=<parameter name>,[parameter name]*  \\
+ type=<parameter type>,[parameter type]*  \\
+ means=<mean>,[mean]*  \\
+ std\_deviations=<standard deviation>,[standard deviation]* 

\examples
\begin{alltt}
.SAMPLING
+ param=R1
+ type=normal
+ means=3K
+ std\_deviations=1K

.SAMPLING
+ param=R1,R2
+ type=uniform,uniform
+ lower\_bounds=1K,2K
+ upper\_bounds=5K,6K

.SAMPLING
+ useExpr=true

.options SAMPLES numsamples=10000

.options SAMPLES numsamples=25000
+ OUTPUTS=\{R1:R\},\{V(1)\}
+ SAMPLE\_TYPE=MC

.options SAMPLES numsamples=1000
+ MEASURES=maxSine
+ SAMPLE\_TYPE=LHS

.options samples numsamples=30
+ covmatrix=1e6,1.0e-3,1.0e-3,4e-14
+ OUTPUTS=\{V(1)\},\{R1:R\},\{C1:C\}
\end{alltt}

\arguments

\begin{Arguments}

\argument{param}
Names of the parameters to be sampled.  This may be any of the parameters that are valid 
for \verb|.STEP|, including device instance, device model, or global parameters.  
  If more than one parameter, then specify as a comma-separated list.

\argument{type}
Distribution type for each parameter.  This may be uniform or normal.  
  If more than one parameter, then specify as a comma-separated list.

\argument{means}
If using normal distributions, the mean for each parameter must be specified.
  If more than one parameter, then specify as a comma-separated list.

\argument{std\_deviations}
If using normal distributions, the standard deviation for each parameter must be specified.
  If more than one parameter, then specify as a comma-separated list.

\argument{lower\_bounds}
If using uniform distributions, the lower bound must be specified.  This is optional for normal distributions.  
  If used with normal distributions, may alter the mean and standard deviation.
  If more than one parameter, then specify as a comma-separated list.

\argument{upper\_bounds}
If using uniform distributions, the upper bound must be specified.  This is optional for normal distributions.
  If used with normal distributions, may alter the mean and standard deviation.
  If more than one parameter, then specify as a comma-separated list.

\argument{useExpr}
If this argument is set to true, then the sampling algorithm will set up random 
  inputs from expression operators such as \verb|AGAUSS| and \verb|AUNIF|.  In 
  this case it will also ignore the list of parameters on the \verb|.SAMPLING| command line.
  For a complete description of expression-based random operators, see the expression
  documentation in section~\ref{ExpressionDocumentation}.

\end{Arguments}

\comments

In addition to the \verb|.SAMPLING| command, this analysis requires a 
  \verb|.options SAMPLES| command as well.  The \verb|.SAMPLING| command specifies 
  parameters and their attributes, either using the \verb|useExpr| option, or 
  with comma-separated lists.  The \verb|.options SAMPLES| command specifies 
  analysis options, including the number of samples, the type of sampling (LHS or MC)
  and the outputs and/or measures for which to compute statistics.
This line also allows one to specify a non-intrusive Polynomial Chaos 
Expansion (PCE) method (either regression or projection PCE).  
To see the details of the \verb|.options SAMPLES| command , see table~\ref{SamplesPKG}.

On the \verb|.SAMPLING| command line, if not using \verb|useExpr|, 
  parameters and their attributes must be specified 
  using comma-separated lists.  The comma-separated lists must all be the same length.

\end{Command}



%%%%%%%%%%%%%%%%%%%%%%%%%%%%%%%%%%%%%%%%%%%%%%%%%%%%%%%%%%%%%%%%%%%%%%%%%%%%%%%%
\newpage
\subsection{\texttt{.SAVE} (Save operating point conditions)}
% Sandia National Laboratories is a multimission laboratory managed and
% operated by National Technology & Engineering Solutions of Sandia, LLC, a
% wholly owned subsidiary of Honeywell International Inc., for the U.S.
% Department of Energy’s National Nuclear Security Administration under
% contract DE-NA0003525.

% Copyright 2002-2024 National Technology & Engineering Solutions of Sandia,
% LLC (NTESS).



\index{\texttt{.SAVE}}
\index{output!save} \index{save operating point conditions}
Stores the operating point of a circuit in the specified file for use in subsequent simulations.  The data may be saved as \texttt{.IC} or \texttt{.NODESET} lines.

\begin{Command}

\format
\begin{alltt}
.SAVE [TYPE=<IC|NODESET>] [FILE=<filename>] [LEVEL=<all|none>]
+ [TIME=<save\_time>]
\end{alltt}

\examples
\begin{alltt}
.SAVE TYPE=IC FILE=mycircuit.ic
.SAVE TYPE=NODESET FILE=myothercircuit.ic

.include mycircuit.ic
\end{alltt}

\comments

  The file created by \texttt{.SAVE} will contain \texttt{.IC} or
  \texttt{.NODESET} lines containing all the voltage node values at the DC
  operating point of the circuit. The default \textrmb{TYPE} is
  \texttt{NODESET}. The default \texttt{filename} is
  \texttt{\emph{netlist.cir}.ic}.
  
  The resulting file may be used in subsequent simulations to obtain quick DC
  convergence simply by including it in the netlist, as in the third example
  line above.  \Xyce{} has no corresponding \texttt{.LOAD} statement.
  
  The \textrmb{LEVEL} parameter is included for compatibility with HSPICE
  netlists. If \texttt{none} is specified, then no save file is created. The
  default \textrmb{LEVEL} is \texttt{all}.
  
  \textrmb{TIME} is also an HSPICE compatibility parameter. This is unsupported
  in \Xyce{}. \Xyce{} outputs the save file only at time=0.0.

\end{Command}



%%%%%%%%%%%%%%%%%%%%%%%%%%%%%%%%%%%%%%%%%%%%%%%%%%%%%%%%%%%%%%%%%%%%%%%%%%%%%%%%
\newpage
\subsection{\texttt{.SENS} (Compute DC, AC or transient sensitivities)}
\label{SensitivityAnalysis}
% Sandia National Laboratories is a multimission laboratory managed and
% operated by National Technology & Engineering Solutions of Sandia, LLC, a
% wholly owned subsidiary of Honeywell International Inc., for the U.S.
% Department of Energy’s National Nuclear Security Administration under
% contract DE-NA0003525.

% Copyright 2002-2024 National Technology & Engineering Solutions of Sandia,
% LLC (NTESS).


\index{\texttt{.SENS}}
\index{results!sens} \index{sensitivity}
Computes sensitivies for a user-specificed objective function
with respect to a user-specified list of circuit parameters.  


\begin{Command}

\format
.SENS objfunc=<output expression(s)> param=<circuit parameter(s)>  

\examples
\begin{alltt}
.SENS objfunc=\{0.5*(V(B)-3.0)**2.0\} param=R1:R,R2:R
.options SENSITIVITY direct=1 adjoint=1

.SENS objfunc=\{I(VM)\},\{V(3)*V(3)\} param= Q2N2222:bf

.param RES=1k
.SENS objfunc=\{RES*V(3)*V(3)\} param=C1:C

.param res=2
.func powerTestFunc(I) \{res*I*I\}
.SENS objfunc=\{powerTestFunc(I(V1))\} param=R1:R

.global\_param res=2
.SENS objfunc=\{res*I(V1)\} param=R1:R

.global\_param res=3.0k
.SENS objfunc=\{res*I(V1)\} param=res

* AC example using objvars
.sens objvars=2,3 param=r1:r,c1:c,v1:acmag

* AC example using acobjfunc
.sens acobjfunc=\{2.0*V(2)\},\{I(VM)\} param=r1:r,c1:c
\end{alltt}

\comments

This capability can be applied to either DC, transient or AC analysis.  
Both direct and adjoint sensitivities are supported. 
The user can optionally request either direct or adjoint sensitivities, 
or both.  

Although \Xyce{} will allow the user to specify both direct and 
adjoint, one would generally not choose to do both.
The best choice of sensitivity method depends on the problem.  For problems 
with a small number of parameters, and (possibly) lots of objective functions, 
then the direct method is a more efficient choice.  For problems with large 
numbers of parameters, but a small number of objective functions, the 
adjoint method is more efficient.

For all variants of sensitivity analysis, it is necessary to specify 
circuit parameters on the \texttt{.SENS} line in a comma-separated list.  Unlike the SPICE version, 
this capability will not automatically use every parameter in the circuit.
It is also necessary for all variations of sensitivity analysis to specify at least 
one objective function.  This capability will not assume any particular 
objective function.  Also, it is possible to specify multiple
objective functions, in a comma-separated list.

As noted, for transient analysis, both types of sensitivities are supported.
Direct sensitivities are computed at each time step during the forward 
calculation.  Transient adjoint sensitivities, in contrast, must be computed
using a reverse time integration method.  The reverse time integration must be 
performed after the original forward calculation is complete.  As such, transient 
adjoint sensitivity calculations can be thought of as a post-processing step.  
One consequence of this is that transient adjoint output must be specified using 
the \texttt{.PRINT TRANADJOINT} type, rather than the \texttt{.PRINT SENS} 
type.

If transient adjoints are specified, the default behavior for the capability is 
for a transient sensitivity calculation be performed for each time step, even 
if the forward transient simulation consists of millions of steps.  For adjoint 
calculations, this can be problematic, as adjoint methods (noted above) are not 
very efficient when applied to problems with a large number of objective functions.
Each time step, from the point of view of transient adjoints, is effectively a 
separate objective function.  As such, this isn't the best use of adjoints.  
One can specify  a list of time points for which to compute transient adjoint
sensitivities. For many practical problems, the sensitivies at only one or a 
handful of points is needed, so this is a good way to mitigate the computational 
cost of adjoints.  The \Xyce{} Users' Guide~\UsersGuide{} provides an example.

If performing a sensitivity calculation with AC analysis, 
there are two options for the specification of the 
objective function. These options are both different from the DC and TRAN method.
Instead of specifying objective functions with the parameter \texttt{objfunc},
one should either use \texttt{objvars} or \texttt{acobjfunc}.  
The parameter \texttt{objvars} should be followed by a comma separated list of voltage nodes.
The parameter \texttt{acobjfunc} should be followed by a comma separated list of objective functions.
It is also possible to use both specifications in the same netlist.

\end{Command}


%%{%%%%%%%%%%%%%%%%%%%%%%%%%%%%%%%%%%%%%%%%%%%%%%%%%%%%%%%%%%%%%%%%%%%%%%%%%%%%%%
\newpage
\subsection{\texttt{.STEP} (Step Parametric Analysis)}\label{.STEP}
% Sandia National Laboratories is a multimission laboratory managed and
% operated by National Technology & Engineering Solutions of Sandia, LLC, a
% wholly owned subsidiary of Honeywell International Inc., for the U.S.
% Department of Energy’s National Nuclear Security Administration under
% contract DE-NA0003525.

% Copyright 2002-2024 National Technology & Engineering Solutions of Sandia,
% LLC (NTESS).


\index{\texttt{.STEP}}
\index{analysis!step} \index{step parametric analysis}
Calculates a full analysis (\verb|.DC|, \verb|.TRAN|, \verb|.AC|, etc.) over a range of
parameter values.  This type of analysis is very similar to .DC analysis.
Similar to .DC analysis, .STEP supports sweeps which are
linear, decade logarithmic, octave logarithmic, a list of values, or over a multivariate data table.

\begin{description}

\item[\tt LIN] Linear sweep \\
The sweep variable is swept linearly from the starting to the ending value.

\item[\tt OCT] Sweep by octaves \\
The sweep variable is swept logarithmically by octaves.

\item[\tt DEC] Sweep by decades \\
The sweep variable is swept logarithmically by decades.

\item[\tt LIST] Sweep over specified values \\
The sweep variable is swept over an enumerated list of values.

\item[\tt DATA] Sweep over table of multivariate values\\
The sweep variables are swept over the rows of a table.  

\end{description}

\subsubsection{Linear Sweeps}
\index{analysis!STEP!Linear sweeps} \index{STEP analysis!Linear sweeps}

\begin{Command}

\format
.STEP [LIN] <parameter name> <initial> <final> <step>

\examples
\begin{alltt}
.STEP R1 45 50 5
.STEP V1 20 10 -1
.STEP LIN V1 20 10 -1
.STEP TEMP -45 -55 -10
.STEP C101:C 45 50 5
.STEP DLEAK:IS 1.0e-12 1.0e-11 1.0e-12

.global_param v1_val=10
V1 1 0 DC \{v1_val\}
.STEP v1_val 20 10 -1

.param v2_val=10
V2 2 0 DC \{v2_val\}
.STEP v2_val 20 10 -1

.data table
+ c1 r1
+ 1e-8  1k
+ 2e-8  0.5k
+ 3e-8  0.25k
.enddata
.STEP data=table
\end{alltt}

\arguments

\begin{Arguments}

\argument{parameter name}
Name of the parameter to be swept.  This may be the special parameter
name \texttt{TEMP} (the ambient simulation temperature), a device
name, device instance or model parameter name, or global parameter
name as defined in a \texttt{.global\_param} or globally-scoped \texttt{.param} statement.  

If a device name is given, the primary parameter for that device is
taken as the parameter; in the first two examples above, the primary
parameters of the devices R1 and V1 are stepped (resistance and DC
voltage, respectively).  The C, L and I devices are then the other
devices with primary parameters, which are the capacitance, inductance 
and DC current, respectively.

To specify a device instance parameter other than the device's primary
parameter, or if the device has no primary parameter, use the 
syntax \texttt{<device name>}:\texttt{<parameter name>}, as
in the fourth example above.

To sweep a device model parameter, use the syntax \texttt{<model
name>}:\texttt{<parameter name>}, as in the fifth example above.

\argument{initial}
Initial value for the parameter.

\argument{final}
Final value for the parameter.

\argument{step}
Value that the parameter is incremented at each step.

\end{Arguments}

\comments

For linear sweeps, the LIN keyword is optional.

STEP parameter analysis will sweep a parameter from its initial value to
its final value, at increments of the step size.  At each step of this
sweep, it will conduct a full analysis (\texttt{.DC}, \texttt{.TRAN},
\texttt{.AC}, etc.) of the circuit.

The specification is similar to that of a \texttt{.DC} sweep, except
that unlike \texttt{.DC}, only one parameter may be swept on
each \texttt{.STEP} line.  Multiple \texttt{.STEP} lines may be
specified, forming nested step loops.  The variables will be stepped
in order such that the first \texttt{.STEP} line that appears in the
netlist will be the innermost loop, and the last \texttt{.STEP} line
will be the outermost.

Output, as designated by a \texttt{.PRINT} statement, is slightly more
complicated in the case of a \texttt{.STEP} simulation.  If the user
has specified a \texttt{.PRINT} line in the input file, \Xyce{} will
output two files.  All steps of the sweep will be output to a single file as
usual, but with the results of each step appearing one after another
with the ``Index'' column starting over at zero.  Additionally, a file
with a ``.res'' suffix will be produced indicating what parameters
were used for each iteration of the step loops; this file will always
be in columnar text format, irrespective of any \texttt{FORMAT=}
option specified on \texttt{.PRINT} lines.  If \texttt{.RESULT} lines
(see section~\ref{.RESULT}) appear in the netlist, the ``.res'' file
will also contain columns for each expression given
on the \texttt{.RESULT} lines, and the value of the result expression
will be printed for each step taken.

Note that analysis lines in \Xyce{} do not currently support use of
expressions to define their parameters (e.g., end times
for \texttt{.TRAN} analysis, or fundamental frequencies
for \texttt{.HB} analysis), and so it is not possible to use stepped
parameters to vary how the analysis will be run at each step.  If each
step requires different analysis parameters, this would have to be
accomplished by performing separate runs of \Xyce{}.

If the stop value is smaller than the start value, the step value
should be negative.  If a positive step value is given in this case,
only a single point (at the start value) will be performed, and a
warning will be emitted.

\end{Command}

\subsubsection{Decade Sweeps}
\index{analysis!STEP!Decade sweeps} \index{STEP analysis!Decade sweeps}

\begin{Command}

\format
.STEP DEC <sweep variable name> <start> <stop> <points>

\examples

\begin{alltt}
.STEP DEC VIN 1 100 2
.STEP DEC R1 100 10000 3 
.STEP DEC TEMP 1.0 10.0 3
\end{alltt}

\comments
The stop value should be larger than the start value.  If a stop value
smaller than the start value is given, only a single point at the
start value will be performed, and a warning will be emitted.  The
points value must be an integer.

\end{Command}

\subsubsection{Octave Sweeps}
\index{analysis!STEP!Octave sweeps} \index{STEP analysis!Octave sweeps}

\begin{Command}

\format
.STEP OCT <sweep variable name> <start> <stop> <points>

\examples

\begin{alltt}
.STEP OCT VIN 0.125 64 2 
.STEP OCT TEMP 0.125 16.0 2 
.STEP OCT R1 0.015625 512 3
\end{alltt}

\comments
The stop value should be larger than the start value.  If a stop value
smaller than the start value is given, only a single point at the
start value will be performed, and a warning will be emitted.  The
points value must be an integer.

\end{Command}

\subsubsection{List Sweeps}
\index{analysis!STEP!List sweeps} \index{STEP analysis!List sweeps}

\begin{Command}

\format
\begin{alltt}
.STEP <sweep variable name> LIST <val> <val> <val>\ldots
\end{alltt}

\examples
\begin{alltt}
.STEP VIN LIST 1.0 2.0 10. 12.0 
.STEP TEMP LIST 8.0 21.0
\end{alltt}

\end{Command}

\subsubsection{Data Sweeps}
\index{analysis!DC!Data Sweeps} \index{DC analysis!Data sweeps}

\begin{Command}
\format
\begin{alltt}
.STEP DATA=<data table name> 
\end{alltt}

\examples
.STEP data=resistorValues

.data resistorValues \\
+ r1   r2 \\
+ 8.0000e+00  4.0000e+00 \\
+ 9.0000e+00  4.0000e+00 \\
.enddata

\end{Command}


%%%%%%%%%%%%%%%%%%%%%%%%%%%%%%%%%%%%%%%%%%%%%%%%%%%%%%%%%%%%%%%%%%%%%%%%%%%%%%%%
\newpage
\subsection{\texttt{.SUBCKT} (Subcircuit)}
\label{SubcircuitDefinition}
\index{subcircuit}
% Sandia National Laboratories is a multimission laboratory managed and
% operated by National Technology & Engineering Solutions of Sandia, LLC, a
% wholly owned subsidiary of Honeywell International Inc., for the U.S.
% Department of Energy’s National Nuclear Security Administration under
% contract DE-NA0003525.

% Copyright 2002-2023 National Technology & Engineering Solutions of Sandia,
% LLC (NTESS).


\label{SUBCKTsection}
The \index{\texttt{.SUBCKT}} \texttt{.SUBCKT} statement begins a subcircuit
definition by giving its name, the number and order of its nodes and the names
and default parameters that direct its behavior.  The \texttt{.ENDS} statement
signifies the end of the subcircuit definition. See Section~\ref{XDevicesection}
for more information on using subcircuits with the \texttt{X} device.

%%%
%%% Subcircuit
%%%

\begin{Command}

\format
\begin{alltt}
.SUBCKT <name> [node]*
+ [PARAMS:] [<name>=<value>]* 
\ldots
.ENDS
\end{alltt}

\examples
\begin{alltt}
.SUBCKT OPAMP 10 12 111 112 13
\ldots
.ENDS

.SUBCKT FILTER1 INPUT OUTPUT PARAMS: CENTER=200kHz,
+ BANDWIDTH=20kHz
\ldots
.ENDS

.SUBCKT PLRD IN1 IN2 IN3 OUT1
+ PARAMS: MNTYMXDELY=0 IO_LEVEL=1
\ldots
.ENDS

.SUBCKT 74LS01 A B Y
+ PARAMS: MNTYMXDELY=0 IO_LEVEL=1
\ldots
.ENDS
\end{alltt}

\arguments

\begin{Arguments}
\argument{name}
The \index{subcircuit!name} name used to reference a subcircuit.

\argument{node}
An optional list of nodes. This is not mandatory since it is
feasible to define a subcircuit without any interface nodes.

\argument{PARAMS:}
Optional keyword that precedes the list of subcircuit parameters.
Parameters specified on the subcircuit instance line are treated as
being local to individual subcircuit instances.  
They can be used inside a subcircuit in the same manner as a .param inside the subcircuit definition.
Parameters defined on the instance line override identically named parameters in the subcircuit definition.

\end{Arguments}

\comments
A subcircuit  designation ends\index{subcircuit!designation} with a
\texttt{.ENDS} command. The entire netlist between \texttt{.SUBCKT} and
\texttt{.ENDS} is part of the definition. Each time the subcircuit is called
via an \texttt{X} device, the entire netlist in the subcircuit definition
replaces the \texttt{X} device.

There must be an equal number of nodes in the subcircuit call and in its
definition.  As soon as the subcircuit is called, the actual nodes (those in
the calling statement) substitute for the argument nodes (those in the
defining statement).

Node zero\index{subcircuit!node zero} cannot be used in this node list, as
it is the global ground node.

Subcircuit references may be nested\index{subcircuit!nesting} to any
level.  Subcircuits definitions may also be nested;
a \texttt{.SUBCKT} statement and its closing \texttt{.ENDS} may
appear between another \texttt{.SUBCKT}/\texttt{.ENDS} pair.  A
subcircuit defined inside another subcircuit definition is local to
the outer subcircuit and may not be used at higher levels of the
circuit netlist.

Subcircuits should include only device instantiations and possibly these
statements:
\begin{XyceItemize}
\item \texttt{.MODEL} (model definition)
\item \texttt{.PARAM} (parameter)
\item \texttt{.FUNC} (function)
\end{XyceItemize}

\index{\texttt{.MODEL}!subcircuit scoping}\index{\texttt{.PARAM}!subcircuit scoping}\index{\texttt{.FUNC}!subcircuit scoping}
Models, parameters, and functions defined within a subcircuit are
scoped\index{subcircuit!scoping} to that definition.  That is they are only
accessible within the subcircuit definition in which they are included.
Further, if a \texttt{.MODEL}, \texttt{.PARAM} or a \texttt{.FUNC} statement
is included in the main circuit netlist, it is accessible from the main
circuit as well as all subcircuits.

Node, device, and model names are scoped\index{subcircuit!scoping} to the
subcircuit in which they are defined.  It is allowable to use a name in a
subcircuit that has been previously used in the main circuit netlist. When
the subcircuit is flattened (expanded into the main netlist), all of its
names are given a prefix via the subcircuit instance name.  For example,
\texttt{Q17} becomes \texttt{X3:Q17} after expansion.  After expansion, all
names are unique.  The single exception occurs in the use of global node
names, which are not expanded.

Additional illustative examples of scoping are given in the
``Working with Subcircuits and Models'' section of the \Xyce{} Users' 
Guide\UsersGuide.  Those examples apply to models and functions also.

\end{Command}


%%%%%%%%%%%%%%%%%%%%%%%%%%%%%%%%%%%%%%%%%%%%%%%%%%%%%%%%%%%%%%%%%%%%%%%%%%%%%%%%
\newpage
\subsection{\texttt{.TRAN} (Transient Analysis)}
% Sandia National Laboratories is a multimission laboratory managed and
% operated by National Technology & Engineering Solutions of Sandia, LLC, a
% wholly owned subsidiary of Honeywell International Inc., for the U.S.
% Department of Energy’s National Nuclear Security Administration under
% contract DE-NA0003525.

% Copyright 2002-2024 National Technology & Engineering Solutions of Sandia,
% LLC (NTESS).


\index{\texttt{.TRAN}}
\index{analysis!transient} \index{transient analysis}
Calculates the time-domain response of a circuit for a specified duration.

\begin{Command}

\format
\begin{alltt}
.TRAN <initial step value> <final time value>
+ [<start time value> [<step ceiling value>]] [NOOP] [UIC]
+ [\{schedule( <time>, <maximum time step>, ... )\}]
\end{alltt}

\examples
\begin{alltt}
.TRAN 1us 100ms
.TRAN 1ms 100ms 0ms .1ms
.TRAN 0 2.0e-3  \{schedule( 0.5e-3, 0, 1.0e-3, 1.0e-6, 2.0e-3, 0 )\}

.param initialStep=1ms, tstop=100ms, tstart=0ms, dtmax=0.1ms
.TRAN \{initialStep\} \{tstop\} \{tstart\} \{dtmax\}
\end{alltt}

\arguments

\begin{Arguments}

\argument{initial step value}

Used to calculate the initial time step (see below).

\argument{final time value}

Sets the end time (duration) for the analysis.

\argument{start time value}

Sets the time at which output of the simulation results is to begin.  Defaults to zero.

\argument{step ceiling value}

Sets a maximum time step.  Defaults to ((final time value)-(start time
value))/10, unless there are breakpoints (see below).

\argument{NOOP or UIC}

These two options are synonyms which specify that no operating point
calculation is to be performed, and that the specified initial
condition (from .IC lines or capacitor ``IC'' parameters) should be
used as the transient initial condition instead. Unspecified values
are set to zero.  Finally, the .IC capability can only set voltage
values, not current values.

\argument{schedule(<time>, <maximum time step>, \ldots)}

Specifies a schedule for maximum allowed time steps. The list of
arguments, $t_0$, $\Delta t_0$, $t_1$, $\Delta t_1$, \emph{ etc.\/}
implies that a maximum time step of $\Delta t_0$ will be used while the
simulation time is greater than or equal to $t_0$ and less than $t_1$.
A maximum time step of $\Delta t_1$ will be used when the simulation
time is greater or equal to than $t_1$ and less than $t_2$. This
sequence will continue for all pairs of $t_i$, $\Delta t_i$ that are
given in the \{schedule()\}.  If $\Delta t$ is zero or negative, then no
maximum time step is enforced (other than hardware limits of the host
computer).

\end{Arguments}

\comments

The transient analysis calculates the circuit's response over an
interval of time beginning with \texttt{TIME=0} and finishing at
\texttt{<final time value>}. Use a \texttt{.PRINT}
(print)\index{\texttt{.PRINT}}\index{results!print}\index{\texttt{.PRINT}!\texttt{TRAN}}
statement to get the results of the transient analysis.

Before calculating the transient response \Xyce{} computes a bias
point\index{bias point} for the circuit that is different from the
regular bias point. This is necessary because at the start of a
transient analysis\index{analysis!transient}, the independent sources
can have different values than their DC values. Specifying \texttt{NOOP}
on the \texttt{.TRAN} line causes \Xyce{} to begin the transient
analysis without performing the usual bias point calculation.

The time integration\index{solvers!time integration} algorithms within
\Xyce{} use adaptive time-stepping methods that adjust the time-step
size\index{time step!size} according to the activity in the
analysis. The default ceiling for the internal time step is
\texttt{(<final time value>-<start time value>)/10}.  This default
ceiling value is automatically adjusted if breakpoints are present, to
ensure that there are always at least 10 time steps between breakpoints.
If the user specifies a ceiling value, however, it overrides any
internally generated ceiling values.

\Xyce{} is not strictly compatible with SPICE in its use of the values on the
\texttt{.TRAN} line. In SPICE, the first number on the \texttt{.TRAN} line
specifies the printing interval. In \Xyce{}, the first number is the
\texttt{<initial step value>}, which is used in determining the initial step
size.  The actual initial step size is chosen to be the smallest of three
quantities: the \texttt{<inital step value>}, the \texttt{<step ceiling
value>}, or 1/200th of the time until the next breakpoint.

The third argument to \texttt{.TRAN} simply determines the earliest time
for which results are to be output.  Simulation of the circuit always
begins at \texttt{TIME=0} irrespective of the setting of \texttt{<start time value>}.

\end{Command}


%%%%%%%%%%%%%%%%%%%%%%%%%%%%%%%%%%%%%%%%%%%%%%%%%%%%%%%%%%%%%%%%%%%%%%%%%%%%%%%%
\newpage
\subsection{Miscellaneous Commands}
\subsubsection{\texttt{*} (Comment)}
\index{netlist!comment} A netlist comment line.  Whitespace at
the beginning of a line is also interpreted as a comment unless it
is followed by a \verb|+| symbol, in which case it treats the line as a continuation.

\subsubsection{\texttt{;} (In-line Comment)}
\index{netlist!in-line comment} Add a netlist in-line comment.

\subsubsection{\texttt{+} (Line Continuation)}
\index{netlist!line continuation} Continue the text of the previous line.

\newpage
\section{Expressions}
\label{ExpressionDocumentation}
\Xyce{} supports use of mathematical expressions in several contexts:
\begin{XyceItemize}
\item for the values of device instance and model parameters.
\item in definition of parameters in \texttt{.PARAM} and \texttt{.GLOBAL\_PARAM} statements.
\item for output on \texttt{.PRINT} lines.
\end{XyceItemize}

In all contexts where expressions are allowed, it is best practice to enclose them
in curly braces (\{\}).  For netlist compatibility with other simulators,
expressions may be enclosed in single quotation marks instead (\texttt{'}). Also, in
some circumstances (such as \texttt{.param} and \texttt{.global\_param} expressions), \Xyce{} will accept 
expressions that are not surrounded by either braces or single quotes. 
However, it is recommended that the braces be used in netlists written specifically for \Xyce{}, as this will
be the most reliable option.

The expression package in \Xyce{} supports all standard arithmetic
operators, trigonometric functions, a collection of arithmetic
functions, and some functions to mimic the pulse, sine, exp, and sffm
time-dependent functions in the independent current and voltage sources.
These functions are listed in tables \ref{Arithmetic_Functions} and
\ref{SPICE_Functions}.

\paragraph{Operators}
\index{netlist!expression operators}
\index{expressions!operators}
% Sandia National Laboratories is a multimission laboratory managed and
% operated by National Technology & Engineering Solutions of Sandia, LLC, a
% wholly owned subsidiary of Honeywell International Inc., for the U.S.
% Department of Energy’s National Nuclear Security Administration under
% contract DE-NA0003525.

% Copyright 2002-2023 National Technology & Engineering Solutions of Sandia,
% LLC (NTESS).

{\renewcommand{\arraystretch}{1.2}
  \newcommand{\category}[1]{\multicolumn{3}{c}{\smallskip\color{XyceDarkBlue}\em\bfseries #1}}
  \begin{longtable}{>{\raggedright\small}m{1in}>{\raggedright\small}m{1in}>{\raggedright\let\\\tabularnewline\small}m{3in}}
    \caption{Operators\label{Expression_Operators}} \\ \hline
    \rowcolor{XyceDarkBlue}
    \color{white}\bf Class of Operator &
    \color{white}\bf Operator & 
    \color{white}\bf Meaning \endfirsthead
    \caption[]{Operators} \\ \hline
    \rowcolor{XyceDarkBlue}
    \color{white}\bf Class of Operator &
    \color{white}\bf Operator & 
    \color{white}\bf Meaning \endhead

    arithmetic
    & \verb|+ | & addition or string concatenation \\ \cline{2-3}
    & \verb|- | & subtraction \\ \cline{2-3}
    & \verb|* | & multiplication \\ \cline{2-3}
    & \verb|/ | & division \\ \cline{2-3}
    & \verb|**| & exponentiation \\ \cline{2-3}
    & \verb|% | & modulus \\ \hline

    logical\footnotemark
    & \verb|~| & unary NOT \\ \cline{2-3}
    & \verb+|+ & boolean OR \\ \cline{2-3}
    & \verb|^| & boolean XOR \\ \cline{2-3}
    & \verb|&| & boolean AND \\ \hline

    relational
    & \verb|==| & equality \\ \cline{2-3}
    & \verb|!=| & non-equality \\ \cline{2-3}
    & \verb|> | & greater-than \\ \cline{2-3}
    & \verb|>=| & greater-than or equal \\ \cline{2-3}
    & \verb|< | & less-than \\ \cline{2-3}
    & \verb|<=| & less-than or equal \\ \hline

    conditional & \verb|? :| & Ternary conditional operator \\ \hline
  \end{longtable}
\footnotetext{Logical and relational operators are used only with the \texttt{IF()} function and the ternary operator for its conditional argument.}
}



\paragraph{Special note on ternary operator}
Note that the ternary operator is available for use in \Xyce{}.  This
operator is the same as the ternary conditional operator in C, C++,
Perl, and others.  The ternary expression \texttt{t?a : b} is
equivalent to the function \texttt{IF(t,a,b)} described below.
However, please be aware that the ternary operator has extremely low
precedence just as it has in these other languages, and if parentheses
are not used to make explicit which expressions are supposed to be
part of the condition or true and false values, the resolution of the
expression may be surprising.

For example, the expression
\begin{alltt}
  1+a==b?1:0+1
\end{alltt}
is equivalent to the expression
\begin{alltt}
  IF(1+a==b,1,0+1)
\end{alltt}
because the ``+'' and ``=='' operators have higher precedence than
either ``?'' or ``:''.  Similarly:
\begin{alltt}
  A==B?1:0 + A==C?2:0 + A==D?3:0
\end{alltt}
is equivalent to
\begin{alltt}
  if(A==B,1,IF(0 + A==C,2,IF(0 + A==D,3,0)))
\end{alltt}
Given the way the original expression is written, it appears that the
intent was that the expression be evaluated as:
\begin{alltt}
  If(A==B,1,0) + IF(A==C,2,0) + IF(A==D,3,0)
\end{alltt}
This is not how the expression will be evaluated.  Fortunately,
because of the use of ``0'' to the right of the colons in each case,
the expression just happens to give the desired result in either
interpretation, but Xyce is using the nested IF equivalent.

Finally, due to restrictions on the expression parser, {\bf it is essential
that ternary operators never be written so that a bare parameter is
directly to the left of a colon}.  This is because colons are actually
legal characters in parameters --- the colon represents hierarchy, so
that \texttt{R1:R} means the R parameter of device R1, and
\texttt{X1:A} refers to the node A of subcircuit X1.  Therefore, it is
necessary to put at least one character that is invalid in parameter names
in between the colon and the parameter.  It is sufficient to use a space.
\begin{alltt}
  \color{red}  \{(A==B)?C:D\} ; this expression will generate a syntax error
    \{(A==B)?C :D\} ; this expression is acceptable
    \{(A==B)?C+0:D\} ; this expression is acceptable
    \{(A==B)?(C):D\} ; this expression is acceptable
\end{alltt}
Note that if using expressions without curly braces or single quotes 
(something that is allowed on \texttt{.param} and \texttt{.global\_param} lines)
this becomes even more restrictive because the parser eats up whitespace.  In
this case, the expression needs to be fixed another way, such as with parenthesis.
\begin{alltt}
  \color{red}  (A==B)?C:D ; this expression will generate a syntax error
    (A==B)?C :D ; parser eliminates whitespace, causing syntax error
    (A==B)?C+0:D ; this expression is acceptable
    (A==B)?(C):D ; this expression is acceptable
\end{alltt}

\paragraph{Arithmetic Functions}
% Sandia National Laboratories is a multimission laboratory managed and
% operated by National Technology & Engineering Solutions of Sandia, LLC, a
% wholly owned subsidiary of Honeywell International Inc., for the U.S.
% Department of Energy’s National Nuclear Security Administration under
% contract DE-NA0003525.

% Copyright 2002-2022 National Technology & Engineering Solutions of Sandia,
% LLC (NTESS).

{\renewcommand{\arraystretch}{1.2}
  \newcommand{\category}[1]{\multicolumn{3}{c}{\smallskip\color{XyceDarkBlue}\em\bfseries #1}}
  \begin{longtable}{>{\raggedright\small}m{1in}>{\raggedright\small}m{2in}>{\raggedright\let\\\tabularnewline\small}m{2in}}
    \caption{Arithmetic Functions\label{Arithmetic_Functions}} \\ \hline
    \rowcolor{XyceDarkBlue}
    \color{white}\bf Function &
    \color{white}\bf Meaning & 
    \color{white}\bf Explanation \endfirsthead
    \caption[]{Arithmetic Functions} \\ \hline
    \rowcolor{XyceDarkBlue}
    \color{white}\bf Function &
    \color{white}\bf Meaning & 
    \color{white}\bf Explanation \endhead

    \category{Arithmetic functions} \\ \hline

    ABS(x) & $|x|$ & absolute value of $x$ \\ \hline


    CEIL(x) & $\lceil x \rceil$ & least integer greater or equal to variable $x$  \\ \hline

    DDT(x)
    &  $\frac{d}{dt} x(t)$ & time derivative of $x$ \\ \hline

    DDX(f(x),x)
    &  $\frac{\partial}{\partial x} f(x)$ & partial derivative of $f(x)$ with respect to $x$ \\ \hline

    FLOOR(x) & $\lfloor x \rfloor$ & greatest integer less than or equal to variable $x$  \\ \hline

    FMOD(x,y) &   & returns the remainder of x/y as a real number \\ \hline

    IF(t,x,y)
    & $x$ if $t$ is true, & $t$ is an expression using the relational
    operators in Table~\ref{Expression_Operators}.\footnotemark[2]\\
    & $y$ otherwise       & \\ \hline
    
    INT(x) & $\mathrm{sgn}(x)\lfloor |x|\rfloor$ & integer part of the real 
    variable $x$  \\ \hline

    LIMIT(x,y,z)
    & $y$ if $x < y$ & \\
    & $x$ if $y < x < z$  & $x$ limited to range $y$ to $z$ \\
    & $z$ if $x > z$ & \\ \hline
    
%% ----- COMMENTED OUT LIMIT
%% NOTE:  LIMIT is indeed defined in Xyce as described below, but this is
%% completely useless and completely incompatible with the intended use of
%% that feature.  LIMIT is supposed to be a random number generator that 
%% generates numbers from the probability function described in the 
%% right-hand column, but as implemented in Xyce, returns just the sum of the 
%% two arguments.  The intended application is Monte Carlo simulation, and as 
%% implemented, this is useless.
%% ----------------------------
%%    LIMIT(x,y) & $x+y$ & upper limit of the random variable $x$ with 
%%    probability mass function $p(z)$ given by: \\ 
%%    & & \hspace*{1in}${p(z=x+y) = 1/2}$ \\
%%    & & \hspace*{1in}${p(z=x-y) = 1/2}$ \\ \hline
    
    M(x) & $|x|$ & absolute value or magnitude of $x$ \\ \hline

    MIN(x,y) & $\min(x,y)$ & minimum of $x$ and $y$  \\ \hline

    MAX(x,y) & $\max(x,y)$ & maximum of $x$ and $y$  \\ \hline

    NINT(x) &            & rounds $x$ up or down, to the nearest integer  \\ \hline

    PWR(x,y) & $x^{y}$ & $x$ raised to $y$ power  \\ \hline
    POW(x,y) & $x^{y}$ & $x$ raised to $y$ power  \\ \hline
    PWRS(x,y)
    & $x^{y}$ if $x > 0$ & \\
    & 0 if $x = 0$ & sign corrected $x$ raised to $y$ power  \\
    & $-(-x)^{y}$ if $x < 0$ & \\ \hline

    SDT(x)
    & $\int x(t)  dt$ & time integral of $x$ \\ \hline
    
    SGN(x)
    & +1 if $x > 0$ & \\
    & 0 if $x = 0$  & sign value of $x$\\
    & -1 if $x < 0$ & \\ \hline

    SIGN(x,y) & $\mathrm{sgn}(y)|x|$ & sign of $y$ times absolute value of $x$
    \\ \hline

    STP(x)
    & 1 if $x > 0$ & step function \\
    & 0 otherwise &                                      \\ \hline

    SQRT(x) & $\sqrt{x}$ & square root of $x$ \\ \hline

    URAMP(x)
    & $x$ if $x > 0$ & ramp function \\
    & 0 otherwise &  \\ \hline

    \category{Operators related to interpolating tabular data} \\ \hline

    TABLE(x,y,z,*)
    & $f(x)$ where $f(y)=z$ & piecewise linear interpolation, multiple ($y$,$z$) pairs can be specified \\ \hline

    TABLE(``filename'') &  
    & \texttt{filename} of $x,y$ data pairs, one pair per line, space separated, is 
      read from disk and put into a \texttt{TABLE} for use as function lookup table. Synonymous with TABLEFILE. \\ \hline

    FASTTABLE(x,y,z,*)
    & $f(x)$ where $f(y)=z$ & piecewise linear interpolation without breakpoints, multiple ($y$,$z$) pairs can be specified \\ \hline

    FASTTABLE(``filename'') &
    & \texttt{filename} of $x,y$ data pairs, one pair per line, space separated, is 
      read from disk and put into a \texttt{FASTTABLE} for use as breakpoint-free function lookup table. \\ \hline

    SPLINE(x,y,z,*)
    & $f(x)$ where $f(y)=z$ & Akima spline interpolation, multiple ($y$,$z$) pairs should be specified \\ \hline

    SPLINE(``filename'') &
    & \texttt{filename} of $x,y$ data pairs, one pair per line, space separated, is 
      read from disk and put into a \texttt{SPLINE} for use in an Akima spline. \\ \hline

    AKIMA(x,y,z,*)
    & $f(x)$ where $f(y)=z$ & Akima spline interpolation~\cite{10.1145/321607.321609}, multiple ($y$,$z$) pairs should be specified (synonymous with SPLINE) \\ \hline

    AKIMA(``filename'') &
    & \texttt{filename} of $x,y$ data pairs, one pair per line, space separated, is 
      read from disk and put into a \texttt{AKIMA} for use in an Akima spline. \\ \hline

    CUBIC(x,y,z,*)
    & $f(x)$ where $f(y)=z$ & Cubic spline interpolation, multiple ($y$,$z$) pairs should be specified \\ \hline

    CUBIC(``filename'') &
    & \texttt{filename} of $x,y$ data pairs, one pair per line, space separated, is 
      read from disk and put into a \texttt{CUBIC} for use in an Cubic spline. \\ \hline

    WODICKA(x,y,z,*)
    & $f(x)$ where $f(y)=z$ & Wodicka spline interpolation~\cite{Engeln1996}, multiple ($y$,$z$) should be specified \\ \hline

    WODICKA(``filename'') &
    & \texttt{filename} of $x,y$ data pairs, one pair per line, space separated, is 
      read from disk and put into a \texttt{WODICKA} for use in a Wodicka spline. \\ \hline

    BLI(x,y,z,*)
    & $f(x)$ where $f(y)=z$ & Barycentric Lagrange interpolation~\cite{Berrut_barycentriclagrange}, multiple ($y$,$z$) pairs should be specified \\ \hline

    BLI(``filename'') &
    & \texttt{filename} of $x,y$ data pairs, one pair per line, space separated, is 
      read from disk and put into a \texttt{BLI} for use in Barycentric Lagrance Interpolation. \\ \hline
    
    TABLEFILE(``filename'') &  
    & \texttt{filename} of $x,y$ data pairs, one pair per line, space separated, is 
      read from disk and put into a \texttt{TABLE} for use as function lookup table. \\ \hline

    \category{Operators related to complex numbers} \\ \hline

    DB (x) &     & output the magnitude of $x$ in decibels \\ \hline

    IMG(x) &     & imaginary part of variable $x$ \\ \hline

    PH(x) &  & phase of variable $x$ \\ \hline 

    R(x) &      & real part of variable $x$ \\ \hline

    RE(x) &      & real part of variable $x$ \\ \hline

    \category{Exponential, logarithmic, and trigonometric functions} \\ \hline

    ACOS(x) & $\arccos(x)$ & result in radians \\ \hline

    ACOSH(x) & $\cosh^{-1}(x)$ & hyperbolic arccosine of $x$ \\ \hline

    ARCTAN(x) & $\arctan(x)$ & result in radians \\ \hline

    ASIN(x) & $\arcsin(x)$ & result in radians \\ \hline

    ASINH(x) & $\sinh^{-1}(x)$ & hyperbolic arcsine of $x$ \\ \hline

    ATAN(x) & $\arctan(x)$ & result in radians \\ \hline

    ATANH(x) & $\tanh^{-1}(x)$ & hyperbolic arctangent of $x$ \\ \hline

    ATAN2(x,y) & $\arctan(x/y)$ & result in radians \\ \hline

    COS(x) & $\cos(x)$ & $x$ in radians \\ \hline

    COSH(x) & $\cosh(x)$ & hyperbolic cosine of $x$ \\ \hline

    EXP(x) & $e^{x}$ & $e$ to the $x$ power \\ \hline

    LN(x) & $\ln(x)$ & log base $e$ \\ \hline

    LOG(x) & $\log(x)$ & log base $10$ \\ \hline

    LOG10(x) & $\log(x)$ & log base $10$ \\ \hline

    SIN(x) & $\sin(x)$ & $x$ in radians \\ \hline

    SINH(x) & $\sinh(x)$ & hyperbolic sine of $x$ \\ \hline

    TAN(x) & $\tan(x)$ & $x$ in radians \\ \hline

    TANH(x) & $\tanh(x)$ & hyperbolic tangent of $x$ \\ \hline

    \category{Operators related to random distributions} \\ \hline

    %AGAUSS($\mu$,$\alpha$,$n$) & $\mu-\alpha < result < \mu+\alpha$ & Random number sampled from normal distribution with mean $\mu$ and standard deviation $\alpha/n$ \\ 
    AGAUSS($\mu$,$\alpha$,$n$) &  & Random number sampled from normal distribution with mean $\mu$ and standard deviation $\alpha/n$ \\ 
    %& & The number returned will differ from the mean by at most $\alpha$ \\
    & & A deviation $\alpha$ will be $n$ standard deviations from the mean. The argument $n$ is optional, and defaults to 1.\footnotemark[1]  \\\hline 

    %GAUSS($\mu$,$\alpha$,$n$) & $\mu*(1-\alpha) < result < \mu*(1+\alpha)$ & Random number sampled from normal distribution with mean $\mu$ and standard deviation $(\alpha*\mu)/n$ \\ 
    GAUSS($\mu$,$\alpha$,$n$) &  & Random number sampled from normal distribution with mean $\mu$ and standard deviation $(\alpha*\mu)/n$ \\ 
    %& & The number returned will differ from the mean by at most $\alpha*\mu$ \\
    & & A deviation $\alpha*\mu$ will be $n$ standard deviations from the mean.  The argument $n$ is optional, and defaults to 1.\footnotemark[1]  \\\hline 

    %AUNIF($\mu$,$\alpha$) & $\mu-\alpha < result < \mu+\alpha$ & Random number sampled from uniform distribution with mean $\mu$ and standard deviation $\alpha/n$ \\ 
    AUNIF($\mu$,$\alpha$) &  & Random number sampled from uniform distribution with mean $\mu$ and standard deviation $\alpha/n$ \\ 
    & & The number returned will differ from the mean by at most $\alpha$ \footnotemark[1]  \\\hline 

    %UNIF($\mu$,$\alpha$) & $\mu*(1-\alpha) < result < \mu*(1+\alpha)$ & Random number sampled from uniform distribution with mean $\mu$ and standard deviation $(\alpha*\mu)/n$ \\ 
    UNIF($\mu$,$\alpha$) &  & Random number sampled from uniform distribution with mean $\mu$ and standard deviation $(\alpha*\mu)/n$ \\ 
    & & The number returned will differ from the mean by at most $\alpha*\mu$ \footnotemark[1]  \\\hline 

    RAND() & & random number between 0 and 1 sampled from a uniform distribution\footnotemark[1] \\ \hline

  \end{longtable}
\footnotetext[1]{The default behavior of the random number functions \texttt{RAND}, \texttt{GAUSS},
  and \texttt{AGAUSS}, if there are not any UQ commands such as \texttt{.SAMPLING} in the netlist, 
  is to return the mean value of the operator.  If a UQ command is present, then these operators 
  can be used to define the distribution of random inputs to the UQ analysis.  However, this
  will only happen if the UQ analysis specifically requests it using the command \texttt{.SAMPLING USEEXPR=TRUE}
  Unless a specific random seed is specified using either the \texttt{-randseed} command line option, or from the netlist, 
  the random number generator will be seeded internally.  In all cases, \Xyce{} will output text to the 
  console indicating what seed is being used.}  
\footnotetext[2]{Use of the \texttt{IF} function to create an expression that has step-function-like behavior as a function of a solution variable is highly likely to produce convergence errors in simulation.  \texttt{IF} statements that have step-like behavior with an explicit time dependence are the exception, as the code will insert breakpoints at the discontinuities.  Do not use step-function or other infinite-slope transitions dependent on variables other than time. Smooth the transition so that it is more easily integrated through. See the ``Analog Behavioral Modeling'' chapter of the \Xyce{} Users' Guide~\UsersGuide{} for guidance on using the \texttt{IF} function with the B-source device.}
}


\newpage
\paragraph{Spice Compatable Functions}
\index{netlist!functions}
\index{expressions!SPICE functions}
Expressions can also use functions that are the equivalent of standard sources.
% Sandia National Laboratories is a multimission laboratory managed and
% operated by National Technology & Engineering Solutions of Sandia, LLC, a
% wholly owned subsidiary of Honeywell International Inc., for the U.S.
% Department of Energy’s National Nuclear Security Administration under
% contract DE-NA0003525.

% Copyright 2002-2023 National Technology & Engineering Solutions of Sandia,
% LLC (NTESS).

{\renewcommand{\arraystretch}{1.2}
  \newcommand{\category}[1]{\multicolumn{2}{c}{\smallskip\color{XyceDarkBlue}\em\bfseries #1}}
  \begin{longtable}{>{\raggedright\small}m{2.5in}>{\raggedright\let\\\tabularnewline\small}m{2.5in}}
    \caption{SPICE Compatibility Functions\label{SPICE_Functions}} \\ \hline
    \rowcolor{XyceDarkBlue}
    \color{white}\bf Function &
    \color{white}\bf Explanation \endfirsthead
    \caption[]{Arithmetic Functions} \\ \hline
    \rowcolor{XyceDarkBlue}
    \color{white}\bf Function &
    \color{white}\bf Explanation \endhead

    SPICE\_EXP(V1,V2,TD1,TAU1,TD2,TAU2)  & SPICE style transient exponential \\
     & V1 = initial value \\
     & V2 = pulsed value \\
     & TD1 = rise delay time \\
     & TAU1 = rise time constant \\
     & TD2 = fall delay time \\
     & TAU2 = fall time constant \\ \hline

    SPICE\_PULSE(V1,V2,TD,TR,TF,PW,PER) & SPICE style transient pulse \\
     & V1 = initial value \\
     & V2 = pulsed value \\
     & TD = delay \\
     & TR = rise time \\
     & TF = fall time \\
     & PW = pulse width \\
     & PER = period \\ \hline

    SPICE\_SFFM(V0,VA,FC,MDI,FS) & SPICE style transient single frequency FM \\
     & V0 = offset \\
     & VA = amplitude \\
     & FC = carrier frequency \\
     & MDI = modulation index \\
     & FS = signal frequency \\ \hline

    SPICE\_SIN(V0,VA,FREQ,TD,THETA) & SPICE style transient sine wave \\
     & V0 = offset \\
     & VA = amplitude \\
     & FREQ = frequency (hz) \\
     & TD = delay \\
     & THETA = damping factor \\ \hline

  \end{longtable}
}


Information about the restrictions on expressions in specific contexts
is given in the subsections that follow.

\subsection{Expressions in \texttt{.PARAM} or \texttt{.GLOBAL\_PARAM} Statements}

Parameters defined using \texttt{.PARAM} and \texttt{.GLOBAL\_PARAM} are 
mostly synonymous and subject to most of the same constraints.
Both types are 
allowed to depend on parameters defined in \texttt{.PARAM} or
\texttt{.GLOBAL\_PARAM} statements, and both may contain special variables
such as \textrm{TIME}, \textrm{FREQ}, \textrm{TEMP} or \textrm{VT}.  Neither type
may contain any references to solution variables or lead currents.

\Example{.PARAM SQUARES=5.0}
\Example{.PARAM SHEETRES=25}
\Example{.PARAM RESISTANCE=\{SQUARES*SHEETRES\}}
\Example{.PARAM a0 = 1.0+2.0J}

\Example{.PARAM dTdt=.01}
\Example{.GLOBAL\_PARAM Temperature=\{27+dTdt*TIME\}}

Both types of parameter are subject to extra constraints if used inside of subcircuits.  
\texttt{.PARAM} parameters may be defined inside of a subcircuit, while \texttt{.GLOBAL\_PARAM} 
parameters may not.  Parameters inside of subcircuits may not be swept by 
analysis commands such as \texttt{.STEP} or \texttt{.SAMPLING}.  Only globally 
scoped user-defined parameters can be swept in this manner.  Note, this restriction 
does not apply to device parameters, just user-defined parameters.

Parameters can be complex-valued.  The suffix letter used for the imaginary 
part of a complex number is the letter \texttt{J}.  For example, in the parameter statement 
\texttt{.param a0=1.0+2.0J}, the parameter \texttt{a0} has \texttt{1.0} for the real part and 
\texttt{2.0} is the imaginary part.  

Although parameters are allowed
to be complex, many uses of parameters are inherently real-valued.  Most 
device parameters, for example, are assumed to be double precision numbers. So,
if a real-valued device parameter is set equal to a complex-valued \texttt{.param}, 
the device parameter will use the real part and the imaginary part
will be ignored.

\subsection{Expressions in \texttt{.PRINT} Lines}

Expressions on \texttt{.PRINT} lines may contain references to
parameters defined in either \texttt{.PARAM} or
\texttt{.GLOBAL\_PARAM} statements, device parameters using the syntax
\texttt{<device name>:<parameter name>}, and may also contain solution
variables.

\begin{centering}
\shadowbox{
\begin{minipage}{0.8\textwidth}
\begin{vquote}
\color{blue}*example with .print expressions\color{black}
.PARAM RES=50
R1 1 0 \{RES\}
V1 1 0 sin(0 5 100khz)
.tran 1u 1m
*Print power dissipated through resistor,
*and actual resistance used in the R1
*device
.print tran \{V(1)*V(1)/RES\} \{R1:R\}
.end
\end{vquote}
\end{minipage}
}
\end{centering}

\subsection{Using Complex Values in Expressions}
\label{ComplexExpressions}
The \Xyce{} expression library was written to work with complex
quantities, and can perform complex arithmetic on any set of 
complex inputs to produce a complex output.
However, the default behavior on the \texttt{.PRINT} line depends 
on the type of analysis being run.  

If running a frequency domain analysis such as \texttt{.AC} or \texttt{.HB}, then outputs of complex valued expressions will automatically include both the real and imaginary part.  For example:
\begin{vquote}
.param r0=\{log(-1)\}
.print ac \{r0\} // automatically output real and imaginary parts
\end{vquote}


If running a real-valued analysis such as \texttt{.DC} or \texttt{.TRAN}, the output on the \texttt{.PRINT} line will, by default, only include the real part.  This is true even if the expression evaluates to a complex number.
If the user is running a real-valued analysis, but desires output of both the real and imaginary part, it is necessary to use operators such as \texttt{Re()} and \texttt{Img()}.  Here are some examples.


\begin{vquote}
.param r0=\{log(-1)\}
.print tran \{r0\} // only output the real part of \{r0\}
.print tran  \{re(r0)\} \{img(r0)\} // output both real and imaginary parts 
\end{vquote}

Note any complex-valued expression will internally be evaluaed using complex math.
The only thing different for \texttt{ \{r0\} } in the above examples is the default outputs for \texttt{.AC} and \texttt{.TRAN}.

\subsection{Expressions for Device Instance and Model Parameters}

Expressions of constants, \texttt{.PARAM} and \texttt{.GLOBAL\_PARAM} parameters may be used
for the values of any device parameters in instance and model lines.

Except in very specific devices, expressions used for device parameter
values must evaluate to a time-independent constant, and must not
contain dependence on solution variables such as nodal voltages or
currents.  

\begin{centering}
\shadowbox{
\begin{minipage}{0.8\textwidth}
\begin{vquote}
\color{blue}*example of use of expressions for device parameters \color{black}
.PARAM RES=50
.GLOBAL\_PARAM theSaturationCurrent=1.5e-14
R1 1 0 \{RES\}
V1 1 0 sin(0 5 100khz)
D1 1 0 DMODEL
.MODEL D DMODEL IS={theSaturationCurrent}

.step theSaturationCurrent 1e-14 5e-14 1e-14
\end{vquote}
\end{minipage}
}
\end{centering}

Some parameters of specific devices are exceptions to the general
rule.  These parameters have no restrictions and may depend on any
parameters, time, or solution variables in the
netlist:
\begin{XyceItemize}
\item The \textrm{V} or \textrm{I} instance parameters of the
  \texttt{B} source.
\item The \textrm{CONTROL} instance parameter of the switch (\texttt{S}
  device).
\item The \textrm{C} (capacitance) instance parameter for the capacitor.
\item The \textrm{Q} (charge) instance parameter for the capacitor.
\item The coupling coefficient instance parameter for the {\em LINEAR}
  mutual inductor (\texttt{K} device with no model card specified)
\end{XyceItemize}

These specific instance parameters may be time-depdendent (i.e. they
may reference the \texttt{TIME} special variable), but may not depend
on any solution variables:

\begin{XyceItemize}
\item The \textrm{TEMP} instance parameter of all devices.
\item The \textrm{L} (inductance) parameter of the inductor.
\item The \textrm{R} (resistance) parameter of the resistor.
\item The \textrm{R}, \textrm{RESISTIVITY}, \textrm{DENSITY}, \textrm{HEATCAPACITY} and \textrm{THERMAL\_HEATCAPACITY} parameters of the thermal resistor (resistor level 2).
\end{XyceItemize}
%\newpage

%%
%% PSpice POLY
%\clearpage
\subsection{POLY expressions}
\label{PspicePoly}
\index{analog behavioral modeling!POLY}
\index{analog behavioral modeling!polynomial expression}
\index{POLY}
% Sandia National Laboratories is a multimission laboratory managed and
% operated by National Technology & Engineering Solutions of Sandia, LLC, a
% wholly owned subsidiary of Honeywell International Inc., for the U.S.
% Department of Energy’s National Nuclear Security Administration under
% contract DE-NA0003525.

% Copyright 2002-2024 National Technology & Engineering Solutions of Sandia,
% LLC (NTESS).


%%-------------------------------------------------------------------------
%% Purpose        : Main LaTeX Xyce Reference Guide
%% Special Notes  : Graphic files (pdf format) work with pdflatex.  To use
%%                  LaTeX, we need to use postcript versions.
%% Creator        : Scott A. Hutchinson, Computational Sciences, SNL
%% Creation Date  : {05/23/2002}
%%
%%-------------------------------------------------------------------------

The \texttt{POLY} keyword is available in the \texttt{E}, \texttt{F},
\texttt{G}, \texttt{H} and \texttt{B} dependent sources. Based on the same
keyword from SPICE2, \texttt{POLY} provides a compact method of specifying
polynomial expressions in which the variables in the polynomial are specified
followed by an ordered list of polynomial coefficients.  All expressions
specified with \texttt{POLY} are ultimately translated by \Xyce{} into an
equivalent, straightforward polynomial expression in a \texttt{B} source.
Since a straightforward polynomial expression can be easier to read, there is
no real benefit to using \texttt{POLY} except to support netlists imported from
other SPICE-based simulators.

There are three different syntax forms for \texttt{POLY}, which can be
a source of confusion.  The \texttt{E} and \texttt{G} sources
(voltage-dependent voltage or current sources) use one form, the
\texttt{F} and \texttt{H} sources (current-dependent voltage or
current sources) use a second form, and the \texttt{B} source (general
nonlinear source) a third form.  During input processing, any of the
\texttt{E,F,G} or \texttt{H} sources that use nonlinear expressions
are first converted into an equivalent \texttt{B} source, and then any
\texttt{B} sources that use the \texttt{POLY} shorthand are further
converted into standard polynomial expressions.  This section describes
how the compact form will be translated into the final form that is
used internally.

All three formats of \texttt{POLY} express the same three components:
a number of variables involved in the expression ($N$, the number in
parentheses after the \texttt{POLY} keyword), the variables
themselves, and an ordered list of coefficients for the polynomial
terms.  Where they differ is in how the variables are expressed.

\subsubsection{Voltage-controlled sources}

The \texttt{E} and \texttt{G} sources are both voltage-controlled, and
so their \texttt{POLY} format requires specification of two nodes for
each voltage on which the source depends, i.e. the positive and
negative nodes from which a voltage drop is computed.  There must
therefore be twice as many nodes as the number of variables specified
in parentheses after the \texttt{POLY} keyword:

\verb|Epoly 1 2 POLY(3) n1p n1m n2p n2m n3p n3m ...|

In this example, the voltage between nodes 1 and 2 is determined by
a polynomial whose variables are \texttt{V(n1p,n1m)},
\texttt{V(n2p,n2m)}, \texttt{V(n3p,n3m)}.  Not shown in this example
are the polynomial coefficients, which will be described later.


\subsubsection{Current-controlled sources}

The \texttt{F} and \texttt{H} sources are both current-controlled, and
so their \texttt{POLY} format requires specification of one voltage
source name for each current on which the source depends.
There must therefore be exactly as many nodes as the number of variables
specified in parentheses after the \texttt{POLY} keyword:

\verb|Fpoly 1 2 POLY(3) V1 V2 V3 ...|

In this example, the voltage between nodes 1 and 2 is determined by
a polynomial whose variables are \texttt{I(V1)},
\texttt{I(V2)}, and \texttt{I(V3)}.  Not shown in this example
are the polynomial coefficients, which will be described later.

\subsubsection{B sources}

Finally, the most general form of \texttt{POLY} is that used in the
general nonlinear dependent source, the \texttt{B} source. In this variant,
each specific variable must be named explicitly (i.e. not simply by node name or
by voltage source name), because currents and voltages may be mixed as
needed.

\verb|Bpoly 1 2 V={POLY(3) I(V1) V(2,3) V(3) ...}|

\verb|Bpoly2 1 2 I={POLY(3) I(V1) V(2,3) V(3) ...}|


In these examples, the source between nodes 1 and 2 is determined by a
polynomial whose variables are \texttt{I(V1)}, \texttt{V(2,3)}, and
\texttt{V(3)}.  In the first example, the polynomial value determines
the voltage between nodes 1 and 2, and in the second the current.

The \texttt{E, F, G} and \texttt{H} formats are all converted internally
in a first step to the \texttt{B} format.  Thus the following pairs of
sources are exactly equivalent:

\verb|Epoly 1 2 POLY(3) n1p n1m n2p n2m n3p n3m ...|

\verb|BEpoly 1 2 V={POLY(3) V(n1p,n1m) V(n2p,n2m) V(n3p,n3m) ...|

\verb|Fpoly 1 2 POLY(3) V1 V2 V3 ...|

\verb|BFpoly 1 2 V={POLY(3) I(V1) I(V2) I(V3) ...|

After conversion to the \texttt{B} source form, the \texttt{POLY} form
is finally converted to a normal expression using the coefficients and
variables given.

Coefficients are given in a standard order, and the polynomial is
built up by terms until the list of coefficients is exhausted.  The
first coefficient is the constant term of the polynomial, followed by
the coefficients of linear terms, then bi-linear, and so on.  For example:

\verb|Epoly 1 2 POLY(3) n1p n1m n2p n2m n3p n3m 1 .5 .5 .5|

In this example, the constant term is 1.0, and the coefficients of the
three terms linear in the input variables are 0.5.  Thus, this
\texttt{E} source is precisely equivalent to the general \texttt{B}
source:

\verb|BEstandard 1 2 V={1.0 + .5*V(n1p,n1m) + .5*V(n2p,n2m) +.5*V(n3p,n3m)}|

The standard ordering for coefficients is:

POLY(N) $X_1 \ldots X_N C_0\ C_1 \ldots C_N\ C_{11} \ldots C_{1N}\ C_{21} \ldots C_{N1} \ldots C_{NN}\ C_{1^21} \ldots  C_{1^2N} \ldots$

with the polynomial then being:

$$Value = C_0 + \sum_{j=1}^{N} C_j X_j + \sum_{i=1}^N\sum_{j=1}^N C_{ij}X_iX_j + \sum_{i=1}^N\sum_{j=1}^N C_{i^2j} X_i^2X_j + \ldots$$

Here we have used the general form $X_i$ for the $i^{th}$ variable,
which may be either a current or voltage variable in the general case.

It should be reiterated that the \texttt{POLY} format is provided primarily for
support of netlists from other simulators, and that its compactness may be a
disadvantage in readability of the netlist and may be more prone to usage
error.  \Xyce{} users are therefore advised that use of the more
straightforward expression format in the \texttt{B} source may be more
appropriate when crafting original netlists for use in \Xyce{}.  Since \Xyce{}
converts \texttt{POLY} format expressions to the simpler format internally,
there is no performance benefit to use of \texttt{POLY}.



%%%%%%%%%%%%%%%%%%%%%%%%%%%%%%%%%%%%%%%%%%%%%%%%%%%%%%%%%%%%%%%%%%%%%%%%%%%%%%%%
%%%%%%%%%%%%%%%%%%%%%%%%%%%%%%%%%%%%%%%%%%%%%%%%%%%%%%%%%%%%%%%%%%%%%%%%%%%%%%%%

\newpage
\section{Devices}
\label{Analog_Devices}
\index{device!analog}

\Xyce{} supports many devices, with an emphasis on analog devices, including
sources, subcircuits and behavioral models.  This section serves as a reference
for the devices supported by \Xyce{}.  Each device is described separately and
includes the following information, if applicable:
\begin{XyceItemize}
\item a description and an example of the correct netlist syntax.
\item the matching model types and their description.
\item the matching list of model parameters and associated descriptions.
\item the corresponding
%circuit diagram and
 characteristic equations for the model (as required).
\item references to publications on which the model is based.
\end{XyceItemize}
User-defined models may be implemented using the \index{netlist!model
definition}\index{\texttt{.MODEL}} \texttt{.MODEL} (model definition)
statement, and macromodels can be created as subcircuits using the
\index{netlist!subcircuit} \texttt{.SUBCKT}\index{\texttt{.SUBCKT}}
(subcircuit) statement.

Please note that the characteristic equations\index{device!equations} are
provided to give a general representation of the device behavior.  The actual
\Xyce{} implementation of the device may be slightly different in order to
improve, for example, the robustness of the device.

Table~\ref{Device_Summary} gives a summary of the device types and the form of
their netlist formats.  Each of these is described below in detail.

% Sandia National Laboratories is a multimission laboratory managed and
% operated by National Technology & Engineering Solutions of Sandia, LLC, a
% wholly owned subsidiary of Honeywell International Inc., for the U.S.
% Department of Energy’s National Nuclear Security Administration under
% contract DE-NA0003525.

% Copyright 2002-2024 National Technology & Engineering Solutions of Sandia,
% LLC (NTESS).

%%
%% Analog Device Description Table.
%%

\newenvironment{DeviceList}[1]
               {\renewcommand{\arraystretch}{1.2}
                 \begin{longtable}{>{\raggedright\small}m{1.5in}>{\ttfamily\small}m{0.75in}<{\normalfont}>{\small}m{3.5in}}
                   \caption{#1} \\ \hline
                   \rowcolor{XyceDarkBlue}
                   \color{white}\normalfont\bf Device Type &
                   \color{white}\bf Letter &
                   \color{white}\bf Typical Netlist Format \endhead}
               {\end{longtable}}

\index{device!analog device summary}
\begin{DeviceList}{Analog Device Quick Reference.  \label{Device_Summary}}
\index{device!nonlinear dependent source}\index{device!B source}%
Nonlinear Dependent Source (B Source) & B &
\verb|B<name> <+ node> <- node>|\linebreak
\verb|+ <I or V>={<expression>}|
 \\ \hline

\index{device!capacitor}%
Capacitor & C &
\verb|C<name> <+ node> <- node> [model name] <value>|\linebreak
\verb|+ [IC=<initial value>]| \\ \hline

\index{device!diode}%
Diode & D &
\verb|D<name> <anode node> <cathode node>|\linebreak
\verb|+ <model name> [area value]| \\ \hline

\index{device!voltage controlled voltage source}%
Voltage Controlled Voltage Source & E&
\verb|E<name> <+ node> <- node> <+ controlling node>|\linebreak
\verb|+ <- controlling node> <gain>| \\ \hline

\index{device!current controlled current source}%
Current Controlled Current Source & F &
\verb|F<name> <+ node> <- node> |\linebreak
\verb|+ <controlling V device name> <gain>| \\ \hline

\index{device!voltage controlled current source}%
Voltage Controlled Current Source & G &
\verb|G<name> <+ node> <- node> <+ controlling node>|\linebreak
\verb|+ <- controlling node> <transconductance>| \\ \hline

\index{device!current controlled voltage source}%
Current Controlled Voltage Source & H &
\verb|H<name> <+ node> <- node>|\linebreak
\verb|+ <controlling V device name> <gain>| \\ \hline

\index{device!independent current source}%
Independent Current Source & I &
\verb|I<name> <+ node> <- node> [[DC] <value>]|\linebreak
\verb|+ [AC [magnitude value [phase value] ] ]|\linebreak
\verb|+ [transient specification]| \\ \hline

\index{device!mutual inductor}%
Mutual Inductor & K &
\verb|K<name> <inductor 1> [<ind. n>*]|\linebreak
\verb|+ <linear coupling or model>| \\ \hline

\index{device!inductor}%
Inductor & L &
\verb|L<name> <+ node> <- node> [model name] <value>|\linebreak
\verb|+ [IC=<initial value>]| \\ \hline

\index{device!JFET}%
JFET & J &
\verb|J<name> <drain node> <gate node> <source node>|\linebreak
\verb|+ <model name> [area value]| \\ \hline

\index{device!MOSFET}%
MOSFET & M &
\verb|M<name> <drain node> <gate node> <source node>|\linebreak
\verb|+ <bulk/substrate node> [SOI node(s)]|\linebreak
\verb|+ <model name> [common model parameter]*| \\ \hline

\index{device!LTRA}%
Lossy Transmission Line (LTRA) & O &
\verb|O<name> <A port (+) node> <A port (-) node>|\linebreak
\verb|+ <B port (+) node> <B port (-) node>|\linebreak
\verb|+ <model name>| \\ \hline

\index{device!bipolar junction transistor (BJT}%
Bipolar Junction Transistor (BJT)& Q &
\verb|Q<name> <collector node> <base node>|\linebreak
\verb|+ <emitter node> [substrate node]|\linebreak
\verb|+ <model name> [area value]| \\ \hline

\index{device!resistor}%
Resistor & R &
\verb|R<name> <+ node> <- node> [model name] <value>|\linebreak
\verb|+  [L=<length>] [W=<width>]| \\ \hline

\index{device!voltage controlled switch}%
Voltage Controlled Switch & S &
\verb|S<name> <+ switch node> <- switch node> |\linebreak
\verb|+ <+ controlling node> <- controlling node>|\linebreak
\verb|+ <model name>| \\ \hline

\index{device!generic switch}%
Generic Switch & S &
\verb|S<name> <+ switch node> <- switch node> |\linebreak
\verb|+ <model name> CONTROL=\{expression\}| \\ \hline

\index{device!transmission line}%
Transmission Line & T &
\verb|T<name> <A port + node> <A port - node>|\linebreak
\verb|+ <B port + node> <B port - node> |\linebreak
\verb|+ <ideal specification> | \\ \hline

\index{device!digital devices}\index{Digital Devices}%
Digital Devices & U &
\verb|U<name> <type> <digital power node> |\linebreak  
\verb|+ <digital ground node> [node]* <model name> | \\ \hline

\index{device!independent voltage source}%
Independent Voltage Source & V &
\verb|V<name> <+ node> <- node> [[DC] <value>]|\linebreak
\verb|+ [AC [magnitude value [phase value] ] ]|\linebreak
\verb|+ [transient specification]| \\ \hline

\index{device!port device}%
Port Device& P &
\verb|P<name> <+ node> <- node>  [[DC] <value>]|\linebreak
\verb|+ port=port number [Z0 = value]  |\linebreak
\verb|+ [AC [magnitude value [phase value] ] ]|\linebreak
\verb|+ [transient specification]| \\ \hline

\index{device!subcircuit}%
Subcircuit & X &
\verb|X<name> [node]* <subcircuit name> |\linebreak
\verb|+ [PARAMS:[<name>=<value>]*]| \\ \hline

\index{device!current controlled switch}%
Current Controlled Switch & W &
\verb|W<name> <+ switch node> <- switch node> |\linebreak
\verb|+ <controlling V device name> <model name>| \\ \hline

\index{device!digital devices}\index{Digital Devices}%
Digital Devices, Y Type (deprecated) & Y<type> &
\verb|Y<type> <name> [node]* <model name>| \\ \hline

\index{device!PDE Devices}\index{PDE Devices}%
PDE Devices & YPDE &
\verb|YPDE <name> [node]* <model name>| \\ \hline

\index{device!ACC Devices}\index{device!accelerated mass devices}\index{Accelerated Mass Devices}%
Accelerated masses & YACC &
\verb|YACC <name> <acceleration> <velocity> <position>|\linebreak
\verb|+ [x0=<initial position>] [v0=<initial velocity>]| \\ \hline

\index{device!linear device}%
Linear Device & YLIN &
\verb|YLIN <name> <+ node> <- node> <model name>|\\ \hline

\index{device!memristor}%
Memristor Device & YMEMRISTOR &
\verb|YMEMRISTOR <name> <+ node> <- node> <model name>|\\ \hline

\index{device!MESFET}%
MESFET & Z &
\verb|Z<name> <drain node> <gate node> <source node>|\linebreak
\verb|+ <model name> [area value]| \\ \hline

\end{DeviceList}



% Sandia National Laboratories is a multimission laboratory managed and
% operated by National Technology & Engineering Solutions of Sandia, LLC, a
% wholly owned subsidiary of Honeywell International Inc., for the U.S.
% Department of Energy’s National Nuclear Security Administration under
% contract DE-NA0003525.

% Copyright 2002-2024 National Technology & Engineering Solutions of Sandia,
% LLC (NTESS).

%%
%% Summary of Features Supported by Each Device.
%%

\begin{longtable}[h] {>{\raggedright\small}m{1.75in}|>{\raggedright\let\\\tabularnewline\small}m{2.15in}
  |>{\center\let\\\tabularnewline\small}m{0.5in}|>{\center\let\\\tabularnewline\small}m{0.5in}
  |>{\center\let\\\tabularnewline\small}m{0.5in}|>{\center\let\\\tabularnewline\small}m{0.6in}}
  \caption{Features Supported by Xyce Device Models\label{deviceFeatureSupportTable}} \\ \hline
  \rowcolor{XyceDarkBlue}
  \color{white}\bf Device &
  \color{white}\bf Comments &
  \color{white}\bf Branch Current &
  \color{white}\bf Power &
  \color{white}\bf Analytic Sensitivity &
  \color{white}\bf Stationary Noise\\ \hline \endfirsthead
  \caption[]{Features Supported by Xyce Device Models} \\ \hline
  \rowcolor{XyceDarkBlue}
  \color{white}\bf Device &
  \color{white}\bf Comments &
  \color{white}\bf Branch Current &
  \color{white}\bf Power &
  \color{white}\bf Analytic Sensitivity &
  \color{white}\bf Stationary Noise\\ \hline \endhead
    Capacitor & Age-aware, semiconductor  & Y & Y & Y & \\ \hline

    Inductor & Coupled mutual inductors (see below) & Y & Y & Y & \\ \hline

    Linear and Nonlinear Mutual Inductor & & Y & Y & & \\ \hline

    Resistor (Level 1) & Normal and Semiconductor & Y & Y & Y & Y \\ \hline

    Resistor (Level 2) & Thermal Resistor  & Y & Y & & \\ \hline

    Diode (Level 1) & & Y & Y & Y & Y \\ \hline

    Diode (Level 2) & Addition of PSpice enhancements & Y & Y & & \\ \hline

    Diode (Level 200) & JUNCAP200 model & Y & Y & Y & Y \\ \hline
    
    Independent Voltage Source (VSRC) & & Y& Y & Y & \\ \hline

    Independent Current Source (ISRC) & & Y & Y & Y & \\ \hline

    Voltage Controlled Voltage Source (VCVS) & & Y & Y & & \\ \hline
    Voltage Controlled Current Source (VCCS) & & Y & Y & & \\ \hline
    Current Controlled Voltage Source (CCVS) & & Y & Y & & \\ \hline
    Current Controlled Current Source (CCCS) & & Y & Y & & \\ \hline

    Port Device   & & Y& Y & & \\ \hline

    Nonlinear Dependent Source  (B Source) & & Y & Y & & \\ \hline

    Bipolar Junction Transistor (BJT) (Level 1) &  & Y & Y & Y & Y\\ \hline

    Bipolar Junction Transistor (BJT) (Level 11)& 
Vertical Bipolar Intercompany (VBIC) model, version 1.3 (3-terminal) & Y & Y & Y & Y \\ \hline

    Bipolar Junction Transistor (BJT) (Level 12)& 
Vertical Bipolar Intercompany (VBIC) model, version 1.3 (4-terminal) & Y & Y & Y & Y\\ \hline

    Bipolar Junction Transistor (BJT) (Level 23)& 
FBH (Ferdinand-Braun-Institut f\"ur H\"ochstfrequenztechnik) HBT model, version 2.1 
    & Y & Y & Y & N\\ \hline

Bipolar Junction Transistor (BJT) (Level 230)& 
HICUM Level 0
    & Y & Y & Y & Y\\ \hline
    
    Bipolar Junction Transistor (BJT) (Level 234)& 
HICUM Level 2
    & Y & Y & Y & Y\\ \hline

    Bipolar Junction Transistor (BJT) (Level 504)& 
MEXTRAM version 504.12.1 & Y & Y & Y & Y \\ \hline

    Bipolar Junction Transistor (BJT) (Level 505)& 
MEXTRAM version 504.12.1 (with self-heating) & Y & Y & Y & Y \\ \hline

    Junction Field Effect Transistor (JFET) (Level 1)  &  
SPICE-compatible JFET model & Y & Y & & \\ \hline
    Junction Field Effect Transistor (JFET) (Level 2) &  
Shockley JFET model & Y & Y & & \\ \hline

    MESFET & & Y & Y & & \\ \hline

    MOSFET (Level 1) &  & Y & Y & & Y \\ \hline
    MOSFET (Level 2) &  SPICE level 2 MOSFET & Y & Y & & Y \\ \hline
    MOSFET (Level 3) &  & Y & Y & & Y \\ \hline
    MOSFET (Level 6) &  SPICE level 6 MOSFET & Y & Y & & Y \\ \hline
    MOSFET (Level 9) &  BSIM3 model & Y & Y & Y & Y \\ \hline
    MOSFET (Level 10) & BSIM SOI model & Y & Y & & \\ \hline
    MOSFET (Level 14 or 54) & BSIM4 model & Y & Y & & \\ \hline
    MOSFET (Level 18) &  VDMOS general model & Y & Y & & \\ \hline
    MOSFET (Level 77) & BSIM6 model version 6.1.1 & Y & Y & Y & Y \\ \hline
    MOSFET (Level 102) & Legacy PSP model & Y & Y & Y & Y\\ \hline
    MOSFET (Level 103) & PSP model & Y & Y & Y & Y\\ \hline
    MOSFET (Level 107)  & BSIM-CMG version 107.0.0 & Y & Y & Y & Y\\ \hline
    MOSFET (Level 110)  & BSIM-CMG version 110.0.0 & Y & Y & Y & Y\\ \hline
    MOSFET (Level 301)& EKV model version 3.0.1 & Y & Y & Y & Y\\ \hline
    MOSFET (Level 2000) & MVS ETSOI model version 2.0.0 & Y & Y & Y & Y\\ \hline
    MOSFET (Level 2001) & MVS HEMT model version 2.0.0 & Y & Y & Y & Y\\ \hline

    Transmission Line (TRA) &  Lossless & Y & Y & & \\ \hline
    Transmission Line (LTRA) &  Lossy  & & & & \\ \hline
    Lumped Transmission Line &  Lossy or Lossless & & & & \\ \hline

    Controlled Switch (S,W) (VSWITCH/ISWITCH) & Voltage or current controlled
    & Y & Y & & \\ \hline

    Generic Switch (SW) & Controlled by an expression & Y & Y & & \\ \hline

    PDE Devices (Level 1) & one-dimensional & Y & & & \\ \hline

    PDE Devices (Level 2) & two-dimensional & Y & & & \\ \hline

    Digital (Level 1)  & Behavioral Digital & NA & NA & & \\ \hline

    ACC & Accelerated mass device, used for simulation of electromechanical and magnetically-driven machines 
    & NA & NA & & \\ \hline

    Power Grid & Separate models for Branch, Bus Shunt, Transformer and Generator 
                 Bus.  The Generator Bus model supports reactive power (Q) limiting & & & & \\ \hline

    Memristor & TEAM formulation & Y & Y &  & \\ \hline

    Memristor & Yakopcic & Y & Y &  & \\ \hline

    Memristor & PEM Formulation & Y & Y &  & \\ \hline    

\end{longtable}



\clearpage
\subsection{Voltage Nodes}
\label{Voltage_Nodes}
\index{device!nodes} \index{Voltage Nodes} \index{netlist!nodes}

Devices in a netlist are connected between {\em nodes}, and all device
types in \Xyce{} require at least two nodes on each instance line. Section
~\ref{legalCharacters} lists the characters that are legal and illegal in
\Xyce{} node and device names.

A node is simply a named point in the circuit.  The naming of nodes is mainly known within the level of circuit hierarchy where they appear.  Nodes can be passed into subcircuits through an argument list, and in this manner subcircuits are given access to nodes from the upper-level circuit.  Historicaly, this is how nodes are passed thru the circuit hierarchy in most circuit simulators, and this is the convention used by most circuit netlists.  However, \Xyce{} has two exceptions to this convention.  Global nodes, described in section~\ref{global_nodes} and fully resolved internal subcircuit nodes, described in section~\ref{subckt_nodes}.

\subsubsection{Global nodes}
\label{global_nodes}
A special syntax is used to designate certain nodes as {\em global}
nodes.  Any node whose name starts with the two characters ``\$G'' is
a global node, and such nodes are available to be used in any
subcircuit.  A typical usage of such global nodes is to define a \texttt{VDD}
or \texttt{VSS} signal that all subcircuits need to be able to access, but
without having to provide \texttt{VSS} and \texttt{VDD} input
nodes to every subcircuit.  In this case, a global \texttt{\$GVDD} node
would be use for the VDD signal.

The node named \texttt{0} is a special global node.  Node \texttt{0}
is always ground, and is accessible to all levels of a hierarchical
netlist.

For compatibility with HSPICE, the \texttt{.GLOBAL} command can
be used to define global nodes that do not start with the two
characters ``\$G''.  See section \ref{GLOBAL_section} for more details.

\subsubsection{Subcircuit Nodes}
\label{subckt_nodes}

Hierarchical netlists may be created using
\texttt{.SUBCKT} [\ref{SubcircuitDefinition}] to define common subcircuit
types, and \texttt{X} [\ref{SubcircuitInstance}] lines to create
instances of those subcircuits.  There are two types of nodes
associated with such subcircuits, {\em interface\/} nodes and {\em
internal\/} nodes.

Interface nodes are the nodes named on the \texttt{.SUBCKT} line.
These are effectively local aliases internal to the subcircuit
definition for the node names used on the \texttt{X} instance lines.
Internal nodes are nodes inside the subcircuit definition that are
strictly local to that subcircuit.  Inside a subcircuit, these node
names may be used without restriction in device instance lines and
expressions on \texttt{B} source lines.

There are some circumstances when it is desirable to access internal
nodes of a subcircuit from outside that subcircuit.  \Xyce{} provides a
syntax that allows this to be done.  The primary context in which this 
is supported is on \texttt{.PRINT} lines, to allow the user to print 
out signals that are usually local to a subcircuit.  However, this 
syntax isn't limited to \texttt{.PRINT} lines, and can work in
other contexts.

The syntax used by \Xyce{} to refer to nodes within a subcircuit is to
prefix the name of the node with the full path of subcircuit instances
in which the node is contained, with colons (:) separating the
instance names.  So, to reference a node ``A'' that is inside a
subcircuit instance called ``Xnot1'' inside another subcircuit
instance called ``Xmain'', one would refer to ``Xmain:Xnot1:A''.

Note that while the default separator in \Xyce{} is the colon (:), the period (.) 
is also optionally supported.  For more information about using a period separator, 
see section~\ref{cmd_line_args}.

The same syntax works on \texttt{.PRINT} lines even if the subcircuit
node is one of the interface nodes on the \texttt{.SUBCKT} line, but
those nodes can also be accessed by using the names of the nodes at
the higher level of circuit hierarchy that are used on its instance
line.

\begin{centering}
\shadowbox{
\begin{minipage}{0.8\textwidth}
\begin{vquote}
\color{blue}* Netlist demonstrating subcircuit node .PRINT access\color{black}
V1  1   0   1
X1  1   2   demosubc
X2  2   0   demosubc
.subckt demosubc A B
R1 A C 1
R2 C B 1
.ends

.dc V1 1 5 1

*V(X1:A) and V(1) are the same signal.
*V(X1:C) is the internal C node of the X1 instance
*V(X2:C) is the internal C node of the X2 instance
*V(X1:B), V(X2:A) and V(2) are the same signal
.print DC V(X1:C) V(X2:C) V(X1:A) V(1)  
+ V(X1:B) V(X2:A) V(2)
.end
\end{vquote}
\end{minipage}
}
\end{centering}

Internal subcircuit nodes may also be accessed from outside of the subcircuit
if one uses the fully resolved syntax.  This works in \texttt{B} source voltage or 
current expressions, and also in standard netlist usage on device instance lines.
This type of usage is outside of the typically 
strict hierarchy required by most circuit simulators, but it can be useful in some contexts.

The one difference between this usage and \texttt{.PRINT} usage is that it is not
possible to use the subcircuit node syntax to access interface nodes.
These must be accessed using the node names being used on the instance
line, as in the ``V(1)'' example in the netlist fragment above.  Two valid examples of 
internal subcircuit node access are given by the fragment below.
\begin{centering}
\shadowbox{
\begin{minipage}{0.8\textwidth}
\begin{vquote}
\color{blue}* Netlist demonstrating resolved subcircuit nodes\color{black}
Vin 1 0 1.0
X1 1 2 test
Rout 2 0 1.0
.subckt test A B
Rt1 A testNode 1.0
Rt2 testNode B 1.0
.ends

\color{blue}* this works: \color{black}
Btest1 3 0 V = {V(X1:testNode)}
Rtest1 3 0 1.0
\color{blue}* this also works:\color{black}
Itest2 0 4 1.0
Rtest2 X1:testNode 4 1.0
Rtest3 X1:testNode 0 1.0\end{vquote}
\end{minipage}
}
\end{centering}


%%
%% Legal and illegal characters in device names and node names
%%
\subsection{Legal Characters in Node and Device Names}
\label{legalCharacters}
\index{legal characters} \index{illegal characters}
\Xyce{} node names and device names can consist of any printable ASCII
characters, with the following exceptions and caveats which may be
different than other SPICE-like circuit simulators.  The exceptions are:
\begin{XyceItemize}
\item White space (space, tab, newline) is not allowed.
\item Parentheses (``('' or ``)''), braces (``\{'' or ``\}''), commas,
colons, semi-colons, double quotes and single quotes are also not
allowed, since they do not work correctly in node names or device names
in all netlist contexts in \Xyce{}.
\end{XyceItemize}
The caveats are as follows:
\begin{XyceItemize}
\item The star (*) and question mark (?) characters are allowed in both
node names and device names.  However, those two characters also function
as ``print wildcards'' in \Xyce{}, per section~\ref{Print_Wildcards}.
So, that usage is discouraged.
\item Global nodes in \Xyce{} begin with the two characters ``\$G''.
\item The node named \texttt{0} (``zero'') is a special global node,
which is always the ground node.
\item These arithmetic operators \verb|%| \verb|^| \verb|&| \verb|?| \verb|:|
\verb|~| \verb|*| \verb|-| \verb|+| \verb|<| \verb|>| \verb|/|
\verb+|+ should not be used in node or device names that will
be used outside of a \Xyce{} ``operator'', such as \texttt{V()}, within
a \Xyce{} expresson.  Examples of this caveat are given below.
\item The \verb|#| character should not be used as the first character
of a node name that will be used within an expression.  Examples of this
caveat are also given below.
\end{XyceItemize}
These are some examples of the caveats of the use of arithmetic operators
and \verb|#| character within expressions:
\begin{verbatim}
* Okay since the + in the node name is enclosed within the V() operator.
.PRINT TRAN {V(1+) - V(+)}
* Okay since the R+ and R- device names are enclosed within the I() operator.
.PRINT TRAN {I(R+) * I(R-)}
* Okay, for printing the resistance value, since the R-1 device name
* is not used in an expression.
.PRINT TRAN R-1:R
* Will produce a parsing error, since the R-1 device name is used outside
* of an operator.  That makes this statement ambiguous within an expression.
.PRINT TRAN {R-1:R}
* These uses of # are okay.
.PRINT TRAN V(#) {V(1#) -1}
* These usages of # are parsing errors, since # is the first character
* in the node names.
.PRINT TRAN {V(#) - 2} {V(#1) -1}
\end{verbatim}

%%
%% Lead Current and Power Calculations
%%
\subsection{Lead Currents and Power Calculations}
\label{leadCurrentPowerCalculations}
\index{lead currents} \index{power calculations} For some devices,
such as independent voltage and current sources, the current through
that device is a ``solution'' variable.  For other devices, the
current through the device is a ``lead current'', whose value is
calculated during a post-processing step.  This approach has
ramifications in \Xyce{} for the availability and accuracy of lead
current values.  In particular, both lead currents and power
calculations need to have been explicitly enabled for a given device,
analysis type (e.g., \texttt{.AC}) or netlist command (e.g.,
\texttt{.MEASURE}).

For voltage sources, both V and I are solution variables.  So, their
accuracy is more likely to be limited by the nonlinear solver
tolerances (\texttt{RELTOL} and \texttt{ABSTOL}).  The lead current
accuracy, for a device like the resistor, can also be limited by the
right-hand side tolerance \texttt{RHSTOL}.  So, the calculated lead
currents through very small resistances (e.g., 1e-12) may be
inaccurate if the default solver tolerances for \Xyce{} are used.

Lead currents have the following additional limitations:
\begin{XyceItemize}
\item They are not enabled for \texttt{.AC} analyses.
\item They are not allowed in the expression controlling a B-Source.
\item They do not work for \texttt{.RESULT} statements.
\end{XyceItemize}

Lead currents and power calculations are available in
\texttt{.MEASURE} and \texttt{.FOUR} statements.

At this time the power calculations are only supported for {\tt .DC}
and {\tt .TRAN} analysis types and for a limited set of devices. In
addition, the results for semiconductor devices (D, J, M, Q and Z
devices) and the lossless transmission device (T device) may differ
from other simulators.  Consult the Features Supported by Xyce Device
Models table in section \ref{Analog_Devices} and the individual
sections on each device for more details.

As an example, the power supplied or dissipated by the voltage source
{\tt V} is calculated as $I \cdot \Delta V$ where the voltage drop is
calculated as $(V_+ - V_-)$ and positive current flows from $V_+$ to
$V_-$.  Dissipated power has a positive sign, while supplied power has
a negative sign.

An important note is that the power calculations are also a
post-processing step, which places a limit on the accuracy of
circuit-wide ``energy conservation'' calculations (e.g., total power
supplied by sources - total power dissipated in non-source devices) in
\Xyce{}.  The accuracy of the inputs ({\tt V} and {\tt I}) to the
power calculations is limited by the nonlinear solver and right-hand
side tolerances, as noted above, and the error in the power
calculations is upper-bounded by the sum of the product-terms of {\tt
  V*(error in I)} and {\tt I*(error in V)}.


%%
%% Capacitor
%%
\clearpage
\subsection{Capacitor}
\index{device!capacitor} \index{capacitor}
% Sandia National Laboratories is a multimission laboratory managed and
% operated by National Technology & Engineering Solutions of Sandia, LLC, a
% wholly owned subsidiary of Honeywell International Inc., for the U.S.
% Department of Energy’s National Nuclear Security Administration under
% contract DE-NA0003525.

% Copyright 2002-2023 National Technology & Engineering Solutions of Sandia,
% LLC (NTESS).


\begin{Device}

\symbol
{\includegraphics{capacitorSymbol}}

\device
\begin{alltt}
C<device name> <(+) node> <(-) node> [model name] [value]
+ [device parameters]
\end{alltt}

\model
\begin{alltt}
.MODEL <model name> C [model parameters]
.MODEL <model name> CAP [model parameters]
\end{alltt}

\examples
\begin{alltt}
CM12 2 4 5.288e-13
CLOAD 1 0 4.540pF IC=1.5V
CFEEDBACK 2 0 CMOD 1.0pF
CAGED 2 3 4.0uF D=0.0233 AGE=86200
CSOLDEP 3 0 C=\{ca*(c0+c1*tanh((V(3,0)-v0)/v1))\}
CSOLDEPQ 3 0 Q=\{ca*(c1*v1*ln(cosh((v(3,0)-v0)/v1))+c0*v(3,0))\}
\end{alltt}

\parameters
\begin{Parameters}
\param{device name}
The name of the device.

\param{\vbox{\hbox{(+) node\hfil}\hbox{(-) node}}}
Polarity definition for a positive voltage across the capacitor. The first
node is defined as positive. Therefore, the voltage across the component is
the first node voltage minus the second node voltage.

\param{model name}
If \texttt{model name} is omitted, then \texttt{value} is the capacitance in
farads.  If [model name] is given then the value is determined from the model
parameters; see the capacitor value formula below.

\param{value}
Positional specification of device parameter C (capacitance).  Alternately,
this can be specified as a parameter, \texttt{C=<value>}, or in the (optional)
model.

\param{device parameters}
Parameters listed in Table~\ref{C_1_Device_Instance_Params} may be provided as
space separated \texttt{<parameter>=<value>} specifications as needed.  Any number
of parameters may be specified.

\param{model parameters}
Parameters listed in Table~\ref{C_1_Device_Model_Params} may be provided as
space separated \texttt{<parameter>=<value>} specifications as needed.  Any number
of parameters may be specified.

\end{Parameters}

\comments
Positive current flows through the capacitor from
the \texttt{(+)} node to the \texttt{(-)} node.  In general, capacitors should
have a positive capacitance value (\texttt{<value>} property). In all cases,
the capacitance must not be zero.  

However, cases exist when a negative capacitance value may be used. This occurs
most often in filter designs that analyze an RLC circuit equivalent to a real
circuit. When transforming from the real to the RLC equivalent, the result may
contain a negative capacitance value.

In a transient run, negative capacitance values may cause the simulation to
fail due to instabilities they cause in the time integration
algorithms.

The power stored or released from the capacitor is calculated 
with $I \cdot \Delta V$ where the voltage drop is calculated as $(V_+ - V_-)$ 
and positive current flows from $V_+$ to $V_-$.

For compatibility with PSpice, either \texttt{C} or \texttt{CAP} can be used in a
\texttt{.MODEL} statement for a capacitor. 

The Multiplicity Factor (M) can be used to specify multiple, identical capacitors
in parallel. The effective capacitance becomes C*M. The M value need not be an
integer. It can be any positive real number. M can not be used as a model
parameter.

\end{Device}

\paragraph{Device Parameters}

% This table was generated by Xyce:
%   Xyce -doc C 1
%
\index{capacitor!device instance parameters}
\begin{DeviceParamTableGenerated}{Capacitor Device Instance Parameters}{C_1_Device_Instance_Params}
AGE & Age of capacitor & hour & 0 \\ \hline
C & Capacitance & F & 1e-06 \\ \hline
D & Age degradation coefficient & -- & 0.0233 \\ \hline
DTEMP & Device delta temperature & $^\circ$C & 0 \\ \hline
IC & Initial voltage drop across device & V & 0 \\ \hline
L & Semiconductor capacitor width & m & 1 \\ \hline
M & Multiplicity Factor & -- & 1 \\ \hline
Q & Charge & C & 0 \\ \hline
TC1 & Linear Temperature Coefficient & $^\circ$C$^{-1}$ & 0 \\ \hline
TC2 & Quadratic Temperature Coefficient & $^\circ$C$^{-2}$ & 0 \\ \hline
TEMP & Device temperature & $^\circ$C & Ambient Temperature \\ \hline
W & Semiconductor capacitor length & m & 1e-06 \\ \hline
\end{DeviceParamTableGenerated}


In addition to the parameters shown in the table, the capacitor supports a
vector parameter for the temperature correction coefficients.
\texttt{TC1=<linear coefficient>} and \texttt{TC2=<quadratic coefficient>} may
therefore be specified compactly as \texttt{TC=<linear coefficient>,<quadratic
coefficient>}.

\paragraph{Model Parameters}

\input{C_1_Device_Model_Params}

\paragraph{Capacitor Equations}

\subparagraph{Capacitance Value Formula}
If \texttt{[model name]} is specified, then the capacitance is given by:

\[
 \mathbf{C} \cdot (1 + \mathbf{TC1} \cdot (T - T_0) +
\mathbf{TC2} \cdot (T - T_0)^2)
\]
where \texttt{C} is the base capacitance specified on the device line
and is normally positive (though it can be negative, but not zero).
$T_0$ is the nominal temperature (set using \textrmb{TNOM} option).

\subparagraph{Age-aware Formula}
If \textrmb{AGE} is given, then the capacitance is:
\[\mathbf{C}[1 - \mathbf{D} \log(\mathbf{AGE})]\]

\subparagraph{Semiconductor Formula}
If \texttt{[model name]} and \textrmb{L} and \textrmb{W} are given, then the capacitance is:
\[
\mathbf{CJ}(\mathbf{L} - \mathbf{NARROW})(\mathbf{W} - \mathbf{NARROW}) + 2
\cdot \mathbf{CJSW}(\mathbf{L} - \mathbf{W} + 2 \cdot \mathbf{NARROW})
\]

\subparagraph{Solution-Dependent Capacitor}
If the capacitance (\texttt{C}) is set equal to an expression then a
``solution-dependent'' capacitor is used, where the capacitance is
a function of other simulation variables.  The formulas for
temperature-dependence and age-dependence, given above, then use that
calculated \texttt{C} value.

If the parameter \texttt{Q} is set equal to an expression {\em
  instead} of specifying a capacitance, this expression is used to
evaluate the charge on the capacitor instead of computing it from
capacitance.  Temperature and age dependence are not computed in this
case, as these effects are applied by modifying the capacitance.

Both solution-dependent charge and capacitance formulations are
implemented to assure charge conservation.  The capacitor:
\begin{alltt}
  c\_mcap 1 2 q=\{ca*(c1*v1*ln(cosh((v(1,2)-v0)/v1))+c0*v(1,2))\}
\end{alltt}
is exactly equivalent to the capacitor
\begin{alltt}
  c\_mcap 1 2 c=\{ca*(c0+c1*tanh((V(1,2)-v0)/v1))\}
\end{alltt}
because the capacitance is the derivative of the charge with respect
to the voltage drop across the capacitor.  Similarly, both are
equivalent to the behavioral source:
\begin{alltt}
  BC 1 2 I={ddt(V(1,2))*(ca*(c0+c1*tanh((V(1,2)-v0)/v1)))}
\end{alltt}
because $I=dQ/dt=dQ/dV*dV/dt=C*dV/dt$.

The restrictions for this formulation are:
\begin{XyceItemize}
  \item The expression used for \texttt{C} or \texttt{Q} must only use
    solution variables, which are node voltages and also branch
    currents for source devices.  It may not use device lead currents,
    which are post-processed quantities that are not solution variables.
  \item The expression must not use time derivatives.
  \item Capacitance (\texttt{C}) and Charge (\texttt{Q}) are the only
    instance or model parameters that are allowed to be
    solution-dependent.
\end{XyceItemize}

\paragraph{Other Restrictions and Caveats}
A netlist parsing error will occur if:
\begin{XyceItemize}
  \item Neither the \texttt{C}, \texttt{Q}, nor \texttt{L} instance
    parameters are specified.
  \item Both \texttt{C} and \texttt{Q} are specified as expressions.
    \item \texttt{Q} is specified in addition to an \texttt{IC=}.
  \item The \texttt{A} instance parameter is specified for a semiconductor 
      capacitor (which is specified via \texttt{L}, \texttt{W} and \texttt{CJSW}).
\end{XyceItemize}
If both the \texttt{C} and \texttt{L} instance parameters are specified then 
\texttt{C} will be used, rather than the semiconductor formulation. 

\paragraph{Special note on Initial Conditions:}

The IC parameter of the capacitor may be used to specify an initial
voltage drop on the capacitor.  Unlike SPICE3F5, this parameter is
never ignored (SPICE3F5 only respects it if UIC is used on a transient
line).  The initial condition is applied differently depending on the
analysis specified.

If one is doing a transient with DC operating point calculation or a
DC operating point analysis, the initial condition is applied by
inserting a voltage source across the capacitor to force the operating
point to find a solution with the capacitor charged to the specific
voltage.  The resulting operating point will be one that is consistent
with the capacitor having the given voltage in steady state.

If one specifies \texttt{UIC} or \texttt{NOOP} on the \texttt{.TRAN}
line, then \Xyce{} does not perform an operating point calculation,
but rather begins a transient simulation directly given an initial
state for the solution.  In this case, \texttt{IC} initial conditions
are applied only for the first iteration of the Newton solve of the
first time step --- the capacitor uses the initial condition to
compute its charge, and the nonlinear solver will therefore find a
solution to the circuit problem consistent with this charge, i.e., one
with the correct voltage drop across the capacitor.

The caveats of this section apply only to initial conditions specified
via \texttt{IC=} parameters on the capacitor, and do not affect how
initial conditions are applied when using \texttt{.IC} lines to
specify initial conditions on node values.

The three different ways of specifying initial conditions can lead to
different circuit behaviors.  Notably, when applying initial
conditions during a DC operating point with \texttt{IC=} on the
capacitor line, the resulting operating point will be a DC solution
with currents everywhere consistent with there being a constant charge
on the capacitor, whereas in general a transient run from an initial
condition {\em without\/} having performed an operating point
calculation will have a quiescent circuit at the first timestep.


%%
%% Inductor
%%
\clearpage
\subsection{Inductor}
\index{device!inductor} \index{inductor}
% Sandia National Laboratories is a multimission laboratory managed and
% operated by National Technology & Engineering Solutions of Sandia, LLC, a
% wholly owned subsidiary of Honeywell International Inc., for the U.S.
% Department of Energy’s National Nuclear Security Administration under
% contract DE-NA0003525.

% Copyright 2002-2024 National Technology & Engineering Solutions of Sandia,
% LLC (NTESS).


\begin{Device}\label{L_DEVICE}

\symbol
{\includegraphics{inductorSymbol}}

\device
L<name> <(+) node> <(-) node> [model] <value> [device parameters]

\model
\begin{alltt}
.MODEL <model name> L [model parameters]
.MODEL <model name> IND [model parameters]
\end{alltt}

\examples
\begin{alltt}
L1 1 5 3.718e-08
LM 7 8 L=5e-3 M=2
LLOAD 3 6 4.540mH IC=2mA
Lmodded 3 6 indmod 4.540mH
.model indmod L (L=.5 TC1=0.010 TC2=0.0094)
\end{alltt}

\parameters

\begin{Parameters}
\param{\vbox{\hbox{(+) node\hfil}\hbox{(-) node}}}

Polarity definition for a positive voltage across the inductor. The
first node is defined as positive. Therefore, the voltage across the
component is the first node voltage minus the second node voltage.

\param{initial value}

The initial current through the inductor during the bias point
calculation.

\end{Parameters}

\comments
In general, inductors should have a positive
inductance value. The inductance must
not be zero.  Also, a netlist parsing error will occur if no value 
is specified for the inductance. 

However, cases exist when a negative value may be used.  This occurs most often
in filter designs that analyze an RLC circuit equivalent to a real circuit.
When transforming from the real to the RLC equivalent, the result may contain a
negative inductance value.

The power stored or released from the inductor is calculated 
with $I \cdot \Delta V$ where the voltage drop is calculated as $(V_+ - V_-)$ 
and positive current flows from $V_+$ to $V_-$.

If a model name is given, the inductance is modified from the value
given on the instance line by the parameters in the model card.  See
``Inductance Value Formula'' below.

When an inductor is named in the list of coupled inductors in a mutual
inductor device line (see page~\pageref{MutualInductor}) , and that
mutual inductor is of the nonlinear-core type, the \verb+<value>+ is
interpreted as a number of turns rather than as an inductance in
Henries.

For compatibility with PSpice, either \texttt{L} or \texttt{IND} can be used in a
\texttt{.MODEL} statement for an inductor. 

The Multiplicity Factor (\texttt{M}) can be used to specify multiple, identical 
inductors in parallel. The effective inductance becomes \texttt{L}/\texttt{M}.
However, the value for the \texttt{IC} instance parameter is not multiplied by 
the \texttt{M} value. The \texttt{M} value need not be an integer.  It can be 
any positive real number. \texttt{M} can not be used as a model parameter.

\end{Device}

\newpage
%\pagebreak

\paragraph{Device Parameters}
% This table was generated by Xyce:
%   Xyce -doc L 1
%
\index{inductor!device instance parameters}
\begin{DeviceParamTableGenerated}{Inductor Device Instance Parameters}{L_1_Device_Instance_Params}
DTEMP & Device delta temperature & $^\circ$C & 0 \\ \hline
IC & Initial current through device & A & 0 \\ \hline
L & Inductance & henry & 0 \\ \hline
M & Multiplicity Factor & -- & 1 \\ \hline
TC1 & Linear Temperature Coefficient & $^\circ$C$^{-1}$ & 0 \\ \hline
TC2 & Quadratic Temperature Coefficient & $^\circ$C$^{-2}$ & 0 \\ \hline
TEMP & Device temperature & $^\circ$C & Ambient Temperature \\ \hline
\end{DeviceParamTableGenerated}


\paragraph{Model Parameters}
\input{L_1_Device_Model_Params}

In addition to the parameters shown in the table, the inductor supports a vector parameter for the temperature correction coefficients.  \texttt{TC1=<linear coefficient>} and \texttt{TC2=<quadratic coefficient>} may therefore be specified compactly as \texttt{TC=<linear coefficient>,<quadratic coefficient>}.

\paragraph{Inductor Equations}

\subparagraph{Inductance Value Formula}
If \verb+[model name]+ is specified, then the inductance is given by:
\[\mathbf{L}_{base} \cdot \mathbf{L} \cdot (1 + \mathbf{TC1} \cdot (T - T_{0}) +
\mathbf{TC2} \cdot (T - T_{0})^{2})\]
where \texttt{$\mathbf{L}_{base}$} is the base inductance specified on the device line and is normally positive (though it can be
negative, but not zero).  $\mathbf{L}$ is the inductance multiplier specified in the model card.  $T_0$ is the nominal temperature (set using
\textrmb{TNOM} option).

%\subparagraph{Inductor Noise Equation}
%There is no noise model for the inductor.


%%
%% Mutual Inductors
%%
\clearpage
\subsection{Mutual Inductors}
\label{MutualInductor}
\index{device!mutualinductor} \index{mutualinductor}
% Sandia National Laboratories is a multimission laboratory managed and
% operated by National Technology & Engineering Solutions of Sandia, LLC, a
% wholly owned subsidiary of Honeywell International Inc., for the U.S.
% Department of Energy’s National Nuclear Security Administration under
% contract DE-NA0003525.

% Copyright 2002-2024 National Technology & Engineering Solutions of Sandia,
% LLC (NTESS).



\begin{Device}\label{K_DEVICE}

\symbol
{\includegraphics{transformerSymbol}}

\device
\begin{alltt}
K<name> L<inductor name> [L<inductor name>*]
+ <coupling value> [model name]
\end{alltt}

\model
.MODEL <model name> CORE [model parameters]

\examples
\begin{alltt}
ktran1 l1 l2 l3 1.0
KTUNED L3OUT  L4IN .8
KTRNSFRM LPRIMARY LSECNDRY 1
KXFRM L1 L2  L3  L4 .98 KPOT\_3C8
\end{alltt}

\parameters
\begin{Parameters}

\param{inductor name}

Identifies the inductors to be coupled. The inductors are coupled and in
the dot notation the dot is placed on the first node of each
inductor. The polarity is determined by the order of the nodes in the L
devices and not by the order of the inductors in the K statement.

If more than two inductors are given on a single K line, each inductor
is coupled to all of the others using the same coupling value.

\param{coupling value}

The coefficient of mutual coupling, which must be between $-1.0$ and
$1.0$.

This coefficient is defined by the equation
\begin{quote}
  \texttt{<coupling value>} = $\frac{M_{ij}}{\sqrt{L_iL_j}}$
\end {quote}

where
\begin{quote}
  $L_i$ is the inductance of the $i$th named inductor in the K-line
\end {quote}
\begin{quote}
    $M_{ij}$ is the mutual inductance between $L_i$ and $L_j$
\end {quote}
For transformers of normal geometry, use $1.0$ as the value. Values less
than $1.0$ occur in air core transformers when the coils do not
completely overlap.

\param{model name}

If \texttt{model name} is present, four things change:
\begin{itemize}
  \item The mutual coupling inductor becomes a nonlinear, magnetic core device.
  \item The inductors become windings, so the number specifying inductance now
        specifies the number of turns.
  \item The list of coupled inductors could be just one inductor.
  \item If two or more inductors are listed, each inductor is coupled to all others through the magnetic core.
  \item A model statement is required to specify the model parameters.
\end{itemize}

\end{Parameters}

\comments
Lead currents and power calculations are supported for the component inductors in both
linear and nonlinear mutual inductors.  They are not supported for the composite
mutual inductor though.  So, if \texttt{L1} is a component inductor for mutual inductor
\texttt{K1}, then requests for \texttt{I(L1)}, \texttt{P(L1)} and \texttt{W(L1)} will 
return lead current and power values as defined in Section~\ref{L_DEVICE}.  However, 
any usage of \texttt{I(K1)}, \texttt{P(K1)} and \texttt{W(K1)} will result in a 
\Xyce{} netlist parsing error.

\end{Device}

\paragraph{Model Parameters}
\input{Min_1_Device_Model_Params}

Note that \Xyce's default value for the $\mbox{GAP}$ parameter as zero.  Some simulators
will use non-zero values of the $\mbox{GAP}$ as a default.  When using netlists from 
other simulators in \Xyce, ensure that the default parameters are consistent. 

\subparagraph{Special Notes}

The coupling coefficient of the linear mutual
inductor (i.e. a mutual inductor without a core model) is permitted to
be a time- or solution variable-dependent expression.  This is
intended to allow simulation of electromechnical devices in which
there might be moving coils that interact with fixed coils.  

Additionally, for linear mutual inductors,
different coupling terms can be applied to different pairs of inductors with this syntax:
{\tt 
\begin{verbatim}
L1 1 2 2.0e-3
L2 0 3 8.1e-3
L3 3 4 8.1e-3 
ktran1 l1 l2 0.7
ktran2 l2 l3 0.9
ktran3 l1 l3 0.99
\end{verbatim}
}

Nonlinear mutual inductors can output $B(t)$ and $H(t)$ variables so that
one can plot $B-H$ loops.  On the {\tt .print} line the $B$ and $H$ variables 
are accessible using the node output syntax as in {\tt n( non-linear-inductor-name\_b )} for $B$
and {\tt n( non-linear-inductor-name\_h )} for $H$.  A confusing aspect of this is that 
the non-linear inductor name is the {\em internal } name used by \Xyce{}.  For example, 
consider this circuit which defines a nonlinear mutual inductor at both the top level of
the circuit and within a subcircuit:

{\tt 
\begin{verbatim}
* Test Circuit for Mutually Coupled Inductors

VS 0 1 SIN(0 169.7 60HZ)
R1 1 2 1K
R2 3 0 1K
L1 2 0 10
L2 3 0 20
K1 L1 L2 0.75 txmod
.model txmod core 

.subckt mysub n1 n2 n3 
r1s n1 n2 1000
r2s n3 0  1000
L1s n2 0  10
L2s n3 0  20   
k1s L1s L2s 0.75 txmod
.ends

xtxs 1 4 5 mysub

.TRAN 100US 25MS

* output the current through each inductor and the B & H values.
.PRINT TRAN I(L1) I(L2) n(ymin!k1_b)  n(ymin!k1_h)  
+ I(xtxs:L1s)  I(xtxs:l2s) n(xtxs:ymin!k1s_b) n(xtxs:ymin!k1s_h)

.END
\end{verbatim}
}

The internal, \Xyce{} name of the non-linear mutual inductor is {\tt
YMIN!K1} or {\tt ymin!k1} as the name is not
case-sensitive.  The device {\tt k1s} is declared within a
subcircuit called {\tt xtxs}. Thus, its full name is {\tt
xtxs:ymin!k1s}. The reason for this is that both the linear and
non-linear mutual inductors are devices that are collections of other
devices, inductors in this case.  Rather than use one of the few
remaining single characters left to signify a new device, \Xyce{} uses
{\tt Y} devices as an indicator of a extended device set, where the
characters after the {\tt Y} denote the device type and then the device
name.  Here, {\tt ymin } means a {\tt min } device which is a {\em
mutual-inductor, non-linear} device.  Thus, to print the $B$ or $H$
variable of the non-linear mutual inductor called {\tt k1} one would
use {\tt n(ymin!k1\_b)} and {\tt n(ymin!k1\_h)}
respectively for a {\tt .print} line that looks like this:

{\tt 
\begin{verbatim}
.PRINT TRAN I(L1) I(L2) n(ymin!k1_b)  n(ymin!k1_h)  
\end{verbatim}
}

And if the mutual inductor is in a subcircuit called {\tt xtxs} then the 
{\tt .print} line would look like this:

{\tt 
\begin{verbatim}
.PRINT TRAN I(xtxs:L1s)  I(xtxs:l2s) n(xtxs:ymin!k1s_b) n(xtxs:ymin!k1s_h)
\end{verbatim}
}

The above example also demonstrates how one outputs the current through inductors 
that are part of mutual inductors.  
The syntax is {\tt I( inductor name )}.

Note that while MKS units are used internally in \Xyce{}, $B$ and $H$ are output
by default in the SI units of Gauss for $B$ and Oersted for $H$.
To convert $B$ to units of Tesla divide  \Xyce{}'s output by $10,000$.  
To convert $H$ to units of $A/m$ divide  \Xyce{}'s output by $4\pi/1000$. 
Additionally, one can set the {\tt .model CORE }
parameter {\tt BHSIUNITS } to 1 to force $B$ and $H$ to be output in MKS
units. 


Finally, one can access the $B$ and $H$ data via the {\tt .model
CORE} line. On the nonlinear mutual inductor's {\tt .model} line, set the
option {\tt OUTPUTSTATEVARS=1}. This will cause \Xyce{} to create a unique
file for each nonlinear mutual inductor that uses this {\tt .model} line
with a name of the form {\tt Inductor\_}device\_name.  There are five
columns of data in this file: time ($t$), magnetic moment ($M$), total
current flux ($R$), flux density ($B$) and magnetic field strength
($H$).  As with data output on the {\tt .print} line, SI units are used
such that $B$ is output with units of Gauss and $H$ in Oersted.  As
mentioned earlier, setting the model flag {\tt BHSIUNITS } to 1 causes
the output of $B$ and $H$ uses MKS units of Tesla and $A/m$
respectively. 

\index{mutualinductor!mutual inductor equations}
\subparagraph{Mutual Inductor Equations}

The voltage to current relationship for a set of linearly coupled inductors is:
\begin{equation}
V_{i} = \sum_{j=1}^{N} c_{ij} \sqrt{ L_{i} L_{j} } \frac{dI_{j}}{dt}
\label{linMIRelation}
\end{equation}

Here, $V_{i}$ is the voltage drop across the $i$th inductor in the coupled set.
The coupling coefficient between a pair of inductors is $c_{ij}$ with a value typically
near unity and $L$ is the inductance of a given inductor which has units of
\emph{Henry's} (1 Henry $ = 1 H = Volt \cdot s / Amp $)

For nonlinearly coupled inductors, the above equation is expanded to the form:
\begin{equation}
V_{i} = \left[1 + \left(1-\frac{\ell_g}{\ell_t}\right)P(M,I_{1} ... I_{N})\right]\sum_{j=1}^{N} Lo_{ij} \frac{dI_{j}}{dt}
\label{nonlinMIRelation}
\end{equation}
This is similar in form to the linearly coupled inductor equation.  However,
the coupling has become more complicated as it now depends on the magnetic moment
created by the current flow, $M$.  Additionally, there are geometric factors,
$\ell_{g}$ and $\ell_{t}$ which are the effective air gap and total mean magnetic
path for the coupled inductors. The matrix of terms, $Lo_{ij}$ is defined as
\begin{equation}
Lo_{ij}=\frac{\mu_0 A_c N_i N_j}{\ell_t}
\end{equation}
and it represents the physical coupling between inductors $i$ and $j$.  In this
expression, $N_i$ is the number of windings around the core of inductor $i$,
$\mu_0$ is the magnetic permeability of free space which has units of Henries per meter
and a value of $4\pi \times 10^{-7}$ and $A_c$ is the mean magnetic cross-sectional area.

The magnetic moment, $M$ is defined by:
\begin{equation}
\frac{dM}{dt} = \frac{1}{\ell_t} P \sum_{i=1}^{N} N_i \frac{dI_i}{dt}
\end{equation}
and the function $P$ is defined as:
\begin{equation}
P = \frac{c M'_{an} + (1-c)M'_{irr}}{1 + \left(\frac{\ell_g}{\ell_t}-\alpha\right) c M'_{an} + \frac{\ell_g}{\ell_t}(1-c)M'_{irr}}
\end{equation}
If $c < \mbox{CLIM}$, then $c$ is treated as zero in the above equation and \Xyce\/ 
simplifies the formulation.  In this case, the magnetic-moment equation will not be 
needed and it will be be dropped form the formulation.  One can controll this behavior by
modifying the value of $\mbox{CLIM}$.

The remaining functions are:
\begin{eqnarray}
M'_{an} & = & \frac{M_s A}{\left(A + |H_e|\right)^2} \\
H_e & = & H + \alpha M \\
H & = & H_{app} - \frac{\ell_g}{\ell_t}M  \\
H_{app} & = & \frac{1}{\ell_t}\sum_{i=1}^{N} N_i I_i \\
M'_{irr} & = & \frac{\Delta M sgn(q) + |\Delta M|}{2\left(K_{irr} - \alpha |\Delta  M|\right)} \\
\Delta M & = & M_{an} - M \\
M_{an} & = & \frac{M_s H_e}{A + |H_e|} \\
q & = & \mbox{DELVSCALING}  \Delta V
\end{eqnarray}

\Xyce\/ dynamically modifies $\mbox{DELVSCALING}$ to be $1000 / $ Maximum Voltage Drop over the 
first inductor.  This typically produces accurate results for both low voltage and high 
voltage applicaitons.  However, it is possible to use a fixed scaling by setting the 
model parameter $\mbox{CONSTDELVSCALING}$ to true and then setting $\mbox{DELVSCALING}$ 
to the desired scaling value.

In \Xyce{}'s formulation, we define $R$ as:
\begin{equation}
R = \frac{dH_{app}}{dt} = \frac{1}{\ell_t}\sum_{i=1}^{N} N_i \frac{dI_i}{dt}
\end{equation}
This simplifies the $M$ equation to:
\begin{equation}
\frac{dM}{dt} =  P R
\end{equation}
\Xyce{} then solves for the additional variables $M$ and $R$ when modeling a nonlinear
mutual inductor device.

\subparagraph{B-H Loop Calculations}
\index{mutualinductor!B-H loop calculations}
To calculate $B$-$H$ loops, $H$ is used as defined above and $B$ is a derived quantity
calculated by:
\begin{eqnarray}
B & = & \mu_0 \left( H + M \right) \\
  & = & \mu_0 \left[ H_{app} + \left(1 - \frac{\ell_g}{\ell_t}\right) M \right]
\end{eqnarray}

\subparagraph{Converting Nonlinear to Linear Inductor Models} 
\index{mutualinductor!Converting nonlinear to linear mutual indcutors}
At times one may have a model for nonlinear mutual inductor, but wish to use a 
simpler linear model in a given circuit.  To convert a non-linear model to an 
equivalent linear form, one can start by equating the coupling components of 
equations~\ref{linMIRelation} and~\ref{nonlinMIRelation} as:

\begin{equation}
 c_{ij} \sqrt{ L_{i} L_{j} } =  \left[1 + \left(1-\frac{\ell_g}{\ell_t}\right)P(M,I_{1} ... I_{N})\right] Lo_{ij}
\label{nonlinMIRelation2}
\end{equation}
In the above relationship, $i$ and $j$ represent the $i$th and $j$th inductors.  Since we would like to 
equate the $i$th inductor's nonliner properties to its linear properties, we will substitute $i\rightarrow j$ 
and simplify assuming steady state where $d/dt = 0$ and $M(t) = 0$.

\begin{equation}
L_{i} = \frac{1}{c_{ii}} \left\{ 1 + \left(1 - \frac{\ell_g}{\ell_t}  \right) 
  \left[\frac{c \frac{Ms}{A}}{1 + \left( \frac{\ell_g}{\ell_t} - \alpha\right) \frac{Ms}{A}} \right] \right\} \frac{\mu A_c}{\ell_t} N_i^2
  \label{inductanceFromWindings}
\end{equation}

In the above equatin, $c_{ii}$ represents the coupling coefficient between the $i$th inductor with itself.  
This will likely be $1$ unless there are very unusual geometry considerations.  Note, that the terms $A$, $Ms$, $A_c$, $\mu$,
$\ell_{g}$ and $\ell_{t}$ all have units of length within them and must use the same unit for this relationship 
to be valid.  Specifically, $\mu$ has units of Henery's per meter and $A$ and $Ms$ have units of Amps per meter.   
$A_c$, $\ell_{g}$ and $\ell_{p}$ have units of length$^2$ and length respectively, but the length unit used in 
the model statement is $cm^2$ and $cm$ respectively.  Thus, one must use consistent units such as meters 
for $A_c$, $\ell_{g}$ and $\ell_{p}$ in equation~\ref{inductanceFromWindings} for a valid inductance approximation.



%%
%% Resistor
%%
\clearpage
\subsection{Resistor}
\index{device!resistor} \index{resistor}
% Sandia National Laboratories is a multimission laboratory managed and
% operated by National Technology & Engineering Solutions of Sandia, LLC, a
% wholly owned subsidiary of Honeywell International Inc., for the U.S.
% Department of Energy’s National Nuclear Security Administration under
% contract DE-NA0003525.

% Copyright 2002-2023 National Technology & Engineering Solutions of Sandia,
% LLC (NTESS).


\begin{Device}

\symbol
{\includegraphics{resistorSymbol}}

\device
R<name> <(+) node> <(-) node> [model name] [value] [device parameters]

\model
\begin{alltt}
.MODEL <model name> R [model parameters]
.MODEL <model name> RES [model parameters]
\end{alltt}

\examples
\begin{alltt}
R1 1 2 2K TEMP=27
RM 4 5 R=4e3 M=2
RSOLDEP 2 0 R=\{1.0+scalar*V(1)\}
RLOAD 3 6 RTCMOD 4.540 TEMP=85
.MODEL RTCMOD R (TC1=.01 TC2=-.001)
RSEMICOND 2 0 RMOD L=1000u W=1u
.MODEL RMOD R (RSH=1)
\end{alltt}

\parameters

\begin{Parameters}

\param{\vbox{\hbox{(+) node\hfil}\hbox{(-) node}}}

Polarity definition for a positive voltage across the resistor. The
first node is defined as positive. Therefore, the voltage across the
component is the first node voltage minus the second node voltage.
Positive current flows from the positive node (first node) to the
negative node (second node).

\param{model name}

If \texttt{[model name]} is omitted, then \texttt{[value]} is the
resistance in Ohms. If \texttt{[model name]} is given then the
resistance is determined from the model parameters; see the resistance
value formula below.

\param{value}

Positional specification of device parameter R (resistance).
Alternately, this can be specified as a parameter, \texttt{R=<value>},
or in the (optional) model.

\param{device parameters}

Parameters listed in Table~\ref{R_1_Device_Instance_Params} may be provided as
space separated \texttt{<parameter>=<value>} specifications as needed.
Any number of parameters may be specified.

\end{Parameters}

\comments

Resistors can have either positive or negative resistance values (R).  A zero 
resistance value (R) is also allowed. 

The power dissipated in the resistor is calculated 
with $I \cdot \Delta V$ where the voltage drop is calculated as $(V_+ - V_-)$ 
and positive current flows from $V_+$ to $V_-$.  The power accessors
(\texttt{P()} and \texttt{W()}) are supported for both the level 1 resistor
and the level 2 (thermal) resistor.

For compatibility with PSpice, either \texttt{R} or \texttt{RES} can be used in a
\texttt{.MODEL} statement for a resistor.

The Multiplicity Factor (\texttt{M}) can be used to specify multiple, identical 
resistors in parallel. The effective resistance becomes \texttt{R}/\texttt{M}.  
The \texttt{M} value need not be an integer.  It can be any positive real number.  
\texttt{M} can not be used as a model parameter. 

\end{Device}

\newpage
%\pagebreak

\paragraph{Device Parameters}
% This table was generated by Xyce:
%   Xyce -doc R 1
%
\index{resistor!device instance parameters}
\begin{DeviceParamTableGenerated}{Resistor Device Instance Parameters}{R_1_Device_Instance_Params}
DTEMP & Device delta temperature & $^\circ$C & 0 \\ \hline
L & Length & m & 0 \\ \hline
M & Multiplicity Factor & -- & 1 \\ \hline
R & Resistance & $\mathsf{\Omega}$ & 1000 \\ \hline
TC1 & Linear Temperature Coefficient & $^\circ$C$^{-1}$ & 0 \\ \hline
TC2 & Quadratic Temperature Coefficient & $^\circ$C$^{-2}$ & 0 \\ \hline
TCE & Exponential Temperature Coefficient & \%$/^\circ$C & 0 \\ \hline
TEMP & Device temperature & $^\circ$C & Ambient Temperature \\ \hline
W & Width & m & 0 \\ \hline
\end{DeviceParamTableGenerated}


In addition to the parameters shown in the table, the resistor supports a vector parameter for the temperature correction coefficients.  \texttt{TC1=<linear coefficient>} and \texttt{TC2=<quadratic coefficient>} may therefore be specified compactly as \texttt{TC=<linear coefficient>,<quadratic coefficient>}.

\paragraph{Model Parameters}
\input{R_1_Device_Model_Params}

Note: There is no model parameter for Default Instance Length.  The use of the semiconductor resistor model requires
the user to specify a non-zero value for the instance parameter \texttt{L}. 

\paragraph{Resistor Equations}

\subparagraph{Resistance Value Formulas}
If the \textrmb{R} parameter is given on the device instance line 
then that value is used.

If the \textrmb{R} parameter is not given then the semiconductor resistor model
will be used if the \textrmb{L} instance parameter and the \textrmb{RSH} model parameter
are given and both are non-zero.  In that case the resistance will be as follows.
(Note: If \textrmb{W} is not given on the instance line then the value
for the model parameter \textrmb{DEFW} will be used instead.)

\[
\mathbf{RSH} \frac{[\mathbf{L} - \mathbf{NARROW}]}
{[\mathbf{W} - \mathbf{NARROW}]}
\]

If neither of these two cases apply then the default value for
the \textrmb{R} parameter will be used.     

\subparagraph{Temperature Dependence}
If \textrmb{TCE} is specified as either an instance or model parameter
for the Level 1 resistor then the resistance at temperature $T$
is given by (where the resistance at the nominal temperature ($T_{0}$) 
was defined above in the resistance value formulas):

\[
\mathbf{R} \cdot pow(1.01,\mathbf{TCE} \cdot (T - T_{0}))
\]

otherwise the resistance is given by:
\[
\mathbf{R} \cdot (1 + \mathbf{TC1} \cdot (T - T_{0}) + \mathbf{TC2}
\cdot (T - T_0)^2)
\]

\paragraph{Thermal (level=2) Resistor}

\Xyce{} supports a thermal resistor model, which is associated with level=2.

\paragraph{Thermal Resistor Instance Parameters}
\input{R_2_Device_Instance_Params}


%%
%% Resistor Model Param Table
%%
\paragraph{Thermal Resistor Model Parameters}
\input{R_2_Device_Model_Params}

The temperature model for the thermal resistor will be enabled
if the \textrmb{A} and \textrmb{L} instance parameters are 
given and the parameters \textrmb{HEATCAPACITY} and
\textrmb{RESISTIVITY} are also given as a pair of either 
instance parameters or model parameters.  Otherwise, the
resistance value and temperature dependence of the Level 2
resistor will follow the equations for the Level 1 resistor 
given above, with the caveat that \texttt{TCE} is only 
allowed as a model parameter for the Level 2 resistor.

If the temperature model for the Level 2 resistor is enabled,
then the resistance ($R$) is given by the following, where the 
\textrmb{RESISTIVITY} can be a temperature-dependent expression:

\[
\frac{\mathbf{RESISTIVITY} \cdot \mathbf{L}}
{\mathbf{A}}
\]

The rate-of-change ($dT/dt$) of the temperature ($T$) of the 
thermal resistor with time is then given by the following where 
$i_{0}$ is the current through the resistor:

\[
\frac{i_{0} \cdot i_{0} \cdot R}
{(\mathbf{A} \cdot \mathbf{L} \cdot \mathbf{HEATCAPACITY}) +
 (\mathbf{THERMAL\_A} \cdot \mathbf{THERMAL\_L} \cdot \mathbf{THERMAL\_HEATCAPACITY})}
\]

\subparagraph{Solution-Dependent Resistor}
If the resistance (\texttt{R}) is set equal to an expression then a
``solution-dependent'' resistor is used, where the resistor is
a function of other simulation variables.  The formulas for
temperature-dependence, given above, then use that
calculated \texttt{R} value.

The restrictions for this solution dependent resistors are:
\begin{XyceItemize}
  \item The expression used for \texttt{R} must only use
    solution variables, which are node voltages and also branch
    currents for source devices.  It may not use device lead currents,
    which are post-processed quantities that are not solution variables.
  \item The expression must not use time derivatives.
  \item Resistance (\texttt{R}) is the only
    instance or model parameters that are allowed to be
    solution-dependent.
\end{XyceItemize}


%%
%% Diode
%%
\clearpage
\subsection{Diode}
\index{device!diode} \index{diode}
% Sandia National Laboratories is a multimission laboratory managed and
% operated by National Technology & Engineering Solutions of Sandia, LLC, a
% wholly owned subsidiary of Honeywell International Inc., for the U.S.
% Department of Energy’s National Nuclear Security Administration under
% contract DE-NA0003525.

% Copyright 2002-2024 National Technology & Engineering Solutions of Sandia,
% LLC (NTESS).


\begin{Device}\label{D_DEVICE}

\symbol
{\includegraphics{diodeSymbol}}

\device
D<name> <(+) node> <(-) node> <model name> [area value]

\model
.MODEL <model name> D [model parameters]

\examples
\begin{alltt}
  DCLAMP 1 0 DMOD
  D2 15 17 SWITCH 1.5
\end{alltt}

\parameters
\begin{Parameters}

\param{\vbox{\hbox{(+) node\hfil}\hbox{(-) node}}}

The anode and the cathode.

\param{area value} 

Scales IS, ISR, IKF, RS, CJO, and IBV, and has a default value of 1.
IBV and BV are both specified as positive values.

\param {PJ value}

Used in computing the junction sidewall effects, and has a default
value of zero (no sidewall effects).

\end{Parameters}

\comments

The diode is modeled as an ohmic resistance (\texttt{RS/area}) in series
with an intrinsic diode.  Positive current is current flowing from the
anode through the diode to the cathode. 

The power through the diode is calculated 
with $I \cdot \Delta V$ where the voltage drop is calculated as $(V_+ - V_-)$ 
and positive current flows from $V_+$ to $V_-$.  This formula may differ from
other simulators, such as HSPICE.

\end{Device}

\paragraph{Diode Operating Temperature}
\index{diode!operating temperature} Model parameters can be assigned unique
measurement temperatures using the \textrmb{TNOM} model parameter.

\paragraph{Diode level selection}

Several distinct implementations of the diode are available.  These are
selected by using the \verb|LEVEL| model parameter.  The default
implementation is based on SPICE 3F5, and may be explicitly specified
using \verb|LEVEL=1| in the model parameters, but is also selected if no
\verb|LEVEL| parameter is specified.  The PSpice implementation
~\cite{PSpiceUG:1998} is obtained by specifying \verb|LEVEL=2|.
The \Xyce{} \verb|LEVEL=200| diode is the JUNCAP200 model.
The \Xyce{} \verb|LEVEL=2002| diode is the DIODE\_CMC model version 2.0.0.


\pagebreak

\paragraph{Level 1 and 2 Diode Instance Parameters}
% This table was generated by Xyce:
%   Xyce -doc D 1
%
\index{diode!device instance parameters}
\begin{DeviceParamTableGenerated}{Diode Device Instance Parameters}{D_1_Device_Instance_Params}
AREA & Area scaling value (scales IS, ISR, IKF, RS, CJ0, and IBV) & -- & 1 \\ \hline
DTEMP & Device delta temperature & $^\circ$C & 0 \\ \hline
IC &  & -- & 0 \\ \hline
M & multiplicity factor & -- & 1 \\ \hline
OFF & Initial voltage drop across device set to zero & logical (T/F) & 0 \\ \hline
PJ & Perimeter scaling value & -- & 0 \\ \hline
TEMP & Device temperature & -- & Ambient Temperature \\ \hline
\end{DeviceParamTableGenerated}


\paragraph{Level 1 and 2 Diode Model Parameters}
% This table was generated by Xyce:
%   Xyce -doc D 1
%
\index{diode!device model parameters}
\begin{DeviceParamTableGenerated}{Diode Device Model Parameters}{D_1_Device_Model_Params}
AF & Flicker noise exponent & -- & 1 \\ \hline
BV & Reverse breakdown "knee" voltage & V & 1e+99 \\ \hline
CJ & Zero-bias p-n depletion capacitance & F & 0 \\ \hline
CJ0 & Zero-bias p-n depletion capacitance & F & 0 \\ \hline
CJO & Zero-bias p-n depletion capacitance & F & 0 \\ \hline
CJP & Sidewall junction capacitance (alias for CJSW) & F & 0 \\ \hline
CJSW & Sidewall junction capacitance & F & 0 \\ \hline
EG & Bandgap voltage (barrier height) & eV & 1.11 \\ \hline
FC & Forward-bias depletion capacitance coefficient & -- & 0.5 \\ \hline
FCS & Forward-bias sidewall depletion capacitance coefficient & -- & 0.5 \\ \hline
IBV & Reverse breakdown "knee" current & A & 0.001 \\ \hline
IBVL & Low-level reverse breakdown "knee" current & A & 0 \\ \hline
IKF & High-injection "knee" current & A & 0 \\ \hline
IS & Saturation current & A & 1e-14 \\ \hline
ISR & Recombination current parameter & A & 0 \\ \hline
JS & Saturation current & A & 1e-14 \\ \hline
JSW & Sidewall Saturation current & A & 0 \\ \hline
KF & Flicker noise coefficient & -- & 0 \\ \hline
M & Grading parameter for p-n junction & -- & 0.5 \\ \hline
MJSW & Grading parameter for sidewall junction & -- & 0.33 \\ \hline
N & Emission coefficient & -- & 1 \\ \hline
NBV & Reverse breakdown ideality factor & -- & 1 \\ \hline
NBVL & Low-level reverse breakdown ideality factor & -- & 1 \\ \hline
NR & Emission coefficient for ISR & -- & 2 \\ \hline
NS & Sidewall emission coefficient & -- & 1 \\ \hline
PHP & Potential for sidewall junction & V & 1 \\ \hline
RS & Parasitic resistance & $\mathsf{\Omega}$ & 0 \\ \hline
TBV1 & BV temperature coefficient (linear) & $^\circ$C$^{-1}$ & 0 \\ \hline
TBV2 & BV temperature coefficient (quadratic) & $^\circ$C$^{-2}$ & 0 \\ \hline
TIKF & IKF temperature coefficient (linear) & $^\circ$C$^{-1}$ & 0 \\ \hline
TNOM &  & -- & Ambient Temperature \\ \hline
TRS & RS temperature coefficient (linear) (alias for TRS1) & $^\circ$C$^{-1}$ & 0 \\ \hline
TRS1 & RS temperature coefficient (linear) & $^\circ$C$^{-1}$ & 0 \\ \hline
TRS2 & RS temperature coefficient (quadratic) & $^\circ$C$^{-2}$ & 0 \\ \hline
TT & Transit time & s & 0 \\ \hline
VB & Reverse breakdown "knee" voltage & V & 1e+99 \\ \hline
VJ & Potential for p-n junction & V & 1 \\ \hline
VJSW & Potential for sidewall junction (alias for PHP) & V & 1 \\ \hline
XTI & IS temperature exponent & -- & 3 \\ \hline
\end{DeviceParamTableGenerated}


\paragraph{JUNCAP200 (level=200) Parameters}
The JUNCAP200 model has the instance and model parameters in the
tables below.  Complete documentation of JUNCAP200 may be found at
\url{http://www.cea.fr/cea-tech/leti/pspsupport/Documents/juncap200p5_summary.pdf}.


The JUNCAP200 device supports output of the internal variables in
table~\ref{D_200_OutputVars} on the \texttt{.PRINT} line of a netlist.
To access them from a print line, use the syntax
\texttt{N(<instance>:<variable>)} where ``\texttt{<instance>}'' refers to the
name of the specific level 200 D device in your netlist.

\input{D_200_Device_Instance_Params}
\input{D_200_Device_Model_Params}
\input{D_200_OutputVars}

\paragraph{DIODE\_CMC (level=2002) Parameters}
The DIODE\_CMC model has the instance and model parameters in the
tables below.  Complete documentation of DIODE\_CMC may be found at
\url{https://si2.org/standard-models}.


The DIODE\_CMC device supports output of the internal variables in
table~\ref{D_2002_OutputVars} on the \texttt{.PRINT} line of a netlist.
To access them from a print line, use the syntax
\texttt{N(<instance>:<variable>)} where ``\texttt{<instance>}'' refers to the
name of the specific level 2002 D device in your netlist.

% This table was generated by Xyce:
%   Xyce -doc D 2002
%
\index{diodecmc 2.0.0!device instance parameters}
\begin{DeviceParamTableGenerated}{DIODE\_CMC 2.0.0 Device Instance Parameters}{D_2002_Device_Instance_Params}
AB & Junction area & m$^{2}$ & 1e-12 \\ \hline
AREA &  Alias for AB & m$^{2}$ & 1e-12 \\ \hline
LG & Gate-edge part of junction perimeter & m & 0 \\ \hline
LS & STI-edge part of junction perimeter & m & 1e-06 \\ \hline
MULT & Number of devices in parallel & --- & 1 \\ \hline
PERIM &  Alias for LS & m & 1e-06 \\ \hline
PJ &  Alias for LS & m & 1e-06 \\ \hline
\end{DeviceParamTableGenerated}

% This table was generated by Xyce:
%   Xyce -doc D 2002
%
\index{diodecmc 2.0.0!device model parameters}
\begin{DeviceParamTableGenerated}{DIODE\_CMC 2.0.0 Device Model Parameters}{D_2002_Device_Model_Params}
ABMAX & maximum allowed junction area & m$^{2}$ & 1 \\ \hline
ABMIN & minimum allowed junction area & m$^{2}$ & 0 \\ \hline
AF & AF parameter for flicker noise & --- & 1 \\ \hline
CBBTBOT & Band-to-band tunneling prefactor of bottom component & A/V$^{3}$ & 1e-12 \\ \hline
CBBTGAT & Band-to-band tunneling prefactor of gate-edge component & Am/V$^{3}$ & 1e-18 \\ \hline
CBBTSTI & Band-to-band tunneling prefactor of STI-edge component & Am/V$^{3}$ & 1e-18 \\ \hline
CJORBOT & Zero-bias capacitance per unit-of-area of bottom component & F/m$^{2}$ & 0.001 \\ \hline
CJORGAT & Zero-bias capacitance per unit-of-length of gate-edge component & F/m & 1e-09 \\ \hline
CJORSTI & Zero-bias capacitance per unit-of-length of STI-edge component & F/m & 1e-09 \\ \hline
CORECOVERY & Flag for recovery equations; 0=original, 1=Hiroshima & --- & 0 \\ \hline
CSRHBOT & Shockley-Read-Hall prefactor of bottom component & A/m$^{3}$ & 100 \\ \hline
CSRHGAT & Shockley-Read-Hall prefactor of gate-edge component & A/m$^{2}$ & 0.0001 \\ \hline
CSRHSTI & Shockley-Read-Hall prefactor of STI-edge component & A/m$^{2}$ & 0.0001 \\ \hline
CTATBOT & Trap-assisted tunneling prefactor of bottom component & A/m$^{3}$ & 100 \\ \hline
CTATGAT & Trap-assisted tunneling prefactor of gate-edge component & A/m$^{2}$ & 0.0001 \\ \hline
CTATSTI & Trap-assisted tunneling prefactor of STI-edge component & A/m$^{2}$ & 0.0001 \\ \hline
DEPNQS & Depletion delay time & --- & 0 \\ \hline
DTA & Temperature offset with respect to ambient temperature & C & 0 \\ \hline
FBBTRBOT & Normalization field at the reference temperature for band-to-band tunneling of bottom component & Vm$^{-1}$ & 1e+09 \\ \hline
FBBTRGAT & Normalization field at the reference temperature for band-to-band tunneling of gate-edge component & Vm$^{-1}$ & 1e+09 \\ \hline
FBBTRSTI & Normalization field at the reference temperature for band-to-band tunneling of STI-edge component & Vm$^{-1}$ & 1e+09 \\ \hline
FJUNQ & Fraction below which junction capacitance components are considered negligible & --- & 0.03 \\ \hline
FREV & Additional parameter for current after breakdown & --- & 1000 \\ \hline
IDSATRBOT & Saturation current density at the reference temperature of bottom component & A/m$^{2}$ & 1e-12 \\ \hline
IDSATRGAT & Saturation current density at the reference temperature of gate-edge component & A/m & 1e-18 \\ \hline
IDSATRSTI & Saturation current density at the reference temperature of STI-edge component & A/m & 1e-18 \\ \hline
IMAX & Maximum current up to which forward current behaves exponentially & A & 1000 \\ \hline
INJ1 & For carrier density & --- & 1 \\ \hline
INJ2 & For carrier density in high-injection condition & --- & 10 \\ \hline
INJT & Temp. co of carrier density in high-injection condition & --- & 0 \\ \hline
KF & KF parameter for flicker noise & --- & 0 \\ \hline
LGMAX & maximum allowed junction gate-edge & m & 1 \\ \hline
LGMIN & minimum allowed junction gate-edge & m & 0 \\ \hline
LSMAX & maximum allowed junction STI-edge & m & 1 \\ \hline
LSMIN & minimum allowed junction STI-edge & m & 0 \\ \hline
MEFFTATBOT & Effective mass (in units of m0) for trap-assisted tunneling of bottom component & --- & 0.25 \\ \hline
MEFFTATGAT & Effective mass (in units of m0) for trap-assisted tunneling of gate-edge component & --- & 0.25 \\ \hline
MEFFTATSTI & Effective mass (in units of m0) for trap-assisted tunneling of STI-edge component & --- & 0.25 \\ \hline
NDIBOT & Doping concentration of drift region & --- & 1e+16 \\ \hline
NDIGAT & Doping concentration of drift region & --- & 1e+16 \\ \hline
NDISTI & Doping concentration of drift region & --- & 1e+16 \\ \hline
NFABOT & ideality factor bottom component & --- & 1 \\ \hline
NFAGAT & ideality factor gate-edge component & --- & 1 \\ \hline
NFASTI & ideality factor STI-edge component & --- & 1 \\ \hline
NJDV & Transition slope of emission coefficient & --- & 0.1 \\ \hline
NJH & High-injection emission coefficient & --- & 1 \\ \hline
NQS & Carrier delay time & --- & 5e-09 \\ \hline
PBOT & Grading coefficient of bottom component & --- & 0.5 \\ \hline
PBRBOT & Breakdown onset tuning parameter of bottom component & V & 4 \\ \hline
PBRGAT & Breakdown onset tuning parameter of gate-edge component & V & 4 \\ \hline
PBRSTI & Breakdown onset tuning parameter of STI-edge component & V & 4 \\ \hline
PGAT & Grading coefficient of gate-edge component & --- & 0.5 \\ \hline
PHIGBOT & Zero-temperature bandgap voltage of bottom component & V & 1.16 \\ \hline
PHIGGAT & Zero-temperature bandgap voltage of gate-edge component & V & 1.16 \\ \hline
PHIGSTI & Zero-temperature bandgap voltage of STI-edge component & V & 1.16 \\ \hline
PSTI & Grading coefficient of STI-edge component & --- & 0.5 \\ \hline
PT &  Alias for XTI & --- & 3 \\ \hline
REVISION & Model revision & --- & 0 \\ \hline
RSBOT & Series resistance per unit-of-area of bottom component & --- & 0 \\ \hline
RSCOM & Common series resistance, no scaling  & --- & 0 \\ \hline
RSGAT & Series resistance per unit-of-length of gate-edge component & --- & 0 \\ \hline
RSSTI & Series resistance per unit-of-length of STI-edge component & --- & 0 \\ \hline
SCALE & Scale parameter & --- & 1 \\ \hline
SHRINK & Scale parameter & --- & 0 \\ \hline
STFBBTBOT & Temperature scaling parameter for band-to-band tunneling of bottom component & 1/K & -0.001 \\ \hline
STFBBTGAT & Temperature scaling parameter for band-to-band tunneling of gate-edge component & 1/K & -0.001 \\ \hline
STFBBTSTI & Temperature scaling parameter for band-to-band tunneling of STI-edge component & 1/K & -0.001 \\ \hline
STRS & Temperature scaling parameter for series resistance & --- & 0 \\ \hline
STVBRBOT1 & Temp. co of breakdown voltage bottom component & 1/K & 0 \\ \hline
STVBRBOT2 & Temp. co of breakdown voltage bottom component & --- & 0 \\ \hline
STVBRGAT1 & Temp. co of breakdown voltage gate-edge component & 1/K & 0 \\ \hline
STVBRGAT2 & Temp. co of breakdown voltage gate-edge component & --- & 0 \\ \hline
STVBRSTI1 & Temp. co of breakdown voltage STI-edge component & 1/K & 0 \\ \hline
STVBRSTI2 & Temp. co of breakdown voltage STI-edge component & --- & 0 \\ \hline
SUBVERSION & Model subversion & --- & 0 \\ \hline
SWJUNEXP & Flag for JUNCAP-express; 0=full model, 1=express model & --- & 0 \\ \hline
TAU & Carrier lifetime & --- & 2e-07 \\ \hline
TAUT & Temp. co of carrier lifetime & --- & 0 \\ \hline
TEMPMAX & maximum allowed junction temp & C & 155 \\ \hline
TEMPMIN & minimum allowed junction temp & C & -55 \\ \hline
TNOM & Alias reference temperature & C & 21 \\ \hline
TRJ & Reference temperature & C & 21 \\ \hline
TT & Transit time & s & 0 \\ \hline
TYPE & Type parameter, in output value 1 reflects n-type, -1 reflects p-type & --- & 1 \\ \hline
VBIRBOT & Built-in voltage at the reference temperature of bottom component & V & 1 \\ \hline
VBIRGAT & Built-in voltage at the reference temperature of gate-edge component & V & 1 \\ \hline
VBIRSTI & Built-in voltage at the reference temperature of STI-edge component & V & 1 \\ \hline
VBRBOT & Breakdown voltage of bottom component & V & 10 \\ \hline
VBRGAT & Breakdown voltage of gate-edge component & V & 10 \\ \hline
VBRSTI & Breakdown voltage of STI-edge component & V & 10 \\ \hline
VERSION & Model version & --- & 2 \\ \hline
VFMAX & maximum allowed forward junction bias & V & 0 \\ \hline
VJUNREF & Typical maximum junction voltage; usually about 2*VSUP & --- & 2.5 \\ \hline
VRMAX & maximum allowed reverse junction bias & V & 0 \\ \hline
WI & Length of drift region & m & 5e-06 \\ \hline
XJUNGAT & Junction depth of gate-edge component & m & 1e-07 \\ \hline
XJUNSTI & Junction depth of STI-edge component & m & 1e-07 \\ \hline
XTI & Temp. co of saturation current & --- & 3 \\ \hline
\end{DeviceParamTableGenerated}

%table generated from Verilog-A input
\index{DIODE level 2002!device output variables}
\begin{DeviceParamTableGenerated}{Diode level 2002 Output Variables}{D_2002_OutputVars}
vak & Voltage between anode and cathode excluding the series resistor &   V & none \\ \hline
cj & Total source junction capacitance &   F & none \\ \hline
cjbot & Junction capacitance (bottom component) &   F & none \\ \hline
cjgat & Junction capacitance (gate-edge component) &   F & none \\ \hline
cjsti & Junction capacitance (STI-edge component) &   F & none \\ \hline
ij & Total source junction current &   A & none \\ \hline
ijbot & Junction current (bottom component) &   A & none \\ \hline
ijgat & Junction current (gate-edge component) &   A & none \\ \hline
ijsti & Junction current (STI-edge component) &   A & none \\ \hline
si & Total junction current noise spectral density &   A$^{2}$/Hz & none \\ \hline
vrs & Voltage across series resistor &   V & none \\ \hline
sf & Total junction flicker noise spectral density &   A$^{2}$/Hz & none \\ \hline
sr & Total series resistor thermal noise spectral density &   A$^{2}$/Hz & none \\ \hline
rseries & Series resistor &   V/A & none \\ \hline
qrr & Recovery charge &   C & none \\ \hline
\end{DeviceParamTableGenerated}


\paragraph{Level 1 Diode Equations}

The equations in this section use the following variables:
\begin{eqnarray*}
V_{di} & = & \mbox{voltage across the intrinsic diode only} \\
V_{th} & = & \mbox{$k \cdot T/q$ (thermal voltage)}         \\
k      & = & \mbox{Boltzmann's constant}                    \\
q      & = & \mbox{electron charge}                         \\
T      & = & \mbox{analysis temperature (Kelvin)}           \\
T_{0}  & = & \mbox{nominal temperature (set using \textrmb{TNOM}
option)} \\
\omega & = & \mbox{Frequency (Hz)}
\end{eqnarray*}
Other variables are listed above in the diode model parameters.

\subparagraph{Level=1}
The level 1 diode is based on the Spice3f5 level 1 model.

\subparagraph{DC Current (Level=1)}

The intrinsic diode current consists of forward and reverse bias regions where
$$
I_D = \left\{ \begin{array}{ll}
\mathbf{IS}\cdot\left[\exp \left(\frac{V_{di}}{\mathbf{N}V_{th}}\right) - 1
\right], & V_{di} > -3.0\cdot\mathbf{N}V_{th} \\
-\mathbf{IS}\cdot\left[1.0 + \left(\frac{3.0\cdot\mathbf{N}V_{th}}{V_{di}\cdot
e}\right)^3\right], & V_{di} < -3.0\cdot\mathbf{N}V_{th}
\end{array}
\right.
$$


When $\bBV$ and an optional parameter $\bIBV$ are explicitly given in the model
statement, an exponential model is used to model reverse breakdown (with a
``knee'' current of $\bIBV$ at a ``knee-on'' voltage of $\bBV$).  The equation
for $I_D$ implemented by \Xyce{} is given by

\[
I_D = -\bIBVeff\cdot\exp \left(-\frac{\bBVeff + V_{di}}{\mathbf{N}V_{th}}
\right), \hspace*{0.5in} V_{di} \leq \bBVeff,
\]
where $\bBVeff$ and $\bIBVeff$ are chosen to satisfy the following
constraints:
\begin{enumerate}
\item Continuity of $I_D$ between reverse bias and reverse breakdown regions
(i.e., continuity of $I_D$ at $V_{di}=-\bBVeff$):
\[
\bIBVeff=\bIS\left(1-\left(\frac{3.0\cdot\bN V_{th}}{e\cdot\bBVeff}
\right)^3\right)
\]
\item ``Knee-on'' voltage/current matching:
\[
\bIBVeff\cdot\exp\left(-\frac{\bBVeff-\bBV}{\bN V_{th}}\right)=\bIBV
\]
\end{enumerate}
Substituting the first expression into the second yields a single constraint
on $\bBVeff$ which cannot be solved for directly.  By performing some basic
algebraic manipulation and rearranging terms, the problem of finding
$\bBVeff$ which satisfies the above two constraints can be cast as finding
the (unique) solution of the equation
\begin{equation}
\bBVeff = f(\bBVeff),
\label{eqn:diode_picardeqn}
\end{equation}
where $f(\cdot)$ is the function that is obtained by solving for the
$\bBVeff$ term which appears in the exponential in terms of $\bBVeff$ and the
other parameters.  \Xyce{} solves Eqn.\ \ref{eqn:diode_picardeqn} by performing
the so-called {\em Picard Iteration} procedure \cite{mattuck:1999}, i.e.
by producing successive estimates of $\bBVeff$ (which we will denote as
$\bBVeff ^ k$) according to
\[
\bBVeff^{k+1}=f(\bBVeff^k)
\]
starting with an initial guess of $\bBVeff^0=\bBV$.  The current iteration
procedure implemented in \Xyce{} can be shown to guarantee at least six
significant digits of accuracy between the numerical estimate of $\bBVeff$ and
the true value.

In addition to the above, \Xyce{} also requires that $\bBVeff$ lie in the
range \mbox{$\bBV \geq \bBVeff \geq 3.0\bN V_{th}$.}  In terms of $\bIBV   $,
this is equivalent to enforcing the following two constraints:
\begin{eqnarray}
\bIS\left(1-\left(\frac{3.0\cdot\bN V_{th}}{e\cdot\bBV}\right)^3\right) & \leq & \bIBV \label{eqn:diode_ibvconstr1}\\
\bIS\left(1-e^{-3}\right)\exp\left(\frac{-3.0\cdot\bN V_{th}+
\bBV}{\bN V_{th}}\right) & \geq & \bIBV \label{eqn:diode_ibvconstr2}
\end{eqnarray}
\Xyce{} first checks the value of $\bIBV$ to ensure that the above two
constraints are satisfied.  If Eqn.\ \ref{eqn:diode_ibvconstr1} is violated,
\Xyce{} sets $\bIBVeff$ to be equal to the left-hand side of Eqn.\
\ref{eqn:diode_ibvconstr1} and, correspondingly, sets $\bBVeff$ to $-3.0\cdot
\bN V_{th}.$  If Eqn.\ \ref{eqn:diode_ibvconstr2} is violated, \Xyce{} sets
$\bIBVeff$ to be equal to the left-hand side of Eqn.\
\ref{eqn:diode_ibvconstr2} and, correspondingly, sets $\bBVeff$ to $\bBV$.

\subparagraph{Capacitance (Level=1)}
The p-n diode capacitance consists of a depletion layer capacitance $C_d$ and a
diffusion capacitance $C_{dif}$.  The first is given by
\[
C_d = \left\{ \begin{array}{ll}
\mathbf{CJ}\cdot\mathbf{AREA} \left(1-\frac{V_{di}}{\mathbf{VJ}} \right)^{-\mathbf{M}}, &
V_{di} \leq \mathbf{FC \cdot VJ} \\
\frac{\mathbf{CJ}\cdot\mathbf{AREA}}{\mathbf{F2}}\left(\mathbf{F3}+\mathbf{M}\frac{V_{di}}{\mathbf{VJ}}\right),
& V_{di} > \mathbf{FC \cdot VJ}
\end{array}
\right.
\]
The diffusion capacitance (sometimes referred to as the transit time
capacitance) is
\[
C_{dif} = \mathbf{TT}G_d = \mathbf{TT}\frac{dI_D}{dV_{di}}
\]
where $G_d$ is the junction conductance.

\subparagraph{Sidewall currents and capacitances}
When the instance parameter \texttt{PJ} (perimeter scaling value) is
specified, the diode currents become the sum of the currents above
(the ``bottom'' of the junction) and those of the periphery (sidewall).

In normal forward and reverse bias regions, the sidewall currents are given by:
$$
I_{D,SW} = \left\{ \begin{array}{ll}
\mathbf{ISatSW}\cdot\left[\exp \left(\frac{V_{di}}{\mathbf{NS}V_{th}}\right) - 1
\right], & V_{di} > -3.0\cdot\mathbf{NS}V_{th} \\
-\mathbf{ISatSW}\cdot\left[1.0 + \left(\frac{3.0\cdot\mathbf{NS}V_{th}}{V_{di}\cdot
e}\right)^3\right], & V_{di} < -3.0\cdot\mathbf{NS}V_{th}
\end{array}
\right.
$$

where $\mathbf{ISatSW}$ is the temperature-adjusted value of JSW multiplied
by the perimeter $\mathbf{PJ}$.

When the breakdown voltage \texttt{BV} has been given and the diode voltage is below $-\mathbf{BV}$, the sidewall current is:
\[
I_{D,sw} = -\mathbf{ISatSW}\cdot\exp \left(-\frac{\mathbf{BVeff} + V_{di}}{\mathbf{NS}V_{th}}
\right), \hspace*{0.5in} V_{di} \leq \mathbf{BVeff},
\]

The sidewall capacitances are computed as:
\[
C_{d,sw} = \left\{ \begin{array}{ll}
\mathbf{CJSW}\cdot\mathbf{PJ} \left(1-\frac{V_{di}}{\mathbf{PHP}} \right)^{-\mathbf{M}}, &
V_{di} \leq \mathbf{FCS \cdot PHP} \\
\frac{\mathbf{CJSW}\cdot\mathbf{PJ}}{\mathbf{F2SW}}\left(\mathbf{F3SW}+\mathbf{MJSW}\frac{V_{di}}{\mathbf{PHP}}\right),
& V_{di} > \mathbf{FCS \cdot PHP}
\end{array}
\right.
\]


\subparagraph{Temperature Effects (Level=1)}
The diode model contains explicit temperature dependencies in the ideal diode
current, the generation/recombination current and the breakdown current.
Further temperature dependencies are present in the diode model via the
saturation current $I_{S}$, the depletion layer junction capacitance $CJ$,
the junction potential $V_J$.
%insert equations here (3)
\begin{eqnarray*}
V_t(T) & = & \frac{kT}{q} \\
V_{tnom}(T) & = & \frac{k\mathbf{TNOM}}{q} \\
E_g(T) & = & E_{g0} - \frac{\alpha T^2}{\beta + T} \\
E_{gNOM}(T) & = & E_{g0} - \frac{\alpha\mathbf{TNOM}^2}{\mathbf{TNOM}+\beta} \\
arg1(T) & = & -\frac{E_g(T)}{2kT} + \frac{E_{g300}}{2kT_0} \\
arg2(T) & = & -\frac{E_{gNOM}(T)}{2k\mathbf{TNOM}} + \frac{E_{g300}}{2kT_0} \\
pbfact1(T) & = & -2.0\cdot V_t(T) \left(1.5\cdot\ln \left(\frac{T}{T_0}\right) + q\cdot arg1(T)\right) \\
pbfact2(T) & = & -2.0\cdot V_{tnom}(T) \left(1.5\cdot\ln \left(\frac{\mathbf{TNOM}}{T_0}\right) + q\cdot arg2(T)\right) \\
pbo(T) & = & \left(\mathbf{VJ}-pbfact2(T)\right)\frac{T_0}{\mathbf{TNOM}} \\
V_J(T) & = & pbfact1(T) + \frac{T}{T_0}pbo(T) \\
gma_{old}(T) & = & \frac{\mathbf{VJ}-pbo(T)}{pbo(T)} \\
gma_{new}(T) & = & \frac{V_J(T)-pbo(T)}{pbo(T)} \\
CJ(T) & = & \mathbf{CJ0}\frac{1.0+\mathbf{M}\left(4.0\times 10^{-4}\left(T-T_0\right)-gma_{new}(T)\right)}{1.0 + \mathbf{M}\left(4.0\times 10^{-4}\left(\mathbf{TNOM}-T_0\right)-gma_{old}(T)\right)} \\
I_S(T) & = & \mathbf{IS} \cdot\exp \left(\left(\frac{T}{\mathbf{TNOM}}-1.0\right) \cdot \frac{\mathbf{EG}}{\mathbf{N}V_t(T)} + \frac{\mathbf{XTI}}{\mathbf{N}} \cdot \ln \left(\frac{T}{\mathbf{TNOM}}\right)\right) \\
\end{eqnarray*}
where, for silicon, $\alpha = 7.02\times 10^{-4}\;eV/K$, $\beta =
1108\; K$ and $E_{g0} = 1.16\;eV$.

%\subparagraph{Noise}
%There is no noise model in this version of \Xyce{}.
%Noise is calculated with a 1.0 Hz bandwidth using the following spectral power
%densities (per unit bandwidth).
%
%Thermal Noise due to Parasitic Resistance: $I_n^2 = \frac{4kT}
%{\mathbf{RS}/\mbox{area}}$ \\
%Intrinsic Diode Shot and Flicker Noise: $I_n^2 = 2qI_D + \mathbf{KF} \cdot
%\frac{I_D^{\mathbf{AF}}}{\omega}$
%
For a more thorough description of p-n junction physics, see [9].  For a
thorough description of the U.C. Berkeley SPICE models see Reference [11].


%%
%% Independent Current Source
%%
\clearpage
\subsection{Independent Current Source}
\index{device!independent voltage source} \index{independent voltage source}
\index{current source!independent}
\label{IndependentCurrentSource}
% Sandia National Laboratories is a multimission laboratory managed and
% operated by National Technology & Engineering Solutions of Sandia, LLC, a
% wholly owned subsidiary of Honeywell International Inc., for the U.S.
% Department of Energy’s National Nuclear Security Administration under
% contract DE-NA0003525.

% Copyright 2002-2024 National Technology & Engineering Solutions of Sandia,
% LLC (NTESS).


\begin{Device}\label{I_DEVICE}

\symbol
{\includegraphics{icsSymbol}}

\device
\begin{alltt}
I<name> <(+) node> <(-) node> [ [DC] <value> ]
+ [AC [magnitude value [phase value] ] ] [transient specification]
\end{alltt}

\examples
\begin{alltt}
ISLOW 1 22 SIN(0.5 1.0ma 1KHz 1ms)
IPULSE 1 3 PULSE(-1 1 2ns 2ns 2ns 50ns 100ns)
IPAT 2 4 PAT(5 0 0 1n 2n 5n b0101)
\end{alltt}

\parameters
\begin{Parameters}

\param{transient specification}

There are five predefined time-varying functions for sources:

\begin{description}
\item[\tt PULSE <parameters>] Pulse waveform
\item[\tt SIN <parameters>] Sinusoidal waveform
\item[\tt EXP <parameters>] Exponential waveform
\item[\tt PAT <parameters>] Pattern waveform
\item[\tt PWL <parameters>] Piecewise linear waveform
\item[\tt SFFM <parameters>] Frequency-modulated waveform
\end{description}

\end{Parameters}

\comments

Positive current flows from the positive node through the source to the
negative node. 

The power supplied or dissipated by the current source is calculated 
with $I \cdot \Delta V$ where the voltage drop is calculated as $(V_+ - V_-)$ 
and positive current flows from $V_+$ to $V_-$.  Dissipated power has a
positive sign, while supplied power has a negative sign.

The default value is zero for the DC, AC, and transient
values. None, any, or all of the DC, AC, and transient values can be
specified. The AC phase value is in degrees.

\end{Device}

\paragraph{Transient Specifications}
This section outlines the available transient specifications. $\Delta t$ and
$T_{F}$ are the time step size and simulation end-time, respectively. 
Parameters marked as -- must have a value specified for them;
otherwise a netlist parsing error will occur.

\subparagraph{Pulse}
\begin{alltt}
PULSE(V1 V2 TD TR TF PW PER)
\end{alltt}

\begin{DeviceParamTable}{Pulse Parameters}
V1 & Initial Value & amp & -- \\ \hline
V2 & Pulse Value & amp & 0.0 \\ \hline
TD & Delay Time & s & 0.0 \\ \hline
TR & Rise Time & s & $\Delta t$ \\ \hline
TF & Fall Time & s & $\Delta t$ \\ \hline
PW & Pulse Width & s & $T_F$ \\ \hline
PER & Period & s & $T_F$ \\ \hline
\end{DeviceParamTable}

\subparagraph{Sine}
\begin{alltt}
SIN(V0 VA FREQ TD THETA PHASE)
\end{alltt}

\begin{DeviceParamTable}{Sine Parameters}
V0 & Offset                & amp & --   \\ \hline
VA & Amplitude             & amp & --   \\ \hline
FREQ & Frequency           & s$^{-1}$  & --   \\ \hline
TD & Delay                 & s    & $\Delta t$ \\ \hline
THETA & Attenuation Factor & s    & $\Delta t$ \\ \hline
PHASE & Phase              & degrees & 0.0 \\ \hline 
\end{DeviceParamTable}

The waveform is shaped according to the following equations,
where $\phi=\pi*\mathbf{PHASE}/180$ :
\[
I = \left\{ \begin{array}{ll}
V_0, & 0 < t < T_D \\
V_0 + V_A \sin[2\pi\cdot\mathbf{FREQ}\cdot(t - T_D)+\phi]
\exp[-(t - T_D) \cdot \mathbf{THETA}], & T_D < t < T_F
\end{array}
\right.
\]

\subparagraph{Exponent}
\begin{alltt}
EXP(V1 V2 TD1 TAU1 TD2 TAU2)
\end{alltt}

\begin{DeviceParamTable}{Exponent Parameters}
V1 & Initial Amplitude & amp & -- \\ \hline
V2 & Amplitude & amp & -- \\ \hline
TD1 & Rise Delay Time & s & 0.0 \\ \hline
TAU1 & Rise Time Constant & s & $\Delta t$ \\ \hline
TD2 & Delay Fall Time & s & TD1 $+ \Delta t$ \\ \hline
TAU2 & Fall Time Constant & s & $\Delta t$ \\ \hline
\end{DeviceParamTable}

The waveform is shaped according to the following equations:
\[
I = \left\{ \begin{array}{ll}
V_1, & 0 < t < \mathrm{TD1} \\
V_1 + (V_2 - V_1) \{1 - \exp[-(t-\mathrm{TD1}) / \mathrm{TAU1}] \} ,
& \mathrm{TD1} < t < \mathrm{TD2} \\
V_1 + (V_2 - V_1) \{1 - \exp[-(t- \mathrm{TD1}) / \mathrm{TAU1}] \} \\
\;\;\;\;\, + (V_1 - V_2) \{1 - \exp[-(t - \mathrm{TD2}) / \mathrm{TAU2}] \} ,
& \mathrm{TD2} < t < T_2
\end{array}
\right. \]

\subparagraph{Pattern}
\begin{alltt}
PAT(VHI VLO TD TR TF TSAMPLE DATA R)
\end{alltt}

\begin{DeviceParamTable}{Pattern Parameters}
VHI & High Value & amp & -- \\ \hline
VLO & Low Value & amp & -- \\ \hline
TD & Delay Time & s & -- \\ \hline
TR & Rise Time & s & -- \\ \hline
TF & Fall Time & s & -- \\ \hline
TSAMPLE & Bit period & s & -- \\ \hline
DATA & Bit pattern & -- & -- \\ \hline
R & Repeat & --  & 0 \\ \hline
\end{DeviceParamTable}

The \texttt{VHI}, \texttt{VLO}, \texttt{TD}, \texttt{TF}, \texttt{TF}
\texttt{TSAMPLE} and \texttt{DATA} parameters are all required, and
hence have no default values.  Negative values for \texttt{TD} are
supported.  The \texttt{R} parameter is optional.  For its default
value of 0, the requested bit pattern will occur once.

The \texttt{DATA} parameter is the requested bit-pattern.  Only the
0' and `1' states are supported.  The `M' and `Z' states are not
supported.  The \texttt{DATA} field should have a leading `b' (or `B')
character (e.g., be specified as `b0101' ).

For times earlier than \texttt{TD}, the waveform value is set by the
first bit in \texttt{DATA}.  For times after the end of the (possibly
repeated) pattern, the waveform value is set by the last bit in
\texttt{DATA}.  Piecewise linear interpolation is used to generate
the output value when transitioning between states.

The \texttt{VHI}, \texttt{VLO}, \texttt{TD}, \texttt{TF}, \texttt{TF}
and \texttt{TSAMPLE} parameters are compatible with \texttt{.STEP}.
The \texttt{DATA} and \texttt{R} parameters are not.

The HSPICE parameters \texttt{RB}, \texttt{ENCODE} and \texttt{RD\_INIT},
for the pattern source, are not supported.

\subparagraph{Piecewise Linear}
\begin{alltt}
PWL  T0 V0 [Tn Vn]*
PWL  FILE "<name>" [TD=<timeDelay>] [R=<repeatTime>]
\end{alltt}

\begin{DeviceParamTable}{Piecewise Linear Parameters}
T$_n$ & Time at Corner & s & none \\ \hline
V$_n$ & Current at Corner & amp & none \\ \hline
TD & Time Delay & s & 0 \\ \hline
R & Repeat Time & s & none \\ \hline 
\end{DeviceParamTable}

When the FILE option is given, \Xyce{} will read the corner points
from the file specified in the \texttt{<name>} field.  This file
should be a plain ASCII text file (or a .CSV file) with the time/current pairs.  
There should be one pair per line, and the time and current values should be
separated by whitespace or commas.  As an example, the file 
specified (e.g., \texttt{ipwl.csv}) could have these five lines:
\begin{alltt}
0.00, 0.00
2.00, 3.00
3.00, 2.00
4.00, 2.00
4.01, 5.00
\end{alltt}
The corresponding example instance lines would be:
\begin{alltt}
IPWL1 1 0 PWL 0S 0A  2S 3A  3S 2A  4S 2A  4.01S 5A
IPWL2 2 0 PWL FILE "ipwl.txt"
IPWL3 3 0 PWL file "ipwl.csv"
IPWL4 4 0 PWL FILE ipwl.csv
\end{alltt}

The double quotes around the file name are optional, as shown above.

It is a best practice to specify all of the time-current pairs in the PWL specification.
However, for compatibility with HSPICE and PSpice, if the user-specified list of 
time/current pairs omits the pair at time=0 as the first pair in the list then \Xyce{} 
will insert a pair at time=0 with the current value at the first user-specified time value.  
As an example, this user-specified list:
\begin{alltt}
2S 3A  3S 2A  4S 2A  4.01S 5A
\end{alltt}
would be implemented in \Xyce{} as follows:  
\begin{alltt} 
0S 3A  2S 3A  3S 2A  4S 2A  4.01S 5A
\end{alltt}

TD has units of seconds, and specifies the length of time to delay
the start of PWL waveform.  The default is to have no delay, and TD is 
an optional parameter.

The Repeat Time (R) is an optional parameter.  If R is omitted then the waveform
will not repeat.  If R is included then the waveform will repeat until the end
of the simulation.  As examples, R=0 means repeat the PWL waveform from time=0
to the last time (T$_N$) specified in the waveform specification. (This would use the
time points 0s, 2s, 3s, 4s and 4.01s for the example waveform given above.).
In general, R=\texttt{<repeatTime>} means repeat the waveform from time equal to
\texttt{<repeatTime>} seconds in the waveform specification to the last time (T$_N$)
specified in the waveform specification. So, the \texttt{<repeatTime>} must be 
greater than or equal to 0 and less than the last time point (T$_N$).
If the R parameter is used then it must have a value.  

The specification \texttt {PWL  FILE "<name>"  R}  \texttt is illegal in \Xyce{} as
a shorthand for R=0.  Also, the \Xyce{} syntax for PWL sources is not compatible
with the PSpice \texttt{REPEAT} syntax for PWL sources.   See section 
\ref{PWL_SOURCE_SYNTAX_DIFF} for more details.

The repeat time (R) does enable the specification of discontinuous piecewise linear
waveforms.  For example, this waveform is a legal \Xyce{} syntax.  
\begin{alltt}
IPWL1 1 0 PWL 0S 0A  2S 3A  3S 2A  4S 2A  4.01S 5A  R=2
\end{alltt}
However, in general, discontinuous source waveforms may cause convergence problems.

\subparagraph{Frequency Modulated}
\begin{alltt}
SFFM (V0 VA FC MDI FS)
\end{alltt}

\begin{DeviceParamTable}{Frequency Modulated Parameters}
V0 & Offset & amp & -- \\ \hline
VA & Amplitude & amp & -- \\ \hline
FC & Carrier Frequency & hertz & 1/\textrmb{TSTOP} \\ \hline
MDI & Modulation Index & - & 0 \\ \hline
FS & Signal Frequency & hertz & 1/\textrmb{TSTOP} \\ \hline
\end{DeviceParamTable}

\textrmb{TSTOP} is the final time, as entered into the transient
(\texttt{.TRANS}) command. The waveform is shaped according to the following
equation:
\[
I = {V_0 + V_A} \cdot \sin(2\pi \cdot \mathrm{FC} \cdot \mathbf{TIME} +
\mathrm{MDI} \cdot \sin(2\pi \cdot \mathrm{FS} \cdot \mathbf{TIME}))
\]
where \textrmb{TIME} is the current simulation time.


%%
%% Independent Voltage Source
%%
\clearpage
\subsection{Independent Voltage Source}
\index{device!independent voltage source} \index{independent voltage source}
\index{voltage source!independent}
% Sandia National Laboratories is a multimission laboratory managed and
% operated by National Technology & Engineering Solutions of Sandia, LLC, a
% wholly owned subsidiary of Honeywell International Inc., for the U.S.
% Department of Energy’s National Nuclear Security Administration under
% contract DE-NA0003525.

% Copyright 2002-2024 National Technology & Engineering Solutions of Sandia,
% LLC (NTESS).


\begin{Device}\label{V_DEVICE}

\symbol
{\includegraphics{ivsSymbol}}

\device
\begin{alltt}
V<name> <(+) node> <(-) node> [ [DC] <value> ]
+ [AC [magnitude value [phase value] ] ] [transient specification]
\end{alltt}

\examples
\begin{alltt}
VSLOW 1 22 SIN(0.5 1.0mV 1KHz 1ms)
VPULSE 1 3 PULSE(-1 1 2ns 2ns 2ns 50ns 100ns)
VPAT 2 4 PAT(5 0 0 1n 2n 5n b0101)
\end{alltt}


\parameters

\begin{Parameters}

\param{transient specification}

There are five predefined time-varying functions for sources:

\begin{description}
\item[\tt PULSE <parameters>] Pulse waveform
\item[\tt SIN <parameters>] Sinusoidal waveform
\item[\tt EXP <parameters>] Exponential waveform
\item[\tt PAT <parameters>] Pattern waveform
\item[\tt PWL <parameters>] Piecewise linear waveform
\item[\tt SFFM <parameters>] Frequency-modulated waveform
\end{description}

\end{Parameters}

\comments

Positive current flows from the positive node
through the source to the negative node. 

The power supplied or dissipated by the voltage source is calculated 
with $I \cdot \Delta V$ where the voltage drop is calculated as $(V_+ - V_-)$ 
and positive current flows from $V_+$ to $V_-$.  Dissipated power has a
positive sign, while supplied power has a negative sign.

None, any, or all of the DC, AC, and transient values
can be specified. The AC phase value is in degrees. 

\end{Device}

\paragraph{Transient Specifications}
This section outlines the available transient specifications. $\Delta t$ and
$T_{F}$ are the time step size and simulation end-time, respectively.
Parameters marked as -- must have a value specified for them;
otherwise a netlist parsing error will occur.

\subparagraph{Pulse}
\begin{alltt}
PULSE(V1 V2 TD TR TF PW PER)
\end{alltt}

\begin{DeviceParamTable}{Pulse Parameters}
V1 & Initial Value & Volt & -- \\ \hline
V2 & Pulse Value & Volt & 0.0 \\ \hline
TD & Delay Time & s & 0.0 \\ \hline
TR & Rise Time & s & $\Delta t$ \\ \hline
TF & Fall Time & s & $\Delta t$ \\ \hline
PW & Pulse Width & s & $T_F$ \\ \hline
PER & Period & s & $T_F$ \\ \hline
\end{DeviceParamTable}

\subparagraph{Sine}
\begin{alltt}
SIN(V0 VA FREQ TD THETA PHASE)
\end{alltt}

\begin{DeviceParamTable}{Sine Parameters}
V0 & Offset                & Volt & --   \\ \hline
VA & Amplitude             & Volt & --   \\ \hline
FREQ & Frequency           & s$^{-1}$  & --   \\ \hline
TD & Delay                 & s    & $\Delta t$ \\ \hline
THETA & Attenuation Factor & s    & $\Delta t$ \\ \hline
PHASE & Phase              & degrees & 0.0 \\ \hline 
\end{DeviceParamTable}

The waveform is shaped according to the following equations, where $\phi=\pi*\mathbf{PHASE}/180$ :
\[
V = \left\{ \begin{array}{ll}
V_0, & 0 < t < T_D \\
V_0 + V_A \sin[2\pi\cdot\mathbf{FREQ}\cdot(t - T_D)+\phi]
\exp[-(t - T_D) \cdot \mathbf{THETA}], & T_D < t < T_F
\end{array}
\right.
\]

\subparagraph{Exponent}
\begin{alltt}
EXP(V1 V2 TD1 TAU1 TD2 TAU2)
\end{alltt}

\begin{DeviceParamTable}{Exponent Parameters}
V1 & Initial Amplitude & Volt & -- \\ \hline
V2 & Amplitude & Volt & -- \\ \hline
TD1 & Rise Delay Time & s & 0.0 \\ \hline
TAU1 & Rise Time Constant & s & $\Delta t$ \\ \hline
TD2 & Delay Fall Time & s & TD1 $+ \Delta t$ \\ \hline
TAU2 & Fall Time Constant & s & $\Delta t$ \\ \hline
\end{DeviceParamTable}

The waveform is shaped according to the following equations:
\[
V = \left\{ \begin{array}{ll}
V_1, & 0 < t < \mathrm{TD1} \\
V_1 + (V_2 - V_1) \{1 - \exp[-(t-\mathrm{TD1}) / \mathrm{TAU1}] \} ,
& \mathrm{TD1} < t < \mathrm{TD2} \\
V_1 + (V_2 - V_1) \{1 - \exp[-(t- \mathrm{TD1}) / \mathrm{TAU1}] \} \\
\;\;\;\;\, + (V_1 - V_2) \{1 - \exp[-(t - \mathrm{TD2}) / \mathrm{TAU2}] \} ,
& \mathrm{TD2} < t < T_2
\end{array}
\right. \]

\subparagraph{Pattern}
\begin{alltt}
PAT(VHI VLO TD TR TF TSAMPLE DATA R)
\end{alltt}

\begin{DeviceParamTable}{Pattern Parameters}
VHI & High Value & Volt & -- \\ \hline
VLO & Low Value & Volt & -- \\ \hline
TD & Delay Time & s & -- \\ \hline
TR & Rise Time & s & -- \\ \hline
TF & Fall Time & s & -- \\ \hline
TSAMPLE & Bit period & s & -- \\ \hline
DATA & Bit pattern & -- & -- \\ \hline
R & Repeat & --  & 0 \\ \hline
\end{DeviceParamTable}

The \texttt{VHI}, \texttt{VLO}, \texttt{TD}, \texttt{TF}, \texttt{TF}
\texttt{TSAMPLE} and \texttt{DATA} parameters are all required, and
hence have no default values.  Negative values for \texttt{TD} are
supported.  The \texttt{R} parameter is optional.  For its default
value of 0, the requested bit pattern will occur once.

The \texttt{DATA} parameter is the requested bit-pattern.  Only the
0' and `1' states are supported.  The `M' and `Z' states are not
supported.  The \texttt{DATA} field should have a leading `b' (or `B')
character (e.g., be specified as `b0101' ).

For times earlier than \texttt{TD}, the waveform value is set by the
first bit in \texttt{DATA}.  For times after the end of the (possibly
repeated) pattern, the waveform value is set by the last bit in
\texttt{DATA}.  Piecewise linear interpolation is used to generate
the output value when transitioning between states.

The relationship between the various source parameters can be
illustrated with the following example:

V1 1 0 PAT(5 0 0 1n 1n 5n b010)

That V1 source definition would produce time-voltages pairs at
(0 0) (4.5ns 0) (5ns 2.5) (5.5ns 5.0) (9.5ns 5.0)
(10ns 2.5) (10.5ns 0).  So, the bit period is 5ns and the
voltage value at the start/end of each ``sample'' is equal to
0.5*(\texttt{VHI} + \texttt{VLO}).  The first rise is centered
around t=5ns, and hence starts at t=4.5ns and ends at t=5.5 ns.

The \texttt{VHI}, \texttt{VLO}, \texttt{TD}, \texttt{TF}, \texttt{TF}
and \texttt{TSAMPLE} parameters are compatible with \texttt{.STEP}.
The \texttt{DATA} and \texttt{R} parameters are not.

The HSPICE parameters \texttt{RB}, \texttt{ENCODE} and \texttt{RD\_INIT},
for the pattern source, are not supported.

\subparagraph{Piecewise Linear}
\begin{alltt}
PWL  T0 V0 [Tn Vn]*
PWL  FILE "<name>" [TD=<timeDelay>] [R=<repeatTime>]
\end{alltt}

\begin{DeviceParamTable}{Piecewise Linear Parameters}
T$_n$ & Time at Corner & s & none \\ \hline
V$_n$ & Voltage at Corner & Volt & none \\ \hline
TD & Time Delay & s & 0 \\ \hline
R & Repeat Time & s & none \\ \hline 
\end{DeviceParamTable}

When the FILE option is given, \Xyce{} will read the corner points
from the file specified in the \texttt{<name>} field.  This file
should be a plain ASCII text file (or a .CSV file) with time/voltage pairs.  
There should be one pair per line, and the time and voltage values should be
separated by whitespace or commas. As an example, the file 
specified (e.g., \texttt{vpwl.csv}) could have these five lines:
\begin{alltt}
0.00, 0.00
2.00, 3.00
3.00, 2.00
4.00, 2.00
4.01, 5.00
\end{alltt}
The corresponding example instance lines would be:
\begin{alltt}
VPWL1 1 0 PWL 0S 0V  2S 3V  3S 2V  4S 2V  4.01S 5V
VPWL2 2 0 PWL FILE "vpwl.txt"
VPWL3 3 0 PWL file "vpwl.csv"
VPWL4 4 0 PWL FILE vpwl.csv
\end{alltt}

The double quotes around the file name are optional, as shown above.

It is a best practice to specify all of the time-voltage pairs in the PWL specification.
However, for compatibility with HSPICE and PSpice, if the user-specified list of 
time/voltage pairs omits the pair at time=0 as the first pair in the list then \Xyce{} 
will insert a pair at time=0 with the voltage value at the first user-specified time value.  
As an example, this user-specified list:
\begin{alltt}
2S 3V  3S 2V  4S 2V  4.01S 5V
\end{alltt}
would be implemented in \Xyce{} as follows:  
\begin{alltt} 
0S 3V  2S 3V  3S 2V  4S 2V  4.01S 5V
\end{alltt}

TD has units of seconds, and specifies the length of time to delay
the start of PWL waveform.  The default is to have no delay, and TD is 
an optional parameter.

The Repeat Time (R) is an optional parameter.  If R is omitted then the waveform
will not repeat.  If R is included then the waveform will repeat until the end
of the simulation.  As examples, R=0 means repeat the PWL waveform from time=0
to the last time (T$_N$) specified in the waveform specification. (This would use the
time points 0s, 2s, 3s, 4s and 4.01s for the example waveform given above.)
In general, R=\texttt{<repeatTime>} means repeat the waveform from time equal to
\texttt{<repeatTime>} seconds in the waveform specification to the last time (T$_N$)
specified in the waveform specification.  So, the \texttt{<repeatTime>} must be 
greater than or equal to 0 and less than the last time point (T$_N$).
If the R parameter is used then it must have a value.  

The specification \texttt {PWL  FILE "<name>"  R}  \texttt is illegal in \Xyce{} as
a shorthand for R=0.  Also, the \Xyce{} syntax for PWL sources is not compatible
with the PSpice \texttt{REPEAT} syntax for PWL sources. See section 
\ref{PWL_SOURCE_SYNTAX_DIFF} for more details.

The repeat time (R) does enable the specification of discontinuous piecewise linear
waveforms.  For example, this waveform is a legal \Xyce{} syntax.  
\begin{alltt}
VPWL1 1 0 PWL 0S 0V  2S 3V  3S 2V  4S 2V  4.01V 5V  R=2
\end{alltt}
However, in general, discontinuous source waveforms may cause convergence problems.

\subparagraph{Frequency Modulated}
\begin{alltt}
SFFM (V0 VA FC MDI FS)
\end{alltt}

\begin{DeviceParamTable}{Frequency Modulated Parameters}
V0 & Offset & Volt & -- \\ \hline
VA & Amplitude & Volt & -- \\ \hline
FC & Carrier Frequency & hertz & 1/\textrmb{TSTOP} \\ \hline
MDI & Modulation Index & - & 0 \\ \hline
FS & Signal Frequency & hertz & 1/\textrmb{TSTOP} \\ \hline
\end{DeviceParamTable}

\textrmb{TSTOP} is the final time, as entered into the transient
(\texttt{.TRANS}) command. The waveform is shaped according to the following
equation:
\[
V = {V_0 + V_A} \cdot \sin(2\pi \cdot \mathrm{FC} \cdot \mathbf{TIME} +
\mathrm{MDI} \cdot \sin(2\pi \cdot \mathrm{FS} \cdot \mathbf{TIME}))
\]
where \textrmb{TIME} is the current simulation time.


%%
%% Voltage-Controlled Voltage-Source Section
%%

%%
%% Port device 
%%
\clearpage
\subsection{Port Device}
\index{device!port device} \index{port device}
% Sandia National Laboratories is a multimission laboratory managed and
% operated by National Technology & Engineering Solutions of Sandia, LLC, a
% wholly owned subsidiary of Honeywell International Inc., for the U.S.
% Department of Energy’s National Nuclear Security Administration under
% contract DE-NA0003525.

% Copyright 2002-2023 National Technology & Engineering Solutions of Sandia,
% LLC (NTESS).


\begin{Device}\label{P_DEVICE}

%\symbol
%{\includegraphics{ivsSymbol}}

\device
\begin{alltt}
P<name> <(+) node> <(-) node> [[DC] <value> ] port=port number
+ [Z0 = value] [AC [magnitude value [phase value] ] ]
+ [transient specification]
\end{alltt}

\examples
\begin{alltt}
P1 1 0 port = 1
P2 12 0  port=1  z0=100
P1 1 0 port=2   sin 0  1 1e5
P2 2 0 port=2 z0=100 AC 1
\end{alltt}


\parameters

\begin{Parameters}

\param{port}

The port number. Numbered sequentially beginning with 1

\param{Z0} 

System impedance. Currently, it only supports a real-valued impedance.

\param{transient specification}

There are six predefined time-varying functions for sources:

\begin{description}
\item[\tt PULSE <parameters>] Pulse waveform
\item[\tt SIN <parameters>] Sinusoidal waveform
\item[\tt EXP <parameters>] Exponential waveform
\item[\tt PAT <parameters>] Pattern waveform
\item[\tt PWL <parameters>] Piecewise linear waveform
\item[\tt SFFM <parameters>] Frequency-modulated waveform
\end{description}

\end{Parameters}

\comments

The port device identifies the ports used in .LIN analysis. Each
port requires a unique port number. For example, if the netlist has
N port devices, it must contain the sequential set of port
numbers, from 1 to N. Each port has an associated impedance Z0. The
default is 50 ohms. 

The port device behaves as a voltage source in series with an impedance
for all other analyses, such as DC, AC and transient.

None, any, or all of the DC, AC, and transient values
can be specified. The AC phase value is in degrees. The port device
accepts the same transient specifications as the voltage (V) sources.

Positive current flows from the positive node through the port
device to the negative node. 

The power supplied or dissipated by the port device is calculated 
with $I \cdot \Delta V$ where the voltage drop is calculated as $(V_+ - V_-)$ 
and positive current flows from $V_+$ to $V_-$.  Dissipated power has a
positive sign, while supplied power has a negative sign.

\end{Device}



\clearpage
\subsection{Voltage Controlled Voltage Source}
\index{device!voltage controlled voltage source} \index{voltage controlled voltage source}
\index{voltage source!voltage controlled}
% Sandia National Laboratories is a multimission laboratory managed and
% operated by National Technology & Engineering Solutions of Sandia, LLC, a
% wholly owned subsidiary of Honeywell International Inc., for the U.S.
% Department of Energy’s National Nuclear Security Administration under
% contract DE-NA0003525.

% Copyright 2002-2023 National Technology & Engineering Solutions of Sandia,
% LLC (NTESS).


\begin{Device}\label{E_DEVICE}

\symbol
{\includegraphics{vcvsSymbol}}

\device
\begin{alltt}
E<name> <(+) node> <(-) node> <(+) controlling node>
+ <(-) controlling node> <gain>
E<name> <(+) node> <(-) node> VALUE = \{ <expression> \}  
+ [device parameters]
E<name> <(+) node> <(-) node> TABLE \{ <expression> \} = 
+ < <input value>,<output value> >*
E<name> <(+) node> <(-) node> POLY(<value>) 
+ [<+ control node> <- control node>]*
+ [<polynomial coefficient value>]*
\end{alltt}

\examples
\begin{alltt}
EBUFFER 1 2 10 11 5.0
ESQROOT   5   0 VALUE = \{5V*SQRT(V(3,2))\}
ET2 2 0 TABLE \{V(ANODE,CATHODE)\} = (0,0) (30,1)
EP1 5 1 POLY(2) 3 0 4 0 0 .5 .5
\end{alltt}

\parameters

\begin{Parameters}

\param{\vbox{\hbox{(+) node\hfil}\hbox{(-) node}}}

Output nodes. Positive current flows from the \texttt{(+)} node through
the source to the \texttt{(-)} node.

\param{\vbox{\hbox{(+) controlling node\hfil}\hbox{(-) controlling node}}}

Node pairs that define a set of controlling voltages. A given node may
appear multiple times and the output and controlling nodes may be the
same.


\param{device parameters} 

The second form supports two instance parameters \texttt{smoothbsrc} and
\texttt{rcconst}. Parameters may be provided as space separated
\texttt{<parameter>=<value>} specifications as needed. The default value for
\texttt{smoothbsrc} is 0 and the default for \texttt{rcconst} is 1e-9.

\end{Parameters}

\comments 

In the first form, a specified voltage drop between controlling nodes is
multiplied by the gain to determine the voltage drop across the output nodes. 

The second through fourth forms allow nonlinear controlled sources using the
\texttt{VALUE}, \texttt{TABLE}, or \texttt{POLY} keywords, respectively, and
are used in analog behavioral modeling.  They are provided primarily for
netlist compatibility with other simulators.  These three forms are
automatically converted within \Xyce{} to its principal ABM device, the
\texttt{B} nonlinear dependent source device. See the B-source section
(\ref{B_Source_Device}) and the \Xyce{} User's Guide for more guidance on
analog behavioral modeling.  For details concerning the use of the
\texttt{POLY} format, see section~\ref{PspicePoly}.

For HSPICE compatibility, \texttt{VOL} is an allowed synonym for
\texttt{VALUE} for the E-source.

The power supplied or dissipated by this source device is calculated 
with $I \cdot \Delta V$ where the voltage drop is calculated as $(V_+ - V_-)$ 
and positive current flows from $V_+$ to $V_-$.  Dissipated power has a
positive sign, while supplied power has a negative sign.

{\bf NOTE:} The expression given on the left hand side of the equals
sign in E source TABLE expressions may be enclosed in braces, but is
not required to be.  Further, if braces are present there must be
exactly one pair of braces and it must enclose the entire expression.
It is not legal to use additional pairs of braces as parentheses
inside these expressions.  So
\begin{alltt}
ET2 2 0 TABLE \{V(ANODE,CATHODE)+5\} = (0,0) (30,1)
ET3 2 0 TABLE V(ANODE,CATHODE)+5 = (0,0) (30,1)
\end{alltt}
are legal, but 
\begin{alltt}
ET2 2 0 TABLE \{V(ANODE,CATHODE)+\{5\}\} = (0,0) (30,1)
\end{alltt}
is not.  This last will result in a parsing error about missing braces.

E-sources were originally developed primarily to support DC and transient analysis.  
As such, their support for frequency domain analysis (AC and HB) has some limitations.  
The main limitation to be aware of is that time-dependent sources will not work with AC or HB analysis.  
These are sources in which the variable \texttt{TIME} is used in the \texttt{VALUE=} expression. 
However, this time-dependent usage is not common.  The most 
common use case is one in which the E-source is purely dependent (depends only 
on other solution variables), and this use case will work with AC and HB.  

\end{Device}


%%
%% Current-Controlled Current-Source Section
%%
\clearpage
\subsection{Current Controlled Current Source}
\index{device!current controlled current source} \index{current controlled current source}
\index{current source!current controlled}
% Sandia National Laboratories is a multimission laboratory managed and
% operated by National Technology & Engineering Solutions of Sandia, LLC, a
% wholly owned subsidiary of Honeywell International Inc., for the U.S.
% Department of Energy’s National Nuclear Security Administration under
% contract DE-NA0003525.

% Copyright 2002-2024 National Technology & Engineering Solutions of Sandia,
% LLC (NTESS).


\begin{Device}

\symbol
{\includegraphics{cccsSymbol}}

\device
\begin{alltt}
F<name> <(+) node> <(-) node>
+ <controlling V device name> <gain>
F<name> <(+) node> <(-) node> POLY(<value>)
+ <controlling V device name>*
+ < <polynomial coefficient value> >*
\end{alltt}

\examples
\begin{alltt}
FSENSE 1 2 VSENSE 10.0
FAMP 13 0 POLY(1) VIN 0 500
FNONLIN 100 101 POLY(2) VCNTRL1 VCINTRL2 0.0 13.6 0.2 0.005
\end{alltt}

\parameters

\begin{Parameters}

\param{\vbox{\hbox{(+) node\hfil}\hbox{(-) node}}}
Output nodes. Positive current flows from the \texttt{(+)} node through
the source to the \texttt{(-)} node.

\param{controlling V device}
The controlling voltage source which must be an independent voltage source
(V device).

\end{Parameters}

\comments

In the first form, a specified current through a controlling device is
multiplied by the gain to determine this device's output current.  The
gain may be expressed either as a number, a parameter, or an arbitrary
brace-delimited ABM expression.

The second form using the \texttt{POLY} keyword is used in analog behavioral
modeling.

Both forms are automatically converted within \Xyce{} to its principal
ABM device, the \texttt{B} nonlinear dependent source device. See the B-source
section (\ref{B_Source_Device}) and the \Xyce{} User's Guide for more guidance
on analog behavioral modeling.  For details concerning the use of the
\texttt{POLY} format, see section~\ref{PspicePoly}.

The power supplied or dissipated by this source device is calculated 
with $I \cdot \Delta V$ where the voltage drop is calculated as $(V_+ - V_-)$ 
and positive current flows from $V_+$ to $V_-$.  Dissipated power has a
positive sign, while supplied power has a negative sign.

F-sources were originally developed primarily to support DC and transient analysis.  
As such, their support for frequency domain analysis (AC and HB) has some limitations.  
The main limitation to be aware of is that time-dependent sources will not work with AC or HB analysis.  
These are sources in which the variable \texttt{TIME} is used in the \texttt{VALUE=} expression. 
However, this time-dependent usage is not common.  The most 
common use case is one in which the F-source is purely dependent (depends only 
on other solution variables), and this use case will work with AC and HB.  

\end{Device}


%%
%% Voltage-Controlled Current-Source Section
%%
\clearpage
\subsection{Voltage Controlled Current Source}
\index{device!voltage controlled current source} \index{voltage controlled current source}
\index{current source!voltage controlled}
% Sandia National Laboratories is a multimission laboratory managed and
% operated by National Technology & Engineering Solutions of Sandia, LLC, a
% wholly owned subsidiary of Honeywell International Inc., for the U.S.
% Department of Energy’s National Nuclear Security Administration under
% contract DE-NA0003525.

% Copyright 2002-2024 National Technology & Engineering Solutions of Sandia,
% LLC (NTESS).



\begin{Device}\label{G_DEVICE}

\symbol
{\includegraphics{vccsSymbol}}

\device
\begin{alltt}
G<name> <(+) node> <(-) node> <(+) controlling node>
  + <(-) controlling node> <transconductance> [M=<value>]
G<name> <(+) <node> <(-) node> VALUE = \{ <expression> \}
G<name> <(+) <node> <(-) node> TABLE \{ <expression> \} =
+ < <input value>,<output value> >*
G<name> <(+) <node> <(-) node> POLY(<value>)
+ [<+ controlling node> <- controlling node>]*
+ [<polynomial coefficient>]*
\end{alltt}

\examples
\begin{alltt}
GBUFFER 1 2 10 11 5.0
GPSK 11 6 VALUE = \{5MA*SIN(6.28*10kHz*TIME+V(3))\}
GA2 2 0 TABLE \{V(5)\} = (0,0) (1,5) (10,5) (11,0)
GMULT 1 2 10 11 3.0 M=5
\end{alltt}

\parameters

\begin{Parameters}

\param{\vbox{\hbox{(+) node\hfil}\hbox{(-) node}}}

Output nodes. Positive current flows from the \texttt{(+)} node through
the source to the \texttt{(-)} node.

\param{\vbox{\hbox{(+) controlling node\hfil}\hbox{(-) controlling node}}}

Node pairs that define a set of controlling voltages. A given node may
appear multiple times and the output and controlling nodes may be the
same.

\end{Parameters}

\comments

In the first form, the voltage drop between the controlling nodes is
multiplied by the transconductance to obtain the current-source
output of the \texttt{G} device. 

The second through fourth forms using the \texttt{VALUE}, \texttt{TABLE}, and
\texttt{POLY} keywords, respectively, are used in analog behavioral modeling.
They are provided primarily for netlist compatibility with other simulators.
These two forms are automatically converted within \Xyce{} to its principal ABM
device, the B nonlinear dependent source device. See the B-source section
(\ref{B_Source_Device}) and the \Xyce{} User's Guide for more guidance on
analog behavioral modeling.  For details concerning the use of the
\texttt{POLY} format, see section~\ref{PspicePoly}.

For HSPICE compatibility, \texttt{CUR} is an allowed synonym for
\texttt{VALUE} for the G-source.  
  Also, this device supports the \texttt{M} multiplier parameter.

The power supplied or dissipated by this source device is calculated 
with $I \cdot \Delta V$ where the voltage drop is calculated as $(V_+ - V_-)$ 
and positive current flows from $V_+$ to $V_-$.  Dissipated power has a
positive sign, while supplied power has a negative sign.

G-sources were originally developed primarily to support DC and transient analysis.  
As such, their support for frequency domain analysis (AC and HB) has some limitations.  
The main limitation to be aware of is that time-dependent sources will not work with AC or HB analysis.  
These are sources in which the variable \texttt{TIME} is used in the \texttt{VALUE=} expression. 
However, this time-dependent usage is not common.  The most 
common use case is one in which the G-source is purely dependent (depends only 
on other solution variables), and this use case will work with AC and HB.  
\end{Device}



%%
%% Current-Controlled Voltage-Source Section
%%
\clearpage
\subsection{Current Controlled Voltage Source}
\index{device!current controlled voltage source} \index{current controlled voltage source}
\index{voltage source!current controlled}
% Sandia National Laboratories is a multimission laboratory managed and
% operated by National Technology & Engineering Solutions of Sandia, LLC, a
% wholly owned subsidiary of Honeywell International Inc., for the U.S.
% Department of Energy’s National Nuclear Security Administration under
% contract DE-NA0003525.

% Copyright 2002-2023 National Technology & Engineering Solutions of Sandia,
% LLC (NTESS).


The syntax of this device is exactly the same as for a Current-Controlled
Current Source.  For a Current-Controlled Voltage Source just substitute an H
for the F. The H device generates a voltage, whereas the F device generates a
current.

\begin{Device}

\symbol
{\includegraphics{ccvsSymbol}}

\device
\begin{alltt}
H<name> <(+) node> <(-) node>
+ <controlling V device name> <transresistance>
H<name> <(+) node> <(-) node> POLY(<value>)
+ <controlling V device name>*
+ < <polynomial coefficient value> >*
\end{alltt}

\examples
\begin{alltt}
HSENSE 1 2 VSENSE 10.0
HAMP 13 0 POLY(1) VIN 0 500
HNONLIN 100 101 POLY(2) VCNTRL1 VCINTRL2 0.0 13.6 0.2 0.005
\end{alltt}

\comments

In the first form, the current through a specified controlling voltage
source is multiplied by the transresistance to obtain the
voltage-source output.  The transresistance may be expressed either as
a number, a parameter, or an arbitrary brace-delimited ABM expression.

The second form using the \texttt{POLY} keyword is used in analog
behavioral modeling.  It is provided primarily for netlist
compatibility with other simulators.

H sources in any form are automatically converted within \Xyce{} to
its principal ABM device, the B nonlinear dependent source device. See
the B-source section (\ref{B_Source_Device}) and the \Xyce{} User's
Guide for more guidance on analog behavioral modeling.  For details
concerning the use of the \texttt{POLY} format, see
section~\ref{PspicePoly}.

The power supplied or dissipated by this source device is calculated
with $I \cdot \Delta V$ where the voltage drop is calculated as $(V_+ - V_-)$
and positive current flows from $V_+$ to $V_-$.  Dissipated power has a
positive sign, while supplied power has a negative sign.

H-sources were originally developed primarily to support DC and transient analysis.  
As such, their support for frequency domain analysis (AC and HB) has some limitations.  
The main limitation to be aware of is that time-dependent sources will not work with AC or HB analysis.  
These are sources in which the variable \texttt{TIME} is used in the \texttt{VALUE=} expression. 
However, this time-dependent usage is not common.  The most 
common use case is one in which the H-source is purely dependent (depends only 
on other solution variables), and this use case will work with AC and HB.  

\end{Device}


%%
%% BSource Section
%%
\clearpage
\subsection{Nonlinear Dependent Source}
\label{B_Source_Device}
\index{device!bsource} \index{bsource}
\index{voltage source!nonlinear dependent} \index{current source!nonlinear dependent}
% Sandia National Laboratories is a multimission laboratory managed and
% operated by National Technology & Engineering Solutions of Sandia, LLC, a
% wholly owned subsidiary of Honeywell International Inc., for the U.S.
% Department of Energy’s National Nuclear Security Administration under
% contract DE-NA0003525.

% Copyright 2002-2024 National Technology & Engineering Solutions of Sandia,
% LLC (NTESS).


\begin{Device}\label{B_DEVICE}

\device
\begin{alltt}
B<name> <(+) node> <(-) node> V={ABM expression} [device parameters]
B<name> <(+) node> <(-) node> I={ABM expression}
\end{alltt}

\examples
\begin{alltt}
B1 2 0 V=\{sqrt(V(1))\}
B2 4 0 V=\{V(1)*TIME\}
B3 4 2 I=\{I(V1) + V(4,2)/100\}
B4 5 0 V=\{Table \{V(5)\}=(0,0) (1.0,2.0) (2.0,3.0) (3.0,10.0)\}
B5 6 0 V=tablefile("file.dat")
B6 7 0 I=tablefile("file.dat")
B5 6 0 V=table("file.dat")
B6 7 0 I=table("file.dat")
B5 6 0 V=\{table("file.dat")\}
B5 6 0 V=\{spline("file.dat")\}
B5 6 0 V=\{BLI("file.dat")\}
B5 6 0 V=\{fasttable("file.dat")\}
\end{alltt}

\comments

The nonlinear dependent source device, also known as the B-source
device, is used in analog behavioral modeling (ABM).  The \texttt{(+)}
and \texttt{(-)} nodes are the output nodes. Positive current flows from
the \texttt{(+)} node through the source to the \texttt{(-)}
node. 

The power supplied or dissipated by the nonlinear dependent source is calculated 
with $I \cdot \Delta V$ where the voltage drop is calculated as $(V_+ - V_-)$ 
and positive current flows from $V_+$ to $V_-$.  Dissipated power has a
positive sign, while supplied power has a negative sign.

The syntax involving the \texttt{tablefile} keyword internally attempts to load the
data in \texttt{"file.dat"} into a \texttt{TABLE} expression.  The data file must
be in plain-text and contain just two pairs of data per line.  For an example see 
the ``Analog Behavioral Modeling'' chapter of the \Xyce{} User's
Guide.  
Either \texttt{table} or \texttt{tablefile} can be used to read a table in 
from a file.  They are synonyms.

Other related table-based features include \texttt{fasttable}, which is the same as \texttt{table} 
but without many breakpoints, and \texttt{bli} for Barycentric Lagrange 
Interpolation~\cite{Berrut_barycentriclagrange}.  Various splines are also supported,
including \texttt{spline}, \texttt{cubic}, \texttt{akima}~\cite{10.1145/321607.321609} 
and \texttt{wodicka}~\cite{Engeln1996}.  \texttt{spline} and \texttt{akima} are synonymous.    
All of these methods use the same syntax as \texttt{table}, and all of them support 
reading tables in from files.

It is important to note that the B-source allows the user to specify
expressions that could have infinite-slope transitions, such as the
following.  (Note: the braces surrounding all expressions are required in this definition.)
\begin{alltt} Bcrtl OUTA 0 V=\{ IF( (V(IN) > 3.5), 5, 0 ) \} \end{alltt}
This can lead to ``timestep too small'' errors when \Xyce{} reaches the
transition point.  Infinite-slope transitions in expressions dependent only on
the \texttt{time} variable are a special case, because \Xyce{} can detect that
they are going to happen in the future and set a ``breakpoint'' to capture
them.  Infinite-slope transitions depending on other solution variables cannot
be predicted in advance, and cause the time integrator to scale back the
timestep repeatedly in an attempt to capture the feature until the timestep is
too small to continue.

One solution to the problem is to modify the expression to allow a continuous transition. 
However, this can become complicated with multiple inputs. The other solution is to specify
device options or instance parameters to allow smooth transitions. The parameter
\texttt{smoothbsrc} enables the smooth transitions. This is done by adding a RC network to the  
output of B sources. For example,

\begin{alltt} Bcrtl OUTA 0 V=\{ IF( (V(IN) > 3.5), 5, 0 ) \} smoothbsrc=1 \end{alltt}

\begin{alltt} .options device  smoothbsrc=1 \end{alltt}

The smoothness of the transition can be controlled by specifying the rc constant of 
the RC network. For example, 

\begin{alltt} Bcrtl OUTA 0 V=\{ IF( (V(IN) > 3.5), 5, 0 ) \} smoothbsrc=1   
 + rcconst = 1e-10 \end{alltt}

Note that this smoothed B-source only applies to voltage sources. The voltage behavioral source supports
two instance parameters \texttt{smoothbsrc} and \texttt{rcconst}. Parameters may be provided as space  
separated \texttt{<parameter>=<value>} specifications as needed. The default value for \texttt{smoothbsrc}
is 0 and the default for \texttt{rcconst} is 1e-9.

See the ``Analog Behavioral Modeling'' chapter of the \Xyce{} User's
Guide~\UsersGuide{} for guidance on using the B-source device and ABM expressions,
and the Expressions Section (\ref{ExpressionDocumentation}) for
complete documentation of expressions and expression operators.
One important note is that time-dependent expressions are supported
for the current and voltage parameters of a B source, but
frequency-dependent expressions are not.

B-sources were originally developed primarily to support DC and transient analysis.  
As such, their support for frequency domain analysis (AC and HB) has some limitations.  
The main limitation to be aware of is that time-dependent sources will not work with AC or HB analysis.  
These are sources in which the variable \texttt{TIME} is used in the \texttt{VALUE=} expression. 
The use case of a purely depedent B-source (depends only on other solution variables) will work with AC and HB.  

\end{Device}


%%
%% BJT Section
%%
\clearpage
\subsection{Bipolar Junction Transistor (BJT)}
\index{device!BJT} \index{BJT}
% Sandia National Laboratories is a multimission laboratory managed and
% operated by National Technology & Engineering Solutions of Sandia, LLC, a
% wholly owned subsidiary of Honeywell International Inc., for the U.S.
% Department of Energy’s National Nuclear Security Administration under
% contract DE-NA0003525.

% Copyright 2002-2023 National Technology & Engineering Solutions of Sandia,
% LLC (NTESS).


%%
%% BJT Description Table
%%

\begin{Device}\label{Q_DEVICE}

\symbol
{\includegraphics{npnSymbol}}
{\includegraphics{pnpSymbol}}

\device
\begin{alltt}
Q<name> <collector node> <base node> <emitter node>
 + [substrate node] <model name> [area value]

Q<name> <collector node> <base node> <emitter node>
 + [thermal node] <VBIC 1.3 3-terminal model name>

Q<name> <collector node> <base node> <emitter node>
 + <substrate> [thermal node] <VBIC 1.3 4-terminal model name>

 Q<name> <collector node> <base node> <emitter node>
 + <substrate> <thermal node> <HICUM model name>
\end{alltt}

\model
\begin{alltt}
.MODEL <model name> NPN [model parameters]
.MODEL <model name> PNP [model parameters]
\end{alltt}

\examples
\begin{alltt}
Q2 10 2 9 PNP1
Q12 14 2 0 1 NPN2 2.0
Q6 VC 4 11 [SUB] LAXPNP
Q7 Coll Base Emit DT VBIC13MODEL2
Q8 Coll Base Emit VBIC13MODEL3 SW\_ET=0
Q9 Coll Base Emit Subst DT VBIC13MODEL4
Q10 Coll Base Emit Subst DT HICUMMMODEL1
\end{alltt}

\parameters
\begin{Parameters}
\param{substrate node}
  Optional and defaults to ground. Since \Xyce{} permits alphanumeric
  node names and because there is no easy way to make a distinction between
  these and the model names, the name (not a number) used for the substrate
  node must be enclosed in square brackets \texttt{[ ]}.  Otherwise, nodes
  would be interpreted as model names. See the fourth example above.

\param{area value}
  The relative device area with a default value of 1.

\end{Parameters}

\comments
The BJT is modeled as an intrinsic transistor using ohmic resistances in series
with the collector (RC/area), with the base (value varies with current, see BJT
equations) and with the emitter (RE/area).For model parameters with optional
names, such as VAF and VA (the optional name is in parentheses), either may be
used.For model types NPN and PNP, the isolation junction capacitance is
connected between the intrinsic-collector and substrate nodes. This is the same
as in SPICE and works well for vertical IC transistor structures.

\textbf{Only the VBIC 1.3 model is available in \Xyce{} 6.11 and
  later.}  The VBIC 1.3 model is provided in both 3-terminal (Q level
  11) and 4-terminal (Q level 12) variants, both supporting
  electrothermal and excess-phase effects.  These variants of the Q line
  are shown in the fourth through sixth examples above. VBIC 1.3
  instance lines have three or four required nodes, depending on model
  level, and an \emph{optional} ``dt'' node.  The first three are the
  normal collector, base,and emitter. In the level 12 (4-terminal) the
  fourth node is the substrate, just as for the level 1 BJT.  If the
  optional ``dt'' node is specified for either variant, it can be used
  to print the local temperature rise due to self-heating, and could
  possibly be used to model coupled heating effects of several VBIC
  devices.  It is, however, unnecessary to specify a ``dt'' node just
  to print the local temperature rise, because when this node is
  omitted from the instance line it simply becomes and internal node,
  and may still be printed using the syntax
  \texttt{N(instancename:dt)}.  For the ``Q8'' example above, one
  could print \texttt{N(Q8:dt)}.

As of release 6.10 of Xyce, the VBIC 1.3 3-terminal device (Q level
11) has been the subject of extensive optimization, and runs much
faster than in previous releases.

\textbf{ The HICUM models require both a substrate and thermal node.}

\end{Device}


% BJT model schematic.
\begin{figure}[ht]
  \centering
  \scalebox{0.6}
  {\includegraphics{bjtSchematic}}
  \caption[BJT model schematic]{BJT model schematic.  Adapted from
reference~\cite{PSpiceUG:1998}. \label{figBJTschematic}}
\end{figure}

\paragraph{BJT Level selection}

\Xyce{} supports the level 1 BJT model, which is based on the
documented standard SPICE 3F5 BJT model, but was coded independently
at Sandia.  It is mostly based on the classic Gummel-Poon BJT
model~\cite{GummelPoon}.

Two variants of the VBIC model are provided as BJT levels 11  and 12.
Levels 11 and 12 are the
3-terminal and 4-terminal variants of the VBIC 1.3.

An experimental release of the FBH HBT\_X model version
2.1\cite{Rudolph_documentationof} is provided as BJT level 23.

Both the HICUM/L0 (level 230) and HICUM/L2 (level 234) models are also
provided (\url{https://www.iee.et.tu-dresden.de/iee/eb/hic_new/hic_start.html}).

The MEXTRAM\cite{MEXTRAM_home} BJT model version 504.12.1 model is provided.
Two variants of this model are available: the level 504 model without
self-heating and without external substrate node, and the level 505 model with
self heating but without external substrate node.  The level 505 instance line
requires a fourth node for the 'dt' node, similar to the usage in all of the
VBIC models (levels 11-12), but is otherwise identical to the level 504 model.

\paragraph{BJT Power Calculations}
Power dissipated in the transistor is calculated with
$|I_{B}*V_{BE}|+|I_{C}*V_{CE}|$, where $I_{B}$ is the base current, $I_{C}$ is
the collector current, $V_{BE}$ is the voltage drop between the base and the
emitter and $V_{CE}$ is the voltage drop between the collector and the emitter.
This formula may differ from other simulators.

\subsubsection{The Level 1 Model}
\paragraph{BJT Equations}\label{bjt_equations}
The Level 1 BJT implementation within \Xyce{} is based on \cite{Fjeldly:1998}.
The equations in this section describe an NPN transistor. For the PNP device,
reverse the signs of all voltages and currents.  The equations use the
following variables:
%insert variables here
\begin{eqnarray*}
V_{be} & = & \mbox{intrinsic base-intrinsic emitter voltage} \\
V_{bc} & = & \mbox{intrinsic base-intrinsic collector voltage} \\
V_{bs} & = & \mbox{intrinsic base-substrate voltage} \\
V_{bw} & = & \mbox{intrinsic base-extrinsic collector voltage
(quasi-saturation only)} \\
V_{bx} & = & \mbox{extrinsic base-intrinsic collector voltage} \\
V_{ce} & = & \mbox{intrinsic collector-intrinsic emitter voltage} \\
V_{js} & = & \mbox{(NPN) intrinsic collector-substrate voltage} \\
&        = & \mbox{(PNP) intrinsic substrate-collector voltage} \\
% THIS DOESN'T EXIST IN XYCE
%&        = & \mbox{(LPNP) intrinsic base-substrate voltage} \\
V_{t}  & = & \mbox{$kT/q$ (thermal voltage)} \\
V_{th} & = & \mbox{threshold voltage} \\
k      & = & \mbox{Boltzmann's constant} \\
q      & = & \mbox{electron charge} \\
T      & = & \mbox{analysis temperature (K)} \\
T_{0}  & = & \mbox{nominal temperature (set using \textrmb{TNOM} option)}
\end{eqnarray*}
Other variables are listed above in BJT Model Parameters.

\subparagraph{DC Current}
The BJT model is based on the Gummel and Poon model~\cite{Grove:1967} where the
different terminal currents are written
%insert equations here
\begin{eqnarray*}
I_{e} & = & -I_{cc} - I_{be} + I_{re} + (C_{dife} +C_{de})\frac{dV_{be}}{dt} \\
I_{c} & = & -I_{cc} + I_{bc} - I_{rc} - (C_{difc}+ C_{dc})\frac{dV_{bc}}{dt} \\
I_b & = & I_e -I_c
\end{eqnarray*}
Here, $C_{dife}$ and $C_{difc}$ are the capacitances related to the hole
charges per unit area in the base, $Q_{dife}$ and $Q_{difc}$,
affiliated with the electrons introduced across the emitter-base and
collector-base junctions, respectively.  Also, $C_{be}$ and $C_{bc}$ are the
capacitances related to donations to the hole charge of the base, $Q_{be}$ and
$Q_{bc}$,  affiliated with the differences in the depletion regions of the
emitter-base and collector-base junctions, respectively.  The
intermediate currents used are defined as
%insert equations here
\begin{eqnarray*}
-I_{be} & = & \mathbf{\frac{IS}{BF}} \left[\exp \left(\frac{V_{be}}
{\mathbf{NF}V_{th}} \right) -1 \right] \\
-I_{cc} & = & \frac{Q_{bo}}{Q_{b}}\mathbf{IS} \left[\exp \left(\frac{V_{be}}
{\mathbf{NF} V_{th}} \right) -
\exp \left(\frac{V_{bc}}{\mathbf{NF}V_{th}} \right) \right] \\
-I_{bc} & = & \mathbf{\frac{IS}{BR}} \left[\exp \left(\frac{V_{bc}}
{\mathbf{NR}V_{th}} \right) - 1 \right] \\
I_{re} & = & \mathbf{ISE} \left[\exp \left(\frac{V_{be}}
{\mathbf{NE} V_{th}} \right) - 1 \right] \\
I_{rc} & = & \mathbf{ISC} \left[\exp \left(\frac{V_{bc}}
{\mathbf{NC}V_{th}} \right) - 1 \right]
\end{eqnarray*}
where the last two terms are the generation/recombination currents related to
the emitter and collector junctions, respectively.  The charge $Q_{b}$ is the
majority carrier charge in the base at large injection levels and is a key
difference in the Gummel-Poon model over the earlier Ebers-Moll model.  The
ratio $Q_b/Q_{bo}$ (where $Q_{bo}$ represents the zero-bias base charge, i.e.
the value of $Q_b$ when $V_{be}=V_{bc}=0$) as computed by \Xyce{} is given by
\[\frac{Q_b}{Q_{bo}} = \frac{q_1}{2}\left(1+\sqrt{1+4q_2}\right)\]
where
\begin{eqnarray*}
q_1 & = & \left(1-\frac{V_{be}}{\mathbf{VAR}}-\frac{V_{bc}}{\mathbf{VAF}}
\right)^{-1} \\
q_2 & = & \frac{\mathbf{IS}}{\mathbf{IKF}}\left[\exp\left(\frac{V_{be}}
{\mathbf{NF}V_{th}}\right)-1\right] + \frac{\mathbf{IS}}{\mathbf{IKR}}
\left[\exp\left(\frac{V_{bc}}{\mathbf{NR}V_{th}}\right)-1\right]
\end{eqnarray*}

\subparagraph{Capacitance Terms}
The capacitances listed in the above DC $I-V$ equations each consist of a
depletion layer capacitance $C_{d}$ and
a diffusion capacitance $C_{dif}$.  The first is given by
\[
C_d = \left\{
\begin{array}{ll}
\mathbf{CJ} \left(1 - \frac{V_{di}}{\mathbf{VJ}} \right)^{\mathbf{-M}} &
V_{di} \leq \mathbf{FC \cdot VJ} \\
\mathbf{CJ} \left(1 - \mathbf{FC} \right)^{-(1+\mathbf{M})} \
\left[1 - \mathbf{FC}(1 + \mathbf{M}) + \mathbf{M}
\frac{V_{di}}{\mathbf{VJ}} \right]
& V_{di} > \mathbf{FC \cdot VJ}
\end{array}
\right. \]

where $\mathbf{CJ}=\mathbf{CJE}$ for $C_{de}$, and where $\mathbf{CJ}=
\mathbf{CJC}$ for $C_{dc}$.
The diffusion capacitance (sometimes referred to as the transit time
capacitance) is
\[
C_{dif} = \mathbf{TT} G_d = \mathbf{TT} \frac{dI}{dV_{di}}
\]
where $I$ is the diode DC current given, $G_d$ is the corresponding junction
conductance, and where $\mathbf{TT}=\mathbf{TF}$ for $C_{dife}$ and
$\mathbf{TT}=\mathbf{TR}$ for $C_{difc}$.

\subparagraph{Temperature Effects}
SPICE temperature effects are default, but all levels of the BJT have a more
advanced temperature compensation available.  By specifying
\texttt{TEMPMODEL=QUADRATIC} in the netlist, parameters can be interpolated
quadratically between measured values extracted from data.  In the BJT, IS and
ISE are interpolated logarithmically because they can change over an order of
magnitude or more for temperature ranges of interest.  See the
Section~\ref{Model_Interpolation} for more details on how to include quadratic
temperature effects.

% \subparagraph{Noise}
% Noise is calculated with a 1.0 Hz bandwidth using the following spectral power
% densities (per unit bandwidth).

% Thermal Noise due to Parasitic Resistance: $I_{n}^{2} =
% \frac{4kT}{\mathbf{RS}/area}$

% Intrinsic Diode Shot and Flicker Noise: $I_{n}^{2} = 2qI_{D} +
% \mathbf{KF}\cdot\frac{I_{D}^{\mathbf{AF}}}{\omega}$

For further information on BJT models, see~\cite{Grove:1967}.  For a thorough
description of the U.C. Berkeley SPICE models see
Reference~\cite{Antognetti:1988}.

\subsubsection{VBIC Temperature Considerations}
\index{VBIC (temperature considerations)}
The VBIC (Q levels 11 and 12) model both support a self-heating
model.  The model works by computing the power dissipated by all
branches of the device, applying this power as a flow through a small
thermal network consisting of a power flow (``current'') source
through a thermal resistance and thermal capacitance, as shown in
Figure~\ref{vbicthermal}.  The circuit node DT will therefore be the
``thermal potential'' (temperature) across the parallel thermal
resistance and capacitance.  This temperature is the temperature rise
due to self heating of the device, which is added to the ambient
temperature and \texttt{TRISE} parameter to obtain the device
operating temperature.
\begin{figure}
  \centering
  \scalebox{0.3}
  {\includegraphics{VBIC_Thermal_Net}}
  \caption[VBIC thermal network schematic]{VBIC thermal network  schematic.}
  \label{vbicthermal}
\end{figure}

In VBIC 1.3, the dt node is optional on the netlist line.  If not
given, the dt node is used internally for thermal effects
calculations, but not accessible from the rest of the netlist.  The
VBIC 1.3 provides an instance parameter \texttt{SW\_ET} that may be
set to zero to turn off electrothermal self-heating effects.  When set
to zero, no thermal power is sourced into the dt node.  This parameter
defaults to 1, meaning that thermal power is computed and flows into
dt even when dt is unspecified on the netlist and remains an internal
node.

In VBIC 1.3, setting RTH to zero does {\em NOT\/} disable the
self-heating model, and does not short the dt node to ground, even
though one might expect that to be the behavior.  Rather, it simply
removes the RTH resistor from the equivalent circuit of
figure~\ref{vbicthermal} and leaves the dt node floating.  This is an
important point to recognize when using the VBIC.

If a node name is given as the fourth node of a VBIC \Xyce{} will emit
warnings about the node not having a DC path to ground and being
connected to only one device.  These warnings may safely be ignored,
and are a harmless artifact of Xyce's connectivity checker.  It is
possible to silence this warning by adding a very large resistance
between the dt node and ground --- 1GOhm or 1TOhm are effectively the
same as leaving the node floating, and will satisfy the connectivity
checker's tests.  This used to be the recommended means of silencing
the connectivity checker for the VBIC 1.2 where dt was a required
node, but it is safe {\em if and only if a nonzero\/} \texttt{RTH}
{\em value is specified for the device.}  If, however, RTH is zero,
then dt would otherwise be floating and your external resistance now
becomes the primary path for thermal power flow; rather than turning
off self-heating effects, it will be as if you had set RTH to a very
large value.  We therefore recommend that you not tie the dt node to
ground via a resistor, and if you are not using it to connect VBIC
devices together via a thermal network, simply leave off the dt node
to silence the connectivity checker warning.  Turn off self-heating
effects ONLY by setting the \texttt{SW\_ET} instance parameter to zero.

Users of earlier versions of \Xyce{} may have been using the VBIC 1.2
model that was removed in release 6.11.  All netlists containing the
old level=10 VBIC 1.2 model must be modified to run in \Xyce{} 6.11
and later.  The following points should be observed when converting an
old VBIC 1.2 netlist and model card to VBIC 1.3.

\begin{itemize}
  \item Generally speaking, most VBIC 1.2 model cards can be converted
    to VBIC 1.3 model cards by the simple substitution of
    \texttt{level=11} for \texttt{level=10}, with the following provisos.

  \item VBIC 1.2 in \Xyce{} 6.10 and earlier did not support excess
    phase effects, and so the \texttt{TD} parameter governing excess
    phase was ignored.

    The \Xyce{} team has observed that some users' VBIC 1.2 parameter
    extractions have a non-zero value for the \texttt{TD} parameter.
    The impact of this is twofold:
    \begin{itemize}
      \item Circuits that use such model cards with only the level
        number changed will likely not produce identical results when
        compared to simulation results of older versions of \Xyce{}
        using VBIC 1.2 due to the excess phase effects.  If strict
        comparison between VBIC 1.3 runs with \Xyce{} 6.11 or later
        against older runs with VBIC 1.2 is desired, change the
        \texttt{TD} parameter to zero.  This will disable the excess
        phase effects and make VBIC 1.3 equivalent to the VBIC 1.2
        that was previously provided.
      \item The \Xyce{} team has seen some instances where the
        previously ignored \texttt{TD} parameter value is such that
        \Xyce{} will fail to converge when the equivalent VBIC 1.3
        model is substituted.  The VBIC 1.2 behavior can be recovered
        by setting the model parameter \texttt{TD} to zero, which will
        disable the excess phase effect in VBIC 1.3.  We can only
        suggest that the model card be re-extracted using VBIC 1.3 to
        determine the correct value for \texttt{TD}.
    \end{itemize}

  \item VBIC 1.2 had a model parameter called \texttt{DTEMP}, which
    \Xyce{} also recognized on the instance line.  In VBIC 1.3 this
    parameter has been replaced by another called \texttt{TRISE},
    which is only an instance parameter, and is unrecognized in model
    cards.  VBIC 1.3 also recognizes \texttt{DTEMP} on the instance
    line as an alias for \texttt{TRISE}.  If you had been specifying
    \texttt{DTEMP} in your VBIC 1.2 model cards, you will need to move it
    to the instance line instead in order for the parameter to be
    properly recognized by both VBIC 1.2 and VBIC 1.3.
  \item Turning off self-heating effects in VBIC 1.2 was done by
    grounding the mandatory dt node.  This is not the recommended way
    of disabling self-heating in VBIC 1.3.  To disable self-heating,
    set the \texttt{SW\_ET} parameter to zero on the instance line (as
    is done in the ``Q8'' example above).
  \item If not using the dt node as a way of thermally coupling
    devices to each other, leave it off of VBIC 1.3 instance lines,
    allowing it to be an internal variable irrespective of whether
    self-heating is enabled or not.  This will silence any connectivity
    warnings from Xyce.  Since the dt node may be printed using the
    N() syntax even when internal, it is unnecessary to put a dt node
    on the instance line just to print the local temperature rise due
    to self-heating.  The only reasons to include it on the instance
    line would be for backward compatibility to VBIC 1.2 netlists, or
    to implement a thermal coupling network between devices.
  \item Finally, VBIC 1.3 introduced a number of constraints on model
    parameters that the previous version did not.  \Xyce{} will emit
    warnings if any parameter on a VBIC 1.3 model card is out of the
    range specified by the VBIC 1.3 authors.  These warnings should
    not be ignored lightly, as they indicate that the model is being
    used in a manner not intended by its authors.  They are generally
    a sign that the model may not be well-behaved, and may indicate an
    improperly extracted model card.

\end{itemize}

\subsubsection{Level 1 BJT Tables}
% This table was generated by Xyce:
%   Xyce -doc Q 1
%
\index{bipolar junction transistor!device instance parameters}
\begin{DeviceParamTableGenerated}{Bipolar Junction Transistor Device Instance Parameters}{Q_1_Device_Instance_Params}
AREA & Relative device area & -- & 1 \\ \hline
DTEMP & Device delta temperature & $^\circ$C & 0 \\ \hline
IC1 & Vector of initial values: Vbe,Vce. Vbe=IC1 & V & 0 \\ \hline
IC2 & Vector of initial values: Vbe,Vce. Vce=IC2 & V & 0 \\ \hline
M & multiplicity factor & -- & 1 \\ \hline
OFF & Initial condition of no voltage drops accross device & logical (T/F) & false \\ \hline
TEMP & Device temperature & $^\circ$C & Ambient Temperature \\ \hline
\end{DeviceParamTableGenerated}

\input{Q_1_Device_Model_Params}

\subsubsection{Level 11 and 12 BJT Tables (VBIC 1.3)}
The VBIC 1.3 (level 11 transistor for 3-terminal, level 12 for
4-terminal) supports a number of instance parameters that are not
available in the VBIC 1.2.  The level 11 and level 12 differ only by
the number of required nodes.  The level 11 is the 3-terminal device,
having only collector, base, and emitter as required nodes.  The level
12 is the 4-terminal device, requiring collector, base, emitter and
substrate nodes.  Both models support an optional 'dt' node as their
last node on the instance line.

\textbf{Model cards extracted for the VBIC 1.2 will mostly work with the VBIC
1.3,  with one notable exception:} in VBIC 1.2 the \texttt{DTEMP} parameter was
a model parameter, and \Xyce{} allowed it also to be specified on the instance
line, overriding whatever was specified in the model.  This parameter was
replaced in VBIC 1.3 with the \texttt{TRISE} parameter, which is {\em only\/}
an instance parameter.  \texttt{DTEMP} and \texttt{DTA} are both supported as
aliases for the \texttt{TRISE} instance parameter.

\input{Q_11_Device_Instance_Params}
\input{Q_11_Device_Model_Params}
\clearpage
\input{Q_12_Device_Instance_Params}
\input{Q_12_Device_Model_Params}
\clearpage

\subsubsection{Level 23 BJT Tables (FBH HBT\_X)}
\input{Q_23_Device_Instance_Params}
\input{Q_23_Device_Model_Params}
\clearpage

\subsubsection{Level 230 BJT Tables (HICUM/L0)}
The HICUM/L0 device supports output of the internal variables in
table~\ref{Q_230_OutputVars} on the \texttt{.PRINT} line of a netlist.
To access them from a print line, use the syntax
\texttt{N(<instance>:<variable>)} where ``\texttt{<instance>}'' refers to the
name of the specific HICUM/L0 Q device in your netlist.

\input{Q_230_Device_Instance_Params}
\input{Q_230_Device_Model_Params}
\input{Q_230_OutputVars}
\clearpage

\subsubsection{Level 234 BJT Table (HICUM/L2)}
\textbf{NOTE:} The HICUM/L2 model has no instance parameters.
The HICUM/L2 device supports output of the internal variables in
table~\ref{Q_234_OutputVars} on the \texttt{.PRINT} line of a netlist.
To access them from a print line, use the syntax
\texttt{N(<instance>:<variable>)} where ``\texttt{<instance>}'' refers to the
name of the specific HICUM/L2 Q device in your netlist.
\input{Q_234_Device_Model_Params}
\input{Q_234_OutputVars}
\clearpage

\subsubsection{Level 504 and 505 BJT Tables (MEXTRAM)}


The MEXTRAM device supports output of the internal variables in
tables~\ref{Q_504_OutputVars} and~\ref{Q_504_OutputVars} on the \texttt{.PRINT} line of a netlist.
To access them from a print line, use the syntax
\texttt{N(<instance>:<variable>)} where ``\texttt{<instance>}'' refers to the
name of the specific MEXTRAM Q device in your netlist.

\input{Q_504_Device_Instance_Params}
\input{Q_504_Device_Model_Params}
\input{Q_504_OutputVars}
\clearpage

\input{Q_505_Device_Instance_Params}
\input{Q_505_Device_Model_Params}
\input{Q_505_OutputVars}
\clearpage



%%
%% JFET Section
%%
\clearpage
\subsection{Junction Field-Effect Transistor (JFET)}
\index{device!JFET} \index{JFET}
% Sandia National Laboratories is a multimission laboratory managed and
% operated by National Technology & Engineering Solutions of Sandia, LLC, a
% wholly owned subsidiary of Honeywell International Inc., for the U.S.
% Department of Energy’s National Nuclear Security Administration under
% contract DE-NA0003525.

% Copyright 2002-2024 National Technology & Engineering Solutions of Sandia,
% LLC (NTESS).


\begin{Device}\label{J_DEVICE}

\symbol
{\includegraphics{njfetSymbol}}
{\includegraphics{pjfetSymbol}}

\device
J<name> <drain node> <gate node> <source node> <model name>
+ [area value] [device parameters]

\examples
\begin{alltt}
JIN 100 1 0 JFAST
J13 22 14 23 JNOM 2.0
J1 1 2 0 2N5114
\end{alltt}

\model
\begin{alltt}
.MODEL <model name> NJF [model parameters]
.MODEL <model name> PJF [model parameters]
\end{alltt}

\parameters

\begin{Parameters}

\param{drain node}
Node connected to drain.

\param{gate node}
Node connected to gate.

\param{source node}
 Node connected to source.

\param{source node}
Name of model defined in .MODEL line.

\param{area value}

The \texttt{JFET} is modeled as an intrinsic FET using an ohmic
resistance (\texttt{RD/area}) in series with the drain and another ohmic
resistance (\texttt{RS/area}) in series with the source.  \texttt{area}
is an area factor with a default of \texttt{1}.

\param{device parameters}

Parameters listed in Table~\ref{J_1_Device_Instance_Params} may be
provided as space separated \texttt{<parameter>=<value>} specifications
as needed.  Any number of parameters may be specified.

\end{Parameters}

\comments

The \texttt{JFET} was first proposed and analyzed by Shockley.  The
SPICE- compatible \texttt{JFET} model is an approximation to the
Shockley analysis that employs an adjustable parameter B.  Both the
Shockley formulation and the SPICE approximation are available in Xyce.

\end{Device}

\pagebreak

\paragraph{Device Parameters}
% This table was generated by Xyce:
%   Xyce -doc J 1
%
\index{jfet!device instance parameters}
\begin{DeviceParamTableGenerated}{JFET Device Instance Parameters}{J_1_Device_Instance_Params}
AREA & Device area & m$^{2}$ & 1 \\ \hline
DTEMP & Device delta temperature & $^\circ$C & 0 \\ \hline
TEMP & Device temperature & -- & Ambient Temperature \\ \hline
\end{DeviceParamTableGenerated}


\paragraph{Model Parameters}
\input{J_1_Device_Model_Params}

\pagebreak

\paragraph{Device Parameters}
% This table was generated by Xyce:
%   Xyce -doc J 2
%
\index{jfet!device instance parameters}
\begin{DeviceParamTableGenerated}{JFET Device Instance Parameters}{J_2_Device_Instance_Params}
AREA & Device area & m$^{2}$ & 1 \\ \hline
DTEMP & Device delta temperature & $^\circ$C & 0 \\ \hline
TEMP & Device temperature & -- & Ambient Temperature \\ \hline
\end{DeviceParamTableGenerated}


\paragraph{Model Parameters}
\input{J_2_Device_Model_Params}

\paragraph{JFET Level selection}
\Xyce{} supports two JFET models.  LEVEL=1, the default, is the SPICE 3f5
treatment.  This model employs a doping profile parameter B.  When B=1,
the original SPICE square law is exactly implemented, and when B=0.6 the
model is close to that of Shockley.

When LEVEL=2 is selected, the Shockley model is used with some additional physics
effects:  channel length modulation and the effect of gate electric field on
mobility.  An additional parameter, DELTA, is added to the LEVEL 2 model that
allows the user to adjust the saturation voltage.

\paragraph{JFET Power Calculations}
Power dissipated in the transistor is calculated with $I_{D}*V_{DS}+I_{G}*V_{GS}$ where
$I_{D}$ is the drain current, $I_{G}$ is the gate current, $V_{DS}$ is the 
voltage drop between the drain and the source and $V_{GS}$ is the voltage drop 
between the gate and the source. This formula may differ from other simulators,
such as HSPICE and PSpice.


%%
%% MESFET Section
%%
\clearpage
\subsection{Metal-Semiconductor FET (MESFET)}
\index{device!MESFET} \index{MESFET}
% Sandia National Laboratories is a multimission laboratory managed and
% operated by National Technology & Engineering Solutions of Sandia, LLC, a
% wholly owned subsidiary of Honeywell International Inc., for the U.S.
% Department of Energy’s National Nuclear Security Administration under
% contract DE-NA0003525.

% Copyright 2002-2023 National Technology & Engineering Solutions of Sandia,
% LLC (NTESS).


\begin{Device}\label{Z_DEVICE}

\symbol
{\includegraphics{nmesfetSymbol}}
{\includegraphics{pmesfetSymbol}}

\device
\begin{alltt}
Z<name> < drain node> <gate node> <source node> <model name>
+ [area value] [device parameters]
\end{alltt}

\model
\begin{alltt}
.MODEL <model name> NMF [model parameters]
.MODEL <model name> PMF [model parameters]
\end{alltt}

\examples
\begin{alltt}
Z1 2 3 0 MESMOD AREA=1.4
Z1 7 2 3 ZM1
\end{alltt}

\parameters

\begin{Parameters}

\param{drain node}
Node connected to drain.

\param{gate node}
Node connected to gate.

\param{source node}
Node connected to source.

\param{source node}
Name of model defined in .MODEL line.

\param{area value}

The \texttt{MESFET} is modeled as an intrinsic FET using an ohmic
resistance (\texttt{RD/area}) in series with the drain and another ohmic
resistance (\texttt{RS/area}) in series with the source.  \texttt{area value}
is a scaling factor with a default of 1.

\param{device parameters}

Parameters listed in Table~\ref{Z_1_Device_Instance_Params} may be
provided as space separated \texttt{<parameter>=<value>} specifications
as needed.  Any number of parameters may be specified.

\end{Parameters}

\comments

Although MESFETs can be made of Si, such devices are not as common as
GaAs MESFETS.  And since the mobility of electrons is much higher than
holes in GaAs, nearly all commercial devices are n-type MESFETS.

\end{Device}

\pagebreak

\paragraph{Device Parameters}
% This table was generated by Xyce:
%   Xyce -doc Z 1
%
\index{mesfet!device instance parameters}
\begin{DeviceParamTableGenerated}{MESFET Device Instance Parameters}{Z_1_Device_Instance_Params}
AREA & device area & m$^{2}$ & 1 \\ \hline
DTEMP & Device delta temperature & $^\circ$C & 0 \\ \hline
TEMP & Device temperature & -- & Ambient Temperature \\ \hline
\end{DeviceParamTableGenerated}


\paragraph{Model Parameters}
\input{Z_1_Device_Model_Params}

\paragraph{MESFET Power Calculations}
Power dissipated in the transistor is calculated with $I_{D}*V_{DS}+I_{G}*V_{GS}$ where
$I_{D}$ is the drain current, $I_{G}$ is the gate current, $V_{DS}$ is the 
voltage drop between the drain and the source and $V_{GS}$ is the voltage drop 
between the gate and the source. This formula may differ from other simulators,
such as HSPICE and PSpice.


%%
%% MOSFET Subsection
%%
\clearpage
\subsection{MOS Field Effect Transistor (MOSFET)}
\index{device!MOSFET} \index{MOSFET}
% Sandia National Laboratories is a multimission laboratory managed and
% operated by National Technology & Engineering Solutions of Sandia, LLC, a
% wholly owned subsidiary of Honeywell International Inc., for the U.S.
% Department of Energy’s National Nuclear Security Administration under
% contract DE-NA0003525.

% Copyright 2002-2022 National Technology & Engineering Solutions of Sandia,
% LLC (NTESS).


\begin{Device}\label{M_DEVICE}

\symbol
{\includegraphics{nmosSymbol}}
{\includegraphics{pmosSymbol}}

\device
\begin{alltt}
M<name> <drain node> <gate node> <source node>
+ <bulk/substrate node> <model name>
+ [L=<value>] [W=<value>]
+ [AD=<value>] [AS=<value>]
+ [PD=<value>] [PS=<value>]
+ [NRD=<value>] [NRS=<value>]
+ [M=<value] [IC=<value, ...>]
\end{alltt}

\vbox{\hrulefill}
\item[Special Form (BSIMSOI)]
\begin{alltt}
M<name> <drain node> <gate node> <source node>
+ <substrate node (E)>
+ [<External body contact (P)>]
+ [<internal body contact (B)>]
+ [<temperature node (T)>]
+ <model name>
+ [L=<value>] [W=<value>]
+ [AD=<value>] [AS=<value>]
+ [PD=<value>] [PS=<value>]
+ [NRD=<value>] [NRS=<value>] [NRB=<value>]
+ [BJTOFF=<value>]
+ [IC=<val>,<val>,<val>,<val>,<val>]
+ [RTH0=<val>] [CTH0=<val>]
+ [NBC=<val>] [NSEG=<val>] [PDBCP=<val>] [PSBCP=<val>]
+ [AGBCP=<val>] [AEBCP=<val>] [VBSUSR=<val>] [TNODEOUT]
+ [FRBODY=<val>] [M=<value>]
\end{alltt}
\vbox{\hrulefill}

\item[Special Form (MVS)]
\begin{alltt}
M<name> <drain node> <gate node> <source node> <model name>
\end{alltt}

\item[Special Form (PSP103 with self-heating)]
\begin{alltt}
M<name> <drain node> <gate node> <source node> <bulk node> <dt node> <model name> [instance parameters]
\end{alltt}

\model
\begin{alltt}
.MODEL <model name> NMOS [model parameters]
.MODEL <model name> PMOS [model parameters]
\end{alltt}

\examples
\begin{alltt}
M5 4 12 3 0 PNOM L=20u W=10u
M3 5 13 10 0 PSTRONG
M6 7 13 10 0 PSTRONG M=2
M8 10 12 100 100 NWEAK L=30u W=20u
+ AD=288p AS=288p PD=60u PS=60u NRD=14 NRS=24
\end{alltt}

\parameters

\begin{Parameters}

\param{\vbox{\hbox{L\hfil}\hbox{M\hfil}}}

The MOSFET channel length and width that are decreased to get the actual
channel length and width. They may be given in the device
\texttt{.MODEL} or \texttt{.OPTIONS} statements. The value in the device
statement overrides the value in the model statement, which overrides
the value in the \texttt{.OPTIONS} statement. If \texttt{L} or \texttt{W}
values are not given, their default value is 100~$\mu$m.

\param{\vbox{\hbox{AD\hfil}\hbox{AS\hfil}}}

The drain and source diffusion areas. Defaults for \texttt{AD} and
\texttt{AS} can be set in the \texttt{.OPTIONS} statement.  If
\texttt{AD} or \texttt{AS} defaults are not set, their default value is
0.

\param{\vbox{\hbox{PD\hfil}\hbox{PS\hfil}}}
The drain and source diffusion perimeters. Their default value is 0.

\param{\vbox{\hbox{NRD\hfil}\hbox{NRS\hfil}}}

Multipliers (in units of $\Box$) that can be multiplied by \texttt{RSH}
to yield the parasitic (ohmic) resistances of the drain (\texttt{RD})
and source (\texttt{RS}), respectively.  \texttt{NRD}, \texttt{NRS}
default to 0.

Consider a square sheet of resistive material. Analysis shows that the
resistance between two parallel edges of such a sheet depends upon its
composition and thickness, but is independent of its size as long as it is
square. In other words, the resistance will be the same whether the square's
edge is 2~mm, 2~cm, or 2~m. For this reason, the \emph{sheet resistance} of
such a layer, abbreviated \texttt{RSH}, has units of Ohms per square,
written $\mathsf{\Omega}/\Box$.

\param{M}

If specified, the value is used as a number of parallel MOSFETs to be
simulated.  For example, if \texttt{M=2} is specified, \Xyce{} simulates two
identical mosfets connected to the same nodes in parallel.

\param{IC}

The BSIM3 (model level 9), BSIM4 (model level 14 or 54) and BSIMSOI (model
level 10) allow one to specify the initial voltage difference across
nodes of the device during the DC operating point calculation.  For the
BSIM3 and BSIM4 the syntax is \texttt{IC=$V_{ds}, V_{gs}, V_{bs}$}
where $V_{ds}$ is the voltage difference between the drain and source,
$V_{gs}$ is the voltage difference between the gate and source and
$V_{bs}$ is the voltage difference between the body and source.  The
BSIMSOI device's initial condition syntax is \texttt{IC=$V_{ds},
  V_{gs}, V_{bs}, V_{es}, V_{ps}$} where the two extra terms are the
voltage difference between the substrate and source, and the external
body and source nodes respectively.  Note that for any of these lists of
voltage differences, fewer than the full number of options may be
specified.  For example, \texttt{IC=5.0} specifies an initial condition on $V_{ds}$
but does not specifiy any initial conditions on the other nodes.
Therefore, one cannot specify $V_{gs}$ without specifying $V_{ds}$, etc.

It is illegal to specify initial conditions on any nodes that are tied
together.  \Xyce{} attempts to catch such errors, but complex circuits may
stymie this error trap.

\end{Parameters}

\vbox{\hrulefill}
\item[BSIM-SOI Options]

There are a large number of extra instance parameters and optional
nodes available for the BSIM-SOI (level 10 (BSIM-SOI 3.2), level 70
(BSIM-SOI 4.6.1), and level 70450 (BSIM-SOI 4.5.0)) MOSFET.  Please
consult the BSIM-SOI technical manual, available at
\url{http://bsim.berkeley.edu/models/bsimsoi/}, for full details.

\begin{Parameters}

\param{substrate node}

The fourth node of the BSIM-SOI device is always the substrate node,
which is referred to as the \texttt{E} node. 

\param{external body contact node}

If given, the fifth node is the external body contact node,
\texttt{P}.  It is connected to the internal body node through a body
tie resistor.  If \texttt{P} is not given, the internal body node is
not accessible from the netlist and floats.

{\em For the BSIM-SOI 3.2 (level=10) only):} If there are only five
nodes specified and \texttt{TNODEOUT} is also specified, the fifth
node is the temperature node instead.

\param{internal body contact node}

If given, the sixth node is the internal body contact node, \texttt{B}.  It is
connected to the external body node through a body tie resistor.  If \texttt{B}
is not given and \texttt{P} is given, the internal body node is not accessible
from the netlist, but is still tied to the external body contact through the
tie resistance.

{\em For the BSIM-SOI 3.2 (level=10) only):} If there are only six
nodes specified and \texttt{TNODEOUT} is also specified, the sixth
node is the temperature node instead.

\param{temperature node}

{\em For the BSIM-SOI 3.2 (level=10) only):} If the parameter \texttt{TNODEOUT} is specified, the final node (fifth, sixth,
or seventh) is interpreted as a temperature node.  The temperature node is
intended for thermal coupling simulation.

{\em For the BSIM-SOI 4.x (level=70 or 70450) only):} The temperature
node is only accessible for thermal coupling if it is the seventh
node.  It is available for printing as an internal node in all other
configurations.

\param{BJTOFF}
Turns off the parasitic BJT currents.

\param{IC}
The \texttt{IC} parameter allows specification of the five junction initial
conditions, $V_{ds}, V_{gs}, V_{bs}, V_{es}$ and $V_{ps}$.  $V_{ps}$ is ignored
in a four-terminal device.

\param{RTH0}
Thermal resistance per unit width.  Taken from model card if not given.

\param{CTH0}
Thermal capacitance per unit width.  Taken from model card if not given.

\param{NBC}
Number of body contact isolation edges.

\param{NSEG}
Number of segments for channel width partitioning.

\param{PDBCP}
Parasitic perimeter length for body contact at drain side.

\param{PSBCP}
Parasitic perimeter length for body contact at source side.

\param{AGBCP}
Parasitic gate-to-body overlap area for body contact.

\param{AEBCP}
Parasitic body-to-substrate overlap area for body contact.

\param{VBSUSR}
Optional initial value of VBS specified by user for use in transient
analysis.  (unused in \Xyce{}).

\param{FRBODY}
Layout-dependent body resistance coefficient.

\end{Parameters}

\comments

The simulator provides three MOSFET device models, which differ in the
formulation of the I-V characteristic. The \texttt{LEVEL} parameter
selects among different models as shown below.

For HSPICE compatibility, the BSIM4 model can be specified with either
level 14 or level 54.

\end{Device}

\paragraph{MOSFET Operating Temperature}
Model parameters may be assigned unique measurement temperatures using the
\textrmb{TNOM} model parameter. See the MOSFET model parameters for more
information.

\paragraph{MOSFET Power Calculations}
Power dissipated in the transistor is calculated with $I_{D}*V_{DS}+I_{G}*V_{GS}$ where
$I_{D}$ is the drain current, $I_{G}$ is the gate current, $V_{DS}$ is the
voltage drop between the drain and the source and $V_{GS}$ is the voltage drop
between the gate and the source. This formula may differ from other simulators,
such as HSPICE and PSpice.

\paragraph{Internal Device Variables Accessible with {\tt N()} Syntax}
For the BSIM3, BSIM4, and BSIM-CMG version 110 models, several
internal variables have been made accessible with the {\tt N()} syntax
on a {\tt .PRINT} line.  They are $g_{m}$ (tranconductance), $V_{th}$,
$V_{ds}$, $V_{gs}$, $V_{bs}$, and $V_{dsat}$.  An example {\tt .PRINT}
line command for a MOSFET device named {\tt m1} would be:
\begin{alltt}
.print dc N(m1:gm) N(m1:Vth) N(m1:Vdsat) N(m1:Vds) N(m1:Vgs) N(m1:Vbs)
\end{alltt}
The BSIM-CMG also supports output of $I_{ds}$ (drain-source current)
in this manner.

If the user runs \texttt{Xyce -namesfile <filename> <netlist>} then
\Xyce{} will output into the first filename a list of all solution
variables generated by that netlist. This can be useful for
determining the ``fully-qualified'' device name, needed for the {\tt
  N()} syntax, if the device is in a subcircuit.

\paragraph{Instance Parameters}
Tables ~\ref{M_1_Device_Instance_Params}, ~\ref{M_2_Device_Instance_Params}, 
~\ref{M_3_Device_Instance_Params},  ~\ref{M_6_Device_Instance_Params},
\ref{M_9_Device_Instance_Params} and \ref{M_10_Device_Instance_Params}  
give the available instance parameters for the levels 1,2,3,6,9 and 10 MOSFETs,
respectively.

In addition to the parameters shown in the tables, where a list of
numbered initial condition parameters are shown, the MOSFETs support a vector
parameter for the initial conditions.  \texttt{IC1} and \texttt{IC2}
may therefore be specified compactly as \texttt{IC=<ic1>,<ic2>}.

\paragraph{Model Parameters}
Tables ~\ref{M_1_Device_Model_Params}, ~\ref{M_2_Device_Model_Params},
~\ref{M_3_Device_Model_Params}, ~\ref{M_6_Device_Model_Params},
~\ref{M_9_Device_Model_Params}, and ~\ref{M_10_Device_Model_Params}
give the available model parameters for the levels 1,2,3,6,9 and 10 MOSFETs,
respectively.

For a thorough description of MOSFET models see~\cite{Antognetti:1988, HLJHCKH,
BLETK:1997, SH:1968, VL:1980,
SSKJ:1987, Pierret:1984, YEC:1983, BSIM3:V3:1, BN}.

\subparagraph{All MOSFET models}
The parameters shared by all MOSFET model levels are principally parasitic
element values (e.g., series resistance, overlap capacitance, etc.).

\subparagraph{Model levels 1 and 3}
The DC behaviors of the level 1 and 3 MOSFET models are defined by the
parameters \textrmb{VTO}, \textrmb{KP}, \textrmb{LAMBDA}, \textrmb{PHI}, and
\textrmb{GAMMA}.  The simulator calculates these if the process parameters
(e.g., \textrmb{TOX}, and \textrmb{NSUB}) are specified, but these are always
overridden by any user-defined values. The \textrmb{VTO} value is positive
(negative) for modeling the enhancement mode and negative (positive) for the
depletion mode of N-channel (P-channel) devices.

For MOSFETs, the capacitance model enforces charge conservation,
influencing just the Level 1 and 3 models.

Effective device parameter lengths and widths are calculated as follows:
\[
P_i = P_0 + P_L / L_e + P_W / W_e
\]
where
\[
\begin{array}{rclcl}
L_e & = & \mbox{effective length} & = & \mathbf{L} - (2 \cdot \mathbf{LD}) \\
W_e & = & \mbox{effective width} & = & \mathbf{W} - (2 \cdot \mathbf{WD})
\end{array}
\]

See \textrmb{.MODEL} (model definition) for more information.

\subparagraph{Model level 9 (BSIM3 version 3.2.2)}
The University of California, Berkeley BSIM3 model is a physical-based model
with a large number of dependencies on essential dimensional and processing
parameters.  It incorporates the key effects that are critical in modeling
deep-submicrometer MOSFETs.  These include threshold voltage reduction,
nonuniform doping, mobility reduction due to the vertical field, bulk charge
effect, carrier velocity saturation, drain-induced barrier lowering (DIBL),
channel length modulation (CLM), hot-carrier-induced output resistance
reduction, subthreshold conduction, source/drain parasitic resistance,
substrate current induced body effect (SCBE) and drain voltage reduction in LDD
structure.

The BSIM3 Version 3.2.2 model is a deep submicron MOSFET model with several major
enhancements over earlier versions.  These include a single I-V formula used
to define the current and output conductance for operating regions, improved
narrow width device modeling, a superior capacitance model with improved short
and narrow geometry models, a new relaxation-time model to better transient
modeling and enhanced model fitting of assorted W/L ratios using a single
parameter set.  This version preserves the large number of integrated
dependencies on dimensional and processing parameters of the Version 2 model.
For further information, see Reference~\cite{HLJHCKH}.

\subparagraph{Additional notes}
\begin{enumerate}
\item If any of the following BSIM3 3.2.2 model parameters are not specified,
they are computed via the following:

If \textrmb{VTHO} is not specified, then:
\[
\mathbf{VTHO} = \mathbf{VFB} + \phi_s \mathbf{K1} \sqrt{\phi_s}
\]
where:
\[
\mathbf{VFB} = -1.0
\]
If \textrmb{VTHO} is given, then:
\begin{eqnarray*}
\mathbf{VFB} & = & \mathbf{VTHO} - \phi_s + \mathbf{K1}\sqrt{phi_s} \\
\mathbf{VBX} & = & \phi_s - \frac{q\cdot\mathbf{NCH} \cdot
\mathbf{XT}^2}{2\varepsilon_{si}} \\
\mathbf{CF} & = & \left( \frac{2\varepsilon_{ox}}{\pi} \right)
\ln \left(1 + \frac{1}{4 \times 10^7\cdot\mathbf{TOX}} \right)
\end{eqnarray*}
where:
\[
E_g(T) = \mbox{the energy bandgap at temperature }T = 1.16 - \frac{T^2}{7.02
\times 10^4(T + 1108)}
\]

\item If \textrmb{K1} and \textrmb{K2} are not given then they are computed via
the following:
\begin{eqnarray*}
\mathbf{K1} &=& \mathbf{GAMMA2} - 2 \cdot \mathbf{K2} \sqrt{\phi_s -
\mathbf{VBM}} \\
\mathbf{K2} &=& \frac{(\mathbf{GAMMA1} -
\mathbf{GAMMA2})(\sqrt{\phi_s - \mathbf{VBX}} -
\sqrt{\phi_s})}{2\sqrt{\phi_s}(\sqrt{\phi_s - \mathbf{VBM}} -
\sqrt{\phi_s}) + \mathbf{VBM}}
\end{eqnarray*}
where:
\begin{eqnarray*}
\phi_s & = & 2V_t \ln \left(\frac{\mathbf{NCH}}{n_i} \right) \\
V_t    & = & kT / q \\
n_i    & = & 1.45 \times 10^{10} \left(\frac{T}{300.15}
\right)^{1.5} \exp \left(21.5565981 - \frac{E_g(T)}{2V_t} \right)
\end{eqnarray*}

\item If \textrmb{NCH} is not specified and \textrmb{GAMMA1} is, then:
\[
\mathbf{NCH} = \frac{\mathbf{GAMMA1^2 \times \mathbf{COX}^2}}
{2q \varepsilon_{si}}
\]
If \textrmb{GAMMA1} and \textrmb{NCH} {\em are not} specified, then
\textrmb{NCH} defaults to $1.7\times10^{23}\;m^{-3}$ and \textrmb{GAMMA1} is
computed using \textrmb{NCH}:
\[
\mathbf{GAMMA1} = \frac{\sqrt{2q\varepsilon_{si} \cdot \mathbf{NCH}}}
{\mathbf{COX}}
\]
If \textrmb{GAMMA2} is not specified, then:
\[
\mathbf{GAMMA2} = \frac{\sqrt{2q\varepsilon_{si} \cdot \mathbf{NSUB}}}
{\mathbf{COX}}
\]

\item If \textrmb{CGSO} is not specified and $\mathbf{DLC} > 0$, then:
\[
\mathbf{CGSO} = \left\{ \begin{array}{ll}
0, & ((\mathbf{DLC \cdot COX) - CGSL)} < 0        \\
0.6 \cdot \mathbf{XJ \cdot COX}, & ((\mathbf{DLC \cdot COX) - CGSL)}
\geq 0
\end{array}
\right.
\]

\item If \textrmb{CGDO} is not specified and $\mathbf{DLC} > 0$, then:
\[
\mathbf{CGDO} = \left\{ \begin{array}{ll}
0, & ((\mathbf{DLC \cdot COX) - CGSL)} < 0 \\
0.6 \cdot \mathbf{XJ \cdot COX},
& ((\mathbf{DLC \cdot COX) - CGSL)} \geq 0
\end{array}
\right. \]
\end{enumerate}

\subparagraph{Model level 10 (BSIM-SOI version 3.2)}

The BSIM-SOI is an international standard model for SOI (silicon on insulator)
circuit design and is formulated on top of the BSIM3v3 framework.
A detailed description can be found in the BSIM-SOI 3.1 User's
Manual~\cite{BSIMSOI:Manual} and the BSIM-SOI 3.2 release
notes~\cite{BSIMSOI:3p2:Notes}.

This version (v3.2) of the BSIM-SOI includes three depletion models;
the partially depleted BSIM-SOI PD (soiMod=0), the fully depleted BSIM-SOI
FD (soiMod=2), and the unified SOI model (soiMod=1).

BSIMPD is the
Partial-Depletion (PD) mode of the BSIM-SOI.  A typical PD SOI MOSFET is formed
on a thin SOI film which is layered on top of a buried oxide.  BSIMPD has
the following features and enhancements:
\begin{XyceItemize}
\item Real floating body simulation of both I-V and C-V.  The body potential is
      determined by the balance of all body current components.
\item An improved parasitic bipolar current model.  This includes enhancements in
      the various diode leakage components, second order effects (high-level
      injection and Early effect), diffusion charge equation, and temperature
      dependence of the diode junction capacitance.
\item An improved impact-ionization current model.  The contribution from BJT
      current is also modeled by the parameter Fbjtii.
\item A gate-to-body tunneling current model, which is important to thin-oxide
      SOI technologies.
\item Enhancements in the threshold voltage and bulk charge formulation of the
      high positive body bias regime.
\item Instance parameters (Pdbcp, Psbcp, Agbcp, Aebcp, Nbc) are provided to model
      the parasitics of devices with various body-contact and isolation structures.
\item An external body node (the 6th node) and other improvements are introduced
      to facilitate the modeling of distributed body resistance.
\item Self heating.  An external temperature node (the 7th node) is supported to
      facilitate the simulation of thermal coupling among neighboring devices.
\item A unique SOI low frequency noise model, including a new excess noise resulting
      from the floating body effect.
\item Width dependence of the body effect is modeled by parameters (K1,K1w1,K1w2).
\item Improved history dependence of the body charges with two new parameters
      (Fbody, DLCB).
\item An instance parameter Vbsusr is provided for users to set the transient initial
      condition of the body potential.
\item The new charge-thickness capacitance model introduced in BSIM3v3.2,
      \texttt{capMod=3}, is included.
\end{XyceItemize}

\paragraph{Quadratic Temperature Compensation}
SPICE temperature effects are the default, but MOSFET levels 18, 19 and 20 have
a more advanced temperature compensation available.  By specifying
\texttt{TEMPMODEL=QUADRATIC} in the netlist, parameters can be interpolated
quadratically between measured values extracted from data.  See
Section~\ref{Model_Interpolation} for more details.

\paragraph{MOSFET Equations}
The following equations define an N-channel MOSFET. The P-channel
devices use a reverse the sign for all voltages and currents.  The
equations use the following variables:
\begin{eqnarray*}
V_{bs}  &=&\mbox{intrinsic substrate-intrinsic source voltage} \\
V_{bd}  &=&\mbox{intrinsic substrate-intrinsic drain voltage} \\
V_{ds}  &=&\mbox{intrinsic drain-substrate source voltage} \\
V_{dsat}&=&\mbox{saturation voltage} \\
V_{gs}  &=&\mbox{intrinsic gate-intrinsic source voltage} \\
V_{gd}  &=&\mbox{intrinsic gate-intrinsic drain voltage} \\
V_t     &=&kT / q \mbox{ (thermal voltage)} \\
V_{th}  &=&\mbox{threshold voltage} \\
C_{ox}  &=&\mbox{the gate oxide capacitance per unit area} \\
f       &=&\mbox{noise frequency} \\
k       &=&\mbox{Boltzmann's constant} \\
q       &=&\mbox{electron charge} \\
Leff    &=&\mbox{effective channel length} \\
Weff    &=&\mbox{effective channel width} \\
T       &=&\mbox{analysis temperature (K)} \\
T_0     &=&\mbox{nominal temperature (set using TNOM option)}
\end{eqnarray*}
Other variables are listed in the BJT Equations section~\ref{bjt_equations}.

\clearpage
\LTXtable{\textwidth}{mosfeteqntbl}

%%
%% MOSFET Equation Capacitance Table
%%
\paragraph{Capacitance}
\LTXtable{\textwidth}{mosfeteqncaptbl}

%%
%% MOSFET Equation Temperature Effects
%%
\clearpage
\paragraph{Temperature Effects}
\LTXtable{\textwidth}{mosfeteqntemptbl}

%%
%% MOSFET Parameters Table
%%
\clearpage
\subsubsection{Level 1 MOSFET Tables (SPICE Level 1)}
% This table was generated by Xyce:
%   Xyce -doc M 1
%
\index{mosfet level 1!device instance parameters}
\begin{DeviceParamTableGenerated}{MOSFET level 1 Device Instance Parameters}{M_1_Device_Instance_Params}
AD & Drain diffusion area & m$^{2}$ & 0 \\ \hline
AS & Source diffusion area & m$^{2}$ & 0 \\ \hline
DTEMP & Device delta temperature & $^\circ$C & 0 \\ \hline
IC1 & Initial condition on Drain-Source voltage & V & 0 \\ \hline
IC2 & Initial condition on Gate-Source voltage & V & 0 \\ \hline
IC3 & Initial condition on Bulk-Source voltage & V & 0 \\ \hline
L & Channel length & m & 0 \\ \hline
M & Multiplier for M devices connected in parallel & -- & 1 \\ \hline
NRD & Multiplier for RSH to yield parasitic resistance of drain & $\Box$ & 1 \\ \hline
NRS & Multiplier for RSH to yield parasitic resistance of source & $\Box$ & 1 \\ \hline
OFF & Initial condition of no voltage drops across device & logical (T/F) & false \\ \hline
PD & Drain diffusion perimeter & m & 0 \\ \hline
PS & Source diffusion perimeter & m & 0 \\ \hline
TEMP & Device temperature & $^\circ$C & Ambient Temperature \\ \hline
W & Channel width & m & 0 \\ \hline
\end{DeviceParamTableGenerated}

% This table was generated by Xyce:
%   Xyce -doc M 1
%
\index{mosfet level 1!device model parameters}
\begin{DeviceParamTableGenerated}{MOSFET level 1 Device Model Parameters}{M_1_Device_Model_Params}
AF & Flicker noise exponent & -- & 1 \\ \hline
CBD & Zero-bias bulk-drain p-n capacitance & F & 0 \\ \hline
CBS & Zero-bias bulk-source p-n capacitance & F & 0 \\ \hline
CGBO & Gate-bulk overlap capacitance/channel length & F/m & 0 \\ \hline
CGDO & Gate-drain overlap capacitance/channel width & F/m & 0 \\ \hline
CGSO & Gate-source overlap capacitance/channel width & F/m & 0 \\ \hline
CJ & Bulk p-n zero-bias bottom capacitance/area & F/m$^{2}$ & 0 \\ \hline
CJSW & Bulk p-n zero-bias sidewall capacitance/area & F/m$^{2}$ & 0 \\ \hline
FC & Bulk p-n forward-bias capacitance coefficient & -- & 0.5 \\ \hline
GAMMA & Bulk threshold parameter & V$^{1/2}$ & 0 \\ \hline
IS & Bulk p-n saturation current & A & 1e-14 \\ \hline
JS & Bulk p-n saturation current density & A/m$^{2}$ & 0 \\ \hline
KF & Flicker noise coefficient & -- & 0 \\ \hline
KP & Transconductance coefficient & A/V$^{2}$ & 2e-05 \\ \hline
L & Default channel length & m & 0.0001 \\ \hline
LAMBDA & Channel-length modulation & V$^{-1}$ & 0 \\ \hline
LD & Lateral diffusion length & m & 0 \\ \hline
MJ & Bulk p-n bottom grading coefficient & -- & 0.5 \\ \hline
MJSW & Bulk p-n sidewall grading coefficient & -- & 0.5 \\ \hline
NSS & Surface state density & cm$^{-2}$ & 0 \\ \hline
NSUB & Substrate doping density & cm$^{-3}$ & 0 \\ \hline
PB & Bulk p-n bottom potential & V & 0.8 \\ \hline
PHI & Surface potential & V & 0.6 \\ \hline
RD & Drain ohmic resistance & $\mathsf{\Omega}$ & 0 \\ \hline
RS & Source ohmic resistance & $\mathsf{\Omega}$ & 0 \\ \hline
RSH & Drain,source diffusion sheet resistance & $\mathsf{\Omega}$ & 0 \\ \hline
TEMPMODEL & Specifies the type of parameter interpolation over temperature & -- & 'NONE' \\ \hline
TNOM & Nominal device temperature & $^\circ$C & 27 \\ \hline
TOX & Gate oxide thickness & m & 1e-07 \\ \hline
TPG & Gate material type (-1 = same as substrate) 0 = aluminum,1 = opposite of substrate) & -- & 0 \\ \hline
U0 & Surface mobility (alias for UO) & 1/(Vcm$^{2}$s) & 600 \\ \hline
UO & Surface mobility & 1/(Vcm$^{2}$s) & 600 \\ \hline
VT0 & Zero-bias threshold voltage (alias for VTO) & V & 0 \\ \hline
VTO & Zero-bias threshold voltage & V & 0 \\ \hline
W & Default channel width & m & 0.0001 \\ \hline
\end{DeviceParamTableGenerated}

\clearpage
\subsubsection{Level 2 MOSFET Tables (SPICE Level 2)}
% This table was generated by Xyce:
%   Xyce -doc M 2
%
\index{mosfet level 2!device instance parameters}
\begin{DeviceParamTableGenerated}{MOSFET level 2 Device Instance Parameters}{M_2_Device_Instance_Params}
AD & Drain diffusion area & m$^{2}$ & 0 \\ \hline
AS & Source diffusion area & m$^{2}$ & 0 \\ \hline
DTEMP & Device delta temperature & $^\circ$C & 0 \\ \hline
IC1 & Initial condition on Drain-Source voltage & V & 0 \\ \hline
IC2 & Initial condition on Gate-Source voltage & V & 0 \\ \hline
IC3 & Initial condition on Bulk-Source voltage & V & 0 \\ \hline
L & Channel length & m & 0 \\ \hline
M & Multiplier for M devices connected in parallel & -- & 1 \\ \hline
NRD & Multiplier for RSH to yield parasitic resistance of drain & $\Box$ & 1 \\ \hline
NRS & Multiplier for RSH to yield parasitic resistance of source & $\Box$ & 1 \\ \hline
OFF & Initial condition of no voltage drops across device & logical (T/F) & false \\ \hline
PD & Drain diffusion perimeter & m & 0 \\ \hline
PS & Source diffusion perimeter & m & 0 \\ \hline
TEMP & Device temperature & $^\circ$C & Ambient Temperature \\ \hline
W & Channel width & m & 0 \\ \hline
\end{DeviceParamTableGenerated}

% This table was generated by Xyce:
%   Xyce -doc M 2
%
\index{mosfet level 2!device model parameters}
\begin{DeviceParamTableGenerated}{MOSFET level 2 Device Model Parameters}{M_2_Device_Model_Params}
AF & Flicker noise exponent & -- & 1 \\ \hline
CBD & Zero-bias bulk-drain p-n capacitance & F & 0 \\ \hline
CBS & Zero-bias bulk-source p-n capacitance & F & 0 \\ \hline
CGBO & Gate-bulk overlap capacitance/channel length & F/m & 0 \\ \hline
CGDO & Gate-drain overlap capacitance/channel width & F/m & 0 \\ \hline
CGSO & Gate-source overlap capacitance/channel width & F/m & 0 \\ \hline
CJ & Bulk p-n zero-bias bottom capacitance/area & F/m$^{2}$ & 0 \\ \hline
CJSW & Bulk p-n zero-bias sidewall capacitance/area & F/m$^{2}$ & 0 \\ \hline
DELTA & Width effect on threshold & -- & 0 \\ \hline
FC & Bulk p-n forward-bias capacitance coefficient & -- & 0.5 \\ \hline
GAMMA & Bulk threshold parameter & V$^{1/2}$ & 0 \\ \hline
IS & Bulk p-n saturation current & A & 1e-14 \\ \hline
JS & Bulk p-n saturation current density & A/m$^{2}$ & 0 \\ \hline
KF & Flicker noise coefficient & -- & 0 \\ \hline
KP & Transconductance coefficient & A/V$^{2}$ & 2e-05 \\ \hline
L & Default channel length & m & 0.0001 \\ \hline
LAMBDA & Channel-length modulation & V$^{-1}$ & 0 \\ \hline
LD & Lateral diffusion length & m & 0 \\ \hline
MJ & Bulk p-n bottom grading coefficient & -- & 0.5 \\ \hline
MJSW & Bulk p-n sidewall grading coefficient & -- & 0.5 \\ \hline
NEFF & Total channel charge coeff. & -- & 1 \\ \hline
NFS & Fast surface state density & -- & 0 \\ \hline
NSS & Surface state density & cm$^{-2}$ & 0 \\ \hline
NSUB & Substrate doping density & cm$^{-3}$ & 0 \\ \hline
PB & Bulk p-n bottom potential & V & 0.8 \\ \hline
PHI & Surface potential & V & 0.6 \\ \hline
RD & Drain ohmic resistance & $\mathsf{\Omega}$ & 0 \\ \hline
RS & Source ohmic resistance & $\mathsf{\Omega}$ & 0 \\ \hline
RSH & Drain,source diffusion sheet resistance & $\mathsf{\Omega}$ & 0 \\ \hline
TEMPMODEL & Specifies the type of parameter interpolation over temperature & -- & 'NONE' \\ \hline
TNOM & Nominal device temperature & $^\circ$C & 27 \\ \hline
TOX & Gate oxide thickness & m & 1e-07 \\ \hline
TPG & Gate material type (-1 = same as substrate, 0 = aluminum,1 = opposite of substrate) & -- & 0 \\ \hline
U0 & Surface mobility (alias for UO) & 1/(Vcm$^{2}$s) & 600 \\ \hline
UCRIT & Crit. field for mob. degradation & -- & 10000 \\ \hline
UEXP & Crit. field exp for mob. deg. & -- & 0 \\ \hline
UO & Surface mobility & 1/(Vcm$^{2}$s) & 600 \\ \hline
VMAX & Maximum carrier drift velocity & -- & 0 \\ \hline
VT0 & Zero-bias threshold voltage (alias for VTO) & V & 0 \\ \hline
VTO & Zero-bias threshold voltage & V & 0 \\ \hline
W & Default channel width & m & 0.0001 \\ \hline
XJ & Junction depth & -- & 0 \\ \hline
\end{DeviceParamTableGenerated}

\clearpage
\subsubsection{Level 3 MOSFET Tables (SPICE Level 3)}
% This table was generated by Xyce:
%   Xyce -doc M 3
%
\index{mosfet level 3!device instance parameters}
\begin{DeviceParamTableGenerated}{MOSFET level 3 Device Instance Parameters}{M_3_Device_Instance_Params}
AD & Drain diffusion area & m$^{2}$ & 0 \\ \hline
AS & Source diffusion area & m$^{2}$ & 0 \\ \hline
DTEMP & Device delta temperature & $^\circ$C & 0 \\ \hline
IC1 & Initial condition on Drain-Source voltage & V & 0 \\ \hline
IC2 & Initial condition on Gate-Source voltage & V & 0 \\ \hline
IC3 & Initial condition on Bulk-Source voltage & V & 0 \\ \hline
L & Channel length & m & 0 \\ \hline
M & Multiplier for M devices connected in parallel & -- & 1 \\ \hline
NRD & Multiplier for RSH to yield parasitic resistance of drain & $\Box$ & 1 \\ \hline
NRS & Multiplier for RSH to yield parasitic resistance of source & $\Box$ & 1 \\ \hline
OFF & Initial condition of no voltage drops across device & logical (T/F) & false \\ \hline
PD & Drain diffusion perimeter & m & 0 \\ \hline
PS & Source diffusion perimeter & m & 0 \\ \hline
TEMP & Device temperature & $^\circ$C & Ambient Temperature \\ \hline
W & Channel width & m & 0 \\ \hline
\end{DeviceParamTableGenerated}

% This table was generated by Xyce:
%   Xyce -doc M 3
%
\index{mosfet level 3!device model parameters}
\begin{DeviceParamTableGenerated}{MOSFET level 3 Device Model Parameters}{M_3_Device_Model_Params}
AF & Flicker noise exponent & -- & 1 \\ \hline
CBD & Zero-bias bulk-drain p-n capacitance & F & 0 \\ \hline
CBS & Zero-bias bulk-source p-n capacitance & F & 0 \\ \hline
CGBO & Gate-bulk overlap capacitance/channel length & F/m & 0 \\ \hline
CGDO & Gate-drain overlap capacitance/channel width & F/m & 0 \\ \hline
CGSO & Gate-source overlap capacitance/channel width & F/m & 0 \\ \hline
CJ & Bulk p-n zero-bias bottom capacitance/area & F/m$^{2}$ & 0 \\ \hline
CJSW & Bulk p-n zero-bias sidewall capacitance/area & F/m$^{2}$ & 0 \\ \hline
DELTA & Width effect on threshold & -- & 0 \\ \hline
ETA & Static feedback & -- & 0 \\ \hline
FC & Bulk p-n forward-bias capacitance coefficient & -- & 0.5 \\ \hline
GAMMA & Bulk threshold parameter & V$^{1/2}$ & 0 \\ \hline
IS & Bulk p-n saturation current & A & 1e-14 \\ \hline
JS & Bulk p-n saturation current density & A/m$^{2}$ & 0 \\ \hline
KAPPA & Saturation field factor & -- & 0.2 \\ \hline
KF & Flicker noise coefficient & -- & 0 \\ \hline
KP & Transconductance coefficient & A/V$^{2}$ & 2e-05 \\ \hline
L & Default channel length & m & 0.0001 \\ \hline
LD & Lateral diffusion length & m & 0 \\ \hline
MJ & Bulk p-n bottom grading coefficient & -- & 0.5 \\ \hline
MJSW & Bulk p-n sidewall grading coefficient & -- & 0.33 \\ \hline
NFS & Fast surface state density & cm$^{-2}$ & 0 \\ \hline
NSS & Surface state density & cm$^{-2}$ & 0 \\ \hline
NSUB & Substrate doping density & cm$^{-3}$ & 0 \\ \hline
PB & Bulk p-n bottom potential & V & 0.8 \\ \hline
PHI & Surface potential & V & 0.6 \\ \hline
RD & Drain ohmic resistance & $\mathsf{\Omega}$ & 0 \\ \hline
RS & Source ohmic resistance & $\mathsf{\Omega}$ & 0 \\ \hline
RSH & Drain,source diffusion sheet resistance & $\mathsf{\Omega}$ & 0 \\ \hline
TEMPMODEL & Specifies the type of parameter interpolation over temperature & -- & 'NONE' \\ \hline
THETA & Mobility modulation & V$^{-1}$ & 0 \\ \hline
TNOM & Nominal device temperature & $^\circ$C & 27 \\ \hline
TOX & Gate oxide thickness & m & 1e-07 \\ \hline
TPG & Gate material type (-1 = same as substrate,0 = aluminum,1 = opposite of substrate) & -- & 1 \\ \hline
U0 & Surface mobility (alias for UO) & 1/(Vcm$^{2}$s) & 600 \\ \hline
UO & Surface mobility & 1/(Vcm$^{2}$s) & 600 \\ \hline
VMAX & Maximum drift velocity & m/s & 0 \\ \hline
VT0 & Zero-bias threshold voltage (alias for VTO) & V & 0 \\ \hline
VTO & Zero-bias threshold voltage & V & 0 \\ \hline
W & Default channel width & m & 0.0001 \\ \hline
XJ & Metallurgical junction depth & m & 0 \\ \hline
\end{DeviceParamTableGenerated}

\clearpage
\subsubsection{Level 6 MOSFET Tables (SPICE Level 6)}
% This table was generated by Xyce:
%   Xyce -doc M 6
%
\index{mosfet level 6!device instance parameters}
\begin{DeviceParamTableGenerated}{MOSFET level 6 Device Instance Parameters}{M_6_Device_Instance_Params}
AD & Drain diffusion area & m$^{2}$ & 0 \\ \hline
AS & Source diffusion area & m$^{2}$ & 0 \\ \hline
DTEMP & Device delta temperature & $^\circ$C & 0 \\ \hline
IC1 & Initial condition on Drain-Source voltage & V & 0 \\ \hline
IC2 & Initial condition on Gate-Source voltage & V & 0 \\ \hline
IC3 & Initial condition on Bulk-Source voltage & V & 0 \\ \hline
L & Channel length & m & 0 \\ \hline
M & Multiplier for M devices connected in parallel & -- & 1 \\ \hline
NRD & Multiplier for RSH to yield parasitic resistance of drain & $\Box$ & 1 \\ \hline
NRS & Multiplier for RSH to yield parasitic resistance of source & $\Box$ & 1 \\ \hline
OFF & Initial condition of no voltage drops across device & logical (T/F) & false \\ \hline
PD & Drain diffusion perimeter & m & 0 \\ \hline
PS & Source diffusion perimeter & m & 0 \\ \hline
TEMP & Device temperature & $^\circ$C & Ambient Temperature \\ \hline
W & Channel width & m & 0 \\ \hline
\end{DeviceParamTableGenerated}

% This table was generated by Xyce:
%   Xyce -doc M 6
%
\index{mosfet level 6!device model parameters}
\begin{DeviceParamTableGenerated}{MOSFET level 6 Device Model Parameters}{M_6_Device_Model_Params}
AF & Flicker noise exponent & -- & 1 \\ \hline
CBD & Zero-bias bulk-drain p-n capacitance & F & 0 \\ \hline
CBS & Zero-bias bulk-source p-n capacitance & F & 0 \\ \hline
CGBO & Gate-bulk overlap capacitance/channel length & F/m & 0 \\ \hline
CGDO & Gate-drain overlap capacitance/channel width & F/m & 0 \\ \hline
CGSO & Gate-source overlap capacitance/channel width & F/m & 0 \\ \hline
CJ & Bulk p-n zero-bias bottom capacitance/area & F/m$^{2}$ & 0 \\ \hline
CJSW & Bulk p-n zero-bias sidewall capacitance/area & F/m$^{2}$ & 0 \\ \hline
FC & Bulk p-n forward-bias capacitance coefficient & -- & 0.5 \\ \hline
GAMMA & Bulk threshold parameter & -- & 0 \\ \hline
GAMMA1 & Bulk threshold parameter 1 & -- & 0 \\ \hline
IS & Bulk p-n saturation current & A & 1e-14 \\ \hline
JS & Bulk p-n saturation current density & A/m$^{2}$ & 0 \\ \hline
KC & Saturation current factor & -- & 5e-05 \\ \hline
KF & Flicker noise coefficient & -- & 0 \\ \hline
KV & Saturation voltage factor & -- & 2 \\ \hline
LAMBDA & Channel length modulation param. & -- & 0 \\ \hline
LAMBDA0 & Channel length modulation param. 0 & -- & 0 \\ \hline
LAMBDA1 & Channel length modulation param. 1 & -- & 0 \\ \hline
LD & Lateral diffusion length & m & 0 \\ \hline
MJ & Bulk p-n bottom grading coefficient & -- & 0.5 \\ \hline
MJSW & Bulk p-n sidewall grading coefficient & -- & 0.5 \\ \hline
NC & Saturation current coeff. & -- & 1 \\ \hline
NSS & Surface state density & cm$^{-2}$ & 0 \\ \hline
NSUB & Substrate doping density & cm$^{-3}$ & 0 \\ \hline
NV & Saturation voltage coeff. & -- & 0.5 \\ \hline
NVTH & Threshold voltage coeff. & -- & 0.5 \\ \hline
PB & Bulk p-n bottom potential & V & 0.8 \\ \hline
PHI & Surface potential & V & 0.6 \\ \hline
PS & Sat. current modification  par. & -- & 0 \\ \hline
RD & Drain ohmic resistance & $\mathsf{\Omega}$ & 0 \\ \hline
RS & Source ohmic resistance & $\mathsf{\Omega}$ & 0 \\ \hline
RSH & Drain,source diffusion sheet resistance & $\mathsf{\Omega}$ & 0 \\ \hline
SIGMA & Static feedback effect par. & -- & 0 \\ \hline
TEMPMODEL & Specifies the type of parameter interpolation over temperature & -- & 'NONE' \\ \hline
TNOM & Nominal device temperature & $^\circ$C & 27 \\ \hline
TOX & Gate oxide thickness & m & 1e-07 \\ \hline
TPG & Gate material type (-1 = same as substrate,0 = aluminum,1 = opposite of substrate) & -- & 1 \\ \hline
U0 & Surface mobility (alias for UO) & 1/(Vcm$^{2}$s) & 600 \\ \hline
UO & Surface mobility & 1/(Vcm$^{2}$s) & 600 \\ \hline
VT0 & Zero-bias threshold voltage (alias for VTO) & V & 0 \\ \hline
VTO & Zero-bias threshold voltage & V & 0 \\ \hline
\end{DeviceParamTableGenerated}

\clearpage
\subsubsection{Level 9 MOSFET Tables (BSIM3)}
For complete documentation of the BSIM3 model, see the users' manual for
the BSIM3, available for download at
\url{http://bsim.berkeley.edu/models/bsim4/bsim3/}.
\Xyce{} implements Version 3.2.2 of the BSIM3.

In addition to the parameters shown in
table~\ref{M_9_Device_Instance_Params}, the BSIM3 supports a vector
parameter for the initial conditions.  \texttt{IC1} through
\texttt{IC3} may therefore be specified compactly as
\texttt{IC=<ic1>,<ic2>,<ic3>}.

\textbf{NOTE:  Many BSIM3 parameters listed in
tables~\ref{M_9_Device_Instance_Params} and \ref{M_9_Device_Model_Params} as
having default values of zero are actually replaced with internally computed
defaults if not given.  Specifying zero in your model card will override this
internal computation.  It is recommended that you only set model parameters
that you are actually changing from defaults and that you not generate model
cards containing default values from the tables.}
\input{M_9_Device_Instance_Params}
\input{M_9_Device_Model_Params}

\clearpage
\subsubsection{Level 10 MOSFET Tables (BSIM-SOI)}
For complete documentation of the BSIM-SOI model, see the users' manual
for the BSIM-SOI, available for download at
\url{http://bsim.berkeley.edu/models/bsimsoi/}.
\Xyce{} implements Version 3.2 of the BSIM-SOI, you will have to get the
documentation from the FTP archive on the Berkeley site.

In addition to the parameters shown in table~\ref{M_10_Device_Instance_Params}, 
the BSIM3SOI supports a vector parameter for the initial conditions.    \texttt{IC1} through \texttt{IC5}
may therefore be specified compactly as \texttt{IC=<ic1>,<ic2>,<ic3>, <ic4>,<ic5>}.

\textbf{NOTE:  Many BSIM SOI parameters listed in
tables~\ref{M_10_Device_Instance_Params} and \ref{M_10_Device_Model_Params} as
having default values of zero are actually replaced with internally computed
defaults if not given.  Specifying zero in your model card will override this
internal computation.  It is recommended that you only set model parameters
that you are actually changing from defaults and that you not generate model
cards containing default values from the tables.}
% This table was generated by Xyce:
%   Xyce -doc_cat M 10
%
\index{bsim3 soi!device instance parameters}
\begin{DeviceParamTableGenerated}{BSIM3 SOI Device Instance Parameters}{M_10_Device_Instance_Params}
BJTOFF & BJT on/off flag & logical (T/F) & 0 \\ \hline
DEBUG & BJT on/off flag & logical (T/F) & 0 \\ \hline
TNODEOUT & Flag indicating external temp node & logical (T/F) & 0 \\ \hline
VLDEBUG &  & logical (T/F) & false \\ \hline

\category{Control Parameters}\\ \hline
M & Multiplier for M devices connected in parallel & -- & 1 \\ \hline
SOIMOD & SIO model selector,SOIMOD=0: BSIMPD,SOIMOD=1: undefined model for PD and FE,SOIMOD=2: ideal FD & -- & 0 \\ \hline

\category{DC Parameters}\\ \hline
VBSUSR & Vbs specified by user & V & 0 \\ \hline

\category{Geometry Parameters}\\ \hline
AD & Drain diffusion area & m$^{2}$ & 0 \\ \hline
AEBCP & Substrate to body overlap area for bc prasitics & m$^{2}$ & 0 \\ \hline
AGBCP & Gate to body overlap area for bc parasitics & m$^{2}$ & 0 \\ \hline
AS & Source diffusion area & m$^{2}$ & 0 \\ \hline
FRBODY & Layout dependent body-resistance coefficient & -- & 1 \\ \hline
L & Channel length & m & 5e-06 \\ \hline
NBC & Number of body contact isolation edge & -- & 0 \\ \hline
NRB & Number of squares in body & -- & 1 \\ \hline
NRD & Multiplier for RSH to yield parasitic resistance of drain & $\Box$ & 1 \\ \hline
NRS & Multiplier for RSH to yield parasitic resistance of source & $\Box$ & 1 \\ \hline
NSEG & Number segments for width partitioning & -- & 1 \\ \hline
PD & Drain diffusion perimeter & m & 0 \\ \hline
PDBCP & Perimeter length for bc parasitics at drain side & m & 0 \\ \hline
PS & Source diffusion perimeter & m & 0 \\ \hline
PSBCP & Perimeter length for bc parasitics at source side & m & 0 \\ \hline
W & Channel width & m & 5e-06 \\ \hline

\category{RF Parameters}\\ \hline
RGATEMOD & Gate resistance model selector & -- & 0 \\ \hline

\category{Temperature Parameters}\\ \hline
CTH0 & Thermal capacitance & F & 0 \\ \hline
DTEMP & Device delta temperature & $^\circ$C & 0 \\ \hline
RTH0 & normalized thermal resistance & $\mathsf{\Omega}$ & 0 \\ \hline
TEMP & Device temperature & $^\circ$C & Ambient Temperature \\ \hline

\category{Voltage Parameters}\\ \hline
IC1 & Initial condition on Vds & V & 0 \\ \hline
IC2 & Initial condition on Vgs & V & 0 \\ \hline
IC3 & Initial condition on Vbs & V & 0 \\ \hline
IC4 & Initial condition on Ves & V & 0 \\ \hline
IC5 & Initial condition on Vps & V & 0 \\ \hline
OFF & Initial condition of no voltage drops accross device & logical (T/F) & false \\ \hline
\end{DeviceParamTableGenerated}

\input{M_10_Device_Model_Params}

\clearpage
\subsubsection{Level 14/54 MOSFET Tables (BSIM4)}
The level 14 MOSFET device in \Xyce{} is based on the Berkeley BSIM4 model
version 4.6.1.  (For HSPICE compatibility, the Xyce BSIM4 model can also be
specified as level 54.)  The model's parameters are given in the following
tables.  Note that the parameters have not all been properly categorized with
units in place.  For complete documentation of the BSIM4 model, see the BSIM4
User’s Manual, available for download at
\url{http://bsim.berkeley.edu/models/bsim4/}.

% This table was generated by Xyce:
%   Xyce -doc_cat M 14
%
\index{bsim4!device instance parameters}
\begin{DeviceParamTableGenerated}{BSIM4 Device Instance Parameters}{M_14_Device_Instance_Params}
AD & Drain area & -- & 0 \\ \hline
AS & Source area & -- & 0 \\ \hline
IC2 &  & -- & 0 \\ \hline
IC3 &  & -- & 0 \\ \hline
L & Length & -- & 5e-06 \\ \hline
M & Number of parallel copies & -- & 1 \\ \hline
MIN & Minimize either D or S & -- & 0 \\ \hline
NF & Number of fingers & -- & 1 \\ \hline
NGCON & Number of gate contacts & -- & 0 \\ \hline
OFF & Device is initially off & -- & false \\ \hline
PD & Drain perimeter & -- & 0 \\ \hline
PS & Source perimeter & -- & 0 \\ \hline
RBDB & Body resistance & -- & 0 \\ \hline
RBPB & Body resistance & -- & 0 \\ \hline
RBPD & Body resistance & -- & 0 \\ \hline
RBPS & Body resistance & -- & 0 \\ \hline
RBSB & Body resistance & -- & 0 \\ \hline
SA & distance between  OD edge to poly of one side  & -- & 0 \\ \hline
SB & distance between  OD edge to poly of the other side & -- & 0 \\ \hline
SC & Distance to a single well edge  & -- & 0 \\ \hline
SCA & Integral of the first distribution function for scattered well dopant & -- & 0 \\ \hline
SCB & Integral of the second distribution function for scattered well dopant & -- & 0 \\ \hline
SCC & Integral of the third distribution function for scattered well dopant & -- & 0 \\ \hline
SD & distance between neighbour fingers & -- & 0 \\ \hline
W & Width & -- & 5e-06 \\ \hline
XGW & Distance from gate contact center to device edge & -- & 0 \\ \hline

\category{Basic Parameters}\\ \hline
DELVT0 & Zero bias threshold voltage variation & V & 0 \\ \hline
DELVTO & Zero bias threshold voltage variation & V & 0 \\ \hline

\category{Control Parameters}\\ \hline
ACNQSMOD & AC NQS model selector & -- & 0 \\ \hline
GEOMOD & Geometry dependent parasitics model selector & -- & 0 \\ \hline
RBODYMOD & Distributed body R model selector & -- & 0 \\ \hline
RGATEMOD & Gate resistance model selector & -- & 0 \\ \hline
RGEOMOD & S/D resistance and contact model selector & -- & 0 \\ \hline
TRNQSMOD & Transient NQS model selector & -- & 0 \\ \hline

\category{Temperature Parameters}\\ \hline
DTEMP & Device delta temperature & $^\circ$C & 0 \\ \hline
TEMP & Device temperature & $^\circ$C & Ambient Temperature \\ \hline

\category{Voltage Parameters}\\ \hline
IC1 & Vector of initial values: Vds,Vgs,Vbs & V & 0 \\ \hline

\category{Asymmetric and Bias-Dependent $R_{ds}$ Parameters}\\ \hline
NRD & Number of squares in drain & -- & 1 \\ \hline
NRS & Number of squares in source & -- & 1 \\ \hline
\end{DeviceParamTableGenerated}

% This table was generated by Xyce:
%   Xyce -doc_cat M 14
%
\index{bsim4!device model parameters}
\begin{DeviceParamTableGenerated}{BSIM4 Device Model Parameters}{M_14_Device_Model_Params}
AF & Flicker noise exponent & -- & 1 \\ \hline
AIGSD & Parameter for Igs,d & -- & 0.0136 \\ \hline
AT & Temperature coefficient of vsat & -- & 33000 \\ \hline
BIGSD & Parameter for Igs,d & -- & 0.00171 \\ \hline
BVD & Drain diode breakdown voltage & -- & 10 \\ \hline
BVS & Source diode breakdown voltage & -- & 10 \\ \hline
CIGSD & Parameter for Igs,d & -- & 0.075 \\ \hline
CJD & Drain bottom junction capacitance per unit area & -- & 0.0005 \\ \hline
CJS & Source bottom junction capacitance per unit area & -- & 0.0005 \\ \hline
CJSWD & Drain sidewall junction capacitance per unit periphery & -- & 5e-10 \\ \hline
CJSWGD & Drain (gate side) sidewall junction capacitance per unit width & -- & 0 \\ \hline
CJSWGS & Source (gate side) sidewall junction capacitance per unit width & -- & 0 \\ \hline
CJSWS & Source sidewall junction capacitance per unit periphery & -- & 5e-10 \\ \hline
DLCIG & Delta L for Ig model & -- & 0 \\ \hline
DMCG & Distance of Mid-Contact to Gate edge & -- & 0 \\ \hline
DMCGT & Distance of Mid-Contact to Gate edge in Test structures & -- & 0 \\ \hline
DMCI & Distance of Mid-Contact to Isolation & -- & 0 \\ \hline
DMDG & Distance of Mid-Diffusion to Gate edge & -- & 0 \\ \hline
DWJ & Delta W for S/D junctions & -- & 0 \\ \hline
EF & Flicker noise frequency exponent & -- & 1 \\ \hline
EM & Flicker noise parameter & -- & 4.1e+07 \\ \hline
EPSRGATE & Dielectric constant of gate relative to vacuum & -- & 11.7 \\ \hline
GBMIN & Minimum body conductance & $\mathsf{\Omega}^{-1}$ & 1e-12 \\ \hline
IJTHDFWD & Forward drain diode forward limiting current & -- & 0.1 \\ \hline
IJTHDREV & Reverse drain diode forward limiting current & -- & 0.1 \\ \hline
IJTHSFWD & Forward source diode forward limiting current & -- & 0.1 \\ \hline
IJTHSREV & Reverse source diode forward limiting current & -- & 0.1 \\ \hline
JSD & Bottom drain junction reverse saturation current density & -- & 0.0001 \\ \hline
JSS & Bottom source junction reverse saturation current density & -- & 0.0001 \\ \hline
JSWD & Isolation edge sidewall drain junction reverse saturation current density & -- & 0 \\ \hline
JSWGD & Gate edge drain junction reverse saturation current density & -- & 0 \\ \hline
JSWGS & Gate edge source junction reverse saturation current density & -- & 0 \\ \hline
JSWS & Isolation edge sidewall source junction reverse saturation current density & -- & 0 \\ \hline
JTSD & Drain bottom trap-assisted saturation current density & -- & 0 \\ \hline
JTSS & Source bottom trap-assisted saturation current density & -- & 0 \\ \hline
JTSSWD & Drain STI sidewall trap-assisted saturation current density & -- & 0 \\ \hline
JTSSWGD & Drain gate-edge sidewall trap-assisted saturation current density & -- & 0 \\ \hline
JTSSWGS & Source gate-edge sidewall trap-assisted saturation current density & -- & 0 \\ \hline
JTSSWS & Source STI sidewall trap-assisted saturation current density & -- & 0 \\ \hline
JTWEFF\newline{\normalfont [Only for versions starting with 4.7]} & TAT current width dependence & m & 0 \\ \hline
K2WE &  K2 shift factor for well proximity effect  & -- & 0 \\ \hline
K3B & Body effect coefficient of k3 & -- & 0 \\ \hline
KF & Flicker noise coefficient & -- & 0 \\ \hline
KT1 & Temperature coefficient of Vth & -- & -0.11 \\ \hline
KT1L & Temperature coefficient of Vth & -- & 0 \\ \hline
KT2 & Body-coefficient of kt1 & -- & 0.022 \\ \hline
KU0 & Mobility degradation/enhancement coefficient for LOD & -- & 0 \\ \hline
KU0WE &  Mobility degradation factor for well proximity effect  & -- & 0 \\ \hline
KVSAT & Saturation velocity degradation/enhancement parameter for LOD & -- & 0 \\ \hline
KVTH0 & Threshold degradation/enhancement parameter for LOD & -- & 0 \\ \hline
KVTH0WE & Threshold shift factor for well proximity effect & -- & 0 \\ \hline
LA0 & Length dependence of a0 & -- & 0 \\ \hline
LA1 & Length dependence of a1 & -- & 0 \\ \hline
LA2 & Length dependence of a2 & -- & 0 \\ \hline
LACDE & Length dependence of acde & -- & 0 \\ \hline
LAGIDL & Length dependence of agidl & -- & 0 \\ \hline
LAGISL & Length dependence of agisl & -- & 0 \\ \hline
LAGS & Length dependence of ags & -- & 0 \\ \hline
LAIGBACC & Length dependence of aigbacc & -- & 0 \\ \hline
LAIGBINV & Length dependence of aigbinv & -- & 0 \\ \hline
LAIGC & Length dependence of aigc & -- & 0 \\ \hline
LAIGD & Length dependence of aigd & -- & 0 \\ \hline
LAIGS & Length dependence of aigs & -- & 0 \\ \hline
LAIGSD & Length dependence of aigsd & -- & 0 \\ \hline
LALPHA0 & Length dependence of alpha0 & -- & 0 \\ \hline
LALPHA1 & Length dependence of alpha1 & -- & 0 \\ \hline
LAT & Length dependence of at & -- & 0 \\ \hline
LB0 & Length dependence of b0 & -- & 0 \\ \hline
LB1 & Length dependence of b1 & -- & 0 \\ \hline
LBETA0 & Length dependence of beta0 & -- & 0 \\ \hline
LBGIDL & Length dependence of bgidl & -- & 0 \\ \hline
LBGISL & Length dependence of bgisl & -- & 0 \\ \hline
LBIGBACC & Length dependence of bigbacc & -- & 0 \\ \hline
LBIGBINV & Length dependence of bigbinv & -- & 0 \\ \hline
LBIGC & Length dependence of bigc & -- & 0 \\ \hline
LBIGD & Length dependence of bigd & -- & 0 \\ \hline
LBIGS & Length dependence of bigs & -- & 0 \\ \hline
LBIGSD & Length dependence of bigsd & -- & 0 \\ \hline
LCDSC & Length dependence of cdsc & -- & 0 \\ \hline
LCDSCB & Length dependence of cdscb & -- & 0 \\ \hline
LCDSCD & Length dependence of cdscd & -- & 0 \\ \hline
LCF & Length dependence of cf & -- & 0 \\ \hline
LCGDL & Length dependence of cgdl & -- & 0 \\ \hline
LCGIDL & Length dependence of cgidl & -- & 0 \\ \hline
LCGISL & Length dependence of cgisl & -- & 0 \\ \hline
LCGSL & Length dependence of cgsl & -- & 0 \\ \hline
LCIGBACC & Length dependence of cigbacc & -- & 0 \\ \hline
LCIGBINV & Length dependence of cigbinv & -- & 0 \\ \hline
LCIGC & Length dependence of cigc & -- & 0 \\ \hline
LCIGD & Length dependence of cigd & -- & 0 \\ \hline
LCIGS & Length dependence of cigs & -- & 0 \\ \hline
LCIGSD & Length dependence of cigsd & -- & 0 \\ \hline
LCIT & Length dependence of cit & -- & 0 \\ \hline
LCKAPPAD & Length dependence of ckappad & -- & 0 \\ \hline
LCKAPPAS & Length dependence of ckappas & -- & 0 \\ \hline
LCLC & Length dependence of clc & -- & 0 \\ \hline
LCLE & Length dependence of cle & -- & 0 \\ \hline
LDELTA & Length dependence of delta & -- & 0 \\ \hline
LDROUT & Length dependence of drout & -- & 0 \\ \hline
LDSUB & Length dependence of dsub & -- & 0 \\ \hline
LDVT0 & Length dependence of dvt0 & -- & 0 \\ \hline
LDVT0W & Length dependence of dvt0w & -- & 0 \\ \hline
LDVT1 & Length dependence of dvt1 & -- & 0 \\ \hline
LDVT1W & Length dependence of dvt1w & -- & 0 \\ \hline
LDVT2 & Length dependence of dvt2 & -- & 0 \\ \hline
LDVT2W & Length dependence of dvt2w & -- & 0 \\ \hline
LDVTP0 & Length dependence of dvtp0 & -- & 0 \\ \hline
LDVTP1 & Length dependence of dvtp1 & -- & 0 \\ \hline
LDVTP2\newline{\normalfont [Only for versions starting with 4.7]} & Length dependence of dvtp2 & -- & 0 \\ \hline
LDVTP3\newline{\normalfont [Only for versions starting with 4.7]} & Length dependence of dvtp3 & -- & 0 \\ \hline
LDVTP4\newline{\normalfont [Only for versions starting with 4.7]} & Length dependence of dvtp4 & -- & 0 \\ \hline
LDVTP5\newline{\normalfont [Only for versions starting with 4.7]} & Length dependence of dvtp5 & -- & 0 \\ \hline
LDWB & Length dependence of dwb & -- & 0 \\ \hline
LDWG & Length dependence of dwg & -- & 0 \\ \hline
LEGIDL & Length dependence of egidl & -- & 0 \\ \hline
LEGISL & Length dependence of egisl & -- & 0 \\ \hline
LEIGBINV & Length dependence for eigbinv & -- & 0 \\ \hline
LETA0 & Length dependence of eta0 & -- & 0 \\ \hline
LETAB & Length dependence of etab & -- & 0 \\ \hline
LEU &  Length dependence of eu & -- & 0 \\ \hline
LFGIDL\newline{\normalfont [Only for versions starting with 4.7]} & Length dependence of fgidl & -- & 0 \\ \hline
LFGISL\newline{\normalfont [Only for versions starting with 4.7]} & Length dependence of fgisl & -- & 0 \\ \hline
LFPROUT & Length dependence of pdiblcb & -- & 0 \\ \hline
LGAMMA1 & Length dependence of gamma1 & -- & 0 \\ \hline
LGAMMA2 & Length dependence of gamma2 & -- & 0 \\ \hline
LINTNOI & lint offset for noise calculation & -- & 0 \\ \hline
LK1 & Length dependence of k1 & -- & 0 \\ \hline
LK2 & Length dependence of k2 & -- & 0 \\ \hline
LK2WE &  Length dependence of k2we  & -- & 0 \\ \hline
LK3 & Length dependence of k3 & -- & 0 \\ \hline
LK3B & Length dependence of k3b & -- & 0 \\ \hline
LKETA & Length dependence of keta & -- & 0 \\ \hline
LKGIDL\newline{\normalfont [Only for versions starting with 4.7]} & Length dependence of kgidl & -- & 0 \\ \hline
LKGISL\newline{\normalfont [Only for versions starting with 4.7]} & Length dependence of kgisl & -- & 0 \\ \hline
LKT1 & Length dependence of kt1 & -- & 0 \\ \hline
LKT1L & Length dependence of kt1l & -- & 0 \\ \hline
LKT2 & Length dependence of kt2 & -- & 0 \\ \hline
LKU0 & Length dependence of ku0 & -- & 0 \\ \hline
LKU0WE &  Length dependence of ku0we  & -- & 0 \\ \hline
LKVTH0 & Length dependence of kvth0 & -- & 0 \\ \hline
LKVTH0WE & Length dependence of kvth0we & -- & 0 \\ \hline
LL & Length reduction parameter & -- & 0 \\ \hline
LLAMBDA & Length dependence of lambda & -- & 0 \\ \hline
LLC & Length reduction parameter for CV & -- & 0 \\ \hline
LLN & Length reduction parameter & -- & 1 \\ \hline
LLODKU0 & Length parameter for u0 LOD effect & -- & 0 \\ \hline
LLODVTH & Length parameter for vth LOD effect & -- & 0 \\ \hline
LLP & Length dependence of lp & -- & 0 \\ \hline
LLPE0 & Length dependence of lpe0 & -- & 0 \\ \hline
LLPEB & Length dependence of lpeb & -- & 0 \\ \hline
LMAX & Maximum length for the model & -- & 1 \\ \hline
LMIN & Minimum length for the model & -- & 0 \\ \hline
LMINV & Length dependence of minv & -- & 0 \\ \hline
LMINVCV & Length dependence of minvcv & -- & 0 \\ \hline
LMOIN & Length dependence of moin & -- & 0 \\ \hline
LNDEP & Length dependence of ndep & -- & 0 \\ \hline
LNFACTOR & Length dependence of nfactor & -- & 0 \\ \hline
LNGATE & Length dependence of ngate & -- & 0 \\ \hline
LNIGBACC & Length dependence of nigbacc & -- & 0 \\ \hline
LNIGBINV & Length dependence of nigbinv & -- & 0 \\ \hline
LNIGC & Length dependence of nigc & -- & 0 \\ \hline
LNOFF & Length dependence of noff & -- & 0 \\ \hline
LNSD & Length dependence of nsd & -- & 0 \\ \hline
LNSUB & Length dependence of nsub & -- & 0 \\ \hline
LNTOX & Length dependence of ntox & -- & 0 \\ \hline
LODETA0 & eta0 shift modification factor for stress effect & -- & 1 \\ \hline
LODK2 & K2 shift modification factor for stress effect & -- & 1 \\ \hline
LPCLM & Length dependence of pclm & -- & 0 \\ \hline
LPDIBLC1 & Length dependence of pdiblc1 & -- & 0 \\ \hline
LPDIBLC2 & Length dependence of pdiblc2 & -- & 0 \\ \hline
LPDIBLCB & Length dependence of pdiblcb & -- & 0 \\ \hline
LPDITS & Length dependence of pdits & -- & 0 \\ \hline
LPDITSD & Length dependence of pditsd & -- & 0 \\ \hline
LPHIN & Length dependence of phin & -- & 0 \\ \hline
LPIGCD & Length dependence for pigcd & -- & 0 \\ \hline
LPOXEDGE & Length dependence for poxedge & -- & 0 \\ \hline
LPRT & Length dependence of prt  & -- & 0 \\ \hline
LPRWB & Length dependence of prwb  & -- & 0 \\ \hline
LPRWG & Length dependence of prwg  & -- & 0 \\ \hline
LPSCBE1 & Length dependence of pscbe1 & -- & 0 \\ \hline
LPSCBE2 & Length dependence of pscbe2 & -- & 0 \\ \hline
LPVAG & Length dependence of pvag & -- & 0 \\ \hline
LRDSW & Length dependence of rdsw  & -- & 0 \\ \hline
LRDW & Length dependence of rdw & -- & 0 \\ \hline
LRGIDL\newline{\normalfont [Only for versions starting with 4.7]} & Length dependence of rgidl & -- & 0 \\ \hline
LRGISL\newline{\normalfont [Only for versions starting with 4.7]} & Length dependence of rgisl & -- & 0 \\ \hline
LRSW & Length dependence of rsw & -- & 0 \\ \hline
LTETA0\newline{\normalfont [Only for versions starting with 4.7]} & Length dependence of teta0 & -- & 0 \\ \hline
LTNFACTOR\newline{\normalfont [Only for versions starting with 4.7]} & Length dependence of tnfactor & -- & 0 \\ \hline
LTVFBSDOFF & Length dependence of tvfbsdoff & -- & 0 \\ \hline
LTVOFF & Length dependence of tvoff & -- & 0 \\ \hline
LTVOFFCV\newline{\normalfont [Only for versions starting with 4.7]} & Length dependence of tvoffcv & -- & 0 \\ \hline
LU0 & Length dependence of u0 & -- & 0 \\ \hline
LUA & Length dependence of ua & -- & 0 \\ \hline
LUA1 & Length dependence of ua1 & -- & 0 \\ \hline
LUB & Length dependence of ub & -- & 0 \\ \hline
LUB1 & Length dependence of ub1 & -- & 0 \\ \hline
LUC & Length dependence of uc & -- & 0 \\ \hline
LUC1 & Length dependence of uc1 & -- & 0 \\ \hline
LUCS\newline{\normalfont [Only for versions starting with 4.7]} &  Length dependence of ucs & -- & 0 \\ \hline
LUCSTE\newline{\normalfont [Only for versions starting with 4.7]} & Length dependence of ucste & -- & 0 \\ \hline
LUD & Length dependence of ud & -- & 0 \\ \hline
LUD1 & Length dependence of ud1 & -- & 0 \\ \hline
LUP & Length dependence of up & -- & 0 \\ \hline
LUTE & Length dependence of ute & -- & 0 \\ \hline
LVBM & Length dependence of vbm & -- & 0 \\ \hline
LVBX & Length dependence of vbx & -- & 0 \\ \hline
LVFB & Length dependence of vfb & -- & 0 \\ \hline
LVFBCV & Length dependence of vfbcv & -- & 0 \\ \hline
LVFBSDOFF & Length dependence of vfbsdoff & -- & 0 \\ \hline
LVOFF & Length dependence of voff & -- & 0 \\ \hline
LVOFFCV & Length dependence of voffcv & -- & 0 \\ \hline
LVSAT & Length dependence of vsat & -- & 0 \\ \hline
LVTH0 &  & -- & 0 \\ \hline
LVTL &  Length dependence of vtl & -- & 0 \\ \hline
LW & Length reduction parameter & -- & 0 \\ \hline
LW0 & Length dependence of w0 & -- & 0 \\ \hline
LWC & Length reduction parameter for CV & -- & 0 \\ \hline
LWL & Length reduction parameter & -- & 0 \\ \hline
LWLC & Length reduction parameter for CV & -- & 0 \\ \hline
LWN & Length reduction parameter & -- & 1 \\ \hline
LWR & Length dependence of wr & -- & 0 \\ \hline
LXJ & Length dependence of xj & -- & 0 \\ \hline
LXN &  Length dependence of xn & -- & 0 \\ \hline
LXRCRG1 & Length dependence of xrcrg1 & -- & 0 \\ \hline
LXRCRG2 & Length dependence of xrcrg2 & -- & 0 \\ \hline
LXT & Length dependence of xt & -- & 0 \\ \hline
MJD & Drain bottom junction capacitance grading coefficient & -- & 0.5 \\ \hline
MJS & Source bottom junction capacitance grading coefficient & -- & 0.5 \\ \hline
MJSWD & Drain sidewall junction capacitance grading coefficient & -- & 0.33 \\ \hline
MJSWGD & Drain (gate side) sidewall junction capacitance grading coefficient & -- & 0.33 \\ \hline
MJSWGS & Source (gate side) sidewall junction capacitance grading coefficient & -- & 0.33 \\ \hline
MJSWS & Source sidewall junction capacitance grading coefficient & -- & 0.33 \\ \hline
NGCON & Number of gate contacts & -- & 1 \\ \hline
NJD & Drain junction emission coefficient & -- & 1 \\ \hline
NJS & Source junction emission coefficient & -- & 1 \\ \hline
NJTS & Non-ideality factor for bottom junction & -- & 20 \\ \hline
NJTSD & Non-ideality factor for bottom junction drain side & -- & 20 \\ \hline
NJTSSW & Non-ideality factor for STI sidewall junction & -- & 20 \\ \hline
NJTSSWD & Non-ideality factor for STI sidewall junction drain side & -- & 20 \\ \hline
NJTSSWG & Non-ideality factor for gate-edge sidewall junction & -- & 20 \\ \hline
NJTSSWGD & Non-ideality factor for gate-edge sidewall junction drain side & -- & 20 \\ \hline
NTNOI & Thermal noise parameter & -- & 1 \\ \hline
PA0 & Cross-term dependence of a0 & -- & 0 \\ \hline
PA1 & Cross-term dependence of a1 & -- & 0 \\ \hline
PA2 & Cross-term dependence of a2 & -- & 0 \\ \hline
PACDE & Cross-term dependence of acde & -- & 0 \\ \hline
PAGIDL & Cross-term dependence of agidl & -- & 0 \\ \hline
PAGISL & Cross-term dependence of agisl & -- & 0 \\ \hline
PAGS & Cross-term dependence of ags & -- & 0 \\ \hline
PAIGBACC & Cross-term dependence of aigbacc & -- & 0 \\ \hline
PAIGBINV & Cross-term dependence of aigbinv & -- & 0 \\ \hline
PAIGC & Cross-term dependence of aigc & -- & 0 \\ \hline
PAIGD & Cross-term dependence of aigd & -- & 0 \\ \hline
PAIGS & Cross-term dependence of aigs & -- & 0 \\ \hline
PAIGSD & Cross-term dependence of aigsd & -- & 0 \\ \hline
PALPHA0 & Cross-term dependence of alpha0 & -- & 0 \\ \hline
PALPHA1 & Cross-term dependence of alpha1 & -- & 0 \\ \hline
PAT & Cross-term dependence of at & -- & 0 \\ \hline
PB0 & Cross-term dependence of b0 & -- & 0 \\ \hline
PB1 & Cross-term dependence of b1 & -- & 0 \\ \hline
PBD & Drain junction built-in potential & -- & 1 \\ \hline
PBETA0 & Cross-term dependence of beta0 & -- & 0 \\ \hline
PBGIDL & Cross-term dependence of bgidl & -- & 0 \\ \hline
PBGISL & Cross-term dependence of bgisl & -- & 0 \\ \hline
PBIGBACC & Cross-term dependence of bigbacc & -- & 0 \\ \hline
PBIGBINV & Cross-term dependence of bigbinv & -- & 0 \\ \hline
PBIGC & Cross-term dependence of bigc & -- & 0 \\ \hline
PBIGD & Cross-term dependence of bigd & -- & 0 \\ \hline
PBIGS & Cross-term dependence of bigs & -- & 0 \\ \hline
PBIGSD & Cross-term dependence of bigsd & -- & 0 \\ \hline
PBS & Source junction built-in potential & -- & 1 \\ \hline
PBSWD & Drain sidewall junction capacitance built in potential & -- & 1 \\ \hline
PBSWGD & Drain (gate side) sidewall junction capacitance built in potential & -- & 0 \\ \hline
PBSWGS & Source (gate side) sidewall junction capacitance built in potential & -- & 0 \\ \hline
PBSWS & Source sidewall junction capacitance built in potential & -- & 1 \\ \hline
PCDSC & Cross-term dependence of cdsc & -- & 0 \\ \hline
PCDSCB & Cross-term dependence of cdscb & -- & 0 \\ \hline
PCDSCD & Cross-term dependence of cdscd & -- & 0 \\ \hline
PCF & Cross-term dependence of cf & -- & 0 \\ \hline
PCGDL & Cross-term dependence of cgdl & -- & 0 \\ \hline
PCGIDL & Cross-term dependence of cgidl & -- & 0 \\ \hline
PCGISL & Cross-term dependence of cgisl & -- & 0 \\ \hline
PCGSL & Cross-term dependence of cgsl & -- & 0 \\ \hline
PCIGBACC & Cross-term dependence of cigbacc & -- & 0 \\ \hline
PCIGBINV & Cross-term dependence of cigbinv & -- & 0 \\ \hline
PCIGC & Cross-term dependence of cigc & -- & 0 \\ \hline
PCIGD & Cross-term dependence of cigd & -- & 0 \\ \hline
PCIGS & Cross-term dependence of cigs & -- & 0 \\ \hline
PCIGSD & Cross-term dependence of cigsd & -- & 0 \\ \hline
PCIT & Cross-term dependence of cit & -- & 0 \\ \hline
PCKAPPAD & Cross-term dependence of ckappad & -- & 0 \\ \hline
PCKAPPAS & Cross-term dependence of ckappas & -- & 0 \\ \hline
PCLC & Cross-term dependence of clc & -- & 0 \\ \hline
PCLE & Cross-term dependence of cle & -- & 0 \\ \hline
PDELTA & Cross-term dependence of delta & -- & 0 \\ \hline
PDROUT & Cross-term dependence of drout & -- & 0 \\ \hline
PDSUB & Cross-term dependence of dsub & -- & 0 \\ \hline
PDVT0 & Cross-term dependence of dvt0 & -- & 0 \\ \hline
PDVT0W & Cross-term dependence of dvt0w & -- & 0 \\ \hline
PDVT1 & Cross-term dependence of dvt1 & -- & 0 \\ \hline
PDVT1W & Cross-term dependence of dvt1w & -- & 0 \\ \hline
PDVT2 & Cross-term dependence of dvt2 & -- & 0 \\ \hline
PDVT2W & Cross-term dependence of dvt2w & -- & 0 \\ \hline
PDVTP0 & Cross-term dependence of dvtp0 & -- & 0 \\ \hline
PDVTP1 & Cross-term dependence of dvtp1 & -- & 0 \\ \hline
PDVTP2\newline{\normalfont [Only for versions starting with 4.7]} & Cross-term dependence of dvtp2 & -- & 0 \\ \hline
PDVTP3\newline{\normalfont [Only for versions starting with 4.7]} & Cross-term dependence of dvtp3 & -- & 0 \\ \hline
PDVTP4\newline{\normalfont [Only for versions starting with 4.7]} & Cross-term dependence of dvtp4 & -- & 0 \\ \hline
PDVTP5\newline{\normalfont [Only for versions starting with 4.7]} & Cross-term dependence of dvtp5 & -- & 0 \\ \hline
PDWB & Cross-term dependence of dwb & -- & 0 \\ \hline
PDWG & Cross-term dependence of dwg & -- & 0 \\ \hline
PEGIDL & Cross-term dependence of egidl & -- & 0 \\ \hline
PEGISL & Cross-term dependence of egisl & -- & 0 \\ \hline
PEIGBINV & Cross-term dependence for eigbinv & -- & 0 \\ \hline
PETA0 & Cross-term dependence of eta0 & -- & 0 \\ \hline
PETAB & Cross-term dependence of etab & -- & 0 \\ \hline
PEU & Cross-term dependence of eu & -- & 0 \\ \hline
PFGIDL\newline{\normalfont [Only for versions starting with 4.7]} & Cross-term dependence of fgidl & -- & 0 \\ \hline
PFGISL\newline{\normalfont [Only for versions starting with 4.7]} & Cross-term dependence of fgisl & -- & 0 \\ \hline
PFPROUT & Cross-term dependence of pdiblcb & -- & 0 \\ \hline
PGAMMA1 & Cross-term dependence of gamma1 & -- & 0 \\ \hline
PGAMMA2 & Cross-term dependence of gamma2 & -- & 0 \\ \hline
PHIG & Work Function of gate & -- & 4.05 \\ \hline
PK1 & Cross-term dependence of k1 & -- & 0 \\ \hline
PK2 & Cross-term dependence of k2 & -- & 0 \\ \hline
PK2WE &  Cross-term dependence of k2we  & -- & 0 \\ \hline
PK3 & Cross-term dependence of k3 & -- & 0 \\ \hline
PK3B & Cross-term dependence of k3b & -- & 0 \\ \hline
PKETA & Cross-term dependence of keta & -- & 0 \\ \hline
PKGIDL\newline{\normalfont [Only for versions starting with 4.7]} & Cross-term dependence of kgidl & -- & 0 \\ \hline
PKGISL\newline{\normalfont [Only for versions starting with 4.7]} & Cross-term dependence of kgisl & -- & 0 \\ \hline
PKT1 & Cross-term dependence of kt1 & -- & 0 \\ \hline
PKT1L & Cross-term dependence of kt1l & -- & 0 \\ \hline
PKT2 & Cross-term dependence of kt2 & -- & 0 \\ \hline
PKU0 & Cross-term dependence of ku0 & -- & 0 \\ \hline
PKU0WE &  Cross-term dependence of ku0we  & -- & 0 \\ \hline
PKVTH0 & Cross-term dependence of kvth0 & -- & 0 \\ \hline
PKVTH0WE & Cross-term dependence of kvth0we & -- & 0 \\ \hline
PLAMBDA & Cross-term dependence of lambda & -- & 0 \\ \hline
PLP & Cross-term dependence of lp & -- & 0 \\ \hline
PLPE0 & Cross-term dependence of lpe0 & -- & 0 \\ \hline
PLPEB & Cross-term dependence of lpeb & -- & 0 \\ \hline
PMINV & Cross-term dependence of minv & -- & 0 \\ \hline
PMINVCV & Cross-term dependence of minvcv & -- & 0 \\ \hline
PMOIN & Cross-term dependence of moin & -- & 0 \\ \hline
PNDEP & Cross-term dependence of ndep & -- & 0 \\ \hline
PNFACTOR & Cross-term dependence of nfactor & -- & 0 \\ \hline
PNGATE & Cross-term dependence of ngate & -- & 0 \\ \hline
PNIGBACC & Cross-term dependence of nigbacc & -- & 0 \\ \hline
PNIGBINV & Cross-term dependence of nigbinv & -- & 0 \\ \hline
PNIGC & Cross-term dependence of nigc & -- & 0 \\ \hline
PNOFF & Cross-term dependence of noff & -- & 0 \\ \hline
PNSD & Cross-term dependence of nsd & -- & 0 \\ \hline
PNSUB & Cross-term dependence of nsub & -- & 0 \\ \hline
PNTOX & Cross-term dependence of ntox & -- & 0 \\ \hline
PPCLM & Cross-term dependence of pclm & -- & 0 \\ \hline
PPDIBLC1 & Cross-term dependence of pdiblc1 & -- & 0 \\ \hline
PPDIBLC2 & Cross-term dependence of pdiblc2 & -- & 0 \\ \hline
PPDIBLCB & Cross-term dependence of pdiblcb & -- & 0 \\ \hline
PPDITS & Cross-term dependence of pdits & -- & 0 \\ \hline
PPDITSD & Cross-term dependence of pditsd & -- & 0 \\ \hline
PPHIN & Cross-term dependence of phin & -- & 0 \\ \hline
PPIGCD & Cross-term dependence for pigcd & -- & 0 \\ \hline
PPOXEDGE & Cross-term dependence for poxedge & -- & 0 \\ \hline
PPRT & Cross-term dependence of prt  & -- & 0 \\ \hline
PPRWB & Cross-term dependence of prwb  & -- & 0 \\ \hline
PPRWG & Cross-term dependence of prwg  & -- & 0 \\ \hline
PPSCBE1 & Cross-term dependence of pscbe1 & -- & 0 \\ \hline
PPSCBE2 & Cross-term dependence of pscbe2 & -- & 0 \\ \hline
PPVAG & Cross-term dependence of pvag & -- & 0 \\ \hline
PRDSW & Cross-term dependence of rdsw  & -- & 0 \\ \hline
PRDW & Cross-term dependence of rdw & -- & 0 \\ \hline
PRGIDL\newline{\normalfont [Only for versions starting with 4.7]} & Cross-term dependence of rgidl & -- & 0 \\ \hline
PRGISL\newline{\normalfont [Only for versions starting with 4.7]} & Cross-term dependence of rgisl & -- & 0 \\ \hline
PRSW & Cross-term dependence of rsw & -- & 0 \\ \hline
PRT & Temperature coefficient of parasitic resistance  & -- & 0 \\ \hline
PTETA0\newline{\normalfont [Only for versions starting with 4.7]} & Cross-term dependence of teta0 & -- & 0 \\ \hline
PTNFACTOR\newline{\normalfont [Only for versions starting with 4.7]} & Cross-term dependence of tnfactor & -- & 0 \\ \hline
PTVFBSDOFF & Cross-term dependence of tvfbsdoff & -- & 0 \\ \hline
PTVOFF & Cross-term dependence of tvoff & -- & 0 \\ \hline
PTVOFFCV\newline{\normalfont [Only for versions starting with 4.7]} & Cross-term dependence of tvoffcv & -- & 0 \\ \hline
PU0 & Cross-term dependence of u0 & -- & 0 \\ \hline
PUA & Cross-term dependence of ua & -- & 0 \\ \hline
PUA1 & Cross-term dependence of ua1 & -- & 0 \\ \hline
PUB & Cross-term dependence of ub & -- & 0 \\ \hline
PUB1 & Cross-term dependence of ub1 & -- & 0 \\ \hline
PUC & Cross-term dependence of uc & -- & 0 \\ \hline
PUC1 & Cross-term dependence of uc1 & -- & 0 \\ \hline
PUCS\newline{\normalfont [Only for versions starting with 4.7]} & Cross-term dependence of ucs & -- & 0 \\ \hline
PUCSTE\newline{\normalfont [Only for versions starting with 4.7]} & Cross-term dependence of ucste & -- & 0 \\ \hline
PUD & Cross-term dependence of ud & -- & 0 \\ \hline
PUD1 & Cross-term dependence of ud1 & -- & 0 \\ \hline
PUP & Cross-term dependence of up & -- & 0 \\ \hline
PUTE & Cross-term dependence of ute & -- & 0 \\ \hline
PVAG & Gate dependence of output resistance parameter & -- & 0 \\ \hline
PVBM & Cross-term dependence of vbm & -- & 0 \\ \hline
PVBX & Cross-term dependence of vbx & -- & 0 \\ \hline
PVFB & Cross-term dependence of vfb & -- & 0 \\ \hline
PVFBCV & Cross-term dependence of vfbcv & -- & 0 \\ \hline
PVFBSDOFF & Cross-term dependence of vfbsdoff & -- & 0 \\ \hline
PVOFF & Cross-term dependence of voff & -- & 0 \\ \hline
PVOFFCV & Cross-term dependence of voffcv & -- & 0 \\ \hline
PVSAT & Cross-term dependence of vsat & -- & 0 \\ \hline
PVTH0 &  & -- & 0 \\ \hline
PVTL & Cross-term dependence of vtl & -- & 0 \\ \hline
PW0 & Cross-term dependence of w0 & -- & 0 \\ \hline
PWR & Cross-term dependence of wr & -- & 0 \\ \hline
PXJ & Cross-term dependence of xj & -- & 0 \\ \hline
PXN & Cross-term dependence of xn & -- & 0 \\ \hline
PXRCRG1 & Cross-term dependence of xrcrg1 & -- & 0 \\ \hline
PXRCRG2 & Cross-term dependence of xrcrg2 & -- & 0 \\ \hline
PXT & Cross-term dependence of xt & -- & 0 \\ \hline
RBDB & Resistance between bNode and dbNode & $\mathsf{\Omega}$ & 50 \\ \hline
RBDBX0 & Body resistance RBDBX  scaling & -- & 100 \\ \hline
RBDBY0 & Body resistance RBDBY  scaling & -- & 100 \\ \hline
RBPB & Resistance between bNodePrime and bNode & $\mathsf{\Omega}$ & 50 \\ \hline
RBPBX0 & Body resistance RBPBX  scaling & -- & 100 \\ \hline
RBPBXL & Body resistance RBPBX L scaling & -- & 0 \\ \hline
RBPBXNF & Body resistance RBPBX NF scaling & -- & 0 \\ \hline
RBPBXW & Body resistance RBPBX W scaling & -- & 0 \\ \hline
RBPBY0 & Body resistance RBPBY  scaling & -- & 100 \\ \hline
RBPBYL & Body resistance RBPBY L scaling & -- & 0 \\ \hline
RBPBYNF & Body resistance RBPBY NF scaling & -- & 0 \\ \hline
RBPBYW & Body resistance RBPBY W scaling & -- & 0 \\ \hline
RBPD & Resistance between bNodePrime and bNode & $\mathsf{\Omega}$ & 50 \\ \hline
RBPD0 & Body resistance RBPD scaling & -- & 50 \\ \hline
RBPDL & Body resistance RBPD L scaling & -- & 0 \\ \hline
RBPDNF & Body resistance RBPD NF scaling & -- & 0 \\ \hline
RBPDW & Body resistance RBPD W scaling & -- & 0 \\ \hline
RBPS & Resistance between bNodePrime and sbNode & $\mathsf{\Omega}$ & 50 \\ \hline
RBPS0 & Body resistance RBPS scaling & -- & 50 \\ \hline
RBPSL & Body resistance RBPS L scaling & -- & 0 \\ \hline
RBPSNF & Body resistance RBPS NF scaling & -- & 0 \\ \hline
RBPSW & Body resistance RBPS W scaling & -- & 0 \\ \hline
RBSB & Resistance between bNode and sbNode & $\mathsf{\Omega}$ & 50 \\ \hline
RBSBX0 & Body resistance RBSBX  scaling & -- & 100 \\ \hline
RBSBY0 & Body resistance RBSBY  scaling & -- & 100 \\ \hline
RBSDBXL & Body resistance RBSDBX L scaling & -- & 0 \\ \hline
RBSDBXNF & Body resistance RBSDBX NF scaling & -- & 0 \\ \hline
RBSDBXW & Body resistance RBSDBX W scaling & -- & 0 \\ \hline
RBSDBYL & Body resistance RBSDBY L scaling & -- & 0 \\ \hline
RBSDBYNF & Body resistance RBSDBY NF scaling & -- & 0 \\ \hline
RBSDBYW & Body resistance RBSDBY W scaling & -- & 0 \\ \hline
RNOIA & Thermal noise coefficient & -- & 0.577 \\ \hline
RNOIB & Thermal noise coefficient & -- & 0.5164 \\ \hline
RNOIC\newline{\normalfont [Only for versions starting with 4.7]} & Thermal noise coefficient & -- & 0.395 \\ \hline
SAREF & Reference distance between OD edge to poly of one side & -- & 1e-06 \\ \hline
SBREF & Reference distance between OD edge to poly of the other side & -- & 1e-06 \\ \hline
SCREF &  Reference distance to calculate SCA,SCB and SCC & -- & 1e-06 \\ \hline
STETA0 & eta0 shift factor related to stress effect on vth & -- & 0 \\ \hline
STK2 & K2 shift factor related to stress effect on vth & -- & 0 \\ \hline
TCJ & Temperature coefficient of cj & -- & 0 \\ \hline
TCJSW & Temperature coefficient of cjsw & -- & 0 \\ \hline
TCJSWG & Temperature coefficient of cjswg & -- & 0 \\ \hline
TETA0\newline{\normalfont [Only for versions starting with 4.7]} & Temperature parameter for eta0 & -- & 0 \\ \hline
TKU0 & Temperature coefficient of KU0 & -- & 0 \\ \hline
TNFACTOR\newline{\normalfont [Only for versions starting with 4.7]} & Temperature parameter for nfactor & -- & 0 \\ \hline
TNJTS & Temperature coefficient for NJTS & -- & 0 \\ \hline
TNJTSD & Temperature coefficient for NJTSD & -- & 0 \\ \hline
TNJTSSW & Temperature coefficient for NJTSSW & -- & 0 \\ \hline
TNJTSSWD & Temperature coefficient for NJTSSWD & -- & 0 \\ \hline
TNJTSSWG & Temperature coefficient for NJTSSWG & -- & 0 \\ \hline
TNJTSSWGD & Temperature coefficient for NJTSSWGD & -- & 0 \\ \hline
TNOIA & Thermal noise parameter & -- & 1.5 \\ \hline
TNOIB & Thermal noise parameter & -- & 3.5 \\ \hline
TNOIC\newline{\normalfont [Only for versions starting with 4.7]} & Thermal noise parameter & -- & 0 \\ \hline
TNOM & Parameter measurement temperature & -- & Ambient Temperature \\ \hline
TPB & Temperature coefficient of pb & -- & 0 \\ \hline
TPBSW & Temperature coefficient of pbsw & -- & 0 \\ \hline
TPBSWG & Temperature coefficient of pbswg & -- & 0 \\ \hline
TVFBSDOFF & Temperature parameter for vfbsdoff & -- & 0 \\ \hline
TVOFF & Temperature parameter for voff & -- & 0 \\ \hline
TVOFFCV\newline{\normalfont [Only for versions starting with 4.7]} & Temperature parameter for tvoffcv & -- & 0 \\ \hline
UA1 & Temperature coefficient of ua & -- & 1e-09 \\ \hline
UB1 & Temperature coefficient of ub & -- & -1e-18 \\ \hline
UC1 & Temperature coefficient of uc & -- & 0 \\ \hline
UCSTE\newline{\normalfont [Only for versions starting with 4.7]} & Temperature coefficient of colombic mobility & -- & -0.004775 \\ \hline
UD1 & Temperature coefficient of ud & -- & 0 \\ \hline
UTE & Temperature coefficient of mobility & -- & -1.5 \\ \hline
VTSD & Drain bottom trap-assisted voltage dependent parameter & -- & 10 \\ \hline
VTSS & Source bottom trap-assisted voltage dependent parameter & -- & 10 \\ \hline
VTSSWD & Drain STI sidewall trap-assisted voltage dependent parameter & -- & 10 \\ \hline
VTSSWGD & Drain gate-edge sidewall trap-assisted voltage dependent parameter & -- & 10 \\ \hline
VTSSWGS & Source gate-edge sidewall trap-assisted voltage dependent parameter & -- & 10 \\ \hline
VTSSWS & Source STI sidewall trap-assisted voltage dependent parameter & -- & 10 \\ \hline
WA0 & Width dependence of a0 & -- & 0 \\ \hline
WA1 & Width dependence of a1 & -- & 0 \\ \hline
WA2 & Width dependence of a2 & -- & 0 \\ \hline
WACDE & Width dependence of acde & -- & 0 \\ \hline
WAGIDL & Width dependence of agidl & -- & 0 \\ \hline
WAGISL & Width dependence of agisl & -- & 0 \\ \hline
WAGS & Width dependence of ags & -- & 0 \\ \hline
WAIGBACC & Width dependence of aigbacc & -- & 0 \\ \hline
WAIGBINV & Width dependence of aigbinv & -- & 0 \\ \hline
WAIGC & Width dependence of aigc & -- & 0 \\ \hline
WAIGD & Width dependence of aigd & -- & 0 \\ \hline
WAIGS & Width dependence of aigs & -- & 0 \\ \hline
WAIGSD & Width dependence of aigsd & -- & 0 \\ \hline
WALPHA0 & Width dependence of alpha0 & -- & 0 \\ \hline
WALPHA1 & Width dependence of alpha1 & -- & 0 \\ \hline
WAT & Width dependence of at & -- & 0 \\ \hline
WB0 & Width dependence of b0 & -- & 0 \\ \hline
WB1 & Width dependence of b1 & -- & 0 \\ \hline
WBETA0 & Width dependence of beta0 & -- & 0 \\ \hline
WBGIDL & Width dependence of bgidl & -- & 0 \\ \hline
WBGISL & Width dependence of bgisl & -- & 0 \\ \hline
WBIGBACC & Width dependence of bigbacc & -- & 0 \\ \hline
WBIGBINV & Width dependence of bigbinv & -- & 0 \\ \hline
WBIGC & Width dependence of bigc & -- & 0 \\ \hline
WBIGD & Width dependence of bigd & -- & 0 \\ \hline
WBIGS & Width dependence of bigs & -- & 0 \\ \hline
WBIGSD & Width dependence of bigsd & -- & 0 \\ \hline
WCDSC & Width dependence of cdsc & -- & 0 \\ \hline
WCDSCB & Width dependence of cdscb & -- & 0 \\ \hline
WCDSCD & Width dependence of cdscd & -- & 0 \\ \hline
WCF & Width dependence of cf & -- & 0 \\ \hline
WCGDL & Width dependence of cgdl & -- & 0 \\ \hline
WCGIDL & Width dependence of cgidl & -- & 0 \\ \hline
WCGISL & Width dependence of cgisl & -- & 0 \\ \hline
WCGSL & Width dependence of cgsl & -- & 0 \\ \hline
WCIGBACC & Width dependence of cigbacc & -- & 0 \\ \hline
WCIGBINV & Width dependence of cigbinv & -- & 0 \\ \hline
WCIGC & Width dependence of cigc & -- & 0 \\ \hline
WCIGD & Width dependence of cigd & -- & 0 \\ \hline
WCIGS & Width dependence of cigs & -- & 0 \\ \hline
WCIGSD & Width dependence of cigsd & -- & 0 \\ \hline
WCIT & Width dependence of cit & -- & 0 \\ \hline
WCKAPPAD & Width dependence of ckappad & -- & 0 \\ \hline
WCKAPPAS & Width dependence of ckappas & -- & 0 \\ \hline
WCLC & Width dependence of clc & -- & 0 \\ \hline
WCLE & Width dependence of cle & -- & 0 \\ \hline
WDELTA & Width dependence of delta & -- & 0 \\ \hline
WDROUT & Width dependence of drout & -- & 0 \\ \hline
WDSUB & Width dependence of dsub & -- & 0 \\ \hline
WDVT0 & Width dependence of dvt0 & -- & 0 \\ \hline
WDVT0W & Width dependence of dvt0w & -- & 0 \\ \hline
WDVT1 & Width dependence of dvt1 & -- & 0 \\ \hline
WDVT1W & Width dependence of dvt1w & -- & 0 \\ \hline
WDVT2 & Width dependence of dvt2 & -- & 0 \\ \hline
WDVT2W & Width dependence of dvt2w & -- & 0 \\ \hline
WDVTP0 & Width dependence of dvtp0 & -- & 0 \\ \hline
WDVTP1 & Width dependence of dvtp1 & -- & 0 \\ \hline
WDVTP2\newline{\normalfont [Only for versions starting with 4.7]} & Width dependence of dvtp2 & -- & 0 \\ \hline
WDVTP3\newline{\normalfont [Only for versions starting with 4.7]} & Width dependence of dvtp3 & -- & 0 \\ \hline
WDVTP4\newline{\normalfont [Only for versions starting with 4.7]} & Width dependence of dvtp4 & -- & 0 \\ \hline
WDVTP5\newline{\normalfont [Only for versions starting with 4.7]} & Width dependence of dvtp5 & -- & 0 \\ \hline
WDWB & Width dependence of dwb & -- & 0 \\ \hline
WDWG & Width dependence of dwg & -- & 0 \\ \hline
WEB & Coefficient for SCB & -- & 0 \\ \hline
WEC & Coefficient for SCC & -- & 0 \\ \hline
WEGIDL & Width dependence of egidl & -- & 0 \\ \hline
WEGISL & Width dependence of egisl & -- & 0 \\ \hline
WEIGBINV & Width dependence for eigbinv & -- & 0 \\ \hline
WETA0 & Width dependence of eta0 & -- & 0 \\ \hline
WETAB & Width dependence of etab & -- & 0 \\ \hline
WEU & Width dependence of eu & -- & 0 \\ \hline
WFGIDL\newline{\normalfont [Only for versions starting with 4.7]} & Width dependence of fgidl & -- & 0 \\ \hline
WFGISL\newline{\normalfont [Only for versions starting with 4.7]} & Width dependence of fgisl & -- & 0 \\ \hline
WFPROUT & Width dependence of pdiblcb & -- & 0 \\ \hline
WGAMMA1 & Width dependence of gamma1 & -- & 0 \\ \hline
WGAMMA2 & Width dependence of gamma2 & -- & 0 \\ \hline
WK1 & Width dependence of k1 & -- & 0 \\ \hline
WK2 & Width dependence of k2 & -- & 0 \\ \hline
WK2WE &  Width dependence of k2we  & -- & 0 \\ \hline
WK3 & Width dependence of k3 & -- & 0 \\ \hline
WK3B & Width dependence of k3b & -- & 0 \\ \hline
WKETA & Width dependence of keta & -- & 0 \\ \hline
WKGIDL\newline{\normalfont [Only for versions starting with 4.7]} & Width dependence of kgidl & -- & 0 \\ \hline
WKGISL\newline{\normalfont [Only for versions starting with 4.7]} & Width dependence of kgisl & -- & 0 \\ \hline
WKT1 & Width dependence of kt1 & -- & 0 \\ \hline
WKT1L & Width dependence of kt1l & -- & 0 \\ \hline
WKT2 & Width dependence of kt2 & -- & 0 \\ \hline
WKU0 & Width dependence of ku0 & -- & 0 \\ \hline
WKU0WE &  Width dependence of ku0we  & -- & 0 \\ \hline
WKVTH0 & Width dependence of kvth0 & -- & 0 \\ \hline
WKVTH0WE & Width dependence of kvth0we & -- & 0 \\ \hline
WL & Width reduction parameter & -- & 0 \\ \hline
WLAMBDA & Width dependence of lambda & -- & 0 \\ \hline
WLC & Width reduction parameter for CV & -- & 0 \\ \hline
WLN & Width reduction parameter & -- & 1 \\ \hline
WLOD & Width parameter for stress effect & -- & 0 \\ \hline
WLODKU0 & Width parameter for u0 LOD effect & -- & 0 \\ \hline
WLODVTH & Width parameter for vth LOD effect & -- & 0 \\ \hline
WLP & Width dependence of lp & -- & 0 \\ \hline
WLPE0 & Width dependence of lpe0 & -- & 0 \\ \hline
WLPEB & Width dependence of lpeb & -- & 0 \\ \hline
WMAX & Maximum width for the model & -- & 1 \\ \hline
WMIN & Minimum width for the model & -- & 0 \\ \hline
WMINV & Width dependence of minv & -- & 0 \\ \hline
WMINVCV & Width dependence of minvcv & -- & 0 \\ \hline
WMOIN & Width dependence of moin & -- & 0 \\ \hline
WNDEP & Width dependence of ndep & -- & 0 \\ \hline
WNFACTOR & Width dependence of nfactor & -- & 0 \\ \hline
WNGATE & Width dependence of ngate & -- & 0 \\ \hline
WNIGBACC & Width dependence of nigbacc & -- & 0 \\ \hline
WNIGBINV & Width dependence of nigbinv & -- & 0 \\ \hline
WNIGC & Width dependence of nigc & -- & 0 \\ \hline
WNOFF & Width dependence of noff & -- & 0 \\ \hline
WNSD & Width dependence of nsd & -- & 0 \\ \hline
WNSUB & Width dependence of nsub & -- & 0 \\ \hline
WNTOX & Width dependence of ntox & -- & 0 \\ \hline
WPCLM & Width dependence of pclm & -- & 0 \\ \hline
WPDIBLC1 & Width dependence of pdiblc1 & -- & 0 \\ \hline
WPDIBLC2 & Width dependence of pdiblc2 & -- & 0 \\ \hline
WPDIBLCB & Width dependence of pdiblcb & -- & 0 \\ \hline
WPDITS & Width dependence of pdits & -- & 0 \\ \hline
WPDITSD & Width dependence of pditsd & -- & 0 \\ \hline
WPEMOD &  Flag for WPE model (WPEMOD=1 to activate this model)  & -- & 0 \\ \hline
WPHIN & Width dependence of phin & -- & 0 \\ \hline
WPIGCD & Width dependence for pigcd & -- & 0 \\ \hline
WPOXEDGE & Width dependence for poxedge & -- & 0 \\ \hline
WPRT & Width dependence of prt & -- & 0 \\ \hline
WPRWB & Width dependence of prwb  & -- & 0 \\ \hline
WPRWG & Width dependence of prwg  & -- & 0 \\ \hline
WPSCBE1 & Width dependence of pscbe1 & -- & 0 \\ \hline
WPSCBE2 & Width dependence of pscbe2 & -- & 0 \\ \hline
WPVAG & Width dependence of pvag & -- & 0 \\ \hline
WRDSW & Width dependence of rdsw  & -- & 0 \\ \hline
WRDW & Width dependence of rdw & -- & 0 \\ \hline
WRGIDL\newline{\normalfont [Only for versions starting with 4.7]} & Width dependence of rgidl & -- & 0 \\ \hline
WRGISL\newline{\normalfont [Only for versions starting with 4.7]} & Width dependence of rgisl & -- & 0 \\ \hline
WRSW & Width dependence of rsw & -- & 0 \\ \hline
WTETA0\newline{\normalfont [Only for versions starting with 4.7]} & Width dependence of teta0 & -- & 0 \\ \hline
WTNFACTOR\newline{\normalfont [Only for versions starting with 4.7]} & Width dependence of tnfactor & -- & 0 \\ \hline
WTVFBSDOFF & Width dependence of tvfbsdoff & -- & 0 \\ \hline
WTVOFF & Width dependence of tvoff & -- & 0 \\ \hline
WTVOFFCV\newline{\normalfont [Only for versions starting with 4.7]} & Width dependence of tvoffcv & -- & 0 \\ \hline
WU0 & Width dependence of u0 & -- & 0 \\ \hline
WUA & Width dependence of ua & -- & 0 \\ \hline
WUA1 & Width dependence of ua1 & -- & 0 \\ \hline
WUB & Width dependence of ub & -- & 0 \\ \hline
WUB1 & Width dependence of ub1 & -- & 0 \\ \hline
WUC & Width dependence of uc & -- & 0 \\ \hline
WUC1 & Width dependence of uc1 & -- & 0 \\ \hline
WUCS\newline{\normalfont [Only for versions starting with 4.7]} & Width dependence of ucs & -- & 0 \\ \hline
WUCSTE\newline{\normalfont [Only for versions starting with 4.7]} & Width dependence of ucste & -- & 0 \\ \hline
WUD & Width dependence of ud & -- & 0 \\ \hline
WUD1 & Width dependence of ud1 & -- & 0 \\ \hline
WUP & Width dependence of up & -- & 0 \\ \hline
WUTE & Width dependence of ute & -- & 0 \\ \hline
WVBM & Width dependence of vbm & -- & 0 \\ \hline
WVBX & Width dependence of vbx & -- & 0 \\ \hline
WVFB & Width dependence of vfb & -- & 0 \\ \hline
WVFBCV & Width dependence of vfbcv & -- & 0 \\ \hline
WVFBSDOFF & Width dependence of vfbsdoff & -- & 0 \\ \hline
WVOFF & Width dependence of voff & -- & 0 \\ \hline
WVOFFCV & Width dependence of voffcv & -- & 0 \\ \hline
WVSAT & Width dependence of vsat & -- & 0 \\ \hline
WVTH0 &  & -- & 0 \\ \hline
WVTL & Width dependence of vtl & -- & 0 \\ \hline
WW & Width reduction parameter & -- & 0 \\ \hline
WW0 & Width dependence of w0 & -- & 0 \\ \hline
WWC & Width reduction parameter for CV & -- & 0 \\ \hline
WWL & Width reduction parameter & -- & 0 \\ \hline
WWLC & Width reduction parameter for CV & -- & 0 \\ \hline
WWN & Width reduction parameter & -- & 1 \\ \hline
WWR & Width dependence of wr & -- & 0 \\ \hline
WXJ & Width dependence of xj & -- & 0 \\ \hline
WXN & Width dependence of xn & -- & 0 \\ \hline
WXRCRG1 & Width dependence of xrcrg1 & -- & 0 \\ \hline
WXRCRG2 & Width dependence of xrcrg2 & -- & 0 \\ \hline
WXT & Width dependence of xt & -- & 0 \\ \hline
XGL & Variation in Ldrawn & -- & 0 \\ \hline
XGW & Distance from gate contact center to device edge & -- & 0 \\ \hline
XJBVD & Fitting parameter for drain diode breakdown current & -- & 1 \\ \hline
XJBVS & Fitting parameter for source diode breakdown current & -- & 1 \\ \hline
XL & L offset for channel length due to mask/etch effect & -- & 0 \\ \hline
XRCRG1 & First fitting parameter the bias-dependent Rg & -- & 12 \\ \hline
XRCRG2 & Second fitting parameter the bias-dependent Rg & -- & 1 \\ \hline
XTID & Drainjunction current temperature exponent & -- & 3 \\ \hline
XTIS & Source junction current temperature exponent & -- & 3 \\ \hline
XTSD & Power dependence of JTSD on temperature & -- & 0.02 \\ \hline
XTSS & Power dependence of JTSS on temperature & -- & 0.02 \\ \hline
XTSSWD & Power dependence of JTSSWD on temperature & -- & 0.02 \\ \hline
XTSSWGD & Power dependence of JTSSWGD on temperature & -- & 0.02 \\ \hline
XTSSWGS & Power dependence of JTSSWGS on temperature & -- & 0.02 \\ \hline
XTSSWS & Power dependence of JTSSWS on temperature & -- & 0.02 \\ \hline
XW & W offset for channel width due to mask/etch effect & -- & 0 \\ \hline

\category{Basic Parameters}\\ \hline
A0 & Non-uniform depletion width effect coefficient. & -- & 1 \\ \hline
A1 & Non-saturation effect coefficient & V$^{-1}$ & 0 \\ \hline
A2 & Non-saturation effect coefficient & -- & 1 \\ \hline
ADOS & Charge centroid parameter & -- & 1 \\ \hline
AGS & Gate bias  coefficient of Abulk. & V$^{-1}$ & 0 \\ \hline
B0 & Abulk narrow width parameter & m & 0 \\ \hline
B1 & Abulk narrow width parameter & m & 0 \\ \hline
BDOS & Charge centroid parameter & -- & 1 \\ \hline
BG0SUB & Band-gap of substrate at T=0K & eV & 1.16 \\ \hline
CDSC & Drain/Source and channel coupling capacitance & F/m$^{2}$ & 0.00024 \\ \hline
CDSCB & Body-bias dependence of cdsc & F/(Vm$^{2}$) & 0 \\ \hline
CDSCD & Drain-bias dependence of cdsc & F/(Vm$^{2}$) & 0 \\ \hline
CIT & Interface state capacitance & F/m$^{2}$ & 0 \\ \hline
DELTA & Effective Vds parameter & V & 0.01 \\ \hline
DROUT & DIBL coefficient of output resistance & -- & 0.56 \\ \hline
DSUB & DIBL coefficient in the subthreshold region & -- & 0 \\ \hline
DVT0 & Short channel effect coeff. 0 & -- & 2.2 \\ \hline
DVT0W & Narrow Width coeff. 0 & -- & 0 \\ \hline
DVT1 & Short channel effect coeff. 1 & -- & 0.53 \\ \hline
DVT1W & Narrow Width effect coeff. 1 & m$^{-1}$ & 5.3e+06 \\ \hline
DVT2 & Short channel effect coeff. 2 & V$^{-1}$ & -0.032 \\ \hline
DVT2W & Narrow Width effect coeff. 2 & V$^{-1}$ & -0.032 \\ \hline
DVTP0 & First parameter for Vth shift due to pocket & m & 0 \\ \hline
DVTP1 & Second parameter for Vth shift due to pocket & V$^{-1}$ & 0 \\ \hline
DVTP2\newline{\normalfont [Only for versions starting with 4.7]} & 3rd parameter for Vth shift due to pocket & Vm$^{X}$ & 0 \\ \hline
DVTP3\newline{\normalfont [Only for versions starting with 4.7]} & 4th parameter for Vth shift due to pocket & -- & 0 \\ \hline
DVTP4\newline{\normalfont [Only for versions starting with 4.7]} & 5th parameter for Vth shift due to pocket & V$^{-1}$ & 0 \\ \hline
DVTP5\newline{\normalfont [Only for versions starting with 4.7]} & 6th parameter for Vth shift due to pocket & V & 0 \\ \hline
DWB & Width reduction parameter & m/V$^{1/2}$ & 0 \\ \hline
DWG & Width reduction parameter & m/V & 0 \\ \hline
EASUB & Electron affinity of substrate & V & 4.05 \\ \hline
EPSRSUB & Dielectric constant of substrate relative to vacuum & -- & 11.7 \\ \hline
ETA0 & Subthreshold region DIBL coefficient & -- & 0.08 \\ \hline
ETAB & Subthreshold region DIBL coefficient & V$^{-1}$ & -0.07 \\ \hline
EU & Mobility exponent & -- & 0 \\ \hline
FPROUT & Rout degradation coefficient for pocket devices & V/m$^{1/2}$ & 0 \\ \hline
K1 & Bulk effect coefficient 1 & V$^{-1/2}$ & 0 \\ \hline
K2 & Bulk effect coefficient 2 & -- & 0 \\ \hline
K3 & Narrow width effect coefficient & -- & 80 \\ \hline
KETA & Body-bias coefficient of non-uniform depletion width effect. & V$^{-1}$ & -0.047 \\ \hline
LAMBDA &  Velocity overshoot parameter & -- & 0 \\ \hline
LC &  back scattering parameter & m & 5e-09 \\ \hline
LEFFEOT\newline{\normalfont [Only for versions starting with 4.7]} & Effective length for extraction of EOT & m & 1e-06 \\ \hline
LINT & Length reduction parameter & m & 0 \\ \hline
LP & Channel length exponential factor of mobility & m & 1e-08 \\ \hline
LPE0 & Equivalent length of pocket region at zero bias & m & 1.74e-07 \\ \hline
LPEB & Equivalent length of pocket region accounting for body bias & m & 0 \\ \hline
MINV & Fitting parameter for moderate inversion in Vgsteff & -- & 0 \\ \hline
NFACTOR & Subthreshold swing Coefficient & -- & 1 \\ \hline
NI0SUB & Intrinsic carrier concentration of substrate at 300.15K & cm$^{-3}$ & 1.45e+10 \\ \hline
PCLM & Channel length modulation Coefficient & -- & 1.3 \\ \hline
PDIBLC1 & Drain-induced barrier lowering coefficient & -- & 0.39 \\ \hline
PDIBLC2 & Drain-induced barrier lowering coefficient & -- & 0.0086 \\ \hline
PDIBLCB & Body-effect on drain-induced barrier lowering & V$^{-1}$ & 0 \\ \hline
PDITS & Coefficient for drain-induced Vth shifts & V$^{-1}$ & 0 \\ \hline
PDITSD & Vds dependence of drain-induced Vth shifts & V$^{-1}$ & 0 \\ \hline
PDITSL & Length dependence of drain-induced Vth shifts & m$^{-1}$ & 0 \\ \hline
PHIN & Adjusting parameter for surface potential due to non-uniform vertical doping & V & 0 \\ \hline
PSCBE1 & Substrate current body-effect coefficient & Vm$^{-1}$ & 4.24e+08 \\ \hline
PSCBE2 & Substrate current body-effect coefficient & m/V & 1e-05 \\ \hline
TBGASUB & First parameter of band-gap change due to temperature & eV/K & 0.000702 \\ \hline
TBGBSUB & Second parameter of band-gap change due to temperature & K & 1108 \\ \hline
TEMPEOT\newline{\normalfont [Only for versions starting with 4.7]} & Temperature for extraction of EOT & -- & 300.15 \\ \hline
U0 & Low-field mobility at Tnom & m$^{2}$/(Vs) & 0 \\ \hline
UA & Linear gate dependence of mobility & m/V & 0 \\ \hline
UB & Quadratic gate dependence of mobility & m$^{2}$/V$^{2}$ & 1e-19 \\ \hline
UC & Body-bias dependence of mobility & V$^{-1}$ & 0 \\ \hline
UCS\newline{\normalfont [Only for versions starting with 4.7]} & Colombic scattering exponent & -- & 1.67 \\ \hline
UD & Coulomb scattering factor of mobility & m$^{-2}$ & 0 \\ \hline
UP & Channel length linear factor of mobility & m$^{-2}$ & 0 \\ \hline
VBM & Maximum body voltage & V & -3 \\ \hline
VDDEOT & Voltage for extraction of equivalent gate oxide thickness & V & 1.5 \\ \hline
VFB & Flat Band Voltage & V & -1 \\ \hline
VOFF & Threshold voltage offset & V & -0.08 \\ \hline
VOFFL & Length dependence parameter for Vth offset & V & 0 \\ \hline
VSAT & Saturation velocity at tnom & m/s & 80000 \\ \hline
VTH0 &  & V & 0 \\ \hline
VTL &  thermal velocity & m/s & 200000 \\ \hline
W0 & Narrow width effect parameter & m & 2.5e-06 \\ \hline
WEFFEOT\newline{\normalfont [Only for versions starting with 4.7]} & Effective width for extraction of EOT & m & 1e-05 \\ \hline
WINT & Width reduction parameter & m & 0 \\ \hline
XN &  back scattering parameter & -- & 3 \\ \hline

\category{Capacitance Parameters}\\ \hline
ACDE & Exponential coefficient for finite charge thickness & m/V & 1 \\ \hline
CF & Fringe capacitance parameter & F/m & 0 \\ \hline
CGBO & Gate-bulk overlap capacitance per length & -- & 0 \\ \hline
CGDL & New C-V model parameter & F/m & 0 \\ \hline
CGDO & Gate-drain overlap capacitance per width & F/m & 0 \\ \hline
CGSL & New C-V model parameter & F/m & 0 \\ \hline
CGSO & Gate-source overlap capacitance per width & F/m & 0 \\ \hline
CKAPPAD & D/G overlap C-V parameter & V & 0.6 \\ \hline
CKAPPAS & S/G overlap C-V parameter  & V & 0.6 \\ \hline
CLC & Vdsat parameter for C-V model & m & 1e-07 \\ \hline
CLE & Vdsat parameter for C-V model & -- & 0.6 \\ \hline
DLC & Delta L for C-V model & m & 0 \\ \hline
DWC & Delta W for C-V model & m & 0 \\ \hline
MINVCV & Fitting parameter for moderate inversion in Vgsteffcv & -- & 0 \\ \hline
MOIN & Coefficient for gate-bias dependent surface potential & -- & 15 \\ \hline
NOFF & C-V turn-on/off parameter & -- & 1 \\ \hline
VFBCV & Flat Band Voltage parameter for capmod=0 only & V & -1 \\ \hline
VOFFCV & C-V lateral-shift parameter & V & 0 \\ \hline
VOFFCVL & Length dependence parameter for Vth offset in CV & -- & 0 \\ \hline
XPART & Channel charge partitioning & F/m & 0 \\ \hline

\category{Control Parameters}\\ \hline
ACNQSMOD & AC NQS model selector & -- & 0 \\ \hline
BINUNIT & Bin  unit  selector & -- & 1 \\ \hline
CAPMOD & Capacitance model selector & -- & 2 \\ \hline
CVCHARGEMOD & Capacitance charge model selector & -- & 0 \\ \hline
DIOMOD & Diode IV model selector & -- & 1 \\ \hline
FNOIMOD & Flicker noise model selector & -- & 1 \\ \hline
GEOMOD & Geometry dependent parasitics model selector & -- & 0 \\ \hline
GIDLMOD\newline{\normalfont [Only for versions starting with 4.7]} & parameter for GIDL selector & -- & 0 \\ \hline
IGBMOD & Gate-to-body Ig model selector & -- & 0 \\ \hline
IGCMOD & Gate-to-channel Ig model selector & -- & 0 \\ \hline
MOBMOD & Mobility model selector & -- & 0 \\ \hline
MTRLCOMPATMOD\newline{\normalfont [Only for versions starting with 4.7]} & New material Mod backward compatibility selector & -- & 0 \\ \hline
MTRLMOD & parameter for nonm-silicon substrate or metal gate selector & -- & 0 \\ \hline
PARAMCHK & Model parameter checking selector & -- & 1 \\ \hline
PERMOD & Pd and Ps model selector & -- & 1 \\ \hline
RBODYMOD & Distributed body R model selector & -- & 0 \\ \hline
RDSMOD & Bias-dependent S/D resistance model selector & -- & 0 \\ \hline
RGATEMOD & Gate R model selector & -- & 0 \\ \hline
RGEOMOD & S/D resistance and contact model selector & -- & 0 \\ \hline
TEMPMOD & Temperature model selector & -- & 0 \\ \hline
TNOIMOD & Thermal noise model selector & -- & 0 \\ \hline
TRNQSMOD & Transient NQS model selector & -- & 0 \\ \hline
VERSION & parameter for model version & -- & '4.6.1' \\ \hline

\category{Flicker and Thermal Noise Parameters}\\ \hline
NOIA & Flicker Noise parameter a & -- & 0 \\ \hline
NOIB & Flicker Noise parameter b & -- & 0 \\ \hline
NOIC & Flicker Noise parameter c & -- & 0 \\ \hline

\category{Process Parameters}\\ \hline
DTOX & Defined as (toxe - toxp)  & m & 0 \\ \hline
EOT & Equivalent gate oxide thickness in meters & m & 1.5e-09 \\ \hline
EPSROX & Dielectric constant of the gate oxide relative to vacuum & -- & 3.9 \\ \hline
GAMMA1 & Vth body coefficient & V$^{1/2}$ & 0 \\ \hline
GAMMA2 & Vth body coefficient & V$^{1/2}$ & 0 \\ \hline
NDEP & Channel doping concentration at the depletion edge & cm$^{-3}$ & 1.7e+17 \\ \hline
NGATE & Poly-gate doping concentration & cm$^{-3}$ & 0 \\ \hline
NSD & S/D doping concentration & cm$^{-3}$ & 1e+20 \\ \hline
NSUB & Substrate doping concentration & cm$^{-3}$ & 6e+16 \\ \hline
RSH & Source-drain sheet resistance & $\mathsf{\Omega}/\Box$ & 0 \\ \hline
RSHG & Gate sheet resistance & $\mathsf{\Omega}/\Box$ & 0.1 \\ \hline
TOXE & Electrical gate oxide thickness in meters & m & 3e-09 \\ \hline
TOXM & Gate oxide thickness at which parameters are extracted & m & 3e-09 \\ \hline
TOXP & Physical gate oxide thickness in meters & m & 3e-09 \\ \hline
VBX & Vth transition body Voltage & V & 0 \\ \hline
XJ & Junction depth in meters & m & 1.5e-07 \\ \hline
XT & Doping depth & m & 1.55e-07 \\ \hline

\category{Tunnelling Parameters}\\ \hline
AIGBACC & Parameter for Igb & (Fs$^2$/g)$^{1/2}$/m & 0.0136 \\ \hline
AIGBINV & Parameter for Igb & (Fs$^2$/g)$^{1/2}$/m & 0.0111 \\ \hline
AIGC & Parameter for Igc & (Fs$^2$/g)$^{1/2}$/m & 0.0136 \\ \hline
AIGD & Parameter for Igd & (Fs$^2$/g)$^{1/2}$/m & 0.0136 \\ \hline
AIGS & Parameter for Igs & (Fs$^2$/g)$^{1/2}$/m & 0.0136 \\ \hline
BIGBACC & Parameter for Igb & (Fs$^2$/g)$^{1/2}$/mV & 0.00171 \\ \hline
BIGBINV & Parameter for Igb & (Fs$^2$/g)$^{1/2}$/mV & 0.000949 \\ \hline
BIGC & Parameter for Igc & (Fs$^2$/g)$^{1/2}$/mV & 0.00171 \\ \hline
BIGD & Parameter for Igd & (Fs$^2$/g)$^{1/2}$/mV & 0.00171 \\ \hline
BIGS & Parameter for Igs & (Fs$^2$/g)$^{1/2}$/mV & 0.00171 \\ \hline
CIGBACC & Parameter for Igb & V$^{-1}$ & 0.075 \\ \hline
CIGBINV & Parameter for Igb & V$^{-1}$ & 0.006 \\ \hline
CIGC & Parameter for Igc & V$^{-1}$ & 0.075 \\ \hline
CIGD & Parameter for Igd & V$^{-1}$ & 0.075 \\ \hline
CIGS & Parameter for Igs & V$^{-1}$ & 0.075 \\ \hline
DLCIGD & Delta L for Ig model drain side & m & 0 \\ \hline
EIGBINV & Parameter for the Si bandgap for Igbinv & V & 1.1 \\ \hline
NIGBACC & Parameter for Igbacc slope & -- & 1 \\ \hline
NIGBINV & Parameter for Igbinv slope & -- & 3 \\ \hline
NIGC & Parameter for Igc slope & -- & 1 \\ \hline
NTOX & Exponent for Tox ratio & -- & 1 \\ \hline
PIGCD & Parameter for Igc partition & -- & 1 \\ \hline
POXEDGE & Factor for the gate edge Tox & -- & 1 \\ \hline
TOXREF & Target tox value & m & 3e-09 \\ \hline
VFBSDOFF & S/D flatband voltage offset & V & 0 \\ \hline

\category{Asymmetric and Bias-Dependent $R_{ds}$ Parameters}\\ \hline
PRWB & Body-effect on parasitic resistance  & V$^{-1}$ & 0 \\ \hline
PRWG & Gate-bias effect on parasitic resistance  & V$^{-1}$ & 1 \\ \hline
RDSW & Source-drain resistance per width & $\mathsf{\Omega}$ $\mu$m & 200 \\ \hline
RDSWMIN & Source-drain resistance per width at high Vg & $\mathsf{\Omega}$ $\mu$m & 0 \\ \hline
RDW & Drain resistance per width & $\mathsf{\Omega}$ $\mu$m & 100 \\ \hline
RDWMIN & Drain resistance per width at high Vg & $\mathsf{\Omega}$ $\mu$m & 0 \\ \hline
RSW & Source resistance per width & $\mathsf{\Omega}$ $\mu$m & 100 \\ \hline
RSWMIN & Source resistance per width at high Vg & $\mathsf{\Omega}$ $\mu$m & 0 \\ \hline
WR & Width dependence of rds & -- & 1 \\ \hline

\category{Impact Ionization Current Parameters}\\ \hline
ALPHA0 & substrate current model parameter & m/V & 0 \\ \hline
ALPHA1 & substrate current model parameter & V$^{-1}$ & 0 \\ \hline
BETA0 & substrate current model parameter & V$^{-1}$ & 0 \\ \hline

\category{Gate-induced Drain Leakage Model Parameters}\\ \hline
AGIDL & Pre-exponential constant for GIDL & $\mathsf{\Omega}^{-1}$ & 0 \\ \hline
AGISL & Pre-exponential constant for GISL & $\mathsf{\Omega}^{-1}$ & 0 \\ \hline
BGIDL & Exponential constant for GIDL & Vm$^{-1}$ & 2.3e+09 \\ \hline
BGISL & Exponential constant for GISL & Vm$^{-1}$ & 2.3e-09 \\ \hline
CGIDL & Parameter for body-bias dependence of GIDL & V$^3$ & 0.5 \\ \hline
CGISL & Parameter for body-bias dependence of GISL & V$^3$ & 0.5 \\ \hline
EGIDL & Fitting parameter for Bandbending & V & 0.8 \\ \hline
EGISL & Fitting parameter for Bandbending & V & 0.8 \\ \hline
FGIDL\newline{\normalfont [Only for versions starting with 4.7]} & GIDL vb parameter & V & 0 \\ \hline
FGISL\newline{\normalfont [Only for versions starting with 4.7]} & Parameter for GISL body bias dependence & V & 0 \\ \hline
KGIDL\newline{\normalfont [Only for versions starting with 4.7]} & GIDL vb parameter & V & 0 \\ \hline
KGISL\newline{\normalfont [Only for versions starting with 4.7]} & Parameter for GISL body bias dependence & V & 0 \\ \hline
RGIDL\newline{\normalfont [Only for versions starting with 4.7]} & GIDL vg parameter & -- & 1 \\ \hline
RGISL\newline{\normalfont [Only for versions starting with 4.7]} & Parameter for GISL gate bias dependence & -- & 1 \\ \hline
\end{DeviceParamTableGenerated}


\clearpage
\subsubsection{Level 18 MOSFET Tables (VDMOS)}
The vertical double-diffused power MOSFET model is based on the uniform charge
control model (UCCM) developed at Rensselaer Polytechnic Institute~\cite{Fjeldly:1998}.
The VDMOS current-voltage characteristics are described by a single, continuous
analytical expression for all regimes of operation.  The physics-based model
includes effects such as velocity saturation in the channel, drain induced barrier
lowering, finite output conductance in saturation, the quasi-saturation effect
through a bias dependent drain parasitic resistance, effects of bulk charge, and
bias dependent low-field mobility.  An important feature of the implementation
is the utilization of a single continuous expression for the drain current, which
is valid below and above threshold, effectively removing discontinuities and
improving convergence properties.

The following tables give parameters for the level 18 MOSFET.

% This table was generated by Xyce:
%   Xyce -doc_cat M 18
%
\index{power mosfet!device instance parameters}
\begin{DeviceParamTableGenerated}{Power MOSFET Device Instance Parameters}{M_18_Device_Instance_Params}

\category{Control Parameters}\\ \hline
M & Multiplier for M devices connected in parallel & -- & 1 \\ \hline

\category{Geometry Parameters}\\ \hline
AD & Drain diffusion area & m$^{2}$ & 0 \\ \hline
AS & Source diffusion area & m$^{2}$ & 0 \\ \hline
L & Channel length & m & 0 \\ \hline
NRD & Multiplier for RSH to yield parasitic resistance of drain & $\Box$ & 1 \\ \hline
NRS & Multiplier for RSH to yield parasitic resistance of source & $\Box$ & 1 \\ \hline
PD & Drain diffusion perimeter & m & 0 \\ \hline
PS & Source diffusion perimeter & m & 0 \\ \hline
W & Channel width & m & 0 \\ \hline

\category{Temperature Parameters}\\ \hline
DTEMP & Device delta temperature & $^\circ$C & 0 \\ \hline
TEMP & Device temperature & $^\circ$C & Ambient Temperature \\ \hline
\end{DeviceParamTableGenerated}

\input{M_18_Device_Model_Params}

\clearpage
\subsubsection{Levels 70 and 70450 MOSFET Tables (BSIM-SOI 4.6.1 and 4.5.0)}
For complete documentation of the BSIM-SOI model, see the users'
manual for the BSIM-SOI, available for download at
\url{http://bsim.berkeley.edu/models/bsimsoi/}.  \Xyce{} implements
Version 4.6.1 of the BSIM-SOI as the level 70 device and version 4.5.0
as level 70450.

Instance and model parameters of the level 70 MOSFET are given in
tables~\ref{M_70_Device_Instance_Params} and
\ref{M_70_Device_Model_Params}.

Beginning with \Xyce{} 7.2, the BSIM-SOI models level 70 and 70450
have {\em limited} support for the optional 5th, 6th, and 7th nodes.
See the BSIM-SOI technical manual at the BSIM web site for details of
what configurations the full device supports.  Only some of these use
cases are supported: Use of the BSIM-SOI 4.x with \texttt{TNODEOUT=0}
(the default) is supported in 4-, 5-, 6-, and 7-node configurations.
\texttt{TNODEOUT=1} is supported only in the 7-node configuration,
with the 7th node being temperature.  No access to the external
temperature node is available in 5- or 6- node configuration.

When \texttt{TNODEOUT=0}, the temperature node is an internal node of
the device even when not specified on the instance line, and its value
may still be printed using the N() notation (see
section~\ref{Print_Device_Info}).  This somewhat minimizes the impact
of the lack of support for \texttt{TNODEOUT=1} in \Xyce{} --- the
temperature rise due to self-heating is always available for printing,
but it is not available for creation of a thermal coupling network
except in the 7-node configuration.

Note that with some choices of model parameters, the BSIM-SOI devices
attempt to ``collapse'' the ``P'' and ``B'' nodes (external and
internal body nodes, 5th and 6th netlist nodes if given, internal
nodes if not given).  Xyce is unable to perform such collapse when the
nodes are externally specified, and will issue warnings when it finds
the model trying to do so.  Depending on the actual nodes used for P
and B, the device may fail to converge or produce invalid results; as
an example, if P and B are actually specified on the netlist line to
be the same node, this failure to collapse will not matter --- the
nodes are already the same.  But if two different node names are used
for the 5th and 6th nodes, the failure to collapse will leave one node
floating and the simulation will likely fail if the printed warnings
are ignored.

A similar problem exists for other choices of model parameter: in some
cases neither the ``P'' nor ``B'' nodes are used, and if the nodes are
specified on the netlist line the BSIM-SOI code attempts to collapse
them to ground.  This is not something \Xyce{} can do, and therefore
instead \Xyce{} ignores the specified nodes.  This can leave those
nodes floating and lead to convergence failures unless the specified
nodes are already the ground node (node 0).  \Xyce{} will issue
appropriate warnings when this condition exists and suggest removal of
the unused external nodes from the instance line.

The BSIM SOI 4.6.1 device supports output of the internal variables in
table~\ref{M_70_OutputVars} on the \texttt{.PRINT} line of a netlist.
To access them from a print line, use the syntax
\texttt{N(<instance>:<variable>)} where ``\texttt{<instance>}'' refers to the
name of the specific level 70 M device in your netlist.

\textbf{NOTE:} It has been observed that the gate capacitance model of
BSIM-SOI 4.6.1 behaves differently than earlier versions, and the team
has seen significant disagreement of gate currents when comparing
identical simulations with other simulators that have only earlier
BSIM-SOI models.  For this reason, we are also providing BSIM-SOI
4.5.0 as the level 70450 MOSFET.  This model does agree with these
other simulators.  The parameters and output variables are given in
tables~\ref{M_70450_Device_Instance_Params},
\ref{M_70450_Device_Model_Params}, and \ref{M_70450_OutputVars}.
Unlike BSIM-SOI 4.6.1, the 4.5.0 model's original Verilog-A source
code does not contain descriptions and units for the parameters, and
these appear blank in the tables.  For descriptions and units, see the
corresponding parameters in the level 70 tables.

\input{M_70_Device_Instance_Params}
\input{M_70_Device_Model_Params}
\input{M_70_OutputVars}
\input{M_70450_Device_Instance_Params}
\input{M_70450_Device_Model_Params}
\input{M_70450_OutputVars}

\clearpage
\subsubsection{Level 77 MOSFET Tables (BSIM6 version 6.1.1)}
\Xyce{} includes the BSIM6 MOSFET model, version 6.1.1.  Full
documentation of the BSIM6 is available at its web site,
\url{http://bsim.berkeley.edu/models/bsim6/}.  Instance and model
parameters for the BSIM6 are given in
tables~\ref{M_77_Device_Instance_Params} and
\ref{M_77_Device_Model_Params}.  These tables are generated directly
from information present in the original Verilog-A implementation of
the BSIM6, and lack many descriptions for the parameters.  Consult the
BSIM6 technical manual from the BSIM group for further details about
these parameters.

Beginning with version 7.2 of \Xyce{}, an optional fifth node may be
specified for BSIM6 devices.  If specified, it is the temperature
node, which is used by the self-heating model and is internal if not
specified on the instance line.

The BSIM6 device supports output of the internal variables in
table~\ref{M_77_OutputVars} on the \texttt{.PRINT} line of a netlist.
To access them from a print line, use the syntax
\texttt{N(<instance>:<variable>)} where ``\texttt{<instance>}'' refers to the
name of the specific level 77 M device in your netlist.

\input{M_77_Device_Instance_Params}
\input{M_77_Device_Model_Params}
\input{M_77_OutputVars}

\clearpage
\subsubsection{Level 102 MOSFET Tables (PSP version 102.5)}

\Xyce{} includes a legacy version of the PSP MOSFET model, version
102.5.  This version is provided because the more recent 103 versions
are not backward compatible with the older 102 versions, and some
foundries provide model cards that use the version 102.  Development
of new model cards should be done using the more recent, supported
versions of PSP.

The PSP102 device supports output of the internal variables in
table~\ref{M_102_OutputVars} on the \texttt{.PRINT} line of a netlist.
To access them from a print line, use the syntax
\texttt{N(<instance>:<variable>)} where ``\texttt{<instance>}'' refers to the
name of the specific PSP102 M device in your netlist.

\input{M_102_Device_Instance_Params}
\input{M_102_Device_Model_Params}
\input{M_102_OutputVars}

\subsubsection{Level 103 and 1031 MOSFET Tables (PSP version 103.4)}

\Xyce{} includes the PSP MOSFET model, version 103.4~\cite{PSP:2006}.
The version without self-heating is the level 103 MOSFET, and the
version with self-heating is the level 1031.  Note that the level 1031
MOSFET requires five nodes on its instance line: drain, gate, source,
bulk, and dt.  The fifth node will be the temperature rise of the
device due to self-heating.

Full documentation for the PSP model is available on its web site,
\url{http://www.cea.fr/cea-tech/leti/pspsupport}.  Instance and model
parameters for the PSP model are given in
tables~\ref{M_103_Device_Instance_Params}, \ref{M_103_Device_Model_Params},
\ref{M_1031_Device_Instance_Params}, and \ref{M_1031_Device_Model_Params}.

The PSP103 devices support output of the internal variables in
table~\ref{M_103_OutputVars} and table~\ref{M_1031_OutputVars} on the \texttt{.PRINT} line of a netlist.
To access them from a print line, use the syntax
\texttt{N(<instance>:<variable>)} where ``\texttt{<instance>}'' refers to the
name of the specific PSP103 M device in your netlist.

% This table was generated by Xyce:
%   Xyce -doc M 103
%
\index{psp103va mosfet!device instance parameters}
\begin{DeviceParamTableGenerated}{PSP103VA MOSFET Device Instance Parameters}{M_103_Device_Instance_Params}
ABDRAIN & Bottom area of drain junction & m$^{2}$ & 1e-12 \\ \hline
ABSOURCE & Bottom area of source junction & m$^{2}$ & 1e-12 \\ \hline
AD & Bottom area of drain junction & m$^{2}$ & 1e-12 \\ \hline
AS & Bottom area of source junction & m$^{2}$ & 1e-12 \\ \hline
DELVTO & Threshold voltage shift parameter & V & 0 \\ \hline
DELVTOEDGE & Threshold voltage shift parameter of edge transistor & V & 0 \\ \hline
DTA & Temperature offset w.r.t. ambient temperature & K & 0 \\ \hline
FACTUO & Zero-field mobility pre-factor & --- & 1 \\ \hline
FACTUOEDGE & Zero-field mobility pre-factor of edge transistor & --- & 1 \\ \hline
JW & Gate-edge length of source/drain junction & m & 1e-06 \\ \hline
L & Design length & m & 1e-05 \\ \hline
LGDRAIN & Gate-edge length of drain junction & m & 1e-06 \\ \hline
LGSOURCE & Gate-edge length of source junction & m & 1e-06 \\ \hline
LSDRAIN & STI-edge length of drain junction & m & 1e-06 \\ \hline
LSSOURCE & STI-edge length of source junction & m & 1e-06 \\ \hline
M &  Alias for MULT & --- & 1 \\ \hline
MULT & Number of devices in parallel & --- & 1 \\ \hline
NF & Number of fingers & --- & 1 \\ \hline
NGCON & Number of gate contacts & --- & 1 \\ \hline
NRD & Number of squares of drain diffusion & --- & 0 \\ \hline
NRS & Number of squares of source diffusion & --- & 0 \\ \hline
PD & Perimeter of drain junction & m & 1e-06 \\ \hline
PS & Perimeter of source junction & m & 1e-06 \\ \hline
SA & Distance between OD-edge and poly from one side & m & 0 \\ \hline
SB & Distance between OD-edge and poly from other side & m & 0 \\ \hline
SC & Distance between OD-edge and nearest well edge & m & 0 \\ \hline
SCA & Integral of the first distribution function for scattered well dopants & --- & 0 \\ \hline
SCB & Integral of the second distribution function for scattered well dopants & --- & 0 \\ \hline
SCC & Integral of the third distribution function for scattered well dopants & --- & 0 \\ \hline
SD & Distance between neighbouring fingers & m & 0 \\ \hline
W & Design width & m & 1e-05 \\ \hline
XGW & Distance from the gate contact to the channel edge & m & 1e-07 \\ \hline
\end{DeviceParamTableGenerated}

\input{M_103_Device_Model_Params}
\input{M_103_OutputVars}
% This table was generated by Xyce:
%   Xyce -doc M 1031
%
\index{psp103va mosfet with self-heating!device instance parameters}
\begin{DeviceParamTableGenerated}{PSP103VA MOSFET with self-heating Device Instance Parameters}{M_1031_Device_Instance_Params}
ABDRAIN & Bottom area of drain junction & m$^{2}$ & 1e-12 \\ \hline
ABSOURCE & Bottom area of source junction & m$^{2}$ & 1e-12 \\ \hline
AD & Bottom area of drain junction & m$^{2}$ & 1e-12 \\ \hline
AS & Bottom area of source junction & m$^{2}$ & 1e-12 \\ \hline
DELVTO & Threshold voltage shift parameter & V & 0 \\ \hline
DELVTOEDGE & Threshold voltage shift parameter of edge transistor & V & 0 \\ \hline
DTA & Temperature offset w.r.t. ambient temperature & K & 0 \\ \hline
FACTUO & Zero-field mobility pre-factor & --- & 1 \\ \hline
FACTUOEDGE & Zero-field mobility pre-factor of edge transistor & --- & 1 \\ \hline
JW & Gate-edge length of source/drain junction & m & 1e-06 \\ \hline
L & Design length & m & 1e-05 \\ \hline
LGDRAIN & Gate-edge length of drain junction & m & 1e-06 \\ \hline
LGSOURCE & Gate-edge length of source junction & m & 1e-06 \\ \hline
LSDRAIN & STI-edge length of drain junction & m & 1e-06 \\ \hline
LSSOURCE & STI-edge length of source junction & m & 1e-06 \\ \hline
M &  Alias for MULT & --- & 1 \\ \hline
MULT & Number of devices in parallel & --- & 1 \\ \hline
NF & Number of fingers & --- & 1 \\ \hline
NGCON & Number of gate contacts & --- & 1 \\ \hline
NRD & Number of squares of drain diffusion & --- & 0 \\ \hline
NRS & Number of squares of source diffusion & --- & 0 \\ \hline
PD & Perimeter of drain junction & m & 1e-06 \\ \hline
PS & Perimeter of source junction & m & 1e-06 \\ \hline
SA & Distance between OD-edge and poly from one side & m & 0 \\ \hline
SB & Distance between OD-edge and poly from other side & m & 0 \\ \hline
SC & Distance between OD-edge and nearest well edge & m & 0 \\ \hline
SCA & Integral of the first distribution function for scattered well dopants & --- & 0 \\ \hline
SCB & Integral of the second distribution function for scattered well dopants & --- & 0 \\ \hline
SCC & Integral of the third distribution function for scattered well dopants & --- & 0 \\ \hline
SD & Distance between neighbouring fingers & m & 0 \\ \hline
W & Design width & m & 1e-05 \\ \hline
XGW & Distance from the gate contact to the channel edge & m & 1e-07 \\ \hline
\end{DeviceParamTableGenerated}

\input{M_1031_Device_Model_Params}
\input{M_1031_OutputVars}

\clearpage
\subsubsection{Level 110 MOSFET Tables (BSIM CMG version 110.0.0)}
\Xyce{} includes the BSIM CMG Common Multi-gate model version 110.
The code in \Xyce{} was generated from the BSIM group's Verilog-A
input using the default ``ifdef'' lines provided, and therefore
supports only the subset of BSIM CMG features those defaults enable.
Instance and model parameters for the BSIM CMG model are given in
tables~\ref{M_110_Device_Instance_Params} and
\ref{M_110_Device_Model_Params}.  Details of the model are documented
in the BSIM-CMG technical report\cite{BSIMCMG:Manual}, available from
the BSIM web site at
\url{http://bsim.berkeley.edu/models/bsimcmg/}.

The BSIM CMG devices support output of the internal variables in
tables~\ref{M_107_OutputVars}, \ref{M_108_OutputVars}, and  \ref{M_110_OutputVars} on the \texttt{.PRINT} line of a netlist.
To access them from a print line, use the syntax
\texttt{N(<instance>:<variable>)} where ``\texttt{<instance>}'' refers to the
name of the specific level 107 or 108 M device in your netlist.

\input{M_110_Device_Instance_Params}
\input{M_110_Device_Model_Params}
\input{M_110_OutputVars}

\subsubsection{Level 107  and 108 MOSFET Tables (BSIM CMG versions 107.0.0 and 108.0.0)}
\Xyce{} includes the legacy BSIM CMG Common Multi-gate model versions 107 and 108.
These models have been superceded by the level 110 version, but has been
retained for backward compatibility with previous versions of Xyce and
older model cards and PDKs.  The code in \Xyce{} was generated from the BSIM
group's Verilog-A input using the default ``ifdef'' lines provided,
and therefore supports only the subset of BSIM CMG features those
defaults enable.  Instance and model parameters for the BSIM CMG model
are given in tables~\ref{M_107_Device_Instance_Params},
\ref{M_107_Device_Model_Params}, \ref{M_108_Device_Instance_Params},
and~\ref{M_108_Device_Model_Params}.  Details of the model are documented
in the BSIM-CMG technical report\cite{BSIMCMG:Manual}, available from
the BSIM web site at \url{http://bsim.berkeley.edu/models/bsimcmg/}.

Note that the TNOIMOD=1 option of BSIM-CMG 108 is not supported in
Xyce, as it uses features of Verilog-A that are not supported in our
Verilog-A compiler.  This noise model was added in version 108 and
removed in version 109.  The TNOIMOD=2 option of BSIM-CMG 108 is the
same as the TNOIMOD=1 option of BSIM-CMG 110.

\input{M_107_Device_Instance_Params}
\input{M_107_Device_Model_Params}
\input{M_107_OutputVars}

\input{M_108_Device_Instance_Params}
\input{M_108_Device_Model_Params}
\input{M_108_OutputVars}


\clearpage
\subsubsection{Levels 2000 and 2001 MOSFET Tables (MVS version 2.0.0)}
\Xyce{} includes the MIT Virtual Source (MVS) MOSFET model version
2.0.0 in both ETSOI and HEMT variants.  The code in \Xyce{} was
generated from the MIT Verilog-A input.  Model parameters for the MVS
model are given in \ref{M_2000_Device_Model_Params} and
\ref{M_2001_Device_Model_Params}.  The MVS model does not have
instance parameters.  Details of the model are documented MVS
Nanotransistor Model 2.0.0 manual, available from the NEEDS web site
at \url{https://nanohub.org/publications/74/1}.

{\bf NOTE: } Unlike all other MOSFET models in Xyce, the MVS model
takes only 3 nodes, the drain, gate and source.  It takes no substrate
node.

\input{M_2000_Device_Model_Params}
\input{M_2001_Device_Model_Params}

\clearpage

\subsubsection{Level 2002 MOSFET Tables (MVSG\_CMC version 1.1.0)}
\Xyce{} includes the MIT Virtual Source GaN HEMT High-Voltage
(MVSG\_CMC) MOSFET model version 1.1.0.  The code in \Xyce{} was
generated from the MIT Verilog-A input.  Model parameters for the MVS
model are given in \ref{M_2002_Device_Instance_Params} and
\ref{M_2002_Device_Model_Params}, and its output variables in
\ref{M_2002_OutputVars}.  More information about this model may be
obtained from the CMC standard models page at
\url{https://si2.org/standard-models}.

\input{M_2002_Device_Instance_Params}
\input{M_2002_Device_Model_Params}
\input{M_2002_OutputVars}


\clearpage
\subsubsection{Level 301 MOSFET Tables (EKV version 3.0.1)}
\Xyce{} includes the EKV MOSFET model, version
3.0.1~\cite{BLETK:1997}\cite{EKV:2006}\cite{EKV:2007}.  Full
documentation for the EKV3 model is available on the \Xyce{} internal web site;
the documentation for the EKV3 model may be freely redistributed.  Instance and
model parameters for the EKV model are given in
tables~\ref{M_301_Device_Instance_Params} and \ref{M_301_Device_Model_Params}.

The EKV3 model is developed by the EKV Team of the Electronics Laboratory-TUC
(Technical University of Crete). It is included in \Xyce{} under license from
Technical University of Crete.  The official web site of the EKV model is
\url{http://ekv.epfl.ch/}.

\textbf{Due to licensing restrictions, the EKV3 MOSFET is not available in
     open-source versions of \Xyce{}.  The license for EKV3 authorizes Sandia
     National Laboratories to distribute EKV3 only in binary versions of code.}


\input{M_301_Device_Instance_Params}
\input{M_301_Device_Model_Params}

\clearpage
\subsubsection{Level 10240 MOSFET Tables (L\_UTSOI Version 102.4.0)}
Select \Xyce{} binaries include the L\_UTSOI MOSFET model as the level
10240 MOSFET.  This model's parameters and output variables are listed in tables~\ref{M_10240_Device_Instance_Params}, \ref{M_10240_Device_Model_Params}, and \ref{M_10240_OutputVars}

\input{M_10240_Device_Instance_Params}
\input{M_10240_Device_Model_Params}
%table generated from Verilog-A input
\index{MOSFET level 10240!device output variables}
\begin{DeviceParamTableGenerated}{MOSFET level 10240 Output Variables}{M_10240_OutputVars}
type & Flag for channel type &    & none \\ \hline
vds & Internal drain-source DC voltage (NMOS convention) &   V & none \\ \hline
vsb & Internal source-bulk DC voltage (NMOS convention) &   V & none \\ \hline
vgs & Internal gate-source DC voltage (NMOS convention) &   V & none \\ \hline
vth & Threshold voltage &   V & none \\ \hline
vth\_drive & Effective gate drive voltage, including back bias, drain bias effects and self-heating &   V & none \\ \hline
vdsat & Drain saturation voltage at the given bias &   V & none \\ \hline
vdsat\_marg & Vds voltage margin &   V & none \\ \hline
id & Total DC drain current flowing into drain terminal &   A & none \\ \hline
ig & Total DC gate current flowing into gate terminal &   A & none \\ \hline
is & Total DC source current flowing into source terminal &   A & none \\ \hline
ib & Total DC bulk current flowing into bulk terminal &   A & none \\ \hline
ids & DC channel current, excluding tunnel, GISL and GIDL currents &   A & none \\ \hline
igidl & DC Gate Induced Drain Leakage current &   A & none \\ \hline
igisl & DC Gate Induced Source Leakage current &   A & none \\ \hline
igs & DC gate-source leakage current &   A & none \\ \hline
igd & DC gate-drain leakage current &   A & none \\ \hline
idb & DC drain-bulk current &   A & none \\ \hline
isb & DC source-bulk current &   A & none \\ \hline
gm & Internal DC transconductance &   A/V & none \\ \hline
gmb & Internal DC bulk transconductance &   A/V & none \\ \hline
gds & Internal DC output conductance &   A/V & none \\ \hline
cgg & Internal AC gate capacitance, including overlap capacitances &   F & none \\ \hline
cgd & Internal AC gate-drain transcapacitance, including overlap capacitances &   F & none \\ \hline
cgs & Internal AC gate-source transcapacitance, including overlap capacitances &   F & none \\ \hline
cgb & Internal AC gate-bulk transcapacitance &   F & none \\ \hline
cdd & Internal AC drain capacitance &   F & none \\ \hline
cdg & Internal AC drain-gate transcapacitance &   F & none \\ \hline
cds & Internal AC drain-source transcapacitance &   F & none \\ \hline
cdb & Internal AC drain-bulk transcapacitance &   F & none \\ \hline
cbb & Internal AC bulk capacitance &   F & none \\ \hline
cbg & Internal AC bulk-gate transcapacitance &   F & none \\ \hline
cbs & Internal AC bulk-source transcapacitance &   F & none \\ \hline
cbd & Internal AC bulk-drain transcapacitance &   F & none \\ \hline
css & Internal AC source capacitance &   F & none \\ \hline
csg & Internal AC source-gate transcapacitance &   F & none \\ \hline
csb & Internal AC source-bulk transcapacitance &   F & none \\ \hline
csd & Internal AC source-drain transcapacitance &   F & none \\ \hline
tk & MOSFET device temperature &   K & none \\ \hline
dtsh & MOSFET device temperature increase due to self-heating &   K & none \\ \hline
self\_gain & Internal L-UTSOI model self gain &    & none \\ \hline
rout & AC output resistance &   Ohm & none \\ \hline
beff & Gain factor in saturation &   A/V$^{2}$ & none \\ \hline
ft & Unity gain frequency at the given bias &   Hz & none \\ \hline
rgate & MOS gate resistance (intrinsic input resistance) &   Ohm & none \\ \hline
gmoverid & Gm over Id &   1/V & none \\ \hline
vearly & Equivalent Early voltage &   V & none \\ \hline
\end{DeviceParamTableGenerated}




%%
%% Lossy Transmission Line Model (LTRA)
%%
\clearpage
\subsection{Lossy Transmission Line (LTRA)}
\index{device!ltra} \index{lossy transmission line}
\index{device!transmission line!lossy}
% Sandia National Laboratories is a multimission laboratory managed and
% operated by National Technology & Engineering Solutions of Sandia, LLC, a
% wholly owned subsidiary of Honeywell International Inc., for the U.S.
% Department of Energy’s National Nuclear Security Administration under
% contract DE-NA0003525.

% Copyright 2002-2023 National Technology & Engineering Solutions of Sandia,
% LLC (NTESS).


\begin{Device}

\symbol
{\includegraphics{translineSymbol}}

\device
\begin{alltt}
O<name> <A port (+) node> <A port (-) node>
+ <B port (+) node> <B port (-) node> [model name]
\end{alltt}

\model
\begin{alltt}
.MODEL <model name> LTRA R=<value> L=<value> C=<value>
+ G=<value> LEN=<value> [model parameters]
\end{alltt}

\examples
\begin{alltt}
Oline1 inp inn outp outn cable1
Oline2 inp inn outp outn cable1
\end{alltt}

\comments

The lossy transmission line, or LTRA, device is a two port (\texttt{A}
and \texttt{B}), bi-directional device. The \texttt{(+)} and \texttt{(-)} nodes
define the polarity of a positive voltage at a port.

\texttt{R}, \texttt{L}, \texttt{C}, and \texttt{G} are the resistance,
inductance, capacitance, and conductance of the transmission line per unit
length, respectively. \texttt{LEN} is the total length of the transmission
line. Supported configurations for the LTRA are \texttt{RLC}, \texttt{RC},
\texttt{LC} (lossless) and \texttt{RG}.

The lossy transmission line, or LTRA, device does not work with AC
analysis at this time.  LTRA models will need to be replaced with
lumped transmission line models (YTRANSLINE) when used in AC analysis.
The LTRA models do work correctly in harmonic balance simulation.
\end{Device}


%\paragraph{Device Parameters}
%\input{O_1_Device_Instance_Params}

\paragraph{Model Parameters}
\input{O_1_Device_Model_Params}

By default time step limiting is on in the LTRA. This means that
simulation step sizes will be reduced if required by the LTRA to
preserve accuracy. This can be disabled by setting
\texttt{NOSTEPLIMIT=1} and \texttt{TRUNCDONTCUT=1} on the
\texttt{.MODEL} line.

The option most worth experimenting with for increasing the speed of
simulation is \texttt{REL}. The default value of 1 is usually safe
from the point of view of accuracy but occasionally increases
computation time. A value greater than 2 eliminates all breakpoints
and may be worth trying depending on the nature of the rest of the
circuit, keeping in mind that it might not be safe from the viewpoint
of accuracy. Breakpoints may be entirely eliminated if the circuit
does not exhibit any sharp discontinuities. Values between 0 and 1 are
usually not required but may be used for setting many breakpoints.

\texttt{COMPACTREL} and \texttt{COMPACTABS} are tolerances that
control when the device should attempt to compact past history. This
can significantly speed up the simulation, and reduce memory usage,
but can negatively impact accuracy and in some cases may cause
problems with the nonlinear solver. In general this capability should
be used with linear type signals, such as square-wave-like
voltages. In order to activate this capability the general device
option \texttt{TRYTOCOMPACT=1} must be set, if it is not no history
compaction will be performed and the \texttt{COMPACT} options will be
ignored.

Example:

\texttt{.OPTIONS DEVICE TRYTOCOMPACT=1}

\paragraph{References}
See references \cite{Roychodhury:1994} and \cite{Spice3f5-user-guide} for more information
about the model.


%%
%% Voltage- or Current-controlled Switch Subsection
%%
\clearpage
\subsection{Voltage- or Current-controlled Switch}
\index{device!controlled switch} \index{controlled switch}
% Sandia National Laboratories is a multimission laboratory managed and
% operated by National Technology & Engineering Solutions of Sandia, LLC, a
% wholly owned subsidiary of Honeywell International Inc., for the U.S.
% Department of Energy’s National Nuclear Security Administration under
% contract DE-NA0003525.

% Copyright 2002-2024 National Technology & Engineering Solutions of Sandia,
% LLC (NTESS).


\begin{Device}

\device
\begin{alltt}
S<name> <(+) switch node> <(-) switch node>
+ <(+) control node> <(-) control node>
+ <model name> [ON] [OFF]

W<name> <(+) switch node> <(-) switch node>
+ <control node voltage source>
+ <model name> [ON] [OFF]
\end{alltt}

\model
\begin{alltt}
.MODEL <model name> VSWITCH [model parameters]
.MODEL <model name> ISWITCH [model parameters]
\end{alltt}

\examples
\begin{alltt}
S1 21 23 12 10 SMOD1
SSET 15 10 1 13 SRELAY
W1 1 2 VCLOCK SWITCHMOD1
W2 3 0 VRAMP SM1 ON
\end{alltt}

\comments

The voltage- or current-controlled switch is a particular type of
controlled resistor. This model is designed to help reduce numerical
issues. See Special Considerations below.

The resistance between the \texttt{<(+) switch node>} and the
\texttt{<(-) switch node>} is dependent on either the voltage between
the \texttt{<(+) control node>} and the \texttt{<(-) control node>} or
the current through the control node voltage source. The resistance
changes in a continuous manner between the \texttt{RON} and
\texttt{ROFF} model parameters.

No resistance is inserted between the control nodes.  It is up to the
user to make sure that these nodes are not floating.

Even though evaluating the switch model is computationally
inexpensive, for transient analysis \Xyce{} steps through the
transition section using small time-steps in order to calculate the
waveform accurately. Thus, a circuit with many switch transitions can
result in lengthy run times.

The ON and OFF parameters are used to specify the initial state of the
switch at the first step of the operating point calculation; this does
not force the switch to be in that state, it only gives the operating
point solver an initial state to work with.  If it is known that the
switch should be in a particular state in the operating point it could
help convergence to specify one of these keywords.

The power dissipated in the switch is calculated with $I \cdot \Delta V$ 
where the voltage drop is calculated as $(V_+ - V_-)$ and positive current 
flows from $V_+$ to $V_-$.  This will essentially be the power dissipated
in either \texttt{RON} or \texttt{ROFF}, since the switch is a particular 
type of controlled resistor.

\textbf{Note:} The voltage- and current-controlled switches specified
in this manner are converted at parse time into equivalent ``generic''
switches.

\end{Device}

\pagebreak

\paragraph{Model Parameters}
% This table was generated by Xyce:
%   Xyce -doc S 1
%
\index{controlled switch!device model parameters}
\begin{DeviceParamTableGenerated}{Controlled Switch Device Model Parameters}{S_1_Device_Model_Params}
IHOFF & Off current & A & 0 \\ \hline
IHON & On current with hysteresis & A & 0.001 \\ \hline
IOFF & Off current with hysteresis & A & 0 \\ \hline
ION & On current & A & 0.001 \\ \hline
OFF & Off control value & -- & 0 \\ \hline
OFFH & Off control value with hysteresis & -- & 0 \\ \hline
ON & On control value & -- & 1 \\ \hline
ONH & On control value with hysteresis & -- & 1 \\ \hline
ROFF & Off resistance & $\mathsf{\Omega}$ & 1e+06 \\ \hline
RON & On resistance & $\mathsf{\Omega}$ & 1 \\ \hline
VHOFF & Off voltage with hysteresis & V & 0 \\ \hline
VHON & On voltage with hysteresis & V & 1 \\ \hline
VOFF & Off voltage & V & 0 \\ \hline
VON & On voltage & V & 1 \\ \hline
\end{DeviceParamTableGenerated}


\paragraph{Special Considerations}

\begin{XyceItemize}
\item Due to numerical limitations, \Xyce{} can only manage a dynamic range of
  approximately 12 decades.  Thus, it is recommended the user limit the ratio
  \textrmb{ROFF}/\textrmb{RON} to less than $10^{12}$.  This soft limitation is not enforced by the code, and larger ratios might converge for some problems.
\item Do not set \textrmb{RON} to 0.0, as the code computes the ``on'' conductance as the inverse of \textrmb{RON}.  Using 0.0 will cause the simulation to fail when this invalid division results in an infinite conductance.  Use a very small, but non-zero, on resistance instead.
\item Furthermore, it is a good idea to limit the narrowness of the transition
  region. This is because in the transition region, the switch has gain and the
  narrower the region, the higher the gain and the more potential for numerical
  problems.  The smallest value recommended for $\|\mathbf{VON - VOFF}\|$ or
  $\|\mathbf{ION - IOFF}\|$ is $1\times10^{-12}$.  This recommendation is not a restriction, and you might find for some problems that narrower transition regions might work well.
\end{XyceItemize}

\paragraph{Controlled switch equations}
The equations in this section use the following variables:
\[
\begin{array}{rllll}
R_s & = & \mbox{switch resistance} \\
V_c & = & \mbox{voltage across control nodes} \\
I_c & = & \mbox{current through control node voltage source} \\
L_m & = & \mbox{log-mean of resistor values} & = &
\ln \left(\sqrt{\mathbf{RON \cdot ROFF}} \right) \\
L_r & = & \mbox{log - ratio of resistor values} & = &
\ln \left(\mathbf{RON / ROFF} \right) \\
V_d & = & \mbox{difference of control voltages} & = & \mathbf{VON - VOFF} \\
I_d & = & \mbox{difference of control currents} & = & \mathbf{ION - IOFF} \\
\end{array}
\]

\subparagraph{Switch Resistance}

To compute the switch resistance, \Xyce{} first calculates the
``switch state'' $S$ as $S=(V_c-\mathbf{VOFF})/V_d$ or
$S=(I_c-\mathbf{IOFF})/I_d$.  The switch resistance is then:
\[ R_s = \left\{ \begin{array}{ll}
\mathbf{RON}, & S \geq 1.0 \\
\mathbf{ROFF}, & S \leq 0.0 \\
\exp \left(L_m + 0.75L_r(2S-1) - 0.25L_r(2S-1)^3 \right),
& 0< S < 1
\end{array}
\right.
\]

%\subparagraph{Noise}
%Noise is computed using a 1.0 Hz bandwidth.  The voltage-controlled switch
%produces thermal noise as though it were a resistor with the resistance a
%switch has at its bias point.  It uses a spectral power density (per unit
%bandwidth):
%\[
%i^2 = 4kT/R_s
%\]


%%
%% Generic Switch Subsection
%%
\clearpage
\subsection{Generic Switch}
\index{device!generic switch} \index{generic switch}
% Sandia National Laboratories is a multimission laboratory managed and
% operated by National Technology & Engineering Solutions of Sandia, LLC, a
% wholly owned subsidiary of Honeywell International Inc., for the U.S.
% Department of Energy’s National Nuclear Security Administration under
% contract DE-NA0003525.

% Copyright 2002-2024 National Technology & Engineering Solutions of Sandia,
% LLC (NTESS).


\begin{Device}

\device
S<name> <(+) switch node> <(-) switch node> <model name> [ON] [OFF] <control = { expression }>

\model
\begin{alltt}
.MODEL <model name> VSWITCH [model parameters]
.MODEL <model name> ISWITCH [model parameters]
.MODEL <model name> SWITCH [model parameters]
\end{alltt}

\examples
\begin{alltt}
S1 1 2 SWI OFF CONTROL=\{I(VMON)\}
SW2 1 2 SWV OFF CONTROL=\{V(3)-V(4)\}
S3 1 2 SW OFF CONTROL=\{if(time>0.001,1,0)\}
\end{alltt}

\comments

The generic switch is similar to the voltage- or current-controlled
switch except that the control variable is anything that can be writen
as an expression.  The examples show how a voltage- or
current-controlled switch can be implemented with the generic switch.
Also shown is a relay that turns on when a certain time is reached.
Model parameters are given in Table ~\ref{S_1_Device_Model_Params}.

The voltage- and current-controlled switch syntaxes are converted at
parse time to their equivalent generic device, and so all three
variants in fact use the same code internally.

The power dissipated in the generic switch is calculated with $I \cdot \Delta V$ 
where the voltage drop is calculated as $(V_+ - V_-)$ and positive current 
flows from $V_+$ to $V_-$.  This will essentially be the power dissipated
in either \texttt{RON} or \texttt{ROFF}, since the generic switch is a particular 
type of controlled resistor.

\end{Device}


%%
%% YLIN device Subsection
%%
\clearpage
\subsection{Linear device}
\index{linear device}
% Sandia National Laboratories is a multimission laboratory managed and
% operated by National Technology & Engineering Solutions of Sandia, LLC, a
% wholly owned subsidiary of Honeywell International Inc., for the U.S.
% Department of Energy’s National Nuclear Security Administration under
% contract DE-NA0003525.

% Copyright 2002-2023 National Technology & Engineering Solutions of Sandia,
% LLC (NTESS).

The linear (YLIN) device allows an S-, Y-, or Z-parameter model to
be used to define an N-port device.  It is most commonly used
as part of a Harmonic Balance (HB) analysis.

\begin{Device}\label{YLIN_DEVICE}

\device
\begin{alltt}
YLIN <name>  <(+) node> <(-) node> [model name]
\end{alltt}

\model
.MODEL <model name> LIN [model parameters]

\examples
\begin{alltt}
YLIN YLIN1 1 0 2 0 YLIN_MOD1
.MODEL YLIN_MOD1 LIN TSTONEFILE=yparams.y2p
\end{alltt}

\parameters
\begin{Parameters}

\param{model name}
  Name of the model defined in a .MODEL line.

\end{Parameters}

\comments
At present, the YLIN device is only supported in the frequency domain for HB analyses.

\end{Device}


\paragraph{Model Parameters}
% This table was generated by Xyce:
%   Xyce -doc Lin 1
%
\index{lin!device model parameters}
\begin{DeviceParamTableGenerated}{LIN Device Model Parameters}{Lin_1_Device_Model_Params}
ISC\_FD & Touchstone file contains frequency-domain short-circuit current data & logical (T/F) & false \\ \hline
%ISC\_TD\_FILE & ISC Time Domain File Name & -- & '' \\ \hline
%ISC\_TD\_FILE\_FORMAT & Format of ISC Time Domain File & -- & 'STD' \\ \hline
TSTONEFILE & Touchstone File Name & -- & '' \\ \hline

INTERPOLATION & Interpolation method. Supported methods:
\begin{XyceItemize}
\item 1 (linear)
\item 2 (akima spline)
\end{XyceItemize}   & -- & 1   \\ \hline

HIGHPASS & Extrapolation method for higher frequency points. Supported methods:
\begin{XyceItemize}
\item 0 (cut off)
\item 1 (use highest frequency point)
\item 2 (linear extrapolation using highest 2 points)
\end{XyceItemize}   & -- & 1   \\ \hline


LOWPASS & Extrapolation method for lower frequency points. Supported methods:
\begin{XyceItemize}
\item 0 (cut off)
\item 1 (use lowest frequency point)
\item 2 (linear extrapolation using lowest 2 points)
\end{XyceItemize}   & -- & 1   \\ \hline


\end{DeviceParamTableGenerated}


The Touchstone file name must be specified.  The YLIN device accepts both
Touchstone 1 and Touchstone 2 formatted input files \cite{touchstone2_std_2009}.

For coupling with EM codes, such as EIGER, the YLIN device also accepts
a non-standard version of the Touchstone input files.  If the \texttt{ISC\_FD}
model parameter is set to true then each row of ``network data'' in the input
file also contains additional columns with the ``per-port frequency-domain
short-circuit currents''.   There are then two such additional columns for each
port.  The format (\texttt{RI}, \texttt{MA} or \texttt{DB}) of those additional
columns will be as specified by the Option line in the Touchstone file.  In this
non-standard case, only the ``Full'' matrix format is supported for
Touchstone 2 input files.






%%
%% Lossless, Ideal Transmission Line Subsection
%%
\clearpage
\subsection{Lossless (Ideal) Transmission Line}
\index{device!lossless transmission line} \index{lossless transmission line}
\index{device!transmission line!lossless}
% Sandia National Laboratories is a multimission laboratory managed and
% operated by National Technology & Engineering Solutions of Sandia, LLC, a
% wholly owned subsidiary of Honeywell International Inc., for the U.S.
% Department of Energy’s National Nuclear Security Administration under
% contract DE-NA0003525.

% Copyright 2002-2023 National Technology & Engineering Solutions of Sandia,
% LLC (NTESS).


\begin{Device}\label{T_DEVICE}

\symbol
{\includegraphics{translineSymbol}}

\device
\begin{alltt}
T<name> <port 1 (+) node> <port 1 (-) node>
+ <port 2 (+) node> <port 2 (-) node>
+ Z0=<value> [TD=<value>] [F=<value> [NL=<value>]]
\end{alltt}

\examples
\begin{alltt}
Tline inp inn outp outn Z0=50 TD=1us
Tline2 inp inn outp outn Z0=50 F=1meg NL=1.0
\end{alltt}

\comments

The lossless transmission line device is a two port (\texttt{A} and
\texttt{B}), bi-directional delay line. The \texttt{(+)} and
\texttt{(-)} nodes define the polarity of a positive voltage at a port.

\texttt{Z0} is the characteristic impedance. For user convenience, 
\texttt{ZO} (``Zee Oh'') is an allowed synonym for \texttt{Z0} (``Zee Zero'').

The transmission line's length is specified by either \texttt{TD} (a delay in
seconds) or by the combination of \texttt{F} and \texttt{NL} (a frequency in Hz and 
the relative wavelength at \texttt{F}). \texttt{NL} defaults to 0.25 (\texttt{F} is 
the quarter-wave frequency).  If \texttt{F} is given, the time delay is 
computed as $\frac{NL}{F}$.  

While both \texttt{TD} and \texttt{F} are optional, at least one of them must be given.
It is an instance line error if both are given.

Lead currents for the two terminals (1 and 2) of the lossless transmission device 
(e.g.,for the T device \texttt{line2}) are accessed via \texttt{I1(Tline2)} and 
\texttt{I2(Tline2)}.  The polarity conventions are that positive current flows into
the positive node of the specified terminal, and negative current flows out of the
positive node of the specified terminal.

Power for the lossless transmission line is calculated as $I_1 \cdot \Delta V_1 + 
I_2 \cdot \Delta V_2$, where the voltage drops ($\Delta V_1$ and $\Delta V_2$) are the 
voltage drops between the positive and negative terminals of each port (e.g., 
$\Delta V = (V_+ - V_-)$).  The sign conventions for the lead currents $I_1$ and $I_2$ 
were given in the previous paragraph.  This definition can be viewed as the instantaneous
sum of the power flowing into terminal 1 and the power flowing into terminal 2.
This definition for power for the lossless transmission line may differ from 
commercial simulators, such as HSPICE.  

The lossless transmission line device does not work with AC analysis
at this time.  Lossless transmission line models will need to be
replaced with lumped transmission line models (YTRANSLINE) when used
in AC analysis.  The lossless transmission line does work correctly in
harmonic balance simulation.

\end{Device}

\paragraph{Instance Parameters}
\input{T_1_Device_Instance_Params}


%%
%% Lumped Transmission Line Subsection
%%
\clearpage
\subsection{Lumped Transmission Line}
\index{device!lumped transmission line} \index{lumped transmission line}
\index{device!transmission line!lumped}
% Sandia National Laboratories is a multimission laboratory managed and
% operated by National Technology & Engineering Solutions of Sandia, LLC, a
% wholly owned subsidiary of Honeywell International Inc., for the U.S.
% Department of Energy’s National Nuclear Security Administration under
% contract DE-NA0003525.

% Copyright 2002-2024 National Technology & Engineering Solutions of Sandia,
% LLC (NTESS).


\begin{Device}

\symbol
{\includegraphics{translineSymbol}}

\device
\begin{alltt}
ytransline <name> <Input port> <Output port> testLine 
+ len=<value> lumps=<value>
\end{alltt}

\model
\begin{alltt}
.model testLine transline r=<value> l=<value> 
+ c=<value> [model parameters]
\end{alltt}

\examples
\begin{alltt}
ytransline line1 inn out  testLine len=12.0 lumps=1440
\end{alltt}

\comments
The lumped transmission line, device is a two port bi-directional device.   The specification
is patterned, loosely, from the netlist specification for the LTRA device.

\texttt{R}, \texttt{L}, and \texttt{C}  are the resistance,
inductance, and capacitance of the transmission line per unit
length, respectively. \texttt{LEN} is the total length of the transmission
line, and \texttt{LUMPS} is the number of lumped elements used to discretize the line. 
Supported configurations for this device are \texttt{RLC} and \texttt{LC}.

Unlike the LTRA device, which is based on an analytic solution, this device is 
based on assembling chains of linear R,L and C devices to approximate the 
solution to the Telegraph equations.  It is the functional equivalent of building a
transmission line in the netlist using subcircuits of linear elements.  The advantage
of using this approach is that it automates the mechanics of this process, and
thus is less prone to error.  It can be used with all analysis types, including
harmonic balance (HB).

The model is based on the assumption that the segments of the line are evenly spaced.
The number of segments is specified by the parameter \texttt{LUMPS} and the larger 
this number, the more accurate the calculation.
\end{Device}

\paragraph{Device Parameters}
\input{Transline_1_Device_Instance_Params}

\paragraph{Model Parameters}
\input{Transline_1_Device_Model_Params}


%\paragraph{References}
%%See references \cite{Roychodhury:1994} and \cite{Spice3f5-user-guide} for more information
%about the model.


%%
%% Delay device Subsection
%%
\clearpage
\subsection{Ideal Delay}
\index{device!delay device}
\index{device!ideal delay device}
% Sandia National Laboratories is a multimission laboratory managed and
% operated by National Technology & Engineering Solutions of Sandia, LLC, a
% wholly owned subsidiary of Honeywell International Inc., for the U.S.
% Department of Energy’s National Nuclear Security Administration under
% contract DE-NA0003525.

% Copyright 2002-2023 National Technology & Engineering Solutions of Sandia,
% LLC (NTESS).


An ideal delay device, operating in a manner similar to a
voltage-controlled voltage source, is provided by the YDELAY device.

\begin{Device}

\device
\begin{alltt}
YDELAY <name> <positive node> <negative node>
+ <positive control node> <negative control node>
+ TD=<time delay>
+ [EXTRAPOLATION=<true|false>] [BPENABLED=<true|false>]
+ [LINEARINTERP=<true|false>] 
\end{alltt}

\examples
\begin{alltt}
YDELAY delay1 2 0 1 0 TD=10N
R1 2 0 1
YDELAY delay1 3 0 2 0 TD=10N LINEARINTERP=true
R2 3 0 1
YDELAY delay1 4 0 3 0 TD=10N BPENABLED=FALSE
R4 4 0 1
YDELAY delay1 5 0 4 0 TD=10N EXTRAPOLATION=false
R5 5 0 1
\end{alltt}

\comments

The voltage between the positive and negative control nodes is
reproduced at the positive and negative output nodes delayed by a time
equal to the specified TD parameter.

The device is equivalent in connectivity to a voltage-controlled
voltage source --- the device puts no load on the control nodes, and
its output must be connected to a valid circuit.

Unlike the transmission line, no impedance matching is required, and
reflections due to impedance mismatch do not occur.

These devices may be chained to create outputs at different delays,
but each instance must have its output connected to a valid closed
circuit.  The examples above are chained correctly so that each of the
output nodes is delayed by 10 nanoseconds from the previous stage.

The device functions by storing a history of its input at each
accepted time point.  At each new time step, interpolation is
performed on this history to determine what the signal would have been
at a time TD in the past.  At each step, the device checks its history
to determine if the previous three saved steps include a discontinuity
in the input.  If so, the device assures that \Xyce{} will correctly
resolve the same discontinuity when it appears on the output.

With no special options specified, three-point quadratic interpolation
is used except after a discontinuity, when linear interpolation is
performed.  If \Xyce{} has advanced the time by more than TD and no
discontinuity has occured, then this interpolation is actually
extrapolation.

When \texttt{LINEARINTERP=true} is specified, the history
interpolation used is always linear interpolation.

When \texttt{EXTRAPOLATION=false}, \Xyce{} will never attempt
extrapolation when it has taken a time step larger than TD.  In this
case, the current, unconverged value of the solution is used as the
third interpolation point and the interpolation is recomputed at every
step of the nonlinear solve.

When \texttt{BPENABLED=false}, the device will not set a simulation
breakpoint to force the time integrator to stop exactly TD seconds
after a detected discontinuity on the input.  It will still force a
maximum time step on the time integrator after such a discontinuity,
and other techniques will be applied to assure the discontinuity is
resolved.  This option may result in \Xyce{} rejecting a lot more time
steps and slower simulation than when it is left at its default.

\end{Device}

\subsubsection{Delay device instance parameters}

The instance parameters for the delay device are shown in
Table~\ref{Delay_1_Device_Instance_Params}.

\input{Delay_1_Device_Instance_Params}

%%
%% U Type Digital Device
%%
\clearpage
\subsection{Behavioral Digital Devices}
\index{device!digital devices} \index{digital devices}
% Sandia National Laboratories is a multimission laboratory managed and
% operated by National Technology & Engineering Solutions of Sandia, LLC, a
% wholly owned subsidiary of Honeywell International Inc., for the U.S.
% Department of Energy’s National Nuclear Security Administration under
% contract DE-NA0003525.

% Copyright 2002-2023 National Technology & Engineering Solutions of Sandia,
% LLC (NTESS).

 
%%
%% Behavioral Digital Description Table
%%
\begin{Device}\label{U_DEVICE}

 
\device
\begin{alltt}
U<name> <type>(<num inputs>) [digital power node] 
+ [digital ground node] <input node>* <output node>* 
+ <model name> [device parameters]
\end{alltt}
 
\model
.MODEL <model name> DIG [model parameters]
 
\examples
\begin{alltt}
UMYAND AND(2) DPWR DGND in1 in2 out DMOD IC=TRUE
UTHEINV INV DPWR DGND in out DMOD
.model DMOD DIG (
+ CLO=1e-12  CHI=1e-12
+ S0RLO=5  S0RHI=5  S0TSW=5e-9
+ S0VLO=-1  S0VHI=1.8
+ S1RLO=200  S1RHI=5  S1TSW=5e-9
+ S1VLO=1  S1VHI=3
+ RLOAD=1000
+ CLOAD=1e-12
+ DELAY=20ns )
\end{alltt}

\parameters 
\begin{Parameters}

\param{type} 

Type of digital device.  Supported devices are: INV, BUF, AND, NAND, OR, NOR, XOR,
NXOR, DFF, JKFF, TFF, DLTCH and ADD.  (Note: NOT is an allowed synonym for INV, but will be
deprecated in future \Xyce{} releases.)

The following gates have a fixed number of inputs.  INV and BUF have only one
input and one output node.  XOR and NXOR have two inputs and one output.
ADD has three inputs (in1, in2, carryIn) and two outputs (sumOut
and carryOut).  DFF has four inputs (PREB, CLRB, Clock and Data) and
two outputs ($Q$ and $\bar{Q}$).  TFF has two inputs (T and CLK) and two
outputs ($Q$ and $\bar{Q}$).  The TFF uses ``positive'' (``rising'') edge clocking.
The JKFF has five inputs (PREB, CLRB, Clock, J and K) and two outputs ($Q$ and $\bar{Q}$).
The JKFF uses ``negative'' (``falling'') edge clocking.
DLTCH has four inputs (PREB, CLRB, Enable and Data) and two outputs ($Q$ and $\bar{Q}$).  

The AND, NAND, OR and NOR gates have one output but a variable number of inputs.  
There is no limit on the number of inputs for AND, NAND, OR and NOR gates, but
there must be at least two inputs.

\param{num inputs}

For AND, NAND, OR and NOR gates, with N inputs, the syntax is (N), as shown for the
MYAND example given above, where AND(2) is specified.  The inclusion of (N) is mandatory
for gates with a variable number of inputs, and both the left and right parentheses must
be used to enclose N.  

This parameter is optional, and typically omitted, for gates with a fixed number of
inputs, such as INV, BUF, XOR, NXOR, DFF, JKFF, TFF, DLTCH and ADD.  This is illustrated by the THEINV
example given above, where the device type is INV rather than INV(1).

\param{digital power node}

Dominant node to be connected to the output node(s) to establish high
output state.  This node is connected to the output by a resistor and
capacitor in parallel, whose values are set by the model.  This node must
be specified on the instance line.

\param{digital ground node}

This node serves two purposes, and must be specified on the instance line.
It is the dominant node to be connected to the output node(s) to establish
low output state.  This node is connected to the output by a resistor and 
capacitor in parallel, whose values are set by the model.  This node is also
connected to the input node by a resistor and capacitor in parallel, whose
values are set by the model.  Determination of the input state is based on the
voltage drop between the input node and this node.

\param{input nodes, output nodes}

Input and output nodes that connect to the circuit.

\param{model name}

 Name of the model defined in a .MODEL line.

\param{device parameters}

Parameter listed in Table~\ref{Digital_1_Device_Instance_Params_Udevice} may be
provided as \texttt{<parameter>=<value>} specifications as needed.  For
devices with more than one output, multiple output initial states may be
provided as Boolean values in either a comma separated list (e.g.
IC=TRUE,FALSE for a device with two outputs) or individually 
(e.g. IC1=TRUE IC2=FALSE or IC2=FALSE).  Finally, the IC specification
must use TRUE and FALSE rather than T and F.

\end{Parameters}
\end{Device}

\paragraph{Device Parameters}

%%
%% Digital Device Param Table
%%
% Sandia National Laboratories is a multimission laboratory managed and
% operated by National Technology & Engineering Solutions of Sandia, LLC, a
% wholly owned subsidiary of Honeywell International Inc., for the U.S.
% Department of Energy’s National Nuclear Security Administration under
% contract DE-NA0003525.

% Copyright 2002-2023 National Technology & Engineering Solutions of Sandia,
% LLC (NTESS).

% This table was generated by Xyce:
%   Xyce -doc Digital 1
%
\index{behavioral digital!device instance parameters}
\begin{DeviceParamTableGenerated}{Behavioral Digital Device Instance Parameters}{Digital_1_Device_Instance_Params_Udevice}
IC1 & Vector of initial values for output(s) & logical (T/F) & false \\ \hline
IC2 &  & -- & false \\ \hline
\end{DeviceParamTableGenerated}


\paragraph{Model Parameters}

%%
%% Digital Model Param Table
%%
% Sandia National Laboratories is a multimission laboratory managed and
% operated by National Technology & Engineering Solutions of Sandia, LLC, a
% wholly owned subsidiary of Honeywell International Inc., for the U.S.
% Department of Energy’s National Nuclear Security Administration under
% contract DE-NA0003525.

% Copyright 2002-2023 National Technology & Engineering Solutions of Sandia,
% LLC (NTESS).

% This table was generated by Xyce:
%   Xyce -doc Digital 1
%
\index{behavioral digital!device model parameters}
\begin{DeviceParamTableGenerated}{Behavioral Digital Device Model Parameters}{Digital_1_Device_Model_Params_Udevice}
CHI & Capacitance between output node and high reference & F & 1e-06 \\ \hline
CLO & Capacitance between output node and low reference & F & 1e-06 \\ \hline
CLOAD & Capacitance between input node and input reference & F & 1e-06 \\ \hline
DELAY & Delay time of device & s & 1e-08 \\ \hline
RLOAD & Resistance between input node and input reference & $\mathsf{\Omega}$ & 1000 \\ \hline
S0RHI & Low state resitance between output node and high reference & $\mathsf{\Omega}$ & 100 \\ \hline
S0RLO & Low state resistance between output node and low reference & $\mathsf{\Omega}$ & 100 \\ \hline
S0TSW & Switching time transition to low state & s & 1e-08 \\ \hline
S0VHI & Maximum voltage to switch to low state & V & 1.7 \\ \hline
S0VLO & Minimum voltage to switch to low state & V & -1.5 \\ \hline
S1RHI & High state resistance between output node and high reference & $\mathsf{\Omega}$ & 100 \\ \hline
S1RLO & High state resistance between output node and low reference & $\mathsf{\Omega}$ & 100 \\ \hline
S1TSW & Switching time transition to high state & s & 1e-08 \\ \hline
S1VHI & Maximum voltage to switch to high state & V & 7 \\ \hline
S1VLO & Minimum voltage to switch to high state & V & 0.9 \\ \hline
\end{DeviceParamTableGenerated}


\paragraph{Model Description}

The input interface model consists of the input node connected with a resistor and
capacitor in parallel to the digital ground node.  The values of these are: \textrmb{RLOAD}
and \textrmb{CLOAD}.  

The logical state of any input node is determined by comparing the voltage relative
to the reference to the range for the low and high state.  The range for the low
state is \textrmb{S0VLO} to \textrmb{S0VHI}.  Similarly, the range for the high state
is \textrmb{S1VLO} to \textrmb{S1VHI}.  The state of an input node will remain fixed as
long as its voltage stays within the range for its current state.  That input node will
transition to the other state only when its state goes outside the voltage range of
its current state.

The output interface model is more complex than the input model, but shares the same
basic configuration of a resistor and capacitor in parallel to simulate loading.  For
the output case, there are such parallel RC connections to two nodes, the digital
ground node and the digital power node.  Both of these nodes must be specified on
the instance line.

The capacitance to the high node is specified by \textrmb{CHI}, and the capacitance to the low
node is \textrmb{CLO}.  The resistors in parallel with these capacitors are variable, and have
values that depend on the state.  In the low state (S0), the resistance values are:
\textrmb{S0RLO} and \textrmb{S0RHI}.  In the high state (S1) ,the resistance values are: 
\textrmb{S1RLO} and \textrmb{S1RHI}.  Transition to the high state occurs exponentially over
a time of \textrmb{S1TSW}, and to the low state \textrmb{S0TSW}.

The device's delay is given by the model parameter \textrmb{DELAY}.  Any input changes
that affect the device's outputs are propagated after this delay.

As a note, the model parameters \textrmb{VREF}, \textrmb{VLO} and \textrmb{VHI} are used
by the now deprecated Y-type digital device, but are
ignored by the U device.  A warning message is emitted if any of these three parameters
are used in the model card for a U device.

Another caveat is that closely spaced input transitions to the \Xyce{} digital behavioral
models may not be accurately reflected in the output states.  In particular, input-state
changes spaced by more than \texttt{DELAY} seconds have independent effects on the output
states. However, two input-state changes (S1 and S2) that occur within \texttt{DELAY} seconds
(e.g., at time=t1 and time=t1+0.5*\texttt{DELAY}) have the effect of masking the effects
of S1 on the device's output states, and only the effects of S2 are propagated to the
device's output states.

\paragraph{DCOP Calculations for Flip-Flops and Latches}
The behavior of the digital devices during the DC Operating Point (DCOP) calculations
can be controlled via the \texttt{IC1} and \texttt{IC2} instance parameters and the
\texttt{DIGINITSTATE} device option.  See ~\ref{Options_Reference} for more details on the
syntax for device options.  Also, this section applies to the Y-Type Behavioral Digital Devices
discussed in ~\ref{YTypeDigitalDevice}.

The \texttt{IC1} instance parameter is supported for all gate types.  The \texttt{IC2}
instance parameter is supported for all gate types that have two outputs.  These instance
parameters allow the outputs of individual gates to be set to known states (either
\texttt{TRUE (1)} or \texttt{FALSE (0)}) during the DCOP calculation, irregardless of their 
input state(s).  There are two caveats.  First, the \texttt{IC1} and \texttt{IC2} settings 
at a given gate will override the global effects of the \texttt{DIGINITSTATE} option, 
discussed below, at that gate.  Second, \texttt{IC1} and \texttt{IC2} do not support the
\texttt{X}, or ``undetermined'', state discussed below.

The \texttt{DIGINITSTATE} option only applies to the DLTCH, DFF, JKFF and TFF devices.  It was added
for improved compatibility with PSpice.  It sets the initial state of all flip-flops and 
latches in the circuit: 0=clear, 1=set, 2=\texttt{X}.  At present, the use of the
\texttt{DIGINITSTATE} option during the DCOP is the only place that \Xyce{} supports the
\texttt{X}, or ``undetermined'', state.  The \texttt{X} state is modeled in \Xyce{} by having the
DLTCH, DFF, JKFF and TFF outputs simultaneously ``pulled-up'' and ``pulled-down''.  That approach typically 
produces an output level, for the \texttt{X} state, that is approximately halfway between the 
voltage levels for \texttt{TRUE} and \texttt{FALSE} (e.g., halfway between \texttt{V\_HI} and 
\texttt{V\_LO}). As mentioned above, the \texttt{IC1} and
\texttt{IC2} instance parameters take precedence at a given gate.

\Xyce{} also supports a default \texttt{DIGINITSTATE}, whose value is 3.  For this default value,
for the DFF, JKFF, TFF and DLTCH devices, \Xyce{} enforces $Q$ and $\bar{Q}$ being different at DCOP, if 
both \texttt{PREB} and \texttt{CLRB} are \texttt{TRUE} . The behavior of the DFF, JKFF and DLTCH 
devices at the DCOP for \texttt{DIGINITSTATE=3} is shown in Tables ~\ref{dffTruthTable},  ~\ref{jkffTruthTable}
and ~\ref{dltchTruthTable}.  In these three tables, the $X$ state denotes the ``Don't Care'' 
condition, where the input state can be 0, 1 or the ``undetermined'' state.
The first row in each truth-table (annotated with $*$) is ``unstable'', and will change to 
a state with $Q$ and $\bar{Q}$ being different once both \texttt{PREB} and \texttt{CLRB} are not 
both in the \texttt{FALSE} state.

The behavior of the TFF device at the DCOP, for the default \texttt{DIGINITSTATE} of 3, is
simpler, and is not shown as a table.  The design decision was to have $Q$ and $\bar{Q}$ be
different, with the $Q$ value equal to the state of the $T$ input.

%%
%% DFF Truth Table for DIGINITSTATE=3
%%
\LTXtable{\textwidth}{dffTruthTableTbl}

%%
%% DLTCH Truth Table for DIGINITSTATE=3
%%
\LTXtable{\textwidth}{dltchTruthTableTbl}

%%
%% JKFF Truth Table for DIGINITSTATE=3
%%
\LTXtable{\textwidth}{jkffTruthTableTbl}


%%
%% Behavioral Digital Devices - Y Type now deprecated
%%
\clearpage
\subsection{Y-Type Behavioral Digital Devices (Deprecated)}
\index{device!digital devices, Y type} \index{Y type digital devices}
% Sandia National Laboratories is a multimission laboratory managed and
% operated by National Technology & Engineering Solutions of Sandia, LLC, a
% wholly owned subsidiary of Honeywell International Inc., for the U.S.
% Department of Energy’s National Nuclear Security Administration under
% contract DE-NA0003525.

% Copyright 2002-2024 National Technology & Engineering Solutions of Sandia,
% LLC (NTESS).

 
%%
%% Behavioral Digital Description Table
%%
\begin{Device}\label{YTypeDigitalDevice}

\device
\begin{alltt}
Y<type> <name> [low output node] [high output node]
+ [input reference node] <input node>* <output node>*
+  <model name> [device parameters]
\end{alltt}

\model
.MODEL <model name> DIG [model parameters]

\examples
\begin{alltt}
YAND MYAND in1 in2 out DMOD IC=TRUE
YNOT THENOT in out DMOD
YNOR ANOR2 vlo vhi vref in1 in2 out DDEF
.model DMOD DIG (
+ CLO=1e-12  CHI=1e-12
+ S0RLO=5  S0RHI=5  S0TSW=5e-9
+ S0VLO=-1  S0VHI=1.8
+ S1RLO=200  S1RHI=5  S1TSW=5e-9
+ S1VLO=1  S1VHI=3
+ RLOAD=1000
+ CLOAD=1e-12
+ VREF=0 VLO=0 VHI=3
+ DELAY=20ns )
.MODEL DDEF DIG
\end{alltt}

\parameters
\begin{Parameters}

\param{type}
Type of digital device.  Supported devices are: NOT, BUF, AND, NAND, OR, NOR, XOR,
NXOR, DFF, JKFF, TFF, DLTCH and ADD.  (Note: INV is now the preferred synonym for NOT.  
The NOT device type will be deprecated in future \Xyce{} releases.)  For Y-type digital
devices, all devices have two input nodes and one output node, except for NOT, DFF and
ADD.  NOT has one input and one output.  ADD has three inputs (in1, in2, carryIn) and 
two outputs (sumOut and carryOut).  DFF has four inputs (PREB, CLRB, Clock and
Data) and two outputs ($Q$ and $\bar{Q}$). TFF has two inputs (T and Clock) and two
outputs ($Q$ and $\bar{Q}$).  The TFF uses ``positive'' (``rising'') edge clocking.
The JKFF has five inputs (PREB, CLRB, Clock, J and K) and two outputs ($Q$ and $\bar{Q}$).
The JKFF uses ``negative'' (``falling'') edge clocking.
DLTCH has four inputs (PREB, CLRB, Enable and Data) and two outputs ($Q$ and $\bar{Q}$). 

\param{name} 
Name of the device instance.  This must be present, and when combined
with the \texttt{Y<type>}, must be unique in the netlist.  In the
examples, MYAND, THENOT and ANOR2 have been used as names for the three
devices.

\param{low output node}

Dominant node to be connected to the output node(s) to establish low
output state.  This node is connected to the output by a resistor and
capacitor in parallel, whose values are set by the model.  If specified
by the model, this node must be omitted from the instance line and a
fixed voltage \textrmb{VLO} is used instead.

\param{high output node}

Dominant node to be connected to the output node(s) to establish high
output state.  This node is connected to the output by a resistor and
capacitor in parallel, whose values are set by the model.  If specified
by the model, this node must be omitted from the instance line and a fixed
voltage \textrmb{VHI} is used instead.

\param{input reference node}

This node is connected to the input node by a resistor and capacitor in
parallel, whose values are set by the model.  Determination if the input
state is based on the voltge drop between the input node and this node.
If specified by the model, this node must be omitted from the instance line and
a fixed voltage \textrmb{VREF} is used instead.

\param{input nodes, output nodes}

Nodes that connect to the circuit.

\param{model name}

Name of the model defined in a .MODEL line.

\param{device parameters}

Parameter listed in Table~\ref{Digital_1_Device_Instance_Params} may be
provided as \texttt{<parameter>=<value>} specifications as needed.  For
devices with more than one output, multiple output initial states may be
provided as Boolean values in either a comma separated list (e.g.
IC=TRUE,FALSE for a device with two outputs) or individually 
(e.g. IC1=TRUE IC2=FALSE or IC2=FALSE).  Finally, the IC specification
must use TRUE and FALSE rather than T and F.


\end{Parameters}
\end{Device}

\paragraph{Device Parameters}

%%
%% Digital Device Param Table
%%
\input{Digital_1_Device_Instance_Params}

\paragraph{Model Parameters}

%%
%% Digital Model Param Table
%%
\input{Digital_1_Device_Model_Params}

\paragraph{Model Description}

The input interface model consists of the input node connected with a resistor and
capacitor in parallel to the digital ground node.  The values of these are: \textrmb{RLOAD}
and \textrmb{CLOAD}.  

The logical state of any input node is determined by comparing the voltage relative
to the reference to the range for the low and high state.  The range for the low
state is \textrmb{S0VLO} to \textrmb{S0VHI}.  Similarly, the range for the high state
is \textrmb{S1VLO} to \textrmb{S1VHI}.  The state of an input node will remain fixed as
long as its voltage stays within the voltage range for its current state.  That input node
will transition to the other state only when its state goes outside the range of its
current state.

The output interface model is more complex than the input model, but shares the same
basic configuration of a resistor and capacitor in parallel to simulate loading.  For
the output case, there are such connections to two nodes, the digital ground node and the
digital power node.  Both of these nodes must be specified on the instance line.

The capacitance to the high node is specified by \textrmb{CHI}, and the capacitance to the low
node is \textrmb{CLO}.  The resistors in parallel with these capacitors are variable, and have
values that depend on the state.  In the low state (S0), the resistance values are:
\textrmb{S0RLO} and \textrmb{S0RHI}.  In the high state (S1) ,the resistance values are: 
\textrmb{S1RLO} and \textrmb{S1RHI}.  Transition to the high state occurs exponentially over
a time of \textrmb{S1TSW}, and to the low state \textrmb{S0TSW}.

The device's delay is given by the model parameter \textrmb{DELAY}.  Any input changes
that affect the device's outputs are propagated after this delay.

Another caveat is that closely spaced input transitions to the \Xyce{} digital behavioral
models may not be accurately reflected in the output states.  In particular, input-state
changes spaced by more than \texttt{DELAY} seconds have independent effects on the output
states. However, two input-state changes (S1 and S2) that occur within \texttt{DELAY} seconds
(e.g., at time=t1 and time=t1+0.5*\texttt{DELAY}) have the effect of masking the effects
of S1 on the device's output states, and only the effects of S2 are propagated to the
device's output states.

\paragraph{DCOP Calculations for Flip-Flops and Latches}
The behavior of the digital devices during the DC Operating Point (DCOP) calculations
can be controlled via the \texttt{IC1} and \texttt{IC2} instance parameters and the
\texttt{DIGINITSTATE} device option.  See ~\ref{U_DEVICE} and ~\ref{Options_Reference} for 
more details on these instance parameters and device option.  

\paragraph{Converting Y-Type Digital Devices to U-Type Digital Devices}
\Xyce{} is migrating the digital behavioral devices to U devices.  The goal is increased
compatibility with PSpice netlists.  This subsection gives four examples of how to 
convert an existing \Xyce{} netlist using Y-type digital devices to the corresponding U device
syntaxes.  The conversion process depends on whether the device has a fixed number of inputs
or a variable number of inputs. In all cases, the the model parameters \textrmb{VREF},
\textrmb{VLO} and \textrmb{VHI} should be omitted from the U device model card.  For
U devices, the nodes \textrmb{vlo} and \textrmb{vhi} are always specified on the 
instance line.

Example 1: Fixed number of inputs, Y-device model card contains \textrmb{VREF},
\textrmb{VLO} and \textrmb{VHI}.  Assume \textrmb{VREF}=\textrmb{VLO}.

\begin{alltt}
YNOT THENOT in out DMOD
.model DMOD DIG (
+ CLO=1e-12  CHI=1e-12
+ S0RLO=5  S0RHI=5  S0TSW=5e-9
+ S0VLO=-1  S0VHI=1.8
+ S1RLO=200  S1RHI=5  S1TSW=5e-9
+ S1VLO=1  S1VHI=3
+ RLOAD=1000
+ CLOAD=1e-12
+ VREF=0 VLO=0 VHI=3
+ DELAY=20ns )

* Digital power node.  Assume digital ground node = GND
V1 DPWR 0 3V 
UTHENOT INV DPWR 0 in out DMOD1
.model DMOD1 DIG (
+ CLO=1e-12  CHI=1e-12
+ S0RLO=5  S0RHI=5  S0TSW=5e-9
+ S0VLO=-1  S0VHI=1.8
+ S1RLO=200  S1RHI=5  S1TSW=5e-9
+ S1VLO=1  S1VHI=3
+ RLOAD=1000
+ CLOAD=1e-12
+ DELAY=20ns )
\end{alltt}

Example 2: Fixed number of inputs, Y-device instance line contains \textrmb{vlo},
\textrmb{vhi} and \textrmb{vref}.  Assume \textrmb{vref}=\textrmb{vlo}.

\begin{alltt}
YNOT THENOT vlo vhi vref in out DMOD1
UTHENOT INV vhi vlo in out DMOD1
\end{alltt}

Example 3: Variable number of inputs, Y-device model card contains \textrmb{VREF},
\textrmb{VLO} and \textrmb{VHI}.  Assume \textrmb{VREF}=\textrmb{VLO}.

\begin{alltt}
YAND MYAND in1 in2 out DMOD
UMYAND AND(2) DPWR 0 in1 in2 out DMOD1
\end{alltt}

Example 4: Variable number of inputs, Y-device instance line contains \textrmb{vlo},
\textrmb{vhi} and \textrmb{vref}.  Assume \textrmb{vref}=\textrmb{vlo}.

\begin{alltt}
YAND MYAND vlo vhi vref in1 in2 out DMOD1
UMYAND AND(2) vhi vlo in1 in2 out DMOD1
\end{alltt}




%%
%% YACC device Subsection
%%
\clearpage
\subsection{Accelerated mass}
\index{accelerated mass device}
% Sandia National Laboratories is a multimission laboratory managed and
% operated by National Technology & Engineering Solutions of Sandia, LLC, a
% wholly owned subsidiary of Honeywell International Inc., for the U.S.
% Department of Energy’s National Nuclear Security Administration under
% contract DE-NA0003525.

% Copyright 2002-2024 National Technology & Engineering Solutions of Sandia,
% LLC (NTESS).


Simulation of electromechanical devices or magnetically driven machines may
require that \Xyce{} simulate the movement of an accelerated mass, that is, to
solve the second order initial value problem
\begin{eqnarray*}
\frac{d^2x}{dt} &= a(t) \\
x(0) &= x_0 \\
\dot{x}_0 &= v_0 \\
\end{eqnarray*}
where $x$ is the position of the object, $\dot{x}$ its velocity, and $a(t)$ the
acceleration.  In \Xyce{}, this simulation capability is provided by the
accelerated mass device.

\begin{Device}

\device
\begin{alltt}
YACC <name> <acceleration node> <velocity node> <position node>
+ [v0=<initial velocity>] [x0=<initial position>]
\end{alltt}

\examples
\begin{alltt}
* Simulate a projectile thrown upward against gravity
V1 acc 0 -9.8
R1 acc 0 1
YACC acc1 acc vel pos v0=10 x0=0
.print tran v(pos)
.tran 1u 10s
.end

* Simulate a damped, forced harmonic oscillator
* assuming K, c, mass, amplitude and frequency
* are defined in .param statements
B1 acc 0 V=\{(-K * v(pos) - c*v(vel))/mass
+            + amplitude*sin(frequency*TIME)\}
R1 acc 0 1
YACC acc2 acc vel pos v0=0 x0=0.4
.print tran v(pos)
.tran 1u 10s
.end
\end{alltt}

\comments

When used as in the examples, \Xyce{} will emit warning messages about the
\texttt{pos} and \texttt{vel} nodes not having a DC path to ground.  This is
normal and should be ignored. The position and velocity nodes should not be
connected to any real circuit elements.  Their values may, however, be used in
behavioral sources; this is done in the second example.

\end{Device}


%%
%% Power Grid device Subsection
\clearpage
\subsection{Power Grid}
\index{power grid}
% Sandia National Laboratories is a multimission laboratory managed and
% operated by National Technology & Engineering Solutions of Sandia, LLC, a
% wholly owned subsidiary of Honeywell International Inc., for the U.S.
% Department of Energy’s National Nuclear Security Administration under
% contract DE-NA0003525.

% Copyright 2002-2024 National Technology & Engineering Solutions of Sandia,
% LLC (NTESS).

 
The Power Grid devices are a family of device models that can be used to model  
steady-state power flow in electric power grids.  They include device models for 
branches, bus shunts, transformers and generator buses.  

Power flow in electric power grids can be modeled as a complex-valued voltage-current
problem with standard admittance-matrix techinques. This approach solves the system of
equations $I=YV$, and is termed \texttt{IV} format in this document.  However, it is 
more typically modeled as a power-flow problem that solves the system of equations 
$S = P + jQ = VI^{*}$, where $S$ is the complex power flow, $V$ and $I$ are complex-valued 
quantities, and $I^{*}$ is the complex conjugate of $I$.  The complex power flow can
then be solved in either rectangular or polar coordinates.  These two solution formats are
termed PQ Rectangular (aka, \texttt{PQR} format) and PQ Polar (aka, \texttt{PQP} format) in 
this document.  The variables for each solution format are described in more detail in the
device descriptions given below. 

In all three formulations,an Equivalent Real Formulation (ERF) \cite{Munankarmy} must be used
for compatibility with the existing solver libraries in \Xyce{}.  More details on these 
equations are given below after the individual device descriptions.
 
%%
%% PowerGridBranch Description 
%%
\paragraph{PowerGridBranch}

\begin{Device}\label{PowerGridBranch}

\device
\begin{alltt}
Y<type> <name> <input node1> <output node1> 
+ <input node2> <output node2> [device parameters] 
\end{alltt}
  
\examples
\begin{alltt}
YPowerGridBranch pg1_2 VR1 VR2 VI1 VI2 AT=IV R=0.05 B=0.1 X=0.05
YPGBR pg1_2a VR1 VR2 VI1 VI2 AT=IV R=0.05 B=0.1 X=0.05
YPowerGridBranch pg1_2b VR1 VR2 VI1 VI2 AT=PQR R=0.05 B=0.1 X=0.05
YPGBR pg1_2c VR1 VR2 VI1 VI2 AT=PQR R=0.05 B=0.1 X=0.05
YPowerGridBranch pg1_2d Th1 Th2 VM1 VM2 AT=PQP R=0.05 B=0.1 X=0.05
YPGBR pg1_2e Th1 Th2 VM1 VM2 AT=PQP R=0.05 B=0.1 X=0.05
\end{alltt}

\parameters 
\begin{Parameters}
\param{type}
The device type has a verbose (\texttt{PowerBranchBranch}) and a shortened
(\texttt{PGBR}) form.  Their usage may be mixed within a netlist.

\param{name}
Name of the device instance. This must be present, and unique amongst the 
\texttt{PowerGridBranch} devices in the netlist.

\param{input node}
There are two input nodes, \texttt{<input node1>} and \texttt{<input node2>}, 
whose definitions depend on the AnalysisType (\texttt{AT}) specified.  Both nodes
must be specified.  This device can be viewed as a generalized 4-port resistor, using
the Equivalent Real Form (ERF) described below in the equation subsections. For 
\texttt{IV} and \texttt{PQR} formats, \texttt{<input node1>} is the real part 
(\texttt{VR}) of the voltage at terminal 1 while \texttt{<input node2>} is the 
imaginary part (\texttt{VI}) of the voltage at terminal 1.  
For \texttt{PQP} format, \texttt{<input node1>} is the angle ($\Theta$ or \texttt{Th}) of the voltage 
at terminal 1 while \texttt{<input node2>} is the magnitude (\texttt{VM} or $|V|$) of the 
voltage at terminal 1.  Finally, by analogy to other \Xyce{} devices, node 1 can be 
considered as the positive terminal for this device, while node 2 is the negative
terminal.

\param{output node}
There are two output nodes, \texttt{<output node1>} and \texttt{<output node2>}, 
whose definitions depend on the AnalysisType (\texttt{AT}) specified.  Both nodes
must be specified.  This device can be viewed as a generalized 4-port resistor, 
using the ERF described below in the equation subsections. For \texttt{IV} 
and \texttt{PQR} formats, \texttt{<output node1>} is the real part (\texttt{VR}) of 
the voltage at terminal 2 while \texttt{<output node2>} is the imaginary part 
(\texttt{VI}) of the voltage at terminal 2.  
For \texttt{PQP} format, \texttt{<output node1>} is the angle ($\Theta$ or \texttt{Th}) of the voltage 
at terminal 2 while \texttt{<output node2>} is the magnitude (\texttt{VM} or $|V|$) of the 
voltage at terminal 2.  Finally, by analogy to other \Xyce{} devices, node 2 can be 
considered as the negative terminal for this device, while node 1 is the positive
terminal.

\param{AT}
This device supports all three analysis types (\texttt{AT}), namely \texttt{IV}, 
\texttt{PQR} and \texttt{PQP}.  The equations for these analysis types are described 
below.  All power grid devices, of all types, in a \Xyce{} netlist must use the 
same analysis type.  This constraint is not checked during netlist parsing.  
Violation of this constraint may cause unpredictable results.

\param{B}
Branch susceptance, given in per unit.  As discussed in the Equation section below, 
the susceptance value given on the branch description lines in 
IEEE Common Data Format (CDF) files is split equally between terminals 1 and 2.

\param{R} 
Branch resistance, given in per unit.

\param{X}
Branch reactance, given in per unit.

\end{Parameters}
\end{Device}

\paragraph{PowerGridBranch Device Parameters}

%%
%% PowerGridBusBranch Instance Parameter Table
%%
\input{PowerGridBranch_1_Device_Instance_Params}


%%
%% PowerGridBusShunt Description
%%
\paragraph{PowerGridBusShunt}

\begin{Device}\label{PowerGridBusShunt}

\device
\begin{alltt}
Y<type> <name> <input node1> <output node1> 
+ <input node2> <output node2 [device parameters] 
\end{alltt}
  
\examples
\begin{alltt}
YPowerGridBusShunt pg1_2 VR1 VR2 VI1 VI2 AT=IV R=0.05 B=0.1 X=0.05
YPGBS pg1_2a VR1 VR2 VI1 VI2 AT=IV R=0.05 B=0.1 X=0.05
YPowerGridBusShunt pg1_2b VR1 VR2 VI1 VI2 AT=PQR R=0.05 B=0.1 X=0.05
YPGBS pg1_2c VR1 VR2 VI1 VI2 AT=PQR R=0.05 B=0.1 X=0.05
YPowerGridBusShunt pg1_2d Th1 Th2 VM1 VM2 AT=PQP R=0.05 B=0.1 X=0.05
YPGBS pg1_2e Th1 Th2 VM1 VM2 AT=PQP R=0.05 B=0.1 X=0.0
\end{alltt}

\parameters 
\begin{Parameters}
\param{type}
The device type has a verbose (\texttt{PowerGridBusShunt}) and a shortened
(\texttt{PGBS}) form.  Their usage may be mixed within a netlist.

\param{name}
Name of the device instance.  This must be present, and unique amongst the 
\texttt{PowerGridBusShunt} devices in the netlist.

\param{input node}
There are two input nodes, \texttt{<input node1>} and \texttt{<input node2>}, 
whose definitions depend on the AnalysisType (\texttt{AT}) specified.  Both nodes
must be specified.  This device can be viewed as a generalized 4-port resistor, using
the Equivalent Real Form (ERF) described below in the equation subsections. For 
\texttt{IV} and \texttt{PQR} formats, \texttt{<input node1>} is the real part 
(\texttt{VR}) of the voltage at terminal 1 while \texttt{<input node2>} is the 
imaginary part (\texttt{VI}) of the voltage at terminal 1.  
For \texttt{PQP} format, \texttt{<input node1>} is the angle ($\Theta$ or \texttt{Th}) of the voltage 
at terminal 1 while \texttt{<input node2>} is the magnitude (\texttt{VM} or $|V|$) of the 
voltage at terminal 1.  Finally, by analogy to other \Xyce{} devices, node 1 can be 
considered as the positive terminal for this device, while node 2 is the negative
terminal.

\param{output node}
There are two output nodes, \texttt{<output node1>} and \texttt{<output node2>}, 
whose definitions depend on the AnalysisType (\texttt{AT}) specified.  Both nodes
must be specified.  This device can be viewed as a generalized 4-port resistor, 
using the ERF described below in the equation subsections. For \texttt{IV} 
and \texttt{PQR} formats, \texttt{<output node1>} is the real part (\texttt{VR}) of 
the voltage at terminal 2 while \texttt{<output node2>} is the imaginary part 
(\texttt{VI}) of the voltage at terminal 2.  
For \texttt{PQP} format, \texttt{<output node1>} is the angle ($\Theta$ or \texttt{Th}) of the voltage 
at terminal 2 while \texttt{<output node2>} is the magnitude (\texttt{VM} or $|V|$) of the 
voltage at terminal 2.  Finally, by analogy to other \Xyce{} devices, node 2 can be 
considered as the negative terminal for this device, while node 1 is the positive
terminal.
  
\param{AT}
This device supports all three analysis types (\texttt{AT}), namely \texttt{IV}, 
\texttt{PQR} and \texttt{PQP}.  The equations for these analysis types are described 
below.  All power grid devices, of all types, in a \Xyce{} netlist must use the same
analysis type.  This constraint is not checked during netlist parsing.  Violation of 
this constraint may cause unpredictable results.

\param{B}
Shunt susceptance, given in per unit.

\param{G}
Shunt conductance, given in per unit.

\end{Parameters}
\end{Device}

\paragraph{Bus Shunt Device Parameters}

%%
%% PowerGridBusShunt Table
%%
\input{PowerGridBusShunt_1_Device_Instance_Params}


%%
%% PowerGridTransformer Description
%%
\paragraph{PowerGridTransformer}

\begin{Device}\label{PowerGridTransformer}

\device
\begin{alltt}
Y<type> <name> <input node1> <output node1> 
+ <input node2> <output node2> [control node] [device parameters] 
\end{alltt}
  
\examples
\begin{alltt}
YPowerGridTransformer pg1_2 VR1 VR2 VI1 VI2 AT=IV R=0.05 X=0.05 
+ TR=0.9 PS=0.1
YPGTR pg1_2a VR1 VR2 VI1 VI2 AT=IV R=0.05 X=0.05 TR=0.9 PS=\{18*PI/180\}
YPowerGridTransformer pg1_2b VR1 VR2 VI1 VI2 AT=PQR R=0.05 X=0.05 
+ TR=0.9 PS=0.1
YPGTR pg1_2c VR1 VR2 VI1 VI2 AT=PQR R=0.05 B=0.1 X=0.05 TR=0.9 PS=0.1
YPowerGridTransformer pg1_2d Th1 Th2 VM1 VM2 AT=PQP 
+ R=0.05 X=0.05 PS=\{18*PI/180\}
YPGTR pg1_2e Th1 Th2 VM1 VM2 AT=PQP R=0.05 X=0.0 TR=0.9 PS=0.1
YPGTR pg1_2f Th1 Th2 VM1 VM2 N AT=PQP R=0.05 X=0.0 TT=VT PS=0.1
YPGTR pg1_2g Th1 Th2 VM1 VM2 Phi AT=PQP R=0.05 X=0.0 TT=PS TR=0.9
\end{alltt}

\parameters 
\begin{Parameters}
\param{type}
The device type has a verbose (\texttt{PowerGridTransformer}) and a shortened
(\texttt{PGTR}) form.  Their usage may be mixed within a netlist.

\param{name}
Name of the device instance.  This must be present, and unique amongst the 
\texttt{PowerGridTransformer} devices in the netlist.

\param{input node}
There are two input nodes, \texttt{<input node1>} and \texttt{<input node2>}, 
whose definitions depend on the AnalysisType (\texttt{AT}) specified.  Both nodes
must be specified.  This device can be viewed as a generalized 4-port resistor, using
the Equivalent Real Form (ERF) described below in the equation subsections. For 
\texttt{IV} and \texttt{PQR} formats, \texttt{<input node1>} is the real part 
(\texttt{VR}) of the voltage at terminal 1 while \texttt{<input node2>} is the 
imaginary part (\texttt{VI}) of the voltage at terminal 1.  
For \texttt{PQP} format, \texttt{<input node1>} is the angle ($\Theta$ or \texttt{Th}) of the voltage 
at terminal 1 while \texttt{<input node2>} is the magnitude (\texttt{VM} or $|V|$) of the 
voltage at terminal 1.  Finally, by analogy to other \Xyce{} devices, node 1 can be 
considered as the positive terminal for this device, while node 2 is the negative
terminal.

\param{output node}
There are two output nodes, \texttt{<output node1>} and \texttt{<output node2>}, 
whose definitions depend on the AnalysisType (\texttt{AT}) specified.  Both nodes
must be specified.  This device can be viewed as a generalized 4-port resistor, 
using the ERF described below in the equation subsections. For \texttt{IV} 
and \texttt{PQR} formats, \texttt{<output node1>} is the real part (\texttt{VR}) of 
the voltage at terminal 2 while \texttt{<output node2>} is the imaginary part 
(\texttt{VI}) of the voltage at terminal 2.  
For \texttt{PQP} format, \texttt{<output node1>} is the angle ($\Theta$ or \texttt{Th}) of the voltage 
at terminal 2 while \texttt{<output node2>} is the magnitude (\texttt{VM} or $|V|$) of the 
voltage at terminal 2.  Finally, by analogy to other \Xyce{} devices, node 2 can be 
considered as the negative terminal for this device, while node 1 is the positive
terminal.

\param{control input}
This is an optional node.  However, it must be specified on the instance line if the 
transformer type (\texttt{TT}) is set to either 2 or 3.  It does not exist, and must 
not be specified on the instance line, for the default of \texttt{TT}=1.  The use of 
the \texttt{control input} node is covered under the definition of the \texttt{TT} 
instance parameter.

\param{AT}
This device supports all three analysis types (\texttt{AT}), namely \texttt{IV}, 
\texttt{PQR} and \texttt{PQP}.  The equations for these analysis types are described 
below.  All power grid devices, of all types, in a \Xyce{} netlist must use the same 
analysis type.  This constraint is not checked during netlist parsing.  Violation of 
this constraint may cause unpredictable results.

\param{PS}
Phase shift given in radians.  As illustrated above, \texttt{PS=\{18*PI/180\}} is a
convenient syntax for converting between decimal degrees and radians on a \Xyce{}
instance line.  This instance parameter is ignored if \texttt{TT}=3, since the
phase shift is set by the optional \texttt{control node} in that case.

\param{R}
Resistance, given in per unit.

\param{TR}
Turns ratio, given in per unit.  This instance parameter is ignored if \texttt{TT}=2,
since this value is set by the optional \texttt{control node} in that case..

\param{X}
Reactance, given in per unit.

\param{TT}
This is the ``Transformer Type''. It allows the user to implement tap-changing
or phase-shifting transformers, by attaching an appropriate control-circuit to the 
\texttt{control input} node.  The allowed values for \texttt{TT} are \texttt{FT}, 
\texttt{VT} or \texttt{PS}, with default value of \texttt{FT}.  Any other values
will cause a netlist parsing error.  A transformer type of \texttt{FT} has a 
fixed turns-ratio, and is a four-terminal device with
two input nodes (\texttt{<input node1>} and \texttt{<input node2>}) and two output
nodes (\texttt{<output node1>} and \texttt{<output node2>}). Let the effective complex 
turns ratio be $r = m + jp = n*(cos(\phi) + j*sin(\phi))$.  The transformer type of
\texttt{VT} exposes the $n$ variable as the \texttt{control input} node, and hence can
operate with a variable turns-ratio. The  transformer type of \texttt{PS} exposes the 
$\phi$ variable as the \texttt{control input} node, and hence can act as a phase shifter. 
The instantaneous value of $n$ (or $\phi$) can be set to the voltage applied to the 
\texttt{control input} node.  There will be no current draw into (or out of) the
\texttt{control input} node.  This device model does not yet support simultaneously
varying both $n$ and $\phi$.
\end{Parameters}
\end{Device}

\paragraph{Transformer Device Parameters}

%%
%% PowerGridTransformer Device Instance Parameter Table
%%
\input{PowerGridTransformer_1_Device_Instance_Params}


%%
%% PowerGridGenBus Description
%%
\paragraph{PowerGridGenBus}

\begin{Device}\label{PowerGridGenBus}

\device
\begin{alltt}
Y<type> <name> <input node1> <output node1> 
+ <input node2> <output node2> [device parameters] 
\end{alltt}
  
\examples
\begin{alltt}
YPowerGridGenBus GenBus1 Th1 0 VM1 0 AT=PQP VM=1.045 P=0.4
YPGGB GenBus2 Th2 GND VM2 GND AT=PQP VM=1.045 P=0.4
\end{alltt}

\parameters 
\begin{Parameters}
\param{type}
The device type has a verbose (\texttt{PowerGridGenBus}) and a shortened
(\texttt{PGGB}) form.  Their usage may be mixed within a netlist.

\param{name}
Name of the device instance.  This must be present, and unique amongst the 
\texttt{PowerGridGenBus} devices in the netlist.

\param{input node}
There are two input nodes, \texttt{<input node1>} and \texttt{<input node2>}, 
whose definitions depend on the AnalysisType (\texttt{AT}) specified.  Both nodes
must be specified.  This device can be viewed as a generalized 4-port resistor, using
the Equivalent Real Form (ERF) described below in the equation subsections. For 
\texttt{IV} and \texttt{PQR} formats, \texttt{<input node1>} is the real part 
(\texttt{VR}) of the voltage at terminal 1 while \texttt{<input node2>} is the 
imaginary part (\texttt{VI}) of the voltage at terminal 1.  
For \texttt{PQP} format, \texttt{<input node1>} is the angle ($\Theta$ or \texttt{Th}) of the voltage 
at terminal 1 while \texttt{<input node2>} is the magnitude (\texttt{VM} or $|V|$) of the 
voltage at terminal 1.  Finally, by analogy to other \Xyce{} devices, node 1 can be 
considered as the positive terminal for this device, while node 2 is the negative
terminal.

\param{output node}
There are two output nodes, \texttt{<output node1>} and \texttt{<output node2>}, 
whose definitions depend on the AnalysisType (\texttt{AT}) specified.  Both nodes
must be specified.  This device can be viewed as a generalized 4-port resistor, 
using the ERF described below in the equation subsections. For \texttt{IV} 
and \texttt{PQR} formats, \texttt{<output node1>} is the real part (\texttt{VR}) of 
the voltage at terminal 2 while \texttt{<output node2>} is the imaginary part 
(\texttt{VI}) of the voltage at terminal 2.  
For \texttt{PQP} format, \texttt{<output node1>} is the angle ($\Theta$ or \texttt{Th}) of the voltage 
at terminal 2 while \texttt{<output node2>} is the magnitude (\texttt{VM} or $|V|$) of the 
voltage at terminal 2.  Finally, by analogy to other \Xyce{} devices, node 2 can be 
considered as the negative terminal for this device, while node 1 is the positive
terminal.

\param{AT}
This device currently only supports the \texttt{PQP} analysis type (\texttt{AT}). 
The equations for the  \texttt{PQP} analysis type are described below.  All power grid
devices, of all types, in a \Xyce{} netlist must use the same analysis type.  This 
constraint is not checked during netlist parsing.  Violation of this constraint may 
cause unpredictable results.

\param{P} 
Generator Output Power, given in per unit.  As noted below, positive real power
(\texttt{P}) and positive reactive power (\texttt{Q}) flow out of the positive 
(\texttt{<input node1>} and \texttt{<input node2>}) terminals into the power grid.
This is opposite from the normal convention for voltage and current sources
in \Xyce{} and SPICE. 

\param{QLED}
This is the Q-Limit Enforcement Delay.  It is only used if either \texttt{QMAX} or
\texttt{QMIN} is specified.  The Q-Limits are not enforced for the first \texttt{QLED}
Newton iterations of the DC Operating Point (DCOP) calculation.  This may be useful if
a given generator bus has, for example, a very small value of \texttt{QMIN} ~\cite{Milano}.
If \texttt{QMAX} or \texttt{QMIN} is specified and \texttt{QLED} is omitted then the default
\texttt{QLED} value of 0 is used.

\param{QMAX}
The upper limit on the reactive power (\texttt{Q}) flow into the power grid, given in per unit.
If this parameter is omitted on the instance line then no upper limit on the reactive power flow
is enforced.  It is recommended that either both \texttt{QMAX} and \texttt{QMIN} be
specified or that both be omitted.

\param{QMIN}
The lower limit on the reactive power (\texttt{Q}) flow into the power grid, given in per unit. 
If this parameter is omitted on the instance line then no lower limit on the reactive power flow
is enforced.  It is recommended that either both \texttt{QMAX} and \texttt{QMIN} be
specified or that both be omitted. 

\param{VM}
Fixed voltage magnitude, given in per unit.

\end{Parameters}
\end{Device}

\paragraph{Generator Bus Device Parameters}

%%
%% PowerGridGenBus Table
%%
\input{PowerGridGenBus_1_Device_Instance_Params}

\paragraph{Branch Current and Power Accessors}
This version of the Power Grid devices does not support the branch current accessor, \texttt{I()}, 
or the power accessors, \texttt{P()} or \texttt{W()}.

\paragraph{Compatibility with .STEP}
\texttt{.STEP} should work with all of the instance parameters for the power grid devices. 
The two exceptions are the Analysis Type (\texttt{AT}) for all of the power grid devices 
and the Transformer Type (\texttt{TT}) for the Transformer device.  Those two parameters
must be constant for all steps. 

\paragraph{Model Limitations and Caveats}
The following features are not supported by this release of the Power Grid device models.
\begin{XyceItemize}
  \item The Generator Bus device model only supports the PQ Polar format. So, reactive 
     power (\texttt{QMAX} and \texttt{QMIN}) limits in the Generator Bus device model are
     also only supported for that format.
  \item Magnetizing susceptance for transformers.
  \item Certain instance parameters, or combinations of instance parameters, will cause
     errors during netlist parsing.  In particular, either \texttt{B}, \texttt{R} or 
     \texttt{X} must be non-zero for the Branch device.  Either \texttt{B} or
     \texttt{G} must be non-zero for the Bus Shunt device.  Either \texttt{R} or
     \texttt{X} must be non-zero for the Transformer device.  \texttt{TR} must not
     be zero for the Transformer device.  \texttt{VM} must be positive for the 
     Generator Bus device. 
\end{XyceItemize}

\paragraph{Equivalent Real Form}
An Equivalent Real Form (ERF) must be used to make the complex-valued voltage-current 
and power-flow equations compatible with the real-valued solvers used by \Xyce{}.  The 
equations given below use a K1 ERF ~\cite{Munankarmy}, which 
solves the complex-valued system of equations $I=YV$ as follows.  Let $Y=(g+jb)$, 
$V=(V_{R}+jV_{I})$ and $I=(I_{R}+jI_{I})$.  Then the equivalent set of real-valued equations is:

\begin{equation}
  \left[ \begin{array}{c} I_{R} \\ I_{I}
         \end{array} \right] =
  \left[ \begin{array}{cc} 
         g & -b \\ g & b \\  
         \end{array} \right] 
  \left[ \begin{array}{c} V_{R} \\ V_{I}
          \end{array} \right]
\end{equation}

\paragraph{Y Matrices for Power Grid Branch and Bus Shunt}
The Y-Matrix for the \texttt{PowerGridBranch} device can be expressed as follows where $A=(R+jX)^{-1}$,
$R$ is the branch resistance, $X$ is the branch reactance and $B$ is the branch shunt susceptance given
on the device's instance line:

\begin{equation}
  \left[ \begin{array}{cc} 
         Y_{11} & Y_{12} \\ Y_{21} & Y_{22} \\  
         \end{array} \right] =
  \left[ \begin{array}{cc} 
         g_{11}+jb_{11} & g_{12}+jb_{12} \\  g_{21}+jb_{21} & g_{22}+jb_{22}
         \end{array} \right] =
  \left[ \begin{array}{cc} 
         A & -A+0.5j*B \\ -A+0.5j*B & A
         \end{array} \right]
\end{equation}
% PI-model for PowerGridBranch
\begin{figure}[ht]
  \centering
  \scalebox{1.0}
  {\includegraphics[width=3.2in,height= 1.14in]{PowerGridBranch.jpg}}
  \caption[Lumped $\Pi$ Model for PowerGridBranch]{Lumped $\Pi$ Model for PowerGridBranch. \label{figPowerGridBranch}}
\end{figure}
The Y-Matrix for the \texttt{PowerGridBusShunt} device can be expressed as follows where 
$G$ is the bus shunt conductance and $B$ is the bus shunt susceptance given
on the device's instance line:

\begin{equation}
  \left[ \begin{array}{cc} 
         Y_{11} & Y_{12} \\ Y_{21} & Y_{22} \\  
         \end{array} \right] =
  \left[ \begin{array}{cc} 
         g_{11}+jb_{11} & g_{12}+jb_{12} \\  g_{21}+jb_{21} & g_{22}+jb_{22}
         \end{array} \right] =
  \left[ \begin{array}{cc} 
         G+jB & -G-jB \\ -G-jB & G+jB
         \end{array} \right]
\end{equation}
% Model for PowerGridBusShunt
\begin{figure}[ht]
  \centering
  \scalebox{1.0}
  {\includegraphics[width=1.91in,height= 1.43in]{PowerGridBusShunt.jpg}}
  \caption[Equivalent Circuit for PowerGridbusShunt]{Equivalent Circuit for PowerGridBusShunt. \label{figPowerGridBusShunt}}
\end{figure}
\paragraph{Equations Common to Power Grid Branch and Bus Shunt}
The \texttt{PowerGridBranch} and \texttt{PowerGridBusShunt} devices use the same
basic equations to model voltage and current flow or voltage and power flow. The 
differences are in the Y-Matrices described above.  There are three options for 
the equations used, namely I=YV, PQ Polar and PQ Rectangular.

For the I=YV format, the device equations for the \texttt{PowerGridBranch} and 
\texttt{PowerGridBusShunt} devices are as follows, where the $g_{ij}$ and $b_{ij}$ 
terms are given above.  Also, $V_{R1}$ and $V_{I1}$ are the real and imaginary parts 
of the voltage at terminal 1.  $I_{R1}$ and $I_{I1}$ are the real and imaginary parts 
of the current at terminal 1.

\begin{equation}
  \left[ \begin{array}{c} I_{R1} \\ I_{R2} \\ I_{I1} \\ I_{I2} 
         \end{array} \right] =
  \left[ \begin{array}{cccc}
      g_{11} & g_{12} & -b_{11} & -b_{12} \\
      g_{21} & g_{22} & -b_{21} & -b_{22} \\
      b_{11} & b_{12} & g_{11} & g_{12} \\
      b_{21} & b_{22} & g_{21} & g_{22} \\
      \end{array} \right] 
  \left[ \begin{array}{c} V_{R1} \\ V_{R2} \\ V_{I1} \\ V_{I2}
         \end{array} \right]
\end{equation}

For the PQ Rectangular format, the device equations are nonlinear ~\cite{Milano}.
\begin{eqnarray}
  P_{1} = g_{11}(V_{R1}^{2} + V_{I1}^{2}) + V_{R1}(g_{12}*V_{R2}-b_{12}*V_{I2})
         + V_{I1}(b_{12}*V_{R2}+g_{12}*V_{I2})\\ 
  P_{2} = g_{22}(V_{R2}^{2} + V_{I2}^{2}) + V_{R2}(g_{21}*V_{R1}-b_{21}*V_{I1})
         + V_{I2}(b_{21}*V_{R1}+g_{21}*V_{I1})\\ 
  Q_{1} = -b_{11}(V_{R1}^{2} + V_{I1}^{2}) + V_{I1}(g_{12}*V_{R2}-b_{12}*V_{I2})
         + V_{R1}(b_{12}*V_{R2}+g_{12}*V_{I2})\\  
  Q_{2} = -b_{22}(V_{R2}^{2} + V_{I2}^{2}) + V_{I2}(g_{21}*V_{R1}-b_{21}*V_{I1})
         + V_{R2}(b_{21}*V_{R1}+g_{21}*V_{I1})
\end{eqnarray}

For the PQ Polar format, the device equations are also nonlinear ~\cite{Milano}. 
Define $|V_{1}|$ as the voltage magnitude at terminal 1 and $\Theta_{1}$ as the 
voltage angle at terminal 1.
\begin{eqnarray}
  P_{1} = g_{11} * |V_{1}|^{2} 
        + |V_{1}| * |V_{2}| * (g_{12}*cos(\Theta_{1}-\Theta_{2}) + b_{12}*sin(\Theta_{1}-\Theta_{2}))\\ 
  P_{2} = g_{22} * |V_{2}|^{2} 
        + |V_{2}| * |V_{1}| * (g_{21}*cos(\Theta_{2}-\Theta_{1}) + b_{21}*sin(\Theta_{2}-\Theta_{1}))\\ 
  Q_{1} =  -b_{11} * |V_{1}|^{2} 
        + |V_{1}| * |V_{2}| * (g_{12}*sin(\Theta_{1}-\Theta_{2}) - b_{12}*cos(\Theta_{1}-\Theta_{2}))\\ 
  Q_{2} =  -b_{22} * |V_{2}|^{2} 
        + |V_{2}| * |V_{1}| * (g_{21}*sin(\Theta_{2}-\Theta_{1}) - b_{21}*cos(\Theta_{2}-\Theta_{1}))
\end{eqnarray}

\paragraph{Equations for Power Grid Transformer}
The equations for the PowerGridTransformer device are similar to those used by the 
PowerGridBranch and PowerGridBusShunt devices.  The circuit diagram for the
PowerGridTransformer is shown below.

For I=YV and PQ Rectangular formats, the equations are the same as for the \texttt{PowerGridBranch}
and \texttt{PowerBusBusShunt} devices.  However, the following Y-Matrix is used where
where $A=(R+jX)^{-1}$, $R$ is the resistance, $X$ is the reactance, $n$ is the turns ratio (which is
the \texttt{TR} instance parameter) and $\phi$ is the phase shift in radians (which is 
the \texttt{PS} instance parameter).

For the I=YV and PQ Rectangular formats, the Y matrix is not symmetric and is given by the following
\cite{Kundur}. Let the effective complex turns ratio be $r = m + jp = n*(cos(\phi) + j*sin(\phi))$:

\begin{equation}
  \left[ \begin{array}{cc} 
         Y_{11} & Y_{12} \\ Y_{21} & Y_{22} \\  
         \end{array} \right] =
  \left[ \begin{array}{cc} 
         g_{11}+jb_{11} & g_{12}+jb_{12} \\  g_{21}+jb_{21} & g_{22}+jb_{22}
         \end{array} \right] =
  \left[ \begin{array}{cc} 
         A*(m^{2}+p^{2})^{-1} & -A*(m-jp)^{-1} \\ -A*(m+jp)^{-1} & A
         \end{array} \right]
\end{equation}

The voltage-current and power flow equations for the I=YV and PQ Rectangular formats
are then the same as for the \texttt{PowerGridBranch}
and \texttt{PowerGridBusShunt} devices, with the modified Y-matrix parameters given above.

For the PQ Polar format, the Y matrix is not symmetric and is given by ~\cite{Milano}:

\begin{equation}
  \left[ \begin{array}{cc} 
         Y_{11} & Y_{12} \\ Y_{21} & Y_{22} \\  
         \end{array} \right] =
  \left[ \begin{array}{cc} 
         g_{11}+jb_{11} & g_{12}+jb_{12} \\  g_{21}+jb_{21} & g_{22}+jb_{22}
         \end{array} \right] =
  \left[ \begin{array}{cc} 
         A*n^{-2} & -A*n^{-1} \\ -A*n^{-1} & A
         \end{array} \right]
\end{equation}
% Model for PowerGridBranch
\begin{figure}[ht]
  \centering
  \scalebox{1.0}
  {\includegraphics[width=5.17in,height= 2.21in]{PowerGridTransformer.jpg}}
  \caption[Equivalent Circuit for PowerGridTransformer]{Equivalent Circuit for PowerGridTransformer. \label{figPowerGridTransformer}}
\end{figure}
The power flow equation for PQ Polar format are then:
\begin{eqnarray}
  P_{1} = g_{11} * |V_{1}|^{2} 
        + |V_{1}| * |V_{2}| * (g_{12}*cos(\Theta_{1}-\Theta_{2}-\phi) 
                            + b_{12}*sin(\Theta_{1}-\Theta_{2}-\phi))\\ 
  P_{2} = g_{22} * |V_{2}|^{2} 
        + |V_{2}| * |V_{1}| * (g_{21}*cos(\Theta_{2}-\Theta_{1}+\phi) 
                            + b_{21}*sin(\Theta_{2}-\Theta_{1}+\phi))\\ 
  Q_{1} =  -b_{11} * |V_{1}|^{2} 
        + |V_{1}| * |V_{2}| * (g_{12}*sin(\Theta_{1}-\Theta_{2}-\phi) 
                            - b_{12}*cos(\Theta_{1}-\Theta_{2}-\phi))\\ 
  Q_{2} =  -b_{22} * |V_{2}|^{2} 
        + |V_{2}| * |V_{1}| * (g_{21}*sin(\Theta_{2}-\Theta_{1}+\phi) 
                             - b_{21}*cos(\Theta_{2}-\Theta_{1}+\phi))
\end{eqnarray}

\paragraph{Equations for Power Grid Gen Bus}
The \texttt{PowerGridGenBus} is an active device that functions as an ideal generator 
with a fixed power output ($P$) and voltage magnitude ($VM$).  Reactive power 
(\texttt{QMAX} and \texttt{QMIN}) limits are also supported.  The device equations
for the PQ Polar format are as follows ~\cite{Milano}.  The other solution formulations are 
not supported in this release.  If reactive power limits are not being enforced then:
\begin{eqnarray}
P_{1} = P \\|V_{1}| = VM
\end{eqnarray}
If reactive power limits are being enforced then $P_{1}$ is still held constant but the behavior
of the $V_{1}$ terminal changes between a constant-voltage and a constant-current source.
In particular, $|V_{1}| = VM$ only if \texttt{QMIN} $< Q_{1} <$ \texttt{QMAX}.  Otherwise, 
$|V_{1}|$ is unconstrained and the appropriate \texttt{QMIN} or \texttt{QMAX} value is 
enforced at the $V_{1}$ terminal instead.

The convention for Power Grids is that positive power is injected into the grid.
So, positive real (P) and reactive power (Q) flow out of the positive terminals 
(\texttt{inputNode 1} and \texttt{inputNode 2}).  This is reversed from the normal
convention for current direction for voltage and current sources in either \Xyce{} 
or SPICE.


%%
%% Anti-Windup Limiter device Subsection
%\clearpage
%\subsection{Anti-Windup Limiter}
%\index{anti-windup limiter}
%% Sandia National Laboratories is a multimission laboratory managed and
% operated by National Technology & Engineering Solutions of Sandia, LLC, a
% wholly owned subsidiary of Honeywell International Inc., for the U.S.
% Department of Energy’s National Nuclear Security Administration under
% contract DE-NA0003525.

% Copyright 2002-2023 National Technology & Engineering Solutions of Sandia,
% LLC (NTESS).

 
The simulation of control systems, such as those used in power grids,
may require two different types of limiters for ``hard limits''.  They
are known as ``windup'' and ``anti-windup'' limiters ~\cite{Milano},
and are typically depicted as shown in Figures \ref{figAntiWindupLimiter} 
and \ref{figWindupLimiter}. The anti-windup limiter has the following 
equations:
\[
  \begin{array}{ll} 
  \mbox{if} \; x \geq x^{max} \; \mbox{and} \; \dot{x} \geq 0 \Rightarrow x = x^{max} \; \mbox{and} \; \dot{x} = 0\\
  \mbox{if} \; x \leq x^{min} \; \mbox{and} \; \dot{x} \leq 0 \Rightarrow x = x^{min} \; \mbox{and} \; \dot{x} = 0\\
  \mbox{otherwise} \; x = (y-x)/T
  \end{array}
\]
The anti-windup limiter has simultaneous constraints on both the output-voltage
level and the first derivative of the output-voltage level.  These equations were 
approximated with a \Xyce{} device model, as described below.  

\begin{figure}[ht]
  \centering
  \scalebox{1.0}
  {\includegraphics[width=2.1in,height= 1.05in]{AntiWindupLimiter.jpg}}
  \caption[Anti-Windup Limiter]{Anti-Windup Limiter. \label{figAntiWindupLimiter}}
\end{figure}

The windup limiter only has a constraint on the output-voltage level.  This is 
straightforward to implement with existing \Xyce{} device models.  The equations
and an example implementation are given below after the description of the \Xyce{} 
anti-windup limiter device model.

\begin{figure}[ht]
  \centering
  \scalebox{1.0}
  {\includegraphics[width=2.0in,height= 1.0in]{WindupLimiter.jpg}}
  \caption[Windup Limiter]{Windup Limiter. \label{figWindupLimiter}}
\end{figure}

\begin{Device}\label{AntiWindupLimiter}

\device
\begin{alltt}
Y<type> <name> <input node> <output node> [device parameters] 
\end{alltt}
  
\examples
\begin{alltt}
YAntiWindupLimiter awl1 in1 out1 T=1 UL=0.5 LL=-0.5
YAWL awl2 in2 out2 T=2 UL=0.5 LL=-0.5
\end{alltt}

\parameters 
\begin{Parameters}
\param{type}
The device type has a verbose (\texttt{AntiWindupLimiter}) and a shortened
(\texttt{AWL}) form.  Their usage may be mixed within a netlist.

\param{name}
Name of the device instance.  This must be present, and unique amongst the 
\texttt{AntiWindupLimiter} devices in the netlist.

\param{input node}
There is one input node, \texttt{<input node>}. 

\param{output node}
There is one output node, \texttt{<output node>}.

\param{T} 
The time-constant, in seconds. 

\param{UL}
The upper control limit, in per-unit. The upper control limit must be 
greater than the lower control limit.  For numerical stability reasons,
the output signal level may exceed this upper limit slightly, depending 
on the time-steps used by \Xyce{}. The reason is that the derivative of 
the AWL output must be continuous in the current implementation of this 
device model.

\param{LL}
The lower control limit, in per-unit. For numerical stability reasons,
the output signal level may be slightly less than this lower limit, depending 
on the time-steps used by \Xyce{}. The reason is that the derivative of 
the AWL output must be continuous in the current implementation of this 
device model.
\end{Parameters}
\end{Device}

\paragraph{Anti-Windup Limiter Device Parameters}
%%
%% AntiWindupLimiter Table
%%
\input{AntiWindupLimiter_1_Device_Instance_Params}

\paragraph{Branch Current and Power Accessors}
This version of the Anti-Windup Limiter device does not 
support the branch current accessor, \texttt{I()}, 
or the power accessors, \texttt{P()} or \texttt{W()}.

\paragraph{Comments}
The model parameters \texttt{UL} and \texttt{LL} are specified 
in ``per-unit'' since the AWL device was developed
for use with the power grid devices.  For other uses, the upper
and lower control limits can be considered to be specified in 
volts.

\paragraph{Windup Limiter}
For a first-order low-pass filter, followed by a hard-limiting function,
the windup limiter has the following equations and can be depicted
as shown in Figure \ref{figWindupLimiter}:
\[
x = \left\{ \begin{array}{ll}
x^{max}, &  \mbox{if} \; x_{1} \geq x^{max}\\
x, &  \mbox{if} \; x^{min} < x_{1} < x^{max}\\
x^{min}, & \mbox{if} \; x_{1} \leq x^{min}
\end{array}
\right.
\]
The version of the windup limiter shown in Figure \ref{figWindupLimiter}
can be implemented as a first-order RC filter followed by a B-Source with
a \texttt{LIMIT} function.  An example \Xyce{} netlist fragment is as follows:
\begin{alltt}
.param timeConstant=1
.param upperLimit=0.5
.param lowerLimit=-0.5
Rlpf y x1 {timeConstant}
Clpf x1 0 1
BLim x 0 V={LIMIT(V(x1),lowerLimit,upperLimit)}
\end{alltt}


%%
%% Memristor Device subsection
%%
\clearpage
\subsection{Memristor Device}
\index{memristor}
% Sandia National Laboratories is a multimission laboratory managed and
% operated by National Technology & Engineering Solutions of Sandia, LLC, a
% wholly owned subsidiary of Honeywell International Inc., for the U.S.
% Department of Energy’s National Nuclear Security Administration under
% contract DE-NA0003525.

% Copyright 2002-2023 National Technology & Engineering Solutions of Sandia,
% LLC (NTESS).


\begin{Device}

\symbol
{\includegraphics[scale=.2, angle=90]{MemristorSymbol}}

\device
ymemristor <name> <(+) node> <(-) node> <model>

\model
\begin{alltt}
.MODEL <model name> MEMRISTOR level=2 [model parameters]
\end{alltt}

\examples
\begin{alltt}

ymemristor mr1 n1 n2 mrm2

.model mrm2 memristor level=2 ron=50 roff=1000 
+ koff=1.46e-18 kon=-4.68e-22 
+ alphaoff=10 alphaon=10 wc=1.0e-12 
+ ioff=115e-6 ion=-8.9e-6 xscaling=1.0e9 wt=4


ymemristor mr2 n1 n2 mrm3 xo=0.11 

.MODEL mrm3 memristor level=3 a1=0.17 a2=0.17 b=0.05 vp=0.16 vn=0.15 
+ ap=4000 an=4000 xp=0.3 xn=0.5 alphap=1 alphan=5 eta=1 


ymemristor mr3 n1 n2 mrm4 

.model mrm4 memristor level=4 
+ fxpdata=fxp_table.csv
+ fxmdata=fxm_table.csv
+ I1=85.37e-6 I2=90.16e-6 V1=0.265 V2=0.265 G0=130.72e-6
+ VP=0.7 VN=1.0 d1=9.87 d2=-4.82
+ C1=1000 C2=1000

\end{alltt}

\parameters

\begin{Parameters}
\param{\vbox{\hbox{(+) node\hfil}\hbox{(-) node}}}

Polarity definition for a positive voltage across the memristor. The
first node is defined as positive. Therefore, the voltage across the
component is the first node voltage minus the second node voltage.


\end{Parameters}

\comments
The {\tt level=2} memristor device is an implementation of the TEAM formulation described 
in~\cite{KvatinskyFriedman2012} and~\cite{KvatinskyFriedman2013}.
The {\tt level=3}  memristor device is an implementation of the Yakopcic formulation described 
in~\cite{ChrisYakopcic2013}. The {\tt level=4} memristor device is an implementation of the 
Piecewise Empirical Model described in~\cite{Niroula2017}.

Positive current flows from
the \texttt{(+)} node through the device to the \texttt{(-)}
node. The power through the device is calculated 
with $I \cdot \Delta V$ where the voltage drop is calculated as $(V_+ - V_-)$ 
and positive current flows from $V_+$ to $V_-$.  
\end{Device}

\paragraph{Device Equations for TEAM Formulation}
The current voltage relationship for the TEAM formulation can be linear or nonlinear and this is selectable with 
the instance parameter {\tt IVRELATION}.  The default is the linear relationship which is:
\begin{equation}
v(t) = \left[ R_{ON} + \frac{R_{OFF} - R_{ON}}{x_{OFF} - x_{ON}} \left( x - x_{ON} \right) \right] i(t)
\end{equation}
The non-linear relationship is:
\begin{equation}
v(t) = R_{ON} e^{\lambda(x - x_{ON})/(x_{OFF}-x_{ON})}  i(t)
\end{equation}
where $\lambda$ is defined as:
\begin{equation}
\frac{R_{OFF}}{R_{ON}} = e^{\lambda}
\end{equation}
In the above equations $x$ represents a doped layer whose growth determines the overall
resistance of the device.  The equation governing the value of $x$ is:
\begin{equation}
\frac{dx}{dt} = \left\{  
  \begin{array}{ll}
    k_{OFF} \left(\frac{i}{i_{OFF}} - 1 \right)^{\alpha_{OFF}} f_{OFF}(x) & 0 < i_{OFF} < i \\
    0       & i_{ON} < i < i_{OFF} \\
    k_{ON} \left(\frac{i}{i_{on}} - 1 \right)^{\alpha_{on}} f_{ON}(x) & i < i_{ON} < 0 \\
  \end{array}
  \right.
\end{equation}

The functions $f_{ON}(x)$ and $f_{OFF}(x)$ are window functions designed to keep $x$ within
the defined limits of $x_{ON}$ and $x_{OFF}$.  Four different types of window functions 
are available and this is selectable with the model parameter {\tt WT}. Note that the 
TEAM memristor device is formulated to work best with the TEAM, Kvatinsky, window 
function {\tt WT=4 }.  Other window functions should be used with caution.

\paragraph{Device Equations for Yakopcic Formulation}
The current voltage relationship for the Yakopcic memristor device is:~\cite{ChrisYakopcic2013}
\begin{equation}
I(t) = \left\{  
  \begin{array}{ll}
    a_1 x(t) \sinh(b V(t)) & V(t) \ge 0 \\
    a_2 x(t) \sinh(b V(t)) & V(t) < 0 \\
  \end{array}
  \right.
\end{equation}

\begin{equation}
g(V(t)) = \left\{  
  \begin{array}{ll}
    A_p\left( \exp^{V(t)} - \exp^{V_p}\right) & V(t) > V_p \\
    -A_n \left( \exp^{-V(t)} - \exp^{V_n}\right) & V(t) < -V_n \\
    0       & -V_n \le V(t) \le V_p \\
  \end{array}
  \right.
\end{equation}

The internal state variable, $x$, is governed by the equation:

\begin{equation}
\frac{dx}{dt} = n g(V(t)) f(x(t))
\end{equation}

where $f(x)$ is defined by:

\begin{equation}
f(x) = \left\{  
  \begin{array}{ll}
   \exp^{-\alpha_p (x-x_p)} w_p(x,x_p) & x \ge x_p \\
   1 & x \le x_p \\
  \end{array}
  \right.
\end{equation}
\begin{equation}
f(x) = \left\{  
  \begin{array}{ll}
   \exp^{\alpha_n (x+x_n-1} w_n(x,x_n) & x \le 1-x_n \\
   1 & x > 1 - x_n \\
  \end{array}
  \right.
\end{equation}
\begin{equation}
w_p(x,x_p) = \frac{x_p -x}{1 - x_p} + 1
\end{equation}
\begin{equation}
w_n(x,x_n) = \frac{x}{1-x_n}
\end{equation}

Note, the quantities, $x_p$, $x_n$, $\alpha_p$, $\alpha_n$, $A_p$, $A_n$, $a_1$, $a_2$ and$b$ are model
parameters that can be specified in the device's model block.


\paragraph{Device Equations for the PEM Formulation}
The PEM memristor device is similar to the TEAM and Yakopcic formulations in that an 
internal state variable, $x$, is used to capture the device's response to its history.

The I-V relationship is
\begin{equation}
I = x \: h(V)
\end{equation}

and $h(V)$ is defined by:
\begin{equation}
 h(V) = I_1 * exp(V/V_1) - I_2 * exp(-V/V_2) + G_0 V - (I_1-I_2)
\end{equation}
where $I_1$, $I_2$, $V_1$, $V_2$ and $G_0$ are model parameters.

The internal variable, $x$, is defined by:
\begin{equation}
  \frac{dx}{dt} = G(V) f(x)
\end{equation}
with
\begin{equation}
G(V) = \left\{  
  \begin{array}{ll}
   C_1 \left( \exp^{d_1\left[ V(t) - V_p\right]} - 1\right) & V > V_p \\
   C_2 \left( \exp^{d_2\left[ V(t) - V_n\right]} - 1\right) & V < -V_n \\
   0 & -V_n \le V(t) \le V_p \\
  \end{array}
  \right.
\end{equation}

Finally, the function $f(x)$ is defined by a user supplied set set data which is 
used with linear interpolation to find the current value of $f(x)$.  Separate data
sets are used for forward bias and reverse bias.
\begin{equation}
f(x) = \left\{  
  \begin{array}{ll}
   F^+ {\tt data set} & V > 0 \\
   F^- {\tt data set} & V < 0 \\
  \end{array}
  \right.
\end{equation}




%%\pagebreak

\paragraph{Device Parameters for TEAM Formulation}
\input{Memristor_2_Device_Instance_Params}

\paragraph{Model Parameters for TEAM Formulation}
\input{Memristor_2_Device_Model_Params.tex}



% Sandia National Laboratories is a multimission laboratory managed and
% operated by National Technology & Engineering Solutions of Sandia, LLC, a
% wholly owned subsidiary of Honeywell International Inc., for the U.S.
% Department of Energy’s National Nuclear Security Administration under
% contract DE-NA0003525.

% Copyright 2002-2023 National Technology & Engineering Solutions of Sandia,
% LLC (NTESS).


\paragraph{Device Parameters for Yakopcic Formulation}
\input{Memristor_3_Device_Instance_Params}

\paragraph{Model Parameters for Yakopcic Formulation}
\input{Memristor_3_Device_Model_Params.tex}




% Sandia National Laboratories is a multimission laboratory managed and
% operated by National Technology & Engineering Solutions of Sandia, LLC, a
% wholly owned subsidiary of Honeywell International Inc., for the U.S.
% Department of Energy’s National Nuclear Security Administration under
% contract DE-NA0003525.

% Copyright 2002-2024 National Technology & Engineering Solutions of Sandia,
% LLC (NTESS).


\paragraph{Device Parameters for PEM Formulation}
\input{Memristor_4_Device_Instance_Params}

\paragraph{Model Parameters for PEM Formulation}
\input{Memristor_4_Device_Model_Params.tex}





%%
%% Subcircuit Subsection
%%
\clearpage
\subsection{Subcircuit}
\label{SubcircuitInstance}
\index{subcircuit}
% Sandia National Laboratories is a multimission laboratory managed and
% operated by National Technology & Engineering Solutions of Sandia, LLC, a
% wholly owned subsidiary of Honeywell International Inc., for the U.S.
% Department of Energy’s National Nuclear Security Administration under
% contract DE-NA0003525.

% Copyright 2002-2024 National Technology & Engineering Solutions of Sandia,
% LLC (NTESS).


\label{XDevicesection}
A subcircuit can be introduced into the circuit netlist
using the specified nodes to substitute for the argument nodes in the
definition.  It provides a building block of circuitry to be defined a single
time and subsequently used multiple times in the overall circuit netlists.
See Section~\ref{SUBCKTsection} for more information about subcircuits.

\begin{Device}

\device
X<name> [node]* <subcircuit name> [PARAMS: [<name> = <value>]*]

\examples
\begin{alltt}
  X12 100 101 200 201 DIFFAMP
  XBUFF 13 15 UNITAMP
  XFOLLOW IN OUT VCC VEE OUT OPAMP
  XFELT 1 2 FILTER PARAMS: CENTER=200kHz
  XNANDI 25 28 7 MYPWR MYGND PARAMS: IO\_LEVEL=2
\end{alltt}

\parameters

\begin{Parameters}

\param{subcircuit name}
The name of the subcircuit's definition.

\param{PARAMS:}

Passed into subcircuits as arguments and into expressions inside the
subcircuit.

\end{Parameters}

\comments

There must be an equal number of nodes in the subcircuit call and in its
definition.

Subcircuit references may be nested to any level. However, the nesting
cannot be circular. For example, if subcircuit \texttt{A}'s definition
includes a call to subcircuit \texttt{B}, then subcircuit \texttt{B}'s
definition cannot include a call to subcircuit \texttt{A}.

\end{Device}


%%

%%% Local Variables:
%%% mode: latex
%%% End:

% END of Xyce_RG_app1.tex ************
