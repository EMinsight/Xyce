% Sandia National Laboratories is a multimission laboratory managed and
% operated by National Technology & Engineering Solutions of Sandia, LLC, a
% wholly owned subsidiary of Honeywell International Inc., for the U.S.
% Department of Energy’s National Nuclear Security Administration under
% contract DE-NA0003525.

% Copyright 2002-2023 National Technology & Engineering Solutions of Sandia,
% LLC (NTESS).

\label{Measure_CONT_section}

``Continuous'' measure results are supported for \texttt{DERIV-AT}, \texttt{DERIV-WHEN},
\texttt{FIND-AT}, \texttt{FIND-WHEN}, \texttt{WHEN} and \texttt{TRIG-TARG} measures
for \texttt{.TRAN}, \texttt{.DC}, \texttt{.AC} and \texttt{.NOISE}
analyses.  They are identical to the ``non-continuous'' versions, except that
they can return more than one measured value in some cases.

\begin{Command}
\format
\begin{alltt}
.MEASURE <AC_CONT|DC_CONT|NOISE_CONT|TRAN_CONT> <result name>
+ DERIV <variable> AT=<value>
+ [MINVAL=<value>] [DEFAULT_VAL=<value>]
+ [PRECISION=<value>] [PRINT=<value>]

.MEASURE TRAN_CONT <result name>
+ DERIV <variable> WHEN <variable>=<variable\(\sb{2}\)>|<value>
+ [MINVAL=<value>] [FROM=<value>] [TO=<value>] [TD=<value>]
+ [RISE=r|LAST] [FALL=f|LAST] [CROSS=c|LAST]
+ [DEFAULT_VAL=<value>] [PRECISION=<value>] [PRINT=<value>]

.MEASURE <AC_CONT|DC_CONT|NOISE_CONT> <result name>
+ DERIV <variable> WHEN <variable>=<variable\(\sb{2}\)>|<value>
+ [MINVAL=<value>] [FROM=<value>] [TO=<value>]
+ [RISE=r|LAST] [FALL=f|LAST] [CROSS=c|LAST]
+ [DEFAULT_VAL=<value>] [PRECISION=<value>] [PRINT=<value>]

.MEASURE <AC_CONT|DC_CONT|NOISE_CONT|TRAN_CONT> <result name>
+ FIND <variable> AT=<value>
+ [MINVAL=<value>] [DEFAULT_VAL=<value>]
+ [PRECISION=<value>] [PRINT=<value>]

.MEASURE TRAN_CONT <result name>
+ FIND <variable> WHEN <variable>=<variable\(\sb{2}\)>|<value>
+ [FROM=<value>] [TO=<value>] [TD=<value>]
+ [RISE=r|LAST] [FALL=f|LAST] [CROSS=c|LAST]
+ [MINVAL=<value>] [DEFAULT_VAL=<value>]
+ [PRECISION=<value>] [PRINT=<value>]

.MEASURE <AC_CONT|DC_CONT|NOISE_CONT> <result name>
+ FIND <variable> WHEN <variable>=<variable\(\sb{2}\)>|<value>
+ [FROM=<value>] [TO=<value>]
+ [RISE=r|LAST] [FALL=f|LAST] [CROSS=c|LAST]
+ [MINVAL=<value>] [DEFAULT_VAL=<value>]
+ [PRECISION=<value>] [PRINT=<value>]

.MEASURE TRAN_CONT <result name>
+ WHEN <variable>=<variable\(\sb{2}\)>|<value>
+ [FROM=<value>] [TO=<value>] [TD=<value>]
+ [RISE=r|LAST] [FALL=f|LAST] [CROSS=c|LAST]
+ [MINVAL=<value>] [DEFAULT_VAL=<value>]
+ [PRECISION=<value>] [PRINT=<value>]

.MEASURE <AC_CONT|DC_CONT|NOISE_CONT> <result name>
+ WHEN <variable>=<variable\(\sb{2}\)>|<value>
+ [FROM=<value>] [TO=<value>]
+ [RISE=r|LAST] [FALL=f|LAST] [CROSS=c|LAST]
+ [MINVAL=<value>] [DEFAULT_VAL=<value>]
+ [PRECISION=<value>] [PRINT=<value>]

.MEASURE <AC_CONT|DC_CONT|NOISE_CONT|TRAN_CONT> <result name>
+ TRIG <variable\(\sb{1}\)>=<variable\(\sb{2}\)>|<value> 
+ [TD=<val>] [RISE=r] [FALL=f] [CROSS=c]
+ TARG <variable\(\sb{3}\)>=<variable\(\sb{4}\)>|<value> 
+ [TD=<val>] [RISE=r] [FALL=f] [CROSS=c]
+ [MINVAL=<value>] [DEFAULT_VAL=<value>]
+ [PRECISION=<value>] [PRINT=<value>]

.MEASURE <AC_CONT|DC_CONT|NOISE_CONT|TRAN_CONT> <result name>
+ TRIG AT=<value> TARG AT=<value> 
+ [MINVAL=<value>] [DEFAULT_VAL=<value>]
+ [PRECISION=<value>] [PRINT=<value>]

\end{alltt}
\index{results!measure continuous}

\examples
\begin{alltt}
.MEASURE TRAN_CONT DERIV1At5 DERIV V(1) AT=5
.MEASURE DC_CONT deriv2 DERIV WHEN V(2)=0.75
.MEASURE AC_CONT find1at5 FIND V(1) AT=5
.MEASURE NOISE_CONT find1 FIND V(1) WHEN V(2)=1 RISE=2
.MEASURE TRAN_CONT whenv1 WHEN V(1)=5
.MEASURE TRAN_CONT TrigTargAT TRIG AT=2ms TARG AT=8ms
.MEASURE TRAN_CONT TrigTargAT1 TRIG V(1)=0.2 CROSS=1
+ TARG AT=8ms
.MEASURE TRAN_CONT TrigTargAT2 TRIG AT=2ms
+ TARG V(1)=0.2 CROSS=1
.MEASURE TRAN_CONT TrigTarg TRIG V(1)=0.2 CROSS=1
+ TARG V(1)=0.3 CROSS=1 TD=8ms
\end{alltt}

\arguments

\begin{Arguments}
\argument{result name}

Measured results are reported to the log file and (possibly) multiple
output files. Section~\ref{Measure_CONT_Measurement_Output} below gives more
information on the output files produced by continuous mode measures.

The \texttt{<result name>} must be a legal \Xyce{} character string.
If multiple measures are defined with the same \texttt{<result name>} then
\Xyce{} uses the last such definition, and issues warning messages about
(and discards) any previous measure definitions with the same
\texttt{<result name>}.

\argument{measure type}

\texttt{DERIV, FIND, WHEN, TRIG, TARG}

The third argument specifies the type of measurement or calculation to
be done.

By default, the measurement is performed over the entire simulation.
The calculations can be limited to a specific measurement window by
using the qualifiers {\tt FROM}, {\tt TO}, {\tt TD}, {\tt RISE}, {\tt
FALL}, {\tt CROSS} and {\tt MINVAL}, which are explained below and in
section~\ref{Measure_section}..

The supported ``continuous'' measure types and their definitions are:

\begin{description}
 \item[\tt DERIV] Computes the derivative of {\tt <variable>} at a
    user-specified time (by using the {\tt AT} qualifier) or when a
    user-specified condition occurs (by using the {\tt WHEN}
    qualifier). If the {\tt WHEN} qualifier is used then the
    measurement window can be limited with the qualifiers {\tt FROM},
    {\tt TO}, {\tt RISE}, {\tt FALL} and {\tt CROSS} for all measure
    modes.  In addition, the {\tt TD} qualifier is supported for
    {\tt TRAN\_CONT} measures. The {\tt MINVAL} qualifier is used as a
    comparison tolerance for both {\tt AT} and {\tt WHEN}.  For HSPICE
    compatibility, {\tt DERIVATIVE} is an allowed synonym for {\tt
    DERIV}.

   \item[\tt FIND-AT] Returns the value of {\tt <variable>} at the
    time when the {\tt AT} clause is satisfied.  The {\tt AT}
    clause is described in more detail later in this list.

  \item[\tt FIND-WHEN] Returns the value of {\tt <variable>} at the
    time when the {\tt WHEN} clause is satisfied.  The {\tt WHEN}
    clause is described in more detail later in this list.

  \item[\tt WHEN] Returns the time (or frequency or DC sweep value) when
    {\tt <variable>} reaches {\tt <variable\(\sb{2}\)>} or the constant
    value, {\tt value}.  The measurement window can be limited with the
    qualifiers {\tt FROM}, {\tt TO}, {\tt RISE}, {\tt FALL} and {\tt CROSS}
    for all measure modes.  In addition, the {\tt TD} qualifier is supported
    for {\tt TRAN\_CONT} measures. The qualifier {\tt MINVAL} acts as a
    tolerance for the comparison.  For example when {\tt <variable\(\sb{2}\)>}
    is specified, the comparison used is when {\tt <variable>} $=$
    {\tt <variable\(\sb{2}\)>} $\pm$ {\tt MINVAL} or when a constant,
    {\tt value} is given: {\tt <variable>} $=$ {\tt value} $\pm$ {\tt
    MINVAL}.  If the conditions specified for finding a given value
    were not found during the simulation then the measure will return
    the default value of {\tt -1}.  The user may change this default
    value with the {\tt DEFAULT\_VAL} qualifier.  Note: The use of
    {\tt FIND} and {\tt WHEN} in one measure statement is also supported.

  \item[\vbox{\hbox{\tt TRIG\hfil}\hbox{\tt TARG\hfil}}] Measures the
    time between a trigger event and a target event.  The trigger is
    specified with {\tt TRIG <variable\(\sb{1}\)>=<variable\(\sb{2}\)>} or {\tt
    TRIG <variable\(\sb{1}\)>=<value>} or {\tt TRIG AT=<value>}.  The target
     is specified as {\tt TARG <variable\(\sb{3}\)>=<variable\(\sb{4}\)>}
    or {\tt TARG <variable\(\sb{3}\)>=<value>} or {\tt TARG AT=<value>}.  The
    measurement window can be limited with the qualifiers {\tt TD}, {\tt RISE},
    {\tt FALL} and {\tt CROSS} for all measure modes.  The qualifier {\tt MINVAL}
     acts as a tolerance for the comparison.  For example when {\tt <variable\(\sb{2}\)>}
    is specified, the comparison used is when {\tt <variable\(\sb{1}\)>} $=$
    {\tt <variable\(\sb{2}\)>} $\pm$ {\tt MINVAL} or when a constant,
    {\tt value} is given: {\tt <variable\(\sb{1}\)>} $=$ {\tt value} $\pm$ {\tt
    MINVAL}.  If the conditions specified for finding a given value
    were not found during the simulation then the measure will return
    the default value of {\tt -1}.  The user may change this default
    value with the {\tt DEFAULT\_VAL} qualifier.  
\end{description}

\argument{\vbox{\hbox{variable\hfil}\hbox{variable\(\sb{n}\)\hfil}\hbox{value}}}

These quantities represents the test for the stated
measurement.  \texttt{<variable>} is a simulation quantity, such as a
voltage or current.  One can compare it to another simulation variable
or a fixed quantity.  Additionally, the \texttt{<variable>} may be
a \Xyce{} expression delimited by \{ \} brackets.  As noted above, an
example is {\tt V(2)=0.75}
\end{Arguments}

Additional information on the \texttt{TO}, \texttt{FROM}, \texttt{TD},
\texttt{RISE}, \texttt{FALL}, \texttt{CROSS}, \texttt{MINVAL},
\texttt{DEFAUAL\_VAL}, \texttt{PRECISION} and \texttt{PRINT} qualifiers
is given in section~\ref{Measure_section}.

\end {Command}

\subsubsection{Measure Output}
\label{Measure_CONT_Measurement_Output}
\index{measure continuous!measurement output}
As discussed in section~\ref{Measure_Measurement_Output}, measured results
for \texttt{AC}, \texttt{DC}, \texttt{NOISE} and \texttt{TRAN} mode measures
are reported to the log file.  Additionally, for \texttt{TRAN} measures, the
results are stored in files called \texttt{circuitFileName.mt\#}, where the
suffixed number (\texttt{\#}) starts at \texttt{0} and increases for multiple
iterations (\texttt{.STEP} iterations) of a given simulation. For \texttt{DC}
measures, the results are stored in the files \texttt{circuitFileName.ms\#},
while \texttt{AC} and \texttt{NOISE} measures use the files
\texttt{circuitFileName.ma\#}.

For \texttt{AC\_CONT}, \texttt{DC\_CONT}, \texttt{NOISE\_CONT} and
\texttt{TRAN\_CONT} mode measures, the output for successful and failed
measures is sent to the standard output (and log files), as described in
section ~\ref{Measure_Measurement_Output}.  There are two options for the
output files though. The default is for each continuous mode measure to generate its
own output file where, for example for a non-step transient analysis, the file name
would be \texttt{circuitFileName\_resultname.mt0} where the result (measure) name is
always output in lower-case. This default matches HSPICE. The second option uses
\texttt{.OPTIONS MEASURE USE\_CONT\_FILES=0}. In that case, the results for all
of the continuous mode measures are sent to the \texttt{circuitFileName.mt\#} file.

An example is as follows.

\begin{alltt}
VPWL1 1 0 pwl(0 0 2.5m 1 5m 0 7.5m 1 10m 0)
R1 1 0 1

.TRAN 0 10ms
.PRINT TRAN V(1)

.MEASURE TRAN MAXV1 MAX V(1)
.MEASURE TRAN_CONT FindV1 WHEN V(1)=0.5
.MEASURE TRAN_CONT FindV1AT FIND V(1) AT=0.6ms

.END
\end{alltt}

The result for measure \texttt{MAXV1} is sent to \texttt{<netlistName>.mt0}. The
results for measures \texttt{FindV1} and \texttt{FindV1AT} are then sent to individual
files, named \texttt{<netlistName>\_findv1.mt0} and \texttt{<netlistName>\_findv1at.mt0}.
Note that the measure names have been lower-cased in the output file names.  The contents
of those files are then as follows.

\begin{alltt}
FINDV1 = 1.250000e-03
FINDV1 = 3.750000e-03
FINDV1 = 6.250000e-03
FINDV1 = 8.750000e-03
\end{alltt}

and:

\begin{alltt}
FINDV1AT = 2.400000e-01
\end{alltt}

Note that \texttt{FIND-AT} measures will still only return one measure value, even for
\texttt{TRAN\_CONT} measure mode.  However, in this simple example, the specified
\texttt{FIND-WHEN} measure returns all four times where \texttt{V(1)} equals 0.5.
The next subsection will describe how the \texttt{RISE}, \texttt{FALL} and \texttt{CROSS}
qualifiers can used to return only a subset of those four crossings.

\subsubsection{RISE, FALL and CROSS Qualifiers}
\label{Measure_CONT_RFC}
\index{measure continuous!rise, fall and cross qualifiers}
\Xyce{} supports non-negative values for the \texttt{RISE}, \texttt{FALL}
and \texttt{CROSS} qualifiers for all continuous measure types.  It supports
negative values for the \texttt{RISE}, \texttt{FALL} and \texttt{CROSS} qualifiers
for the \texttt{DERIV-WHEN}, \texttt{FIND-WHEN} and \texttt{WHEN} measure types.
However, their interpretation is slightly different for
\texttt{TRAN} and \texttt{TRAN\_CONT} measure modes, as illustrated by the following
netlist for the \texttt{TRAN\_CONT} measure mode and \texttt{WHEN} measure. The rules
are then the same for the other continuous measures modes and the \texttt{RISE} and
\texttt{FALL} qualifiers.

\begin{alltt}
VPWL1 1 0 pwl(0 0 2.5m 1 5m 0 7.5m 1 10m 0)
R1 1 0 1

.TRAN 0 10ms
.PRINT TRAN V(1)

.MEASURE TRAN FindV1_CROSS3 WHEN V(1)=0.5 CROSS=3
.MEASURE TRAN_CONT FindV1_CONT_CROSS3 WHEN V(1)=0.5 CROSS=3

.MEASURE TRAN FindV1_CROSS_NEG3 WHEN V(1)=0.5 CROSS=-3
.MEASURE TRAN_CONT FindV1_CONT_CROSS_NEG3 WHEN V(1)=0.5 CROSS=-3

.END
\end{alltt}

The \texttt{<netlistName>.mt0} file will contain the results for both
\texttt{TRAN} mode measures.  The result for the \texttt{FindV1\_CROSS3}
is the time of the third crossing.  The result for the \texttt{FindV1\_CROSS\_NEG3}
is the time of the second crossing, which is also the ``third to last'' (or
negative third) crossing in this case.

\begin{alltt}
FINDV1_CROSS3 = 6.250000e-03
FINDV1_CROSS_NEG3 = 3.750000e-03
\end{alltt}

The \texttt{<netlistName>\_findv1\_cont\_cross3.mt0} output file will have two
values.  For non-negative values of \texttt{CROSS}, a \texttt{TRAN\_CONT} measure
will return all crossings, starting with the specified value.  This is the third
and fourth crossings in this case.

\begin{alltt}
FINDV1_CONT_CROSS3 = 6.250000e-03
FINDV1_CONT_CROSS3 = 8.750000e-03
\end{alltt}

For negative values of \texttt{CROSS}, a \texttt{TRAN\_CONT} measure will only
return one value.  That is the third-to-last crossing in this case.  So, the
\texttt{<netlistName>\_findv1\_cont\_cross3\_neg3.mt0} file only has one
value in it.  As a final note, a \texttt{CROSS} value of either 5 or -5 would
produce failed measures in this example.

\begin{alltt}
FINDV1_CONT_CROSS_NEG3 = 3.750000e-03
\end{alltt}

\subsubsection{AT and TD Qualifiers for TRIG-TARG}
\index{measure continuous!AT and TD qualifiers for TRIG-TARG}
The following rules apply to the \texttt{AT} and \texttt{TD} qualifiers
for \texttt{TRIG-TARG} measures:

\begin{XyceItemize}
  \item Separate \texttt{AT} values can be given for the \texttt{TRIG} and
\texttt{TARG} clauses.
  \item Separate \texttt{TD} values can be given for the \texttt{TRIG} and
\texttt{TARG} clauses.
  \item The \texttt{AT} value takes precedence over the \texttt{TD} qualifier if
both are given in a \texttt{TRIG} or \texttt{TARG} clause.
  \item If the \texttt{TD} value is only given for the \texttt{TRIG} clause then
that value will be used for both the \texttt{TRIG} and \texttt{TARG} clauses.
  \item An \texttt{AT} value that is outside of the simulation window, or a 
\texttt{TD} value that is greater than the end simulation time or the largest
\texttt{AC}, \texttt{DC} or \texttt{NOISE} sweep value, will produce a failed
measure.
  \item A \texttt{TD} value that is less than 0,  or the smallest \texttt{AC},
\texttt{DC} or \texttt{NOISE} sweep value, is essentially ignored.
\end{XyceItemize}

\subsubsection{HSPICE Compatibility}
\label{Measure_CONT__HSpice_Compatibility}
\index{measure continuous!HSPICE compatibility}
There are known incompatibilities between the \Xyce{} and HSPICE implementation
of continuous measures.  They include the following:

\begin{XyceItemize}
  \item \Xyce{} will not return a trig or targ value that is outside of the simulation
bounds.  In some case, HSPICE will return a trig or targ value that is earlier than
the start of the simulation window.
  \item \Xyce{} does not support negative values for the \texttt{RISE}, \texttt{FALL} or
\texttt{CROSS} qualifiers for the continuous version of the \texttt{TRIG-TARG} measure.

\end{XyceItemize}

