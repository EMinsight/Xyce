% Sandia National Laboratories is a multimission laboratory managed and
% operated by National Technology & Engineering Solutions of Sandia, LLC, a
% wholly owned subsidiary of Honeywell International Inc., for the U.S.
% Department of Energy’s National Nuclear Security Administration under
% contract DE-NA0003525.

% Copyright 2002-2023 National Technology & Engineering Solutions of Sandia,
% LLC (NTESS).


\begin{Device}

\symbol
{\includegraphics{translineSymbol}}

\device
\begin{alltt}
O<name> <A port (+) node> <A port (-) node>
+ <B port (+) node> <B port (-) node> [model name]
\end{alltt}

\model
\begin{alltt}
.MODEL <model name> LTRA R=<value> L=<value> C=<value>
+ G=<value> LEN=<value> [model parameters]
\end{alltt}

\examples
\begin{alltt}
Oline1 inp inn outp outn cable1
Oline2 inp inn outp outn cable1
\end{alltt}

\comments

The lossy transmission line, or LTRA, device is a two port (\texttt{A}
and \texttt{B}), bi-directional device. The \texttt{(+)} and \texttt{(-)} nodes
define the polarity of a positive voltage at a port.

\texttt{R}, \texttt{L}, \texttt{C}, and \texttt{G} are the resistance,
inductance, capacitance, and conductance of the transmission line per unit
length, respectively. \texttt{LEN} is the total length of the transmission
line. Supported configurations for the LTRA are \texttt{RLC}, \texttt{RC},
\texttt{LC} (lossless) and \texttt{RG}.

The lossy transmission line, or LTRA, device does not work with AC
analysis at this time.  LTRA models will need to be replaced with
lumped transmission line models (YTRANSLINE) when used in AC analysis.
The LTRA models do work correctly in harmonic balance simulation.
\end{Device}


%\paragraph{Device Parameters}
%\input{O_1_Device_Instance_Params}

\paragraph{Model Parameters}
\input{O_1_Device_Model_Params}

By default time step limiting is on in the LTRA. This means that
simulation step sizes will be reduced if required by the LTRA to
preserve accuracy. This can be disabled by setting
\texttt{NOSTEPLIMIT=1} and \texttt{TRUNCDONTCUT=1} on the
\texttt{.MODEL} line.

The option most worth experimenting with for increasing the speed of
simulation is \texttt{REL}. The default value of 1 is usually safe
from the point of view of accuracy but occasionally increases
computation time. A value greater than 2 eliminates all breakpoints
and may be worth trying depending on the nature of the rest of the
circuit, keeping in mind that it might not be safe from the viewpoint
of accuracy. Breakpoints may be entirely eliminated if the circuit
does not exhibit any sharp discontinuities. Values between 0 and 1 are
usually not required but may be used for setting many breakpoints.

\texttt{COMPACTREL} and \texttt{COMPACTABS} are tolerances that
control when the device should attempt to compact past history. This
can significantly speed up the simulation, and reduce memory usage,
but can negatively impact accuracy and in some cases may cause
problems with the nonlinear solver. In general this capability should
be used with linear type signals, such as square-wave-like
voltages. In order to activate this capability the general device
option \texttt{TRYTOCOMPACT=1} must be set, if it is not no history
compaction will be performed and the \texttt{COMPACT} options will be
ignored.

Example:

\texttt{.OPTIONS DEVICE TRYTOCOMPACT=1}

\paragraph{References}
See references \cite{Roychodhury:1994} and \cite{Spice3f5-user-guide} for more information
about the model.
