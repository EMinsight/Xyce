% Sandia National Laboratories is a multimission laboratory managed and
% operated by National Technology & Engineering Solutions of Sandia, LLC, a
% wholly owned subsidiary of Honeywell International Inc., for the U.S.
% Department of Energy’s National Nuclear Security Administration under
% contract DE-NA0003525.

% Copyright 2002-2024 National Technology & Engineering Solutions of Sandia,
% LLC (NTESS).

\newpage
\subsection{\texttt {.PREPROCESS REPLACEGROUND} (Ground Synonym)}

The purpose of ground synonym replacement is to treat nodes with the names
{\tt GND}, {\tt GND!}, {\tt GROUND} or any capital/lowercase variant thereof
as synonyms for node {\tt 0}.  The general invocation is
\index{\texttt{.PREPROCESS}!{\tt REPLACEGROUND}}

\begin{Command}

\format
.PREPROCESS REPLACEGROUND <bool>

\arguments

\begin{Arguments}

\argument{bool} 

If {\tt TRUE}, \Xyce~will treat all instances of {\tt GND}, {\tt GND!}, {\tt GROUND},
or any capital/lowercase variant thereof, as synonyms for node {\tt 0}.  If
{\tt FALSE}, \Xyce~will consider each term as a separate node.  Only one
{\tt .PREPROCESS REPLACEGROUND} statement is permissible in a given netlist file.

\end{Arguments}
\end{Command}

\newpage
\subsection{\texttt {.PREPROCESS REMOVEUNUSED} (Removal of Unused Components)}
\index{\texttt{.PREPROCESS}!{\tt REMOVEUNUSED}}
If a given netlist file contains devices whose terminals are all connected to
the same node (\emph{e.g.},~\texttt{R2~1~1~1M}), it may be desirable to remove such
components from the netlist before simulation begins.  This is the purpose of
the command

\begin{Command}

\format
.PREPROCESS REMOVEUNUSED [<value>]

\arguments

\begin{Arguments}

\argument{value}

is a list of components separated by commas.  The allowed values are 
\begin{basedescript}{
    \desclabelstyle{\multilinelabel}
    \desclabelwidth{1in}
    \renewcommand{\makelabel}[1]{\tt #1\hfill}}
\item[\tt C] Capacitor
\item[\tt D] Diode
\item[\tt I] Independent Current Source
\item[\tt L] Inductor
\item[\tt M] MOSFET
\item[\tt Q] BJT
\item[\tt R] Resistor
\item[\tt V] Independent Voltage Source
\end{basedescript}

\end{Arguments}

\examples
.PREPROCESS REMOVEUNUSED R,C

\normalfont
\texttt{.PREPROCESS} will attempt to search for all resistors and capacitors in
a given netlist file whose individual device terminals are connected to the
same node and remove these components from the netlist before simulation
ensues.  A list of components that are supported for removal is given above.
Note that for MOSFETS and BJTs, three terminals on each device (the gate,
source, and drain in the case of a MOSFET and the collector, base, and emitter
in the case of a BJT) must be the same for the device to be removed from the
netlist.  As before, only one \texttt{.PREPROCESS REMOVEUNUSED} line is allowed
in a given netlist file.

\end{Command}

\newpage
\subsection{\texttt {.PREPROCESS ADDRESISTORS} (Adding Resistors to Dangling Nodes)}
\index{\texttt{.PREPROCESS}!{\tt ADDRESISTORS}}

We refer to a \emph{dangling node} as a circuit node in one of the following
two scenarios:  either the node is connected to only one device terminal,
and/or the node has no DC path to ground.  If several such nodes exist in a
given netlist file, it may be desirable to automatically append a resistor of a
specified value between the dangling node and ground.  To add resistors to
nodes which are connected to only one device terminal, one may use the command

\begin{Command}

\format
.PREPROCESS ADDRESISTORS ONETERMINAL <value>

\arguments

\begin{Arguments}

\argument{value}

is the value of the resistor to be placed between
nodes with only one device terminal connection and ground.  For instance,
the command

\end{Arguments}

\examples
.PREPROCESS ADDRESISTORS ONETERMINAL 1G

\normalfont
will add resistors of value 1G between ground and nodes with only one
device terminal connection and ground.  The command

\examples

.PREPROCESS ADDRESISTORS NODCPATH <value>

\normalfont

acts similarly, adding resistors of value {\tt <VALUE>} between ground and
all nodes which have no DC path to ground.

The {\tt .PREPROCESS ADDRESISTORS} command is functionally different from
either of the prior {\tt .PREPROCESS} commands in the following way:  while
the other commands augment the netlist file for the current simulation, a
{\tt .PREPROCESS ADDRESISTORS} statement creates an auxiliary netlist file
which explicitly contains a set of resistors that connect dangling nodes to
ground.  If the original netlist file containing a {\tt .PREPROCESS
ADDRESISTORS} statement is called {\tt filename}, invoking \Xyce~on this file
will produce a file {\tt filename\_xyce.cir} which contains the resistors that
connect dangling nodes to ground.  One can then run \Xyce~on this file to
run a simulation in which the dangling nodes are tied to ground.  Note that,
in the original run on the file {\tt filename}, \Xyce~will continue to run a
simulation as usual after producing the file {\tt filename\_xyce.cir}, but this
simulation will {\em not} include the effects of adding resistors between the
dangling nodes and ground.  Refer to the \Xyce{} Users' Guide~\UsersGuide{} for more detailed
examples on the use of {\tt .PREPROCESS ADDRESISTOR} statements.

Note that it is possible for a node to have one device terminal connection
and, simultaneously, have no DC path to ground.  In this case, if both a
{\tt ONETERMINAL} and {\tt NODCPATH} command are invoked, only the resistor
for the {\tt ONETERMINAL} connection is added to the netlist; the
{\tt NODCPATH} connection is omitted.

As before, each netlist file is allowed to contain only one {\tt .PREPROCESS
ADDRESISTORS ONETERMINAL} and one {\tt .PREPROCESS ADDRESISTORS NODCPATH} line
each, or else \Xyce~will exit in error.

\end{Command}
