% Sandia National Laboratories is a multimission laboratory managed and
% operated by National Technology & Engineering Solutions of Sandia, LLC, a
% wholly owned subsidiary of Honeywell International Inc., for the U.S.
% Department of Energy’s National Nuclear Security Administration under
% contract DE-NA0003525.

% Copyright 2002-2024 National Technology & Engineering Solutions of Sandia,
% LLC (NTESS).


Simulation of electromechanical devices or magnetically driven machines may
require that \Xyce{} simulate the movement of an accelerated mass, that is, to
solve the second order initial value problem
\begin{eqnarray*}
\frac{d^2x}{dt} &= a(t) \\
x(0) &= x_0 \\
\dot{x}_0 &= v_0 \\
\end{eqnarray*}
where $x$ is the position of the object, $\dot{x}$ its velocity, and $a(t)$ the
acceleration.  In \Xyce{}, this simulation capability is provided by the
accelerated mass device.

\begin{Device}

\device
\begin{alltt}
YACC <name> <acceleration node> <velocity node> <position node>
+ [v0=<initial velocity>] [x0=<initial position>]
\end{alltt}

\examples
\begin{alltt}
* Simulate a projectile thrown upward against gravity
V1 acc 0 -9.8
R1 acc 0 1
YACC acc1 acc vel pos v0=10 x0=0
.print tran v(pos)
.tran 1u 10s
.end

* Simulate a damped, forced harmonic oscillator
* assuming K, c, mass, amplitude and frequency
* are defined in .param statements
B1 acc 0 V=\{(-K * v(pos) - c*v(vel))/mass
+            + amplitude*sin(frequency*TIME)\}
R1 acc 0 1
YACC acc2 acc vel pos v0=0 x0=0.4
.print tran v(pos)
.tran 1u 10s
.end
\end{alltt}

\comments

When used as in the examples, \Xyce{} will emit warning messages about the
\texttt{pos} and \texttt{vel} nodes not having a DC path to ground.  This is
normal and should be ignored. The position and velocity nodes should not be
connected to any real circuit elements.  Their values may, however, be used in
behavioral sources; this is done in the second example.

\end{Device}
