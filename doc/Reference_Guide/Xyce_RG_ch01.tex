% Sandia National Laboratories is a multimission laboratory managed and
% operated by National Technology & Engineering Solutions of Sandia, LLC, a
% wholly owned subsidiary of Honeywell International Inc., for the U.S.
% Department of Energy’s National Nuclear Security Administration under
% contract DE-NA0003525.

% Copyright 2002-2023 National Technology & Engineering Solutions of Sandia,
% LLC (NTESS).


\chapter{Introduction}
\label{Introduction}

\chapteroverview{Welcome to \XyceTitle{}}
{
The \XyceTM{} Parallel Electronic Simulator has been written to support, in a
rigorous manner, the simulation needs of the Sandia National Laboratories
\index{Sandia National Laboratories} electrical designers.  It is targeted 
specifically to run on large-scale parallel computing
\index{parallel!computing} platforms but also runs well on a variety
of architectures including single processor workstations.  It also aims to
support a variety of devices and models specific to Sandia needs.
}
\section{Overview}
\label{Overview}

This document is intended to complement the \Xyce{} Users' 
Guide\UsersGuide.  It contains comprehensive, detailed
information about a number of topics pertinent to the usage of \Xyce{}.  
Included in this document is a
netlist reference for the input-file commands and elements supported 
within \Xyce{}; a command line reference, which describes the 
available command line arguments for \Xyce{}; and quick-references for 
users of other circuit codes, such as Orcad's PSpice~\cite{PSpiceUG:1998}.

\section{How to Use this Guide}
\label{HowTo_Guide}

This guide is designed so you can quickly find the information you need to 
use \Xyce{}.  It assumes that you are familiar with basic Unix-type 
commands, how \index{Unix}Unix manages applications and files to 
perform routine tasks (e.g., starting applications, opening files and 
saving your work).  Note that while Windows versions of \Xyce{} are 
available, they are command-line programs meant to be run under the 
\emph{Command Prompt}, and are used almost identically to their Unix counterparts.

\section{Typographical conventions}
Before continuing in this Reference Guide, it is important to understand 
the terms and typographical conventions used.  Procedures for performing 
an operation are generally indicated with the following typographical 
conventions.

\begin{table}[htbp]
  \caption{\Xyce{} typographical conventions.}
  \begin{tabularx}{\linewidth}{|Y|Y|Y|}
    \rowcolor{XyceDarkBlue} \color{white}\bf Notation & \color{white}\bf
    Example & \color{white}\bf Description \\ \hline

    \texttt{Typewriter text} & \texttt{mpirun -np 4}
    & Commands entered
    from the keyboard on the command line or text entered in a netlist. \\
    \hline

    \textrmb{Bold Roman Font} & Set nominal temperature using the
    \textrmb{TNOM} option. & SPICE-type parameters used in models, etc. \\
    \hline

    \cellcolor[gray]{0.75} Gray Shaded Text & \cellcolor[gray]{0.75} DEBUGLEVEL
    & Feature that is designed primarily for use by \Xyce{}
    developers. \\ \hline

    \texttt{[text in brackets]} & \texttt{Xyce [options] <netlist>} & Optional parameters. \\ \hline

    \texttt{<text in angle brackets>} & \texttt{Xyce [options] <netlist>} &
    Parameters to be inserted by the user. \\ \hline

    \texttt{<object with asterisk>*} & \texttt{K1 <ind. 1> [<ind. n>*]} &
    Parameter that may be multiply specified. \\ \hline

    \texttt{<TEXT1|TEXT2>}&
    \texttt{.PRINT TRAN}
    \verb-+     DELIMITER=<TAB|COMMA>- & Parameters that may only take specified values. \\ \hline

  \end{tabularx}
\end{table}

