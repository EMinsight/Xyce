% Sandia National Laboratories is a multimission laboratory managed and
% operated by National Technology & Engineering Solutions of Sandia, LLC, a
% wholly owned subsidiary of Honeywell International Inc., for the U.S.
% Department of Energy’s National Nuclear Security Administration under
% contract DE-NA0003525.

% Copyright 2002-2023 National Technology & Engineering Solutions of Sandia,
% LLC (NTESS).


%%
%% BJT Description Table
%%

\begin{Device}\label{Q_DEVICE}

\symbol
{\includegraphics{npnSymbol}}
{\includegraphics{pnpSymbol}}

\device
\begin{alltt}
Q<name> <collector node> <base node> <emitter node>
 + [substrate node] <model name> [area value]

Q<name> <collector node> <base node> <emitter node>
 + [thermal node] <VBIC 1.3 3-terminal model name>

Q<name> <collector node> <base node> <emitter node>
 + <substrate> [thermal node] <VBIC 1.3 4-terminal model name>

 Q<name> <collector node> <base node> <emitter node>
 + <substrate> <thermal node> <HICUM model name>
\end{alltt}

\model
\begin{alltt}
.MODEL <model name> NPN [model parameters]
.MODEL <model name> PNP [model parameters]
\end{alltt}

\examples
\begin{alltt}
Q2 10 2 9 PNP1
Q12 14 2 0 1 NPN2 2.0
Q6 VC 4 11 [SUB] LAXPNP
Q7 Coll Base Emit DT VBIC13MODEL2
Q8 Coll Base Emit VBIC13MODEL3 SW\_ET=0
Q9 Coll Base Emit Subst DT VBIC13MODEL4
Q10 Coll Base Emit Subst DT HICUMMMODEL1
\end{alltt}

\parameters
\begin{Parameters}
\param{substrate node}
  Optional and defaults to ground. Since \Xyce{} permits alphanumeric
  node names and because there is no easy way to make a distinction between
  these and the model names, the name (not a number) used for the substrate
  node must be enclosed in square brackets \texttt{[ ]}.  Otherwise, nodes
  would be interpreted as model names. See the fourth example above.

\param{area value}
  The relative device area with a default value of 1.

\end{Parameters}

\comments
The BJT is modeled as an intrinsic transistor using ohmic resistances in series
with the collector (RC/area), with the base (value varies with current, see BJT
equations) and with the emitter (RE/area).For model parameters with optional
names, such as VAF and VA (the optional name is in parentheses), either may be
used.For model types NPN and PNP, the isolation junction capacitance is
connected between the intrinsic-collector and substrate nodes. This is the same
as in SPICE and works well for vertical IC transistor structures.

\textbf{Only the VBIC 1.3 model is available in \Xyce{} 6.11 and
  later.}  The VBIC 1.3 model is provided in both 3-terminal (Q level
  11) and 4-terminal (Q level 12) variants, both supporting
  electrothermal and excess-phase effects.  These variants of the Q line
  are shown in the fourth through sixth examples above. VBIC 1.3
  instance lines have three or four required nodes, depending on model
  level, and an \emph{optional} ``dt'' node.  The first three are the
  normal collector, base,and emitter. In the level 12 (4-terminal) the
  fourth node is the substrate, just as for the level 1 BJT.  If the
  optional ``dt'' node is specified for either variant, it can be used
  to print the local temperature rise due to self-heating, and could
  possibly be used to model coupled heating effects of several VBIC
  devices.  It is, however, unnecessary to specify a ``dt'' node just
  to print the local temperature rise, because when this node is
  omitted from the instance line it simply becomes and internal node,
  and may still be printed using the syntax
  \texttt{N(instancename:dt)}.  For the ``Q8'' example above, one
  could print \texttt{N(Q8:dt)}.

As of release 6.10 of Xyce, the VBIC 1.3 3-terminal device (Q level
11) has been the subject of extensive optimization, and runs much
faster than in previous releases.

\textbf{ The HICUM models require both a substrate and thermal node.}

\end{Device}


% BJT model schematic.
\begin{figure}[ht]
  \centering
  \scalebox{0.6}
  {\includegraphics{bjtSchematic}}
  \caption[BJT model schematic]{BJT model schematic.  Adapted from
reference~\cite{PSpiceUG:1998}. \label{figBJTschematic}}
\end{figure}

\paragraph{BJT Level selection}

\Xyce{} supports the level 1 BJT model, which is based on the
documented standard SPICE 3F5 BJT model, but was coded independently
at Sandia.  It is mostly based on the classic Gummel-Poon BJT
model~\cite{GummelPoon}.

Two variants of the VBIC model are provided as BJT levels 11  and 12.
Levels 11 and 12 are the
3-terminal and 4-terminal variants of the VBIC 1.3.

An experimental release of the FBH HBT\_X model version
2.1\cite{Rudolph_documentationof} is provided as BJT level 23.

Both the HICUM/L0 (level 230) and HICUM/L2 (level 234) models are also
provided (\url{https://www.iee.et.tu-dresden.de/iee/eb/hic_new/hic_start.html}).

The MEXTRAM\cite{MEXTRAM_home} BJT model version 504.12.1 model is provided.
Two variants of this model are available: the level 504 model without
self-heating and without external substrate node, and the level 505 model with
self heating but without external substrate node.  The level 505 instance line
requires a fourth node for the 'dt' node, similar to the usage in all of the
VBIC models (levels 11-12), but is otherwise identical to the level 504 model.

\paragraph{BJT Power Calculations}
Power dissipated in the transistor is calculated with
$|I_{B}*V_{BE}|+|I_{C}*V_{CE}|$, where $I_{B}$ is the base current, $I_{C}$ is
the collector current, $V_{BE}$ is the voltage drop between the base and the
emitter and $V_{CE}$ is the voltage drop between the collector and the emitter.
This formula may differ from other simulators.

\subsubsection{The Level 1 Model}
\paragraph{BJT Equations}\label{bjt_equations}
The Level 1 BJT implementation within \Xyce{} is based on \cite{Fjeldly:1998}.
The equations in this section describe an NPN transistor. For the PNP device,
reverse the signs of all voltages and currents.  The equations use the
following variables:
%insert variables here
\begin{eqnarray*}
V_{be} & = & \mbox{intrinsic base-intrinsic emitter voltage} \\
V_{bc} & = & \mbox{intrinsic base-intrinsic collector voltage} \\
V_{bs} & = & \mbox{intrinsic base-substrate voltage} \\
V_{bw} & = & \mbox{intrinsic base-extrinsic collector voltage
(quasi-saturation only)} \\
V_{bx} & = & \mbox{extrinsic base-intrinsic collector voltage} \\
V_{ce} & = & \mbox{intrinsic collector-intrinsic emitter voltage} \\
V_{js} & = & \mbox{(NPN) intrinsic collector-substrate voltage} \\
&        = & \mbox{(PNP) intrinsic substrate-collector voltage} \\
% THIS DOESN'T EXIST IN XYCE
%&        = & \mbox{(LPNP) intrinsic base-substrate voltage} \\
V_{t}  & = & \mbox{$kT/q$ (thermal voltage)} \\
V_{th} & = & \mbox{threshold voltage} \\
k      & = & \mbox{Boltzmann's constant} \\
q      & = & \mbox{electron charge} \\
T      & = & \mbox{analysis temperature (K)} \\
T_{0}  & = & \mbox{nominal temperature (set using \textrmb{TNOM} option)}
\end{eqnarray*}
Other variables are listed above in BJT Model Parameters.

\subparagraph{DC Current}
The BJT model is based on the Gummel and Poon model~\cite{Grove:1967} where the
different terminal currents are written
%insert equations here
\begin{eqnarray*}
I_{e} & = & -I_{cc} - I_{be} + I_{re} + (C_{dife} +C_{de})\frac{dV_{be}}{dt} \\
I_{c} & = & -I_{cc} + I_{bc} - I_{rc} - (C_{difc}+ C_{dc})\frac{dV_{bc}}{dt} \\
I_b & = & I_e -I_c
\end{eqnarray*}
Here, $C_{dife}$ and $C_{difc}$ are the capacitances related to the hole
charges per unit area in the base, $Q_{dife}$ and $Q_{difc}$,
affiliated with the electrons introduced across the emitter-base and
collector-base junctions, respectively.  Also, $C_{be}$ and $C_{bc}$ are the
capacitances related to donations to the hole charge of the base, $Q_{be}$ and
$Q_{bc}$,  affiliated with the differences in the depletion regions of the
emitter-base and collector-base junctions, respectively.  The
intermediate currents used are defined as
%insert equations here
\begin{eqnarray*}
-I_{be} & = & \mathbf{\frac{IS}{BF}} \left[\exp \left(\frac{V_{be}}
{\mathbf{NF}V_{th}} \right) -1 \right] \\
-I_{cc} & = & \frac{Q_{bo}}{Q_{b}}\mathbf{IS} \left[\exp \left(\frac{V_{be}}
{\mathbf{NF} V_{th}} \right) -
\exp \left(\frac{V_{bc}}{\mathbf{NF}V_{th}} \right) \right] \\
-I_{bc} & = & \mathbf{\frac{IS}{BR}} \left[\exp \left(\frac{V_{bc}}
{\mathbf{NR}V_{th}} \right) - 1 \right] \\
I_{re} & = & \mathbf{ISE} \left[\exp \left(\frac{V_{be}}
{\mathbf{NE} V_{th}} \right) - 1 \right] \\
I_{rc} & = & \mathbf{ISC} \left[\exp \left(\frac{V_{bc}}
{\mathbf{NC}V_{th}} \right) - 1 \right]
\end{eqnarray*}
where the last two terms are the generation/recombination currents related to
the emitter and collector junctions, respectively.  The charge $Q_{b}$ is the
majority carrier charge in the base at large injection levels and is a key
difference in the Gummel-Poon model over the earlier Ebers-Moll model.  The
ratio $Q_b/Q_{bo}$ (where $Q_{bo}$ represents the zero-bias base charge, i.e.
the value of $Q_b$ when $V_{be}=V_{bc}=0$) as computed by \Xyce{} is given by
\[\frac{Q_b}{Q_{bo}} = \frac{q_1}{2}\left(1+\sqrt{1+4q_2}\right)\]
where
\begin{eqnarray*}
q_1 & = & \left(1-\frac{V_{be}}{\mathbf{VAR}}-\frac{V_{bc}}{\mathbf{VAF}}
\right)^{-1} \\
q_2 & = & \frac{\mathbf{IS}}{\mathbf{IKF}}\left[\exp\left(\frac{V_{be}}
{\mathbf{NF}V_{th}}\right)-1\right] + \frac{\mathbf{IS}}{\mathbf{IKR}}
\left[\exp\left(\frac{V_{bc}}{\mathbf{NR}V_{th}}\right)-1\right]
\end{eqnarray*}

\subparagraph{Capacitance Terms}
The capacitances listed in the above DC $I-V$ equations each consist of a
depletion layer capacitance $C_{d}$ and
a diffusion capacitance $C_{dif}$.  The first is given by
\[
C_d = \left\{
\begin{array}{ll}
\mathbf{CJ} \left(1 - \frac{V_{di}}{\mathbf{VJ}} \right)^{\mathbf{-M}} &
V_{di} \leq \mathbf{FC \cdot VJ} \\
\mathbf{CJ} \left(1 - \mathbf{FC} \right)^{-(1+\mathbf{M})} \
\left[1 - \mathbf{FC}(1 + \mathbf{M}) + \mathbf{M}
\frac{V_{di}}{\mathbf{VJ}} \right]
& V_{di} > \mathbf{FC \cdot VJ}
\end{array}
\right. \]

where $\mathbf{CJ}=\mathbf{CJE}$ for $C_{de}$, and where $\mathbf{CJ}=
\mathbf{CJC}$ for $C_{dc}$.
The diffusion capacitance (sometimes referred to as the transit time
capacitance) is
\[
C_{dif} = \mathbf{TT} G_d = \mathbf{TT} \frac{dI}{dV_{di}}
\]
where $I$ is the diode DC current given, $G_d$ is the corresponding junction
conductance, and where $\mathbf{TT}=\mathbf{TF}$ for $C_{dife}$ and
$\mathbf{TT}=\mathbf{TR}$ for $C_{difc}$.

\subparagraph{Temperature Effects}
SPICE temperature effects are default, but all levels of the BJT have a more
advanced temperature compensation available.  By specifying
\texttt{TEMPMODEL=QUADRATIC} in the netlist, parameters can be interpolated
quadratically between measured values extracted from data.  In the BJT, IS and
ISE are interpolated logarithmically because they can change over an order of
magnitude or more for temperature ranges of interest.  See the
Section~\ref{Model_Interpolation} for more details on how to include quadratic
temperature effects.

% \subparagraph{Noise}
% Noise is calculated with a 1.0 Hz bandwidth using the following spectral power
% densities (per unit bandwidth).

% Thermal Noise due to Parasitic Resistance: $I_{n}^{2} =
% \frac{4kT}{\mathbf{RS}/area}$

% Intrinsic Diode Shot and Flicker Noise: $I_{n}^{2} = 2qI_{D} +
% \mathbf{KF}\cdot\frac{I_{D}^{\mathbf{AF}}}{\omega}$

For further information on BJT models, see~\cite{Grove:1967}.  For a thorough
description of the U.C. Berkeley SPICE models see
Reference~\cite{Antognetti:1988}.

\subsubsection{VBIC Temperature Considerations}
\index{VBIC (temperature considerations)}
The VBIC (Q levels 11 and 12) model both support a self-heating
model.  The model works by computing the power dissipated by all
branches of the device, applying this power as a flow through a small
thermal network consisting of a power flow (``current'') source
through a thermal resistance and thermal capacitance, as shown in
Figure~\ref{vbicthermal}.  The circuit node DT will therefore be the
``thermal potential'' (temperature) across the parallel thermal
resistance and capacitance.  This temperature is the temperature rise
due to self heating of the device, which is added to the ambient
temperature and \texttt{TRISE} parameter to obtain the device
operating temperature.
\begin{figure}
  \centering
  \scalebox{0.3}
  {\includegraphics{VBIC_Thermal_Net}}
  \caption[VBIC thermal network schematic]{VBIC thermal network  schematic.}
  \label{vbicthermal}
\end{figure}

In VBIC 1.3, the dt node is optional on the netlist line.  If not
given, the dt node is used internally for thermal effects
calculations, but not accessible from the rest of the netlist.  The
VBIC 1.3 provides an instance parameter \texttt{SW\_ET} that may be
set to zero to turn off electrothermal self-heating effects.  When set
to zero, no thermal power is sourced into the dt node.  This parameter
defaults to 1, meaning that thermal power is computed and flows into
dt even when dt is unspecified on the netlist and remains an internal
node.

In VBIC 1.3, setting RTH to zero does {\em NOT\/} disable the
self-heating model, and does not short the dt node to ground, even
though one might expect that to be the behavior.  Rather, it simply
removes the RTH resistor from the equivalent circuit of
figure~\ref{vbicthermal} and leaves the dt node floating.  This is an
important point to recognize when using the VBIC.

If a node name is given as the fourth node of a VBIC \Xyce{} will emit
warnings about the node not having a DC path to ground and being
connected to only one device.  These warnings may safely be ignored,
and are a harmless artifact of Xyce's connectivity checker.  It is
possible to silence this warning by adding a very large resistance
between the dt node and ground --- 1GOhm or 1TOhm are effectively the
same as leaving the node floating, and will satisfy the connectivity
checker's tests.  This used to be the recommended means of silencing
the connectivity checker for the VBIC 1.2 where dt was a required
node, but it is safe {\em if and only if a nonzero\/} \texttt{RTH}
{\em value is specified for the device.}  If, however, RTH is zero,
then dt would otherwise be floating and your external resistance now
becomes the primary path for thermal power flow; rather than turning
off self-heating effects, it will be as if you had set RTH to a very
large value.  We therefore recommend that you not tie the dt node to
ground via a resistor, and if you are not using it to connect VBIC
devices together via a thermal network, simply leave off the dt node
to silence the connectivity checker warning.  Turn off self-heating
effects ONLY by setting the \texttt{SW\_ET} instance parameter to zero.

Users of earlier versions of \Xyce{} may have been using the VBIC 1.2
model that was removed in release 6.11.  All netlists containing the
old level=10 VBIC 1.2 model must be modified to run in \Xyce{} 6.11
and later.  The following points should be observed when converting an
old VBIC 1.2 netlist and model card to VBIC 1.3.

\begin{itemize}
  \item Generally speaking, most VBIC 1.2 model cards can be converted
    to VBIC 1.3 model cards by the simple substitution of
    \texttt{level=11} for \texttt{level=10}, with the following provisos.

  \item VBIC 1.2 in \Xyce{} 6.10 and earlier did not support excess
    phase effects, and so the \texttt{TD} parameter governing excess
    phase was ignored.

    The \Xyce{} team has observed that some users' VBIC 1.2 parameter
    extractions have a non-zero value for the \texttt{TD} parameter.
    The impact of this is twofold:
    \begin{itemize}
      \item Circuits that use such model cards with only the level
        number changed will likely not produce identical results when
        compared to simulation results of older versions of \Xyce{}
        using VBIC 1.2 due to the excess phase effects.  If strict
        comparison between VBIC 1.3 runs with \Xyce{} 6.11 or later
        against older runs with VBIC 1.2 is desired, change the
        \texttt{TD} parameter to zero.  This will disable the excess
        phase effects and make VBIC 1.3 equivalent to the VBIC 1.2
        that was previously provided.
      \item The \Xyce{} team has seen some instances where the
        previously ignored \texttt{TD} parameter value is such that
        \Xyce{} will fail to converge when the equivalent VBIC 1.3
        model is substituted.  The VBIC 1.2 behavior can be recovered
        by setting the model parameter \texttt{TD} to zero, which will
        disable the excess phase effect in VBIC 1.3.  We can only
        suggest that the model card be re-extracted using VBIC 1.3 to
        determine the correct value for \texttt{TD}.
    \end{itemize}

  \item VBIC 1.2 had a model parameter called \texttt{DTEMP}, which
    \Xyce{} also recognized on the instance line.  In VBIC 1.3 this
    parameter has been replaced by another called \texttt{TRISE},
    which is only an instance parameter, and is unrecognized in model
    cards.  VBIC 1.3 also recognizes \texttt{DTEMP} on the instance
    line as an alias for \texttt{TRISE}.  If you had been specifying
    \texttt{DTEMP} in your VBIC 1.2 model cards, you will need to move it
    to the instance line instead in order for the parameter to be
    properly recognized by both VBIC 1.2 and VBIC 1.3.
  \item Turning off self-heating effects in VBIC 1.2 was done by
    grounding the mandatory dt node.  This is not the recommended way
    of disabling self-heating in VBIC 1.3.  To disable self-heating,
    set the \texttt{SW\_ET} parameter to zero on the instance line (as
    is done in the ``Q8'' example above).
  \item If not using the dt node as a way of thermally coupling
    devices to each other, leave it off of VBIC 1.3 instance lines,
    allowing it to be an internal variable irrespective of whether
    self-heating is enabled or not.  This will silence any connectivity
    warnings from Xyce.  Since the dt node may be printed using the
    N() syntax even when internal, it is unnecessary to put a dt node
    on the instance line just to print the local temperature rise due
    to self-heating.  The only reasons to include it on the instance
    line would be for backward compatibility to VBIC 1.2 netlists, or
    to implement a thermal coupling network between devices.
  \item Finally, VBIC 1.3 introduced a number of constraints on model
    parameters that the previous version did not.  \Xyce{} will emit
    warnings if any parameter on a VBIC 1.3 model card is out of the
    range specified by the VBIC 1.3 authors.  These warnings should
    not be ignored lightly, as they indicate that the model is being
    used in a manner not intended by its authors.  They are generally
    a sign that the model may not be well-behaved, and may indicate an
    improperly extracted model card.

\end{itemize}

\subsubsection{Level 1 BJT Tables}
% This table was generated by Xyce:
%   Xyce -doc Q 1
%
\index{bipolar junction transistor!device instance parameters}
\begin{DeviceParamTableGenerated}{Bipolar Junction Transistor Device Instance Parameters}{Q_1_Device_Instance_Params}
AREA & Relative device area & -- & 1 \\ \hline
DTEMP & Device delta temperature & $^\circ$C & 0 \\ \hline
IC1 & Vector of initial values: Vbe,Vce. Vbe=IC1 & V & 0 \\ \hline
IC2 & Vector of initial values: Vbe,Vce. Vce=IC2 & V & 0 \\ \hline
M & multiplicity factor & -- & 1 \\ \hline
OFF & Initial condition of no voltage drops accross device & logical (T/F) & false \\ \hline
TEMP & Device temperature & $^\circ$C & Ambient Temperature \\ \hline
\end{DeviceParamTableGenerated}

\input{Q_1_Device_Model_Params}

\subsubsection{Level 11 and 12 BJT Tables (VBIC 1.3)}
The VBIC 1.3 (level 11 transistor for 3-terminal, level 12 for
4-terminal) supports a number of instance parameters that are not
available in the VBIC 1.2.  The level 11 and level 12 differ only by
the number of required nodes.  The level 11 is the 3-terminal device,
having only collector, base, and emitter as required nodes.  The level
12 is the 4-terminal device, requiring collector, base, emitter and
substrate nodes.  Both models support an optional 'dt' node as their
last node on the instance line.

\textbf{Model cards extracted for the VBIC 1.2 will mostly work with the VBIC
1.3,  with one notable exception:} in VBIC 1.2 the \texttt{DTEMP} parameter was
a model parameter, and \Xyce{} allowed it also to be specified on the instance
line, overriding whatever was specified in the model.  This parameter was
replaced in VBIC 1.3 with the \texttt{TRISE} parameter, which is {\em only\/}
an instance parameter.  \texttt{DTEMP} and \texttt{DTA} are both supported as
aliases for the \texttt{TRISE} instance parameter.

\input{Q_11_Device_Instance_Params}
\input{Q_11_Device_Model_Params}
\clearpage
\input{Q_12_Device_Instance_Params}
\input{Q_12_Device_Model_Params}
\clearpage

\subsubsection{Level 23 BJT Tables (FBH HBT\_X)}
\input{Q_23_Device_Instance_Params}
\input{Q_23_Device_Model_Params}
\clearpage

\subsubsection{Level 230 BJT Tables (HICUM/L0)}
The HICUM/L0 device supports output of the internal variables in
table~\ref{Q_230_OutputVars} on the \texttt{.PRINT} line of a netlist.
To access them from a print line, use the syntax
\texttt{N(<instance>:<variable>)} where ``\texttt{<instance>}'' refers to the
name of the specific HICUM/L0 Q device in your netlist.

\input{Q_230_Device_Instance_Params}
\input{Q_230_Device_Model_Params}
\input{Q_230_OutputVars}
\clearpage

\subsubsection{Level 234 BJT Table (HICUM/L2)}
\textbf{NOTE:} The HICUM/L2 model has no instance parameters.
The HICUM/L2 device supports output of the internal variables in
table~\ref{Q_234_OutputVars} on the \texttt{.PRINT} line of a netlist.
To access them from a print line, use the syntax
\texttt{N(<instance>:<variable>)} where ``\texttt{<instance>}'' refers to the
name of the specific HICUM/L2 Q device in your netlist.
\input{Q_234_Device_Model_Params}
\input{Q_234_OutputVars}
\clearpage

\subsubsection{Level 504 and 505 BJT Tables (MEXTRAM)}


The MEXTRAM device supports output of the internal variables in
tables~\ref{Q_504_OutputVars} and~\ref{Q_504_OutputVars} on the \texttt{.PRINT} line of a netlist.
To access them from a print line, use the syntax
\texttt{N(<instance>:<variable>)} where ``\texttt{<instance>}'' refers to the
name of the specific MEXTRAM Q device in your netlist.

\input{Q_504_Device_Instance_Params}
\input{Q_504_Device_Model_Params}
\input{Q_504_OutputVars}
\clearpage

\input{Q_505_Device_Instance_Params}
\input{Q_505_Device_Model_Params}
\input{Q_505_OutputVars}
\clearpage

