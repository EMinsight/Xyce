% Sandia National Laboratories is a multimission laboratory managed and
% operated by National Technology & Engineering Solutions of Sandia, LLC, a
% wholly owned subsidiary of Honeywell International Inc., for the U.S.
% Department of Energy’s National Nuclear Security Administration under
% contract DE-NA0003525.

% Copyright 2002-2023 National Technology & Engineering Solutions of Sandia,
% LLC (NTESS).


\begin{Device}

\device
\begin{alltt}
S<name> <(+) switch node> <(-) switch node>
+ <(+) control node> <(-) control node>
+ <model name> [ON] [OFF]

W<name> <(+) switch node> <(-) switch node>
+ <control node voltage source>
+ <model name> [ON] [OFF]
\end{alltt}

\model
\begin{alltt}
.MODEL <model name> VSWITCH [model parameters]
.MODEL <model name> ISWITCH [model parameters]
\end{alltt}

\examples
\begin{alltt}
S1 21 23 12 10 SMOD1
SSET 15 10 1 13 SRELAY
W1 1 2 VCLOCK SWITCHMOD1
W2 3 0 VRAMP SM1 ON
\end{alltt}

\comments

The voltage- or current-controlled switch is a particular type of
controlled resistor. This model is designed to help reduce numerical
issues. See Special Considerations below.

The resistance between the \texttt{<(+) switch node>} and the
\texttt{<(-) switch node>} is dependent on either the voltage between
the \texttt{<(+) control node>} and the \texttt{<(-) control node>} or
the current through the control node voltage source. The resistance
changes in a continuous manner between the \texttt{RON} and
\texttt{ROFF} model parameters.

No resistance is inserted between the control nodes.  It is up to the
user to make sure that these nodes are not floating.

Even though evaluating the switch model is computationally
inexpensive, for transient analysis \Xyce{} steps through the
transition section using small time-steps in order to calculate the
waveform accurately. Thus, a circuit with many switch transitions can
result in lengthy run times.

The ON and OFF parameters are used to specify the initial state of the
switch at the first step of the operating point calculation; this does
not force the switch to be in that state, it only gives the operating
point solver an initial state to work with.  If it is known that the
switch should be in a particular state in the operating point it could
help convergence to specify one of these keywords.

The power dissipated in the switch is calculated with $I \cdot \Delta V$ 
where the voltage drop is calculated as $(V_+ - V_-)$ and positive current 
flows from $V_+$ to $V_-$.  This will essentially be the power dissipated
in either \texttt{RON} or \texttt{ROFF}, since the switch is a particular 
type of controlled resistor.

\textbf{Note:} The voltage- and current-controlled switches specified
in this manner are converted at parse time into equivalent ``generic''
switches.

\end{Device}

\pagebreak

\paragraph{Model Parameters}
% This table was generated by Xyce:
%   Xyce -doc S 1
%
\index{controlled switch!device model parameters}
\begin{DeviceParamTableGenerated}{Controlled Switch Device Model Parameters}{S_1_Device_Model_Params}
IHOFF & Off current & A & 0 \\ \hline
IHON & On current with hysteresis & A & 0.001 \\ \hline
IOFF & Off current with hysteresis & A & 0 \\ \hline
ION & On current & A & 0.001 \\ \hline
OFF & Off control value & -- & 0 \\ \hline
OFFH & Off control value with hysteresis & -- & 0 \\ \hline
ON & On control value & -- & 1 \\ \hline
ONH & On control value with hysteresis & -- & 1 \\ \hline
ROFF & Off resistance & $\mathsf{\Omega}$ & 1e+06 \\ \hline
RON & On resistance & $\mathsf{\Omega}$ & 1 \\ \hline
VHOFF & Off voltage with hysteresis & V & 0 \\ \hline
VHON & On voltage with hysteresis & V & 1 \\ \hline
VOFF & Off voltage & V & 0 \\ \hline
VON & On voltage & V & 1 \\ \hline
\end{DeviceParamTableGenerated}


\paragraph{Special Considerations}

\begin{XyceItemize}
\item Due to numerical limitations, \Xyce{} can only manage a dynamic range of
  approximately 12 decades.  Thus, it is recommended the user limit the ratio
  \textrmb{ROFF}/\textrmb{RON} to less than $10^{12}$.  This soft limitation is not enforced by the code, and larger ratios might converge for some problems.
\item Do not set \textrmb{RON} to 0.0, as the code computes the ``on'' conductance as the inverse of \textrmb{RON}.  Using 0.0 will cause the simulation to fail when this invalid division results in an infinite conductance.  Use a very small, but non-zero, on resistance instead.
\item Furthermore, it is a good idea to limit the narrowness of the transition
  region. This is because in the transition region, the switch has gain and the
  narrower the region, the higher the gain and the more potential for numerical
  problems.  The smallest value recommended for $\|\mathbf{VON - VOFF}\|$ or
  $\|\mathbf{ION - IOFF}\|$ is $1\times10^{-12}$.  This recommendation is not a restriction, and you might find for some problems that narrower transition regions might work well.
\end{XyceItemize}

\paragraph{Controlled switch equations}
The equations in this section use the following variables:
\[
\begin{array}{rllll}
R_s & = & \mbox{switch resistance} \\
V_c & = & \mbox{voltage across control nodes} \\
I_c & = & \mbox{current through control node voltage source} \\
L_m & = & \mbox{log-mean of resistor values} & = &
\ln \left(\sqrt{\mathbf{RON \cdot ROFF}} \right) \\
L_r & = & \mbox{log - ratio of resistor values} & = &
\ln \left(\mathbf{RON / ROFF} \right) \\
V_d & = & \mbox{difference of control voltages} & = & \mathbf{VON - VOFF} \\
I_d & = & \mbox{difference of control currents} & = & \mathbf{ION - IOFF} \\
\end{array}
\]

\subparagraph{Switch Resistance}

To compute the switch resistance, \Xyce{} first calculates the
``switch state'' $S$ as $S=(V_c-\mathbf{VOFF})/V_d$ or
$S=(I_c-\mathbf{IOFF})/I_d$.  The switch resistance is then:
\[ R_s = \left\{ \begin{array}{ll}
\mathbf{RON}, & S \geq 1.0 \\
\mathbf{ROFF}, & S \leq 0.0 \\
\exp \left(L_m + 0.75L_r(2S-1) - 0.25L_r(2S-1)^3 \right),
& 0< S < 1
\end{array}
\right.
\]

%\subparagraph{Noise}
%Noise is computed using a 1.0 Hz bandwidth.  The voltage-controlled switch
%produces thermal noise as though it were a resistor with the resistance a
%switch has at its bias point.  It uses a spectral power density (per unit
%bandwidth):
%\[
%i^2 = 4kT/R_s
%\]
