% Sandia National Laboratories is a multimission laboratory managed and
% operated by National Technology & Engineering Solutions of Sandia, LLC, a
% wholly owned subsidiary of Honeywell International Inc., for the U.S.
% Department of Energy’s National Nuclear Security Administration under
% contract DE-NA0003525.

% Copyright 2002-2022 National Technology & Engineering Solutions of Sandia,
% LLC (NTESS).


\begin{Device}\label{M_DEVICE}

\symbol
{\includegraphics{nmosSymbol}}
{\includegraphics{pmosSymbol}}

\device
\begin{alltt}
M<name> <drain node> <gate node> <source node>
+ <bulk/substrate node> <model name>
+ [L=<value>] [W=<value>]
+ [AD=<value>] [AS=<value>]
+ [PD=<value>] [PS=<value>]
+ [NRD=<value>] [NRS=<value>]
+ [M=<value] [IC=<value, ...>]
\end{alltt}

\vbox{\hrulefill}
\item[Special Form (BSIMSOI)]
\begin{alltt}
M<name> <drain node> <gate node> <source node>
+ <substrate node (E)>
+ [<External body contact (P)>]
+ [<internal body contact (B)>]
+ [<temperature node (T)>]
+ <model name>
+ [L=<value>] [W=<value>]
+ [AD=<value>] [AS=<value>]
+ [PD=<value>] [PS=<value>]
+ [NRD=<value>] [NRS=<value>] [NRB=<value>]
+ [BJTOFF=<value>]
+ [IC=<val>,<val>,<val>,<val>,<val>]
+ [RTH0=<val>] [CTH0=<val>]
+ [NBC=<val>] [NSEG=<val>] [PDBCP=<val>] [PSBCP=<val>]
+ [AGBCP=<val>] [AEBCP=<val>] [VBSUSR=<val>] [TNODEOUT]
+ [FRBODY=<val>] [M=<value>]
\end{alltt}
\vbox{\hrulefill}

\item[Special Form (MVS)]
\begin{alltt}
M<name> <drain node> <gate node> <source node> <model name>
\end{alltt}

\item[Special Form (PSP103 with self-heating)]
\begin{alltt}
M<name> <drain node> <gate node> <source node> <bulk node> <dt node> <model name> [instance parameters]
\end{alltt}

\model
\begin{alltt}
.MODEL <model name> NMOS [model parameters]
.MODEL <model name> PMOS [model parameters]
\end{alltt}

\examples
\begin{alltt}
M5 4 12 3 0 PNOM L=20u W=10u
M3 5 13 10 0 PSTRONG
M6 7 13 10 0 PSTRONG M=2
M8 10 12 100 100 NWEAK L=30u W=20u
+ AD=288p AS=288p PD=60u PS=60u NRD=14 NRS=24
\end{alltt}

\parameters

\begin{Parameters}

\param{\vbox{\hbox{L\hfil}\hbox{M\hfil}}}

The MOSFET channel length and width that are decreased to get the actual
channel length and width. They may be given in the device
\texttt{.MODEL} or \texttt{.OPTIONS} statements. The value in the device
statement overrides the value in the model statement, which overrides
the value in the \texttt{.OPTIONS} statement. If \texttt{L} or \texttt{W}
values are not given, their default value is 100~$\mu$m.

\param{\vbox{\hbox{AD\hfil}\hbox{AS\hfil}}}

The drain and source diffusion areas. Defaults for \texttt{AD} and
\texttt{AS} can be set in the \texttt{.OPTIONS} statement.  If
\texttt{AD} or \texttt{AS} defaults are not set, their default value is
0.

\param{\vbox{\hbox{PD\hfil}\hbox{PS\hfil}}}
The drain and source diffusion perimeters. Their default value is 0.

\param{\vbox{\hbox{NRD\hfil}\hbox{NRS\hfil}}}

Multipliers (in units of $\Box$) that can be multiplied by \texttt{RSH}
to yield the parasitic (ohmic) resistances of the drain (\texttt{RD})
and source (\texttt{RS}), respectively.  \texttt{NRD}, \texttt{NRS}
default to 0.

Consider a square sheet of resistive material. Analysis shows that the
resistance between two parallel edges of such a sheet depends upon its
composition and thickness, but is independent of its size as long as it is
square. In other words, the resistance will be the same whether the square's
edge is 2~mm, 2~cm, or 2~m. For this reason, the \emph{sheet resistance} of
such a layer, abbreviated \texttt{RSH}, has units of Ohms per square,
written $\mathsf{\Omega}/\Box$.

\param{M}

If specified, the value is used as a number of parallel MOSFETs to be
simulated.  For example, if \texttt{M=2} is specified, \Xyce{} simulates two
identical mosfets connected to the same nodes in parallel.

\param{IC}

The BSIM3 (model level 9), BSIM4 (model level 14 or 54) and BSIMSOI (model
level 10) allow one to specify the initial voltage difference across
nodes of the device during the DC operating point calculation.  For the
BSIM3 and BSIM4 the syntax is \texttt{IC=$V_{ds}, V_{gs}, V_{bs}$}
where $V_{ds}$ is the voltage difference between the drain and source,
$V_{gs}$ is the voltage difference between the gate and source and
$V_{bs}$ is the voltage difference between the body and source.  The
BSIMSOI device's initial condition syntax is \texttt{IC=$V_{ds},
  V_{gs}, V_{bs}, V_{es}, V_{ps}$} where the two extra terms are the
voltage difference between the substrate and source, and the external
body and source nodes respectively.  Note that for any of these lists of
voltage differences, fewer than the full number of options may be
specified.  For example, \texttt{IC=5.0} specifies an initial condition on $V_{ds}$
but does not specifiy any initial conditions on the other nodes.
Therefore, one cannot specify $V_{gs}$ without specifying $V_{ds}$, etc.

It is illegal to specify initial conditions on any nodes that are tied
together.  \Xyce{} attempts to catch such errors, but complex circuits may
stymie this error trap.

\end{Parameters}

\vbox{\hrulefill}
\item[BSIM-SOI Options]

There are a large number of extra instance parameters and optional
nodes available for the BSIM-SOI (level 10 (BSIM-SOI 3.2), level 70
(BSIM-SOI 4.6.1), and level 70450 (BSIM-SOI 4.5.0)) MOSFET.  Please
consult the BSIM-SOI technical manual, available at
\url{http://bsim.berkeley.edu/models/bsimsoi/}, for full details.

\begin{Parameters}

\param{substrate node}

The fourth node of the BSIM-SOI device is always the substrate node,
which is referred to as the \texttt{E} node. 

\param{external body contact node}

If given, the fifth node is the external body contact node,
\texttt{P}.  It is connected to the internal body node through a body
tie resistor.  If \texttt{P} is not given, the internal body node is
not accessible from the netlist and floats.

{\em For the BSIM-SOI 3.2 (level=10) only):} If there are only five
nodes specified and \texttt{TNODEOUT} is also specified, the fifth
node is the temperature node instead.

\param{internal body contact node}

If given, the sixth node is the internal body contact node, \texttt{B}.  It is
connected to the external body node through a body tie resistor.  If \texttt{B}
is not given and \texttt{P} is given, the internal body node is not accessible
from the netlist, but is still tied to the external body contact through the
tie resistance.

{\em For the BSIM-SOI 3.2 (level=10) only):} If there are only six
nodes specified and \texttt{TNODEOUT} is also specified, the sixth
node is the temperature node instead.

\param{temperature node}

{\em For the BSIM-SOI 3.2 (level=10) only):} If the parameter \texttt{TNODEOUT} is specified, the final node (fifth, sixth,
or seventh) is interpreted as a temperature node.  The temperature node is
intended for thermal coupling simulation.

{\em For the BSIM-SOI 4.x (level=70 or 70450) only):} The temperature
node is only accessible for thermal coupling if it is the seventh
node.  It is available for printing as an internal node in all other
configurations.

\param{BJTOFF}
Turns off the parasitic BJT currents.

\param{IC}
The \texttt{IC} parameter allows specification of the five junction initial
conditions, $V_{ds}, V_{gs}, V_{bs}, V_{es}$ and $V_{ps}$.  $V_{ps}$ is ignored
in a four-terminal device.

\param{RTH0}
Thermal resistance per unit width.  Taken from model card if not given.

\param{CTH0}
Thermal capacitance per unit width.  Taken from model card if not given.

\param{NBC}
Number of body contact isolation edges.

\param{NSEG}
Number of segments for channel width partitioning.

\param{PDBCP}
Parasitic perimeter length for body contact at drain side.

\param{PSBCP}
Parasitic perimeter length for body contact at source side.

\param{AGBCP}
Parasitic gate-to-body overlap area for body contact.

\param{AEBCP}
Parasitic body-to-substrate overlap area for body contact.

\param{VBSUSR}
Optional initial value of VBS specified by user for use in transient
analysis.  (unused in \Xyce{}).

\param{FRBODY}
Layout-dependent body resistance coefficient.

\end{Parameters}

\comments

The simulator provides multiple MOSFET device models, which differ in the
formulation of the I-V characteristic. The \texttt{LEVEL} parameter
selects among different models as shown below.

For HSPICE compatibility, the BSIM4 model can be specified with either
level 14 or level 54.

If a model supports parameter aliases (e.g. ``U0'' and ``UO'' or
``VT0'' and ``VTO'' in the levels 1-6 MOSFETS), it would be a mistake
to specify both parameters and give them different values.  There is
no warning or error message if you do that.  Don't do that.

\end{Device}

\paragraph{MOSFET Operating Temperature}
Model parameters may be assigned unique measurement temperatures using the
\textrmb{TNOM} model parameter. See the MOSFET model parameters for more
information.

\paragraph{MOSFET Power Calculations}
Power dissipated in the transistor is calculated with $I_{D}*V_{DS}+I_{G}*V_{GS}$ where
$I_{D}$ is the drain current, $I_{G}$ is the gate current, $V_{DS}$ is the
voltage drop between the drain and the source and $V_{GS}$ is the voltage drop
between the gate and the source. This formula may differ from other simulators,
such as HSPICE and PSpice.

\paragraph{Internal Device Variables Accessible with {\tt N()} Syntax}
For the BSIM3, BSIM4, and BSIM-CMG version 110 models, several
internal variables have been made accessible with the {\tt N()} syntax
on a {\tt .PRINT} line.  They are $g_{m}$ (tranconductance), $V_{th}$,
$V_{ds}$, $V_{gs}$, $V_{bs}$, and $V_{dsat}$.  An example {\tt .PRINT}
line command for a MOSFET device named {\tt m1} would be:
\begin{alltt}
.print dc N(m1:gm) N(m1:Vth) N(m1:Vdsat) N(m1:Vds) N(m1:Vgs) N(m1:Vbs)
\end{alltt}
The BSIM-CMG also supports output of $I_{ds}$ (drain-source current)
in this manner.

If the user runs \texttt{Xyce -namesfile <filename> <netlist>} then
\Xyce{} will output into the first filename a list of all solution
variables generated by that netlist. This can be useful for
determining the ``fully-qualified'' device name, needed for the {\tt
  N()} syntax, if the device is in a subcircuit.

\paragraph{Instance Parameters}
Tables ~\ref{M_1_Device_Instance_Params}, ~\ref{M_2_Device_Instance_Params}, 
~\ref{M_3_Device_Instance_Params},  ~\ref{M_6_Device_Instance_Params},
\ref{M_9_Device_Instance_Params} and \ref{M_10_Device_Instance_Params}  
give the available instance parameters for the levels 1,2,3,6,9 and 10 MOSFETs,
respectively.

In addition to the parameters shown in the tables, where a list of
numbered initial condition parameters are shown, the MOSFETs support a vector
parameter for the initial conditions.  \texttt{IC1} and \texttt{IC2}
may therefore be specified compactly as \texttt{IC=<ic1>,<ic2>}.

\paragraph{Model Parameters}
Tables ~\ref{M_1_Device_Model_Params}, ~\ref{M_2_Device_Model_Params},
~\ref{M_3_Device_Model_Params}, ~\ref{M_6_Device_Model_Params},
~\ref{M_9_Device_Model_Params}, and ~\ref{M_10_Device_Model_Params}
give the available model parameters for the levels 1,2,3,6,9 and 10 MOSFETs,
respectively.

For a thorough description of MOSFET models see~\cite{Antognetti:1988, HLJHCKH,
BLETK:1997, SH:1968, VL:1980,
SSKJ:1987, Pierret:1984, YEC:1983, BSIM3:V3:1, BN}.

\subparagraph{All MOSFET models}
The parameters shared by all MOSFET model levels are principally parasitic
element values (e.g., series resistance, overlap capacitance, etc.).

\subparagraph{Model levels 1 and 3}
The DC behaviors of the level 1 and 3 MOSFET models are defined by the
parameters \textrmb{VTO}, \textrmb{KP}, \textrmb{LAMBDA}, \textrmb{PHI}, and
\textrmb{GAMMA}.  The simulator calculates these if the process parameters
(e.g., \textrmb{TOX}, and \textrmb{NSUB}) are specified, but these are always
overridden by any user-defined values. The \textrmb{VTO} value is positive
(negative) for modeling the enhancement mode and negative (positive) for the
depletion mode of N-channel (P-channel) devices.

For MOSFETs, the capacitance model enforces charge conservation,
influencing just the Level 1 and 3 models.

Effective device parameter lengths and widths are calculated as follows:
\[
P_i = P_0 + P_L / L_e + P_W / W_e
\]
where
\[
\begin{array}{rclcl}
L_e & = & \mbox{effective length} & = & \mathbf{L} - (2 \cdot \mathbf{LD}) \\
W_e & = & \mbox{effective width} & = & \mathbf{W} - (2 \cdot \mathbf{WD})
\end{array}
\]

See \textrmb{.MODEL} (model definition) for more information.

\subparagraph{Model level 9 (BSIM3 version 3.2.2)}
The University of California, Berkeley BSIM3 model is a physical-based model
with a large number of dependencies on essential dimensional and processing
parameters.  It incorporates the key effects that are critical in modeling
deep-submicrometer MOSFETs.  These include threshold voltage reduction,
nonuniform doping, mobility reduction due to the vertical field, bulk charge
effect, carrier velocity saturation, drain-induced barrier lowering (DIBL),
channel length modulation (CLM), hot-carrier-induced output resistance
reduction, subthreshold conduction, source/drain parasitic resistance,
substrate current induced body effect (SCBE) and drain voltage reduction in LDD
structure.

The BSIM3 Version 3.2.2 model is a deep submicron MOSFET model with several major
enhancements over earlier versions.  These include a single I-V formula used
to define the current and output conductance for operating regions, improved
narrow width device modeling, a superior capacitance model with improved short
and narrow geometry models, a new relaxation-time model to better transient
modeling and enhanced model fitting of assorted W/L ratios using a single
parameter set.  This version preserves the large number of integrated
dependencies on dimensional and processing parameters of the Version 2 model.
For further information, see Reference~\cite{HLJHCKH}.

\subparagraph{Additional notes}
\begin{enumerate}
\item If any of the following BSIM3 3.2.2 model parameters are not specified,
they are computed via the following:

If \textrmb{VTHO} is not specified, then:
\[
\mathbf{VTHO} = \mathbf{VFB} + \phi_s \mathbf{K1} \sqrt{\phi_s}
\]
where:
\[
\mathbf{VFB} = -1.0
\]
If \textrmb{VTHO} is given, then:
\begin{eqnarray*}
\mathbf{VFB} & = & \mathbf{VTHO} - \phi_s + \mathbf{K1}\sqrt{phi_s} \\
\mathbf{VBX} & = & \phi_s - \frac{q\cdot\mathbf{NCH} \cdot
\mathbf{XT}^2}{2\varepsilon_{si}} \\
\mathbf{CF} & = & \left( \frac{2\varepsilon_{ox}}{\pi} \right)
\ln \left(1 + \frac{1}{4 \times 10^7\cdot\mathbf{TOX}} \right)
\end{eqnarray*}
where:
\[
E_g(T) = \mbox{the energy bandgap at temperature }T = 1.16 - \frac{T^2}{7.02
\times 10^4(T + 1108)}
\]

\item If \textrmb{K1} and \textrmb{K2} are not given then they are computed via
the following:
\begin{eqnarray*}
\mathbf{K1} &=& \mathbf{GAMMA2} - 2 \cdot \mathbf{K2} \sqrt{\phi_s -
\mathbf{VBM}} \\
\mathbf{K2} &=& \frac{(\mathbf{GAMMA1} -
\mathbf{GAMMA2})(\sqrt{\phi_s - \mathbf{VBX}} -
\sqrt{\phi_s})}{2\sqrt{\phi_s}(\sqrt{\phi_s - \mathbf{VBM}} -
\sqrt{\phi_s}) + \mathbf{VBM}}
\end{eqnarray*}
where:
\begin{eqnarray*}
\phi_s & = & 2V_t \ln \left(\frac{\mathbf{NCH}}{n_i} \right) \\
V_t    & = & kT / q \\
n_i    & = & 1.45 \times 10^{10} \left(\frac{T}{300.15}
\right)^{1.5} \exp \left(21.5565981 - \frac{E_g(T)}{2V_t} \right)
\end{eqnarray*}

\item If \textrmb{NCH} is not specified and \textrmb{GAMMA1} is, then:
\[
\mathbf{NCH} = \frac{\mathbf{GAMMA1^2 \times \mathbf{COX}^2}}
{2q \varepsilon_{si}}
\]
If \textrmb{GAMMA1} and \textrmb{NCH} {\em are not} specified, then
\textrmb{NCH} defaults to $1.7\times10^{23}\;m^{-3}$ and \textrmb{GAMMA1} is
computed using \textrmb{NCH}:
\[
\mathbf{GAMMA1} = \frac{\sqrt{2q\varepsilon_{si} \cdot \mathbf{NCH}}}
{\mathbf{COX}}
\]
If \textrmb{GAMMA2} is not specified, then:
\[
\mathbf{GAMMA2} = \frac{\sqrt{2q\varepsilon_{si} \cdot \mathbf{NSUB}}}
{\mathbf{COX}}
\]

\item If \textrmb{CGSO} is not specified and $\mathbf{DLC} > 0$, then:
\[
\mathbf{CGSO} = \left\{ \begin{array}{ll}
0, & ((\mathbf{DLC \cdot COX) - CGSL)} < 0        \\
0.6 \cdot \mathbf{XJ \cdot COX}, & ((\mathbf{DLC \cdot COX) - CGSL)}
\geq 0
\end{array}
\right.
\]

\item If \textrmb{CGDO} is not specified and $\mathbf{DLC} > 0$, then:
\[
\mathbf{CGDO} = \left\{ \begin{array}{ll}
0, & ((\mathbf{DLC \cdot COX) - CGSL)} < 0 \\
0.6 \cdot \mathbf{XJ \cdot COX},
& ((\mathbf{DLC \cdot COX) - CGSL)} \geq 0
\end{array}
\right. \]
\end{enumerate}

\subparagraph{Model level 10 (BSIM-SOI version 3.2)}

The BSIM-SOI is an international standard model for SOI (silicon on insulator)
circuit design and is formulated on top of the BSIM3v3 framework.
A detailed description can be found in the BSIM-SOI 3.1 User's
Manual~\cite{BSIMSOI:Manual} and the BSIM-SOI 3.2 release
notes~\cite{BSIMSOI:3p2:Notes}.

This version (v3.2) of the BSIM-SOI includes three depletion models;
the partially depleted BSIM-SOI PD (soiMod=0), the fully depleted BSIM-SOI
FD (soiMod=2), and the unified SOI model (soiMod=1).

BSIMPD is the
Partial-Depletion (PD) mode of the BSIM-SOI.  A typical PD SOI MOSFET is formed
on a thin SOI film which is layered on top of a buried oxide.  BSIMPD has
the following features and enhancements:
\begin{XyceItemize}
\item Real floating body simulation of both I-V and C-V.  The body potential is
      determined by the balance of all body current components.
\item An improved parasitic bipolar current model.  This includes enhancements in
      the various diode leakage components, second order effects (high-level
      injection and Early effect), diffusion charge equation, and temperature
      dependence of the diode junction capacitance.
\item An improved impact-ionization current model.  The contribution from BJT
      current is also modeled by the parameter Fbjtii.
\item A gate-to-body tunneling current model, which is important to thin-oxide
      SOI technologies.
\item Enhancements in the threshold voltage and bulk charge formulation of the
      high positive body bias regime.
\item Instance parameters (Pdbcp, Psbcp, Agbcp, Aebcp, Nbc) are provided to model
      the parasitics of devices with various body-contact and isolation structures.
\item An external body node (the 6th node) and other improvements are introduced
      to facilitate the modeling of distributed body resistance.
\item Self heating.  An external temperature node (the 7th node) is supported to
      facilitate the simulation of thermal coupling among neighboring devices.
\item A unique SOI low frequency noise model, including a new excess noise resulting
      from the floating body effect.
\item Width dependence of the body effect is modeled by parameters (K1,K1w1,K1w2).
\item Improved history dependence of the body charges with two new parameters
      (Fbody, DLCB).
\item An instance parameter Vbsusr is provided for users to set the transient initial
      condition of the body potential.
\item The new charge-thickness capacitance model introduced in BSIM3v3.2,
      \texttt{capMod=3}, is included.
\end{XyceItemize}

\paragraph{Quadratic Temperature Compensation}
SPICE temperature effects are the default, but MOSFET levels 18, 19 and 20 have
a more advanced temperature compensation available.  By specifying
\texttt{TEMPMODEL=QUADRATIC} in the netlist, parameters can be interpolated
quadratically between measured values extracted from data.  See
Section~\ref{Model_Interpolation} for more details.

\paragraph{MOSFET Equations}
The following equations define an N-channel MOSFET. The P-channel
devices use a reverse the sign for all voltages and currents.  The
equations use the following variables:
\begin{eqnarray*}
V_{bs}  &=&\mbox{intrinsic substrate-intrinsic source voltage} \\
V_{bd}  &=&\mbox{intrinsic substrate-intrinsic drain voltage} \\
V_{ds}  &=&\mbox{intrinsic drain-substrate source voltage} \\
V_{dsat}&=&\mbox{saturation voltage} \\
V_{gs}  &=&\mbox{intrinsic gate-intrinsic source voltage} \\
V_{gd}  &=&\mbox{intrinsic gate-intrinsic drain voltage} \\
V_t     &=&kT / q \mbox{ (thermal voltage)} \\
V_{th}  &=&\mbox{threshold voltage} \\
C_{ox}  &=&\mbox{the gate oxide capacitance per unit area} \\
f       &=&\mbox{noise frequency} \\
k       &=&\mbox{Boltzmann's constant} \\
q       &=&\mbox{electron charge} \\
Leff    &=&\mbox{effective channel length} \\
Weff    &=&\mbox{effective channel width} \\
T       &=&\mbox{analysis temperature (K)} \\
T_0     &=&\mbox{nominal temperature (set using TNOM option)}
\end{eqnarray*}
Other variables are listed in the BJT Equations section~\ref{bjt_equations}.

\clearpage
\LTXtable{\textwidth}{mosfeteqntbl}

%%
%% MOSFET Equation Capacitance Table
%%
\paragraph{Capacitance}
\LTXtable{\textwidth}{mosfeteqncaptbl}

%%
%% MOSFET Equation Temperature Effects
%%
\clearpage
\paragraph{Temperature Effects}
\LTXtable{\textwidth}{mosfeteqntemptbl}

%%
%% MOSFET Parameters Table
%%
\clearpage
\subsubsection{Level 1 MOSFET Tables (SPICE Level 1)}
% This table was generated by Xyce:
%   Xyce -doc M 1
%
\index{mosfet level 1!device instance parameters}
\begin{DeviceParamTableGenerated}{MOSFET level 1 Device Instance Parameters}{M_1_Device_Instance_Params}
AD & Drain diffusion area & m$^{2}$ & 0 \\ \hline
AS & Source diffusion area & m$^{2}$ & 0 \\ \hline
DTEMP & Device delta temperature & $^\circ$C & 0 \\ \hline
IC1 & Initial condition on Drain-Source voltage & V & 0 \\ \hline
IC2 & Initial condition on Gate-Source voltage & V & 0 \\ \hline
IC3 & Initial condition on Bulk-Source voltage & V & 0 \\ \hline
L & Channel length & m & 0 \\ \hline
M & Multiplier for M devices connected in parallel & -- & 1 \\ \hline
NRD & Multiplier for RSH to yield parasitic resistance of drain & $\Box$ & 1 \\ \hline
NRS & Multiplier for RSH to yield parasitic resistance of source & $\Box$ & 1 \\ \hline
OFF & Initial condition of no voltage drops across device & logical (T/F) & false \\ \hline
PD & Drain diffusion perimeter & m & 0 \\ \hline
PS & Source diffusion perimeter & m & 0 \\ \hline
TEMP & Device temperature & $^\circ$C & Ambient Temperature \\ \hline
W & Channel width & m & 0 \\ \hline
\end{DeviceParamTableGenerated}

% This table was generated by Xyce:
%   Xyce -doc M 1
%
\index{mosfet level 1!device model parameters}
\begin{DeviceParamTableGenerated}{MOSFET level 1 Device Model Parameters}{M_1_Device_Model_Params}
AF & Flicker noise exponent & -- & 1 \\ \hline
CBD & Zero-bias bulk-drain p-n capacitance & F & 0 \\ \hline
CBS & Zero-bias bulk-source p-n capacitance & F & 0 \\ \hline
CGBO & Gate-bulk overlap capacitance/channel length & F/m & 0 \\ \hline
CGDO & Gate-drain overlap capacitance/channel width & F/m & 0 \\ \hline
CGSO & Gate-source overlap capacitance/channel width & F/m & 0 \\ \hline
CJ & Bulk p-n zero-bias bottom capacitance/area & F/m$^{2}$ & 0 \\ \hline
CJSW & Bulk p-n zero-bias sidewall capacitance/area & F/m$^{2}$ & 0 \\ \hline
FC & Bulk p-n forward-bias capacitance coefficient & -- & 0.5 \\ \hline
GAMMA & Bulk threshold parameter & V$^{1/2}$ & 0 \\ \hline
IS & Bulk p-n saturation current & A & 1e-14 \\ \hline
JS & Bulk p-n saturation current density & A/m$^{2}$ & 0 \\ \hline
KF & Flicker noise coefficient & -- & 0 \\ \hline
KP & Transconductance coefficient & A/V$^{2}$ & 2e-05 \\ \hline
L & Default channel length & m & 0.0001 \\ \hline
LAMBDA & Channel-length modulation & V$^{-1}$ & 0 \\ \hline
LD & Lateral diffusion length & m & 0 \\ \hline
MJ & Bulk p-n bottom grading coefficient & -- & 0.5 \\ \hline
MJSW & Bulk p-n sidewall grading coefficient & -- & 0.5 \\ \hline
NSS & Surface state density & cm$^{-2}$ & 0 \\ \hline
NSUB & Substrate doping density & cm$^{-3}$ & 0 \\ \hline
PB & Bulk p-n bottom potential & V & 0.8 \\ \hline
PHI & Surface potential & V & 0.6 \\ \hline
RD & Drain ohmic resistance & $\mathsf{\Omega}$ & 0 \\ \hline
RS & Source ohmic resistance & $\mathsf{\Omega}$ & 0 \\ \hline
RSH & Drain,source diffusion sheet resistance & $\mathsf{\Omega}$ & 0 \\ \hline
TEMPMODEL & Specifies the type of parameter interpolation over temperature & -- & 'NONE' \\ \hline
TNOM & Nominal device temperature & $^\circ$C & 27 \\ \hline
TOX & Gate oxide thickness & m & 1e-07 \\ \hline
TPG & Gate material type (-1 = same as substrate) 0 = aluminum,1 = opposite of substrate) & -- & 0 \\ \hline
U0 & Surface mobility (alias for UO) & 1/(Vcm$^{2}$s) & 600 \\ \hline
UO & Surface mobility & 1/(Vcm$^{2}$s) & 600 \\ \hline
VT0 & Zero-bias threshold voltage (alias for VTO) & V & 0 \\ \hline
VTO & Zero-bias threshold voltage & V & 0 \\ \hline
W & Default channel width & m & 0.0001 \\ \hline
\end{DeviceParamTableGenerated}

\clearpage
\subsubsection{Level 2 MOSFET Tables (SPICE Level 2)}
% This table was generated by Xyce:
%   Xyce -doc M 2
%
\index{mosfet level 2!device instance parameters}
\begin{DeviceParamTableGenerated}{MOSFET level 2 Device Instance Parameters}{M_2_Device_Instance_Params}
AD & Drain diffusion area & m$^{2}$ & 0 \\ \hline
AS & Source diffusion area & m$^{2}$ & 0 \\ \hline
DTEMP & Device delta temperature & $^\circ$C & 0 \\ \hline
IC1 & Initial condition on Drain-Source voltage & V & 0 \\ \hline
IC2 & Initial condition on Gate-Source voltage & V & 0 \\ \hline
IC3 & Initial condition on Bulk-Source voltage & V & 0 \\ \hline
L & Channel length & m & 0 \\ \hline
M & Multiplier for M devices connected in parallel & -- & 1 \\ \hline
NRD & Multiplier for RSH to yield parasitic resistance of drain & $\Box$ & 1 \\ \hline
NRS & Multiplier for RSH to yield parasitic resistance of source & $\Box$ & 1 \\ \hline
OFF & Initial condition of no voltage drops across device & logical (T/F) & false \\ \hline
PD & Drain diffusion perimeter & m & 0 \\ \hline
PS & Source diffusion perimeter & m & 0 \\ \hline
TEMP & Device temperature & $^\circ$C & Ambient Temperature \\ \hline
W & Channel width & m & 0 \\ \hline
\end{DeviceParamTableGenerated}

% This table was generated by Xyce:
%   Xyce -doc M 2
%
\index{mosfet level 2!device model parameters}
\begin{DeviceParamTableGenerated}{MOSFET level 2 Device Model Parameters}{M_2_Device_Model_Params}
AF & Flicker noise exponent & -- & 1 \\ \hline
CBD & Zero-bias bulk-drain p-n capacitance & F & 0 \\ \hline
CBS & Zero-bias bulk-source p-n capacitance & F & 0 \\ \hline
CGBO & Gate-bulk overlap capacitance/channel length & F/m & 0 \\ \hline
CGDO & Gate-drain overlap capacitance/channel width & F/m & 0 \\ \hline
CGSO & Gate-source overlap capacitance/channel width & F/m & 0 \\ \hline
CJ & Bulk p-n zero-bias bottom capacitance/area & F/m$^{2}$ & 0 \\ \hline
CJSW & Bulk p-n zero-bias sidewall capacitance/area & F/m$^{2}$ & 0 \\ \hline
DELTA & Width effect on threshold & -- & 0 \\ \hline
FC & Bulk p-n forward-bias capacitance coefficient & -- & 0.5 \\ \hline
GAMMA & Bulk threshold parameter & V$^{1/2}$ & 0 \\ \hline
IS & Bulk p-n saturation current & A & 1e-14 \\ \hline
JS & Bulk p-n saturation current density & A/m$^{2}$ & 0 \\ \hline
KF & Flicker noise coefficient & -- & 0 \\ \hline
KP & Transconductance coefficient & A/V$^{2}$ & 2e-05 \\ \hline
L & Default channel length & m & 0.0001 \\ \hline
LAMBDA & Channel-length modulation & V$^{-1}$ & 0 \\ \hline
LD & Lateral diffusion length & m & 0 \\ \hline
MJ & Bulk p-n bottom grading coefficient & -- & 0.5 \\ \hline
MJSW & Bulk p-n sidewall grading coefficient & -- & 0.5 \\ \hline
NEFF & Total channel charge coeff. & -- & 1 \\ \hline
NFS & Fast surface state density & -- & 0 \\ \hline
NSS & Surface state density & cm$^{-2}$ & 0 \\ \hline
NSUB & Substrate doping density & cm$^{-3}$ & 0 \\ \hline
PB & Bulk p-n bottom potential & V & 0.8 \\ \hline
PHI & Surface potential & V & 0.6 \\ \hline
RD & Drain ohmic resistance & $\mathsf{\Omega}$ & 0 \\ \hline
RS & Source ohmic resistance & $\mathsf{\Omega}$ & 0 \\ \hline
RSH & Drain,source diffusion sheet resistance & $\mathsf{\Omega}$ & 0 \\ \hline
TEMPMODEL & Specifies the type of parameter interpolation over temperature & -- & 'NONE' \\ \hline
TNOM & Nominal device temperature & $^\circ$C & 27 \\ \hline
TOX & Gate oxide thickness & m & 1e-07 \\ \hline
TPG & Gate material type (-1 = same as substrate, 0 = aluminum,1 = opposite of substrate) & -- & 0 \\ \hline
U0 & Surface mobility (alias for UO) & 1/(Vcm$^{2}$s) & 600 \\ \hline
UCRIT & Crit. field for mob. degradation & -- & 10000 \\ \hline
UEXP & Crit. field exp for mob. deg. & -- & 0 \\ \hline
UO & Surface mobility & 1/(Vcm$^{2}$s) & 600 \\ \hline
VMAX & Maximum carrier drift velocity & -- & 0 \\ \hline
VT0 & Zero-bias threshold voltage (alias for VTO) & V & 0 \\ \hline
VTO & Zero-bias threshold voltage & V & 0 \\ \hline
W & Default channel width & m & 0.0001 \\ \hline
XJ & Junction depth & -- & 0 \\ \hline
\end{DeviceParamTableGenerated}

\clearpage
\subsubsection{Level 3 MOSFET Tables (SPICE Level 3)}
% This table was generated by Xyce:
%   Xyce -doc M 3
%
\index{mosfet level 3!device instance parameters}
\begin{DeviceParamTableGenerated}{MOSFET level 3 Device Instance Parameters}{M_3_Device_Instance_Params}
AD & Drain diffusion area & m$^{2}$ & 0 \\ \hline
AS & Source diffusion area & m$^{2}$ & 0 \\ \hline
DTEMP & Device delta temperature & $^\circ$C & 0 \\ \hline
IC1 & Initial condition on Drain-Source voltage & V & 0 \\ \hline
IC2 & Initial condition on Gate-Source voltage & V & 0 \\ \hline
IC3 & Initial condition on Bulk-Source voltage & V & 0 \\ \hline
L & Channel length & m & 0 \\ \hline
M & Multiplier for M devices connected in parallel & -- & 1 \\ \hline
NRD & Multiplier for RSH to yield parasitic resistance of drain & $\Box$ & 1 \\ \hline
NRS & Multiplier for RSH to yield parasitic resistance of source & $\Box$ & 1 \\ \hline
OFF & Initial condition of no voltage drops across device & logical (T/F) & false \\ \hline
PD & Drain diffusion perimeter & m & 0 \\ \hline
PS & Source diffusion perimeter & m & 0 \\ \hline
TEMP & Device temperature & $^\circ$C & Ambient Temperature \\ \hline
W & Channel width & m & 0 \\ \hline
\end{DeviceParamTableGenerated}

% This table was generated by Xyce:
%   Xyce -doc M 3
%
\index{mosfet level 3!device model parameters}
\begin{DeviceParamTableGenerated}{MOSFET level 3 Device Model Parameters}{M_3_Device_Model_Params}
AF & Flicker noise exponent & -- & 1 \\ \hline
CBD & Zero-bias bulk-drain p-n capacitance & F & 0 \\ \hline
CBS & Zero-bias bulk-source p-n capacitance & F & 0 \\ \hline
CGBO & Gate-bulk overlap capacitance/channel length & F/m & 0 \\ \hline
CGDO & Gate-drain overlap capacitance/channel width & F/m & 0 \\ \hline
CGSO & Gate-source overlap capacitance/channel width & F/m & 0 \\ \hline
CJ & Bulk p-n zero-bias bottom capacitance/area & F/m$^{2}$ & 0 \\ \hline
CJSW & Bulk p-n zero-bias sidewall capacitance/area & F/m$^{2}$ & 0 \\ \hline
DELTA & Width effect on threshold & -- & 0 \\ \hline
ETA & Static feedback & -- & 0 \\ \hline
FC & Bulk p-n forward-bias capacitance coefficient & -- & 0.5 \\ \hline
GAMMA & Bulk threshold parameter & V$^{1/2}$ & 0 \\ \hline
IS & Bulk p-n saturation current & A & 1e-14 \\ \hline
JS & Bulk p-n saturation current density & A/m$^{2}$ & 0 \\ \hline
KAPPA & Saturation field factor & -- & 0.2 \\ \hline
KF & Flicker noise coefficient & -- & 0 \\ \hline
KP & Transconductance coefficient & A/V$^{2}$ & 2e-05 \\ \hline
L & Default channel length & m & 0.0001 \\ \hline
LD & Lateral diffusion length & m & 0 \\ \hline
MJ & Bulk p-n bottom grading coefficient & -- & 0.5 \\ \hline
MJSW & Bulk p-n sidewall grading coefficient & -- & 0.33 \\ \hline
NFS & Fast surface state density & cm$^{-2}$ & 0 \\ \hline
NSS & Surface state density & cm$^{-2}$ & 0 \\ \hline
NSUB & Substrate doping density & cm$^{-3}$ & 0 \\ \hline
PB & Bulk p-n bottom potential & V & 0.8 \\ \hline
PHI & Surface potential & V & 0.6 \\ \hline
RD & Drain ohmic resistance & $\mathsf{\Omega}$ & 0 \\ \hline
RS & Source ohmic resistance & $\mathsf{\Omega}$ & 0 \\ \hline
RSH & Drain,source diffusion sheet resistance & $\mathsf{\Omega}$ & 0 \\ \hline
TEMPMODEL & Specifies the type of parameter interpolation over temperature & -- & 'NONE' \\ \hline
THETA & Mobility modulation & V$^{-1}$ & 0 \\ \hline
TNOM & Nominal device temperature & $^\circ$C & 27 \\ \hline
TOX & Gate oxide thickness & m & 1e-07 \\ \hline
TPG & Gate material type (-1 = same as substrate,0 = aluminum,1 = opposite of substrate) & -- & 1 \\ \hline
U0 & Surface mobility (alias for UO) & 1/(Vcm$^{2}$s) & 600 \\ \hline
UO & Surface mobility & 1/(Vcm$^{2}$s) & 600 \\ \hline
VMAX & Maximum drift velocity & m/s & 0 \\ \hline
VT0 & Zero-bias threshold voltage (alias for VTO) & V & 0 \\ \hline
VTO & Zero-bias threshold voltage & V & 0 \\ \hline
W & Default channel width & m & 0.0001 \\ \hline
XJ & Metallurgical junction depth & m & 0 \\ \hline
\end{DeviceParamTableGenerated}

\clearpage
\subsubsection{Level 6 MOSFET Tables (SPICE Level 6)}
% This table was generated by Xyce:
%   Xyce -doc M 6
%
\index{mosfet level 6!device instance parameters}
\begin{DeviceParamTableGenerated}{MOSFET level 6 Device Instance Parameters}{M_6_Device_Instance_Params}
AD & Drain diffusion area & m$^{2}$ & 0 \\ \hline
AS & Source diffusion area & m$^{2}$ & 0 \\ \hline
DTEMP & Device delta temperature & $^\circ$C & 0 \\ \hline
IC1 & Initial condition on Drain-Source voltage & V & 0 \\ \hline
IC2 & Initial condition on Gate-Source voltage & V & 0 \\ \hline
IC3 & Initial condition on Bulk-Source voltage & V & 0 \\ \hline
L & Channel length & m & 0 \\ \hline
M & Multiplier for M devices connected in parallel & -- & 1 \\ \hline
NRD & Multiplier for RSH to yield parasitic resistance of drain & $\Box$ & 1 \\ \hline
NRS & Multiplier for RSH to yield parasitic resistance of source & $\Box$ & 1 \\ \hline
OFF & Initial condition of no voltage drops across device & logical (T/F) & false \\ \hline
PD & Drain diffusion perimeter & m & 0 \\ \hline
PS & Source diffusion perimeter & m & 0 \\ \hline
TEMP & Device temperature & $^\circ$C & Ambient Temperature \\ \hline
W & Channel width & m & 0 \\ \hline
\end{DeviceParamTableGenerated}

% This table was generated by Xyce:
%   Xyce -doc M 6
%
\index{mosfet level 6!device model parameters}
\begin{DeviceParamTableGenerated}{MOSFET level 6 Device Model Parameters}{M_6_Device_Model_Params}
AF & Flicker noise exponent & -- & 1 \\ \hline
CBD & Zero-bias bulk-drain p-n capacitance & F & 0 \\ \hline
CBS & Zero-bias bulk-source p-n capacitance & F & 0 \\ \hline
CGBO & Gate-bulk overlap capacitance/channel length & F/m & 0 \\ \hline
CGDO & Gate-drain overlap capacitance/channel width & F/m & 0 \\ \hline
CGSO & Gate-source overlap capacitance/channel width & F/m & 0 \\ \hline
CJ & Bulk p-n zero-bias bottom capacitance/area & F/m$^{2}$ & 0 \\ \hline
CJSW & Bulk p-n zero-bias sidewall capacitance/area & F/m$^{2}$ & 0 \\ \hline
FC & Bulk p-n forward-bias capacitance coefficient & -- & 0.5 \\ \hline
GAMMA & Bulk threshold parameter & -- & 0 \\ \hline
GAMMA1 & Bulk threshold parameter 1 & -- & 0 \\ \hline
IS & Bulk p-n saturation current & A & 1e-14 \\ \hline
JS & Bulk p-n saturation current density & A/m$^{2}$ & 0 \\ \hline
KC & Saturation current factor & -- & 5e-05 \\ \hline
KF & Flicker noise coefficient & -- & 0 \\ \hline
KV & Saturation voltage factor & -- & 2 \\ \hline
LAMBDA & Channel length modulation param. & -- & 0 \\ \hline
LAMBDA0 & Channel length modulation param. 0 & -- & 0 \\ \hline
LAMBDA1 & Channel length modulation param. 1 & -- & 0 \\ \hline
LD & Lateral diffusion length & m & 0 \\ \hline
MJ & Bulk p-n bottom grading coefficient & -- & 0.5 \\ \hline
MJSW & Bulk p-n sidewall grading coefficient & -- & 0.5 \\ \hline
NC & Saturation current coeff. & -- & 1 \\ \hline
NSS & Surface state density & cm$^{-2}$ & 0 \\ \hline
NSUB & Substrate doping density & cm$^{-3}$ & 0 \\ \hline
NV & Saturation voltage coeff. & -- & 0.5 \\ \hline
NVTH & Threshold voltage coeff. & -- & 0.5 \\ \hline
PB & Bulk p-n bottom potential & V & 0.8 \\ \hline
PHI & Surface potential & V & 0.6 \\ \hline
PS & Sat. current modification  par. & -- & 0 \\ \hline
RD & Drain ohmic resistance & $\mathsf{\Omega}$ & 0 \\ \hline
RS & Source ohmic resistance & $\mathsf{\Omega}$ & 0 \\ \hline
RSH & Drain,source diffusion sheet resistance & $\mathsf{\Omega}$ & 0 \\ \hline
SIGMA & Static feedback effect par. & -- & 0 \\ \hline
TEMPMODEL & Specifies the type of parameter interpolation over temperature & -- & 'NONE' \\ \hline
TNOM & Nominal device temperature & $^\circ$C & 27 \\ \hline
TOX & Gate oxide thickness & m & 1e-07 \\ \hline
TPG & Gate material type (-1 = same as substrate,0 = aluminum,1 = opposite of substrate) & -- & 1 \\ \hline
U0 & Surface mobility (alias for UO) & 1/(Vcm$^{2}$s) & 600 \\ \hline
UO & Surface mobility & 1/(Vcm$^{2}$s) & 600 \\ \hline
VT0 & Zero-bias threshold voltage (alias for VTO) & V & 0 \\ \hline
VTO & Zero-bias threshold voltage & V & 0 \\ \hline
\end{DeviceParamTableGenerated}

\clearpage
\subsubsection{Level 9 MOSFET Tables (BSIM3)}
For complete documentation of the BSIM3 model, see the users' manual for
the BSIM3, available for download at
\url{http://bsim.berkeley.edu/models/bsim4/bsim3/}.
\Xyce{} implements Version 3.2.2 of the BSIM3.

In addition to the parameters shown in
table~\ref{M_9_Device_Instance_Params}, the BSIM3 supports a vector
parameter for the initial conditions.  \texttt{IC1} through
\texttt{IC3} may therefore be specified compactly as
\texttt{IC=<ic1>,<ic2>,<ic3>}.

\textbf{NOTE:  Many BSIM3 parameters listed in
tables~\ref{M_9_Device_Instance_Params} and \ref{M_9_Device_Model_Params} as
having default values of zero are actually replaced with internally computed
defaults if not given.  Specifying zero in your model card will override this
internal computation.  It is recommended that you only set model parameters
that you are actually changing from defaults and that you not generate model
cards containing default values from the tables.}
\input{M_9_Device_Instance_Params}
\input{M_9_Device_Model_Params}

\clearpage
\subsubsection{Level 10 MOSFET Tables (BSIM-SOI)}
For complete documentation of the BSIM-SOI model, see the users' manual
for the BSIM-SOI, available for download at
\url{http://bsim.berkeley.edu/models/bsimsoi/}.
\Xyce{} implements Version 3.2 of the BSIM-SOI, you will have to get the
documentation from the FTP archive on the Berkeley site.

In addition to the parameters shown in table~\ref{M_10_Device_Instance_Params}, 
the BSIM3SOI supports a vector parameter for the initial conditions.    \texttt{IC1} through \texttt{IC5}
may therefore be specified compactly as \texttt{IC=<ic1>,<ic2>,<ic3>, <ic4>,<ic5>}.

\textbf{NOTE:  Many BSIM SOI parameters listed in
tables~\ref{M_10_Device_Instance_Params} and \ref{M_10_Device_Model_Params} as
having default values of zero are actually replaced with internally computed
defaults if not given.  Specifying zero in your model card will override this
internal computation.  It is recommended that you only set model parameters
that you are actually changing from defaults and that you not generate model
cards containing default values from the tables.}
% This table was generated by Xyce:
%   Xyce -doc_cat M 10
%
\index{bsim3 soi!device instance parameters}
\begin{DeviceParamTableGenerated}{BSIM3 SOI Device Instance Parameters}{M_10_Device_Instance_Params}
BJTOFF & BJT on/off flag & logical (T/F) & 0 \\ \hline
DEBUG & BJT on/off flag & logical (T/F) & 0 \\ \hline
TNODEOUT & Flag indicating external temp node & logical (T/F) & 0 \\ \hline
VLDEBUG &  & logical (T/F) & false \\ \hline

\category{Control Parameters}\\ \hline
M & Multiplier for M devices connected in parallel & -- & 1 \\ \hline
SOIMOD & SIO model selector,SOIMOD=0: BSIMPD,SOIMOD=1: undefined model for PD and FE,SOIMOD=2: ideal FD & -- & 0 \\ \hline

\category{DC Parameters}\\ \hline
VBSUSR & Vbs specified by user & V & 0 \\ \hline

\category{Geometry Parameters}\\ \hline
AD & Drain diffusion area & m$^{2}$ & 0 \\ \hline
AEBCP & Substrate to body overlap area for bc prasitics & m$^{2}$ & 0 \\ \hline
AGBCP & Gate to body overlap area for bc parasitics & m$^{2}$ & 0 \\ \hline
AS & Source diffusion area & m$^{2}$ & 0 \\ \hline
FRBODY & Layout dependent body-resistance coefficient & -- & 1 \\ \hline
L & Channel length & m & 5e-06 \\ \hline
NBC & Number of body contact isolation edge & -- & 0 \\ \hline
NRB & Number of squares in body & -- & 1 \\ \hline
NRD & Multiplier for RSH to yield parasitic resistance of drain & $\Box$ & 1 \\ \hline
NRS & Multiplier for RSH to yield parasitic resistance of source & $\Box$ & 1 \\ \hline
NSEG & Number segments for width partitioning & -- & 1 \\ \hline
PD & Drain diffusion perimeter & m & 0 \\ \hline
PDBCP & Perimeter length for bc parasitics at drain side & m & 0 \\ \hline
PS & Source diffusion perimeter & m & 0 \\ \hline
PSBCP & Perimeter length for bc parasitics at source side & m & 0 \\ \hline
W & Channel width & m & 5e-06 \\ \hline

\category{RF Parameters}\\ \hline
RGATEMOD & Gate resistance model selector & -- & 0 \\ \hline

\category{Temperature Parameters}\\ \hline
CTH0 & Thermal capacitance & F & 0 \\ \hline
DTEMP & Device delta temperature & $^\circ$C & 0 \\ \hline
RTH0 & normalized thermal resistance & $\mathsf{\Omega}$ & 0 \\ \hline
TEMP & Device temperature & $^\circ$C & Ambient Temperature \\ \hline

\category{Voltage Parameters}\\ \hline
IC1 & Initial condition on Vds & V & 0 \\ \hline
IC2 & Initial condition on Vgs & V & 0 \\ \hline
IC3 & Initial condition on Vbs & V & 0 \\ \hline
IC4 & Initial condition on Ves & V & 0 \\ \hline
IC5 & Initial condition on Vps & V & 0 \\ \hline
OFF & Initial condition of no voltage drops accross device & logical (T/F) & false \\ \hline
\end{DeviceParamTableGenerated}

\input{M_10_Device_Model_Params}

\clearpage
\subsubsection{Level 14/54 MOSFET Tables (BSIM4)}
The level 14 MOSFET device in \Xyce{} is based on the Berkeley BSIM4 model
version 4.6.1.  (For HSPICE compatibility, the Xyce BSIM4 model can also be
specified as level 54.)  The model's parameters are given in the following
tables.  Note that the parameters have not all been properly categorized with
units in place.  For complete documentation of the BSIM4 model, see the BSIM4
User’s Manual, available for download at
\url{http://bsim.berkeley.edu/models/bsim4/}.

Note that the BSIM4 device in Xyce now supports multiple versions
selectable with the \texttt{VERSION} parameter in the model card.  At
this time versions 4.6.1, 4.7.0, and 4.8.2 are supported.  This
version parameter may be specified either in legacy text format
(``4.6.1'' or ``4.8.2'') or in the CMC standard floating point format
(``4.61'' or ``4.82'').

If a \texttt{VERSION} parameter is not given, the latest version
supported is used.

If a \texttt{VERSION} parameter is given that is not one of the
supported version numbers, the closest matching supported version is
used instead and a warning given.  If a version older than the lowest
supported version is chosen, the lowest supported version (4.6.1) is
used and a warning given.  If a model lower than version 4.7.0 is
requested, version 4.6.1 is used (and a warning given).  If a version
newer than 4.7.0 but older than 4.8.0 is requested, 4.7.0 is used and
a warning given.  If a version 4.8.0 or later is requested, 4.8.2 is
used with an appropriate warning.

Specifying any model parameter that is
not supported in the chosen version results in a warning and the
parameter being ignored.  Parameters that are only valid for specific
ranges of versions are noted as such in the
tables~\ref{M_14_Device_Instance_Params} and
\ref{M_14_Device_Model_Params}.

At this time, the BSIM4 is the only device in Xyce that supports
multiple versions in this manner.  All other devices that have
multiple version in Xyce are handled by having a unique level number
for each version.

\textbf{NOTE: Many BSIM4 parameters listed in
  tables~\ref{M_14_Device_Instance_Params} and
  \ref{M_14_Device_Model_Params} as having default values of zero are
  actually replaced with internally computed defaults if not given.
  Specifying zero in your model card will override this internal
  computation.  It is recommended that you only set model parameters
  that you are actually changing from defaults and that you not
  generate model cards containing default values from the tables.
}

\textbf{
  Furthermore, the value of \texttt{FGIDL} changed from 0 to 1 with
  version 4.8.2 of the BSIM4.  This change is NOT reflected in the
  table, which is generated automatically, and which shows only the
  default value of this parameter that applies to versions 4.6.1 and
  4.7.0.}


% This table was generated by Xyce:
%   Xyce -doc_cat M 14
%
\index{bsim4!device instance parameters}
\begin{DeviceParamTableGenerated}{BSIM4 Device Instance Parameters}{M_14_Device_Instance_Params}
AD & Drain area & -- & 0 \\ \hline
AS & Source area & -- & 0 \\ \hline
IC2 &  & -- & 0 \\ \hline
IC3 &  & -- & 0 \\ \hline
L & Length & -- & 5e-06 \\ \hline
M & Number of parallel copies & -- & 1 \\ \hline
MIN & Minimize either D or S & -- & 0 \\ \hline
NF & Number of fingers & -- & 1 \\ \hline
NGCON & Number of gate contacts & -- & 0 \\ \hline
OFF & Device is initially off & -- & false \\ \hline
PD & Drain perimeter & -- & 0 \\ \hline
PS & Source perimeter & -- & 0 \\ \hline
RBDB & Body resistance & -- & 0 \\ \hline
RBPB & Body resistance & -- & 0 \\ \hline
RBPD & Body resistance & -- & 0 \\ \hline
RBPS & Body resistance & -- & 0 \\ \hline
RBSB & Body resistance & -- & 0 \\ \hline
SA & distance between  OD edge to poly of one side  & -- & 0 \\ \hline
SB & distance between  OD edge to poly of the other side & -- & 0 \\ \hline
SC & Distance to a single well edge  & -- & 0 \\ \hline
SCA & Integral of the first distribution function for scattered well dopant & -- & 0 \\ \hline
SCB & Integral of the second distribution function for scattered well dopant & -- & 0 \\ \hline
SCC & Integral of the third distribution function for scattered well dopant & -- & 0 \\ \hline
SD & distance between neighbour fingers & -- & 0 \\ \hline
W & Width & -- & 5e-06 \\ \hline
XGW & Distance from gate contact center to device edge & -- & 0 \\ \hline

\category{Basic Parameters}\\ \hline
DELVT0 & Zero bias threshold voltage variation & V & 0 \\ \hline
DELVTO & Zero bias threshold voltage variation & V & 0 \\ \hline

\category{Control Parameters}\\ \hline
ACNQSMOD & AC NQS model selector & -- & 0 \\ \hline
GEOMOD & Geometry dependent parasitics model selector & -- & 0 \\ \hline
RBODYMOD & Distributed body R model selector & -- & 0 \\ \hline
RGATEMOD & Gate resistance model selector & -- & 0 \\ \hline
RGEOMOD & S/D resistance and contact model selector & -- & 0 \\ \hline
TRNQSMOD & Transient NQS model selector & -- & 0 \\ \hline

\category{Temperature Parameters}\\ \hline
DTEMP & Device delta temperature & $^\circ$C & 0 \\ \hline
TEMP & Device temperature & $^\circ$C & Ambient Temperature \\ \hline

\category{Voltage Parameters}\\ \hline
IC1 & Vector of initial values: Vds,Vgs,Vbs & V & 0 \\ \hline

\category{Asymmetric and Bias-Dependent $R_{ds}$ Parameters}\\ \hline
NRD & Number of squares in drain & -- & 1 \\ \hline
NRS & Number of squares in source & -- & 1 \\ \hline
\end{DeviceParamTableGenerated}

% This table was generated by Xyce:
%   Xyce -doc_cat M 14
%
\index{bsim4!device model parameters}
\begin{DeviceParamTableGenerated}{BSIM4 Device Model Parameters}{M_14_Device_Model_Params}
AF & Flicker noise exponent & -- & 1 \\ \hline
AIGSD & Parameter for Igs,d & -- & 0.0136 \\ \hline
AT & Temperature coefficient of vsat & -- & 33000 \\ \hline
BIGSD & Parameter for Igs,d & -- & 0.00171 \\ \hline
BVD & Drain diode breakdown voltage & -- & 10 \\ \hline
BVS & Source diode breakdown voltage & -- & 10 \\ \hline
CIGSD & Parameter for Igs,d & -- & 0.075 \\ \hline
CJD & Drain bottom junction capacitance per unit area & -- & 0.0005 \\ \hline
CJS & Source bottom junction capacitance per unit area & -- & 0.0005 \\ \hline
CJSWD & Drain sidewall junction capacitance per unit periphery & -- & 5e-10 \\ \hline
CJSWGD & Drain (gate side) sidewall junction capacitance per unit width & -- & 0 \\ \hline
CJSWGS & Source (gate side) sidewall junction capacitance per unit width & -- & 0 \\ \hline
CJSWS & Source sidewall junction capacitance per unit periphery & -- & 5e-10 \\ \hline
DLCIG & Delta L for Ig model & -- & 0 \\ \hline
DMCG & Distance of Mid-Contact to Gate edge & -- & 0 \\ \hline
DMCGT & Distance of Mid-Contact to Gate edge in Test structures & -- & 0 \\ \hline
DMCI & Distance of Mid-Contact to Isolation & -- & 0 \\ \hline
DMDG & Distance of Mid-Diffusion to Gate edge & -- & 0 \\ \hline
DWJ & Delta W for S/D junctions & -- & 0 \\ \hline
EF & Flicker noise frequency exponent & -- & 1 \\ \hline
EM & Flicker noise parameter & -- & 4.1e+07 \\ \hline
EPSRGATE & Dielectric constant of gate relative to vacuum & -- & 11.7 \\ \hline
GBMIN & Minimum body conductance & $\mathsf{\Omega}^{-1}$ & 1e-12 \\ \hline
IJTHDFWD & Forward drain diode forward limiting current & -- & 0.1 \\ \hline
IJTHDREV & Reverse drain diode forward limiting current & -- & 0.1 \\ \hline
IJTHSFWD & Forward source diode forward limiting current & -- & 0.1 \\ \hline
IJTHSREV & Reverse source diode forward limiting current & -- & 0.1 \\ \hline
JSD & Bottom drain junction reverse saturation current density & -- & 0.0001 \\ \hline
JSS & Bottom source junction reverse saturation current density & -- & 0.0001 \\ \hline
JSWD & Isolation edge sidewall drain junction reverse saturation current density & -- & 0 \\ \hline
JSWGD & Gate edge drain junction reverse saturation current density & -- & 0 \\ \hline
JSWGS & Gate edge source junction reverse saturation current density & -- & 0 \\ \hline
JSWS & Isolation edge sidewall source junction reverse saturation current density & -- & 0 \\ \hline
JTSD & Drain bottom trap-assisted saturation current density & -- & 0 \\ \hline
JTSS & Source bottom trap-assisted saturation current density & -- & 0 \\ \hline
JTSSWD & Drain STI sidewall trap-assisted saturation current density & -- & 0 \\ \hline
JTSSWGD & Drain gate-edge sidewall trap-assisted saturation current density & -- & 0 \\ \hline
JTSSWGS & Source gate-edge sidewall trap-assisted saturation current density & -- & 0 \\ \hline
JTSSWS & Source STI sidewall trap-assisted saturation current density & -- & 0 \\ \hline
JTWEFF\newline{\normalfont [Only for versions starting with 4.7]} & TAT current width dependence & m & 0 \\ \hline
K2WE &  K2 shift factor for well proximity effect  & -- & 0 \\ \hline
K3B & Body effect coefficient of k3 & -- & 0 \\ \hline
KF & Flicker noise coefficient & -- & 0 \\ \hline
KT1 & Temperature coefficient of Vth & -- & -0.11 \\ \hline
KT1L & Temperature coefficient of Vth & -- & 0 \\ \hline
KT2 & Body-coefficient of kt1 & -- & 0.022 \\ \hline
KU0 & Mobility degradation/enhancement coefficient for LOD & -- & 0 \\ \hline
KU0WE &  Mobility degradation factor for well proximity effect  & -- & 0 \\ \hline
KVSAT & Saturation velocity degradation/enhancement parameter for LOD & -- & 0 \\ \hline
KVTH0 & Threshold degradation/enhancement parameter for LOD & -- & 0 \\ \hline
KVTH0WE & Threshold shift factor for well proximity effect & -- & 0 \\ \hline
LA0 & Length dependence of a0 & -- & 0 \\ \hline
LA1 & Length dependence of a1 & -- & 0 \\ \hline
LA2 & Length dependence of a2 & -- & 0 \\ \hline
LACDE & Length dependence of acde & -- & 0 \\ \hline
LAGIDL & Length dependence of agidl & -- & 0 \\ \hline
LAGISL & Length dependence of agisl & -- & 0 \\ \hline
LAGS & Length dependence of ags & -- & 0 \\ \hline
LAIGBACC & Length dependence of aigbacc & -- & 0 \\ \hline
LAIGBINV & Length dependence of aigbinv & -- & 0 \\ \hline
LAIGC & Length dependence of aigc & -- & 0 \\ \hline
LAIGD & Length dependence of aigd & -- & 0 \\ \hline
LAIGS & Length dependence of aigs & -- & 0 \\ \hline
LAIGSD & Length dependence of aigsd & -- & 0 \\ \hline
LALPHA0 & Length dependence of alpha0 & -- & 0 \\ \hline
LALPHA1 & Length dependence of alpha1 & -- & 0 \\ \hline
LAT & Length dependence of at & -- & 0 \\ \hline
LB0 & Length dependence of b0 & -- & 0 \\ \hline
LB1 & Length dependence of b1 & -- & 0 \\ \hline
LBETA0 & Length dependence of beta0 & -- & 0 \\ \hline
LBGIDL & Length dependence of bgidl & -- & 0 \\ \hline
LBGISL & Length dependence of bgisl & -- & 0 \\ \hline
LBIGBACC & Length dependence of bigbacc & -- & 0 \\ \hline
LBIGBINV & Length dependence of bigbinv & -- & 0 \\ \hline
LBIGC & Length dependence of bigc & -- & 0 \\ \hline
LBIGD & Length dependence of bigd & -- & 0 \\ \hline
LBIGS & Length dependence of bigs & -- & 0 \\ \hline
LBIGSD & Length dependence of bigsd & -- & 0 \\ \hline
LCDSC & Length dependence of cdsc & -- & 0 \\ \hline
LCDSCB & Length dependence of cdscb & -- & 0 \\ \hline
LCDSCD & Length dependence of cdscd & -- & 0 \\ \hline
LCF & Length dependence of cf & -- & 0 \\ \hline
LCGDL & Length dependence of cgdl & -- & 0 \\ \hline
LCGIDL & Length dependence of cgidl & -- & 0 \\ \hline
LCGISL & Length dependence of cgisl & -- & 0 \\ \hline
LCGSL & Length dependence of cgsl & -- & 0 \\ \hline
LCIGBACC & Length dependence of cigbacc & -- & 0 \\ \hline
LCIGBINV & Length dependence of cigbinv & -- & 0 \\ \hline
LCIGC & Length dependence of cigc & -- & 0 \\ \hline
LCIGD & Length dependence of cigd & -- & 0 \\ \hline
LCIGS & Length dependence of cigs & -- & 0 \\ \hline
LCIGSD & Length dependence of cigsd & -- & 0 \\ \hline
LCIT & Length dependence of cit & -- & 0 \\ \hline
LCKAPPAD & Length dependence of ckappad & -- & 0 \\ \hline
LCKAPPAS & Length dependence of ckappas & -- & 0 \\ \hline
LCLC & Length dependence of clc & -- & 0 \\ \hline
LCLE & Length dependence of cle & -- & 0 \\ \hline
LDELTA & Length dependence of delta & -- & 0 \\ \hline
LDROUT & Length dependence of drout & -- & 0 \\ \hline
LDSUB & Length dependence of dsub & -- & 0 \\ \hline
LDVT0 & Length dependence of dvt0 & -- & 0 \\ \hline
LDVT0W & Length dependence of dvt0w & -- & 0 \\ \hline
LDVT1 & Length dependence of dvt1 & -- & 0 \\ \hline
LDVT1W & Length dependence of dvt1w & -- & 0 \\ \hline
LDVT2 & Length dependence of dvt2 & -- & 0 \\ \hline
LDVT2W & Length dependence of dvt2w & -- & 0 \\ \hline
LDVTP0 & Length dependence of dvtp0 & -- & 0 \\ \hline
LDVTP1 & Length dependence of dvtp1 & -- & 0 \\ \hline
LDVTP2\newline{\normalfont [Only for versions starting with 4.7]} & Length dependence of dvtp2 & -- & 0 \\ \hline
LDVTP3\newline{\normalfont [Only for versions starting with 4.7]} & Length dependence of dvtp3 & -- & 0 \\ \hline
LDVTP4\newline{\normalfont [Only for versions starting with 4.7]} & Length dependence of dvtp4 & -- & 0 \\ \hline
LDVTP5\newline{\normalfont [Only for versions starting with 4.7]} & Length dependence of dvtp5 & -- & 0 \\ \hline
LDWB & Length dependence of dwb & -- & 0 \\ \hline
LDWG & Length dependence of dwg & -- & 0 \\ \hline
LEGIDL & Length dependence of egidl & -- & 0 \\ \hline
LEGISL & Length dependence of egisl & -- & 0 \\ \hline
LEIGBINV & Length dependence for eigbinv & -- & 0 \\ \hline
LETA0 & Length dependence of eta0 & -- & 0 \\ \hline
LETAB & Length dependence of etab & -- & 0 \\ \hline
LEU &  Length dependence of eu & -- & 0 \\ \hline
LFGIDL\newline{\normalfont [Only for versions starting with 4.7]} & Length dependence of fgidl & -- & 0 \\ \hline
LFGISL\newline{\normalfont [Only for versions starting with 4.7]} & Length dependence of fgisl & -- & 0 \\ \hline
LFPROUT & Length dependence of pdiblcb & -- & 0 \\ \hline
LGAMMA1 & Length dependence of gamma1 & -- & 0 \\ \hline
LGAMMA2 & Length dependence of gamma2 & -- & 0 \\ \hline
LINTNOI & lint offset for noise calculation & -- & 0 \\ \hline
LK1 & Length dependence of k1 & -- & 0 \\ \hline
LK2 & Length dependence of k2 & -- & 0 \\ \hline
LK2WE &  Length dependence of k2we  & -- & 0 \\ \hline
LK3 & Length dependence of k3 & -- & 0 \\ \hline
LK3B & Length dependence of k3b & -- & 0 \\ \hline
LKETA & Length dependence of keta & -- & 0 \\ \hline
LKGIDL\newline{\normalfont [Only for versions starting with 4.7]} & Length dependence of kgidl & -- & 0 \\ \hline
LKGISL\newline{\normalfont [Only for versions starting with 4.7]} & Length dependence of kgisl & -- & 0 \\ \hline
LKT1 & Length dependence of kt1 & -- & 0 \\ \hline
LKT1L & Length dependence of kt1l & -- & 0 \\ \hline
LKT2 & Length dependence of kt2 & -- & 0 \\ \hline
LKU0 & Length dependence of ku0 & -- & 0 \\ \hline
LKU0WE &  Length dependence of ku0we  & -- & 0 \\ \hline
LKVTH0 & Length dependence of kvth0 & -- & 0 \\ \hline
LKVTH0WE & Length dependence of kvth0we & -- & 0 \\ \hline
LL & Length reduction parameter & -- & 0 \\ \hline
LLAMBDA & Length dependence of lambda & -- & 0 \\ \hline
LLC & Length reduction parameter for CV & -- & 0 \\ \hline
LLN & Length reduction parameter & -- & 1 \\ \hline
LLODKU0 & Length parameter for u0 LOD effect & -- & 0 \\ \hline
LLODVTH & Length parameter for vth LOD effect & -- & 0 \\ \hline
LLP & Length dependence of lp & -- & 0 \\ \hline
LLPE0 & Length dependence of lpe0 & -- & 0 \\ \hline
LLPEB & Length dependence of lpeb & -- & 0 \\ \hline
LMAX & Maximum length for the model & -- & 1 \\ \hline
LMIN & Minimum length for the model & -- & 0 \\ \hline
LMINV & Length dependence of minv & -- & 0 \\ \hline
LMINVCV & Length dependence of minvcv & -- & 0 \\ \hline
LMOIN & Length dependence of moin & -- & 0 \\ \hline
LNDEP & Length dependence of ndep & -- & 0 \\ \hline
LNFACTOR & Length dependence of nfactor & -- & 0 \\ \hline
LNGATE & Length dependence of ngate & -- & 0 \\ \hline
LNIGBACC & Length dependence of nigbacc & -- & 0 \\ \hline
LNIGBINV & Length dependence of nigbinv & -- & 0 \\ \hline
LNIGC & Length dependence of nigc & -- & 0 \\ \hline
LNOFF & Length dependence of noff & -- & 0 \\ \hline
LNSD & Length dependence of nsd & -- & 0 \\ \hline
LNSUB & Length dependence of nsub & -- & 0 \\ \hline
LNTOX & Length dependence of ntox & -- & 0 \\ \hline
LODETA0 & eta0 shift modification factor for stress effect & -- & 1 \\ \hline
LODK2 & K2 shift modification factor for stress effect & -- & 1 \\ \hline
LPCLM & Length dependence of pclm & -- & 0 \\ \hline
LPDIBLC1 & Length dependence of pdiblc1 & -- & 0 \\ \hline
LPDIBLC2 & Length dependence of pdiblc2 & -- & 0 \\ \hline
LPDIBLCB & Length dependence of pdiblcb & -- & 0 \\ \hline
LPDITS & Length dependence of pdits & -- & 0 \\ \hline
LPDITSD & Length dependence of pditsd & -- & 0 \\ \hline
LPHIN & Length dependence of phin & -- & 0 \\ \hline
LPIGCD & Length dependence for pigcd & -- & 0 \\ \hline
LPOXEDGE & Length dependence for poxedge & -- & 0 \\ \hline
LPRT & Length dependence of prt  & -- & 0 \\ \hline
LPRWB & Length dependence of prwb  & -- & 0 \\ \hline
LPRWG & Length dependence of prwg  & -- & 0 \\ \hline
LPSCBE1 & Length dependence of pscbe1 & -- & 0 \\ \hline
LPSCBE2 & Length dependence of pscbe2 & -- & 0 \\ \hline
LPVAG & Length dependence of pvag & -- & 0 \\ \hline
LRDSW & Length dependence of rdsw  & -- & 0 \\ \hline
LRDW & Length dependence of rdw & -- & 0 \\ \hline
LRGIDL\newline{\normalfont [Only for versions starting with 4.7]} & Length dependence of rgidl & -- & 0 \\ \hline
LRGISL\newline{\normalfont [Only for versions starting with 4.7]} & Length dependence of rgisl & -- & 0 \\ \hline
LRSW & Length dependence of rsw & -- & 0 \\ \hline
LTETA0\newline{\normalfont [Only for versions starting with 4.7]} & Length dependence of teta0 & -- & 0 \\ \hline
LTNFACTOR\newline{\normalfont [Only for versions starting with 4.7]} & Length dependence of tnfactor & -- & 0 \\ \hline
LTVFBSDOFF & Length dependence of tvfbsdoff & -- & 0 \\ \hline
LTVOFF & Length dependence of tvoff & -- & 0 \\ \hline
LTVOFFCV\newline{\normalfont [Only for versions starting with 4.7]} & Length dependence of tvoffcv & -- & 0 \\ \hline
LU0 & Length dependence of u0 & -- & 0 \\ \hline
LUA & Length dependence of ua & -- & 0 \\ \hline
LUA1 & Length dependence of ua1 & -- & 0 \\ \hline
LUB & Length dependence of ub & -- & 0 \\ \hline
LUB1 & Length dependence of ub1 & -- & 0 \\ \hline
LUC & Length dependence of uc & -- & 0 \\ \hline
LUC1 & Length dependence of uc1 & -- & 0 \\ \hline
LUCS\newline{\normalfont [Only for versions starting with 4.7]} &  Length dependence of ucs & -- & 0 \\ \hline
LUCSTE\newline{\normalfont [Only for versions starting with 4.7]} & Length dependence of ucste & -- & 0 \\ \hline
LUD & Length dependence of ud & -- & 0 \\ \hline
LUD1 & Length dependence of ud1 & -- & 0 \\ \hline
LUP & Length dependence of up & -- & 0 \\ \hline
LUTE & Length dependence of ute & -- & 0 \\ \hline
LVBM & Length dependence of vbm & -- & 0 \\ \hline
LVBX & Length dependence of vbx & -- & 0 \\ \hline
LVFB & Length dependence of vfb & -- & 0 \\ \hline
LVFBCV & Length dependence of vfbcv & -- & 0 \\ \hline
LVFBSDOFF & Length dependence of vfbsdoff & -- & 0 \\ \hline
LVOFF & Length dependence of voff & -- & 0 \\ \hline
LVOFFCV & Length dependence of voffcv & -- & 0 \\ \hline
LVSAT & Length dependence of vsat & -- & 0 \\ \hline
LVTH0 &  & -- & 0 \\ \hline
LVTL &  Length dependence of vtl & -- & 0 \\ \hline
LW & Length reduction parameter & -- & 0 \\ \hline
LW0 & Length dependence of w0 & -- & 0 \\ \hline
LWC & Length reduction parameter for CV & -- & 0 \\ \hline
LWL & Length reduction parameter & -- & 0 \\ \hline
LWLC & Length reduction parameter for CV & -- & 0 \\ \hline
LWN & Length reduction parameter & -- & 1 \\ \hline
LWR & Length dependence of wr & -- & 0 \\ \hline
LXJ & Length dependence of xj & -- & 0 \\ \hline
LXN &  Length dependence of xn & -- & 0 \\ \hline
LXRCRG1 & Length dependence of xrcrg1 & -- & 0 \\ \hline
LXRCRG2 & Length dependence of xrcrg2 & -- & 0 \\ \hline
LXT & Length dependence of xt & -- & 0 \\ \hline
MJD & Drain bottom junction capacitance grading coefficient & -- & 0.5 \\ \hline
MJS & Source bottom junction capacitance grading coefficient & -- & 0.5 \\ \hline
MJSWD & Drain sidewall junction capacitance grading coefficient & -- & 0.33 \\ \hline
MJSWGD & Drain (gate side) sidewall junction capacitance grading coefficient & -- & 0.33 \\ \hline
MJSWGS & Source (gate side) sidewall junction capacitance grading coefficient & -- & 0.33 \\ \hline
MJSWS & Source sidewall junction capacitance grading coefficient & -- & 0.33 \\ \hline
NGCON & Number of gate contacts & -- & 1 \\ \hline
NJD & Drain junction emission coefficient & -- & 1 \\ \hline
NJS & Source junction emission coefficient & -- & 1 \\ \hline
NJTS & Non-ideality factor for bottom junction & -- & 20 \\ \hline
NJTSD & Non-ideality factor for bottom junction drain side & -- & 20 \\ \hline
NJTSSW & Non-ideality factor for STI sidewall junction & -- & 20 \\ \hline
NJTSSWD & Non-ideality factor for STI sidewall junction drain side & -- & 20 \\ \hline
NJTSSWG & Non-ideality factor for gate-edge sidewall junction & -- & 20 \\ \hline
NJTSSWGD & Non-ideality factor for gate-edge sidewall junction drain side & -- & 20 \\ \hline
NTNOI & Thermal noise parameter & -- & 1 \\ \hline
PA0 & Cross-term dependence of a0 & -- & 0 \\ \hline
PA1 & Cross-term dependence of a1 & -- & 0 \\ \hline
PA2 & Cross-term dependence of a2 & -- & 0 \\ \hline
PACDE & Cross-term dependence of acde & -- & 0 \\ \hline
PAGIDL & Cross-term dependence of agidl & -- & 0 \\ \hline
PAGISL & Cross-term dependence of agisl & -- & 0 \\ \hline
PAGS & Cross-term dependence of ags & -- & 0 \\ \hline
PAIGBACC & Cross-term dependence of aigbacc & -- & 0 \\ \hline
PAIGBINV & Cross-term dependence of aigbinv & -- & 0 \\ \hline
PAIGC & Cross-term dependence of aigc & -- & 0 \\ \hline
PAIGD & Cross-term dependence of aigd & -- & 0 \\ \hline
PAIGS & Cross-term dependence of aigs & -- & 0 \\ \hline
PAIGSD & Cross-term dependence of aigsd & -- & 0 \\ \hline
PALPHA0 & Cross-term dependence of alpha0 & -- & 0 \\ \hline
PALPHA1 & Cross-term dependence of alpha1 & -- & 0 \\ \hline
PAT & Cross-term dependence of at & -- & 0 \\ \hline
PB0 & Cross-term dependence of b0 & -- & 0 \\ \hline
PB1 & Cross-term dependence of b1 & -- & 0 \\ \hline
PBD & Drain junction built-in potential & -- & 1 \\ \hline
PBETA0 & Cross-term dependence of beta0 & -- & 0 \\ \hline
PBGIDL & Cross-term dependence of bgidl & -- & 0 \\ \hline
PBGISL & Cross-term dependence of bgisl & -- & 0 \\ \hline
PBIGBACC & Cross-term dependence of bigbacc & -- & 0 \\ \hline
PBIGBINV & Cross-term dependence of bigbinv & -- & 0 \\ \hline
PBIGC & Cross-term dependence of bigc & -- & 0 \\ \hline
PBIGD & Cross-term dependence of bigd & -- & 0 \\ \hline
PBIGS & Cross-term dependence of bigs & -- & 0 \\ \hline
PBIGSD & Cross-term dependence of bigsd & -- & 0 \\ \hline
PBS & Source junction built-in potential & -- & 1 \\ \hline
PBSWD & Drain sidewall junction capacitance built in potential & -- & 1 \\ \hline
PBSWGD & Drain (gate side) sidewall junction capacitance built in potential & -- & 0 \\ \hline
PBSWGS & Source (gate side) sidewall junction capacitance built in potential & -- & 0 \\ \hline
PBSWS & Source sidewall junction capacitance built in potential & -- & 1 \\ \hline
PCDSC & Cross-term dependence of cdsc & -- & 0 \\ \hline
PCDSCB & Cross-term dependence of cdscb & -- & 0 \\ \hline
PCDSCD & Cross-term dependence of cdscd & -- & 0 \\ \hline
PCF & Cross-term dependence of cf & -- & 0 \\ \hline
PCGDL & Cross-term dependence of cgdl & -- & 0 \\ \hline
PCGIDL & Cross-term dependence of cgidl & -- & 0 \\ \hline
PCGISL & Cross-term dependence of cgisl & -- & 0 \\ \hline
PCGSL & Cross-term dependence of cgsl & -- & 0 \\ \hline
PCIGBACC & Cross-term dependence of cigbacc & -- & 0 \\ \hline
PCIGBINV & Cross-term dependence of cigbinv & -- & 0 \\ \hline
PCIGC & Cross-term dependence of cigc & -- & 0 \\ \hline
PCIGD & Cross-term dependence of cigd & -- & 0 \\ \hline
PCIGS & Cross-term dependence of cigs & -- & 0 \\ \hline
PCIGSD & Cross-term dependence of cigsd & -- & 0 \\ \hline
PCIT & Cross-term dependence of cit & -- & 0 \\ \hline
PCKAPPAD & Cross-term dependence of ckappad & -- & 0 \\ \hline
PCKAPPAS & Cross-term dependence of ckappas & -- & 0 \\ \hline
PCLC & Cross-term dependence of clc & -- & 0 \\ \hline
PCLE & Cross-term dependence of cle & -- & 0 \\ \hline
PDELTA & Cross-term dependence of delta & -- & 0 \\ \hline
PDROUT & Cross-term dependence of drout & -- & 0 \\ \hline
PDSUB & Cross-term dependence of dsub & -- & 0 \\ \hline
PDVT0 & Cross-term dependence of dvt0 & -- & 0 \\ \hline
PDVT0W & Cross-term dependence of dvt0w & -- & 0 \\ \hline
PDVT1 & Cross-term dependence of dvt1 & -- & 0 \\ \hline
PDVT1W & Cross-term dependence of dvt1w & -- & 0 \\ \hline
PDVT2 & Cross-term dependence of dvt2 & -- & 0 \\ \hline
PDVT2W & Cross-term dependence of dvt2w & -- & 0 \\ \hline
PDVTP0 & Cross-term dependence of dvtp0 & -- & 0 \\ \hline
PDVTP1 & Cross-term dependence of dvtp1 & -- & 0 \\ \hline
PDVTP2\newline{\normalfont [Only for versions starting with 4.7]} & Cross-term dependence of dvtp2 & -- & 0 \\ \hline
PDVTP3\newline{\normalfont [Only for versions starting with 4.7]} & Cross-term dependence of dvtp3 & -- & 0 \\ \hline
PDVTP4\newline{\normalfont [Only for versions starting with 4.7]} & Cross-term dependence of dvtp4 & -- & 0 \\ \hline
PDVTP5\newline{\normalfont [Only for versions starting with 4.7]} & Cross-term dependence of dvtp5 & -- & 0 \\ \hline
PDWB & Cross-term dependence of dwb & -- & 0 \\ \hline
PDWG & Cross-term dependence of dwg & -- & 0 \\ \hline
PEGIDL & Cross-term dependence of egidl & -- & 0 \\ \hline
PEGISL & Cross-term dependence of egisl & -- & 0 \\ \hline
PEIGBINV & Cross-term dependence for eigbinv & -- & 0 \\ \hline
PETA0 & Cross-term dependence of eta0 & -- & 0 \\ \hline
PETAB & Cross-term dependence of etab & -- & 0 \\ \hline
PEU & Cross-term dependence of eu & -- & 0 \\ \hline
PFGIDL\newline{\normalfont [Only for versions starting with 4.7]} & Cross-term dependence of fgidl & -- & 0 \\ \hline
PFGISL\newline{\normalfont [Only for versions starting with 4.7]} & Cross-term dependence of fgisl & -- & 0 \\ \hline
PFPROUT & Cross-term dependence of pdiblcb & -- & 0 \\ \hline
PGAMMA1 & Cross-term dependence of gamma1 & -- & 0 \\ \hline
PGAMMA2 & Cross-term dependence of gamma2 & -- & 0 \\ \hline
PHIG & Work Function of gate & -- & 4.05 \\ \hline
PK1 & Cross-term dependence of k1 & -- & 0 \\ \hline
PK2 & Cross-term dependence of k2 & -- & 0 \\ \hline
PK2WE &  Cross-term dependence of k2we  & -- & 0 \\ \hline
PK3 & Cross-term dependence of k3 & -- & 0 \\ \hline
PK3B & Cross-term dependence of k3b & -- & 0 \\ \hline
PKETA & Cross-term dependence of keta & -- & 0 \\ \hline
PKGIDL\newline{\normalfont [Only for versions starting with 4.7]} & Cross-term dependence of kgidl & -- & 0 \\ \hline
PKGISL\newline{\normalfont [Only for versions starting with 4.7]} & Cross-term dependence of kgisl & -- & 0 \\ \hline
PKT1 & Cross-term dependence of kt1 & -- & 0 \\ \hline
PKT1L & Cross-term dependence of kt1l & -- & 0 \\ \hline
PKT2 & Cross-term dependence of kt2 & -- & 0 \\ \hline
PKU0 & Cross-term dependence of ku0 & -- & 0 \\ \hline
PKU0WE &  Cross-term dependence of ku0we  & -- & 0 \\ \hline
PKVTH0 & Cross-term dependence of kvth0 & -- & 0 \\ \hline
PKVTH0WE & Cross-term dependence of kvth0we & -- & 0 \\ \hline
PLAMBDA & Cross-term dependence of lambda & -- & 0 \\ \hline
PLP & Cross-term dependence of lp & -- & 0 \\ \hline
PLPE0 & Cross-term dependence of lpe0 & -- & 0 \\ \hline
PLPEB & Cross-term dependence of lpeb & -- & 0 \\ \hline
PMINV & Cross-term dependence of minv & -- & 0 \\ \hline
PMINVCV & Cross-term dependence of minvcv & -- & 0 \\ \hline
PMOIN & Cross-term dependence of moin & -- & 0 \\ \hline
PNDEP & Cross-term dependence of ndep & -- & 0 \\ \hline
PNFACTOR & Cross-term dependence of nfactor & -- & 0 \\ \hline
PNGATE & Cross-term dependence of ngate & -- & 0 \\ \hline
PNIGBACC & Cross-term dependence of nigbacc & -- & 0 \\ \hline
PNIGBINV & Cross-term dependence of nigbinv & -- & 0 \\ \hline
PNIGC & Cross-term dependence of nigc & -- & 0 \\ \hline
PNOFF & Cross-term dependence of noff & -- & 0 \\ \hline
PNSD & Cross-term dependence of nsd & -- & 0 \\ \hline
PNSUB & Cross-term dependence of nsub & -- & 0 \\ \hline
PNTOX & Cross-term dependence of ntox & -- & 0 \\ \hline
PPCLM & Cross-term dependence of pclm & -- & 0 \\ \hline
PPDIBLC1 & Cross-term dependence of pdiblc1 & -- & 0 \\ \hline
PPDIBLC2 & Cross-term dependence of pdiblc2 & -- & 0 \\ \hline
PPDIBLCB & Cross-term dependence of pdiblcb & -- & 0 \\ \hline
PPDITS & Cross-term dependence of pdits & -- & 0 \\ \hline
PPDITSD & Cross-term dependence of pditsd & -- & 0 \\ \hline
PPHIN & Cross-term dependence of phin & -- & 0 \\ \hline
PPIGCD & Cross-term dependence for pigcd & -- & 0 \\ \hline
PPOXEDGE & Cross-term dependence for poxedge & -- & 0 \\ \hline
PPRT & Cross-term dependence of prt  & -- & 0 \\ \hline
PPRWB & Cross-term dependence of prwb  & -- & 0 \\ \hline
PPRWG & Cross-term dependence of prwg  & -- & 0 \\ \hline
PPSCBE1 & Cross-term dependence of pscbe1 & -- & 0 \\ \hline
PPSCBE2 & Cross-term dependence of pscbe2 & -- & 0 \\ \hline
PPVAG & Cross-term dependence of pvag & -- & 0 \\ \hline
PRDSW & Cross-term dependence of rdsw  & -- & 0 \\ \hline
PRDW & Cross-term dependence of rdw & -- & 0 \\ \hline
PRGIDL\newline{\normalfont [Only for versions starting with 4.7]} & Cross-term dependence of rgidl & -- & 0 \\ \hline
PRGISL\newline{\normalfont [Only for versions starting with 4.7]} & Cross-term dependence of rgisl & -- & 0 \\ \hline
PRSW & Cross-term dependence of rsw & -- & 0 \\ \hline
PRT & Temperature coefficient of parasitic resistance  & -- & 0 \\ \hline
PTETA0\newline{\normalfont [Only for versions starting with 4.7]} & Cross-term dependence of teta0 & -- & 0 \\ \hline
PTNFACTOR\newline{\normalfont [Only for versions starting with 4.7]} & Cross-term dependence of tnfactor & -- & 0 \\ \hline
PTVFBSDOFF & Cross-term dependence of tvfbsdoff & -- & 0 \\ \hline
PTVOFF & Cross-term dependence of tvoff & -- & 0 \\ \hline
PTVOFFCV\newline{\normalfont [Only for versions starting with 4.7]} & Cross-term dependence of tvoffcv & -- & 0 \\ \hline
PU0 & Cross-term dependence of u0 & -- & 0 \\ \hline
PUA & Cross-term dependence of ua & -- & 0 \\ \hline
PUA1 & Cross-term dependence of ua1 & -- & 0 \\ \hline
PUB & Cross-term dependence of ub & -- & 0 \\ \hline
PUB1 & Cross-term dependence of ub1 & -- & 0 \\ \hline
PUC & Cross-term dependence of uc & -- & 0 \\ \hline
PUC1 & Cross-term dependence of uc1 & -- & 0 \\ \hline
PUCS\newline{\normalfont [Only for versions starting with 4.7]} & Cross-term dependence of ucs & -- & 0 \\ \hline
PUCSTE\newline{\normalfont [Only for versions starting with 4.7]} & Cross-term dependence of ucste & -- & 0 \\ \hline
PUD & Cross-term dependence of ud & -- & 0 \\ \hline
PUD1 & Cross-term dependence of ud1 & -- & 0 \\ \hline
PUP & Cross-term dependence of up & -- & 0 \\ \hline
PUTE & Cross-term dependence of ute & -- & 0 \\ \hline
PVAG & Gate dependence of output resistance parameter & -- & 0 \\ \hline
PVBM & Cross-term dependence of vbm & -- & 0 \\ \hline
PVBX & Cross-term dependence of vbx & -- & 0 \\ \hline
PVFB & Cross-term dependence of vfb & -- & 0 \\ \hline
PVFBCV & Cross-term dependence of vfbcv & -- & 0 \\ \hline
PVFBSDOFF & Cross-term dependence of vfbsdoff & -- & 0 \\ \hline
PVOFF & Cross-term dependence of voff & -- & 0 \\ \hline
PVOFFCV & Cross-term dependence of voffcv & -- & 0 \\ \hline
PVSAT & Cross-term dependence of vsat & -- & 0 \\ \hline
PVTH0 &  & -- & 0 \\ \hline
PVTL & Cross-term dependence of vtl & -- & 0 \\ \hline
PW0 & Cross-term dependence of w0 & -- & 0 \\ \hline
PWR & Cross-term dependence of wr & -- & 0 \\ \hline
PXJ & Cross-term dependence of xj & -- & 0 \\ \hline
PXN & Cross-term dependence of xn & -- & 0 \\ \hline
PXRCRG1 & Cross-term dependence of xrcrg1 & -- & 0 \\ \hline
PXRCRG2 & Cross-term dependence of xrcrg2 & -- & 0 \\ \hline
PXT & Cross-term dependence of xt & -- & 0 \\ \hline
RBDB & Resistance between bNode and dbNode & $\mathsf{\Omega}$ & 50 \\ \hline
RBDBX0 & Body resistance RBDBX  scaling & -- & 100 \\ \hline
RBDBY0 & Body resistance RBDBY  scaling & -- & 100 \\ \hline
RBPB & Resistance between bNodePrime and bNode & $\mathsf{\Omega}$ & 50 \\ \hline
RBPBX0 & Body resistance RBPBX  scaling & -- & 100 \\ \hline
RBPBXL & Body resistance RBPBX L scaling & -- & 0 \\ \hline
RBPBXNF & Body resistance RBPBX NF scaling & -- & 0 \\ \hline
RBPBXW & Body resistance RBPBX W scaling & -- & 0 \\ \hline
RBPBY0 & Body resistance RBPBY  scaling & -- & 100 \\ \hline
RBPBYL & Body resistance RBPBY L scaling & -- & 0 \\ \hline
RBPBYNF & Body resistance RBPBY NF scaling & -- & 0 \\ \hline
RBPBYW & Body resistance RBPBY W scaling & -- & 0 \\ \hline
RBPD & Resistance between bNodePrime and bNode & $\mathsf{\Omega}$ & 50 \\ \hline
RBPD0 & Body resistance RBPD scaling & -- & 50 \\ \hline
RBPDL & Body resistance RBPD L scaling & -- & 0 \\ \hline
RBPDNF & Body resistance RBPD NF scaling & -- & 0 \\ \hline
RBPDW & Body resistance RBPD W scaling & -- & 0 \\ \hline
RBPS & Resistance between bNodePrime and sbNode & $\mathsf{\Omega}$ & 50 \\ \hline
RBPS0 & Body resistance RBPS scaling & -- & 50 \\ \hline
RBPSL & Body resistance RBPS L scaling & -- & 0 \\ \hline
RBPSNF & Body resistance RBPS NF scaling & -- & 0 \\ \hline
RBPSW & Body resistance RBPS W scaling & -- & 0 \\ \hline
RBSB & Resistance between bNode and sbNode & $\mathsf{\Omega}$ & 50 \\ \hline
RBSBX0 & Body resistance RBSBX  scaling & -- & 100 \\ \hline
RBSBY0 & Body resistance RBSBY  scaling & -- & 100 \\ \hline
RBSDBXL & Body resistance RBSDBX L scaling & -- & 0 \\ \hline
RBSDBXNF & Body resistance RBSDBX NF scaling & -- & 0 \\ \hline
RBSDBXW & Body resistance RBSDBX W scaling & -- & 0 \\ \hline
RBSDBYL & Body resistance RBSDBY L scaling & -- & 0 \\ \hline
RBSDBYNF & Body resistance RBSDBY NF scaling & -- & 0 \\ \hline
RBSDBYW & Body resistance RBSDBY W scaling & -- & 0 \\ \hline
RNOIA & Thermal noise coefficient & -- & 0.577 \\ \hline
RNOIB & Thermal noise coefficient & -- & 0.5164 \\ \hline
RNOIC\newline{\normalfont [Only for versions starting with 4.7]} & Thermal noise coefficient & -- & 0.395 \\ \hline
SAREF & Reference distance between OD edge to poly of one side & -- & 1e-06 \\ \hline
SBREF & Reference distance between OD edge to poly of the other side & -- & 1e-06 \\ \hline
SCREF &  Reference distance to calculate SCA,SCB and SCC & -- & 1e-06 \\ \hline
STETA0 & eta0 shift factor related to stress effect on vth & -- & 0 \\ \hline
STK2 & K2 shift factor related to stress effect on vth & -- & 0 \\ \hline
TCJ & Temperature coefficient of cj & -- & 0 \\ \hline
TCJSW & Temperature coefficient of cjsw & -- & 0 \\ \hline
TCJSWG & Temperature coefficient of cjswg & -- & 0 \\ \hline
TETA0\newline{\normalfont [Only for versions starting with 4.7]} & Temperature parameter for eta0 & -- & 0 \\ \hline
TKU0 & Temperature coefficient of KU0 & -- & 0 \\ \hline
TNFACTOR\newline{\normalfont [Only for versions starting with 4.7]} & Temperature parameter for nfactor & -- & 0 \\ \hline
TNJTS & Temperature coefficient for NJTS & -- & 0 \\ \hline
TNJTSD & Temperature coefficient for NJTSD & -- & 0 \\ \hline
TNJTSSW & Temperature coefficient for NJTSSW & -- & 0 \\ \hline
TNJTSSWD & Temperature coefficient for NJTSSWD & -- & 0 \\ \hline
TNJTSSWG & Temperature coefficient for NJTSSWG & -- & 0 \\ \hline
TNJTSSWGD & Temperature coefficient for NJTSSWGD & -- & 0 \\ \hline
TNOIA & Thermal noise parameter & -- & 1.5 \\ \hline
TNOIB & Thermal noise parameter & -- & 3.5 \\ \hline
TNOIC\newline{\normalfont [Only for versions starting with 4.7]} & Thermal noise parameter & -- & 0 \\ \hline
TNOM & Parameter measurement temperature & -- & Ambient Temperature \\ \hline
TPB & Temperature coefficient of pb & -- & 0 \\ \hline
TPBSW & Temperature coefficient of pbsw & -- & 0 \\ \hline
TPBSWG & Temperature coefficient of pbswg & -- & 0 \\ \hline
TVFBSDOFF & Temperature parameter for vfbsdoff & -- & 0 \\ \hline
TVOFF & Temperature parameter for voff & -- & 0 \\ \hline
TVOFFCV\newline{\normalfont [Only for versions starting with 4.7]} & Temperature parameter for tvoffcv & -- & 0 \\ \hline
UA1 & Temperature coefficient of ua & -- & 1e-09 \\ \hline
UB1 & Temperature coefficient of ub & -- & -1e-18 \\ \hline
UC1 & Temperature coefficient of uc & -- & 0 \\ \hline
UCSTE\newline{\normalfont [Only for versions starting with 4.7]} & Temperature coefficient of colombic mobility & -- & -0.004775 \\ \hline
UD1 & Temperature coefficient of ud & -- & 0 \\ \hline
UTE & Temperature coefficient of mobility & -- & -1.5 \\ \hline
VTSD & Drain bottom trap-assisted voltage dependent parameter & -- & 10 \\ \hline
VTSS & Source bottom trap-assisted voltage dependent parameter & -- & 10 \\ \hline
VTSSWD & Drain STI sidewall trap-assisted voltage dependent parameter & -- & 10 \\ \hline
VTSSWGD & Drain gate-edge sidewall trap-assisted voltage dependent parameter & -- & 10 \\ \hline
VTSSWGS & Source gate-edge sidewall trap-assisted voltage dependent parameter & -- & 10 \\ \hline
VTSSWS & Source STI sidewall trap-assisted voltage dependent parameter & -- & 10 \\ \hline
WA0 & Width dependence of a0 & -- & 0 \\ \hline
WA1 & Width dependence of a1 & -- & 0 \\ \hline
WA2 & Width dependence of a2 & -- & 0 \\ \hline
WACDE & Width dependence of acde & -- & 0 \\ \hline
WAGIDL & Width dependence of agidl & -- & 0 \\ \hline
WAGISL & Width dependence of agisl & -- & 0 \\ \hline
WAGS & Width dependence of ags & -- & 0 \\ \hline
WAIGBACC & Width dependence of aigbacc & -- & 0 \\ \hline
WAIGBINV & Width dependence of aigbinv & -- & 0 \\ \hline
WAIGC & Width dependence of aigc & -- & 0 \\ \hline
WAIGD & Width dependence of aigd & -- & 0 \\ \hline
WAIGS & Width dependence of aigs & -- & 0 \\ \hline
WAIGSD & Width dependence of aigsd & -- & 0 \\ \hline
WALPHA0 & Width dependence of alpha0 & -- & 0 \\ \hline
WALPHA1 & Width dependence of alpha1 & -- & 0 \\ \hline
WAT & Width dependence of at & -- & 0 \\ \hline
WB0 & Width dependence of b0 & -- & 0 \\ \hline
WB1 & Width dependence of b1 & -- & 0 \\ \hline
WBETA0 & Width dependence of beta0 & -- & 0 \\ \hline
WBGIDL & Width dependence of bgidl & -- & 0 \\ \hline
WBGISL & Width dependence of bgisl & -- & 0 \\ \hline
WBIGBACC & Width dependence of bigbacc & -- & 0 \\ \hline
WBIGBINV & Width dependence of bigbinv & -- & 0 \\ \hline
WBIGC & Width dependence of bigc & -- & 0 \\ \hline
WBIGD & Width dependence of bigd & -- & 0 \\ \hline
WBIGS & Width dependence of bigs & -- & 0 \\ \hline
WBIGSD & Width dependence of bigsd & -- & 0 \\ \hline
WCDSC & Width dependence of cdsc & -- & 0 \\ \hline
WCDSCB & Width dependence of cdscb & -- & 0 \\ \hline
WCDSCD & Width dependence of cdscd & -- & 0 \\ \hline
WCF & Width dependence of cf & -- & 0 \\ \hline
WCGDL & Width dependence of cgdl & -- & 0 \\ \hline
WCGIDL & Width dependence of cgidl & -- & 0 \\ \hline
WCGISL & Width dependence of cgisl & -- & 0 \\ \hline
WCGSL & Width dependence of cgsl & -- & 0 \\ \hline
WCIGBACC & Width dependence of cigbacc & -- & 0 \\ \hline
WCIGBINV & Width dependence of cigbinv & -- & 0 \\ \hline
WCIGC & Width dependence of cigc & -- & 0 \\ \hline
WCIGD & Width dependence of cigd & -- & 0 \\ \hline
WCIGS & Width dependence of cigs & -- & 0 \\ \hline
WCIGSD & Width dependence of cigsd & -- & 0 \\ \hline
WCIT & Width dependence of cit & -- & 0 \\ \hline
WCKAPPAD & Width dependence of ckappad & -- & 0 \\ \hline
WCKAPPAS & Width dependence of ckappas & -- & 0 \\ \hline
WCLC & Width dependence of clc & -- & 0 \\ \hline
WCLE & Width dependence of cle & -- & 0 \\ \hline
WDELTA & Width dependence of delta & -- & 0 \\ \hline
WDROUT & Width dependence of drout & -- & 0 \\ \hline
WDSUB & Width dependence of dsub & -- & 0 \\ \hline
WDVT0 & Width dependence of dvt0 & -- & 0 \\ \hline
WDVT0W & Width dependence of dvt0w & -- & 0 \\ \hline
WDVT1 & Width dependence of dvt1 & -- & 0 \\ \hline
WDVT1W & Width dependence of dvt1w & -- & 0 \\ \hline
WDVT2 & Width dependence of dvt2 & -- & 0 \\ \hline
WDVT2W & Width dependence of dvt2w & -- & 0 \\ \hline
WDVTP0 & Width dependence of dvtp0 & -- & 0 \\ \hline
WDVTP1 & Width dependence of dvtp1 & -- & 0 \\ \hline
WDVTP2\newline{\normalfont [Only for versions starting with 4.7]} & Width dependence of dvtp2 & -- & 0 \\ \hline
WDVTP3\newline{\normalfont [Only for versions starting with 4.7]} & Width dependence of dvtp3 & -- & 0 \\ \hline
WDVTP4\newline{\normalfont [Only for versions starting with 4.7]} & Width dependence of dvtp4 & -- & 0 \\ \hline
WDVTP5\newline{\normalfont [Only for versions starting with 4.7]} & Width dependence of dvtp5 & -- & 0 \\ \hline
WDWB & Width dependence of dwb & -- & 0 \\ \hline
WDWG & Width dependence of dwg & -- & 0 \\ \hline
WEB & Coefficient for SCB & -- & 0 \\ \hline
WEC & Coefficient for SCC & -- & 0 \\ \hline
WEGIDL & Width dependence of egidl & -- & 0 \\ \hline
WEGISL & Width dependence of egisl & -- & 0 \\ \hline
WEIGBINV & Width dependence for eigbinv & -- & 0 \\ \hline
WETA0 & Width dependence of eta0 & -- & 0 \\ \hline
WETAB & Width dependence of etab & -- & 0 \\ \hline
WEU & Width dependence of eu & -- & 0 \\ \hline
WFGIDL\newline{\normalfont [Only for versions starting with 4.7]} & Width dependence of fgidl & -- & 0 \\ \hline
WFGISL\newline{\normalfont [Only for versions starting with 4.7]} & Width dependence of fgisl & -- & 0 \\ \hline
WFPROUT & Width dependence of pdiblcb & -- & 0 \\ \hline
WGAMMA1 & Width dependence of gamma1 & -- & 0 \\ \hline
WGAMMA2 & Width dependence of gamma2 & -- & 0 \\ \hline
WK1 & Width dependence of k1 & -- & 0 \\ \hline
WK2 & Width dependence of k2 & -- & 0 \\ \hline
WK2WE &  Width dependence of k2we  & -- & 0 \\ \hline
WK3 & Width dependence of k3 & -- & 0 \\ \hline
WK3B & Width dependence of k3b & -- & 0 \\ \hline
WKETA & Width dependence of keta & -- & 0 \\ \hline
WKGIDL\newline{\normalfont [Only for versions starting with 4.7]} & Width dependence of kgidl & -- & 0 \\ \hline
WKGISL\newline{\normalfont [Only for versions starting with 4.7]} & Width dependence of kgisl & -- & 0 \\ \hline
WKT1 & Width dependence of kt1 & -- & 0 \\ \hline
WKT1L & Width dependence of kt1l & -- & 0 \\ \hline
WKT2 & Width dependence of kt2 & -- & 0 \\ \hline
WKU0 & Width dependence of ku0 & -- & 0 \\ \hline
WKU0WE &  Width dependence of ku0we  & -- & 0 \\ \hline
WKVTH0 & Width dependence of kvth0 & -- & 0 \\ \hline
WKVTH0WE & Width dependence of kvth0we & -- & 0 \\ \hline
WL & Width reduction parameter & -- & 0 \\ \hline
WLAMBDA & Width dependence of lambda & -- & 0 \\ \hline
WLC & Width reduction parameter for CV & -- & 0 \\ \hline
WLN & Width reduction parameter & -- & 1 \\ \hline
WLOD & Width parameter for stress effect & -- & 0 \\ \hline
WLODKU0 & Width parameter for u0 LOD effect & -- & 0 \\ \hline
WLODVTH & Width parameter for vth LOD effect & -- & 0 \\ \hline
WLP & Width dependence of lp & -- & 0 \\ \hline
WLPE0 & Width dependence of lpe0 & -- & 0 \\ \hline
WLPEB & Width dependence of lpeb & -- & 0 \\ \hline
WMAX & Maximum width for the model & -- & 1 \\ \hline
WMIN & Minimum width for the model & -- & 0 \\ \hline
WMINV & Width dependence of minv & -- & 0 \\ \hline
WMINVCV & Width dependence of minvcv & -- & 0 \\ \hline
WMOIN & Width dependence of moin & -- & 0 \\ \hline
WNDEP & Width dependence of ndep & -- & 0 \\ \hline
WNFACTOR & Width dependence of nfactor & -- & 0 \\ \hline
WNGATE & Width dependence of ngate & -- & 0 \\ \hline
WNIGBACC & Width dependence of nigbacc & -- & 0 \\ \hline
WNIGBINV & Width dependence of nigbinv & -- & 0 \\ \hline
WNIGC & Width dependence of nigc & -- & 0 \\ \hline
WNOFF & Width dependence of noff & -- & 0 \\ \hline
WNSD & Width dependence of nsd & -- & 0 \\ \hline
WNSUB & Width dependence of nsub & -- & 0 \\ \hline
WNTOX & Width dependence of ntox & -- & 0 \\ \hline
WPCLM & Width dependence of pclm & -- & 0 \\ \hline
WPDIBLC1 & Width dependence of pdiblc1 & -- & 0 \\ \hline
WPDIBLC2 & Width dependence of pdiblc2 & -- & 0 \\ \hline
WPDIBLCB & Width dependence of pdiblcb & -- & 0 \\ \hline
WPDITS & Width dependence of pdits & -- & 0 \\ \hline
WPDITSD & Width dependence of pditsd & -- & 0 \\ \hline
WPEMOD &  Flag for WPE model (WPEMOD=1 to activate this model)  & -- & 0 \\ \hline
WPHIN & Width dependence of phin & -- & 0 \\ \hline
WPIGCD & Width dependence for pigcd & -- & 0 \\ \hline
WPOXEDGE & Width dependence for poxedge & -- & 0 \\ \hline
WPRT & Width dependence of prt & -- & 0 \\ \hline
WPRWB & Width dependence of prwb  & -- & 0 \\ \hline
WPRWG & Width dependence of prwg  & -- & 0 \\ \hline
WPSCBE1 & Width dependence of pscbe1 & -- & 0 \\ \hline
WPSCBE2 & Width dependence of pscbe2 & -- & 0 \\ \hline
WPVAG & Width dependence of pvag & -- & 0 \\ \hline
WRDSW & Width dependence of rdsw  & -- & 0 \\ \hline
WRDW & Width dependence of rdw & -- & 0 \\ \hline
WRGIDL\newline{\normalfont [Only for versions starting with 4.7]} & Width dependence of rgidl & -- & 0 \\ \hline
WRGISL\newline{\normalfont [Only for versions starting with 4.7]} & Width dependence of rgisl & -- & 0 \\ \hline
WRSW & Width dependence of rsw & -- & 0 \\ \hline
WTETA0\newline{\normalfont [Only for versions starting with 4.7]} & Width dependence of teta0 & -- & 0 \\ \hline
WTNFACTOR\newline{\normalfont [Only for versions starting with 4.7]} & Width dependence of tnfactor & -- & 0 \\ \hline
WTVFBSDOFF & Width dependence of tvfbsdoff & -- & 0 \\ \hline
WTVOFF & Width dependence of tvoff & -- & 0 \\ \hline
WTVOFFCV\newline{\normalfont [Only for versions starting with 4.7]} & Width dependence of tvoffcv & -- & 0 \\ \hline
WU0 & Width dependence of u0 & -- & 0 \\ \hline
WUA & Width dependence of ua & -- & 0 \\ \hline
WUA1 & Width dependence of ua1 & -- & 0 \\ \hline
WUB & Width dependence of ub & -- & 0 \\ \hline
WUB1 & Width dependence of ub1 & -- & 0 \\ \hline
WUC & Width dependence of uc & -- & 0 \\ \hline
WUC1 & Width dependence of uc1 & -- & 0 \\ \hline
WUCS\newline{\normalfont [Only for versions starting with 4.7]} & Width dependence of ucs & -- & 0 \\ \hline
WUCSTE\newline{\normalfont [Only for versions starting with 4.7]} & Width dependence of ucste & -- & 0 \\ \hline
WUD & Width dependence of ud & -- & 0 \\ \hline
WUD1 & Width dependence of ud1 & -- & 0 \\ \hline
WUP & Width dependence of up & -- & 0 \\ \hline
WUTE & Width dependence of ute & -- & 0 \\ \hline
WVBM & Width dependence of vbm & -- & 0 \\ \hline
WVBX & Width dependence of vbx & -- & 0 \\ \hline
WVFB & Width dependence of vfb & -- & 0 \\ \hline
WVFBCV & Width dependence of vfbcv & -- & 0 \\ \hline
WVFBSDOFF & Width dependence of vfbsdoff & -- & 0 \\ \hline
WVOFF & Width dependence of voff & -- & 0 \\ \hline
WVOFFCV & Width dependence of voffcv & -- & 0 \\ \hline
WVSAT & Width dependence of vsat & -- & 0 \\ \hline
WVTH0 &  & -- & 0 \\ \hline
WVTL & Width dependence of vtl & -- & 0 \\ \hline
WW & Width reduction parameter & -- & 0 \\ \hline
WW0 & Width dependence of w0 & -- & 0 \\ \hline
WWC & Width reduction parameter for CV & -- & 0 \\ \hline
WWL & Width reduction parameter & -- & 0 \\ \hline
WWLC & Width reduction parameter for CV & -- & 0 \\ \hline
WWN & Width reduction parameter & -- & 1 \\ \hline
WWR & Width dependence of wr & -- & 0 \\ \hline
WXJ & Width dependence of xj & -- & 0 \\ \hline
WXN & Width dependence of xn & -- & 0 \\ \hline
WXRCRG1 & Width dependence of xrcrg1 & -- & 0 \\ \hline
WXRCRG2 & Width dependence of xrcrg2 & -- & 0 \\ \hline
WXT & Width dependence of xt & -- & 0 \\ \hline
XGL & Variation in Ldrawn & -- & 0 \\ \hline
XGW & Distance from gate contact center to device edge & -- & 0 \\ \hline
XJBVD & Fitting parameter for drain diode breakdown current & -- & 1 \\ \hline
XJBVS & Fitting parameter for source diode breakdown current & -- & 1 \\ \hline
XL & L offset for channel length due to mask/etch effect & -- & 0 \\ \hline
XRCRG1 & First fitting parameter the bias-dependent Rg & -- & 12 \\ \hline
XRCRG2 & Second fitting parameter the bias-dependent Rg & -- & 1 \\ \hline
XTID & Drainjunction current temperature exponent & -- & 3 \\ \hline
XTIS & Source junction current temperature exponent & -- & 3 \\ \hline
XTSD & Power dependence of JTSD on temperature & -- & 0.02 \\ \hline
XTSS & Power dependence of JTSS on temperature & -- & 0.02 \\ \hline
XTSSWD & Power dependence of JTSSWD on temperature & -- & 0.02 \\ \hline
XTSSWGD & Power dependence of JTSSWGD on temperature & -- & 0.02 \\ \hline
XTSSWGS & Power dependence of JTSSWGS on temperature & -- & 0.02 \\ \hline
XTSSWS & Power dependence of JTSSWS on temperature & -- & 0.02 \\ \hline
XW & W offset for channel width due to mask/etch effect & -- & 0 \\ \hline

\category{Basic Parameters}\\ \hline
A0 & Non-uniform depletion width effect coefficient. & -- & 1 \\ \hline
A1 & Non-saturation effect coefficient & V$^{-1}$ & 0 \\ \hline
A2 & Non-saturation effect coefficient & -- & 1 \\ \hline
ADOS & Charge centroid parameter & -- & 1 \\ \hline
AGS & Gate bias  coefficient of Abulk. & V$^{-1}$ & 0 \\ \hline
B0 & Abulk narrow width parameter & m & 0 \\ \hline
B1 & Abulk narrow width parameter & m & 0 \\ \hline
BDOS & Charge centroid parameter & -- & 1 \\ \hline
BG0SUB & Band-gap of substrate at T=0K & eV & 1.16 \\ \hline
CDSC & Drain/Source and channel coupling capacitance & F/m$^{2}$ & 0.00024 \\ \hline
CDSCB & Body-bias dependence of cdsc & F/(Vm$^{2}$) & 0 \\ \hline
CDSCD & Drain-bias dependence of cdsc & F/(Vm$^{2}$) & 0 \\ \hline
CIT & Interface state capacitance & F/m$^{2}$ & 0 \\ \hline
DELTA & Effective Vds parameter & V & 0.01 \\ \hline
DROUT & DIBL coefficient of output resistance & -- & 0.56 \\ \hline
DSUB & DIBL coefficient in the subthreshold region & -- & 0 \\ \hline
DVT0 & Short channel effect coeff. 0 & -- & 2.2 \\ \hline
DVT0W & Narrow Width coeff. 0 & -- & 0 \\ \hline
DVT1 & Short channel effect coeff. 1 & -- & 0.53 \\ \hline
DVT1W & Narrow Width effect coeff. 1 & m$^{-1}$ & 5.3e+06 \\ \hline
DVT2 & Short channel effect coeff. 2 & V$^{-1}$ & -0.032 \\ \hline
DVT2W & Narrow Width effect coeff. 2 & V$^{-1}$ & -0.032 \\ \hline
DVTP0 & First parameter for Vth shift due to pocket & m & 0 \\ \hline
DVTP1 & Second parameter for Vth shift due to pocket & V$^{-1}$ & 0 \\ \hline
DVTP2\newline{\normalfont [Only for versions starting with 4.7]} & 3rd parameter for Vth shift due to pocket & Vm$^{X}$ & 0 \\ \hline
DVTP3\newline{\normalfont [Only for versions starting with 4.7]} & 4th parameter for Vth shift due to pocket & -- & 0 \\ \hline
DVTP4\newline{\normalfont [Only for versions starting with 4.7]} & 5th parameter for Vth shift due to pocket & V$^{-1}$ & 0 \\ \hline
DVTP5\newline{\normalfont [Only for versions starting with 4.7]} & 6th parameter for Vth shift due to pocket & V & 0 \\ \hline
DWB & Width reduction parameter & m/V$^{1/2}$ & 0 \\ \hline
DWG & Width reduction parameter & m/V & 0 \\ \hline
EASUB & Electron affinity of substrate & V & 4.05 \\ \hline
EPSRSUB & Dielectric constant of substrate relative to vacuum & -- & 11.7 \\ \hline
ETA0 & Subthreshold region DIBL coefficient & -- & 0.08 \\ \hline
ETAB & Subthreshold region DIBL coefficient & V$^{-1}$ & -0.07 \\ \hline
EU & Mobility exponent & -- & 0 \\ \hline
FPROUT & Rout degradation coefficient for pocket devices & V/m$^{1/2}$ & 0 \\ \hline
K1 & Bulk effect coefficient 1 & V$^{-1/2}$ & 0 \\ \hline
K2 & Bulk effect coefficient 2 & -- & 0 \\ \hline
K3 & Narrow width effect coefficient & -- & 80 \\ \hline
KETA & Body-bias coefficient of non-uniform depletion width effect. & V$^{-1}$ & -0.047 \\ \hline
LAMBDA &  Velocity overshoot parameter & -- & 0 \\ \hline
LC &  back scattering parameter & m & 5e-09 \\ \hline
LEFFEOT\newline{\normalfont [Only for versions starting with 4.7]} & Effective length for extraction of EOT & m & 1e-06 \\ \hline
LINT & Length reduction parameter & m & 0 \\ \hline
LP & Channel length exponential factor of mobility & m & 1e-08 \\ \hline
LPE0 & Equivalent length of pocket region at zero bias & m & 1.74e-07 \\ \hline
LPEB & Equivalent length of pocket region accounting for body bias & m & 0 \\ \hline
MINV & Fitting parameter for moderate inversion in Vgsteff & -- & 0 \\ \hline
NFACTOR & Subthreshold swing Coefficient & -- & 1 \\ \hline
NI0SUB & Intrinsic carrier concentration of substrate at 300.15K & cm$^{-3}$ & 1.45e+10 \\ \hline
PCLM & Channel length modulation Coefficient & -- & 1.3 \\ \hline
PDIBLC1 & Drain-induced barrier lowering coefficient & -- & 0.39 \\ \hline
PDIBLC2 & Drain-induced barrier lowering coefficient & -- & 0.0086 \\ \hline
PDIBLCB & Body-effect on drain-induced barrier lowering & V$^{-1}$ & 0 \\ \hline
PDITS & Coefficient for drain-induced Vth shifts & V$^{-1}$ & 0 \\ \hline
PDITSD & Vds dependence of drain-induced Vth shifts & V$^{-1}$ & 0 \\ \hline
PDITSL & Length dependence of drain-induced Vth shifts & m$^{-1}$ & 0 \\ \hline
PHIN & Adjusting parameter for surface potential due to non-uniform vertical doping & V & 0 \\ \hline
PSCBE1 & Substrate current body-effect coefficient & Vm$^{-1}$ & 4.24e+08 \\ \hline
PSCBE2 & Substrate current body-effect coefficient & m/V & 1e-05 \\ \hline
TBGASUB & First parameter of band-gap change due to temperature & eV/K & 0.000702 \\ \hline
TBGBSUB & Second parameter of band-gap change due to temperature & K & 1108 \\ \hline
TEMPEOT\newline{\normalfont [Only for versions starting with 4.7]} & Temperature for extraction of EOT & -- & 300.15 \\ \hline
U0 & Low-field mobility at Tnom & m$^{2}$/(Vs) & 0 \\ \hline
UA & Linear gate dependence of mobility & m/V & 0 \\ \hline
UB & Quadratic gate dependence of mobility & m$^{2}$/V$^{2}$ & 1e-19 \\ \hline
UC & Body-bias dependence of mobility & V$^{-1}$ & 0 \\ \hline
UCS\newline{\normalfont [Only for versions starting with 4.7]} & Colombic scattering exponent & -- & 1.67 \\ \hline
UD & Coulomb scattering factor of mobility & m$^{-2}$ & 0 \\ \hline
UP & Channel length linear factor of mobility & m$^{-2}$ & 0 \\ \hline
VBM & Maximum body voltage & V & -3 \\ \hline
VDDEOT & Voltage for extraction of equivalent gate oxide thickness & V & 1.5 \\ \hline
VFB & Flat Band Voltage & V & -1 \\ \hline
VOFF & Threshold voltage offset & V & -0.08 \\ \hline
VOFFL & Length dependence parameter for Vth offset & V & 0 \\ \hline
VSAT & Saturation velocity at tnom & m/s & 80000 \\ \hline
VTH0 &  & V & 0 \\ \hline
VTL &  thermal velocity & m/s & 200000 \\ \hline
W0 & Narrow width effect parameter & m & 2.5e-06 \\ \hline
WEFFEOT\newline{\normalfont [Only for versions starting with 4.7]} & Effective width for extraction of EOT & m & 1e-05 \\ \hline
WINT & Width reduction parameter & m & 0 \\ \hline
XN &  back scattering parameter & -- & 3 \\ \hline

\category{Capacitance Parameters}\\ \hline
ACDE & Exponential coefficient for finite charge thickness & m/V & 1 \\ \hline
CF & Fringe capacitance parameter & F/m & 0 \\ \hline
CGBO & Gate-bulk overlap capacitance per length & -- & 0 \\ \hline
CGDL & New C-V model parameter & F/m & 0 \\ \hline
CGDO & Gate-drain overlap capacitance per width & F/m & 0 \\ \hline
CGSL & New C-V model parameter & F/m & 0 \\ \hline
CGSO & Gate-source overlap capacitance per width & F/m & 0 \\ \hline
CKAPPAD & D/G overlap C-V parameter & V & 0.6 \\ \hline
CKAPPAS & S/G overlap C-V parameter  & V & 0.6 \\ \hline
CLC & Vdsat parameter for C-V model & m & 1e-07 \\ \hline
CLE & Vdsat parameter for C-V model & -- & 0.6 \\ \hline
DLC & Delta L for C-V model & m & 0 \\ \hline
DWC & Delta W for C-V model & m & 0 \\ \hline
MINVCV & Fitting parameter for moderate inversion in Vgsteffcv & -- & 0 \\ \hline
MOIN & Coefficient for gate-bias dependent surface potential & -- & 15 \\ \hline
NOFF & C-V turn-on/off parameter & -- & 1 \\ \hline
VFBCV & Flat Band Voltage parameter for capmod=0 only & V & -1 \\ \hline
VOFFCV & C-V lateral-shift parameter & V & 0 \\ \hline
VOFFCVL & Length dependence parameter for Vth offset in CV & -- & 0 \\ \hline
XPART & Channel charge partitioning & F/m & 0 \\ \hline

\category{Control Parameters}\\ \hline
ACNQSMOD & AC NQS model selector & -- & 0 \\ \hline
BINUNIT & Bin  unit  selector & -- & 1 \\ \hline
CAPMOD & Capacitance model selector & -- & 2 \\ \hline
CVCHARGEMOD & Capacitance charge model selector & -- & 0 \\ \hline
DIOMOD & Diode IV model selector & -- & 1 \\ \hline
FNOIMOD & Flicker noise model selector & -- & 1 \\ \hline
GEOMOD & Geometry dependent parasitics model selector & -- & 0 \\ \hline
GIDLMOD\newline{\normalfont [Only for versions starting with 4.7]} & parameter for GIDL selector & -- & 0 \\ \hline
IGBMOD & Gate-to-body Ig model selector & -- & 0 \\ \hline
IGCMOD & Gate-to-channel Ig model selector & -- & 0 \\ \hline
MOBMOD & Mobility model selector & -- & 0 \\ \hline
MTRLCOMPATMOD\newline{\normalfont [Only for versions starting with 4.7]} & New material Mod backward compatibility selector & -- & 0 \\ \hline
MTRLMOD & parameter for nonm-silicon substrate or metal gate selector & -- & 0 \\ \hline
PARAMCHK & Model parameter checking selector & -- & 1 \\ \hline
PERMOD & Pd and Ps model selector & -- & 1 \\ \hline
RBODYMOD & Distributed body R model selector & -- & 0 \\ \hline
RDSMOD & Bias-dependent S/D resistance model selector & -- & 0 \\ \hline
RGATEMOD & Gate R model selector & -- & 0 \\ \hline
RGEOMOD & S/D resistance and contact model selector & -- & 0 \\ \hline
TEMPMOD & Temperature model selector & -- & 0 \\ \hline
TNOIMOD & Thermal noise model selector & -- & 0 \\ \hline
TRNQSMOD & Transient NQS model selector & -- & 0 \\ \hline
VERSION & parameter for model version & -- & '4.6.1' \\ \hline

\category{Flicker and Thermal Noise Parameters}\\ \hline
NOIA & Flicker Noise parameter a & -- & 0 \\ \hline
NOIB & Flicker Noise parameter b & -- & 0 \\ \hline
NOIC & Flicker Noise parameter c & -- & 0 \\ \hline

\category{Process Parameters}\\ \hline
DTOX & Defined as (toxe - toxp)  & m & 0 \\ \hline
EOT & Equivalent gate oxide thickness in meters & m & 1.5e-09 \\ \hline
EPSROX & Dielectric constant of the gate oxide relative to vacuum & -- & 3.9 \\ \hline
GAMMA1 & Vth body coefficient & V$^{1/2}$ & 0 \\ \hline
GAMMA2 & Vth body coefficient & V$^{1/2}$ & 0 \\ \hline
NDEP & Channel doping concentration at the depletion edge & cm$^{-3}$ & 1.7e+17 \\ \hline
NGATE & Poly-gate doping concentration & cm$^{-3}$ & 0 \\ \hline
NSD & S/D doping concentration & cm$^{-3}$ & 1e+20 \\ \hline
NSUB & Substrate doping concentration & cm$^{-3}$ & 6e+16 \\ \hline
RSH & Source-drain sheet resistance & $\mathsf{\Omega}/\Box$ & 0 \\ \hline
RSHG & Gate sheet resistance & $\mathsf{\Omega}/\Box$ & 0.1 \\ \hline
TOXE & Electrical gate oxide thickness in meters & m & 3e-09 \\ \hline
TOXM & Gate oxide thickness at which parameters are extracted & m & 3e-09 \\ \hline
TOXP & Physical gate oxide thickness in meters & m & 3e-09 \\ \hline
VBX & Vth transition body Voltage & V & 0 \\ \hline
XJ & Junction depth in meters & m & 1.5e-07 \\ \hline
XT & Doping depth & m & 1.55e-07 \\ \hline

\category{Tunnelling Parameters}\\ \hline
AIGBACC & Parameter for Igb & (Fs$^2$/g)$^{1/2}$/m & 0.0136 \\ \hline
AIGBINV & Parameter for Igb & (Fs$^2$/g)$^{1/2}$/m & 0.0111 \\ \hline
AIGC & Parameter for Igc & (Fs$^2$/g)$^{1/2}$/m & 0.0136 \\ \hline
AIGD & Parameter for Igd & (Fs$^2$/g)$^{1/2}$/m & 0.0136 \\ \hline
AIGS & Parameter for Igs & (Fs$^2$/g)$^{1/2}$/m & 0.0136 \\ \hline
BIGBACC & Parameter for Igb & (Fs$^2$/g)$^{1/2}$/mV & 0.00171 \\ \hline
BIGBINV & Parameter for Igb & (Fs$^2$/g)$^{1/2}$/mV & 0.000949 \\ \hline
BIGC & Parameter for Igc & (Fs$^2$/g)$^{1/2}$/mV & 0.00171 \\ \hline
BIGD & Parameter for Igd & (Fs$^2$/g)$^{1/2}$/mV & 0.00171 \\ \hline
BIGS & Parameter for Igs & (Fs$^2$/g)$^{1/2}$/mV & 0.00171 \\ \hline
CIGBACC & Parameter for Igb & V$^{-1}$ & 0.075 \\ \hline
CIGBINV & Parameter for Igb & V$^{-1}$ & 0.006 \\ \hline
CIGC & Parameter for Igc & V$^{-1}$ & 0.075 \\ \hline
CIGD & Parameter for Igd & V$^{-1}$ & 0.075 \\ \hline
CIGS & Parameter for Igs & V$^{-1}$ & 0.075 \\ \hline
DLCIGD & Delta L for Ig model drain side & m & 0 \\ \hline
EIGBINV & Parameter for the Si bandgap for Igbinv & V & 1.1 \\ \hline
NIGBACC & Parameter for Igbacc slope & -- & 1 \\ \hline
NIGBINV & Parameter for Igbinv slope & -- & 3 \\ \hline
NIGC & Parameter for Igc slope & -- & 1 \\ \hline
NTOX & Exponent for Tox ratio & -- & 1 \\ \hline
PIGCD & Parameter for Igc partition & -- & 1 \\ \hline
POXEDGE & Factor for the gate edge Tox & -- & 1 \\ \hline
TOXREF & Target tox value & m & 3e-09 \\ \hline
VFBSDOFF & S/D flatband voltage offset & V & 0 \\ \hline

\category{Asymmetric and Bias-Dependent $R_{ds}$ Parameters}\\ \hline
PRWB & Body-effect on parasitic resistance  & V$^{-1}$ & 0 \\ \hline
PRWG & Gate-bias effect on parasitic resistance  & V$^{-1}$ & 1 \\ \hline
RDSW & Source-drain resistance per width & $\mathsf{\Omega}$ $\mu$m & 200 \\ \hline
RDSWMIN & Source-drain resistance per width at high Vg & $\mathsf{\Omega}$ $\mu$m & 0 \\ \hline
RDW & Drain resistance per width & $\mathsf{\Omega}$ $\mu$m & 100 \\ \hline
RDWMIN & Drain resistance per width at high Vg & $\mathsf{\Omega}$ $\mu$m & 0 \\ \hline
RSW & Source resistance per width & $\mathsf{\Omega}$ $\mu$m & 100 \\ \hline
RSWMIN & Source resistance per width at high Vg & $\mathsf{\Omega}$ $\mu$m & 0 \\ \hline
WR & Width dependence of rds & -- & 1 \\ \hline

\category{Impact Ionization Current Parameters}\\ \hline
ALPHA0 & substrate current model parameter & m/V & 0 \\ \hline
ALPHA1 & substrate current model parameter & V$^{-1}$ & 0 \\ \hline
BETA0 & substrate current model parameter & V$^{-1}$ & 0 \\ \hline

\category{Gate-induced Drain Leakage Model Parameters}\\ \hline
AGIDL & Pre-exponential constant for GIDL & $\mathsf{\Omega}^{-1}$ & 0 \\ \hline
AGISL & Pre-exponential constant for GISL & $\mathsf{\Omega}^{-1}$ & 0 \\ \hline
BGIDL & Exponential constant for GIDL & Vm$^{-1}$ & 2.3e+09 \\ \hline
BGISL & Exponential constant for GISL & Vm$^{-1}$ & 2.3e-09 \\ \hline
CGIDL & Parameter for body-bias dependence of GIDL & V$^3$ & 0.5 \\ \hline
CGISL & Parameter for body-bias dependence of GISL & V$^3$ & 0.5 \\ \hline
EGIDL & Fitting parameter for Bandbending & V & 0.8 \\ \hline
EGISL & Fitting parameter for Bandbending & V & 0.8 \\ \hline
FGIDL\newline{\normalfont [Only for versions starting with 4.7]} & GIDL vb parameter & V & 0 \\ \hline
FGISL\newline{\normalfont [Only for versions starting with 4.7]} & Parameter for GISL body bias dependence & V & 0 \\ \hline
KGIDL\newline{\normalfont [Only for versions starting with 4.7]} & GIDL vb parameter & V & 0 \\ \hline
KGISL\newline{\normalfont [Only for versions starting with 4.7]} & Parameter for GISL body bias dependence & V & 0 \\ \hline
RGIDL\newline{\normalfont [Only for versions starting with 4.7]} & GIDL vg parameter & -- & 1 \\ \hline
RGISL\newline{\normalfont [Only for versions starting with 4.7]} & Parameter for GISL gate bias dependence & -- & 1 \\ \hline
\end{DeviceParamTableGenerated}


\clearpage
\subsubsection{Level 18 MOSFET Tables (VDMOS)}
The vertical double-diffused power MOSFET model is based on the uniform charge
control model (UCCM) developed at Rensselaer Polytechnic Institute~\cite{Fjeldly:1998}.
The VDMOS current-voltage characteristics are described by a single, continuous
analytical expression for all regimes of operation.  The physics-based model
includes effects such as velocity saturation in the channel, drain induced barrier
lowering, finite output conductance in saturation, the quasi-saturation effect
through a bias dependent drain parasitic resistance, effects of bulk charge, and
bias dependent low-field mobility.  An important feature of the implementation
is the utilization of a single continuous expression for the drain current, which
is valid below and above threshold, effectively removing discontinuities and
improving convergence properties.

The following tables give parameters for the level 18 MOSFET.

% This table was generated by Xyce:
%   Xyce -doc_cat M 18
%
\index{power mosfet!device instance parameters}
\begin{DeviceParamTableGenerated}{Power MOSFET Device Instance Parameters}{M_18_Device_Instance_Params}

\category{Control Parameters}\\ \hline
M & Multiplier for M devices connected in parallel & -- & 1 \\ \hline

\category{Geometry Parameters}\\ \hline
AD & Drain diffusion area & m$^{2}$ & 0 \\ \hline
AS & Source diffusion area & m$^{2}$ & 0 \\ \hline
L & Channel length & m & 0 \\ \hline
NRD & Multiplier for RSH to yield parasitic resistance of drain & $\Box$ & 1 \\ \hline
NRS & Multiplier for RSH to yield parasitic resistance of source & $\Box$ & 1 \\ \hline
PD & Drain diffusion perimeter & m & 0 \\ \hline
PS & Source diffusion perimeter & m & 0 \\ \hline
W & Channel width & m & 0 \\ \hline

\category{Temperature Parameters}\\ \hline
DTEMP & Device delta temperature & $^\circ$C & 0 \\ \hline
TEMP & Device temperature & $^\circ$C & Ambient Temperature \\ \hline
\end{DeviceParamTableGenerated}

\input{M_18_Device_Model_Params}

\clearpage
\subsubsection{Levels 70 and 70450 MOSFET Tables (BSIM-SOI 4.6.1 and 4.5.0)}
For complete documentation of the BSIM-SOI model, see the users'
manual for the BSIM-SOI, available for download at
\url{http://bsim.berkeley.edu/models/bsimsoi/}.  \Xyce{} implements
Version 4.6.1 of the BSIM-SOI as the level 70 device and version 4.5.0
as level 70450.

Instance and model parameters of the level 70 MOSFET are given in
tables~\ref{M_70_Device_Instance_Params} and
\ref{M_70_Device_Model_Params}.

Beginning with \Xyce{} 7.2, the BSIM-SOI models level 70 and 70450
have {\em limited} support for the optional 5th, 6th, and 7th nodes.
See the BSIM-SOI technical manual at the BSIM web site for details of
what configurations the full device supports.  Only some of these use
cases are supported: Use of the BSIM-SOI 4.x with \texttt{TNODEOUT=0}
(the default) is supported in 4-, 5-, 6-, and 7-node configurations.
\texttt{TNODEOUT=1} is supported only in the 7-node configuration,
with the 7th node being temperature.  No access to the external
temperature node is available in 5- or 6- node configuration.

When \texttt{TNODEOUT=0}, the temperature node is an internal node of
the device even when not specified on the instance line, and its value
may still be printed using the N() notation (see
section~\ref{Print_Device_Info}).  This somewhat minimizes the impact
of the lack of support for \texttt{TNODEOUT=1} in \Xyce{} --- the
temperature rise due to self-heating is always available for printing,
but it is not available for creation of a thermal coupling network
except in the 7-node configuration.

Note that with some choices of model parameters, the BSIM-SOI devices
attempt to ``collapse'' the ``P'' and ``B'' nodes (external and
internal body nodes, 5th and 6th netlist nodes if given, internal
nodes if not given).  Xyce is unable to perform such collapse when the
nodes are externally specified, and will issue warnings when it finds
the model trying to do so.  Depending on the actual nodes used for P
and B, the device may fail to converge or produce invalid results; as
an example, if P and B are actually specified on the netlist line to
be the same node, this failure to collapse will not matter --- the
nodes are already the same.  But if two different node names are used
for the 5th and 6th nodes, the failure to collapse will leave one node
floating and the simulation will likely fail if the printed warnings
are ignored.

A similar problem exists for other choices of model parameter: in some
cases neither the ``P'' nor ``B'' nodes are used, and if the nodes are
specified on the netlist line the BSIM-SOI code attempts to collapse
them to ground.  This is not something \Xyce{} can do, and therefore
instead \Xyce{} ignores the specified nodes.  This can leave those
nodes floating and lead to convergence failures unless the specified
nodes are already the ground node (node 0).  \Xyce{} will issue
appropriate warnings when this condition exists and suggest removal of
the unused external nodes from the instance line.

The BSIM SOI 4.6.1 device supports output of the internal variables in
table~\ref{M_70_OutputVars} on the \texttt{.PRINT} line of a netlist.
To access them from a print line, use the syntax
\texttt{N(<instance>:<variable>)} where ``\texttt{<instance>}'' refers to the
name of the specific level 70 M device in your netlist.

\textbf{NOTE:} It has been observed that the gate capacitance model of
BSIM-SOI 4.6.1 behaves differently than earlier versions, and the team
has seen significant disagreement of gate currents when comparing
identical simulations with other simulators that have only earlier
BSIM-SOI models.  For this reason, we are also providing BSIM-SOI
4.5.0 as the level 70450 MOSFET.  This model does agree with these
other simulators.  The parameters and output variables are given in
tables~\ref{M_70450_Device_Instance_Params},
\ref{M_70450_Device_Model_Params}, and \ref{M_70450_OutputVars}.
Unlike BSIM-SOI 4.6.1, the 4.5.0 model's original Verilog-A source
code does not contain descriptions and units for the parameters, and
these appear blank in the tables.  For descriptions and units, see the
corresponding parameters in the level 70 tables.

\input{M_70_Device_Instance_Params}
\input{M_70_Device_Model_Params}
\input{M_70_OutputVars}
\input{M_70450_Device_Instance_Params}
\input{M_70450_Device_Model_Params}
\input{M_70450_OutputVars}

\clearpage
\subsubsection{Level 77 MOSFET Tables (BSIM6 version 6.1.1)}
\Xyce{} includes the BSIM6 MOSFET model, version 6.1.1.  Full
documentation of the BSIM6 is available at its web site,
\url{http://bsim.berkeley.edu/models/bsim6/}.  Instance and model
parameters for the BSIM6 are given in
tables~\ref{M_77_Device_Instance_Params} and
\ref{M_77_Device_Model_Params}.  These tables are generated directly
from information present in the original Verilog-A implementation of
the BSIM6, and lack many descriptions for the parameters.  Consult the
BSIM6 technical manual from the BSIM group for further details about
these parameters.

Beginning with version 7.2 of \Xyce{}, an optional fifth node may be
specified for BSIM6 devices.  If specified, it is the temperature
node, which is used by the self-heating model and is internal if not
specified on the instance line.

The BSIM6 device supports output of the internal variables in
table~\ref{M_77_OutputVars} on the \texttt{.PRINT} line of a netlist.
To access them from a print line, use the syntax
\texttt{N(<instance>:<variable>)} where ``\texttt{<instance>}'' refers to the
name of the specific level 77 M device in your netlist.

\input{M_77_Device_Instance_Params}
\input{M_77_Device_Model_Params}
\input{M_77_OutputVars}

\clearpage
\subsubsection{Level 102 MOSFET Tables (PSP version 102.5)}

\Xyce{} includes a legacy version of the PSP MOSFET model, version
102.5.  This version is provided because the more recent 103 versions
are not backward compatible with the older 102 versions, and some
foundries provide model cards that use the version 102.  Development
of new model cards should be done using the more recent, supported
versions of PSP.

The PSP102 device supports output of the internal variables in
table~\ref{M_102_OutputVars} on the \texttt{.PRINT} line of a netlist.
To access them from a print line, use the syntax
\texttt{N(<instance>:<variable>)} where ``\texttt{<instance>}'' refers to the
name of the specific PSP102 M device in your netlist.

\input{M_102_Device_Instance_Params}
\input{M_102_Device_Model_Params}
\input{M_102_OutputVars}

\subsubsection{Level 103 and 1031 MOSFET Tables (PSP version 103.4)}

\Xyce{} includes the PSP MOSFET model, version 103.4~\cite{PSP:2006}.
The version without self-heating is the level 103 MOSFET, and the
version with self-heating is the level 1031.  Note that the level 1031
MOSFET requires five nodes on its instance line: drain, gate, source,
bulk, and dt.  The fifth node will be the temperature rise of the
device due to self-heating.

Full documentation for the PSP model is available on its web site,
\url{http://www.cea.fr/cea-tech/leti/pspsupport}.  Instance and model
parameters for the PSP model are given in
tables~\ref{M_103_Device_Instance_Params}, \ref{M_103_Device_Model_Params},
\ref{M_1031_Device_Instance_Params}, and \ref{M_1031_Device_Model_Params}.

The PSP103 devices support output of the internal variables in
table~\ref{M_103_OutputVars} and table~\ref{M_1031_OutputVars} on the \texttt{.PRINT} line of a netlist.
To access them from a print line, use the syntax
\texttt{N(<instance>:<variable>)} where ``\texttt{<instance>}'' refers to the
name of the specific PSP103 M device in your netlist.

% This table was generated by Xyce:
%   Xyce -doc M 103
%
\index{psp103va mosfet!device instance parameters}
\begin{DeviceParamTableGenerated}{PSP103VA MOSFET Device Instance Parameters}{M_103_Device_Instance_Params}
ABDRAIN & Bottom area of drain junction & m$^{2}$ & 1e-12 \\ \hline
ABSOURCE & Bottom area of source junction & m$^{2}$ & 1e-12 \\ \hline
AD & Bottom area of drain junction & m$^{2}$ & 1e-12 \\ \hline
AS & Bottom area of source junction & m$^{2}$ & 1e-12 \\ \hline
DELVTO & Threshold voltage shift parameter & V & 0 \\ \hline
DELVTOEDGE & Threshold voltage shift parameter of edge transistor & V & 0 \\ \hline
DTA & Temperature offset w.r.t. ambient temperature & K & 0 \\ \hline
FACTUO & Zero-field mobility pre-factor & --- & 1 \\ \hline
FACTUOEDGE & Zero-field mobility pre-factor of edge transistor & --- & 1 \\ \hline
JW & Gate-edge length of source/drain junction & m & 1e-06 \\ \hline
L & Design length & m & 1e-05 \\ \hline
LGDRAIN & Gate-edge length of drain junction & m & 1e-06 \\ \hline
LGSOURCE & Gate-edge length of source junction & m & 1e-06 \\ \hline
LSDRAIN & STI-edge length of drain junction & m & 1e-06 \\ \hline
LSSOURCE & STI-edge length of source junction & m & 1e-06 \\ \hline
M &  Alias for MULT & --- & 1 \\ \hline
MULT & Number of devices in parallel & --- & 1 \\ \hline
NF & Number of fingers & --- & 1 \\ \hline
NGCON & Number of gate contacts & --- & 1 \\ \hline
NRD & Number of squares of drain diffusion & --- & 0 \\ \hline
NRS & Number of squares of source diffusion & --- & 0 \\ \hline
PD & Perimeter of drain junction & m & 1e-06 \\ \hline
PS & Perimeter of source junction & m & 1e-06 \\ \hline
SA & Distance between OD-edge and poly from one side & m & 0 \\ \hline
SB & Distance between OD-edge and poly from other side & m & 0 \\ \hline
SC & Distance between OD-edge and nearest well edge & m & 0 \\ \hline
SCA & Integral of the first distribution function for scattered well dopants & --- & 0 \\ \hline
SCB & Integral of the second distribution function for scattered well dopants & --- & 0 \\ \hline
SCC & Integral of the third distribution function for scattered well dopants & --- & 0 \\ \hline
SD & Distance between neighbouring fingers & m & 0 \\ \hline
W & Design width & m & 1e-05 \\ \hline
XGW & Distance from the gate contact to the channel edge & m & 1e-07 \\ \hline
\end{DeviceParamTableGenerated}

\input{M_103_Device_Model_Params}
\input{M_103_OutputVars}
% This table was generated by Xyce:
%   Xyce -doc M 1031
%
\index{psp103va mosfet with self-heating!device instance parameters}
\begin{DeviceParamTableGenerated}{PSP103VA MOSFET with self-heating Device Instance Parameters}{M_1031_Device_Instance_Params}
ABDRAIN & Bottom area of drain junction & m$^{2}$ & 1e-12 \\ \hline
ABSOURCE & Bottom area of source junction & m$^{2}$ & 1e-12 \\ \hline
AD & Bottom area of drain junction & m$^{2}$ & 1e-12 \\ \hline
AS & Bottom area of source junction & m$^{2}$ & 1e-12 \\ \hline
DELVTO & Threshold voltage shift parameter & V & 0 \\ \hline
DELVTOEDGE & Threshold voltage shift parameter of edge transistor & V & 0 \\ \hline
DTA & Temperature offset w.r.t. ambient temperature & K & 0 \\ \hline
FACTUO & Zero-field mobility pre-factor & --- & 1 \\ \hline
FACTUOEDGE & Zero-field mobility pre-factor of edge transistor & --- & 1 \\ \hline
JW & Gate-edge length of source/drain junction & m & 1e-06 \\ \hline
L & Design length & m & 1e-05 \\ \hline
LGDRAIN & Gate-edge length of drain junction & m & 1e-06 \\ \hline
LGSOURCE & Gate-edge length of source junction & m & 1e-06 \\ \hline
LSDRAIN & STI-edge length of drain junction & m & 1e-06 \\ \hline
LSSOURCE & STI-edge length of source junction & m & 1e-06 \\ \hline
M &  Alias for MULT & --- & 1 \\ \hline
MULT & Number of devices in parallel & --- & 1 \\ \hline
NF & Number of fingers & --- & 1 \\ \hline
NGCON & Number of gate contacts & --- & 1 \\ \hline
NRD & Number of squares of drain diffusion & --- & 0 \\ \hline
NRS & Number of squares of source diffusion & --- & 0 \\ \hline
PD & Perimeter of drain junction & m & 1e-06 \\ \hline
PS & Perimeter of source junction & m & 1e-06 \\ \hline
SA & Distance between OD-edge and poly from one side & m & 0 \\ \hline
SB & Distance between OD-edge and poly from other side & m & 0 \\ \hline
SC & Distance between OD-edge and nearest well edge & m & 0 \\ \hline
SCA & Integral of the first distribution function for scattered well dopants & --- & 0 \\ \hline
SCB & Integral of the second distribution function for scattered well dopants & --- & 0 \\ \hline
SCC & Integral of the third distribution function for scattered well dopants & --- & 0 \\ \hline
SD & Distance between neighbouring fingers & m & 0 \\ \hline
W & Design width & m & 1e-05 \\ \hline
XGW & Distance from the gate contact to the channel edge & m & 1e-07 \\ \hline
\end{DeviceParamTableGenerated}

\input{M_1031_Device_Model_Params}
\input{M_1031_OutputVars}

\clearpage
\subsubsection{Level 110 MOSFET Tables (BSIM CMG version 110.0.0)}
\Xyce{} includes the BSIM CMG Common Multi-gate model version 110.
The code in \Xyce{} was generated from the BSIM group's Verilog-A
input using the default ``ifdef'' lines provided, and therefore
supports only the subset of BSIM CMG features those defaults enable.
Instance and model parameters for the BSIM CMG model are given in
tables~\ref{M_110_Device_Instance_Params} and
\ref{M_110_Device_Model_Params}.  Details of the model are documented
in the BSIM-CMG technical report\cite{BSIMCMG:Manual}, available from
the BSIM web site at
\url{http://bsim.berkeley.edu/models/bsimcmg/}.

The BSIM CMG devices support output of the internal variables in
tables~\ref{M_107_OutputVars}, \ref{M_108_OutputVars}, and  \ref{M_110_OutputVars} on the \texttt{.PRINT} line of a netlist.
To access them from a print line, use the syntax
\texttt{N(<instance>:<variable>)} where ``\texttt{<instance>}'' refers to the
name of the specific level 107 or 108 M device in your netlist.

\input{M_110_Device_Instance_Params}
\input{M_110_Device_Model_Params}
\input{M_110_OutputVars}

\subsubsection{Level 107  and 108 MOSFET Tables (BSIM CMG versions 107.0.0 and 108.0.0)}
\Xyce{} includes the legacy BSIM CMG Common Multi-gate model versions 107 and 108.
These models have been superceded by the level 110 version, but has been
retained for backward compatibility with previous versions of Xyce and
older model cards and PDKs.  The code in \Xyce{} was generated from the BSIM
group's Verilog-A input using the default ``ifdef'' lines provided,
and therefore supports only the subset of BSIM CMG features those
defaults enable.  Instance and model parameters for the BSIM CMG model
are given in tables~\ref{M_107_Device_Instance_Params},
\ref{M_107_Device_Model_Params}, \ref{M_108_Device_Instance_Params},
and~\ref{M_108_Device_Model_Params}.  Details of the model are documented
in the BSIM-CMG technical report\cite{BSIMCMG:Manual}, available from
the BSIM web site at \url{http://bsim.berkeley.edu/models/bsimcmg/}.

Note that the TNOIMOD=1 option of BSIM-CMG 108 is not supported in
Xyce, as it uses features of Verilog-A that are not supported in our
Verilog-A compiler.  This noise model was added in version 108 and
removed in version 109.  The TNOIMOD=2 option of BSIM-CMG 108 is the
same as the TNOIMOD=1 option of BSIM-CMG 110.

\input{M_107_Device_Instance_Params}
\input{M_107_Device_Model_Params}
\input{M_107_OutputVars}

\input{M_108_Device_Instance_Params}
\input{M_108_Device_Model_Params}
\input{M_108_OutputVars}


\clearpage
\subsubsection{Levels 2000 and 2001 MOSFET Tables (MVS version 2.0.0)}
\Xyce{} includes the MIT Virtual Source (MVS) MOSFET model version
2.0.0 in both ETSOI and HEMT variants.  The code in \Xyce{} was
generated from the MIT Verilog-A input.  Model parameters for the MVS
model are given in \ref{M_2000_Device_Model_Params} and
\ref{M_2001_Device_Model_Params}.  The MVS model does not have
instance parameters.  Details of the model are documented MVS
Nanotransistor Model 2.0.0 manual, available from the NEEDS web site
at \url{https://nanohub.org/publications/74/1}.

{\bf NOTE: } Unlike all other MOSFET models in Xyce, the MVS model
takes only 3 nodes, the drain, gate and source.  It takes no substrate
node.

\input{M_2000_Device_Model_Params}
\input{M_2001_Device_Model_Params}

\clearpage

\subsubsection{Level 2002 MOSFET Tables (MVSG\_CMC version 1.1.0)}
\Xyce{} includes the MIT Virtual Source GaN HEMT High-Voltage
(MVSG\_CMC) MOSFET model version 1.1.0.  The code in \Xyce{} was
generated from the MIT Verilog-A input.  Model parameters for the MVS
model are given in \ref{M_2002_Device_Instance_Params} and
\ref{M_2002_Device_Model_Params}, and its output variables in
\ref{M_2002_OutputVars}.  More information about this model may be
obtained from the CMC standard models page at
\url{https://si2.org/standard-models}.

\input{M_2002_Device_Instance_Params}
\input{M_2002_Device_Model_Params}
\input{M_2002_OutputVars}


\clearpage
\subsubsection{Level 260 MOSFET Tables (EKV version 2.6)}

\Xyce{} includes the EKV MOSFET model, version 2.6 as the level 260
MOSFET device.

Official documentation of this model may be found at \url{https://www.epfl.ch/labs/iclab/wp-content/uploads/2019/02/ekv_v262.pdf}.

We have implemented EKV 2.6 directly from the Verilog-A source
published by its authors at \url{https://github.com/ekv26/model}.
While it is a faithful implementation of the model provided there, we
have had anecdotal evidence that other simulators have different
implementations that contain additional parameters and possibly a
different extrinsic model.  Model cards containing parameters
extracted from other simulators may not result in \Xyce{} simulations
that match those other simulators.  Watch carefully for any warnings
from \Xyce{} regarding unrecognized model parameters, as these are a
strong indication that the model card is not extracted using the exact
version of EKV provided by \Xyce{}.

Tables of EKV MOSFET 2.6 parameters are in
tables~\ref{M_260_Device_Instance_Params} and
\ref{M_260_Device_Model_Params}.

Note that in the tables the device claims that the default
\texttt{TNOM} and \texttt{TEMP} parameter values is 1e21.  This is
merely an artifact of an unusual way the authors have defined those
parameters in the Verilog-A source.  In fact, if not given
\texttt{TNOM} defaults to 25 $^\circ$ C, and if not given \texttt{TEMP}
defaults to the ambient temperature of the simulation.

% This table was generated by Xyce:
%   Xyce -doc M 260
%
\index{ekv mosfet version 2.6!device instance parameters}
\begin{DeviceParamTableGenerated}{EKV MOSFET version 2.6 Device Instance Parameters}{M_260_Device_Instance_Params}
AD &  & -- & 0 \\ \hline
AS &  & -- & 0 \\ \hline
L &  & -- & 1e-05 \\ \hline
M &  & -- & 1 \\ \hline
NS &  & -- & 1 \\ \hline
PD &  & -- & 0 \\ \hline
PS &  & -- & 0 \\ \hline
W &  & -- & 1e-05 \\ \hline
\end{DeviceParamTableGenerated}

% This table was generated by Xyce:
%   Xyce -doc M 260
%
\index{ekv mosfet version 2.6!device model parameters}
\begin{DeviceParamTableGenerated}{EKV MOSFET version 2.6 Device Model Parameters}{M_260_Device_Model_Params}
AD &  & -- & 0 \\ \hline
AF &  & -- & 1 \\ \hline
AGAMMA &  & -- & 1e-06 \\ \hline
AKP &  & -- & 1e-06 \\ \hline
AS &  & -- & 0 \\ \hline
AVTO &  & -- & 1e-06 \\ \hline
BEX &  & -- & -1.5 \\ \hline
COX &  & -- & 0.002 \\ \hline
DL &  & -- & -1e-08 \\ \hline
DW &  & -- & -1e-08 \\ \hline
E0 &  & -- & 1e+08 \\ \hline
GAMMA &  & -- & 0.7 \\ \hline
HDIF &  & -- & 5e-07 \\ \hline
IBA &  & -- & 5e+08 \\ \hline
IBB &  & -- & 4e+08 \\ \hline
IBBT &  & -- & 0.0009 \\ \hline
IBN &  & -- & 1 \\ \hline
KF &  & -- & 0 \\ \hline
KP &  & -- & 0.00015 \\ \hline
L &  & -- & 1e-05 \\ \hline
LAMBDA &  & -- & 0.8 \\ \hline
LETA &  & -- & 0.3 \\ \hline
LK &  & -- & 4e-07 \\ \hline
LMAX &  & -- & 100 \\ \hline
LMIN &  & -- & 0 \\ \hline
M &  & -- & 1 \\ \hline
NOISE &  & -- & 1 \\ \hline
NS &  & -- & 1 \\ \hline
PD &  & -- & 0 \\ \hline
PHI &  & -- & 0.5 \\ \hline
PS &  & -- & 0 \\ \hline
Q0 &  & -- & 0.00023 \\ \hline
RSH &  & -- & 0 \\ \hline
TCV &  & -- & 0.001 \\ \hline
TEMP &  & -- & 1e+21 \\ \hline
THETA &  & -- & 0 \\ \hline
TNOM &  & -- & 1e+21 \\ \hline
TP\_CJ &  & -- & 0 \\ \hline
TP\_CJSW &  & -- & 0 \\ \hline
TP\_CJSWG &  & -- & 0 \\ \hline
TP\_NJTS &  & -- & 0 \\ \hline
TP\_NJTSSW &  & -- & 0 \\ \hline
TP\_NJTSSWG &  & -- & 0 \\ \hline
TP\_PB &  & -- & 0 \\ \hline
TP\_PBSW &  & -- & 0 \\ \hline
TP\_PBSWG &  & -- & 0 \\ \hline
TP\_XTI &  & -- & 3 \\ \hline
TRISE &  & -- & 0 \\ \hline
TYPE &  & -- & 1 \\ \hline
UCEX &  & -- & 0.8 \\ \hline
UCRIT &  & -- & 2e+06 \\ \hline
VTO &  & -- & 0.5 \\ \hline
W &  & -- & 1e-05 \\ \hline
WETA &  & -- & 0.2 \\ \hline
WMAX &  & -- & 100 \\ \hline
WMIN &  & -- & 0 \\ \hline
XD\_BV &  & -- & 10 \\ \hline
XD\_CJ &  & -- & 1e-09 \\ \hline
XD\_CJSW &  & -- & 1e-12 \\ \hline
XD\_CJSWG &  & -- & 1e-12 \\ \hline
XD\_GMIN &  & -- & 0 \\ \hline
XD\_JS &  & -- & 1e-09 \\ \hline
XD\_JSW &  & -- & 1e-12 \\ \hline
XD\_JSWG &  & -- & 1e-12 \\ \hline
XD\_MJ &  & -- & 0.9 \\ \hline
XD\_MJSW &  & -- & 0.7 \\ \hline
XD\_MJSWG &  & -- & 0.7 \\ \hline
XD\_N &  & -- & 1 \\ \hline
XD\_NJTS &  & -- & 1 \\ \hline
XD\_NJTSSW &  & -- & 1 \\ \hline
XD\_NJTSSWG &  & -- & 1 \\ \hline
XD\_PB &  & -- & 0.8 \\ \hline
XD\_PBSW &  & -- & 0.6 \\ \hline
XD\_PBSWG &  & -- & 0.6 \\ \hline
XD\_VTS &  & -- & 0 \\ \hline
XD\_VTSSW &  & -- & 0 \\ \hline
XD\_VTSSWG &  & -- & 0 \\ \hline
XD\_XJBV &  & -- & 0 \\ \hline
XJ &  & -- & 3e-07 \\ \hline
\end{DeviceParamTableGenerated}


\clearpage
\subsubsection{Level 301 MOSFET Tables (EKV version 3.0.1)}
\Xyce{} includes the EKV MOSFET model, version
3.0.1~\cite{BLETK:1997}\cite{EKV:2006}\cite{EKV:2007}.  Full
documentation for the EKV3 model is available on the \Xyce{} internal web site;
the documentation for the EKV3 model may be freely redistributed.  Instance and
model parameters for the EKV model are given in
tables~\ref{M_301_Device_Instance_Params} and \ref{M_301_Device_Model_Params}.

The EKV3 model is developed by the EKV Team of the Electronics Laboratory-TUC
(Technical University of Crete). It is included in \Xyce{} under license from
Technical University of Crete.  The official web site of the EKV model is
\url{http://ekv.epfl.ch/}.

\textbf{Due to licensing restrictions, the EKV3 MOSFET is not available in
     open-source versions of \Xyce{}.  The license for EKV3 authorizes Sandia
     National Laboratories to distribute EKV3 only in binary versions of code.}


\input{M_301_Device_Instance_Params}
\input{M_301_Device_Model_Params}

\clearpage
\subsubsection{Level 10240 MOSFET Tables (L\_UTSOI Version 102.4.0)}
Select \Xyce{} binaries include the L\_UTSOI MOSFET model as the level
10240 MOSFET.  This model's parameters and output variables are listed in tables~\ref{M_10240_Device_Instance_Params}, \ref{M_10240_Device_Model_Params}, and \ref{M_10240_OutputVars}

\input{M_10240_Device_Instance_Params}
\input{M_10240_Device_Model_Params}
%table generated from Verilog-A input
\index{MOSFET level 10240!device output variables}
\begin{DeviceParamTableGenerated}{MOSFET level 10240 Output Variables}{M_10240_OutputVars}
type & Flag for channel type &    & none \\ \hline
vds & Internal drain-source DC voltage (NMOS convention) &   V & none \\ \hline
vsb & Internal source-bulk DC voltage (NMOS convention) &   V & none \\ \hline
vgs & Internal gate-source DC voltage (NMOS convention) &   V & none \\ \hline
vth & Threshold voltage &   V & none \\ \hline
vth\_drive & Effective gate drive voltage, including back bias, drain bias effects and self-heating &   V & none \\ \hline
vdsat & Drain saturation voltage at the given bias &   V & none \\ \hline
vdsat\_marg & Vds voltage margin &   V & none \\ \hline
id & Total DC drain current flowing into drain terminal &   A & none \\ \hline
ig & Total DC gate current flowing into gate terminal &   A & none \\ \hline
is & Total DC source current flowing into source terminal &   A & none \\ \hline
ib & Total DC bulk current flowing into bulk terminal &   A & none \\ \hline
ids & DC channel current, excluding tunnel, GISL and GIDL currents &   A & none \\ \hline
igidl & DC Gate Induced Drain Leakage current &   A & none \\ \hline
igisl & DC Gate Induced Source Leakage current &   A & none \\ \hline
igs & DC gate-source leakage current &   A & none \\ \hline
igd & DC gate-drain leakage current &   A & none \\ \hline
idb & DC drain-bulk current &   A & none \\ \hline
isb & DC source-bulk current &   A & none \\ \hline
gm & Internal DC transconductance &   A/V & none \\ \hline
gmb & Internal DC bulk transconductance &   A/V & none \\ \hline
gds & Internal DC output conductance &   A/V & none \\ \hline
cgg & Internal AC gate capacitance, including overlap capacitances &   F & none \\ \hline
cgd & Internal AC gate-drain transcapacitance, including overlap capacitances &   F & none \\ \hline
cgs & Internal AC gate-source transcapacitance, including overlap capacitances &   F & none \\ \hline
cgb & Internal AC gate-bulk transcapacitance &   F & none \\ \hline
cdd & Internal AC drain capacitance &   F & none \\ \hline
cdg & Internal AC drain-gate transcapacitance &   F & none \\ \hline
cds & Internal AC drain-source transcapacitance &   F & none \\ \hline
cdb & Internal AC drain-bulk transcapacitance &   F & none \\ \hline
cbb & Internal AC bulk capacitance &   F & none \\ \hline
cbg & Internal AC bulk-gate transcapacitance &   F & none \\ \hline
cbs & Internal AC bulk-source transcapacitance &   F & none \\ \hline
cbd & Internal AC bulk-drain transcapacitance &   F & none \\ \hline
css & Internal AC source capacitance &   F & none \\ \hline
csg & Internal AC source-gate transcapacitance &   F & none \\ \hline
csb & Internal AC source-bulk transcapacitance &   F & none \\ \hline
csd & Internal AC source-drain transcapacitance &   F & none \\ \hline
tk & MOSFET device temperature &   K & none \\ \hline
dtsh & MOSFET device temperature increase due to self-heating &   K & none \\ \hline
self\_gain & Internal L-UTSOI model self gain &    & none \\ \hline
rout & AC output resistance &   Ohm & none \\ \hline
beff & Gain factor in saturation &   A/V$^{2}$ & none \\ \hline
ft & Unity gain frequency at the given bias &   Hz & none \\ \hline
rgate & MOS gate resistance (intrinsic input resistance) &   Ohm & none \\ \hline
gmoverid & Gm over Id &   1/V & none \\ \hline
vearly & Equivalent Early voltage &   V & none \\ \hline
\end{DeviceParamTableGenerated}


