% Sandia National Laboratories is a multimission laboratory managed and
% operated by National Technology & Engineering Solutions of Sandia, LLC, a
% wholly owned subsidiary of Honeywell International Inc., for the U.S.
% Department of Energy’s National Nuclear Security Administration under
% contract DE-NA0003525.

% Copyright 2002-2024 National Technology & Engineering Solutions of Sandia,
% LLC (NTESS).


\begin{Device}\label{M_DEVICE}

\symbol
{\includegraphics{nmosSymbol}}
{\includegraphics{pmosSymbol}}

\device
\begin{alltt}
M<name> <drain node> <gate node> <source node>
+ <bulk/substrate node> <model name>
+ [L=<value>] [W=<value>]
+ [AD=<value>] [AS=<value>]
+ [PD=<value>] [PS=<value>]
+ [NRD=<value>] [NRS=<value>]
+ [M=<value] [IC=<value, ...>]
\end{alltt}

\vbox{\hrulefill}
\item[Special Form (BSIMSOI)]
\begin{alltt}
M<name> <drain node> <gate node> <source node>
+ <substrate node (E)>
+ [<External body contact (P)>]
+ [<internal body contact (B)>]
+ [<temperature node (T)>]
+ <model name>
+ [L=<value>] [W=<value>]
+ [AD=<value>] [AS=<value>]
+ [PD=<value>] [PS=<value>]
+ [NRD=<value>] [NRS=<value>] [NRB=<value>]
+ [BJTOFF=<value>]
+ [IC=<val>,<val>,<val>,<val>,<val>]
+ [RTH0=<val>] [CTH0=<val>]
+ [NBC=<val>] [NSEG=<val>] [PDBCP=<val>] [PSBCP=<val>]
+ [AGBCP=<val>] [AEBCP=<val>] [VBSUSR=<val>] [TNODEOUT]
+ [FRBODY=<val>] [M=<value>]
\end{alltt}
\vbox{\hrulefill}

\item[Special Form (MVS)]
\begin{alltt}
M<name> <drain node> <gate node> <source node> <model name>
\end{alltt}

\item[Special Form (PSP103 with self-heating)]
\begin{alltt}
M<name> <drain node> <gate node> <source node> <bulk node> <dt node> <model name> [instance parameters]
\end{alltt}

\model
\begin{alltt}
.MODEL <model name> NMOS [model parameters]
.MODEL <model name> PMOS [model parameters]
\end{alltt}

\examples
\begin{alltt}
M5 4 12 3 0 PNOM L=20u W=10u
M3 5 13 10 0 PSTRONG
M6 7 13 10 0 PSTRONG M=2
M8 10 12 100 100 NWEAK L=30u W=20u
+ AD=288p AS=288p PD=60u PS=60u NRD=14 NRS=24
\end{alltt}

\parameters

\begin{Parameters}

\param{\vbox{\hbox{L\hfil}\hbox{M\hfil}}}

The MOSFET channel length and width that are decreased to get the actual
channel length and width. They may be given in the device
\texttt{.MODEL} or \texttt{.OPTIONS} statements. The value in the device
statement overrides the value in the model statement, which overrides
the value in the \texttt{.OPTIONS} statement. If \texttt{L} or \texttt{W}
values are not given, their default value is 100~$\mu$m.

\param{\vbox{\hbox{AD\hfil}\hbox{AS\hfil}}}

The drain and source diffusion areas. Defaults for \texttt{AD} and
\texttt{AS} can be set in the \texttt{.OPTIONS} statement.  If
\texttt{AD} or \texttt{AS} defaults are not set, their default value is
0.

\param{\vbox{\hbox{PD\hfil}\hbox{PS\hfil}}}
The drain and source diffusion perimeters. Their default value is 0.

\param{\vbox{\hbox{NRD\hfil}\hbox{NRS\hfil}}}

Multipliers (in units of $\Box$) that can be multiplied by \texttt{RSH}
to yield the parasitic (ohmic) resistances of the drain (\texttt{RD})
and source (\texttt{RS}), respectively.  \texttt{NRD}, \texttt{NRS}
default to 0.

Consider a square sheet of resistive material. Analysis shows that the
resistance between two parallel edges of such a sheet depends upon its
composition and thickness, but is independent of its size as long as it is
square. In other words, the resistance will be the same whether the square's
edge is 2~mm, 2~cm, or 2~m. For this reason, the \emph{sheet resistance} of
such a layer, abbreviated \texttt{RSH}, has units of Ohms per square,
written $\mathsf{\Omega}/\Box$.

\param{M}

If specified, the value is used as a number of parallel MOSFETs to be
simulated.  For example, if \texttt{M=2} is specified, \Xyce{} simulates two
identical mosfets connected to the same nodes in parallel.

\param{IC}

The BSIM3 (model level 9), BSIM4 (model level 14 or 54) and BSIMSOI (model
level 10) allow one to specify the initial voltage difference across
nodes of the device during the DC operating point calculation.  For the
BSIM3 and BSIM4 the syntax is \texttt{IC=$V_{ds}, V_{gs}, V_{bs}$}
where $V_{ds}$ is the voltage difference between the drain and source,
$V_{gs}$ is the voltage difference between the gate and source and
$V_{bs}$ is the voltage difference between the body and source.  The
BSIMSOI device's initial condition syntax is \texttt{IC=$V_{ds},
  V_{gs}, V_{bs}, V_{es}, V_{ps}$} where the two extra terms are the
voltage difference between the substrate and source, and the external
body and source nodes respectively.  Note that for any of these lists of
voltage differences, fewer than the full number of options may be
specified.  For example, \texttt{IC=5.0} specifies an initial condition on $V_{ds}$
but does not specifiy any initial conditions on the other nodes.
Therefore, one cannot specify $V_{gs}$ without specifying $V_{ds}$, etc.

It is illegal to specify initial conditions on any nodes that are tied
together.  \Xyce{} attempts to catch such errors, but complex circuits may
stymie this error trap.

\end{Parameters}

\vbox{\hrulefill}
\item[BSIM-SOI Options]

There are a large number of extra instance parameters and optional
nodes available for the BSIM-SOI (level 10 (BSIM-SOI 3.2), level 70
(BSIM-SOI 4.6.1), and level 70450 (BSIM-SOI 4.5.0)) MOSFET.  Please
consult the BSIM-SOI technical manual, available at
\url{http://bsim.berkeley.edu/models/bsimsoi/}, for full details.

\begin{Parameters}

\param{substrate node}

The fourth node of the BSIM-SOI device is always the substrate node,
which is referred to as the \texttt{E} node. 

\param{external body contact node}

If given, the fifth node is the external body contact node,
\texttt{P}.  It is connected to the internal body node through a body
tie resistor.  If \texttt{P} is not given, the internal body node is
not accessible from the netlist and floats.

{\em For the BSIM-SOI 3.2 (level=10) only):} If there are only five
nodes specified and \texttt{TNODEOUT} is also specified, the fifth
node is the temperature node instead.

\param{internal body contact node}

If given, the sixth node is the internal body contact node, \texttt{B}.  It is
connected to the external body node through a body tie resistor.  If \texttt{B}
is not given and \texttt{P} is given, the internal body node is not accessible
from the netlist, but is still tied to the external body contact through the
tie resistance.

{\em For the BSIM-SOI 3.2 (level=10) only):} If there are only six
nodes specified and \texttt{TNODEOUT} is also specified, the sixth
node is the temperature node instead.

\param{temperature node}

{\em For the BSIM-SOI 3.2 (level=10) only):} If the parameter \texttt{TNODEOUT} is specified, the final node (fifth, sixth,
or seventh) is interpreted as a temperature node.  The temperature node is
intended for thermal coupling simulation.

{\em For the BSIM-SOI 4.x (level=70 or 70450) only):} The temperature
node is only accessible for thermal coupling if it is the seventh
node.  It is available for printing as an internal node in all other
configurations.

\param{BJTOFF}
Turns off the parasitic BJT currents.

\param{IC}
The \texttt{IC} parameter allows specification of the five junction initial
conditions, $V_{ds}, V_{gs}, V_{bs}, V_{es}$ and $V_{ps}$.  $V_{ps}$ is ignored
in a four-terminal device.

\param{RTH0}
Thermal resistance per unit width.  Taken from model card if not given.

\param{CTH0}
Thermal capacitance per unit width.  Taken from model card if not given.

\param{NBC}
Number of body contact isolation edges.

\param{NSEG}
Number of segments for channel width partitioning.

\param{PDBCP}
Parasitic perimeter length for body contact at drain side.

\param{PSBCP}
Parasitic perimeter length for body contact at source side.

\param{AGBCP}
Parasitic gate-to-body overlap area for body contact.

\param{AEBCP}
Parasitic body-to-substrate overlap area for body contact.

\param{VBSUSR}
Optional initial value of VBS specified by user for use in transient
analysis.  (unused in \Xyce{}).

\param{FRBODY}
Layout-dependent body resistance coefficient.

\end{Parameters}

\comments

The simulator provides multiple MOSFET device models, which differ in the
formulation of the I-V characteristic. The \texttt{LEVEL} parameter
selects among different models as shown below.

For HSPICE compatibility, the BSIM4 model can be specified with either
level 14 or level 54.

If a model supports parameter aliases (e.g. ``U0'' and ``UO'' or
``VT0'' and ``VTO'' in the levels 1-6 MOSFETS), it would be a mistake
to specify both parameters and give them different values.  There is
no warning or error message if you do that.  Don't do that.

\end{Device}

\paragraph{MOSFET Operating Temperature}
Model parameters may be assigned unique measurement temperatures using the
\textrmb{TNOM} model parameter. See the MOSFET model parameters for more
information.

\paragraph{MOSFET Power Calculations}
Power dissipated in the transistor is calculated with $I_{D}*V_{DS}+I_{G}*V_{GS}$ where
$I_{D}$ is the drain current, $I_{G}$ is the gate current, $V_{DS}$ is the
voltage drop between the drain and the source and $V_{GS}$ is the voltage drop
between the gate and the source. This formula may differ from other simulators,
such as HSPICE and PSpice.

\paragraph{Internal Device Variables Accessible with {\tt N()} Syntax}
For the BSIM3, BSIM4, and BSIM-CMG version 110 models, several
internal variables have been made accessible with the {\tt N()} syntax
on a {\tt .PRINT} line.  They are $g_{m}$ (tranconductance), $V_{th}$,
$V_{ds}$, $V_{gs}$, $V_{bs}$, and $V_{dsat}$.  An example {\tt .PRINT}
line command for a MOSFET device named {\tt m1} would be:
\begin{alltt}
.print dc N(m1:gm) N(m1:Vth) N(m1:Vdsat) N(m1:Vds) N(m1:Vgs) N(m1:Vbs)
\end{alltt}
The BSIM-CMG also supports output of $I_{ds}$ (drain-source current)
in this manner.

If the user runs \texttt{Xyce -namesfile <filename> <netlist>} then
\Xyce{} will output into the first filename a list of all solution
variables generated by that netlist. This can be useful for
determining the ``fully-qualified'' device name, needed for the {\tt
  N()} syntax, if the device is in a subcircuit.

\paragraph{Instance Parameters}
Tables ~\ref{M_1_Device_Instance_Params}, ~\ref{M_2_Device_Instance_Params}, 
~\ref{M_3_Device_Instance_Params},  ~\ref{M_6_Device_Instance_Params},
\ref{M_9_Device_Instance_Params} and \ref{M_10_Device_Instance_Params}  
give the available instance parameters for the levels 1,2,3,6,9 and 10 MOSFETs,
respectively.

In addition to the parameters shown in the tables, where a list of
numbered initial condition parameters are shown, the MOSFETs support a vector
parameter for the initial conditions.  \texttt{IC1} and \texttt{IC2}
may therefore be specified compactly as \texttt{IC=<ic1>,<ic2>}.

\paragraph{Model Parameters}
Tables ~\ref{M_1_Device_Model_Params}, ~\ref{M_2_Device_Model_Params},
~\ref{M_3_Device_Model_Params}, ~\ref{M_6_Device_Model_Params},
~\ref{M_9_Device_Model_Params}, and ~\ref{M_10_Device_Model_Params}
give the available model parameters for the levels 1,2,3,6,9 and 10 MOSFETs,
respectively.

For a thorough description of MOSFET models see~\cite{Antognetti:1988, HLJHCKH,
BLETK:1997, SH:1968, VL:1980,
SSKJ:1987, Pierret:1984, YEC:1983, BSIM3:V3:1, BN}.

\subparagraph{All MOSFET models}
The parameters shared by all MOSFET model levels are principally parasitic
element values (e.g., series resistance, overlap capacitance, etc.).

\subparagraph{Model levels 1 and 3}
The DC behaviors of the level 1 and 3 MOSFET models are defined by the
parameters \textrmb{VTO}, \textrmb{KP}, \textrmb{LAMBDA}, \textrmb{PHI}, and
\textrmb{GAMMA}.  The simulator calculates these if the process parameters
(e.g., \textrmb{TOX}, and \textrmb{NSUB}) are specified, but these are always
overridden by any user-defined values. The \textrmb{VTO} value is positive
(negative) for modeling the enhancement mode and negative (positive) for the
depletion mode of N-channel (P-channel) devices.

For MOSFETs, the capacitance model enforces charge conservation,
influencing just the Level 1 and 3 models.

Effective device parameter lengths and widths are calculated as follows:
\[
P_i = P_0 + P_L / L_e + P_W / W_e
\]
where
\[
\begin{array}{rclcl}
L_e & = & \mbox{effective length} & = & \mathbf{L} - (2 \cdot \mathbf{LD}) \\
W_e & = & \mbox{effective width} & = & \mathbf{W} - (2 \cdot \mathbf{WD})
\end{array}
\]

See \textrmb{.MODEL} (model definition) for more information.

\subparagraph{Model level 9 (BSIM3 version 3.2.2)}
The University of California, Berkeley BSIM3 model is a physical-based model
with a large number of dependencies on essential dimensional and processing
parameters.  It incorporates the key effects that are critical in modeling
deep-submicrometer MOSFETs.  These include threshold voltage reduction,
nonuniform doping, mobility reduction due to the vertical field, bulk charge
effect, carrier velocity saturation, drain-induced barrier lowering (DIBL),
channel length modulation (CLM), hot-carrier-induced output resistance
reduction, subthreshold conduction, source/drain parasitic resistance,
substrate current induced body effect (SCBE) and drain voltage reduction in LDD
structure.

The BSIM3 Version 3.2.2 model is a deep submicron MOSFET model with several major
enhancements over earlier versions.  These include a single I-V formula used
to define the current and output conductance for operating regions, improved
narrow width device modeling, a superior capacitance model with improved short
and narrow geometry models, a new relaxation-time model to better transient
modeling and enhanced model fitting of assorted W/L ratios using a single
parameter set.  This version preserves the large number of integrated
dependencies on dimensional and processing parameters of the Version 2 model.
For further information, see Reference~\cite{HLJHCKH}.

\subparagraph{Additional notes}
\begin{enumerate}
\item If any of the following BSIM3 3.2.2 model parameters are not specified,
they are computed via the following:

If \textrmb{VTHO} is not specified, then:
\[
\mathbf{VTHO} = \mathbf{VFB} + \phi_s \mathbf{K1} \sqrt{\phi_s}
\]
where:
\[
\mathbf{VFB} = -1.0
\]
If \textrmb{VTHO} is given, then:
\begin{eqnarray*}
\mathbf{VFB} & = & \mathbf{VTHO} - \phi_s + \mathbf{K1}\sqrt{phi_s} \\
\mathbf{VBX} & = & \phi_s - \frac{q\cdot\mathbf{NCH} \cdot
\mathbf{XT}^2}{2\varepsilon_{si}} \\
\mathbf{CF} & = & \left( \frac{2\varepsilon_{ox}}{\pi} \right)
\ln \left(1 + \frac{1}{4 \times 10^7\cdot\mathbf{TOX}} \right)
\end{eqnarray*}
where:
\[
E_g(T) = \mbox{the energy bandgap at temperature }T = 1.16 - \frac{T^2}{7.02
\times 10^4(T + 1108)}
\]

\item If \textrmb{K1} and \textrmb{K2} are not given then they are computed via
the following:
\begin{eqnarray*}
\mathbf{K1} &=& \mathbf{GAMMA2} - 2 \cdot \mathbf{K2} \sqrt{\phi_s -
\mathbf{VBM}} \\
\mathbf{K2} &=& \frac{(\mathbf{GAMMA1} -
\mathbf{GAMMA2})(\sqrt{\phi_s - \mathbf{VBX}} -
\sqrt{\phi_s})}{2\sqrt{\phi_s}(\sqrt{\phi_s - \mathbf{VBM}} -
\sqrt{\phi_s}) + \mathbf{VBM}}
\end{eqnarray*}
where:
\begin{eqnarray*}
\phi_s & = & 2V_t \ln \left(\frac{\mathbf{NCH}}{n_i} \right) \\
V_t    & = & kT / q \\
n_i    & = & 1.45 \times 10^{10} \left(\frac{T}{300.15}
\right)^{1.5} \exp \left(21.5565981 - \frac{E_g(T)}{2V_t} \right)
\end{eqnarray*}

\item If \textrmb{NCH} is not specified and \textrmb{GAMMA1} is, then:
\[
\mathbf{NCH} = \frac{\mathbf{GAMMA1^2 \times \mathbf{COX}^2}}
{2q \varepsilon_{si}}
\]
If \textrmb{GAMMA1} and \textrmb{NCH} {\em are not} specified, then
\textrmb{NCH} defaults to $1.7\times10^{23}\;m^{-3}$ and \textrmb{GAMMA1} is
computed using \textrmb{NCH}:
\[
\mathbf{GAMMA1} = \frac{\sqrt{2q\varepsilon_{si} \cdot \mathbf{NCH}}}
{\mathbf{COX}}
\]
If \textrmb{GAMMA2} is not specified, then:
\[
\mathbf{GAMMA2} = \frac{\sqrt{2q\varepsilon_{si} \cdot \mathbf{NSUB}}}
{\mathbf{COX}}
\]

\item If \textrmb{CGSO} is not specified and $\mathbf{DLC} > 0$, then:
\[
\mathbf{CGSO} = \left\{ \begin{array}{ll}
0, & ((\mathbf{DLC \cdot COX) - CGSL)} < 0        \\
0.6 \cdot \mathbf{XJ \cdot COX}, & ((\mathbf{DLC \cdot COX) - CGSL)}
\geq 0
\end{array}
\right.
\]

\item If \textrmb{CGDO} is not specified and $\mathbf{DLC} > 0$, then:
\[
\mathbf{CGDO} = \left\{ \begin{array}{ll}
0, & ((\mathbf{DLC \cdot COX) - CGSL)} < 0 \\
0.6 \cdot \mathbf{XJ \cdot COX},
& ((\mathbf{DLC \cdot COX) - CGSL)} \geq 0
\end{array}
\right. \]
\end{enumerate}

\subparagraph{Model level 10 (BSIM-SOI version 3.2)}

The BSIM-SOI is an international standard model for SOI (silicon on insulator)
circuit design and is formulated on top of the BSIM3v3 framework.
A detailed description can be found in the BSIM-SOI 3.1 User's
Manual~\cite{BSIMSOI:Manual} and the BSIM-SOI 3.2 release
notes~\cite{BSIMSOI:3p2:Notes}.

This version (v3.2) of the BSIM-SOI includes three depletion models;
the partially depleted BSIM-SOI PD (soiMod=0), the fully depleted BSIM-SOI
FD (soiMod=2), and the unified SOI model (soiMod=1).

BSIMPD is the
Partial-Depletion (PD) mode of the BSIM-SOI.  A typical PD SOI MOSFET is formed
on a thin SOI film which is layered on top of a buried oxide.  BSIMPD has
the following features and enhancements:
\begin{XyceItemize}
\item Real floating body simulation of both I-V and C-V.  The body potential is
      determined by the balance of all body current components.
\item An improved parasitic bipolar current model.  This includes enhancements in
      the various diode leakage components, second order effects (high-level
      injection and Early effect), diffusion charge equation, and temperature
      dependence of the diode junction capacitance.
\item An improved impact-ionization current model.  The contribution from BJT
      current is also modeled by the parameter Fbjtii.
\item A gate-to-body tunneling current model, which is important to thin-oxide
      SOI technologies.
\item Enhancements in the threshold voltage and bulk charge formulation of the
      high positive body bias regime.
\item Instance parameters (Pdbcp, Psbcp, Agbcp, Aebcp, Nbc) are provided to model
      the parasitics of devices with various body-contact and isolation structures.
\item An external body node (the 6th node) and other improvements are introduced
      to facilitate the modeling of distributed body resistance.
\item Self heating.  An external temperature node (the 7th node) is supported to
      facilitate the simulation of thermal coupling among neighboring devices.
\item A unique SOI low frequency noise model, including a new excess noise resulting
      from the floating body effect.
\item Width dependence of the body effect is modeled by parameters (K1,K1w1,K1w2).
\item Improved history dependence of the body charges with two new parameters
      (Fbody, DLCB).
\item An instance parameter Vbsusr is provided for users to set the transient initial
      condition of the body potential.
\item The new charge-thickness capacitance model introduced in BSIM3v3.2,
      \texttt{capMod=3}, is included.
\end{XyceItemize}

\paragraph{Quadratic Temperature Compensation}
SPICE temperature effects are the default, but MOSFET levels 18, 19 and 20 have
a more advanced temperature compensation available.  By specifying
\texttt{TEMPMODEL=QUADRATIC} in the netlist, parameters can be interpolated
quadratically between measured values extracted from data.  See
Section~\ref{Model_Interpolation} for more details.

\paragraph{MOSFET Equations}
The following equations define an N-channel MOSFET. The P-channel
devices use a reverse the sign for all voltages and currents.  The
equations use the following variables:
\begin{eqnarray*}
V_{bs}  &=&\mbox{intrinsic substrate-intrinsic source voltage} \\
V_{bd}  &=&\mbox{intrinsic substrate-intrinsic drain voltage} \\
V_{ds}  &=&\mbox{intrinsic drain-substrate source voltage} \\
V_{dsat}&=&\mbox{saturation voltage} \\
V_{gs}  &=&\mbox{intrinsic gate-intrinsic source voltage} \\
V_{gd}  &=&\mbox{intrinsic gate-intrinsic drain voltage} \\
V_t     &=&kT / q \mbox{ (thermal voltage)} \\
V_{th}  &=&\mbox{threshold voltage} \\
C_{ox}  &=&\mbox{the gate oxide capacitance per unit area} \\
f       &=&\mbox{noise frequency} \\
k       &=&\mbox{Boltzmann's constant} \\
q       &=&\mbox{electron charge} \\
Leff    &=&\mbox{effective channel length} \\
Weff    &=&\mbox{effective channel width} \\
T       &=&\mbox{analysis temperature (K)} \\
T_0     &=&\mbox{nominal temperature (set using TNOM option)}
\end{eqnarray*}
Other variables are listed in the BJT Equations section~\ref{bjt_equations}.

\clearpage
\LTXtable{\textwidth}{mosfeteqntbl}

%%
%% MOSFET Equation Capacitance Table
%%
\paragraph{Capacitance}
\LTXtable{\textwidth}{mosfeteqncaptbl}

%%
%% MOSFET Equation Temperature Effects
%%
\clearpage
\paragraph{Temperature Effects}
\LTXtable{\textwidth}{mosfeteqntemptbl}

%%
%% MOSFET Parameters Table
%%
\clearpage
\subsubsection{Level 1 MOSFET Tables (SPICE Level 1)}
% This table was generated by Xyce:
%   Xyce -doc M 1
%
\index{mosfet level 1!device instance parameters}
\begin{DeviceParamTableGenerated}{MOSFET level 1 Device Instance Parameters}{M_1_Device_Instance_Params}
AD & Drain diffusion area & m$^{2}$ & 0 \\ \hline
AS & Source diffusion area & m$^{2}$ & 0 \\ \hline
DTEMP & Device delta temperature & $^\circ$C & 0 \\ \hline
IC1 & Initial condition on Drain-Source voltage & V & 0 \\ \hline
IC2 & Initial condition on Gate-Source voltage & V & 0 \\ \hline
IC3 & Initial condition on Bulk-Source voltage & V & 0 \\ \hline
L & Channel length & m & 0 \\ \hline
M & Multiplier for M devices connected in parallel & -- & 1 \\ \hline
NRD & Multiplier for RSH to yield parasitic resistance of drain & $\Box$ & 1 \\ \hline
NRS & Multiplier for RSH to yield parasitic resistance of source & $\Box$ & 1 \\ \hline
OFF & Initial condition of no voltage drops across device & logical (T/F) & false \\ \hline
PD & Drain diffusion perimeter & m & 0 \\ \hline
PS & Source diffusion perimeter & m & 0 \\ \hline
TEMP & Device temperature & $^\circ$C & Ambient Temperature \\ \hline
W & Channel width & m & 0 \\ \hline
\end{DeviceParamTableGenerated}

% This table was generated by Xyce:
%   Xyce -doc M 1
%
\index{mosfet level 1!device model parameters}
\begin{DeviceParamTableGenerated}{MOSFET level 1 Device Model Parameters}{M_1_Device_Model_Params}
AF & Flicker noise exponent & -- & 1 \\ \hline
CBD & Zero-bias bulk-drain p-n capacitance & F & 0 \\ \hline
CBS & Zero-bias bulk-source p-n capacitance & F & 0 \\ \hline
CGBO & Gate-bulk overlap capacitance/channel length & F/m & 0 \\ \hline
CGDO & Gate-drain overlap capacitance/channel width & F/m & 0 \\ \hline
CGSO & Gate-source overlap capacitance/channel width & F/m & 0 \\ \hline
CJ & Bulk p-n zero-bias bottom capacitance/area & F/m$^{2}$ & 0 \\ \hline
CJSW & Bulk p-n zero-bias sidewall capacitance/area & F/m$^{2}$ & 0 \\ \hline
FC & Bulk p-n forward-bias capacitance coefficient & -- & 0.5 \\ \hline
GAMMA & Bulk threshold parameter & V$^{1/2}$ & 0 \\ \hline
IS & Bulk p-n saturation current & A & 1e-14 \\ \hline
JS & Bulk p-n saturation current density & A/m$^{2}$ & 0 \\ \hline
KF & Flicker noise coefficient & -- & 0 \\ \hline
KP & Transconductance coefficient & A/V$^{2}$ & 2e-05 \\ \hline
L & Default channel length & m & 0.0001 \\ \hline
LAMBDA & Channel-length modulation & V$^{-1}$ & 0 \\ \hline
LD & Lateral diffusion length & m & 0 \\ \hline
MJ & Bulk p-n bottom grading coefficient & -- & 0.5 \\ \hline
MJSW & Bulk p-n sidewall grading coefficient & -- & 0.5 \\ \hline
NSS & Surface state density & cm$^{-2}$ & 0 \\ \hline
NSUB & Substrate doping density & cm$^{-3}$ & 0 \\ \hline
PB & Bulk p-n bottom potential & V & 0.8 \\ \hline
PHI & Surface potential & V & 0.6 \\ \hline
RD & Drain ohmic resistance & $\mathsf{\Omega}$ & 0 \\ \hline
RS & Source ohmic resistance & $\mathsf{\Omega}$ & 0 \\ \hline
RSH & Drain,source diffusion sheet resistance & $\mathsf{\Omega}$ & 0 \\ \hline
TEMPMODEL & Specifies the type of parameter interpolation over temperature & -- & 'NONE' \\ \hline
TNOM & Nominal device temperature & $^\circ$C & 27 \\ \hline
TOX & Gate oxide thickness & m & 1e-07 \\ \hline
TPG & Gate material type (-1 = same as substrate) 0 = aluminum,1 = opposite of substrate) & -- & 0 \\ \hline
U0 & Surface mobility (alias for UO) & 1/(Vcm$^{2}$s) & 600 \\ \hline
UO & Surface mobility & 1/(Vcm$^{2}$s) & 600 \\ \hline
VT0 & Zero-bias threshold voltage (alias for VTO) & V & 0 \\ \hline
VTO & Zero-bias threshold voltage & V & 0 \\ \hline
W & Default channel width & m & 0.0001 \\ \hline
\end{DeviceParamTableGenerated}

\clearpage
\subsubsection{Level 2 MOSFET Tables (SPICE Level 2)}
% This table was generated by Xyce:
%   Xyce -doc M 2
%
\index{mosfet level 2!device instance parameters}
\begin{DeviceParamTableGenerated}{MOSFET level 2 Device Instance Parameters}{M_2_Device_Instance_Params}
AD & Drain diffusion area & m$^{2}$ & 0 \\ \hline
AS & Source diffusion area & m$^{2}$ & 0 \\ \hline
DTEMP & Device delta temperature & $^\circ$C & 0 \\ \hline
IC1 & Initial condition on Drain-Source voltage & V & 0 \\ \hline
IC2 & Initial condition on Gate-Source voltage & V & 0 \\ \hline
IC3 & Initial condition on Bulk-Source voltage & V & 0 \\ \hline
L & Channel length & m & 0 \\ \hline
M & Multiplier for M devices connected in parallel & -- & 1 \\ \hline
NRD & Multiplier for RSH to yield parasitic resistance of drain & $\Box$ & 1 \\ \hline
NRS & Multiplier for RSH to yield parasitic resistance of source & $\Box$ & 1 \\ \hline
OFF & Initial condition of no voltage drops across device & logical (T/F) & false \\ \hline
PD & Drain diffusion perimeter & m & 0 \\ \hline
PS & Source diffusion perimeter & m & 0 \\ \hline
TEMP & Device temperature & $^\circ$C & Ambient Temperature \\ \hline
W & Channel width & m & 0 \\ \hline
\end{DeviceParamTableGenerated}

% This table was generated by Xyce:
%   Xyce -doc M 2
%
\index{mosfet level 2!device model parameters}
\begin{DeviceParamTableGenerated}{MOSFET level 2 Device Model Parameters}{M_2_Device_Model_Params}
AF & Flicker noise exponent & -- & 1 \\ \hline
CBD & Zero-bias bulk-drain p-n capacitance & F & 0 \\ \hline
CBS & Zero-bias bulk-source p-n capacitance & F & 0 \\ \hline
CGBO & Gate-bulk overlap capacitance/channel length & F/m & 0 \\ \hline
CGDO & Gate-drain overlap capacitance/channel width & F/m & 0 \\ \hline
CGSO & Gate-source overlap capacitance/channel width & F/m & 0 \\ \hline
CJ & Bulk p-n zero-bias bottom capacitance/area & F/m$^{2}$ & 0 \\ \hline
CJSW & Bulk p-n zero-bias sidewall capacitance/area & F/m$^{2}$ & 0 \\ \hline
DELTA & Width effect on threshold & -- & 0 \\ \hline
FC & Bulk p-n forward-bias capacitance coefficient & -- & 0.5 \\ \hline
GAMMA & Bulk threshold parameter & V$^{1/2}$ & 0 \\ \hline
IS & Bulk p-n saturation current & A & 1e-14 \\ \hline
JS & Bulk p-n saturation current density & A/m$^{2}$ & 0 \\ \hline
KF & Flicker noise coefficient & -- & 0 \\ \hline
KP & Transconductance coefficient & A/V$^{2}$ & 2e-05 \\ \hline
L & Default channel length & m & 0.0001 \\ \hline
LAMBDA & Channel-length modulation & V$^{-1}$ & 0 \\ \hline
LD & Lateral diffusion length & m & 0 \\ \hline
MJ & Bulk p-n bottom grading coefficient & -- & 0.5 \\ \hline
MJSW & Bulk p-n sidewall grading coefficient & -- & 0.5 \\ \hline
NEFF & Total channel charge coeff. & -- & 1 \\ \hline
NFS & Fast surface state density & -- & 0 \\ \hline
NSS & Surface state density & cm$^{-2}$ & 0 \\ \hline
NSUB & Substrate doping density & cm$^{-3}$ & 0 \\ \hline
PB & Bulk p-n bottom potential & V & 0.8 \\ \hline
PHI & Surface potential & V & 0.6 \\ \hline
RD & Drain ohmic resistance & $\mathsf{\Omega}$ & 0 \\ \hline
RS & Source ohmic resistance & $\mathsf{\Omega}$ & 0 \\ \hline
RSH & Drain,source diffusion sheet resistance & $\mathsf{\Omega}$ & 0 \\ \hline
TEMPMODEL & Specifies the type of parameter interpolation over temperature & -- & 'NONE' \\ \hline
TNOM & Nominal device temperature & $^\circ$C & 27 \\ \hline
TOX & Gate oxide thickness & m & 1e-07 \\ \hline
TPG & Gate material type (-1 = same as substrate, 0 = aluminum,1 = opposite of substrate) & -- & 0 \\ \hline
U0 & Surface mobility (alias for UO) & 1/(Vcm$^{2}$s) & 600 \\ \hline
UCRIT & Crit. field for mob. degradation & -- & 10000 \\ \hline
UEXP & Crit. field exp for mob. deg. & -- & 0 \\ \hline
UO & Surface mobility & 1/(Vcm$^{2}$s) & 600 \\ \hline
VMAX & Maximum carrier drift velocity & -- & 0 \\ \hline
VT0 & Zero-bias threshold voltage (alias for VTO) & V & 0 \\ \hline
VTO & Zero-bias threshold voltage & V & 0 \\ \hline
W & Default channel width & m & 0.0001 \\ \hline
XJ & Junction depth & -- & 0 \\ \hline
\end{DeviceParamTableGenerated}

\clearpage
\subsubsection{Level 3 MOSFET Tables (SPICE Level 3)}
% This table was generated by Xyce:
%   Xyce -doc M 3
%
\index{mosfet level 3!device instance parameters}
\begin{DeviceParamTableGenerated}{MOSFET level 3 Device Instance Parameters}{M_3_Device_Instance_Params}
AD & Drain diffusion area & m$^{2}$ & 0 \\ \hline
AS & Source diffusion area & m$^{2}$ & 0 \\ \hline
DTEMP & Device delta temperature & $^\circ$C & 0 \\ \hline
IC1 & Initial condition on Drain-Source voltage & V & 0 \\ \hline
IC2 & Initial condition on Gate-Source voltage & V & 0 \\ \hline
IC3 & Initial condition on Bulk-Source voltage & V & 0 \\ \hline
L & Channel length & m & 0 \\ \hline
M & Multiplier for M devices connected in parallel & -- & 1 \\ \hline
NRD & Multiplier for RSH to yield parasitic resistance of drain & $\Box$ & 1 \\ \hline
NRS & Multiplier for RSH to yield parasitic resistance of source & $\Box$ & 1 \\ \hline
OFF & Initial condition of no voltage drops across device & logical (T/F) & false \\ \hline
PD & Drain diffusion perimeter & m & 0 \\ \hline
PS & Source diffusion perimeter & m & 0 \\ \hline
TEMP & Device temperature & $^\circ$C & Ambient Temperature \\ \hline
W & Channel width & m & 0 \\ \hline
\end{DeviceParamTableGenerated}

% This table was generated by Xyce:
%   Xyce -doc M 3
%
\index{mosfet level 3!device model parameters}
\begin{DeviceParamTableGenerated}{MOSFET level 3 Device Model Parameters}{M_3_Device_Model_Params}
AF & Flicker noise exponent & -- & 1 \\ \hline
CBD & Zero-bias bulk-drain p-n capacitance & F & 0 \\ \hline
CBS & Zero-bias bulk-source p-n capacitance & F & 0 \\ \hline
CGBO & Gate-bulk overlap capacitance/channel length & F/m & 0 \\ \hline
CGDO & Gate-drain overlap capacitance/channel width & F/m & 0 \\ \hline
CGSO & Gate-source overlap capacitance/channel width & F/m & 0 \\ \hline
CJ & Bulk p-n zero-bias bottom capacitance/area & F/m$^{2}$ & 0 \\ \hline
CJSW & Bulk p-n zero-bias sidewall capacitance/area & F/m$^{2}$ & 0 \\ \hline
DELTA & Width effect on threshold & -- & 0 \\ \hline
ETA & Static feedback & -- & 0 \\ \hline
FC & Bulk p-n forward-bias capacitance coefficient & -- & 0.5 \\ \hline
GAMMA & Bulk threshold parameter & V$^{1/2}$ & 0 \\ \hline
IS & Bulk p-n saturation current & A & 1e-14 \\ \hline
JS & Bulk p-n saturation current density & A/m$^{2}$ & 0 \\ \hline
KAPPA & Saturation field factor & -- & 0.2 \\ \hline
KF & Flicker noise coefficient & -- & 0 \\ \hline
KP & Transconductance coefficient & A/V$^{2}$ & 2e-05 \\ \hline
L & Default channel length & m & 0.0001 \\ \hline
LD & Lateral diffusion length & m & 0 \\ \hline
MJ & Bulk p-n bottom grading coefficient & -- & 0.5 \\ \hline
MJSW & Bulk p-n sidewall grading coefficient & -- & 0.33 \\ \hline
NFS & Fast surface state density & cm$^{-2}$ & 0 \\ \hline
NSS & Surface state density & cm$^{-2}$ & 0 \\ \hline
NSUB & Substrate doping density & cm$^{-3}$ & 0 \\ \hline
PB & Bulk p-n bottom potential & V & 0.8 \\ \hline
PHI & Surface potential & V & 0.6 \\ \hline
RD & Drain ohmic resistance & $\mathsf{\Omega}$ & 0 \\ \hline
RS & Source ohmic resistance & $\mathsf{\Omega}$ & 0 \\ \hline
RSH & Drain,source diffusion sheet resistance & $\mathsf{\Omega}$ & 0 \\ \hline
TEMPMODEL & Specifies the type of parameter interpolation over temperature & -- & 'NONE' \\ \hline
THETA & Mobility modulation & V$^{-1}$ & 0 \\ \hline
TNOM & Nominal device temperature & $^\circ$C & 27 \\ \hline
TOX & Gate oxide thickness & m & 1e-07 \\ \hline
TPG & Gate material type (-1 = same as substrate,0 = aluminum,1 = opposite of substrate) & -- & 1 \\ \hline
U0 & Surface mobility (alias for UO) & 1/(Vcm$^{2}$s) & 600 \\ \hline
UO & Surface mobility & 1/(Vcm$^{2}$s) & 600 \\ \hline
VMAX & Maximum drift velocity & m/s & 0 \\ \hline
VT0 & Zero-bias threshold voltage (alias for VTO) & V & 0 \\ \hline
VTO & Zero-bias threshold voltage & V & 0 \\ \hline
W & Default channel width & m & 0.0001 \\ \hline
XJ & Metallurgical junction depth & m & 0 \\ \hline
\end{DeviceParamTableGenerated}

\clearpage
\subsubsection{Level 6 MOSFET Tables (SPICE Level 6)}
% This table was generated by Xyce:
%   Xyce -doc M 6
%
\index{mosfet level 6!device instance parameters}
\begin{DeviceParamTableGenerated}{MOSFET level 6 Device Instance Parameters}{M_6_Device_Instance_Params}
AD & Drain diffusion area & m$^{2}$ & 0 \\ \hline
AS & Source diffusion area & m$^{2}$ & 0 \\ \hline
DTEMP & Device delta temperature & $^\circ$C & 0 \\ \hline
IC1 & Initial condition on Drain-Source voltage & V & 0 \\ \hline
IC2 & Initial condition on Gate-Source voltage & V & 0 \\ \hline
IC3 & Initial condition on Bulk-Source voltage & V & 0 \\ \hline
L & Channel length & m & 0 \\ \hline
M & Multiplier for M devices connected in parallel & -- & 1 \\ \hline
NRD & Multiplier for RSH to yield parasitic resistance of drain & $\Box$ & 1 \\ \hline
NRS & Multiplier for RSH to yield parasitic resistance of source & $\Box$ & 1 \\ \hline
OFF & Initial condition of no voltage drops across device & logical (T/F) & false \\ \hline
PD & Drain diffusion perimeter & m & 0 \\ \hline
PS & Source diffusion perimeter & m & 0 \\ \hline
TEMP & Device temperature & $^\circ$C & Ambient Temperature \\ \hline
W & Channel width & m & 0 \\ \hline
\end{DeviceParamTableGenerated}

% This table was generated by Xyce:
%   Xyce -doc M 6
%
\index{mosfet level 6!device model parameters}
\begin{DeviceParamTableGenerated}{MOSFET level 6 Device Model Parameters}{M_6_Device_Model_Params}
AF & Flicker noise exponent & -- & 1 \\ \hline
CBD & Zero-bias bulk-drain p-n capacitance & F & 0 \\ \hline
CBS & Zero-bias bulk-source p-n capacitance & F & 0 \\ \hline
CGBO & Gate-bulk overlap capacitance/channel length & F/m & 0 \\ \hline
CGDO & Gate-drain overlap capacitance/channel width & F/m & 0 \\ \hline
CGSO & Gate-source overlap capacitance/channel width & F/m & 0 \\ \hline
CJ & Bulk p-n zero-bias bottom capacitance/area & F/m$^{2}$ & 0 \\ \hline
CJSW & Bulk p-n zero-bias sidewall capacitance/area & F/m$^{2}$ & 0 \\ \hline
FC & Bulk p-n forward-bias capacitance coefficient & -- & 0.5 \\ \hline
GAMMA & Bulk threshold parameter & -- & 0 \\ \hline
GAMMA1 & Bulk threshold parameter 1 & -- & 0 \\ \hline
IS & Bulk p-n saturation current & A & 1e-14 \\ \hline
JS & Bulk p-n saturation current density & A/m$^{2}$ & 0 \\ \hline
KC & Saturation current factor & -- & 5e-05 \\ \hline
KF & Flicker noise coefficient & -- & 0 \\ \hline
KV & Saturation voltage factor & -- & 2 \\ \hline
LAMBDA & Channel length modulation param. & -- & 0 \\ \hline
LAMBDA0 & Channel length modulation param. 0 & -- & 0 \\ \hline
LAMBDA1 & Channel length modulation param. 1 & -- & 0 \\ \hline
LD & Lateral diffusion length & m & 0 \\ \hline
MJ & Bulk p-n bottom grading coefficient & -- & 0.5 \\ \hline
MJSW & Bulk p-n sidewall grading coefficient & -- & 0.5 \\ \hline
NC & Saturation current coeff. & -- & 1 \\ \hline
NSS & Surface state density & cm$^{-2}$ & 0 \\ \hline
NSUB & Substrate doping density & cm$^{-3}$ & 0 \\ \hline
NV & Saturation voltage coeff. & -- & 0.5 \\ \hline
NVTH & Threshold voltage coeff. & -- & 0.5 \\ \hline
PB & Bulk p-n bottom potential & V & 0.8 \\ \hline
PHI & Surface potential & V & 0.6 \\ \hline
PS & Sat. current modification  par. & -- & 0 \\ \hline
RD & Drain ohmic resistance & $\mathsf{\Omega}$ & 0 \\ \hline
RS & Source ohmic resistance & $\mathsf{\Omega}$ & 0 \\ \hline
RSH & Drain,source diffusion sheet resistance & $\mathsf{\Omega}$ & 0 \\ \hline
SIGMA & Static feedback effect par. & -- & 0 \\ \hline
TEMPMODEL & Specifies the type of parameter interpolation over temperature & -- & 'NONE' \\ \hline
TNOM & Nominal device temperature & $^\circ$C & 27 \\ \hline
TOX & Gate oxide thickness & m & 1e-07 \\ \hline
TPG & Gate material type (-1 = same as substrate,0 = aluminum,1 = opposite of substrate) & -- & 1 \\ \hline
U0 & Surface mobility (alias for UO) & 1/(Vcm$^{2}$s) & 600 \\ \hline
UO & Surface mobility & 1/(Vcm$^{2}$s) & 600 \\ \hline
VT0 & Zero-bias threshold voltage (alias for VTO) & V & 0 \\ \hline
VTO & Zero-bias threshold voltage & V & 0 \\ \hline
\end{DeviceParamTableGenerated}

\clearpage
\subsubsection{Level 9 MOSFET Tables (BSIM3)}
For complete documentation of the BSIM3 model, see the users' manual for
the BSIM3, available for download at
\url{http://bsim.berkeley.edu/models/bsim4/bsim3/}.
\Xyce{} implements Version 3.2.2 of the BSIM3.

In addition to the parameters shown in
table~\ref{M_9_Device_Instance_Params}, the BSIM3 supports a vector
parameter for the initial conditions.  \texttt{IC1} through
\texttt{IC3} may therefore be specified compactly as
\texttt{IC=<ic1>,<ic2>,<ic3>}.

\textbf{NOTE:  Many BSIM3 parameters listed in
tables~\ref{M_9_Device_Instance_Params} and \ref{M_9_Device_Model_Params} as
having default values of zero are actually replaced with internally computed
defaults if not given.  Specifying zero in your model card will override this
internal computation.  It is recommended that you only set model parameters
that you are actually changing from defaults and that you not generate model
cards containing default values from the tables.}
\input{M_9_Device_Instance_Params}
\input{M_9_Device_Model_Params}

\clearpage
\subsubsection{Level 10 MOSFET Tables (BSIM-SOI)}
For complete documentation of the BSIM-SOI model, see the users' manual
for the BSIM-SOI, available for download at
\url{http://bsim.berkeley.edu/models/bsimsoi/}.
\Xyce{} implements Version 3.2 of the BSIM-SOI, you will have to get the
documentation from the FTP archive on the Berkeley site.

In addition to the parameters shown in table~\ref{M_10_Device_Instance_Params}, 
the BSIM3SOI supports a vector parameter for the initial conditions.    \texttt{IC1} through \texttt{IC5}
may therefore be specified compactly as \texttt{IC=<ic1>,<ic2>,<ic3>, <ic4>,<ic5>}.

\textbf{NOTE:  Many BSIM SOI parameters listed in
tables~\ref{M_10_Device_Instance_Params} and \ref{M_10_Device_Model_Params} as
having default values of zero are actually replaced with internally computed
defaults if not given.  Specifying zero in your model card will override this
internal computation.  It is recommended that you only set model parameters
that you are actually changing from defaults and that you not generate model
cards containing default values from the tables.}
% This table was generated by Xyce:
%   Xyce -doc_cat M 10
%
\index{bsim3 soi!device instance parameters}
\begin{DeviceParamTableGenerated}{BSIM3 SOI Device Instance Parameters}{M_10_Device_Instance_Params}
BJTOFF & BJT on/off flag & logical (T/F) & 0 \\ \hline
DEBUG & BJT on/off flag & logical (T/F) & 0 \\ \hline
TNODEOUT & Flag indicating external temp node & logical (T/F) & 0 \\ \hline
VLDEBUG &  & logical (T/F) & false \\ \hline

\category{Control Parameters}\\ \hline
M & Multiplier for M devices connected in parallel & -- & 1 \\ \hline
SOIMOD & SIO model selector,SOIMOD=0: BSIMPD,SOIMOD=1: undefined model for PD and FE,SOIMOD=2: ideal FD & -- & 0 \\ \hline

\category{DC Parameters}\\ \hline
VBSUSR & Vbs specified by user & V & 0 \\ \hline

\category{Geometry Parameters}\\ \hline
AD & Drain diffusion area & m$^{2}$ & 0 \\ \hline
AEBCP & Substrate to body overlap area for bc prasitics & m$^{2}$ & 0 \\ \hline
AGBCP & Gate to body overlap area for bc parasitics & m$^{2}$ & 0 \\ \hline
AS & Source diffusion area & m$^{2}$ & 0 \\ \hline
FRBODY & Layout dependent body-resistance coefficient & -- & 1 \\ \hline
L & Channel length & m & 5e-06 \\ \hline
NBC & Number of body contact isolation edge & -- & 0 \\ \hline
NRB & Number of squares in body & -- & 1 \\ \hline
NRD & Multiplier for RSH to yield parasitic resistance of drain & $\Box$ & 1 \\ \hline
NRS & Multiplier for RSH to yield parasitic resistance of source & $\Box$ & 1 \\ \hline
NSEG & Number segments for width partitioning & -- & 1 \\ \hline
PD & Drain diffusion perimeter & m & 0 \\ \hline
PDBCP & Perimeter length for bc parasitics at drain side & m & 0 \\ \hline
PS & Source diffusion perimeter & m & 0 \\ \hline
PSBCP & Perimeter length for bc parasitics at source side & m & 0 \\ \hline
W & Channel width & m & 5e-06 \\ \hline

\category{RF Parameters}\\ \hline
RGATEMOD & Gate resistance model selector & -- & 0 \\ \hline

\category{Temperature Parameters}\\ \hline
CTH0 & Thermal capacitance & F & 0 \\ \hline
DTEMP & Device delta temperature & $^\circ$C & 0 \\ \hline
RTH0 & normalized thermal resistance & $\mathsf{\Omega}$ & 0 \\ \hline
TEMP & Device temperature & $^\circ$C & Ambient Temperature \\ \hline

\category{Voltage Parameters}\\ \hline
IC1 & Initial condition on Vds & V & 0 \\ \hline
IC2 & Initial condition on Vgs & V & 0 \\ \hline
IC3 & Initial condition on Vbs & V & 0 \\ \hline
IC4 & Initial condition on Ves & V & 0 \\ \hline
IC5 & Initial condition on Vps & V & 0 \\ \hline
OFF & Initial condition of no voltage drops accross device & logical (T/F) & false \\ \hline
\end{DeviceParamTableGenerated}

\input{M_10_Device_Model_Params}

\clearpage
\subsubsection{Level 14/54 MOSFET Tables (BSIM4)}
The level 14 MOSFET device in \Xyce{} is based on the Berkeley BSIM4 model
version 4.6.1.  (For HSPICE compatibility, the Xyce BSIM4 model can also be
specified as level 54.)  The model's parameters are given in the following
tables.  Note that the parameters have not all been properly categorized with
units in place.  For complete documentation of the BSIM4 model, see the BSIM4
User’s Manual, available for download at
\url{http://bsim.berkeley.edu/models/bsim4/}.

Note that the BSIM4 device in Xyce now supports multiple versions
selectable with the \texttt{VERSION} parameter in the model card.  At
this time versions 4.6.1, 4.7.0, and 4.8.2 are supported.  This
version parameter may be specified either in legacy text format
(``4.6.1'' or ``4.8.2'') or in the CMC standard floating point format
(``4.61'' or ``4.82'').

If a \texttt{VERSION} parameter is not given, the latest version
supported is used.

If a \texttt{VERSION} parameter is given that is not one of the
supported version numbers, the closest matching supported version is
used instead and a warning given.  If a version older than the lowest
supported version is chosen, the lowest supported version (4.6.1) is
used and a warning given.  If a model lower than version 4.7.0 is
requested, version 4.6.1 is used (and a warning given).  If a version
newer than 4.7.0 but older than 4.8.0 is requested, 4.7.0 is used and
a warning given.  If a version 4.8.0 or later is requested, 4.8.2 is
used with an appropriate warning.

Specifying any model parameter that is
not supported in the chosen version results in a warning and the
parameter being ignored.  Parameters that are only valid for specific
ranges of versions are noted as such in the
tables~\ref{M_14_Device_Instance_Params} and
\ref{M_14_Device_Model_Params}.

At this time, the BSIM4 is the only device in Xyce that supports
multiple versions in this manner.  All other devices that have
multiple version in Xyce are handled by having a unique level number
for each version.

\textbf{NOTE: Many BSIM4 parameters listed in
  tables~\ref{M_14_Device_Instance_Params} and
  \ref{M_14_Device_Model_Params} as having default values of zero are
  actually replaced with internally computed defaults if not given.
  Specifying zero in your model card will override this internal
  computation.  It is recommended that you only set model parameters
  that you are actually changing from defaults and that you not
  generate model cards containing default values from the tables.
}

\textbf{
  Furthermore, the value of \texttt{FGIDL} changed from 0 to 1 with
  version 4.8.2 of the BSIM4.  This change is NOT reflected in the
  table, which is generated automatically, and which shows only the
  default value of this parameter that applies to versions 4.6.1 and
  4.7.0.}


% This table was generated by Xyce:
%   Xyce -doc_cat M 14
%
\index{bsim4!device instance parameters}
\begin{DeviceParamTableGenerated}{BSIM4 Device Instance Parameters}{M_14_Device_Instance_Params}
AD & Drain area & -- & 0 \\ \hline
AS & Source area & -- & 0 \\ \hline
IC2 &  & -- & 0 \\ \hline
IC3 &  & -- & 0 \\ \hline
L & Length & -- & 5e-06 \\ \hline
M & Number of parallel copies & -- & 1 \\ \hline
MIN & Minimize either D or S & -- & 0 \\ \hline
NF & Number of fingers & -- & 1 \\ \hline
NGCON & Number of gate contacts & -- & 0 \\ \hline
OFF & Device is initially off & -- & false \\ \hline
PD & Drain perimeter & -- & 0 \\ \hline
PS & Source perimeter & -- & 0 \\ \hline
RBDB & Body resistance & -- & 0 \\ \hline
RBPB & Body resistance & -- & 0 \\ \hline
RBPD & Body resistance & -- & 0 \\ \hline
RBPS & Body resistance & -- & 0 \\ \hline
RBSB & Body resistance & -- & 0 \\ \hline
SA & distance between  OD edge to poly of one side  & -- & 0 \\ \hline
SB & distance between  OD edge to poly of the other side & -- & 0 \\ \hline
SC & Distance to a single well edge  & -- & 0 \\ \hline
SCA & Integral of the first distribution function for scattered well dopant & -- & 0 \\ \hline
SCB & Integral of the second distribution function for scattered well dopant & -- & 0 \\ \hline
SCC & Integral of the third distribution function for scattered well dopant & -- & 0 \\ \hline
SD & distance between neighbour fingers & -- & 0 \\ \hline
W & Width & -- & 5e-06 \\ \hline
XGW & Distance from gate contact center to device edge & -- & 0 \\ \hline

\category{Basic Parameters}\\ \hline
DELVT0 & Zero bias threshold voltage variation & V & 0 \\ \hline
DELVTO & Zero bias threshold voltage variation & V & 0 \\ \hline

\category{Control Parameters}\\ \hline
ACNQSMOD & AC NQS model selector & -- & 0 \\ \hline
GEOMOD & Geometry dependent parasitics model selector & -- & 0 \\ \hline
RBODYMOD & Distributed body R model selector & -- & 0 \\ \hline
RGATEMOD & Gate resistance model selector & -- & 0 \\ \hline
RGEOMOD & S/D resistance and contact model selector & -- & 0 \\ \hline
TRNQSMOD & Transient NQS model selector & -- & 0 \\ \hline

\category{Temperature Parameters}\\ \hline
DTEMP & Device delta temperature & $^\circ$C & 0 \\ \hline
TEMP & Device temperature & $^\circ$C & Ambient Temperature \\ \hline

\category{Voltage Parameters}\\ \hline
IC1 & Vector of initial values: Vds,Vgs,Vbs & V & 0 \\ \hline

\category{Asymmetric and Bias-Dependent $R_{ds}$ Parameters}\\ \hline
NRD & Number of squares in drain & -- & 1 \\ \hline
NRS & Number of squares in source & -- & 1 \\ \hline
\end{DeviceParamTableGenerated}

% This table was generated by Xyce:
%   Xyce -doc_cat M 14
%
\index{bsim4!device model parameters}
\begin{DeviceParamTableGenerated}{BSIM4 Device Model Parameters}{M_14_Device_Model_Params}
AF & Flicker noise exponent & -- & 1 \\ \hline
AIGSD & Parameter for Igs,d & -- & 0.0136 \\ \hline
AT & Temperature coefficient of vsat & -- & 33000 \\ \hline
BIGSD & Parameter for Igs,d & -- & 0.00171 \\ \hline
BVD & Drain diode breakdown voltage & -- & 10 \\ \hline
BVS & Source diode breakdown voltage & -- & 10 \\ \hline
CIGSD & Parameter for Igs,d & -- & 0.075 \\ \hline
CJD & Drain bottom junction capacitance per unit area & -- & 0.0005 \\ \hline
CJS & Source bottom junction capacitance per unit area & -- & 0.0005 \\ \hline
CJSWD & Drain sidewall junction capacitance per unit periphery & -- & 5e-10 \\ \hline
CJSWGD & Drain (gate side) sidewall junction capacitance per unit width & -- & 0 \\ \hline
CJSWGS & Source (gate side) sidewall junction capacitance per unit width & -- & 0 \\ \hline
CJSWS & Source sidewall junction capacitance per unit periphery & -- & 5e-10 \\ \hline
DLCIG & Delta L for Ig model & -- & 0 \\ \hline
DMCG & Distance of Mid-Contact to Gate edge & -- & 0 \\ \hline
DMCGT & Distance of Mid-Contact to Gate edge in Test structures & -- & 0 \\ \hline
DMCI & Distance of Mid-Contact to Isolation & -- & 0 \\ \hline
DMDG & Distance of Mid-Diffusion to Gate edge & -- & 0 \\ \hline
DWJ & Delta W for S/D junctions & -- & 0 \\ \hline
EF & Flicker noise frequency exponent & -- & 1 \\ \hline
EM & Flicker noise parameter & -- & 4.1e+07 \\ \hline
EPSRGATE & Dielectric constant of gate relative to vacuum & -- & 11.7 \\ \hline
GBMIN & Minimum body conductance & $\mathsf{\Omega}^{-1}$ & 1e-12 \\ \hline
IJTHDFWD & Forward drain diode forward limiting current & -- & 0.1 \\ \hline
IJTHDREV & Reverse drain diode forward limiting current & -- & 0.1 \\ \hline
IJTHSFWD & Forward source diode forward limiting current & -- & 0.1 \\ \hline
IJTHSREV & Reverse source diode forward limiting current & -- & 0.1 \\ \hline
JSD & Bottom drain junction reverse saturation current density & -- & 0.0001 \\ \hline
JSS & Bottom source junction reverse saturation current density & -- & 0.0001 \\ \hline
JSWD & Isolation edge sidewall drain junction reverse saturation current density & -- & 0 \\ \hline
JSWGD & Gate edge drain junction reverse saturation current density & -- & 0 \\ \hline
JSWGS & Gate edge source junction reverse saturation current density & -- & 0 \\ \hline
JSWS & Isolation edge sidewall source junction reverse saturation current density & -- & 0 \\ \hline
JTSD & Drain bottom trap-assisted saturation current density & -- & 0 \\ \hline
JTSS & Source bottom trap-assisted saturation current density & -- & 0 \\ \hline
JTSSWD & Drain STI sidewall trap-assisted saturation current density & -- & 0 \\ \hline
JTSSWGD & Drain gate-edge sidewall trap-assisted saturation current density & -- & 0 \\ \hline
JTSSWGS & Source gate-edge sidewall trap-assisted saturation current density & -- & 0 \\ \hline
JTSSWS & Source STI sidewall trap-assisted saturation current density & -- & 0 \\ \hline
JTWEFF\newline{\normalfont [Only for versions starting with 4.7]} & TAT current width dependence & m & 0 \\ \hline
K2WE &  K2 shift factor for well proximity effect  & -- & 0 \\ \hline
K3B & Body effect coefficient of k3 & -- & 0 \\ \hline
KF & Flicker noise coefficient & -- & 0 \\ \hline
KT1 & Temperature coefficient of Vth & -- & -0.11 \\ \hline
KT1L & Temperature coefficient of Vth & -- & 0 \\ \hline
KT2 & Body-coefficient of kt1 & -- & 0.022 \\ \hline
KU0 & Mobility degradation/enhancement coefficient for LOD & -- & 0 \\ \hline
KU0WE &  Mobility degradation factor for well proximity effect  & -- & 0 \\ \hline
KVSAT & Saturation velocity degradation/enhancement parameter for LOD & -- & 0 \\ \hline
KVTH0 & Threshold degradation/enhancement parameter for LOD & -- & 0 \\ \hline
KVTH0WE & Threshold shift factor for well proximity effect & -- & 0 \\ \hline
LA0 & Length dependence of a0 & -- & 0 \\ \hline
LA1 & Length dependence of a1 & -- & 0 \\ \hline
LA2 & Length dependence of a2 & -- & 0 \\ \hline
LACDE & Length dependence of acde & -- & 0 \\ \hline
LAGIDL & Length dependence of agidl & -- & 0 \\ \hline
LAGISL & Length dependence of agisl & -- & 0 \\ \hline
LAGS & Length dependence of ags & -- & 0 \\ \hline
LAIGBACC & Length dependence of aigbacc & -- & 0 \\ \hline
LAIGBINV & Length dependence of aigbinv & -- & 0 \\ \hline
LAIGC & Length dependence of aigc & -- & 0 \\ \hline
LAIGD & Length dependence of aigd & -- & 0 \\ \hline
LAIGS & Length dependence of aigs & -- & 0 \\ \hline
LAIGSD & Length dependence of aigsd & -- & 0 \\ \hline
LALPHA0 & Length dependence of alpha0 & -- & 0 \\ \hline
LALPHA1 & Length dependence of alpha1 & -- & 0 \\ \hline
LAT & Length dependence of at & -- & 0 \\ \hline
LB0 & Length dependence of b0 & -- & 0 \\ \hline
LB1 & Length dependence of b1 & -- & 0 \\ \hline
LBETA0 & Length dependence of beta0 & -- & 0 \\ \hline
LBGIDL & Length dependence of bgidl & -- & 0 \\ \hline
LBGISL & Length dependence of bgisl & -- & 0 \\ \hline
LBIGBACC & Length dependence of bigbacc & -- & 0 \\ \hline
LBIGBINV & Length dependence of bigbinv & -- & 0 \\ \hline
LBIGC & Length dependence of bigc & -- & 0 \\ \hline
LBIGD & Length dependence of bigd & -- & 0 \\ \hline
LBIGS & Length dependence of bigs & -- & 0 \\ \hline
LBIGSD & Length dependence of bigsd & -- & 0 \\ \hline
LCDSC & Length dependence of cdsc & -- & 0 \\ \hline
LCDSCB & Length dependence of cdscb & -- & 0 \\ \hline
LCDSCD & Length dependence of cdscd & -- & 0 \\ \hline
LCF & Length dependence of cf & -- & 0 \\ \hline
LCGDL & Length dependence of cgdl & -- & 0 \\ \hline
LCGIDL & Length dependence of cgidl & -- & 0 \\ \hline
LCGISL & Length dependence of cgisl & -- & 0 \\ \hline
LCGSL & Length dependence of cgsl & -- & 0 \\ \hline
LCIGBACC & Length dependence of cigbacc & -- & 0 \\ \hline
LCIGBINV & Length dependence of cigbinv & -- & 0 \\ \hline
LCIGC & Length dependence of cigc & -- & 0 \\ \hline
LCIGD & Length dependence of cigd & -- & 0 \\ \hline
LCIGS & Length dependence of cigs & -- & 0 \\ \hline
LCIGSD & Length dependence of cigsd & -- & 0 \\ \hline
LCIT & Length dependence of cit & -- & 0 \\ \hline
LCKAPPAD & Length dependence of ckappad & -- & 0 \\ \hline
LCKAPPAS & Length dependence of ckappas & -- & 0 \\ \hline
LCLC & Length dependence of clc & -- & 0 \\ \hline
LCLE & Length dependence of cle & -- & 0 \\ \hline
LDELTA & Length dependence of delta & -- & 0 \\ \hline
LDROUT & Length dependence of drout & -- & 0 \\ \hline
LDSUB & Length dependence of dsub & -- & 0 \\ \hline
LDVT0 & Length dependence of dvt0 & -- & 0 \\ \hline
LDVT0W & Length dependence of dvt0w & -- & 0 \\ \hline
LDVT1 & Length dependence of dvt1 & -- & 0 \\ \hline
LDVT1W & Length dependence of dvt1w & -- & 0 \\ \hline
LDVT2 & Length dependence of dvt2 & -- & 0 \\ \hline
LDVT2W & Length dependence of dvt2w & -- & 0 \\ \hline
LDVTP0 & Length dependence of dvtp0 & -- & 0 \\ \hline
LDVTP1 & Length dependence of dvtp1 & -- & 0 \\ \hline
LDVTP2\newline{\normalfont [Only for versions starting with 4.7]} & Length dependence of dvtp2 & -- & 0 \\ \hline
LDVTP3\newline{\normalfont [Only for versions starting with 4.7]} & Length dependence of dvtp3 & -- & 0 \\ \hline
LDVTP4\newline{\normalfont [Only for versions starting with 4.7]} & Length dependence of dvtp4 & -- & 0 \\ \hline
LDVTP5\newline{\normalfont [Only for versions starting with 4.7]} & Length dependence of dvtp5 & -- & 0 \\ \hline
LDWB & Length dependence of dwb & -- & 0 \\ \hline
LDWG & Length dependence of dwg & -- & 0 \\ \hline
LEGIDL & Length dependence of egidl & -- & 0 \\ \hline
LEGISL & Length dependence of egisl & -- & 0 \\ \hline
LEIGBINV & Length dependence for eigbinv & -- & 0 \\ \hline
LETA0 & Length dependence of eta0 & -- & 0 \\ \hline
LETAB & Length dependence of etab & -- & 0 \\ \hline
LEU &  Length dependence of eu & -- & 0 \\ \hline
LFGIDL\newline{\normalfont [Only for versions starting with 4.7]} & Length dependence of fgidl & -- & 0 \\ \hline
LFGISL\newline{\normalfont [Only for versions starting with 4.7]} & Length dependence of fgisl & -- & 0 \\ \hline
LFPROUT & Length dependence of pdiblcb & -- & 0 \\ \hline
LGAMMA1 & Length dependence of gamma1 & -- & 0 \\ \hline
LGAMMA2 & Length dependence of gamma2 & -- & 0 \\ \hline
LINTNOI & lint offset for noise calculation & -- & 0 \\ \hline
LK1 & Length dependence of k1 & -- & 0 \\ \hline
LK2 & Length dependence of k2 & -- & 0 \\ \hline
LK2WE &  Length dependence of k2we  & -- & 0 \\ \hline
LK3 & Length dependence of k3 & -- & 0 \\ \hline
LK3B & Length dependence of k3b & -- & 0 \\ \hline
LKETA & Length dependence of keta & -- & 0 \\ \hline
LKGIDL\newline{\normalfont [Only for versions starting with 4.7]} & Length dependence of kgidl & -- & 0 \\ \hline
LKGISL\newline{\normalfont [Only for versions starting with 4.7]} & Length dependence of kgisl & -- & 0 \\ \hline
LKT1 & Length dependence of kt1 & -- & 0 \\ \hline
LKT1L & Length dependence of kt1l & -- & 0 \\ \hline
LKT2 & Length dependence of kt2 & -- & 0 \\ \hline
LKU0 & Length dependence of ku0 & -- & 0 \\ \hline
LKU0WE &  Length dependence of ku0we  & -- & 0 \\ \hline
LKVTH0 & Length dependence of kvth0 & -- & 0 \\ \hline
LKVTH0WE & Length dependence of kvth0we & -- & 0 \\ \hline
LL & Length reduction parameter & -- & 0 \\ \hline
LLAMBDA & Length dependence of lambda & -- & 0 \\ \hline
LLC & Length reduction parameter for CV & -- & 0 \\ \hline
LLN & Length reduction parameter & -- & 1 \\ \hline
LLODKU0 & Length parameter for u0 LOD effect & -- & 0 \\ \hline
LLODVTH & Length parameter for vth LOD effect & -- & 0 \\ \hline
LLP & Length dependence of lp & -- & 0 \\ \hline
LLPE0 & Length dependence of lpe0 & -- & 0 \\ \hline
LLPEB & Length dependence of lpeb & -- & 0 \\ \hline
LMAX & Maximum length for the model & -- & 1 \\ \hline
LMIN & Minimum length for the model & -- & 0 \\ \hline
LMINV & Length dependence of minv & -- & 0 \\ \hline
LMINVCV & Length dependence of minvcv & -- & 0 \\ \hline
LMOIN & Length dependence of moin & -- & 0 \\ \hline
LNDEP & Length dependence of ndep & -- & 0 \\ \hline
LNFACTOR & Length dependence of nfactor & -- & 0 \\ \hline
LNGATE & Length dependence of ngate & -- & 0 \\ \hline
LNIGBACC & Length dependence of nigbacc & -- & 0 \\ \hline
LNIGBINV & Length dependence of nigbinv & -- & 0 \\ \hline
LNIGC & Length dependence of nigc & -- & 0 \\ \hline
LNOFF & Length dependence of noff & -- & 0 \\ \hline
LNSD & Length dependence of nsd & -- & 0 \\ \hline
LNSUB & Length dependence of nsub & -- & 0 \\ \hline
LNTOX & Length dependence of ntox & -- & 0 \\ \hline
LODETA0 & eta0 shift modification factor for stress effect & -- & 1 \\ \hline
LODK2 & K2 shift modification factor for stress effect & -- & 1 \\ \hline
LPCLM & Length dependence of pclm & -- & 0 \\ \hline
LPDIBLC1 & Length dependence of pdiblc1 & -- & 0 \\ \hline
LPDIBLC2 & Length dependence of pdiblc2 & -- & 0 \\ \hline
LPDIBLCB & Length dependence of pdiblcb & -- & 0 \\ \hline
LPDITS & Length dependence of pdits & -- & 0 \\ \hline
LPDITSD & Length dependence of pditsd & -- & 0 \\ \hline
LPHIN & Length dependence of phin & -- & 0 \\ \hline
LPIGCD & Length dependence for pigcd & -- & 0 \\ \hline
LPOXEDGE & Length dependence for poxedge & -- & 0 \\ \hline
LPRT & Length dependence of prt  & -- & 0 \\ \hline
LPRWB & Length dependence of prwb  & -- & 0 \\ \hline
LPRWG & Length dependence of prwg  & -- & 0 \\ \hline
LPSCBE1 & Length dependence of pscbe1 & -- & 0 \\ \hline
LPSCBE2 & Length dependence of pscbe2 & -- & 0 \\ \hline
LPVAG & Length dependence of pvag & -- & 0 \\ \hline
LRDSW & Length dependence of rdsw  & -- & 0 \\ \hline
LRDW & Length dependence of rdw & -- & 0 \\ \hline
LRGIDL\newline{\normalfont [Only for versions starting with 4.7]} & Length dependence of rgidl & -- & 0 \\ \hline
LRGISL\newline{\normalfont [Only for versions starting with 4.7]} & Length dependence of rgisl & -- & 0 \\ \hline
LRSW & Length dependence of rsw & -- & 0 \\ \hline
LTETA0\newline{\normalfont [Only for versions starting with 4.7]} & Length dependence of teta0 & -- & 0 \\ \hline
LTNFACTOR\newline{\normalfont [Only for versions starting with 4.7]} & Length dependence of tnfactor & -- & 0 \\ \hline
LTVFBSDOFF & Length dependence of tvfbsdoff & -- & 0 \\ \hline
LTVOFF & Length dependence of tvoff & -- & 0 \\ \hline
LTVOFFCV\newline{\normalfont [Only for versions starting with 4.7]} & Length dependence of tvoffcv & -- & 0 \\ \hline
LU0 & Length dependence of u0 & -- & 0 \\ \hline
LUA & Length dependence of ua & -- & 0 \\ \hline
LUA1 & Length dependence of ua1 & -- & 0 \\ \hline
LUB & Length dependence of ub & -- & 0 \\ \hline
LUB1 & Length dependence of ub1 & -- & 0 \\ \hline
LUC & Length dependence of uc & -- & 0 \\ \hline
LUC1 & Length dependence of uc1 & -- & 0 \\ \hline
LUCS\newline{\normalfont [Only for versions starting with 4.7]} &  Length dependence of ucs & -- & 0 \\ \hline
LUCSTE\newline{\normalfont [Only for versions starting with 4.7]} & Length dependence of ucste & -- & 0 \\ \hline
LUD & Length dependence of ud & -- & 0 \\ \hline
LUD1 & Length dependence of ud1 & -- & 0 \\ \hline
LUP & Length dependence of up & -- & 0 \\ \hline
LUTE & Length dependence of ute & -- & 0 \\ \hline
LVBM & Length dependence of vbm & -- & 0 \\ \hline
LVBX & Length dependence of vbx & -- & 0 \\ \hline
LVFB & Length dependence of vfb & -- & 0 \\ \hline
LVFBCV & Length dependence of vfbcv & -- & 0 \\ \hline
LVFBSDOFF & Length dependence of vfbsdoff & -- & 0 \\ \hline
LVOFF & Length dependence of voff & -- & 0 \\ \hline
LVOFFCV & Length dependence of voffcv & -- & 0 \\ \hline
LVSAT & Length dependence of vsat & -- & 0 \\ \hline
LVTH0 &  & -- & 0 \\ \hline
LVTL &  Length dependence of vtl & -- & 0 \\ \hline
LW & Length reduction parameter & -- & 0 \\ \hline
LW0 & Length dependence of w0 & -- & 0 \\ \hline
LWC & Length reduction parameter for CV & -- & 0 \\ \hline
LWL & Length reduction parameter & -- & 0 \\ \hline
LWLC & Length reduction parameter for CV & -- & 0 \\ \hline
LWN & Length reduction parameter & -- & 1 \\ \hline
LWR & Length dependence of wr & -- & 0 \\ \hline
LXJ & Length dependence of xj & -- & 0 \\ \hline
LXN &  Length dependence of xn & -- & 0 \\ \hline
LXRCRG1 & Length dependence of xrcrg1 & -- & 0 \\ \hline
LXRCRG2 & Length dependence of xrcrg2 & -- & 0 \\ \hline
LXT & Length dependence of xt & -- & 0 \\ \hline
MJD & Drain bottom junction capacitance grading coefficient & -- & 0.5 \\ \hline
MJS & Source bottom junction capacitance grading coefficient & -- & 0.5 \\ \hline
MJSWD & Drain sidewall junction capacitance grading coefficient & -- & 0.33 \\ \hline
MJSWGD & Drain (gate side) sidewall junction capacitance grading coefficient & -- & 0.33 \\ \hline
MJSWGS & Source (gate side) sidewall junction capacitance grading coefficient & -- & 0.33 \\ \hline
MJSWS & Source sidewall junction capacitance grading coefficient & -- & 0.33 \\ \hline
NGCON & Number of gate contacts & -- & 1 \\ \hline
NJD & Drain junction emission coefficient & -- & 1 \\ \hline
NJS & Source junction emission coefficient & -- & 1 \\ \hline
NJTS & Non-ideality factor for bottom junction & -- & 20 \\ \hline
NJTSD & Non-ideality factor for bottom junction drain side & -- & 20 \\ \hline
NJTSSW & Non-ideality factor for STI sidewall junction & -- & 20 \\ \hline
NJTSSWD & Non-ideality factor for STI sidewall junction drain side & -- & 20 \\ \hline
NJTSSWG & Non-ideality factor for gate-edge sidewall junction & -- & 20 \\ \hline
NJTSSWGD & Non-ideality factor for gate-edge sidewall junction drain side & -- & 20 \\ \hline
NTNOI & Thermal noise parameter & -- & 1 \\ \hline
PA0 & Cross-term dependence of a0 & -- & 0 \\ \hline
PA1 & Cross-term dependence of a1 & -- & 0 \\ \hline
PA2 & Cross-term dependence of a2 & -- & 0 \\ \hline
PACDE & Cross-term dependence of acde & -- & 0 \\ \hline
PAGIDL & Cross-term dependence of agidl & -- & 0 \\ \hline
PAGISL & Cross-term dependence of agisl & -- & 0 \\ \hline
PAGS & Cross-term dependence of ags & -- & 0 \\ \hline
PAIGBACC & Cross-term dependence of aigbacc & -- & 0 \\ \hline
PAIGBINV & Cross-term dependence of aigbinv & -- & 0 \\ \hline
PAIGC & Cross-term dependence of aigc & -- & 0 \\ \hline
PAIGD & Cross-term dependence of aigd & -- & 0 \\ \hline
PAIGS & Cross-term dependence of aigs & -- & 0 \\ \hline
PAIGSD & Cross-term dependence of aigsd & -- & 0 \\ \hline
PALPHA0 & Cross-term dependence of alpha0 & -- & 0 \\ \hline
PALPHA1 & Cross-term dependence of alpha1 & -- & 0 \\ \hline
PAT & Cross-term dependence of at & -- & 0 \\ \hline
PB0 & Cross-term dependence of b0 & -- & 0 \\ \hline
PB1 & Cross-term dependence of b1 & -- & 0 \\ \hline
PBD & Drain junction built-in potential & -- & 1 \\ \hline
PBETA0 & Cross-term dependence of beta0 & -- & 0 \\ \hline
PBGIDL & Cross-term dependence of bgidl & -- & 0 \\ \hline
PBGISL & Cross-term dependence of bgisl & -- & 0 \\ \hline
PBIGBACC & Cross-term dependence of bigbacc & -- & 0 \\ \hline
PBIGBINV & Cross-term dependence of bigbinv & -- & 0 \\ \hline
PBIGC & Cross-term dependence of bigc & -- & 0 \\ \hline
PBIGD & Cross-term dependence of bigd & -- & 0 \\ \hline
PBIGS & Cross-term dependence of bigs & -- & 0 \\ \hline
PBIGSD & Cross-term dependence of bigsd & -- & 0 \\ \hline
PBS & Source junction built-in potential & -- & 1 \\ \hline
PBSWD & Drain sidewall junction capacitance built in potential & -- & 1 \\ \hline
PBSWGD & Drain (gate side) sidewall junction capacitance built in potential & -- & 0 \\ \hline
PBSWGS & Source (gate side) sidewall junction capacitance built in potential & -- & 0 \\ \hline
PBSWS & Source sidewall junction capacitance built in potential & -- & 1 \\ \hline
PCDSC & Cross-term dependence of cdsc & -- & 0 \\ \hline
PCDSCB & Cross-term dependence of cdscb & -- & 0 \\ \hline
PCDSCD & Cross-term dependence of cdscd & -- & 0 \\ \hline
PCF & Cross-term dependence of cf & -- & 0 \\ \hline
PCGDL & Cross-term dependence of cgdl & -- & 0 \\ \hline
PCGIDL & Cross-term dependence of cgidl & -- & 0 \\ \hline
PCGISL & Cross-term dependence of cgisl & -- & 0 \\ \hline
PCGSL & Cross-term dependence of cgsl & -- & 0 \\ \hline
PCIGBACC & Cross-term dependence of cigbacc & -- & 0 \\ \hline
PCIGBINV & Cross-term dependence of cigbinv & -- & 0 \\ \hline
PCIGC & Cross-term dependence of cigc & -- & 0 \\ \hline
PCIGD & Cross-term dependence of cigd & -- & 0 \\ \hline
PCIGS & Cross-term dependence of cigs & -- & 0 \\ \hline
PCIGSD & Cross-term dependence of cigsd & -- & 0 \\ \hline
PCIT & Cross-term dependence of cit & -- & 0 \\ \hline
PCKAPPAD & Cross-term dependence of ckappad & -- & 0 \\ \hline
PCKAPPAS & Cross-term dependence of ckappas & -- & 0 \\ \hline
PCLC & Cross-term dependence of clc & -- & 0 \\ \hline
PCLE & Cross-term dependence of cle & -- & 0 \\ \hline
PDELTA & Cross-term dependence of delta & -- & 0 \\ \hline
PDROUT & Cross-term dependence of drout & -- & 0 \\ \hline
PDSUB & Cross-term dependence of dsub & -- & 0 \\ \hline
PDVT0 & Cross-term dependence of dvt0 & -- & 0 \\ \hline
PDVT0W & Cross-term dependence of dvt0w & -- & 0 \\ \hline
PDVT1 & Cross-term dependence of dvt1 & -- & 0 \\ \hline
PDVT1W & Cross-term dependence of dvt1w & -- & 0 \\ \hline
PDVT2 & Cross-term dependence of dvt2 & -- & 0 \\ \hline
PDVT2W & Cross-term dependence of dvt2w & -- & 0 \\ \hline
PDVTP0 & Cross-term dependence of dvtp0 & -- & 0 \\ \hline
PDVTP1 & Cross-term dependence of dvtp1 & -- & 0 \\ \hline
PDVTP2\newline{\normalfont [Only for versions starting with 4.7]} & Cross-term dependence of dvtp2 & -- & 0 \\ \hline
PDVTP3\newline{\normalfont [Only for versions starting with 4.7]} & Cross-term dependence of dvtp3 & -- & 0 \\ \hline
PDVTP4\newline{\normalfont [Only for versions starting with 4.7]} & Cross-term dependence of dvtp4 & -- & 0 \\ \hline
PDVTP5\newline{\normalfont [Only for versions starting with 4.7]} & Cross-term dependence of dvtp5 & -- & 0 \\ \hline
PDWB & Cross-term dependence of dwb & -- & 0 \\ \hline
PDWG & Cross-term dependence of dwg & -- & 0 \\ \hline
PEGIDL & Cross-term dependence of egidl & -- & 0 \\ \hline
PEGISL & Cross-term dependence of egisl & -- & 0 \\ \hline
PEIGBINV & Cross-term dependence for eigbinv & -- & 0 \\ \hline
PETA0 & Cross-term dependence of eta0 & -- & 0 \\ \hline
PETAB & Cross-term dependence of etab & -- & 0 \\ \hline
PEU & Cross-term dependence of eu & -- & 0 \\ \hline
PFGIDL\newline{\normalfont [Only for versions starting with 4.7]} & Cross-term dependence of fgidl & -- & 0 \\ \hline
PFGISL\newline{\normalfont [Only for versions starting with 4.7]} & Cross-term dependence of fgisl & -- & 0 \\ \hline
PFPROUT & Cross-term dependence of pdiblcb & -- & 0 \\ \hline
PGAMMA1 & Cross-term dependence of gamma1 & -- & 0 \\ \hline
PGAMMA2 & Cross-term dependence of gamma2 & -- & 0 \\ \hline
PHIG & Work Function of gate & -- & 4.05 \\ \hline
PK1 & Cross-term dependence of k1 & -- & 0 \\ \hline
PK2 & Cross-term dependence of k2 & -- & 0 \\ \hline
PK2WE &  Cross-term dependence of k2we  & -- & 0 \\ \hline
PK3 & Cross-term dependence of k3 & -- & 0 \\ \hline
PK3B & Cross-term dependence of k3b & -- & 0 \\ \hline
PKETA & Cross-term dependence of keta & -- & 0 \\ \hline
PKGIDL\newline{\normalfont [Only for versions starting with 4.7]} & Cross-term dependence of kgidl & -- & 0 \\ \hline
PKGISL\newline{\normalfont [Only for versions starting with 4.7]} & Cross-term dependence of kgisl & -- & 0 \\ \hline
PKT1 & Cross-term dependence of kt1 & -- & 0 \\ \hline
PKT1L & Cross-term dependence of kt1l & -- & 0 \\ \hline
PKT2 & Cross-term dependence of kt2 & -- & 0 \\ \hline
PKU0 & Cross-term dependence of ku0 & -- & 0 \\ \hline
PKU0WE &  Cross-term dependence of ku0we  & -- & 0 \\ \hline
PKVTH0 & Cross-term dependence of kvth0 & -- & 0 \\ \hline
PKVTH0WE & Cross-term dependence of kvth0we & -- & 0 \\ \hline
PLAMBDA & Cross-term dependence of lambda & -- & 0 \\ \hline
PLP & Cross-term dependence of lp & -- & 0 \\ \hline
PLPE0 & Cross-term dependence of lpe0 & -- & 0 \\ \hline
PLPEB & Cross-term dependence of lpeb & -- & 0 \\ \hline
PMINV & Cross-term dependence of minv & -- & 0 \\ \hline
PMINVCV & Cross-term dependence of minvcv & -- & 0 \\ \hline
PMOIN & Cross-term dependence of moin & -- & 0 \\ \hline
PNDEP & Cross-term dependence of ndep & -- & 0 \\ \hline
PNFACTOR & Cross-term dependence of nfactor & -- & 0 \\ \hline
PNGATE & Cross-term dependence of ngate & -- & 0 \\ \hline
PNIGBACC & Cross-term dependence of nigbacc & -- & 0 \\ \hline
PNIGBINV & Cross-term dependence of nigbinv & -- & 0 \\ \hline
PNIGC & Cross-term dependence of nigc & -- & 0 \\ \hline
PNOFF & Cross-term dependence of noff & -- & 0 \\ \hline
PNSD & Cross-term dependence of nsd & -- & 0 \\ \hline
PNSUB & Cross-term dependence of nsub & -- & 0 \\ \hline
PNTOX & Cross-term dependence of ntox & -- & 0 \\ \hline
PPCLM & Cross-term dependence of pclm & -- & 0 \\ \hline
PPDIBLC1 & Cross-term dependence of pdiblc1 & -- & 0 \\ \hline
PPDIBLC2 & Cross-term dependence of pdiblc2 & -- & 0 \\ \hline
PPDIBLCB & Cross-term dependence of pdiblcb & -- & 0 \\ \hline
PPDITS & Cross-term dependence of pdits & -- & 0 \\ \hline
PPDITSD & Cross-term dependence of pditsd & -- & 0 \\ \hline
PPHIN & Cross-term dependence of phin & -- & 0 \\ \hline
PPIGCD & Cross-term dependence for pigcd & -- & 0 \\ \hline
PPOXEDGE & Cross-term dependence for poxedge & -- & 0 \\ \hline
PPRT & Cross-term dependence of prt  & -- & 0 \\ \hline
PPRWB & Cross-term dependence of prwb  & -- & 0 \\ \hline
PPRWG & Cross-term dependence of prwg  & -- & 0 \\ \hline
PPSCBE1 & Cross-term dependence of pscbe1 & -- & 0 \\ \hline
PPSCBE2 & Cross-term dependence of pscbe2 & -- & 0 \\ \hline
PPVAG & Cross-term dependence of pvag & -- & 0 \\ \hline
PRDSW & Cross-term dependence of rdsw  & -- & 0 \\ \hline
PRDW & Cross-term dependence of rdw & -- & 0 \\ \hline
PRGIDL\newline{\normalfont [Only for versions starting with 4.7]} & Cross-term dependence of rgidl & -- & 0 \\ \hline
PRGISL\newline{\normalfont [Only for versions starting with 4.7]} & Cross-term dependence of rgisl & -- & 0 \\ \hline
PRSW & Cross-term dependence of rsw & -- & 0 \\ \hline
PRT & Temperature coefficient of parasitic resistance  & -- & 0 \\ \hline
PTETA0\newline{\normalfont [Only for versions starting with 4.7]} & Cross-term dependence of teta0 & -- & 0 \\ \hline
PTNFACTOR\newline{\normalfont [Only for versions starting with 4.7]} & Cross-term dependence of tnfactor & -- & 0 \\ \hline
PTVFBSDOFF & Cross-term dependence of tvfbsdoff & -- & 0 \\ \hline
PTVOFF & Cross-term dependence of tvoff & -- & 0 \\ \hline
PTVOFFCV\newline{\normalfont [Only for versions starting with 4.7]} & Cross-term dependence of tvoffcv & -- & 0 \\ \hline
PU0 & Cross-term dependence of u0 & -- & 0 \\ \hline
PUA & Cross-term dependence of ua & -- & 0 \\ \hline
PUA1 & Cross-term dependence of ua1 & -- & 0 \\ \hline
PUB & Cross-term dependence of ub & -- & 0 \\ \hline
PUB1 & Cross-term dependence of ub1 & -- & 0 \\ \hline
PUC & Cross-term dependence of uc & -- & 0 \\ \hline
PUC1 & Cross-term dependence of uc1 & -- & 0 \\ \hline
PUCS\newline{\normalfont [Only for versions starting with 4.7]} & Cross-term dependence of ucs & -- & 0 \\ \hline
PUCSTE\newline{\normalfont [Only for versions starting with 4.7]} & Cross-term dependence of ucste & -- & 0 \\ \hline
PUD & Cross-term dependence of ud & -- & 0 \\ \hline
PUD1 & Cross-term dependence of ud1 & -- & 0 \\ \hline
PUP & Cross-term dependence of up & -- & 0 \\ \hline
PUTE & Cross-term dependence of ute & -- & 0 \\ \hline
PVAG & Gate dependence of output resistance parameter & -- & 0 \\ \hline
PVBM & Cross-term dependence of vbm & -- & 0 \\ \hline
PVBX & Cross-term dependence of vbx & -- & 0 \\ \hline
PVFB & Cross-term dependence of vfb & -- & 0 \\ \hline
PVFBCV & Cross-term dependence of vfbcv & -- & 0 \\ \hline
PVFBSDOFF & Cross-term dependence of vfbsdoff & -- & 0 \\ \hline
PVOFF & Cross-term dependence of voff & -- & 0 \\ \hline
PVOFFCV & Cross-term dependence of voffcv & -- & 0 \\ \hline
PVSAT & Cross-term dependence of vsat & -- & 0 \\ \hline
PVTH0 &  & -- & 0 \\ \hline
PVTL & Cross-term dependence of vtl & -- & 0 \\ \hline
PW0 & Cross-term dependence of w0 & -- & 0 \\ \hline
PWR & Cross-term dependence of wr & -- & 0 \\ \hline
PXJ & Cross-term dependence of xj & -- & 0 \\ \hline
PXN & Cross-term dependence of xn & -- & 0 \\ \hline
PXRCRG1 & Cross-term dependence of xrcrg1 & -- & 0 \\ \hline
PXRCRG2 & Cross-term dependence of xrcrg2 & -- & 0 \\ \hline
PXT & Cross-term dependence of xt & -- & 0 \\ \hline
RBDB & Resistance between bNode and dbNode & $\mathsf{\Omega}$ & 50 \\ \hline
RBDBX0 & Body resistance RBDBX  scaling & -- & 100 \\ \hline
RBDBY0 & Body resistance RBDBY  scaling & -- & 100 \\ \hline
RBPB & Resistance between bNodePrime and bNode & $\mathsf{\Omega}$ & 50 \\ \hline
RBPBX0 & Body resistance RBPBX  scaling & -- & 100 \\ \hline
RBPBXL & Body resistance RBPBX L scaling & -- & 0 \\ \hline
RBPBXNF & Body resistance RBPBX NF scaling & -- & 0 \\ \hline
RBPBXW & Body resistance RBPBX W scaling & -- & 0 \\ \hline
RBPBY0 & Body resistance RBPBY  scaling & -- & 100 \\ \hline
RBPBYL & Body resistance RBPBY L scaling & -- & 0 \\ \hline
RBPBYNF & Body resistance RBPBY NF scaling & -- & 0 \\ \hline
RBPBYW & Body resistance RBPBY W scaling & -- & 0 \\ \hline
RBPD & Resistance between bNodePrime and bNode & $\mathsf{\Omega}$ & 50 \\ \hline
RBPD0 & Body resistance RBPD scaling & -- & 50 \\ \hline
RBPDL & Body resistance RBPD L scaling & -- & 0 \\ \hline
RBPDNF & Body resistance RBPD NF scaling & -- & 0 \\ \hline
RBPDW & Body resistance RBPD W scaling & -- & 0 \\ \hline
RBPS & Resistance between bNodePrime and sbNode & $\mathsf{\Omega}$ & 50 \\ \hline
RBPS0 & Body resistance RBPS scaling & -- & 50 \\ \hline
RBPSL & Body resistance RBPS L scaling & -- & 0 \\ \hline
RBPSNF & Body resistance RBPS NF scaling & -- & 0 \\ \hline
RBPSW & Body resistance RBPS W scaling & -- & 0 \\ \hline
RBSB & Resistance between bNode and sbNode & $\mathsf{\Omega}$ & 50 \\ \hline
RBSBX0 & Body resistance RBSBX  scaling & -- & 100 \\ \hline
RBSBY0 & Body resistance RBSBY  scaling & -- & 100 \\ \hline
RBSDBXL & Body resistance RBSDBX L scaling & -- & 0 \\ \hline
RBSDBXNF & Body resistance RBSDBX NF scaling & -- & 0 \\ \hline
RBSDBXW & Body resistance RBSDBX W scaling & -- & 0 \\ \hline
RBSDBYL & Body resistance RBSDBY L scaling & -- & 0 \\ \hline
RBSDBYNF & Body resistance RBSDBY NF scaling & -- & 0 \\ \hline
RBSDBYW & Body resistance RBSDBY W scaling & -- & 0 \\ \hline
RNOIA & Thermal noise coefficient & -- & 0.577 \\ \hline
RNOIB & Thermal noise coefficient & -- & 0.5164 \\ \hline
RNOIC\newline{\normalfont [Only for versions starting with 4.7]} & Thermal noise coefficient & -- & 0.395 \\ \hline
SAREF & Reference distance between OD edge to poly of one side & -- & 1e-06 \\ \hline
SBREF & Reference distance between OD edge to poly of the other side & -- & 1e-06 \\ \hline
SCREF &  Reference distance to calculate SCA,SCB and SCC & -- & 1e-06 \\ \hline
STETA0 & eta0 shift factor related to stress effect on vth & -- & 0 \\ \hline
STK2 & K2 shift factor related to stress effect on vth & -- & 0 \\ \hline
TCJ & Temperature coefficient of cj & -- & 0 \\ \hline
TCJSW & Temperature coefficient of cjsw & -- & 0 \\ \hline
TCJSWG & Temperature coefficient of cjswg & -- & 0 \\ \hline
TETA0\newline{\normalfont [Only for versions starting with 4.7]} & Temperature parameter for eta0 & -- & 0 \\ \hline
TKU0 & Temperature coefficient of KU0 & -- & 0 \\ \hline
TNFACTOR\newline{\normalfont [Only for versions starting with 4.7]} & Temperature parameter for nfactor & -- & 0 \\ \hline
TNJTS & Temperature coefficient for NJTS & -- & 0 \\ \hline
TNJTSD & Temperature coefficient for NJTSD & -- & 0 \\ \hline
TNJTSSW & Temperature coefficient for NJTSSW & -- & 0 \\ \hline
TNJTSSWD & Temperature coefficient for NJTSSWD & -- & 0 \\ \hline
TNJTSSWG & Temperature coefficient for NJTSSWG & -- & 0 \\ \hline
TNJTSSWGD & Temperature coefficient for NJTSSWGD & -- & 0 \\ \hline
TNOIA & Thermal noise parameter & -- & 1.5 \\ \hline
TNOIB & Thermal noise parameter & -- & 3.5 \\ \hline
TNOIC\newline{\normalfont [Only for versions starting with 4.7]} & Thermal noise parameter & -- & 0 \\ \hline
TNOM & Parameter measurement temperature & -- & Ambient Temperature \\ \hline
TPB & Temperature coefficient of pb & -- & 0 \\ \hline
TPBSW & Temperature coefficient of pbsw & -- & 0 \\ \hline
TPBSWG & Temperature coefficient of pbswg & -- & 0 \\ \hline
TVFBSDOFF & Temperature parameter for vfbsdoff & -- & 0 \\ \hline
TVOFF & Temperature parameter for voff & -- & 0 \\ \hline
TVOFFCV\newline{\normalfont [Only for versions starting with 4.7]} & Temperature parameter for tvoffcv & -- & 0 \\ \hline
UA1 & Temperature coefficient of ua & -- & 1e-09 \\ \hline
UB1 & Temperature coefficient of ub & -- & -1e-18 \\ \hline
UC1 & Temperature coefficient of uc & -- & 0 \\ \hline
UCSTE\newline{\normalfont [Only for versions starting with 4.7]} & Temperature coefficient of colombic mobility & -- & -0.004775 \\ \hline
UD1 & Temperature coefficient of ud & -- & 0 \\ \hline
UTE & Temperature coefficient of mobility & -- & -1.5 \\ \hline
VTSD & Drain bottom trap-assisted voltage dependent parameter & -- & 10 \\ \hline
VTSS & Source bottom trap-assisted voltage dependent parameter & -- & 10 \\ \hline
VTSSWD & Drain STI sidewall trap-assisted voltage dependent parameter & -- & 10 \\ \hline
VTSSWGD & Drain gate-edge sidewall trap-assisted voltage dependent parameter & -- & 10 \\ \hline
VTSSWGS & Source gate-edge sidewall trap-assisted voltage dependent parameter & -- & 10 \\ \hline
VTSSWS & Source STI sidewall trap-assisted voltage dependent parameter & -- & 10 \\ \hline
WA0 & Width dependence of a0 & -- & 0 \\ \hline
WA1 & Width dependence of a1 & -- & 0 \\ \hline
WA2 & Width dependence of a2 & -- & 0 \\ \hline
WACDE & Width dependence of acde & -- & 0 \\ \hline
WAGIDL & Width dependence of agidl & -- & 0 \\ \hline
WAGISL & Width dependence of agisl & -- & 0 \\ \hline
WAGS & Width dependence of ags & -- & 0 \\ \hline
WAIGBACC & Width dependence of aigbacc & -- & 0 \\ \hline
WAIGBINV & Width dependence of aigbinv & -- & 0 \\ \hline
WAIGC & Width dependence of aigc & -- & 0 \\ \hline
WAIGD & Width dependence of aigd & -- & 0 \\ \hline
WAIGS & Width dependence of aigs & -- & 0 \\ \hline
WAIGSD & Width dependence of aigsd & -- & 0 \\ \hline
WALPHA0 & Width dependence of alpha0 & -- & 0 \\ \hline
WALPHA1 & Width dependence of alpha1 & -- & 0 \\ \hline
WAT & Width dependence of at & -- & 0 \\ \hline
WB0 & Width dependence of b0 & -- & 0 \\ \hline
WB1 & Width dependence of b1 & -- & 0 \\ \hline
WBETA0 & Width dependence of beta0 & -- & 0 \\ \hline
WBGIDL & Width dependence of bgidl & -- & 0 \\ \hline
WBGISL & Width dependence of bgisl & -- & 0 \\ \hline
WBIGBACC & Width dependence of bigbacc & -- & 0 \\ \hline
WBIGBINV & Width dependence of bigbinv & -- & 0 \\ \hline
WBIGC & Width dependence of bigc & -- & 0 \\ \hline
WBIGD & Width dependence of bigd & -- & 0 \\ \hline
WBIGS & Width dependence of bigs & -- & 0 \\ \hline
WBIGSD & Width dependence of bigsd & -- & 0 \\ \hline
WCDSC & Width dependence of cdsc & -- & 0 \\ \hline
WCDSCB & Width dependence of cdscb & -- & 0 \\ \hline
WCDSCD & Width dependence of cdscd & -- & 0 \\ \hline
WCF & Width dependence of cf & -- & 0 \\ \hline
WCGDL & Width dependence of cgdl & -- & 0 \\ \hline
WCGIDL & Width dependence of cgidl & -- & 0 \\ \hline
WCGISL & Width dependence of cgisl & -- & 0 \\ \hline
WCGSL & Width dependence of cgsl & -- & 0 \\ \hline
WCIGBACC & Width dependence of cigbacc & -- & 0 \\ \hline
WCIGBINV & Width dependence of cigbinv & -- & 0 \\ \hline
WCIGC & Width dependence of cigc & -- & 0 \\ \hline
WCIGD & Width dependence of cigd & -- & 0 \\ \hline
WCIGS & Width dependence of cigs & -- & 0 \\ \hline
WCIGSD & Width dependence of cigsd & -- & 0 \\ \hline
WCIT & Width dependence of cit & -- & 0 \\ \hline
WCKAPPAD & Width dependence of ckappad & -- & 0 \\ \hline
WCKAPPAS & Width dependence of ckappas & -- & 0 \\ \hline
WCLC & Width dependence of clc & -- & 0 \\ \hline
WCLE & Width dependence of cle & -- & 0 \\ \hline
WDELTA & Width dependence of delta & -- & 0 \\ \hline
WDROUT & Width dependence of drout & -- & 0 \\ \hline
WDSUB & Width dependence of dsub & -- & 0 \\ \hline
WDVT0 & Width dependence of dvt0 & -- & 0 \\ \hline
WDVT0W & Width dependence of dvt0w & -- & 0 \\ \hline
WDVT1 & Width dependence of dvt1 & -- & 0 \\ \hline
WDVT1W & Width dependence of dvt1w & -- & 0 \\ \hline
WDVT2 & Width dependence of dvt2 & -- & 0 \\ \hline
WDVT2W & Width dependence of dvt2w & -- & 0 \\ \hline
WDVTP0 & Width dependence of dvtp0 & -- & 0 \\ \hline
WDVTP1 & Width dependence of dvtp1 & -- & 0 \\ \hline
WDVTP2\newline{\normalfont [Only for versions starting with 4.7]} & Width dependence of dvtp2 & -- & 0 \\ \hline
WDVTP3\newline{\normalfont [Only for versions starting with 4.7]} & Width dependence of dvtp3 & -- & 0 \\ \hline
WDVTP4\newline{\normalfont [Only for versions starting with 4.7]} & Width dependence of dvtp4 & -- & 0 \\ \hline
WDVTP5\newline{\normalfont [Only for versions starting with 4.7]} & Width dependence of dvtp5 & -- & 0 \\ \hline
WDWB & Width dependence of dwb & -- & 0 \\ \hline
WDWG & Width dependence of dwg & -- & 0 \\ \hline
WEB & Coefficient for SCB & -- & 0 \\ \hline
WEC & Coefficient for SCC & -- & 0 \\ \hline
WEGIDL & Width dependence of egidl & -- & 0 \\ \hline
WEGISL & Width dependence of egisl & -- & 0 \\ \hline
WEIGBINV & Width dependence for eigbinv & -- & 0 \\ \hline
WETA0 & Width dependence of eta0 & -- & 0 \\ \hline
WETAB & Width dependence of etab & -- & 0 \\ \hline
WEU & Width dependence of eu & -- & 0 \\ \hline
WFGIDL\newline{\normalfont [Only for versions starting with 4.7]} & Width dependence of fgidl & -- & 0 \\ \hline
WFGISL\newline{\normalfont [Only for versions starting with 4.7]} & Width dependence of fgisl & -- & 0 \\ \hline
WFPROUT & Width dependence of pdiblcb & -- & 0 \\ \hline
WGAMMA1 & Width dependence of gamma1 & -- & 0 \\ \hline
WGAMMA2 & Width dependence of gamma2 & -- & 0 \\ \hline
WK1 & Width dependence of k1 & -- & 0 \\ \hline
WK2 & Width dependence of k2 & -- & 0 \\ \hline
WK2WE &  Width dependence of k2we  & -- & 0 \\ \hline
WK3 & Width dependence of k3 & -- & 0 \\ \hline
WK3B & Width dependence of k3b & -- & 0 \\ \hline
WKETA & Width dependence of keta & -- & 0 \\ \hline
WKGIDL\newline{\normalfont [Only for versions starting with 4.7]} & Width dependence of kgidl & -- & 0 \\ \hline
WKGISL\newline{\normalfont [Only for versions starting with 4.7]} & Width dependence of kgisl & -- & 0 \\ \hline
WKT1 & Width dependence of kt1 & -- & 0 \\ \hline
WKT1L & Width dependence of kt1l & -- & 0 \\ \hline
WKT2 & Width dependence of kt2 & -- & 0 \\ \hline
WKU0 & Width dependence of ku0 & -- & 0 \\ \hline
WKU0WE &  Width dependence of ku0we  & -- & 0 \\ \hline
WKVTH0 & Width dependence of kvth0 & -- & 0 \\ \hline
WKVTH0WE & Width dependence of kvth0we & -- & 0 \\ \hline
WL & Width reduction parameter & -- & 0 \\ \hline
WLAMBDA & Width dependence of lambda & -- & 0 \\ \hline
WLC & Width reduction parameter for CV & -- & 0 \\ \hline
WLN & Width reduction parameter & -- & 1 \\ \hline
WLOD & Width parameter for stress effect & -- & 0 \\ \hline
WLODKU0 & Width parameter for u0 LOD effect & -- & 0 \\ \hline
WLODVTH & Width parameter for vth LOD effect & -- & 0 \\ \hline
WLP & Width dependence of lp & -- & 0 \\ \hline
WLPE0 & Width dependence of lpe0 & -- & 0 \\ \hline
WLPEB & Width dependence of lpeb & -- & 0 \\ \hline
WMAX & Maximum width for the model & -- & 1 \\ \hline
WMIN & Minimum width for the model & -- & 0 \\ \hline
WMINV & Width dependence of minv & -- & 0 \\ \hline
WMINVCV & Width dependence of minvcv & -- & 0 \\ \hline
WMOIN & Width dependence of moin & -- & 0 \\ \hline
WNDEP & Width dependence of ndep & -- & 0 \\ \hline
WNFACTOR & Width dependence of nfactor & -- & 0 \\ \hline
WNGATE & Width dependence of ngate & -- & 0 \\ \hline
WNIGBACC & Width dependence of nigbacc & -- & 0 \\ \hline
WNIGBINV & Width dependence of nigbinv & -- & 0 \\ \hline
WNIGC & Width dependence of nigc & -- & 0 \\ \hline
WNOFF & Width dependence of noff & -- & 0 \\ \hline
WNSD & Width dependence of nsd & -- & 0 \\ \hline
WNSUB & Width dependence of nsub & -- & 0 \\ \hline
WNTOX & Width dependence of ntox & -- & 0 \\ \hline
WPCLM & Width dependence of pclm & -- & 0 \\ \hline
WPDIBLC1 & Width dependence of pdiblc1 & -- & 0 \\ \hline
WPDIBLC2 & Width dependence of pdiblc2 & -- & 0 \\ \hline
WPDIBLCB & Width dependence of pdiblcb & -- & 0 \\ \hline
WPDITS & Width dependence of pdits & -- & 0 \\ \hline
WPDITSD & Width dependence of pditsd & -- & 0 \\ \hline
WPEMOD &  Flag for WPE model (WPEMOD=1 to activate this model)  & -- & 0 \\ \hline
WPHIN & Width dependence of phin & -- & 0 \\ \hline
WPIGCD & Width dependence for pigcd & -- & 0 \\ \hline
WPOXEDGE & Width dependence for poxedge & -- & 0 \\ \hline
WPRT & Width dependence of prt & -- & 0 \\ \hline
WPRWB & Width dependence of prwb  & -- & 0 \\ \hline
WPRWG & Width dependence of prwg  & -- & 0 \\ \hline
WPSCBE1 & Width dependence of pscbe1 & -- & 0 \\ \hline
WPSCBE2 & Width dependence of pscbe2 & -- & 0 \\ \hline
WPVAG & Width dependence of pvag & -- & 0 \\ \hline
WRDSW & Width dependence of rdsw  & -- & 0 \\ \hline
WRDW & Width dependence of rdw & -- & 0 \\ \hline
WRGIDL\newline{\normalfont [Only for versions starting with 4.7]} & Width dependence of rgidl & -- & 0 \\ \hline
WRGISL\newline{\normalfont [Only for versions starting with 4.7]} & Width dependence of rgisl & -- & 0 \\ \hline
WRSW & Width dependence of rsw & -- & 0 \\ \hline
WTETA0\newline{\normalfont [Only for versions starting with 4.7]} & Width dependence of teta0 & -- & 0 \\ \hline
WTNFACTOR\newline{\normalfont [Only for versions starting with 4.7]} & Width dependence of tnfactor & -- & 0 \\ \hline
WTVFBSDOFF & Width dependence of tvfbsdoff & -- & 0 \\ \hline
WTVOFF & Width dependence of tvoff & -- & 0 \\ \hline
WTVOFFCV\newline{\normalfont [Only for versions starting with 4.7]} & Width dependence of tvoffcv & -- & 0 \\ \hline
WU0 & Width dependence of u0 & -- & 0 \\ \hline
WUA & Width dependence of ua & -- & 0 \\ \hline
WUA1 & Width dependence of ua1 & -- & 0 \\ \hline
WUB & Width dependence of ub & -- & 0 \\ \hline
WUB1 & Width dependence of ub1 & -- & 0 \\ \hline
WUC & Width dependence of uc & -- & 0 \\ \hline
WUC1 & Width dependence of uc1 & -- & 0 \\ \hline
WUCS\newline{\normalfont [Only for versions starting with 4.7]} & Width dependence of ucs & -- & 0 \\ \hline
WUCSTE\newline{\normalfont [Only for versions starting with 4.7]} & Width dependence of ucste & -- & 0 \\ \hline
WUD & Width dependence of ud & -- & 0 \\ \hline
WUD1 & Width dependence of ud1 & -- & 0 \\ \hline
WUP & Width dependence of up & -- & 0 \\ \hline
WUTE & Width dependence of ute & -- & 0 \\ \hline
WVBM & Width dependence of vbm & -- & 0 \\ \hline
WVBX & Width dependence of vbx & -- & 0 \\ \hline
WVFB & Width dependence of vfb & -- & 0 \\ \hline
WVFBCV & Width dependence of vfbcv & -- & 0 \\ \hline
WVFBSDOFF & Width dependence of vfbsdoff & -- & 0 \\ \hline
WVOFF & Width dependence of voff & -- & 0 \\ \hline
WVOFFCV & Width dependence of voffcv & -- & 0 \\ \hline
WVSAT & Width dependence of vsat & -- & 0 \\ \hline
WVTH0 &  & -- & 0 \\ \hline
WVTL & Width dependence of vtl & -- & 0 \\ \hline
WW & Width reduction parameter & -- & 0 \\ \hline
WW0 & Width dependence of w0 & -- & 0 \\ \hline
WWC & Width reduction parameter for CV & -- & 0 \\ \hline
WWL & Width reduction parameter & -- & 0 \\ \hline
WWLC & Width reduction parameter for CV & -- & 0 \\ \hline
WWN & Width reduction parameter & -- & 1 \\ \hline
WWR & Width dependence of wr & -- & 0 \\ \hline
WXJ & Width dependence of xj & -- & 0 \\ \hline
WXN & Width dependence of xn & -- & 0 \\ \hline
WXRCRG1 & Width dependence of xrcrg1 & -- & 0 \\ \hline
WXRCRG2 & Width dependence of xrcrg2 & -- & 0 \\ \hline
WXT & Width dependence of xt & -- & 0 \\ \hline
XGL & Variation in Ldrawn & -- & 0 \\ \hline
XGW & Distance from gate contact center to device edge & -- & 0 \\ \hline
XJBVD & Fitting parameter for drain diode breakdown current & -- & 1 \\ \hline
XJBVS & Fitting parameter for source diode breakdown current & -- & 1 \\ \hline
XL & L offset for channel length due to mask/etch effect & -- & 0 \\ \hline
XRCRG1 & First fitting parameter the bias-dependent Rg & -- & 12 \\ \hline
XRCRG2 & Second fitting parameter the bias-dependent Rg & -- & 1 \\ \hline
XTID & Drainjunction current temperature exponent & -- & 3 \\ \hline
XTIS & Source junction current temperature exponent & -- & 3 \\ \hline
XTSD & Power dependence of JTSD on temperature & -- & 0.02 \\ \hline
XTSS & Power dependence of JTSS on temperature & -- & 0.02 \\ \hline
XTSSWD & Power dependence of JTSSWD on temperature & -- & 0.02 \\ \hline
XTSSWGD & Power dependence of JTSSWGD on temperature & -- & 0.02 \\ \hline
XTSSWGS & Power dependence of JTSSWGS on temperature & -- & 0.02 \\ \hline
XTSSWS & Power dependence of JTSSWS on temperature & -- & 0.02 \\ \hline
XW & W offset for channel width due to mask/etch effect & -- & 0 \\ \hline

\category{Basic Parameters}\\ \hline
A0 & Non-uniform depletion width effect coefficient. & -- & 1 \\ \hline
A1 & Non-saturation effect coefficient & V$^{-1}$ & 0 \\ \hline
A2 & Non-saturation effect coefficient & -- & 1 \\ \hline
ADOS & Charge centroid parameter & -- & 1 \\ \hline
AGS & Gate bias  coefficient of Abulk. & V$^{-1}$ & 0 \\ \hline
B0 & Abulk narrow width parameter & m & 0 \\ \hline
B1 & Abulk narrow width parameter & m & 0 \\ \hline
BDOS & Charge centroid parameter & -- & 1 \\ \hline
BG0SUB & Band-gap of substrate at T=0K & eV & 1.16 \\ \hline
CDSC & Drain/Source and channel coupling capacitance & F/m$^{2}$ & 0.00024 \\ \hline
CDSCB & Body-bias dependence of cdsc & F/(Vm$^{2}$) & 0 \\ \hline
CDSCD & Drain-bias dependence of cdsc & F/(Vm$^{2}$) & 0 \\ \hline
CIT & Interface state capacitance & F/m$^{2}$ & 0 \\ \hline
DELTA & Effective Vds parameter & V & 0.01 \\ \hline
DROUT & DIBL coefficient of output resistance & -- & 0.56 \\ \hline
DSUB & DIBL coefficient in the subthreshold region & -- & 0 \\ \hline
DVT0 & Short channel effect coeff. 0 & -- & 2.2 \\ \hline
DVT0W & Narrow Width coeff. 0 & -- & 0 \\ \hline
DVT1 & Short channel effect coeff. 1 & -- & 0.53 \\ \hline
DVT1W & Narrow Width effect coeff. 1 & m$^{-1}$ & 5.3e+06 \\ \hline
DVT2 & Short channel effect coeff. 2 & V$^{-1}$ & -0.032 \\ \hline
DVT2W & Narrow Width effect coeff. 2 & V$^{-1}$ & -0.032 \\ \hline
DVTP0 & First parameter for Vth shift due to pocket & m & 0 \\ \hline
DVTP1 & Second parameter for Vth shift due to pocket & V$^{-1}$ & 0 \\ \hline
DVTP2\newline{\normalfont [Only for versions starting with 4.7]} & 3rd parameter for Vth shift due to pocket & Vm$^{X}$ & 0 \\ \hline
DVTP3\newline{\normalfont [Only for versions starting with 4.7]} & 4th parameter for Vth shift due to pocket & -- & 0 \\ \hline
DVTP4\newline{\normalfont [Only for versions starting with 4.7]} & 5th parameter for Vth shift due to pocket & V$^{-1}$ & 0 \\ \hline
DVTP5\newline{\normalfont [Only for versions starting with 4.7]} & 6th parameter for Vth shift due to pocket & V & 0 \\ \hline
DWB & Width reduction parameter & m/V$^{1/2}$ & 0 \\ \hline
DWG & Width reduction parameter & m/V & 0 \\ \hline
EASUB & Electron affinity of substrate & V & 4.05 \\ \hline
EPSRSUB & Dielectric constant of substrate relative to vacuum & -- & 11.7 \\ \hline
ETA0 & Subthreshold region DIBL coefficient & -- & 0.08 \\ \hline
ETAB & Subthreshold region DIBL coefficient & V$^{-1}$ & -0.07 \\ \hline
EU & Mobility exponent & -- & 0 \\ \hline
FPROUT & Rout degradation coefficient for pocket devices & V/m$^{1/2}$ & 0 \\ \hline
K1 & Bulk effect coefficient 1 & V$^{-1/2}$ & 0 \\ \hline
K2 & Bulk effect coefficient 2 & -- & 0 \\ \hline
K3 & Narrow width effect coefficient & -- & 80 \\ \hline
KETA & Body-bias coefficient of non-uniform depletion width effect. & V$^{-1}$ & -0.047 \\ \hline
LAMBDA &  Velocity overshoot parameter & -- & 0 \\ \hline
LC &  back scattering parameter & m & 5e-09 \\ \hline
LEFFEOT\newline{\normalfont [Only for versions starting with 4.7]} & Effective length for extraction of EOT & m & 1e-06 \\ \hline
LINT & Length reduction parameter & m & 0 \\ \hline
LP & Channel length exponential factor of mobility & m & 1e-08 \\ \hline
LPE0 & Equivalent length of pocket region at zero bias & m & 1.74e-07 \\ \hline
LPEB & Equivalent length of pocket region accounting for body bias & m & 0 \\ \hline
MINV & Fitting parameter for moderate inversion in Vgsteff & -- & 0 \\ \hline
NFACTOR & Subthreshold swing Coefficient & -- & 1 \\ \hline
NI0SUB & Intrinsic carrier concentration of substrate at 300.15K & cm$^{-3}$ & 1.45e+10 \\ \hline
PCLM & Channel length modulation Coefficient & -- & 1.3 \\ \hline
PDIBLC1 & Drain-induced barrier lowering coefficient & -- & 0.39 \\ \hline
PDIBLC2 & Drain-induced barrier lowering coefficient & -- & 0.0086 \\ \hline
PDIBLCB & Body-effect on drain-induced barrier lowering & V$^{-1}$ & 0 \\ \hline
PDITS & Coefficient for drain-induced Vth shifts & V$^{-1}$ & 0 \\ \hline
PDITSD & Vds dependence of drain-induced Vth shifts & V$^{-1}$ & 0 \\ \hline
PDITSL & Length dependence of drain-induced Vth shifts & m$^{-1}$ & 0 \\ \hline
PHIN & Adjusting parameter for surface potential due to non-uniform vertical doping & V & 0 \\ \hline
PSCBE1 & Substrate current body-effect coefficient & Vm$^{-1}$ & 4.24e+08 \\ \hline
PSCBE2 & Substrate current body-effect coefficient & m/V & 1e-05 \\ \hline
TBGASUB & First parameter of band-gap change due to temperature & eV/K & 0.000702 \\ \hline
TBGBSUB & Second parameter of band-gap change due to temperature & K & 1108 \\ \hline
TEMPEOT\newline{\normalfont [Only for versions starting with 4.7]} & Temperature for extraction of EOT & -- & 300.15 \\ \hline
U0 & Low-field mobility at Tnom & m$^{2}$/(Vs) & 0 \\ \hline
UA & Linear gate dependence of mobility & m/V & 0 \\ \hline
UB & Quadratic gate dependence of mobility & m$^{2}$/V$^{2}$ & 1e-19 \\ \hline
UC & Body-bias dependence of mobility & V$^{-1}$ & 0 \\ \hline
UCS\newline{\normalfont [Only for versions starting with 4.7]} & Colombic scattering exponent & -- & 1.67 \\ \hline
UD & Coulomb scattering factor of mobility & m$^{-2}$ & 0 \\ \hline
UP & Channel length linear factor of mobility & m$^{-2}$ & 0 \\ \hline
VBM & Maximum body voltage & V & -3 \\ \hline
VDDEOT & Voltage for extraction of equivalent gate oxide thickness & V & 1.5 \\ \hline
VFB & Flat Band Voltage & V & -1 \\ \hline
VOFF & Threshold voltage offset & V & -0.08 \\ \hline
VOFFL & Length dependence parameter for Vth offset & V & 0 \\ \hline
VSAT & Saturation velocity at tnom & m/s & 80000 \\ \hline
VTH0 &  & V & 0 \\ \hline
VTL &  thermal velocity & m/s & 200000 \\ \hline
W0 & Narrow width effect parameter & m & 2.5e-06 \\ \hline
WEFFEOT\newline{\normalfont [Only for versions starting with 4.7]} & Effective width for extraction of EOT & m & 1e-05 \\ \hline
WINT & Width reduction parameter & m & 0 \\ \hline
XN &  back scattering parameter & -- & 3 \\ \hline

\category{Capacitance Parameters}\\ \hline
ACDE & Exponential coefficient for finite charge thickness & m/V & 1 \\ \hline
CF & Fringe capacitance parameter & F/m & 0 \\ \hline
CGBO & Gate-bulk overlap capacitance per length & -- & 0 \\ \hline
CGDL & New C-V model parameter & F/m & 0 \\ \hline
CGDO & Gate-drain overlap capacitance per width & F/m & 0 \\ \hline
CGSL & New C-V model parameter & F/m & 0 \\ \hline
CGSO & Gate-source overlap capacitance per width & F/m & 0 \\ \hline
CKAPPAD & D/G overlap C-V parameter & V & 0.6 \\ \hline
CKAPPAS & S/G overlap C-V parameter  & V & 0.6 \\ \hline
CLC & Vdsat parameter for C-V model & m & 1e-07 \\ \hline
CLE & Vdsat parameter for C-V model & -- & 0.6 \\ \hline
DLC & Delta L for C-V model & m & 0 \\ \hline
DWC & Delta W for C-V model & m & 0 \\ \hline
MINVCV & Fitting parameter for moderate inversion in Vgsteffcv & -- & 0 \\ \hline
MOIN & Coefficient for gate-bias dependent surface potential & -- & 15 \\ \hline
NOFF & C-V turn-on/off parameter & -- & 1 \\ \hline
VFBCV & Flat Band Voltage parameter for capmod=0 only & V & -1 \\ \hline
VOFFCV & C-V lateral-shift parameter & V & 0 \\ \hline
VOFFCVL & Length dependence parameter for Vth offset in CV & -- & 0 \\ \hline
XPART & Channel charge partitioning & F/m & 0 \\ \hline

\category{Control Parameters}\\ \hline
ACNQSMOD & AC NQS model selector & -- & 0 \\ \hline
BINUNIT & Bin  unit  selector & -- & 1 \\ \hline
CAPMOD & Capacitance model selector & -- & 2 \\ \hline
CVCHARGEMOD & Capacitance charge model selector & -- & 0 \\ \hline
DIOMOD & Diode IV model selector & -- & 1 \\ \hline
FNOIMOD & Flicker noise model selector & -- & 1 \\ \hline
GEOMOD & Geometry dependent parasitics model selector & -- & 0 \\ \hline
GIDLMOD\newline{\normalfont [Only for versions starting with 4.7]} & parameter for GIDL selector & -- & 0 \\ \hline
IGBMOD & Gate-to-body Ig model selector & -- & 0 \\ \hline
IGCMOD & Gate-to-channel Ig model selector & -- & 0 \\ \hline
MOBMOD & Mobility model selector & -- & 0 \\ \hline
MTRLCOMPATMOD\newline{\normalfont [Only for versions starting with 4.7]} & New material Mod backward compatibility selector & -- & 0 \\ \hline
MTRLMOD & parameter for nonm-silicon substrate or metal gate selector & -- & 0 \\ \hline
PARAMCHK & Model parameter checking selector & -- & 1 \\ \hline
PERMOD & Pd and Ps model selector & -- & 1 \\ \hline
RBODYMOD & Distributed body R model selector & -- & 0 \\ \hline
RDSMOD & Bias-dependent S/D resistance model selector & -- & 0 \\ \hline
RGATEMOD & Gate R model selector & -- & 0 \\ \hline
RGEOMOD & S/D resistance and contact model selector & -- & 0 \\ \hline
TEMPMOD & Temperature model selector & -- & 0 \\ \hline
TNOIMOD & Thermal noise model selector & -- & 0 \\ \hline
TRNQSMOD & Transient NQS model selector & -- & 0 \\ \hline
VERSION & parameter for model version & -- & '4.6.1' \\ \hline

\category{Flicker and Thermal Noise Parameters}\\ \hline
NOIA & Flicker Noise parameter a & -- & 0 \\ \hline
NOIB & Flicker Noise parameter b & -- & 0 \\ \hline
NOIC & Flicker Noise parameter c & -- & 0 \\ \hline

\category{Process Parameters}\\ \hline
DTOX & Defined as (toxe - toxp)  & m & 0 \\ \hline
EOT & Equivalent gate oxide thickness in meters & m & 1.5e-09 \\ \hline
EPSROX & Dielectric constant of the gate oxide relative to vacuum & -- & 3.9 \\ \hline
GAMMA1 & Vth body coefficient & V$^{1/2}$ & 0 \\ \hline
GAMMA2 & Vth body coefficient & V$^{1/2}$ & 0 \\ \hline
NDEP & Channel doping concentration at the depletion edge & cm$^{-3}$ & 1.7e+17 \\ \hline
NGATE & Poly-gate doping concentration & cm$^{-3}$ & 0 \\ \hline
NSD & S/D doping concentration & cm$^{-3}$ & 1e+20 \\ \hline
NSUB & Substrate doping concentration & cm$^{-3}$ & 6e+16 \\ \hline
RSH & Source-drain sheet resistance & $\mathsf{\Omega}/\Box$ & 0 \\ \hline
RSHG & Gate sheet resistance & $\mathsf{\Omega}/\Box$ & 0.1 \\ \hline
TOXE & Electrical gate oxide thickness in meters & m & 3e-09 \\ \hline
TOXM & Gate oxide thickness at which parameters are extracted & m & 3e-09 \\ \hline
TOXP & Physical gate oxide thickness in meters & m & 3e-09 \\ \hline
VBX & Vth transition body Voltage & V & 0 \\ \hline
XJ & Junction depth in meters & m & 1.5e-07 \\ \hline
XT & Doping depth & m & 1.55e-07 \\ \hline

\category{Tunnelling Parameters}\\ \hline
AIGBACC & Parameter for Igb & (Fs$^2$/g)$^{1/2}$/m & 0.0136 \\ \hline
AIGBINV & Parameter for Igb & (Fs$^2$/g)$^{1/2}$/m & 0.0111 \\ \hline
AIGC & Parameter for Igc & (Fs$^2$/g)$^{1/2}$/m & 0.0136 \\ \hline
AIGD & Parameter for Igd & (Fs$^2$/g)$^{1/2}$/m & 0.0136 \\ \hline
AIGS & Parameter for Igs & (Fs$^2$/g)$^{1/2}$/m & 0.0136 \\ \hline
BIGBACC & Parameter for Igb & (Fs$^2$/g)$^{1/2}$/mV & 0.00171 \\ \hline
BIGBINV & Parameter for Igb & (Fs$^2$/g)$^{1/2}$/mV & 0.000949 \\ \hline
BIGC & Parameter for Igc & (Fs$^2$/g)$^{1/2}$/mV & 0.00171 \\ \hline
BIGD & Parameter for Igd & (Fs$^2$/g)$^{1/2}$/mV & 0.00171 \\ \hline
BIGS & Parameter for Igs & (Fs$^2$/g)$^{1/2}$/mV & 0.00171 \\ \hline
CIGBACC & Parameter for Igb & V$^{-1}$ & 0.075 \\ \hline
CIGBINV & Parameter for Igb & V$^{-1}$ & 0.006 \\ \hline
CIGC & Parameter for Igc & V$^{-1}$ & 0.075 \\ \hline
CIGD & Parameter for Igd & V$^{-1}$ & 0.075 \\ \hline
CIGS & Parameter for Igs & V$^{-1}$ & 0.075 \\ \hline
DLCIGD & Delta L for Ig model drain side & m & 0 \\ \hline
EIGBINV & Parameter for the Si bandgap for Igbinv & V & 1.1 \\ \hline
NIGBACC & Parameter for Igbacc slope & -- & 1 \\ \hline
NIGBINV & Parameter for Igbinv slope & -- & 3 \\ \hline
NIGC & Parameter for Igc slope & -- & 1 \\ \hline
NTOX & Exponent for Tox ratio & -- & 1 \\ \hline
PIGCD & Parameter for Igc partition & -- & 1 \\ \hline
POXEDGE & Factor for the gate edge Tox & -- & 1 \\ \hline
TOXREF & Target tox value & m & 3e-09 \\ \hline
VFBSDOFF & S/D flatband voltage offset & V & 0 \\ \hline

\category{Asymmetric and Bias-Dependent $R_{ds}$ Parameters}\\ \hline
PRWB & Body-effect on parasitic resistance  & V$^{-1}$ & 0 \\ \hline
PRWG & Gate-bias effect on parasitic resistance  & V$^{-1}$ & 1 \\ \hline
RDSW & Source-drain resistance per width & $\mathsf{\Omega}$ $\mu$m & 200 \\ \hline
RDSWMIN & Source-drain resistance per width at high Vg & $\mathsf{\Omega}$ $\mu$m & 0 \\ \hline
RDW & Drain resistance per width & $\mathsf{\Omega}$ $\mu$m & 100 \\ \hline
RDWMIN & Drain resistance per width at high Vg & $\mathsf{\Omega}$ $\mu$m & 0 \\ \hline
RSW & Source resistance per width & $\mathsf{\Omega}$ $\mu$m & 100 \\ \hline
RSWMIN & Source resistance per width at high Vg & $\mathsf{\Omega}$ $\mu$m & 0 \\ \hline
WR & Width dependence of rds & -- & 1 \\ \hline

\category{Impact Ionization Current Parameters}\\ \hline
ALPHA0 & substrate current model parameter & m/V & 0 \\ \hline
ALPHA1 & substrate current model parameter & V$^{-1}$ & 0 \\ \hline
BETA0 & substrate current model parameter & V$^{-1}$ & 0 \\ \hline

\category{Gate-induced Drain Leakage Model Parameters}\\ \hline
AGIDL & Pre-exponential constant for GIDL & $\mathsf{\Omega}^{-1}$ & 0 \\ \hline
AGISL & Pre-exponential constant for GISL & $\mathsf{\Omega}^{-1}$ & 0 \\ \hline
BGIDL & Exponential constant for GIDL & Vm$^{-1}$ & 2.3e+09 \\ \hline
BGISL & Exponential constant for GISL & Vm$^{-1}$ & 2.3e-09 \\ \hline
CGIDL & Parameter for body-bias dependence of GIDL & V$^3$ & 0.5 \\ \hline
CGISL & Parameter for body-bias dependence of GISL & V$^3$ & 0.5 \\ \hline
EGIDL & Fitting parameter for Bandbending & V & 0.8 \\ \hline
EGISL & Fitting parameter for Bandbending & V & 0.8 \\ \hline
FGIDL\newline{\normalfont [Only for versions starting with 4.7]} & GIDL vb parameter & V & 0 \\ \hline
FGISL\newline{\normalfont [Only for versions starting with 4.7]} & Parameter for GISL body bias dependence & V & 0 \\ \hline
KGIDL\newline{\normalfont [Only for versions starting with 4.7]} & GIDL vb parameter & V & 0 \\ \hline
KGISL\newline{\normalfont [Only for versions starting with 4.7]} & Parameter for GISL body bias dependence & V & 0 \\ \hline
RGIDL\newline{\normalfont [Only for versions starting with 4.7]} & GIDL vg parameter & -- & 1 \\ \hline
RGISL\newline{\normalfont [Only for versions starting with 4.7]} & Parameter for GISL gate bias dependence & -- & 1 \\ \hline
\end{DeviceParamTableGenerated}


\clearpage
\subsubsection{Level 18 MOSFET Tables (VDMOS)}
The vertical double-diffused power MOSFET model is based on the uniform charge
control model (UCCM) developed at Rensselaer Polytechnic Institute~\cite{Fjeldly:1998}.
The VDMOS current-voltage characteristics are described by a single, continuous
analytical expression for all regimes of operation.  The physics-based model
includes effects such as velocity saturation in the channel, drain induced barrier
lowering, finite output conductance in saturation, the quasi-saturation effect
through a bias dependent drain parasitic resistance, effects of bulk charge, and
bias dependent low-field mobility.  An important feature of the implementation
is the utilization of a single continuous expression for the drain current, which
is valid below and above threshold, effectively removing discontinuities and
improving convergence properties.

The following tables give parameters for the level 18 MOSFET.

% This table was generated by Xyce:
%   Xyce -doc_cat M 18
%
\index{power mosfet!device instance parameters}
\begin{DeviceParamTableGenerated}{Power MOSFET Device Instance Parameters}{M_18_Device_Instance_Params}

\category{Control Parameters}\\ \hline
M & Multiplier for M devices connected in parallel & -- & 1 \\ \hline

\category{Geometry Parameters}\\ \hline
AD & Drain diffusion area & m$^{2}$ & 0 \\ \hline
AS & Source diffusion area & m$^{2}$ & 0 \\ \hline
L & Channel length & m & 0 \\ \hline
NRD & Multiplier for RSH to yield parasitic resistance of drain & $\Box$ & 1 \\ \hline
NRS & Multiplier for RSH to yield parasitic resistance of source & $\Box$ & 1 \\ \hline
PD & Drain diffusion perimeter & m & 0 \\ \hline
PS & Source diffusion perimeter & m & 0 \\ \hline
W & Channel width & m & 0 \\ \hline

\category{Temperature Parameters}\\ \hline
DTEMP & Device delta temperature & $^\circ$C & 0 \\ \hline
TEMP & Device temperature & $^\circ$C & Ambient Temperature \\ \hline
\end{DeviceParamTableGenerated}

\input{M_18_Device_Model_Params}

\clearpage
\subsubsection{Levels 70 and 70450 MOSFET Tables (BSIM-SOI 4.6.1 and 4.5.0)}
For complete documentation of the BSIM-SOI model, see the users'
manual for the BSIM-SOI, available for download at
\url{http://bsim.berkeley.edu/models/bsimsoi/}.  \Xyce{} implements
Version 4.6.1 of the BSIM-SOI as the level 70 device and version 4.5.0
as level 70450.

Instance and model parameters of the level 70 MOSFET are given in
tables~\ref{M_70_Device_Instance_Params} and
\ref{M_70_Device_Model_Params}.

Beginning with \Xyce{} 7.2, the BSIM-SOI models level 70 and 70450
have {\em limited} support for the optional 5th, 6th, and 7th nodes.
See the BSIM-SOI technical manual at the BSIM web site for details of
what configurations the full device supports.  Only some of these use
cases are supported: Use of the BSIM-SOI 4.x with \texttt{TNODEOUT=0}
(the default) is supported in 4-, 5-, 6-, and 7-node configurations.
\texttt{TNODEOUT=1} is supported only in the 7-node configuration,
with the 7th node being temperature.  No access to the external
temperature node is available in 5- or 6- node configuration.

When \texttt{TNODEOUT=0}, the temperature node is an internal node of
the device even when not specified on the instance line, and its value
may still be printed using the N() notation (see
section~\ref{Print_Device_Info}).  This somewhat minimizes the impact
of the lack of support for \texttt{TNODEOUT=1} in \Xyce{} --- the
temperature rise due to self-heating is always available for printing,
but it is not available for creation of a thermal coupling network
except in the 7-node configuration.

Note that with some choices of model parameters, the BSIM-SOI devices
attempt to ``collapse'' the ``P'' and ``B'' nodes (external and
internal body nodes, 5th and 6th netlist nodes if given, internal
nodes if not given).  Xyce is unable to perform such collapse when the
nodes are externally specified, and will issue warnings when it finds
the model trying to do so.  Depending on the actual nodes used for P
and B, the device may fail to converge or produce invalid results; as
an example, if P and B are actually specified on the netlist line to
be the same node, this failure to collapse will not matter --- the
nodes are already the same.  But if two different node names are used
for the 5th and 6th nodes, the failure to collapse will leave one node
floating and the simulation will likely fail if the printed warnings
are ignored.

A similar problem exists for other choices of model parameter: in some
cases neither the ``P'' nor ``B'' nodes are used, and if the nodes are
specified on the netlist line the BSIM-SOI code attempts to collapse
them to ground.  This is not something \Xyce{} can do, and therefore
instead \Xyce{} ignores the specified nodes.  This can leave those
nodes floating and lead to convergence failures unless the specified
nodes are already the ground node (node 0).  \Xyce{} will issue
appropriate warnings when this condition exists and suggest removal of
the unused external nodes from the instance line.

The BSIM SOI 4.6.1 device supports output of the internal variables in
table~\ref{M_70_OutputVars} on the \texttt{.PRINT} line of a netlist.
To access them from a print line, use the syntax
\texttt{N(<instance>:<variable>)} where ``\texttt{<instance>}'' refers to the
name of the specific level 70 M device in your netlist.

\textbf{NOTE:} It has been observed that the gate capacitance model of
BSIM-SOI 4.6.1 behaves differently than earlier versions, and the team
has seen significant disagreement of gate currents when comparing
identical simulations with other simulators that have only earlier
BSIM-SOI models.  For this reason, we are also providing BSIM-SOI
4.5.0 as the level 70450 MOSFET.  This model does agree with these
other simulators.  The parameters and output variables are given in
tables~\ref{M_70450_Device_Instance_Params},
\ref{M_70450_Device_Model_Params}, and \ref{M_70450_OutputVars}.
Unlike BSIM-SOI 4.6.1, the 4.5.0 model's original Verilog-A source
code does not contain descriptions and units for the parameters, and
these appear blank in the tables.  For descriptions and units, see the
corresponding parameters in the level 70 tables.

\input{M_70_Device_Instance_Params}
\input{M_70_Device_Model_Params}
\input{M_70_OutputVars}
\input{M_70450_Device_Instance_Params}
\input{M_70450_Device_Model_Params}
\input{M_70450_OutputVars}

\clearpage
\subsubsection{Level 77 MOSFET Tables (BSIM6 version 6.1.1)}
\Xyce{} includes the BSIM6 MOSFET model, version 6.1.1.  Full
documentation of the BSIM6 is available at its web site,
\url{http://bsim.berkeley.edu/models/bsim6/}.  Instance and model
parameters for the BSIM6 are given in
tables~\ref{M_77_Device_Instance_Params} and
\ref{M_77_Device_Model_Params}.  These tables are generated directly
from information present in the original Verilog-A implementation of
the BSIM6, and lack many descriptions for the parameters.  Consult the
BSIM6 technical manual from the BSIM group for further details about
these parameters.

Beginning with version 7.2 of \Xyce{}, an optional fifth node may be
specified for BSIM6 devices.  If specified, it is the temperature
node, which is used by the self-heating model and is internal if not
specified on the instance line.

The BSIM6 device supports output of the internal variables in
table~\ref{M_77_OutputVars} on the \texttt{.PRINT} line of a netlist.
To access them from a print line, use the syntax
\texttt{N(<instance>:<variable>)} where ``\texttt{<instance>}'' refers to the
name of the specific level 77 M device in your netlist.

\input{M_77_Device_Instance_Params}
\input{M_77_Device_Model_Params}
\input{M_77_OutputVars}

\clearpage
\subsubsection{Level 102 MOSFET Tables (PSP version 102.5)}

\Xyce{} includes a legacy version of the PSP MOSFET model, version
102.5.  This version is provided because the more recent 103 versions
are not backward compatible with the older 102 versions, and some
foundries provide model cards that use the version 102.  Development
of new model cards should be done using the more recent, supported
versions of PSP.

The PSP102 device supports output of the internal variables in
table~\ref{M_102_OutputVars} on the \texttt{.PRINT} line of a netlist.
To access them from a print line, use the syntax
\texttt{N(<instance>:<variable>)} where ``\texttt{<instance>}'' refers to the
name of the specific PSP102 M device in your netlist.

\input{M_102_Device_Instance_Params}
\input{M_102_Device_Model_Params}
\input{M_102_OutputVars}

\subsubsection{Level 103 and 1031 MOSFET Tables (PSP version 103.4)}

\Xyce{} includes the PSP MOSFET model, version 103.4~\cite{PSP:2006}.
The version without self-heating is the level 103 MOSFET, and the
version with self-heating is the level 1031.  Note that the level 1031
MOSFET requires five nodes on its instance line: drain, gate, source,
bulk, and dt.  The fifth node will be the temperature rise of the
device due to self-heating.

Full documentation for the PSP model is available on its web site,
\url{http://www.cea.fr/cea-tech/leti/pspsupport}.  Instance and model
parameters for the PSP model are given in
tables~\ref{M_103_Device_Instance_Params}, \ref{M_103_Device_Model_Params},
\ref{M_1031_Device_Instance_Params}, and \ref{M_1031_Device_Model_Params}.

The PSP103 devices support output of the internal variables in
table~\ref{M_103_OutputVars} and table~\ref{M_1031_OutputVars} on the \texttt{.PRINT} line of a netlist.
To access them from a print line, use the syntax
\texttt{N(<instance>:<variable>)} where ``\texttt{<instance>}'' refers to the
name of the specific PSP103 M device in your netlist.

% This table was generated by Xyce:
%   Xyce -doc M 103
%
\index{psp103va mosfet!device instance parameters}
\begin{DeviceParamTableGenerated}{PSP103VA MOSFET Device Instance Parameters}{M_103_Device_Instance_Params}
ABDRAIN & Bottom area of drain junction & m$^{2}$ & 1e-12 \\ \hline
ABSOURCE & Bottom area of source junction & m$^{2}$ & 1e-12 \\ \hline
AD & Bottom area of drain junction & m$^{2}$ & 1e-12 \\ \hline
AS & Bottom area of source junction & m$^{2}$ & 1e-12 \\ \hline
DELVTO & Threshold voltage shift parameter & V & 0 \\ \hline
DELVTOEDGE & Threshold voltage shift parameter of edge transistor & V & 0 \\ \hline
DTA & Temperature offset w.r.t. ambient temperature & K & 0 \\ \hline
FACTUO & Zero-field mobility pre-factor & --- & 1 \\ \hline
FACTUOEDGE & Zero-field mobility pre-factor of edge transistor & --- & 1 \\ \hline
JW & Gate-edge length of source/drain junction & m & 1e-06 \\ \hline
L & Design length & m & 1e-05 \\ \hline
LGDRAIN & Gate-edge length of drain junction & m & 1e-06 \\ \hline
LGSOURCE & Gate-edge length of source junction & m & 1e-06 \\ \hline
LSDRAIN & STI-edge length of drain junction & m & 1e-06 \\ \hline
LSSOURCE & STI-edge length of source junction & m & 1e-06 \\ \hline
M &  Alias for MULT & --- & 1 \\ \hline
MULT & Number of devices in parallel & --- & 1 \\ \hline
NF & Number of fingers & --- & 1 \\ \hline
NGCON & Number of gate contacts & --- & 1 \\ \hline
NRD & Number of squares of drain diffusion & --- & 0 \\ \hline
NRS & Number of squares of source diffusion & --- & 0 \\ \hline
PD & Perimeter of drain junction & m & 1e-06 \\ \hline
PS & Perimeter of source junction & m & 1e-06 \\ \hline
SA & Distance between OD-edge and poly from one side & m & 0 \\ \hline
SB & Distance between OD-edge and poly from other side & m & 0 \\ \hline
SC & Distance between OD-edge and nearest well edge & m & 0 \\ \hline
SCA & Integral of the first distribution function for scattered well dopants & --- & 0 \\ \hline
SCB & Integral of the second distribution function for scattered well dopants & --- & 0 \\ \hline
SCC & Integral of the third distribution function for scattered well dopants & --- & 0 \\ \hline
SD & Distance between neighbouring fingers & m & 0 \\ \hline
W & Design width & m & 1e-05 \\ \hline
XGW & Distance from the gate contact to the channel edge & m & 1e-07 \\ \hline
\end{DeviceParamTableGenerated}

\input{M_103_Device_Model_Params}
\input{M_103_OutputVars}
% This table was generated by Xyce:
%   Xyce -doc M 1031
%
\index{psp103va mosfet with self-heating!device instance parameters}
\begin{DeviceParamTableGenerated}{PSP103VA MOSFET with self-heating Device Instance Parameters}{M_1031_Device_Instance_Params}
ABDRAIN & Bottom area of drain junction & m$^{2}$ & 1e-12 \\ \hline
ABSOURCE & Bottom area of source junction & m$^{2}$ & 1e-12 \\ \hline
AD & Bottom area of drain junction & m$^{2}$ & 1e-12 \\ \hline
AS & Bottom area of source junction & m$^{2}$ & 1e-12 \\ \hline
DELVTO & Threshold voltage shift parameter & V & 0 \\ \hline
DELVTOEDGE & Threshold voltage shift parameter of edge transistor & V & 0 \\ \hline
DTA & Temperature offset w.r.t. ambient temperature & K & 0 \\ \hline
FACTUO & Zero-field mobility pre-factor & --- & 1 \\ \hline
FACTUOEDGE & Zero-field mobility pre-factor of edge transistor & --- & 1 \\ \hline
JW & Gate-edge length of source/drain junction & m & 1e-06 \\ \hline
L & Design length & m & 1e-05 \\ \hline
LGDRAIN & Gate-edge length of drain junction & m & 1e-06 \\ \hline
LGSOURCE & Gate-edge length of source junction & m & 1e-06 \\ \hline
LSDRAIN & STI-edge length of drain junction & m & 1e-06 \\ \hline
LSSOURCE & STI-edge length of source junction & m & 1e-06 \\ \hline
M &  Alias for MULT & --- & 1 \\ \hline
MULT & Number of devices in parallel & --- & 1 \\ \hline
NF & Number of fingers & --- & 1 \\ \hline
NGCON & Number of gate contacts & --- & 1 \\ \hline
NRD & Number of squares of drain diffusion & --- & 0 \\ \hline
NRS & Number of squares of source diffusion & --- & 0 \\ \hline
PD & Perimeter of drain junction & m & 1e-06 \\ \hline
PS & Perimeter of source junction & m & 1e-06 \\ \hline
SA & Distance between OD-edge and poly from one side & m & 0 \\ \hline
SB & Distance between OD-edge and poly from other side & m & 0 \\ \hline
SC & Distance between OD-edge and nearest well edge & m & 0 \\ \hline
SCA & Integral of the first distribution function for scattered well dopants & --- & 0 \\ \hline
SCB & Integral of the second distribution function for scattered well dopants & --- & 0 \\ \hline
SCC & Integral of the third distribution function for scattered well dopants & --- & 0 \\ \hline
SD & Distance between neighbouring fingers & m & 0 \\ \hline
W & Design width & m & 1e-05 \\ \hline
XGW & Distance from the gate contact to the channel edge & m & 1e-07 \\ \hline
\end{DeviceParamTableGenerated}

\input{M_1031_Device_Model_Params}
\input{M_1031_OutputVars}

\clearpage

\subsubsection{Level 111 MOSFET Tables (BSIM CMG version 111.2.1)}
\Xyce{} includes the BSIM CMG Common Multi-gate model version 111.
The code in \Xyce{} was generated from the BSIM group's Verilog-A
input using the default ``ifdef'' lines provided, and therefore
supports only the subset of BSIM CMG features those defaults enable.
Instance and model parameters for the BSIM CMG model are given in
tables~\ref{M_111_Device_Instance_Params} and
\ref{M_111_Device_Model_Params}.  Details of the model are documented
in the BSIM-CMG technical report\cite{BSIMCMG:Manual}, available from
the BSIM web site at
\url{http://bsim.berkeley.edu/models/bsimcmg/}.

The BSIM CMG devices support output of the internal variables in
tables~\ref{M_107_OutputVars}, \ref{M_108_OutputVars}, and  \ref{M_111_OutputVars} on the \texttt{.PRINT} line of a netlist.
To access them from a print line, use the syntax
\texttt{N(<instance>:<variable>)} where ``\texttt{<instance>}'' refers to the
name of the specific level 107 or 108 M device in your netlist.

% This table was generated by Xyce:
%   Xyce -doc M 111
%
\index{bsim-cmg finfet v111.2.1!device instance parameters}
\begin{DeviceParamTableGenerated}{BSIM-CMG FINFET v111.2.1 Device Instance Parameters}{M_111_Device_Instance_Params}
ADEJ & Drain junction area (BULKMOD = 1 or 2) & m$^{2}$ & 0 \\ \hline
ADEO & Drain-to-substrate overlap area through oxide & m$^{2}$ & 0 \\ \hline
ASEJ & Source junction area (BULKMOD = 1 or 2) & m$^{2}$ & 0 \\ \hline
ASEO & Source-to-substrate overlap area through oxide & m$^{2}$ & 0 \\ \hline
CDSP & Constant drain-to-source fringe capacitance (all CGEOMOD) & F & 0 \\ \hline
CGDP & Constant gate-to-drain fringe capacitance (CGEOMOD = 1) & --- & 0 \\ \hline
CGSP & Constant gate-to-source fringe capacitance (CGEOMOD = 1) & --- & 0 \\ \hline
COVD & Constant gate-to-drain overlap capacitance (CGEOMOD = 1) & --- & 0 \\ \hline
COVS & Constant gate-to-source overlap capacitance (CGEOMOD = 1) & --- & 0 \\ \hline
D & Diameter of the cylinder (GEOMOD = 3) & m & 4e-08 \\ \hline
DACH1 & Total area correction for 1st GAA body & m$^{2}$ & 0 \\ \hline
DACH2 & Total area correction for 2nd GAA body & m$^{2}$ & 0 \\ \hline
DACH3 & Total area correction for 3rd GAA body & m$^{2}$ & 0 \\ \hline
DACH4 & Total area correction for 4th GAA body & m$^{2}$ & 0 \\ \hline
DACH5 & Total area correction for 5th GAA body & m$^{2}$ & 0 \\ \hline
DACH6 & Total area correction for 6th GAA body & m$^{2}$ & 0 \\ \hline
DELVTRAND & Variability in Vth & V & 0 \\ \hline
DTEMP & Variability in device temperature & $^\circ$C & 0 \\ \hline
DWS1 & Total width correction for 1st GAA body & m & 0 \\ \hline
DWS2 & Total width correction for 2nd GAA body & m & 0 \\ \hline
DWS3 & Total width correction for 3rd GAA body & m & 0 \\ \hline
DWS4 & Total width correction for 4th GAA body & m & 0 \\ \hline
DWS5 & Total width correction for 5th GAA body & m & 0 \\ \hline
DWS6 & Total width correction for 6th GAA body & m & 0 \\ \hline
FPITCH & Fin pitch & m & 8e-08 \\ \hline
HPFF & Fin-height of parasitic finFET & m & 5e-09 \\ \hline
IDS0MULT & Variability in drain current for miscellaneous reasons & --- & 1 \\ \hline
IGB0MULT & Gate to body current scale factor & --- & 1 \\ \hline
IGC0MULT & Gate to channel current scale factor & --- & 1 \\ \hline
L & Designed gate length & m & 3e-08 \\ \hline
LRSD & Length of the source/drain & m & 0 \\ \hline
M & multiplicity factor & --- & 1 \\ \hline
MOBSCMOD & Switch for GAAFET geometry dependent mobility model (0: off; 1: on) & --- & 0 \\ \hline
NF & Number of fingers & --- & 1 \\ \hline
NFIN & Number of fins per finger (real number enables optimization) & --- & 1 \\ \hline
NFINNOM & If non-zero, nominal number of fins per finger & --- & 0 \\ \hline
NGAA & Number of GAA bodies per fin & --- & 1 \\ \hline
NGCON & Number of gate contact (1 or 2 sided) & --- & 1 \\ \hline
NRD & Number of drain diffusion squares & --- & 0 \\ \hline
NRS & Number of source diffusion squares & --- & 0 \\ \hline
PDEJ & Drain-to-substrate PN junction perimeter (BULKMOD = 1 or 2) & m & 0 \\ \hline
PDEO & Perimeter of drain-to-substrate overlap region through oxide & m & 0 \\ \hline
PSEJ & Source-to-substrate PN junction perimeter (BULKMOD = 1 or 2) & m & 0 \\ \hline
PSEO & Perimeter of source-to-substrate overlap region through oxide & m & 0 \\ \hline
SUBBANDMOD & Switch for GAAFET quantum subband model (0: off; 1: on) & --- & 0 \\ \hline
TFIN & Fin thickness & m & 1.5e-08 \\ \hline
TGAA & Thickness of individual GAA bodies & m & 5e-09 \\ \hline
TSUS & Separation between GAA bodies & m & 2e-09 \\ \hline
U0MULT & Variability in carrier mobility & --- & 1 \\ \hline
WGAA & GAA body width & m & 6e-09 \\ \hline
XL & L offset for channel length due to mask/etch effect & m & 0 \\ \hline
XW & W offset for GAA channel width due to mask/etch effect & m & 0 \\ \hline
\end{DeviceParamTableGenerated}

% This table was generated by Xyce:
%   Xyce -doc M 111
%
\index{bsim-cmg finfet v111.2.1!device model parameters}
\begin{DeviceParamTableGenerated}{BSIM-CMG FINFET v111.2.1 Device Model Parameters}{M_111_Device_Model_Params}
A1 & Non-saturation effect parameter for strong inversion Region & --- & 0 \\ \hline
A11 & Temperature dependence of A1 & --- & 0 \\ \hline
A2 & Non-saturation effect parameter for moderate Inversion Region & V$^{-1}$ & 0 \\ \hline
A21 & Temperature dependence of A2 & --- & 0 \\ \hline
ACH\_UFCM & Area of the channel for the unified model & m$^{2}$ & 1 \\ \hline
ADEJ & Drain junction area (BULKMOD = 1 or 2) & m$^{2}$ & 0 \\ \hline
ADEO & Drain-to-substrate overlap area through oxide & m$^{2}$ & 0 \\ \hline
ADVTP0 & Pre-exponential coefficient for DITS & --- & 0 \\ \hline
ADVTP1 & Pre-exponential coefficient for DVTP1 & --- & 0 \\ \hline
AEU & Pre-exponential coefficient for EU & --- & 0 \\ \hline
AEUR & Reverse-mode pre-exponential coefficient for EU & --- & 0 \\ \hline
AGIDL & Pre-exponential coefficient for GIDL & --- & 6.055e-12 \\ \hline
AGIDLB & Pre-exponential coefficient for parasitic substrate GIDL & --- & 6.055e-12 \\ \hline
AGISL & Pre-exponential coefficient for GISL & --- & 0 \\ \hline
AGISLB & Pre-exponential coefficient for parasitic substrate GISL & --- & 0 \\ \hline
AIGBACC & Parameter for Igb in accumulation & --- & 0.0136 \\ \hline
AIGBACC1 & Parameter for Igb in accumulation & --- & 0 \\ \hline
AIGBINV & Parameter for Igb in inversion & --- & 0.0111 \\ \hline
AIGBINV1 & Parameter for Igb in inversion & --- & 0 \\ \hline
AIGC & Parameter for Igc in inversion & --- & 0.0136 \\ \hline
AIGC1 & Parameter for Igc in inversion & --- & 0 \\ \hline
AIGD & Parameter for Igd in inversion & --- & 0 \\ \hline
AIGD1 & Parameter for Igd in inversion & --- & 0 \\ \hline
AIGEN & Thermal generation current parameter & --- & 0 \\ \hline
AIGS & Parameter for Igs in inversion & --- & 0.0136 \\ \hline
AIGS1 & Parameter for Igs in inversion & --- & 0 \\ \hline
ALPHA0 & First parameter of Iii & m/V & 0 \\ \hline
ALPHA01 & Temperature dependence of ALPHA0 & --- & 0 \\ \hline
ALPHA1 & L scaling parameter of Iii & V$^{-1}$ & 0 \\ \hline
ALPHA11 & Temperature dependence ALPHA1 & --- & 0 \\ \hline
ALPHA\_UFCM & Mobile charge scaling term taking QM effects into account & --- & 0.5556 \\ \hline
ALPHAII0 & First parameter of Iii for IIMOD = 2 & m/V & 0 \\ \hline
ALPHAII01 & Temperature dependence of ALPHAII0 & --- & 0 \\ \hline
ALPHAII1 & L scaling parameter of Iii for IIMOD = 2 & V$^{-1}$ & 0 \\ \hline
ALPHAII11 & Temperature dependence of ALPHAII1 & --- & 0 \\ \hline
AMEXP & Pre-exponential coefficient for MEXP & --- & 0 \\ \hline
AMEXPR & Pre-exponential coefficient for MEXPR & --- & 0 \\ \hline
APCLM & Pre-exponential coefficient for PCLM & --- & 0 \\ \hline
APCLMR & Reverse-mode pre-exponential coefficient for PCLM & --- & 0 \\ \hline
APSAT & Pre-exponential coefficient for PSAT & --- & 0 \\ \hline
APSATCV & Pre-exponential coefficient for PSATCV & --- & 0 \\ \hline
APTWG & Pre-exponential coefficient for PTWG & --- & 0 \\ \hline
AQMTCEN & Parameter for geometric dependence of Tcen on R/TFIN/HFIN & --- & 0 \\ \hline
ARDSW & Pre-exponential coefficient for RDSW & --- & 0 \\ \hline
ARDW & Pre-exponential coefficient for RDW & --- & 0 \\ \hline
ARSDEND & Extra raised source/drain cross sectional areaat the two ends of the finFET & m$^{2}$ & 0 \\ \hline
ARSW & Pre-exponential coefficient for RSW & --- & 0 \\ \hline
ASEJ & Source junction area (BULKMOD = 1 or 2) & m$^{2}$ & 0 \\ \hline
ASEO & Source-to-substrate overlap area through oxide & m$^{2}$ & 0 \\ \hline
ASHEXP & Exponent to tune RTH dependence of NFINTOTAL & --- & 1 \\ \hline
ASILIEND & Extra silicide cross sectional area at the two ends of the FinFET & m$^{2}$ & 0 \\ \hline
ASYMMOD & 0: Turn off asymmetry model - forward mode parameters used; 1: Turn on asymmetry model & --- & 0 \\ \hline
AT & Saturation velocity temperature coefficient & --- & -0.00156 \\ \hline
AT2 & CRYOMOD != 0 saturation velocity temperature coefficient & --- & 2e-06 \\ \hline
AT2CV & CRYOMOD != 0 saturation velocity temperature coefficient for CV & --- & 0 \\ \hline
ATCV & Saturation velocity temperature coefficient for CV & --- & 0 \\ \hline
ATR & Reverse-mode saturation velocity temperature coefficient & --- & 0 \\ \hline
ATVSRSD & Temperature coefficient for S/D region saturation velocity & --- & 0 \\ \hline
AUA & Pre-exponential coefficient for UA & --- & 0 \\ \hline
AUAR & Reverse-mode pre-exponential coefficient for UA & --- & 0 \\ \hline
AUD & Pre-exponential coefficient for UD & --- & 0 \\ \hline
AUDR & Reverse-mode pre-exponential coefficient for UD & --- & 0 \\ \hline
AVSAT & Pre-exponential coefficient for VSAT & --- & 0 \\ \hline
AVSAT1 & Pre-exponential coefficient for VSAT1 & --- & 0 \\ \hline
AVSATCV & Pre-exponential coefficient for VSATCV & --- & 0 \\ \hline
BDVTP0 & Exponential coefficient for DITS & --- & 1e-07 \\ \hline
BDVTP1 & Exponential coefficient for DVTP1 & --- & 1e-07 \\ \hline
BETA0 & Vds dependence parameter of Iii & V$^{-1}$ & 0 \\ \hline
BETAII0 & Vds dependence parameter of Iii & V$^{-1}$ & 0 \\ \hline
BETAII1 & Vds dependence parameter of Iii & --- & 0 \\ \hline
BETAII2 & Vds dependence parameter of Iii & V & 0.1 \\ \hline
BEU & Exponential coefficient for EU & --- & 1e-07 \\ \hline
BEUR & Reverse-mode exponential coefficient for EU & --- & 0 \\ \hline
BG0SUB & Bandgap of substrate at 300.15K & --- & 1.12 \\ \hline
BGIDL & Exponential coefficient for GIDL & --- & 3e+08 \\ \hline
BGIDLB & Exponential coefficient for parasitic substrate GIDL & --- & 3e+08 \\ \hline
BGISL & Exponential coefficient for GISL & --- & 0 \\ \hline
BGISLB & Exponential coefficient for parasitic substrate GISL & --- & 0 \\ \hline
BIGBACC & Parameter for Igb in accumulation & --- & 0.00171 \\ \hline
BIGBINV & Parameter for Igb in inversion & --- & 0.000949 \\ \hline
BIGC & Parameter for Igc in inversion & --- & 0.00171 \\ \hline
BIGD & Parameter for Igd in inversion & --- & 0 \\ \hline
BIGEN & Thermal generation current parameter & --- & 0 \\ \hline
BIGS & Parameter for Igs in inversion & --- & 0.00171 \\ \hline
BMEXP & Exponential coefficient for MEXP & --- & 1 \\ \hline
BMEXPR & Exponential coefficient for MEXPR & --- & 0 \\ \hline
BPCLM & Exponential coefficient for PCLM & --- & 1e-07 \\ \hline
BPCLMR & Reverse-mode exponential coefficient for PCLM & --- & 0 \\ \hline
BPSAT & Exponential coefficient for PSAT & --- & 1 \\ \hline
BPSATCV & Exponential coefficient for PSATCV & --- & 0 \\ \hline
BPTWG & Exponential coefficient for PTWG & --- & 1e-07 \\ \hline
BQMTCEN & Parameter for geometric dependence of Tcen on R/TFIN/HFIN & --- & 1.2e-08 \\ \hline
BRDSW & exponential coefficient for RDSW & --- & 1e-07 \\ \hline
BRDW & Exponential coefficient for RDW & --- & 1e-07 \\ \hline
BRSW & Exponential coefficient for RSW & --- & 1e-07 \\ \hline
BSHEXP & Exponent to tune RTH dependence of NF & --- & 1 \\ \hline
BUA & Exponential coefficient for UA & --- & 1e-07 \\ \hline
BUAR & Reverse-mode exponential coefficient for UAR & --- & 0 \\ \hline
BUD & Exponential coefficient for UD & --- & 5e-08 \\ \hline
BUDR & Reverse-mode exponential coefficient for UD & --- & 0 \\ \hline
BULKMOD & 0: SOI multi-gate; 1: Bulk multi-gate; 2: for decoupled bulk multi-gate & --- & 0 \\ \hline
BVD & Drain diode breakdown voltage & V & 0 \\ \hline
BVS & Source diode breakdown voltage & V & 10 \\ \hline
BVSAT & Exponential coefficient for VSAT & --- & 1e-07 \\ \hline
BVSAT1 & Exponential coefficient for VSAT1 & --- & 0 \\ \hline
BVSATCV & Exponential coefficient for VSATCV & --- & 0 \\ \hline
CDSC & Coupling capacitance between S/D and channel & --- & 0.007 \\ \hline
CDSCD & Drain-bias sensitivity of CDSC & --- & 0.007 \\ \hline
CDSCDN1 & NFIN dependence of CDSCD & --- & 0 \\ \hline
CDSCDN2 & NFIN dependence of CDSCD & --- & 100000 \\ \hline
CDSCDR & Reverse-mode drain-bias sensitivity of CDSC & --- & 0 \\ \hline
CDSCDRN1 & NFIN dependence of CDSCD & --- & 0 \\ \hline
CDSCDRN2 & NFIN dependence of CDSCD & --- & 0 \\ \hline
CDSCN1 & NFIN dependence of CDSC & --- & 0 \\ \hline
CDSCN2 & NFIN dependence of CDSC & --- & 100000 \\ \hline
CDSP & Constant drain-to-source fringe capacitance (all CGEOMOD) & F & 0 \\ \hline
CFD & Outer fringe capacitance at drain side & --- & 0 \\ \hline
CFS & Outer fringe capacitance at source side & --- & 2.5e-11 \\ \hline
CGBL & Bias-dependent component of gate-to-substrate overlap capacitance per unit channel length per fin per finger & --- & 0 \\ \hline
CGBN & Gate-to-substrate overlap capacitance per unit channel length per fin per finger & --- & 0 \\ \hline
CGBO & Gate-to-substrate overlap capacitance per unit channel length per finger per NGCON & --- & 0 \\ \hline
CGBW & GAA gate-to-substrate overlap capacitance per unit area per fin per finger & --- & 0 \\ \hline
CGDL & Overlap capacitance between gate and lightly-doped drain region (CGEOMOD = 0, 2, 3) & --- & 0 \\ \hline
CGDO & User-designated non-LDD region drain-gate overlap capacitance per unit channel width & --- & 0 \\ \hline
CGDP & Constant gate-to-drain fringe capacitance (CGEOMOD = 1) & --- & 0 \\ \hline
CGEO1SW & For CGEOMOD = 1 only, this switch enables the parameters COVS, COVD, CGSP, and CGDP to be in F per fin, per gate-finger, per unit channel width & --- & 0 \\ \hline
CGEOA & Fitting parameter for CGEOMOD = 2 and 3 & --- & 1 \\ \hline
CGEOB & Fitting parameter for CGEOMOD = 2 and 3 & --- & 0 \\ \hline
CGEOC & Fitting parameter for CGEOMOD = 2 and 3 & --- & 0 \\ \hline
CGEOD & Fitting parameter for CGEOMOD = 2 and 3 & --- & 0 \\ \hline
CGEOE & Fitting parameter for CGEOMOD = 2 and 3 & --- & 1 \\ \hline
CGEOMOD & Geometry-dependent parasitic capacitance model selector & --- & 0 \\ \hline
CGIDL & Parameter for body-effect of GIDL & --- & 0.5 \\ \hline
CGIDLB & Parameter for body-effect of parasitic substrate GIDL & --- & 0.5 \\ \hline
CGISL & Parameter for body-effect of GISL & --- & 0 \\ \hline
CGISLB & Parameter for body-effect of parasitic substrate GISL & --- & 0 \\ \hline
CGSL & Overlap capacitance between gate and lightly-doped source region (CGEOMOD = 0, 2, 3) & --- & 0 \\ \hline
CGSO & User-designated non-LDD region source-gate overlap capacitance per unit channel width & --- & 0 \\ \hline
CGSP & Constant gate-to-source fringe capacitance (CGEOMOD = 1) & --- & 0 \\ \hline
CHARGEWF & Average channel charge weighting factor, 1: source-side, 0: middle, -1: drain-side & --- & 0 \\ \hline
CIGBACC & Parameter for Igb in accumulation & V$^{-1}$ & 0.075 \\ \hline
CIGBINV & Parameter for Igb in inversion & V$^{-1}$ & 0.006 \\ \hline
CIGC & Parameter for Igc in inversion & V$^{-1}$ & 0.075 \\ \hline
CIGD & Parameter for Igd in inversion & V$^{-1}$ & 0 \\ \hline
CIGS & Parameter for Igs in inversion & V$^{-1}$ & 0.075 \\ \hline
CINS\_UFCM & Insulator capacitance for the unified model & --- & 1 \\ \hline
CIT & Parameter for interface traps & --- & 0 \\ \hline
CITR & Parameter for interface traps in reverse mode for asymmetric model & --- & 0 \\ \hline
CJD & Unit area drain-side junction capacitance at zero bias & --- & 0 \\ \hline
CJS & Unit area source-side junction capacitance at zero bias & --- & 0.0005 \\ \hline
CJSWD & Unit length drain-side sidewall junction capacitance at zero bias & --- & 0 \\ \hline
CJSWGD & Unit length drain-side gate sidewall junction capacitance at zero bias & --- & 0 \\ \hline
CJSWGS & Unit length source-side gate sidewall junction capacitance at zero bias & --- & 0 \\ \hline
CJSWS & Unit length source-side sidewall junction capacitance at zero bias & --- & 5e-10 \\ \hline
CKAPPAB & Bias-dependent gate-to-substrate parasitic capacitance & V & 0.6 \\ \hline
CKAPPAD & Coefficient of bias-dependent overlap capacitance for the drain side (CGEOMOD = 0, 2, 3) & V & 0 \\ \hline
CKAPPAS & Coefficient of bias-dependent overlap capacitance for the source side (CGEOMOD = 0, 2, 3) & V & 0.6 \\ \hline
COVD & Constant gate-to-drain overlap capacitance (CGEOMOD = 1) & --- & 0 \\ \hline
COVS & Constant gate-to-source overlap capacitance (CGEOMOD = 1) & --- & 0 \\ \hline
CRATIO & Ratio of the corner area filled with silicon to the total corner area & --- & 0.5 \\ \hline
CRYOMOD & 0: Turn off cryogenic temperature model; 1: Most physical cryogenic temperature models; 2: Cryogenic models converge to BSIM-CMG 111.1.0 temperature models for T > 210 K & --- & 0 \\ \hline
CSDESW & Coefficient for source/drain-to-substrate sidewall capacitance & --- & 0 \\ \hline
CTH0 & Thermal capacitance & --- & 1e-05 \\ \hline
CVMOD & 0: Consistent I-V and C-V, 1: Decoupled I-V and C-V & --- & 0 \\ \hline
D & Diameter of the cylinder (GEOMOD = 3) & m & 4e-08 \\ \hline
DACH1 & Total area correction for 1st GAA body & m$^{2}$ & 0 \\ \hline
DACH2 & Total area correction for 2nd GAA body & m$^{2}$ & 0 \\ \hline
DACH3 & Total area correction for 3rd GAA body & m$^{2}$ & 0 \\ \hline
DACH4 & Total area correction for 4th GAA body & m$^{2}$ & 0 \\ \hline
DACH5 & Total area correction for 5th GAA body & m$^{2}$ & 0 \\ \hline
DACH6 & Total area correction for 6th GAA body & m$^{2}$ & 0 \\ \hline
DELTAPRSD & Change in silicon/silicide interface length due to non-rectangular epi & m & 0 \\ \hline
DELTAVSAT & velocity saturation parameter in the linear region & --- & 1 \\ \hline
DELTAVSATCV & Velocity saturation parameter in the linear region for the capacitance model & --- & 0 \\ \hline
DELTAW & Change of effective width due to shape of fin/cylinder & m & 0 \\ \hline
DELTAWCV & CV change of effective width due to shape of fin/cylinder & m & 0 \\ \hline
DELVFBACC & Change in flatband voltage: Vfb\_accumulation - Vfb\_inversion & V & 0 \\ \hline
DELVTRAND & Variability in Vth & V & 0 \\ \hline
DIM1H & Max dimension for 1st subband (real number enables optimization) & --- & 3 \\ \hline
DIM2H & Max dimension for 2nd subband (real number enables optimization) & --- & 3 \\ \hline
DIM3H & Max dimension for 3rd subband (real number enables optimization) & --- & 3 \\ \hline
DIMENSION1 & Dimension for 1st subband (real number enables optimization) & --- & 2 \\ \hline
DIMENSION2 & Dimension for 2nd subband (real number enables optimization) & --- & 2.6 \\ \hline
DIMENSION3 & Dimension for 3rd subband (real number enables optimization) & --- & 2.6 \\ \hline
DLBIN & Delta L for binning & m & 0 \\ \hline
DLC & Delta L for C-V model & m & 0 \\ \hline
DLCACC & Delta L for C-V model in accumulation region (BULKMOD = 1 or 2) & m & 0 \\ \hline
DLCIGD & Delta L for Igd model & m & 0 \\ \hline
DLCIGS & Delta L for Igs model & m & 0 \\ \hline
DMOBCLAMP & Minimum clamp on Dmob & m & 0.01 \\ \hline
DROUT & L dependence of DIBL effect on Rout & --- & 1.06 \\ \hline
DSSP1 & Change in SSP1 with WGAA scaling for large WGAA>WSSP0 & --- & 2 \\ \hline
DSSP2 & Change in SSP2 with WGAA scaling for large WGAA>WSSP0 & m & 0 \\ \hline
DSSP3 & Change in SPP3 with WGAA scaling for large WGAA>WSSP0 & --- & 0 \\ \hline
DSUB & DIBL exponent coefficient & --- & 1.06 \\ \hline
DTEMP & Variability in device temperature & $^\circ$C & 0 \\ \hline
DTLOW & CRYOMOD != 0 smoothing parameter for TLOW & K & 1 \\ \hline
DTLOW1 & CRYOMOD != 0 smoothing parameter for TLOW1 & K & 0 \\ \hline
DVSATCLAMP & Minimum clamp on Dvsat & m & 0.01 \\ \hline
DVT0 & SCE coefficient & --- & 0 \\ \hline
DVT1 & SCE exponent coefficient. After binning it should be within (0 : inf) & --- & 0.6 \\ \hline
DVT1SS & Subthreshold swing exponent coefficient. After binning it should be within (0 : inf) & --- & 0 \\ \hline
DVTP0 & Coefficient for drain-induced Vth shift (DITS) & --- & 0 \\ \hline
DVTP1 & DITS exponent coefficient & --- & 0 \\ \hline
DVTP2 & DITS model parameter & --- & 0 \\ \hline
DVTSHIFT & Vth shift handle & V & 0 \\ \hline
DVTSHIFTR & Vth shift handle for asymmetric mode & V & 0 \\ \hline
DWBIN & GAA Delta W for binning & m & 0 \\ \hline
DWCACC & GAA delta W for C-V model in accumulation region (BULKMOD = 1 or 2) & m & 0 \\ \hline
DWS1 & Total width correction for 1st GAA body & m & 0 \\ \hline
DWS2 & Total width correction for 2nd GAA body & m & 0 \\ \hline
DWS3 & Total width correction for 3rd GAA body & m & 0 \\ \hline
DWS4 & Total width correction for 4th GAA body & m & 0 \\ \hline
DWS5 & Total width correction for 5th GAA body & m & 0 \\ \hline
DWS6 & Total width correction for 6th GAA body & m & 0 \\ \hline
E2NOM & 2nd subband energy for nominal WGAA & --- & 0.139 \\ \hline
E3NOM & 3rd subband energy for nominal WGAA & --- & 2 \\ \hline
EASUB & Electron affinity of substrate & --- & 4.05 \\ \hline
EF & Flicker noise frequency exponent & --- & 1 \\ \hline
EGBULK & Bulk band-gap & --- & 1.1 \\ \hline
EGIDL & Band bending parameter for GIDL & V & 0.2 \\ \hline
EGIDLB & Band bending parameter for parasitic substrate GIDL & V & 0.2 \\ \hline
EGISL & Band bending parameter for GISL & V & 0 \\ \hline
EGISLB & Band bending parameter for parasitic substrate GISL & V & 0 \\ \hline
EIGBINV & Parameter for Igb in inversion & V & 1.1 \\ \hline
EM & Flicker noise parameter & --- & 4.1e+07 \\ \hline
EMOBT & Temperature coefficient of ETAMOB & --- & 0 \\ \hline
EOT & Equivalent oxide thickness & m & 1e-09 \\ \hline
EOTACC & Equivalent oxide thickness for accumulation region & m & 0 \\ \hline
EOTBOX & Equivalent oxide thickness of the buried oxide (SOI FinFET) & m & 1.4e-07 \\ \hline
EPSROX & Relative dielectric constant of the gate dielectric & --- & 3.9 \\ \hline
EPSRSP & Relative dielectric constant of the spacer & --- & 3.9 \\ \hline
EPSRSUB & Relative dielectric constant of the channel material & --- & 11.9 \\ \hline
ESATII & Saturation channel E-field for Iii & --- & 1e+07 \\ \hline
ETA0 & DIBL coefficient & --- & 0.6 \\ \hline
ETA0CV & CVMOD = 1 DIBL coefficient & --- & 0 \\ \hline
ETA0LT & Coupled NFIN and length dependence of ETA0 & --- & 0 \\ \hline
ETA0LTCV & Coupled NFIN and length dependence of ETA0CV & --- & 0 \\ \hline
ETA0N1 & NFIN dependence of ETA0 & --- & 0 \\ \hline
ETA0N1CV & NFIN dependence of ETA0CV & --- & 0 \\ \hline
ETA0N2 & NFIN dependence of ETA0 & --- & 100000 \\ \hline
ETA0N2CV & NFIN dependence of ETA0CV & --- & 0 \\ \hline
ETA0R & Reverse-mode DIBL coefficient & --- & 0 \\ \hline
ETAMOB & Effective field parameter & --- & 2 \\ \hline
ETAMOBIR & Ideality parameter & --- & 0.1 \\ \hline
ETAMOBTHIN & Effective field parameter for thin GAA bodies & --- & 0 \\ \hline
ETAMOBTNI & Critical TGAA for non-ideality & m & 7.5e-09 \\ \hline
EU & Phonon/surface roughness scattering parameter & --- & 2.5 \\ \hline
EU1 & CRYOMOD != 0 mobility temperature coefficient for EU & --- & -0.001 \\ \hline
EUIR & Ideality parameter & --- & 0.2 \\ \hline
EUPTSC & Exponent for TGAA scaling power law & --- & 3.5 \\ \hline
EUR & Reverse-mode phonon/surface roughness scattering parameter & --- & 0 \\ \hline
EUTHIN & Phonon/surface roughness scattering parameter for thin GAA bodies & --- & 0 \\ \hline
EUTNI & Critical TGAA for non-ideality & m & 6e-09 \\ \hline
FNMOD & 0: Old flicker noise model; 1: New flicker noise model & --- & 0 \\ \hline
FPITCH & Fin pitch & m & 8e-08 \\ \hline
GAVSRD & Parameter for Isat\_rd variation with drain voltage & --- & 0 \\ \hline
GEOMOD & 0: Double gate; 1: Triple gate; 2: Quadruple gate; 3: Cylindrical gate; 4: Unified fin Shape; 5: GAAFETs & --- & 0 \\ \hline
GIDLMOD & 0: Turn off GIDL/GISL current; 1: Turn on GIDL/GISL current; 2: Turn on GIDL/GISL with parasitic substrate component for GEOMOD = 2 or 3 or 5 and BULKMOD != 0 & --- & 0 \\ \hline
HEPI & Height of the raised source/drain on top of the fin & m & 1e-08 \\ \hline
HFIN & Fin height & m & 3e-08 \\ \hline
HPFF & Fin-height of parasitic finFET & m & 5e-09 \\ \hline
IDS0MULT & Variability in drain current for miscellaneous reasons & --- & 1 \\ \hline
IGB0MULT & Gate to body current scale factor & --- & 1 \\ \hline
IGBACCCLAMP & Clamping value of the exponent for Igb in accumulation & --- & 0.001 \\ \hline
IGBINVCLAMP & Clamping value of the exponent for Igb in inversion & --- & 0.001 \\ \hline
IGBMOD & 0: Turn off Igb; 1: Turn on Igb & --- & 0 \\ \hline
IGC0MULT & Gate to channel current scale factor & --- & 1 \\ \hline
IGCINVCLAMP & Clamping value of the exponent for Igc in inversion & --- & 0.0005 \\ \hline
IGCLAMP & 0: Disable gate current clamps; 1: Enable gate current clamps & --- & 1 \\ \hline
IGCMOD & 0: Turn off Igc, Igs and Igd; 1: Turn on Igc, Igs and Igd & --- & 0 \\ \hline
IGT & Gate current temperature dependence & --- & 2.5 \\ \hline
IIMOD & 0: Turn off impact ionization current; 1: BSIM4-based model; 2: BSIMSOI-based model & --- & 0 \\ \hline
IIMOD2CLAMP1 & Clamp1 of SII1 * Vg term in IIMOD = 2 & V & 0.1 \\ \hline
IIMOD2CLAMP2 & Clamp2 of SII0 * Vg term in IIMOD = 2 & V & 0.1 \\ \hline
IIMOD2CLAMP3 & Clamp3 of ratio term in IIMOD = 2 & --- & 0.1 \\ \hline
IIT & Impact ionization temperature dependence for IIMOD = 1 & --- & -0.5 \\ \hline
IJTHDFWD & Forward drain diode breakdown limiting current & A & 0 \\ \hline
IJTHDREV & Reverse drain diode breakdown limiting current & A & 0 \\ \hline
IJTHSFWD & Forward source diode breakdown limiting current & A & 0.1 \\ \hline
IJTHSREV & Reverse source diode breakdown limiting current & A & 0.1 \\ \hline
IMIN & Parameter for Vgs clamping for inversion region calculation in accumulation & --- & 1e-15 \\ \hline
JSD & Bottom drain junction reverse saturation current density & --- & 0 \\ \hline
JSS & Bottom source junction reverse saturation current density & --- & 0.0001 \\ \hline
JSWD & Unit length reverse saturation current for sidewall drain junction & --- & 0 \\ \hline
JSWGD & Unit length reverse saturation current for gate-edge sidewall drain junction & --- & 0 \\ \hline
JSWGS & Unit length reverse saturation current for gate-edge sidewall source junction & --- & 0 \\ \hline
JSWS & Unit length reverse saturation current for sidewall source junction & --- & 0 \\ \hline
JTSD & Bottom drain junction trap-assisted saturation current density & --- & 0 \\ \hline
JTSS & Bottom source junction trap-assisted saturation current density & --- & 0 \\ \hline
JTSSWD & Unit length trap-assisted saturation current for sidewall drain junction & --- & 0 \\ \hline
JTSSWGD & Unit length trap-assisted saturation current for gate-edge sidewall drain junction & --- & 0 \\ \hline
JTSSWGS & Unit length trap-assisted saturation current for gate-edge sidewall source junction & --- & 0 \\ \hline
JTSSWS & Unit length trap-assisted saturation current for sidewall source junction & --- & 0 \\ \hline
JTWEFF & Trap-assisted tunneling current width dependence & m & 0 \\ \hline
K0 & Lateral NUD voltage parameter & V & 0 \\ \hline
K01 & Temperature dependence of lateral NUD voltage parameter & V/K & 0 \\ \hline
K0SI & Correction factor for strong inversion used in Mnud. After binning it should be within (0 : inf) & --- & 1 \\ \hline
K0SI1 & Temperature dependence of K0SI & --- & 0 \\ \hline
K0SISAT & Correction factor for strong inversion used in Mnud & --- & 0 \\ \hline
K0SISAT1 & Temperature dependence of K0SISAT & --- & 0 \\ \hline
K1 & Body effect coefficient for subthreshold region & --- & 1e-06 \\ \hline
K11 & Temperature dependence of K1 & --- & 0 \\ \hline
K1RSCE & K1 for reverse short channel effect calculation & --- & 0 \\ \hline
K2 & Body effect coefficient for BULKMOD = 2 & --- & 0 \\ \hline
K21 & Temperature dependence of K2 & --- & 0 \\ \hline
K2SAT & Correction factor for K2 in saturation (high Vds) & --- & 0 \\ \hline
K2SAT1 & Temperature dependence of K2SAT & --- & 0 \\ \hline
K2SI & Correction factor for strong inversion used in Mob & --- & 0 \\ \hline
K2SI1 & Temperature dependence of K2SI & --- & 0 \\ \hline
K2SISAT & Correction factor for strong inversion used in Mob & --- & 0 \\ \hline
K2SISAT1 & Temperature dependence of K2SISAT & --- & 0 \\ \hline
KLOW1 & CRYOMOD != 0 slope magnitude of effective temperature below TLOW1 & --- & 0 \\ \hline
KSATIV & Parameter for long channel Vdsat & --- & 1 \\ \hline
KSATIVR & KSATIV in asymmetric mode & --- & 0 \\ \hline
KSATIVT1 & CRYOMOD != 0 temperature coefficient for KSATIV & --- & -0.0002 \\ \hline
KSATIVT2 & CRYOMOD != 0 temperature coefficient for KSATIV & --- & -2e-07 \\ \hline
KT1 & Vth temperature coefficient & V & 0 \\ \hline
KT11 & CRYOMOD != 0 Vth temperature coefficient & V & 0.01 \\ \hline
KT12 & CRYOMOD != 0 Vth temperature coefficient & --- & 0.1 \\ \hline
KT1L & Vth temperature L coefficient & --- & 0 \\ \hline
L & Designed gate length & m & 3e-08 \\ \hline
LA1 & L-term of A1 & --- & 0 \\ \hline
LA11 & L-term of A11 & --- & 0 \\ \hline
LA2 & L-term of A2 & m/V & 0 \\ \hline
LA21 & L-term of A21 & --- & 0 \\ \hline
LAGIDL & L-term of AGIDL & --- & 0 \\ \hline
LAGIDLB & L-term of AGIDLB & --- & 0 \\ \hline
LAGISL & L-term of AGISL & --- & 0 \\ \hline
LAGISLB & L-term of AGISLB & --- & 0 \\ \hline
LAIGBACC & L-term of AIGBACC & --- & 0 \\ \hline
LAIGBACC1 & L-term of AIGBACC1 & --- & 0 \\ \hline
LAIGBINV & L-term of AIGBINV & --- & 0 \\ \hline
LAIGBINV1 & L-term of AIGBINV1 & --- & 0 \\ \hline
LAIGC & L-term of AIGC & --- & 0 \\ \hline
LAIGC1 & L-term of AIGC1 & --- & 0 \\ \hline
LAIGD & L-term of AIGD & --- & 0 \\ \hline
LAIGD1 & L-term of AIGD1 & --- & 0 \\ \hline
LAIGEN & L-term of AIGEN & --- & 0 \\ \hline
LAIGS & L-term of AIGS & --- & 0 \\ \hline
LAIGS1 & L-term of AIGS1 & --- & 0 \\ \hline
LALPHA0 & L-term of ALPHA0 & --- & 0 \\ \hline
LALPHA1 & L-term of ALPHA1 & m/V & 0 \\ \hline
LALPHAII0 & L-term of ALPHAII0 & --- & 0 \\ \hline
LALPHAII1 & L-term of ALPHAII1 & m/V & 0 \\ \hline
LAT & L-term of AT & --- & 0 \\ \hline
LATCV & L-term of ATCV & --- & 0 \\ \hline
LATR & L-term of ATR & --- & 0 \\ \hline
LBETA0 & L-term of BETA0 & m/V & 0 \\ \hline
LBETAII0 & L-term of BETAII0 & m/V & 0 \\ \hline
LBETAII1 & L-term of BETAII1 & m & 0 \\ \hline
LBETAII2 & L-term of BETAII2 & --- & 0 \\ \hline
LBGIDL & L-term of BGIDL & V & 0 \\ \hline
LBGIDLB & L-term of BGIDLB & V & 0 \\ \hline
LBGISL & L-term of BGISL & V & 0 \\ \hline
LBGISLB & L-term of BGISLB & V & 0 \\ \hline
LBIGBACC & L-term of BIGBACC & --- & 0 \\ \hline
LBIGBINV & L-term of BIGBINV & --- & 0 \\ \hline
LBIGC & L-term of BIGC & --- & 0 \\ \hline
LBIGD & L-term of BIGD & --- & 0 \\ \hline
LBIGEN & L-term of BIGEN & --- & 0 \\ \hline
LBIGS & L-term of BIGS & --- & 0 \\ \hline
LCDSC & L-term of CDSC & --- & 0 \\ \hline
LCDSCD & L-term of CDSCD & --- & 0 \\ \hline
LCDSCDR & L-term of CDSCDR & --- & 0 \\ \hline
LCFD & L-term of CFD & F & 0 \\ \hline
LCFS & L-term of CFS & F & 0 \\ \hline
LCGBL & L-term of CGBL & F & 0 \\ \hline
LCGDL & L-term of CGDL & F & 0 \\ \hline
LCGIDL & L-term of CGIDL & --- & 0 \\ \hline
LCGIDLB & L-term of CGIDLB & --- & 0 \\ \hline
LCGISL & L-term of CGISL & --- & 0 \\ \hline
LCGISLB & L-term of CGISLB & --- & 0 \\ \hline
LCGSL & L-term of CGSL & F & 0 \\ \hline
LCIGBACC & L-term of CIGBACC & m/V & 0 \\ \hline
LCIGBINV & L-term of CIGBINV & m/V & 0 \\ \hline
LCIGC & L-term of CIGC & m/V & 0 \\ \hline
LCIGD & L-term of CIGD & m/V & 0 \\ \hline
LCIGS & L-term of CIGS & m/V & 0 \\ \hline
LCIT & L-term of CIT & --- & 0 \\ \hline
LCITR & L-term of CITR & --- & 0 \\ \hline
LCKAPPAB & L-term of CKAPPAB & --- & 0 \\ \hline
LCKAPPAD & L-term of CKAPPAD & --- & 0 \\ \hline
LCKAPPAS & L-term of CKAPPAS & --- & 0 \\ \hline
LCOVD & L-term of COVD & F & 0 \\ \hline
LCOVS & L-term of COVS & F & 0 \\ \hline
LDELTAVSAT & L-term of DELTAVSAT & m & 0 \\ \hline
LDELTAVSATCV & L-term of DELTAVSATCV & m & 0 \\ \hline
LDIMENSION1 & L-term of DIMENSION1 & m & 0 \\ \hline
LDIMENSION2 & L-term of DIMENSION2 & m & 0 \\ \hline
LDIMENSION3 & L-term of DIMENSION3 & m & 0 \\ \hline
LDLBIN & L-term of DLBIN & m$^{2}$ & 0 \\ \hline
LDROUT & L-term of DROUT & m & 0 \\ \hline
LDSUB & L-term of DSUB & m & 0 \\ \hline
LDVT0 & L-term of DVT0 & m & 0 \\ \hline
LDVT1 & L-term of DVT1 & m & 0 \\ \hline
LDVT1SS & L-term of DVT1SS & m & 0 \\ \hline
LDVTP0 & L-term of DVTP0 & m & 0 \\ \hline
LDVTP1 & L-term of DVTP1 & m & 0 \\ \hline
LDVTSHIFT & L-term of DVTSHIFT & --- & 0 \\ \hline
LDVTSHIFTR & L-term of DVTSHIFTR & --- & 0 \\ \hline
LDWBIN & L-term of DWBIN & m$^{2}$ & 0 \\ \hline
LE2NOM & L-term of E2NOM & --- & 0 \\ \hline
LE3NOM & L-term of E3NOM & --- & 0 \\ \hline
LEGIDL & L-term of EGIDL & --- & 0 \\ \hline
LEGIDLB & L-term of EGIDLB & --- & 0 \\ \hline
LEGISL & L-term of EGISL & --- & 0 \\ \hline
LEGISLB & L-term of EGISLB & --- & 0 \\ \hline
LEIGBINV & L-term of EIGBINV & --- & 0 \\ \hline
LEMOBT & L-term of EMOBT & m & 0 \\ \hline
LESATII & L-term of ESATII & V & 0 \\ \hline
LETA0 & L-term of ETA0 & m & 0 \\ \hline
LETA0CV & L-term of ETA0CV & m & 0 \\ \hline
LETA0R & L-term of ETA0R & m & 0 \\ \hline
LETAMOB & L-term of ETAMOB & m & 0 \\ \hline
LEU & L-term of EU & --- & 0 \\ \hline
LEU1 & L-term of EU1 & m & 0 \\ \hline
LEUR & L-term of EUR & --- & 0 \\ \hline
LIGT & L-term of IGT & m & 0 \\ \hline
LII & Channel length dependence parameter of Iii & --- & 5e-10 \\ \hline
LIIT & L-term of IIT & m & 0 \\ \hline
LINT & Length reduction parameter (dopant diffusion effect) & m & 0 \\ \hline
LINTIGEN & Lint for thermal generation current & m & 0 \\ \hline
LINTNOI & L offset for flicker noise calculation & m & 0 \\ \hline
LK0 & L-term of K0 & --- & 0 \\ \hline
LK01 & L-term of K01 & --- & 0 \\ \hline
LK0SI & L-term of K0SI & m & 0 \\ \hline
LK0SI1 & L-term of K0SI1 & --- & 0 \\ \hline
LK0SISAT & L-term of K0SISAT & m & 0 \\ \hline
LK0SISAT1 & L-term of K0SISAT1 & m & 0 \\ \hline
LK1 & L-term of K1 & --- & 0 \\ \hline
LK11 & L-term of K11 & --- & 0 \\ \hline
LK1RSCE & L-term of K1RSCE & --- & 0 \\ \hline
LK2 & L-term of K2 & m & 0 \\ \hline
LK21 & L-term of K21 & m & 0 \\ \hline
LK2SAT & L-term of K2SAT & m & 0 \\ \hline
LK2SAT1 & L-term of K2SAT1 & m & 0 \\ \hline
LK2SI & L-term of K2SI & m & 0 \\ \hline
LK2SI1 & L-term of K2SI1 & --- & 0 \\ \hline
LK2SISAT & L-term of K2SISAT & m & 0 \\ \hline
LK2SISAT1 & L-term of K2SISAT1 & m & 0 \\ \hline
LKSATIV & L-term of KSATIV & m & 0 \\ \hline
LKSATIVR & L-term of KSATIVR & m & 0 \\ \hline
LKT1 & L-term of KT1 & --- & 0 \\ \hline
LL & Length reduction parameter (dopant diffusion effect) & --- & 0 \\ \hline
LLC & Length reduction parameter (dopant diffusion effect) & --- & 0 \\ \hline
LLII & L-term of LII & --- & 0 \\ \hline
LLINT & L-term of LINT & m$^{2}$ & 0 \\ \hline
LLN & Length reduction parameter (dopant diffusion effect) & --- & 1 \\ \hline
LLPE0 & L-term of LPE0 & m$^{2}$ & 0 \\ \hline
LMAX & Maximum length for which this model should be used & m & 100 \\ \hline
LMEXP & L-term of MEXP & m & 0 \\ \hline
LMEXPR & L-term of MEXPR & m & 0 \\ \hline
LMFQ1NOM & L-term of MFQ1NOM & m & 0 \\ \hline
LMFQ2NOM & L-term of MFQ2NOM & m & 0 \\ \hline
LMFQ3NOM & L-term of MFQ3NOM & m & 0 \\ \hline
LMIN & Minimum length for which this model should be used & m & 0 \\ \hline
LMPOWER & L-term for MPOWER & m & 0 \\ \hline
LNBODY & L-term of NBODY & --- & 0 \\ \hline
LNGATE & L-term of NGATE & --- & 0 \\ \hline
LNIGBACC & L-term of NIGBACC & m & 0 \\ \hline
LNIGBINV & L-term of NIGBINV & m & 0 \\ \hline
LNOIA2 & L-term for NOIA2 & --- & 0 \\ \hline
LNTGEN & L-term of NTGEN & m & 0 \\ \hline
LNTOX & L-term of NTOX & m & 0 \\ \hline
LPA & Mobility L power coefficient & --- & 1 \\ \hline
LPAR & Reverse-mode mobility L power coefficient & --- & 0 \\ \hline
LPCLM & L-term of PCLM & m & 0 \\ \hline
LPCLMCV & L-term of PCLMCV & m & 0 \\ \hline
LPCLMG & L-term of PCLMG & m/V & 0 \\ \hline
LPCLMR & L-term of PCLMR & m & 0 \\ \hline
LPDIBL1 & L-term of PDIBL1 & m & 0 \\ \hline
LPDIBL1R & L-term of PDIBL1R & m & 0 \\ \hline
LPDIBL2 & L-term of PDIBL2 & m & 0 \\ \hline
LPDIBL2R & L-term of PDIBL2R & m & 0 \\ \hline
LPE0 & Equivalent length of pocket region at zero bias & m & 5e-09 \\ \hline
LPGIDL & L-term of PGIDL & m & 0 \\ \hline
LPGIDLB & L-term of PGIDLB & m & 0 \\ \hline
LPGISL & L-term of PGISL & m & 0 \\ \hline
LPGISLB & L-term of PGISLB & m & 0 \\ \hline
LPHIBE & L-term of PHIBE & --- & 0 \\ \hline
LPHIG & L-term of PHIG & --- & 0 \\ \hline
LPHIN & L-term of PHIN & --- & 0 \\ \hline
LPIGCD & L-term of PIGCD & m & 0 \\ \hline
LPOXEDGE & L-term of POXEDGE & m & 0 \\ \hline
LPRT & L-term of PRT & --- & 0 \\ \hline
LPRT1 & L-term of PRT1 & --- & 0 \\ \hline
LPRWGD & L-term of PRWGD & m/V & 0 \\ \hline
LPRWGS & L-term of PRWGS & m/V & 0 \\ \hline
LPSAT & L-term of PSAT & m & 0 \\ \hline
LPSATCV & L-term of PSATCV & m & 0 \\ \hline
LPTWG & L-term of PTWG & --- & 0 \\ \hline
LPTWGR & L-term of PTWGR & --- & 0 \\ \hline
LPTWGT & L-term of PTWGT & --- & 0 \\ \hline
LPVAG & L-term of PVAG & m & 0 \\ \hline
LQMFACTOR & L-term of QMFACTOR & m & 0 \\ \hline
LQMTCENCV & L-term of QMTCENCV & m & 0 \\ \hline
LQMTCENCVA & L-term of QMTCENCVA & m & 0 \\ \hline
LQSREF & L-term for QSREF & m & 0 \\ \hline
LRDSW & L-term of RDSW & --- & 0 \\ \hline
LRDW & L-term of RDW & --- & 0 \\ \hline
LRSD & Length of the source/drain & m & 0 \\ \hline
LRSW & L-term of RSW & --- & 0 \\ \hline
LSII0 & L-term of SII0 & m/V & 0 \\ \hline
LSII1 & L-term of SII1 & m & 0 \\ \hline
LSII2 & L-term of SII2 & --- & 0 \\ \hline
LSIID & L-term of SIID & --- & 0 \\ \hline
LSP & Thickness of the gate sidewall spacer & m & 6e-09 \\ \hline
LSPRT & L-term of SPRT & m & 0 \\ \hline
LSSP1 & L-term of SSP1 & m & 0 \\ \hline
LSSP2 & L-term of SSP2 & m & 0 \\ \hline
LSSP3 & L-term of SSP3 & m & 0 \\ \hline
LTGIDL & L-term of TGIDL & --- & 0 \\ \hline
LTII & L-term of TII & m & 0 \\ \hline
LTR0 & L-term of TR0 & --- & 0 \\ \hline
LTSS & L-term of TSS & --- & 0 \\ \hline
LU0 & L-term of U0 & --- & 0 \\ \hline
LU0CV & L-term of U0CV & --- & 0 \\ \hline
LU0R & L-term of U0R & --- & 0 \\ \hline
LUA & L-term of UA & --- & 0 \\ \hline
LUA1 & L-term of UA1 & m & 0 \\ \hline
LUA1CV & L-term of UA1CV & m & 0 \\ \hline
LUA1R & L-term of UA1R & m & 0 \\ \hline
LUA2 & L-term of UA2 & m & 0 \\ \hline
LUA2CV & L-term of UA2CV & m & 0 \\ \hline
LUACV & L-term of UACV & --- & 0 \\ \hline
LUAR & L-term of UAR & --- & 0 \\ \hline
LUC & L-term of UC & --- & 0 \\ \hline
LUC1 & L-term of UC1 & m & 0 \\ \hline
LUC1CV & L-term of UC1CV & m & 0 \\ \hline
LUC1R & L-term of UC1R & m & 0 \\ \hline
LUCCV & L-term of UCCV & --- & 0 \\ \hline
LUCR & L-term of UCR & --- & 0 \\ \hline
LUCS & L-term of UCS & m & 0 \\ \hline
LUCSTE & L-term of UCSTE & m & 0 \\ \hline
LUCSTE1 & L-term of UCSTE1 & m & 0 \\ \hline
LUD & L-term of UD & --- & 0 \\ \hline
LUD1 & L-term of UD1 & m & 0 \\ \hline
LUD1CV & L-term of UD1CV & m & 0 \\ \hline
LUD1R & L-term of UD1R & m & 0 \\ \hline
LUD2 & L-term of UD2 & m & 0 \\ \hline
LUD2CV & L-term of UD2CV & m & 0 \\ \hline
LUDCV & L-term of UDCV & --- & 0 \\ \hline
LUDD & L-term of UDD & m & 0 \\ \hline
LUDD1 & L-term of UDD1 & m & 0 \\ \hline
LUDR & L-term of UDR & --- & 0 \\ \hline
LUDS & L-term of UDS & m & 0 \\ \hline
LUDS1 & L-term of UDS1 & m & 0 \\ \hline
LUP & L-term of UP & --- & 0 \\ \hline
LUPR & L-term of UPR & --- & 0 \\ \hline
LUTE & L-term of UTE & m & 0 \\ \hline
LUTE1 & L-term of UTE1 & m & 0 \\ \hline
LUTE1CV & L-term of UTE1CV & m & 0 \\ \hline
LUTECV & L-term of UTECV & m & 0 \\ \hline
LUTER & L-term of UTER & m & 0 \\ \hline
LUTL & L-term of UTL & m & 0 \\ \hline
LUTLCV & L-term of UTLCV & m & 0 \\ \hline
LUTLR & L-term of UTLR & m & 0 \\ \hline
LVSAT & L-term of VSAT & --- & 0 \\ \hline
LVSAT1 & L-term of VSAT1 & --- & 0 \\ \hline
LVSAT1R & L-term of VSAT1R & --- & 0 \\ \hline
LVSATCV & L-term of VSATCV & --- & 0 \\ \hline
LVSATR & L-term of VSATR & --- & 0 \\ \hline
LWR & L-term of WR & m & 0 \\ \hline
LXL & L-term of XL & m$^{2}$ & 0 \\ \hline
LXRCRG1 & L-term of XRCRG1 & m & 0 \\ \hline
LXRCRG2 & L-term of XRCRG2 & m & 0 \\ \hline
LXW & L-term of XW & m$^{2}$ & 0 \\ \hline
MEXP & Smoothing function factor for Vdsat & --- & 4 \\ \hline
MEXPR & Reverse-mode smoothing function factor for Vdsat & --- & 0 \\ \hline
MFE2 & Rate of change in 2nd subband energy with WGAA and TGAA scaling & --- & 1 \\ \hline
MFE3 & Rate of change in 3rd subband energy with WGAA and TGAA scaling & --- & 1 \\ \hline
MFQ1 & Rate of change in 1st subband charge with WGAA and TGAA scaling & --- & 1 \\ \hline
MFQ1NOM & Sacling factor for 1st subband charge for nominal WGAA & --- & 11.2 \\ \hline
MFQ2 & Rate of change in 2nd subband charge with WGAA and TGAA scaling & --- & 1 \\ \hline
MFQ2NOM & Scaling factor for 2nd subband charge for nominal WGAA & --- & 8.02 \\ \hline
MFQ3 & Rate of change in 3rd subband charge with WGAA and TGAA scaling & --- & 1 \\ \hline
MFQ3NOM & Scaling factor for 3rd subband charge for niminal WHSEET & --- & 6.18 \\ \hline
MINR & minr is the value below which the simulator expects elimination of source/drain resitance and it will improve simulation efficiency without significantly altering the results & $\mathsf{\Omega}$ & 0 \\ \hline
MJD & Drain bottom junction capacitance grading coefficient & --- & 0 \\ \hline
MJD2 & Drain bottom two-step second junction capacitance grading coefficient & --- & 0 \\ \hline
MJS & Source bottom junction capacitance grading coefficient & --- & 0.5 \\ \hline
MJS2 & Source bottom two-step second junction capacitance grading coefficient & --- & 0.125 \\ \hline
MJSWD & Drain sidewall junction capacitance grading coefficient & --- & 0 \\ \hline
MJSWD2 & Drain sidewall two-step second junction capacitance grading coefficient & --- & 0 \\ \hline
MJSWGD & Drain-side gate sidewall junction capacitance grading coefficient & --- & 0 \\ \hline
MJSWGD2 & Drain-side gate sidewall two-step second junction capacitance grading coefficient & --- & 0 \\ \hline
MJSWGS & Source-side gate sidewall junction capacitance grading coefficient & --- & 0 \\ \hline
MJSWGS2 & Source-side gate sidewall two-step second junction capacitance grading coefficient & --- & 0 \\ \hline
MJSWS & Source sidewall junction capacitance grading coefficient & --- & 0.33 \\ \hline
MJSWS2 & Source sidewall two-step second junction capacitance grading coefficient & --- & 0.083 \\ \hline
MOBSCMOD & Switch for GAAFET geometry dependent mobility model (0: off; 1: on) & --- & 0 \\ \hline
MPOWER & Sub-threshold to strong inversion transition slope Parameter & --- & 1.2 \\ \hline
MVSRSD & Non-linear resistance parameter & --- & 1 \\ \hline
NA1 & N-term of A1 & --- & 0 \\ \hline
NA11 & N-term of A11 & --- & 0 \\ \hline
NA2 & N-term of A2 & V$^{-1}$ & 0 \\ \hline
NA21 & N-term of A21 & --- & 0 \\ \hline
NAGIDL & N-term of AGIDL & --- & 0 \\ \hline
NAGIDLB & N-term of AGIDLB & --- & 0 \\ \hline
NAGISL & N-term of AGISL & --- & 0 \\ \hline
NAGISLB & N-term of AGISLB & --- & 0 \\ \hline
NAIGBACC & N-term of AIGBACC & --- & 0 \\ \hline
NAIGBACC1 & N-term of AIGBACC1 & --- & 0 \\ \hline
NAIGBINV & N-term of AIGBINV & --- & 0 \\ \hline
NAIGBINV1 & N-term of AIGBINV1 & --- & 0 \\ \hline
NAIGC & N-term of AIGC & --- & 0 \\ \hline
NAIGC1 & N-term of AIGC1 & --- & 0 \\ \hline
NAIGD & N-term of AIGD & --- & 0 \\ \hline
NAIGD1 & N-term of AIGD1 & --- & 0 \\ \hline
NAIGEN & N-term of AIGEN & --- & 0 \\ \hline
NAIGS & N-term of AIGS & --- & 0 \\ \hline
NAIGS1 & N-term of AIGS1 & --- & 0 \\ \hline
NALPHA0 & N-term of ALPHA0 & m/V & 0 \\ \hline
NALPHA1 & N-term of ALPHA1 & V$^{-1}$ & 0 \\ \hline
NALPHAII0 & N-term of ALPHAII0 & m/V & 0 \\ \hline
NALPHAII1 & N-term of ALPHAII1 & V$^{-1}$ & 0 \\ \hline
NAT & N-term of AT & --- & 0 \\ \hline
NATCV & N-term of ATCV & --- & 0 \\ \hline
NATR & N-term of ATR & --- & 0 \\ \hline
NBETA0 & N-term of BETA0 & V$^{-1}$ & 0 \\ \hline
NBETAII0 & N-term of BETAII0 & V$^{-1}$ & 0 \\ \hline
NBETAII1 & N-term of BETAII1 & --- & 0 \\ \hline
NBETAII2 & N-term of BETAII2 & V & 0 \\ \hline
NBGIDL & N-term of BGIDL & --- & 0 \\ \hline
NBGIDLB & N-term of BGIDLB & --- & 0 \\ \hline
NBGISL & N-term of BGISL & --- & 0 \\ \hline
NBGISLB & N-term of BGISLB & --- & 0 \\ \hline
NBIGBACC & N-term of BIGBACC & --- & 0 \\ \hline
NBIGBINV & N-term of BIGBINV & --- & 0 \\ \hline
NBIGC & N-term of BIGC & --- & 0 \\ \hline
NBIGD & N-term of BIGD & --- & 0 \\ \hline
NBIGEN & N-term of BIGEN & --- & 0 \\ \hline
NBIGS & N-term of BIGS & --- & 0 \\ \hline
NBODY & Channel (body) doping & --- & 1e+22 \\ \hline
NBODYN1 & NFIN dependence of channel (body) doping & --- & 0 \\ \hline
NBODYN2 & NFIN dependence of channel (body) doping & --- & 100000 \\ \hline
NC0SUB & Conduction band density of states & --- & 2.86e+25 \\ \hline
NCDSC & N-term of CDSC & --- & 0 \\ \hline
NCDSCD & N-term of CDSCD & --- & 0 \\ \hline
NCDSCDR & N-term of CDSCDR & --- & 0 \\ \hline
NCFD & N-term of CFD & --- & 0 \\ \hline
NCFS & N-term of CFS & --- & 0 \\ \hline
NCGBL & N-term of CGBL & --- & 0 \\ \hline
NCGDL & N-term of CGDL & --- & 0 \\ \hline
NCGIDL & N-term of CGIDL & --- & 0 \\ \hline
NCGIDLB & N-term of CGIDLB & --- & 0 \\ \hline
NCGISL & N-term of CGISL & --- & 0 \\ \hline
NCGISLB & N-term of CGISLB & --- & 0 \\ \hline
NCGSL & N-term of CGSL & --- & 0 \\ \hline
NCIGBACC & N-term of CIGBACC & V$^{-1}$ & 0 \\ \hline
NCIGBINV & N-term of CIGBINV & V$^{-1}$ & 0 \\ \hline
NCIGC & N-term of CIGC & V$^{-1}$ & 0 \\ \hline
NCIGD & N-term of CIGD & V$^{-1}$ & 0 \\ \hline
NCIGS & N-term of CIGS & V$^{-1}$ & 0 \\ \hline
NCIT & N-term of CIT & --- & 0 \\ \hline
NCITR & N-term of CITR & --- & 0 \\ \hline
NCKAPPAB & N-term of CKAPPAB & V & 0 \\ \hline
NCKAPPAD & N-term of CKAPPAD & V & 0 \\ \hline
NCKAPPAS & N-term of CKAPPAS & V & 0 \\ \hline
NCOVD & N-term of COVD & --- & 0 \\ \hline
NCOVS & N-term of COVS & --- & 0 \\ \hline
NDELTAVSAT & N-term of DELTAVSAT & --- & 0 \\ \hline
NDELTAVSATCV & N-term of DELTAVSATCV & --- & 0 \\ \hline
NDIMENSION1 & N-term of DIMENSION1 & --- & 0 \\ \hline
NDIMENSION2 & N-term of DIMENSION2 & --- & 0 \\ \hline
NDIMENSION3 & N-term of DIMENSION3 & --- & 0 \\ \hline
NDLBIN & N-term of DLBIN & m & 0 \\ \hline
NDROUT & N-term of DROUT & --- & 0 \\ \hline
NDSUB & N-term of DSUB & --- & 0 \\ \hline
NDVT0 & N-term of DVT0 & --- & 0 \\ \hline
NDVT1 & N-term of DVT1 & --- & 0 \\ \hline
NDVT1SS & N-term of DVT1SS & --- & 0 \\ \hline
NDVTP0 & N-term of DVTP0 & --- & 0 \\ \hline
NDVTP1 & N-term of DVTP1 & --- & 0 \\ \hline
NDVTSHIFT & N-term of DVTSHIFT & V & 0 \\ \hline
NDVTSHIFTR & N-term of DVTSHIFTR & V & 0 \\ \hline
NDWBIN & N-term of DWBIN & m & 0 \\ \hline
NE2NOM & N-term of E2NOM & --- & 0 \\ \hline
NE3NOM & N-term of E3NOM & --- & 0 \\ \hline
NEGIDL & N-term of EGIDL & V & 0 \\ \hline
NEGIDLB & N-term of EGIDLB & V & 0 \\ \hline
NEGISL & N-term of EGISL & V & 0 \\ \hline
NEGISLB & N-term of EGISLB & V & 0 \\ \hline
NEIGBINV & N-term of EIGBINV & V & 0 \\ \hline
NEMOBT & N-term of EMOBT & --- & 0 \\ \hline
NESATII & N-term of ESATII & --- & 0 \\ \hline
NETA0 & N-term of ETA0 & --- & 0 \\ \hline
NETA0CV & N-term of ETA0CV & --- & 0 \\ \hline
NETA0R & N-term of ETA0R & --- & 0 \\ \hline
NETAMOB & N-term of ETAMOB & --- & 0 \\ \hline
NEU & N-term of EU & --- & 0 \\ \hline
NEU1 & N-term of EU1 & --- & 0 \\ \hline
NEUR & N-term of EUR & --- & 0 \\ \hline
NFIN & Number of fins per finger (real number enables optimization) & --- & 1 \\ \hline
NFINNOM & If non-zero, nominal number of fins per finger & --- & 0 \\ \hline
NGAA & Number of GAA bodies per fin & --- & 1 \\ \hline
NGATE & Parameter for poly gate doping. For metal gate please set NGATE = 0 & --- & 0 \\ \hline
NGCON & Number of gate contact (1 or 2 sided) & --- & 1 \\ \hline
NI0SUB & Intrinsic carrier constant at 300.15K & --- & 1.1e+16 \\ \hline
NIGBACC & Parameter for Igb in accumulation & --- & 1 \\ \hline
NIGBINV & Parameter for Igb in inversion & --- & 3 \\ \hline
NIGT & N-term of IGT & --- & 0 \\ \hline
NIIT & N-term of IIT & --- & 0 \\ \hline
NJD & Drain junction emission coefficient & --- & 0 \\ \hline
NJS & Source junction emission coefficient & --- & 1 \\ \hline
NJTS & Non-ideality factor for JTSS & --- & 20 \\ \hline
NJTSD & Non-ideality factor for JTSD & --- & 0 \\ \hline
NJTSSW & Non-ideality factor for JTSSWS & --- & 20 \\ \hline
NJTSSWD & Non-ideality factor for JTSSWD & --- & 0 \\ \hline
NJTSSWG & Non-ideality factor for JTSSWGS & --- & 20 \\ \hline
NJTSSWGD & Non-ideality factor for JTSSWGD & --- & 0 \\ \hline
NK0 & N-term of K0 & V & 0 \\ \hline
NK01 & N-term of K01 & V/K & 0 \\ \hline
NK0SI & N-term of K0SI & --- & 0 \\ \hline
NK0SI1 & N-term of K0SI1 & --- & 0 \\ \hline
NK0SISAT & N-term of K0SISAT & --- & 0 \\ \hline
NK0SISAT1 & N-term of K0SISAT1 & --- & 0 \\ \hline
NK1 & N-term of K1 & --- & 0 \\ \hline
NK11 & N-term of K11 & --- & 0 \\ \hline
NK1RSCE & N-term of K1RSCE & --- & 0 \\ \hline
NK2 & N-term of K2 & --- & 0 \\ \hline
NK21 & N-term of K21 & --- & 0 \\ \hline
NK2SAT & N-term of K2SAT & --- & 0 \\ \hline
NK2SAT1 & N-term of K2SAT1 & --- & 0 \\ \hline
NK2SI & N-term of K2SI & --- & 0 \\ \hline
NK2SI1 & N-term of K2SI1 & --- & 0 \\ \hline
NK2SISAT & N-term of K2SISAT & --- & 0 \\ \hline
NK2SISAT1 & N-term of K2SISAT1 & --- & 0 \\ \hline
NKSATIV & N-term of KSATIV & --- & 0 \\ \hline
NKSATIVR & N-term of KSATIVR & --- & 0 \\ \hline
NKT1 & N-term of KT1 & V & 0 \\ \hline
NLII & N-term of LII & --- & 0 \\ \hline
NLINT & N-term of LINT & m & 0 \\ \hline
NLPE0 & N-term of LPE0 & m & 0 \\ \hline
NMEXP & N-term of MEXP & --- & 0 \\ \hline
NMEXPR & N-term of MEXPR & --- & 0 \\ \hline
NMFQ1NOM & N-term of MFQ1NOM & --- & 0 \\ \hline
NMFQ2NOM & N-term of MFQ2NOM & --- & 0 \\ \hline
NMFQ3NOM & N-term of MFQ3NOM & --- & 0 \\ \hline
NMPOWER & N-term for MPOWER & --- & 0 \\ \hline
NNBODY & N-term of NBODY & --- & 0 \\ \hline
NNGATE & N-term of NGATE & --- & 0 \\ \hline
NNIGBACC & N-term of NIGBACC & --- & 0 \\ \hline
NNIGBINV & N-term of NIGBINV & --- & 0 \\ \hline
NNOIA2 & N-term for NOIA2 & --- & 0 \\ \hline
NNTGEN & N-term of NTGEN & --- & 0 \\ \hline
NNTOX & N-term of NTOX & --- & 0 \\ \hline
NOIA & Flicker noise parameter & --- & 6.25e+39 \\ \hline
NOIA2 & Flicker noise parameter for sub-threshold region & --- & 0 \\ \hline
NOIB & Flicker noise parameter & --- & 3.125e+24 \\ \hline
NOIC & Flicker noise parameter & --- & 8.75e+07 \\ \hline
NPCLM & N-term of PCLM & --- & 0 \\ \hline
NPCLMCV & N-term of PCLMCV & --- & 0 \\ \hline
NPCLMG & N-term of PCLMG & V$^{-1}$ & 0 \\ \hline
NPCLMR & N-term of PCLMR & --- & 0 \\ \hline
NPDIBL1 & N-term of PDIBL1 & --- & 0 \\ \hline
NPDIBL1R & N-term of PDIBL1R & --- & 0 \\ \hline
NPDIBL2 & N-term of PDIBL2 & --- & 0 \\ \hline
NPDIBL2R & N-term of PDIBL2R & --- & 0 \\ \hline
NPGIDL & N-term of PGIDL & --- & 0 \\ \hline
NPGIDLB & N-term of PGIDLB & --- & 0 \\ \hline
NPGISL & N-term of PGISL & --- & 0 \\ \hline
NPGISLB & N-term of PGISLB & --- & 0 \\ \hline
NPHIBE & N-term of PHIBE & V & 0 \\ \hline
NPHIG & N-term of PHIG & --- & 0 \\ \hline
NPHIN & N-term of PHIN & V & 0 \\ \hline
NPIGCD & N-term of PIGCD & --- & 0 \\ \hline
NPOXEDGE & N-term of POXEDGE & --- & 0 \\ \hline
NPRT & N-term of PRT & --- & 0 \\ \hline
NPRT1 & N-term of PRT1 & --- & 0 \\ \hline
NPRWGD & N-term of PRWGD & V$^{-1}$ & 0 \\ \hline
NPRWGS & N-term of PRWGS & V$^{-1}$ & 0 \\ \hline
NPSAT & N-term of PSAT & --- & 0 \\ \hline
NPSATCV & N-term of PSATCV & --- & 0 \\ \hline
NPTWG & N-term of PTWG & --- & 0 \\ \hline
NPTWGR & N-term of PTWGR & --- & 0 \\ \hline
NPTWGT & N-term of PTWGT & --- & 0 \\ \hline
NPVAG & N-term of PVAG & --- & 0 \\ \hline
NQMFACTOR & N-term of QMFACTOR & --- & 0 \\ \hline
NQMTCENCV & N-term of QMTCENCV & --- & 0 \\ \hline
NQMTCENCVA & N-term of QMTCENCVA & --- & 0 \\ \hline
NQSMOD & 0: Turn off NQS model; 1: NQS gate resistance (with gi node); 2: NQS charge deficit model from BSIM4 (with q node) & --- & 0 \\ \hline
NQSREF & N-term for QSREF & --- & 0 \\ \hline
NRD & Number of drain diffusion squares & --- & 0 \\ \hline
NRDSW & N-term of RDSW & --- & 0 \\ \hline
NRDW & N-term of RDW & --- & 0 \\ \hline
NRS & Number of source diffusion squares & --- & 0 \\ \hline
NRSW & N-term of RSW & --- & 0 \\ \hline
NSD & Source/drain active doping concentration & --- & 2e+26 \\ \hline
NSII0 & N-term of SII0 & V$^{-1}$ & 0 \\ \hline
NSII1 & N-term of SII1 & --- & 0 \\ \hline
NSII2 & N-term of SII2 & V & 0 \\ \hline
NSIID & N-term of SIID & V & 0 \\ \hline
NSPRT & N-term of SPRT & --- & 0 \\ \hline
NSSP1 & N-term of SSP1 & --- & 0 \\ \hline
NSSP2 & N-term of SSP2 & --- & 0 \\ \hline
NSSP3 & N-term of SSP3 & --- & 0 \\ \hline
NTGEN & Thermal generation current parameter & --- & 1 \\ \hline
NTGIDL & N-term of TGIDL & --- & 0 \\ \hline
NTII & N-term of TII & --- & 0 \\ \hline
NTNOI & Thermal noise parameter & --- & 1 \\ \hline
NTOX & Exponent for Tox ratio & --- & 1 \\ \hline
NTR0 & N-term of TR0 & K & 0 \\ \hline
NTSS & N-term of TSS & --- & 0 \\ \hline
NU0 & N-term of U0 & --- & 0 \\ \hline
NU0CV & N-term of U0CV & --- & 0 \\ \hline
NU0R & N-term of U0R & --- & 0 \\ \hline
NUA & N-term of UA & --- & 0 \\ \hline
NUA1 & N-term of UA1 & --- & 0 \\ \hline
NUA1CV & N-term of UA1CV & --- & 0 \\ \hline
NUA1R & N-term of UA1R & --- & 0 \\ \hline
NUA2 & N-term of UA2 & --- & 0 \\ \hline
NUA2CV & N-term of UA2CV & --- & 0 \\ \hline
NUACV & N-term of UACV & --- & 0 \\ \hline
NUAR & N-term of UAR & --- & 0 \\ \hline
NUC & N-term of UC & --- & 0 \\ \hline
NUC1 & N-term of UC1 & --- & 0 \\ \hline
NUC1CV & N-term of UC1CV & --- & 0 \\ \hline
NUC1R & N-term of UC1R & --- & 0 \\ \hline
NUCCV & N-term of UCCV & --- & 0 \\ \hline
NUCR & N-term of UCR & --- & 0 \\ \hline
NUCS & N-term of UCS & --- & 0 \\ \hline
NUCSTE & N-term of UCSTE & --- & 0 \\ \hline
NUCSTE1 & N-term of UCSTE1 & --- & 0 \\ \hline
NUD & N-term of UD & --- & 0 \\ \hline
NUD1 & N-term of UD1 & --- & 0 \\ \hline
NUD1CV & N-term of UD1CV & --- & 0 \\ \hline
NUD1R & N-term of UD1R & --- & 0 \\ \hline
NUD2 & N-term of UD2 & --- & 0 \\ \hline
NUD2CV & N-term of UD2CV & --- & 0 \\ \hline
NUDCV & N-term of UDCV & --- & 0 \\ \hline
NUDD & N-term of UDD & --- & 0 \\ \hline
NUDD1 & N-term of UDD1 & --- & 0 \\ \hline
NUDR & N-term of UDR & --- & 0 \\ \hline
NUDS & N-term of UDS & --- & 0 \\ \hline
NUDS1 & N-term of UDS1 & --- & 0 \\ \hline
NUP & N-term of UP & --- & 0 \\ \hline
NUPR & N-term of UPR & --- & 0 \\ \hline
NUTE & N-term of UTE & --- & 0 \\ \hline
NUTE1 & N-term of UTE1 & --- & 0 \\ \hline
NUTE1CV & N-term of UTE1CV & --- & 0 \\ \hline
NUTECV & N-term of UTECV & --- & 0 \\ \hline
NUTER & N-term of UTER & --- & 0 \\ \hline
NUTL & N-term of UTL & --- & 0 \\ \hline
NUTLCV & N-term of UTLCV & --- & 0 \\ \hline
NUTLR & N-term of UTLR & --- & 0 \\ \hline
NVSAT & N-term of VSAT & --- & 0 \\ \hline
NVSAT1 & N-term of VSAT1 & --- & 0 \\ \hline
NVSAT1R & N-term of VSAT1R & --- & 0 \\ \hline
NVSATCV & N-term of VSATCV & --- & 0 \\ \hline
NVSATR & N-term of VSATR & --- & 0 \\ \hline
NVSRD & Charge density in the drain region & --- & 5e+16 \\ \hline
NVSRS & Charge density in the source region & --- & 0 \\ \hline
NVTM & If non-zero, subthreshold swing factor multiplied by Vtm. & V & 0 \\ \hline
NWR & N-term of WR & --- & 0 \\ \hline
NXL & N-term of XL & m & 0 \\ \hline
NXRCRG1 & N-term of XRCRG1 & --- & 0 \\ \hline
NXRCRG2 & N-term of XRCRG2 & --- & 0 \\ \hline
NXW & N-term of XW & m & 0 \\ \hline
P2A1 & WL-term of A1 & --- & 0 \\ \hline
P2A11 & WL-term of A11 & --- & 0 \\ \hline
P2A2 & WL-term of A2 & --- & 0 \\ \hline
P2A21 & WL-term of A21 & --- & 0 \\ \hline
P2AGIDL & WL-term of AGIDL & --- & 0 \\ \hline
P2AGIDLB & WL-term of AGIDLB & --- & 0 \\ \hline
P2AGISL & WL-term of AGISL & --- & 0 \\ \hline
P2AGISLB & WL-term of AGISLB & --- & 0 \\ \hline
P2AIGBACC & WL-term of AIGBACC & --- & 0 \\ \hline
P2AIGBACC1 & WL-term of AIGBACC1 & --- & 0 \\ \hline
P2AIGBINV & WL-term of AIGBINV & --- & 0 \\ \hline
P2AIGBINV1 & WL-term of AIGBINV1 & --- & 0 \\ \hline
P2AIGC & WL-term of AIGC & --- & 0 \\ \hline
P2AIGC1 & WL-term of AIGC1 & --- & 0 \\ \hline
P2AIGD & WL-term of AIGD & --- & 0 \\ \hline
P2AIGD1 & WL-term of AIGD1 & --- & 0 \\ \hline
P2AIGEN & WL-term of AIGEN & --- & 0 \\ \hline
P2AIGS & WL-term of AIGS & --- & 0 \\ \hline
P2AIGS1 & WL-term of AIGS1 & --- & 0 \\ \hline
P2ALPHA0 & WL-term of ALPHA0 & --- & 0 \\ \hline
P2ALPHA1 & WL-term of ALPHA1 & --- & 0 \\ \hline
P2ALPHAII0 & WL-term of ALPHAII0 & --- & 0 \\ \hline
P2ALPHAII1 & WL-term of ALPHAII1 & --- & 0 \\ \hline
P2AT & WL-term of AT & --- & 0 \\ \hline
P2ATCV & WL-term of ATCV & --- & 0 \\ \hline
P2ATR & WL-term of ATR & --- & 0 \\ \hline
P2BETA0 & WL-term of BETA0 & --- & 0 \\ \hline
P2BETAII0 & WL-term of BETAII0 & --- & 0 \\ \hline
P2BETAII1 & WL-term of BETAII1 & m$^{2}$ & 0 \\ \hline
P2BETAII2 & WL-term of BETAII2 & --- & 0 \\ \hline
P2BGIDL & WL-term of BGIDL & --- & 0 \\ \hline
P2BGIDLB & WL-term of BGIDLB & --- & 0 \\ \hline
P2BGISL & WL-term of BGISL & --- & 0 \\ \hline
P2BGISLB & WL-term of BGISLB & --- & 0 \\ \hline
P2BIGBACC & WL-term of BIGBACC & --- & 0 \\ \hline
P2BIGBINV & WL-term of BIGBINV & --- & 0 \\ \hline
P2BIGC & WL-term of BIGC & --- & 0 \\ \hline
P2BIGD & WL-term of BIGD & --- & 0 \\ \hline
P2BIGEN & WL-term of BIGEN & --- & 0 \\ \hline
P2BIGS & WL-term of BIGS & --- & 0 \\ \hline
P2CDSC & WL-term of CDSC & F & 0 \\ \hline
P2CDSCD & WL-term of CDSCD & F & 0 \\ \hline
P2CDSCDR & WL-term of CDSCDR & F & 0 \\ \hline
P2CFD & WL-term of CFD & --- & 0 \\ \hline
P2CFS & WL-term of CFS & --- & 0 \\ \hline
P2CGBL & WL-term of CGBL & --- & 0 \\ \hline
P2CGDL & WL-term of CGDL & --- & 0 \\ \hline
P2CGIDL & WL-term of CGIDL & --- & 0 \\ \hline
P2CGIDLB & WL-term of CGIDLB & --- & 0 \\ \hline
P2CGISL & WL-term of CGISL & --- & 0 \\ \hline
P2CGISLB & WL-term of CGISLB & --- & 0 \\ \hline
P2CGSL & WL-term of CGSL & --- & 0 \\ \hline
P2CIGBACC & WL-term of CIGBACC & --- & 0 \\ \hline
P2CIGBINV & WL-term of CIGBINV & --- & 0 \\ \hline
P2CIGC & WL-term of CIGC & --- & 0 \\ \hline
P2CIGD & WL-term of CIGD & --- & 0 \\ \hline
P2CIGS & WL-term of CIGS & --- & 0 \\ \hline
P2CIT & WL-term of CIT & F & 0 \\ \hline
P2CITR & WL-term of CITR & F & 0 \\ \hline
P2CKAPPAB & WL-term of CKAPPAB & --- & 0 \\ \hline
P2CKAPPAD & WL-term of CKAPPAD & --- & 0 \\ \hline
P2CKAPPAS & WL-term of CKAPPAS & --- & 0 \\ \hline
P2COVD & WL-term of COVD & --- & 0 \\ \hline
P2COVS & WL-term of COVS & --- & 0 \\ \hline
P2DELTAVSAT & WL-term of DELTAVSAT & m$^{2}$ & 0 \\ \hline
P2DELTAVSATCV & WL-term of DELTAVSATCV & m$^{2}$ & 0 \\ \hline
P2DIMENSION1 & WL-term of DIMENSION1 & m$^{2}$ & 0 \\ \hline
P2DIMENSION2 & WL-term of DIMENSION2 & m$^{2}$ & 0 \\ \hline
P2DIMENSION3 & WL-term of DIMENSION3 & m$^{2}$ & 0 \\ \hline
P2DROUT & WL-term of DROUT & m$^{2}$ & 0 \\ \hline
P2DSUB & WL-term of DSUB & m$^{2}$ & 0 \\ \hline
P2DVT0 & WL-term of DVT0 & m$^{2}$ & 0 \\ \hline
P2DVT1 & WL-term of DVT1 & m$^{2}$ & 0 \\ \hline
P2DVT1SS & WL-term of DVT1SS & m$^{2}$ & 0 \\ \hline
P2DVTP0 & WL-term of DVTP0 & m$^{2}$ & 0 \\ \hline
P2DVTP1 & WL-term of DVTP1 & m$^{2}$ & 0 \\ \hline
P2DVTSHIFT & WL-term of DVTSHIFT & --- & 0 \\ \hline
P2DVTSHIFTR & WL-term of DVTSHIFTR & --- & 0 \\ \hline
P2DWBIN & WL-term of DWBIN & --- & 0 \\ \hline
P2E2NOM & WL-term of E2NOM & --- & 0 \\ \hline
P2E3NOM & WL-term of E3NOM & --- & 0 \\ \hline
P2EGIDL & WL-term of EGIDL & --- & 0 \\ \hline
P2EGIDLB & WL-term of EGIDLB & --- & 0 \\ \hline
P2EGISL & WL-term of EGISL & --- & 0 \\ \hline
P2EGISLB & WL-term of EGISLB & --- & 0 \\ \hline
P2EIGBINV & WL-term of EIGBINV & --- & 0 \\ \hline
P2EMOBT & WL-term of EMOBT & m$^{2}$ & 0 \\ \hline
P2ESATII & WL-term of ESATII & --- & 0 \\ \hline
P2ETA0 & WL-term of ETA0 & m$^{2}$ & 0 \\ \hline
P2ETA0CV & WL-term of ETA0CV & m$^{2}$ & 0 \\ \hline
P2ETA0R & WL-term of ETA0R & m$^{2}$ & 0 \\ \hline
P2ETAMOB & WL-term of ETAMOB & m$^{2}$ & 0 \\ \hline
P2EU & WL-term of EU & --- & 0 \\ \hline
P2EU1 & WL-term of EU1 & m$^{2}$ & 0 \\ \hline
P2EUR & WL-term of EUR & --- & 0 \\ \hline
P2IGT & WL-term of IGT & m$^{2}$ & 0 \\ \hline
P2IIT & WL-term of IIT & m$^{2}$ & 0 \\ \hline
P2K0 & WL-term of K0 & --- & 0 \\ \hline
P2K01 & WL-term of K01 & --- & 0 \\ \hline
P2K0SI & WL-term of K0SI & m$^{2}$ & 0 \\ \hline
P2K0SI1 & WL-term of K0SI1 & --- & 0 \\ \hline
P2K0SISAT & WL-term of K0SISAT & m$^{2}$ & 0 \\ \hline
P2K0SISAT1 & WL-term of K0SISAT1 & m$^{2}$ & 0 \\ \hline
P2K1 & WL-term of K1 & --- & 0 \\ \hline
P2K11 & WL-term of K11 & --- & 0 \\ \hline
P2K1RSCE & WL-term of K1RSCE & --- & 0 \\ \hline
P2K2 & WL-term of K2 & m$^{2}$ & 0 \\ \hline
P2K21 & WL-term of K21 & m$^{2}$ & 0 \\ \hline
P2K2SAT & WL-term of K2SAT & m$^{2}$ & 0 \\ \hline
P2K2SAT1 & WL-term of K2SAT1 & m$^{2}$ & 0 \\ \hline
P2K2SI & WL-term of K2SI & m$^{2}$ & 0 \\ \hline
P2K2SI1 & WL-term of K2SI1 & --- & 0 \\ \hline
P2K2SISAT & WL-term of K2SISAT & m$^{2}$ & 0 \\ \hline
P2K2SISAT1 & WL-term of K2SISAT1 & m$^{2}$ & 0 \\ \hline
P2KSATIV & WL-term of KSATIV & m$^{2}$ & 0 \\ \hline
P2KSATIVR & WL-term of KSATIVR & m$^{2}$ & 0 \\ \hline
P2KT1 & WL-term of KT1 & --- & 0 \\ \hline
P2LII & WL-term of LII & --- & 0 \\ \hline
P2LPE0 & WL-term of LPE0 & --- & 0 \\ \hline
P2MEXP & WL-term of MEXP & m$^{2}$ & 0 \\ \hline
P2MEXPR & WL-term of MEXPR & m$^{2}$ & 0 \\ \hline
P2MFQ1NOM & WL-term of MFQ1NOM & m$^{2}$ & 0 \\ \hline
P2MFQ2NOM & WL-term of MFQ2NOM & m$^{2}$ & 0 \\ \hline
P2MFQ3NOM & WL-term of MFQ3NOM & m$^{2}$ & 0 \\ \hline
P2MPOWER & WL-term for MPOWER & m$^{2}$ & 0 \\ \hline
P2NGATE & WL-term of NGATE & --- & 0 \\ \hline
P2NIGBACC & WL-term of NIGBACC & m$^{2}$ & 0 \\ \hline
P2NIGBINV & WL-term of NIGBINV & m$^{2}$ & 0 \\ \hline
P2NOIA2 & WL-term for NOIA2 & --- & 0 \\ \hline
P2NTGEN & WL-term of NTGEN & m$^{2}$ & 0 \\ \hline
P2NTOX & WL-term of NTOX & m$^{2}$ & 0 \\ \hline
P2PCLM & WL-term of PCLM & m$^{2}$ & 0 \\ \hline
P2PCLMCV & WL-term of PCLMCV & m$^{2}$ & 0 \\ \hline
P2PCLMG & WL-term of PCLMG & --- & 0 \\ \hline
P2PCLMR & WL-term of PCLMR & m$^{2}$ & 0 \\ \hline
P2PDIBL1 & WL-term of PDIBL1 & m$^{2}$ & 0 \\ \hline
P2PDIBL1R & WL-term of PDIBL1R & m$^{2}$ & 0 \\ \hline
P2PDIBL2 & WL-term of PDIBL2 & m$^{2}$ & 0 \\ \hline
P2PDIBL2R & WL-term of PDIBL2R & m$^{2}$ & 0 \\ \hline
P2PGIDL & WL-term of PGIDL & m$^{2}$ & 0 \\ \hline
P2PGIDLB & WL-term of PGIDLB & m$^{2}$ & 0 \\ \hline
P2PGISL & WL-term of PGISL & m$^{2}$ & 0 \\ \hline
P2PGISLB & WL-term of PGISLB & m$^{2}$ & 0 \\ \hline
P2PHIBE & WL-term of PHIBE & --- & 0 \\ \hline
P2PHIG & WL-term of PHIG & --- & 0 \\ \hline
P2PHIN & WL-term of PHIN & --- & 0 \\ \hline
P2PIGCD & WL-term of PIGCD & m$^{2}$ & 0 \\ \hline
P2POXEDGE & WL-term of POXEDGE & m$^{2}$ & 0 \\ \hline
P2PRT & WL-term of PRT & --- & 0 \\ \hline
P2PRT1 & WL-term of PRT1 & --- & 0 \\ \hline
P2PRWGD & WL-term of PRWGD & --- & 0 \\ \hline
P2PRWGS & WL-term of PRWGS & --- & 0 \\ \hline
P2PSAT & WL-term of PSAT & m$^{2}$ & 0 \\ \hline
P2PSATCV & WL-term of PSATCV & m$^{2}$ & 0 \\ \hline
P2PTWG & WL-term of PTWG & --- & 0 \\ \hline
P2PTWGR & WL-term of PTWGR & --- & 0 \\ \hline
P2PTWGT & WL-term of PTWGT & --- & 0 \\ \hline
P2PVAG & WL-term of PVAG & m$^{2}$ & 0 \\ \hline
P2QMFACTOR & WL-term of QMFACTOR & m$^{2}$ & 0 \\ \hline
P2QMTCENCV & WL-term of QMTCENCV & m$^{2}$ & 0 \\ \hline
P2QMTCENCVA & WL-term of QMTCENCVA & m$^{2}$ & 0 \\ \hline
P2QSREF & WL-term for QSREF & m$^{2}$ & 0 \\ \hline
P2RDSW & WL-term of RDSW & --- & 0 \\ \hline
P2RDW & WL-term of RDW & --- & 0 \\ \hline
P2RSW & WL-term of RSW & --- & 0 \\ \hline
P2SII0 & WL-term of SII0 & --- & 0 \\ \hline
P2SII1 & WL-term of SII1 & m$^{2}$ & 0 \\ \hline
P2SII2 & WL-term of SII2 & --- & 0 \\ \hline
P2SIID & WL-term of SIID & --- & 0 \\ \hline
P2SPRT & WL-term of SPRT & m$^{2}$ & 0 \\ \hline
P2SSP1 & WL-term of SSP1 & m$^{2}$ & 0 \\ \hline
P2SSP2 & WL-term of SSP2 & m$^{2}$ & 0 \\ \hline
P2SSP3 & WL-term of SSP3 & m$^{2}$ & 0 \\ \hline
P2TGIDL & WL-term of TGIDL & --- & 0 \\ \hline
P2TII & WL-term of TII & m$^{2}$ & 0 \\ \hline
P2TR0 & WL-term of TR0 & --- & 0 \\ \hline
P2TSS & WL-term of TSS & --- & 0 \\ \hline
P2U0 & WL-term of U0 & --- & 0 \\ \hline
P2U0CV & WL-term of U0CV & --- & 0 \\ \hline
P2U0R & WL-term of U0R & --- & 0 \\ \hline
P2UA & WL-term of UA & --- & 0 \\ \hline
P2UA1 & WL-term of UA1 & m$^{2}$ & 0 \\ \hline
P2UA1CV & WL-term of UA1CV & m$^{2}$ & 0 \\ \hline
P2UA1R & WL-term of UA1R & m$^{2}$ & 0 \\ \hline
P2UA2 & WL-term of UA2 & m$^{2}$ & 0 \\ \hline
P2UA2CV & WL-term of UA2CV & m$^{2}$ & 0 \\ \hline
P2UACV & WL-term of UACV & --- & 0 \\ \hline
P2UAR & WL-term of UAR & --- & 0 \\ \hline
P2UC & WL-term of UC & --- & 0 \\ \hline
P2UC1 & WL-term of UC1 & m$^{2}$ & 0 \\ \hline
P2UC1CV & WL-term of UC1CV & m$^{2}$ & 0 \\ \hline
P2UC1R & WL-term of UC1R & m$^{2}$ & 0 \\ \hline
P2UCCV & WL-term of UCCV & --- & 0 \\ \hline
P2UCR & WL-term of UCR & --- & 0 \\ \hline
P2UCS & WL-term of UCS & m$^{2}$ & 0 \\ \hline
P2UCSTE & WL-term of UCSTE & m$^{2}$ & 0 \\ \hline
P2UCSTE1 & WL-term of UCSTE1 & m$^{2}$ & 0 \\ \hline
P2UD & WL-term of UD & --- & 0 \\ \hline
P2UD1 & WL-term of UD1 & m$^{2}$ & 0 \\ \hline
P2UD1CV & WL-term of UD1CV & m$^{2}$ & 0 \\ \hline
P2UD1R & WL-term of UD1R & m$^{2}$ & 0 \\ \hline
P2UD2 & WL-term of UD2 & m$^{2}$ & 0 \\ \hline
P2UD2CV & WL-term of UD2CV & m$^{2}$ & 0 \\ \hline
P2UDCV & WL-term of UDCV & --- & 0 \\ \hline
P2UDD & WL-term of UDD & m$^{2}$ & 0 \\ \hline
P2UDD1 & WL-term of UDD1 & m$^{2}$ & 0 \\ \hline
P2UDR & WL-term of UDR & --- & 0 \\ \hline
P2UDS & WL-term of UDS & m$^{2}$ & 0 \\ \hline
P2UDS1 & WL-term of UDS1 & m$^{2}$ & 0 \\ \hline
P2UP & WL-term of UP & --- & 0 \\ \hline
P2UPR & WL-term of UPR & --- & 0 \\ \hline
P2UTE & WL-term of UTE & m$^{2}$ & 0 \\ \hline
P2UTE1 & WL-term of UTE1 & m$^{2}$ & 0 \\ \hline
P2UTE1CV & WL-term of UTE1CV & m$^{2}$ & 0 \\ \hline
P2UTECV & WL-term of UTECV & m$^{2}$ & 0 \\ \hline
P2UTER & WL-term of UTER & m$^{2}$ & 0 \\ \hline
P2UTL & WL-term of UTL & m$^{2}$ & 0 \\ \hline
P2UTLCV & WL-term of UTLCV & m$^{2}$ & 0 \\ \hline
P2UTLR & WL-term of UTLR & m$^{2}$ & 0 \\ \hline
P2VSAT & WL-term of VSAT & --- & 0 \\ \hline
P2VSAT1 & WL-term of VSAT1 & --- & 0 \\ \hline
P2VSAT1R & WL-term of VSAT1R & --- & 0 \\ \hline
P2VSATCV & WL-term of VSATCV & --- & 0 \\ \hline
P2VSATR & WL-term of VSATR & --- & 0 \\ \hline
P2WR & WL-term of WR & m$^{2}$ & 0 \\ \hline
P2XRCRG1 & WL-term of XRCRG1 & m$^{2}$ & 0 \\ \hline
P2XRCRG2 & WL-term of XRCRG2 & m$^{2}$ & 0 \\ \hline
P2XW & WL-term of XW & --- & 0 \\ \hline
PA1 & P-term of A1 & --- & 0 \\ \hline
PA11 & P-term of A11 & --- & 0 \\ \hline
PA2 & P-term of A2 & m/V & 0 \\ \hline
PA21 & P-term of A21 & --- & 0 \\ \hline
PAGIDL & P-term of AGIDL & --- & 0 \\ \hline
PAGIDLB & P-term of AGIDLB & --- & 0 \\ \hline
PAGISL & P-term of AGISL & --- & 0 \\ \hline
PAGISLB & P-term of AGISLB & --- & 0 \\ \hline
PAIGBACC & P-term of AIGBACC & --- & 0 \\ \hline
PAIGBACC1 & P-term of AIGBACC1 & --- & 0 \\ \hline
PAIGBINV & P-term of AIGBINV & --- & 0 \\ \hline
PAIGBINV1 & P-term of AIGBINV1 & --- & 0 \\ \hline
PAIGC & P-term of AIGC & --- & 0 \\ \hline
PAIGC1 & P-term of AIGC1 & --- & 0 \\ \hline
PAIGD & P-term of AIGD & --- & 0 \\ \hline
PAIGD1 & P-term of AIGD1 & --- & 0 \\ \hline
PAIGEN & P-term of AIGEN & --- & 0 \\ \hline
PAIGS & P-term of AIGS & --- & 0 \\ \hline
PAIGS1 & P-term of AIGS1 & --- & 0 \\ \hline
PALPHA0 & P-term of ALPHA0 & --- & 0 \\ \hline
PALPHA1 & P-term of ALPHA1 & m/V & 0 \\ \hline
PALPHAII0 & P-term of ALPHAII0 & --- & 0 \\ \hline
PALPHAII1 & P-term of ALPHAII1 & m/V & 0 \\ \hline
PAT & P-term of AT & --- & 0 \\ \hline
PATCV & P-term of ATCV & --- & 0 \\ \hline
PATR & P-term of ATR & --- & 0 \\ \hline
PBD & Drain-side bulk junction built-in potential & V & 0 \\ \hline
PBETA0 & P-term of BETA0 & m/V & 0 \\ \hline
PBETAII0 & P-term of BETAII0 & m/V & 0 \\ \hline
PBETAII1 & P-term of BETAII1 & m & 0 \\ \hline
PBETAII2 & P-term of BETAII2 & --- & 0 \\ \hline
PBGIDL & P-term of BGIDL & V & 0 \\ \hline
PBGIDLB & P-term of BGIDLB & V & 0 \\ \hline
PBGISL & P-term of BGISL & V & 0 \\ \hline
PBGISLB & P-term of BGISLB & V & 0 \\ \hline
PBIGBACC & P-term of BIGBACC & --- & 0 \\ \hline
PBIGBINV & P-term of BIGBINV & --- & 0 \\ \hline
PBIGC & P-term of BIGC & --- & 0 \\ \hline
PBIGD & P-term of BIGD & --- & 0 \\ \hline
PBIGEN & P-term of BIGEN & --- & 0 \\ \hline
PBIGS & P-term of BIGS & --- & 0 \\ \hline
PBS & Source-side bulk junction built-in potential & V & 1 \\ \hline
PBSWD & Built-in potential for Drain-side sidewall junction capacitance & V & 0 \\ \hline
PBSWGD & Built-in potential for Drain-side gate sidewall junction capacitance & V & 0 \\ \hline
PBSWGS & Built-in potential for Source-side gate sidewall junction capacitance & V & 0 \\ \hline
PBSWS & Built-in potential for Source-side sidewall junction capacitance & V & 1 \\ \hline
PCDSC & P-term of CDSC & --- & 0 \\ \hline
PCDSCD & P-term of CDSCD & --- & 0 \\ \hline
PCDSCDR & P-term of CDSCDR & --- & 0 \\ \hline
PCFD & P-term of CFD & F & 0 \\ \hline
PCFS & P-term of CFS & F & 0 \\ \hline
PCGBL & P-term of CGBL & F & 0 \\ \hline
PCGDL & P-term of CGDL & F & 0 \\ \hline
PCGIDL & P-term of CGIDL & --- & 0 \\ \hline
PCGIDLB & P-term of CGIDLB & --- & 0 \\ \hline
PCGISL & P-term of CGISL & --- & 0 \\ \hline
PCGISLB & P-term of CGISLB & --- & 0 \\ \hline
PCGSL & P-term of CGSL & F & 0 \\ \hline
PCIGBACC & P-term of CIGBACC & m/V & 0 \\ \hline
PCIGBINV & P-term of CIGBINV & m/V & 0 \\ \hline
PCIGC & P-term of CIGC & m/V & 0 \\ \hline
PCIGD & P-term of CIGD & m/V & 0 \\ \hline
PCIGS & P-term of CIGS & m/V & 0 \\ \hline
PCIT & P-term of CIT & --- & 0 \\ \hline
PCITR & P-term of CITR & --- & 0 \\ \hline
PCKAPPAB & P-term of CKAPPAB & --- & 0 \\ \hline
PCKAPPAD & P-term of CKAPPAD & --- & 0 \\ \hline
PCKAPPAS & P-term of CKAPPAS & --- & 0 \\ \hline
PCLM & Channel length modulation (CLM) parameter & --- & 0.013 \\ \hline
PCLMCV & CLM parameter for short-channel CV & --- & 0 \\ \hline
PCLMG & Gate bias dependence parameter for CLM & V$^{-1}$ & 0 \\ \hline
PCLMR & Reverse model PCLM parameter & --- & 0 \\ \hline
PCLMT & CRYOMOD != 0 temperature coefficient for PCLM & --- & -2e-05 \\ \hline
PCOVD & P-term of COVD & F & 0 \\ \hline
PCOVS & P-term of COVS & F & 0 \\ \hline
PDEJ & Drain-to-substrate PN junction perimeter (BULKMOD = 1 or 2) & m & 0 \\ \hline
PDELTAVSAT & P-term of DELTAVSAT & m & 0 \\ \hline
PDELTAVSATCV & P-term of DELTAVSATCV & m & 0 \\ \hline
PDEO & Perimeter of drain-to-substrate overlap region through oxide & m & 0 \\ \hline
PDIBL1 & DIBL output conductance parameter - forward mode & --- & 1.3 \\ \hline
PDIBL1R & DIBL output conductance parameter - reverse mode & --- & 0 \\ \hline
PDIBL2 & DIBL output conductance parameter & --- & 0.0002 \\ \hline
PDIBL2R & DIBL output conductance parameter - reverse mode & --- & 0 \\ \hline
PDIMENSION1 & P-term of DIMENSION1 & m & 0 \\ \hline
PDIMENSION2 & P-term of DIMENSION2 & m & 0 \\ \hline
PDIMENSION3 & P-term of DIMENSION3 & m & 0 \\ \hline
PDLBIN & P-term of DLBIN & m$^{2}$ & 0 \\ \hline
PDROUT & P-term of DROUT & m & 0 \\ \hline
PDSUB & P-term of DSUB & m & 0 \\ \hline
PDVT0 & P-term of DVT0 & m & 0 \\ \hline
PDVT1 & P-term of DVT1 & m & 0 \\ \hline
PDVT1SS & P-term of DVT1SS & m & 0 \\ \hline
PDVTP0 & P-term of DVTP0 & m & 0 \\ \hline
PDVTP1 & P-term of DVTP1 & m & 0 \\ \hline
PDVTSHIFT & P-term of DVTSHIFT & --- & 0 \\ \hline
PDVTSHIFTR & P-term of DVTSHIFTR & --- & 0 \\ \hline
PDWBIN & P-term of DWBIN & m$^{2}$ & 0 \\ \hline
PE2NOM & P-term of E2NOM & --- & 0 \\ \hline
PE3NOM & P-term of E3NOM & --- & 0 \\ \hline
PEGIDL & P-term of EGIDL & --- & 0 \\ \hline
PEGIDLB & P-term of EGIDLB & --- & 0 \\ \hline
PEGISL & P-term of EGISL & --- & 0 \\ \hline
PEGISLB & P-term of EGISLB & --- & 0 \\ \hline
PEIGBINV & P-term of EIGBINV & --- & 0 \\ \hline
PEMOBT & P-term of EMOBT & m & 0 \\ \hline
PESATII & P-term of ESATII & V & 0 \\ \hline
PETA0 & P-term of ETA0 & m & 0 \\ \hline
PETA0CV & P-term of ETA0CV & m & 0 \\ \hline
PETA0R & P-term of ETA0R & m & 0 \\ \hline
PETAMOB & P-term of ETAMOB & m & 0 \\ \hline
PEU & P-term of EU & --- & 0 \\ \hline
PEU1 & P-term of EU1 & m & 0 \\ \hline
PEUR & P-term of EUR & --- & 0 \\ \hline
PGIDL & Parameter for body-bias effect on GIDL & --- & 1 \\ \hline
PGIDLB & Parameter for body-bias effect on parasitic substrate GIDL & --- & 1 \\ \hline
PGISL & Parameter for body-bias effect on GISL & --- & 0 \\ \hline
PGISLB & Parameter for body-bias effect on parasitic substrate GISL & --- & 0 \\ \hline
PHIBE & Body effect voltage parameter. After binning it should be within [0.2 : 1.2] & V & 0.7 \\ \hline
PHIG & Gate workfunction & --- & 4.61 \\ \hline
PHIGL & Length dependence of gate workfunction & --- & 0 \\ \hline
PHIGLT & Coupled NFIN and length dependence of gate workfunction & --- & 0 \\ \hline
PHIGN1 & NFIN dependence of gate workfunction & --- & 0 \\ \hline
PHIGN2 & NFIN dependence of gate workfunction & --- & 100000 \\ \hline
PHIN & Nonuniform vertical doping effect on surface potential & V & 0.05 \\ \hline
PIGCD & Parameter for Igc partition & --- & 1 \\ \hline
PIGT & P-term of IGT & m & 0 \\ \hline
PIIT & P-term of IIT & m & 0 \\ \hline
PK0 & P-term of K0 & --- & 0 \\ \hline
PK01 & P-term of K01 & --- & 0 \\ \hline
PK0SI & P-term of K0SI & m & 0 \\ \hline
PK0SI1 & P-term of K0SI1 & --- & 0 \\ \hline
PK0SISAT & P-term of K0SISAT & m & 0 \\ \hline
PK0SISAT1 & P-term of K0SISAT1 & m & 0 \\ \hline
PK1 & P-term of K1 & --- & 0 \\ \hline
PK11 & P-term of K11 & --- & 0 \\ \hline
PK1RSCE & P-term of K1RSCE & --- & 0 \\ \hline
PK2 & P-term of K2 & m & 0 \\ \hline
PK21 & P-term of K21 & m & 0 \\ \hline
PK2SAT & P-term of K2SAT & m & 0 \\ \hline
PK2SAT1 & P-term of K2SAT1 & m & 0 \\ \hline
PK2SI & P-term of K2SI & m & 0 \\ \hline
PK2SI1 & P-term of K2SI1 & --- & 0 \\ \hline
PK2SISAT & P-term of K2SISAT & m & 0 \\ \hline
PK2SISAT1 & P-term of K2SISAT1 & m & 0 \\ \hline
PKSATIV & P-term of KSATIV & m & 0 \\ \hline
PKSATIVR & P-term of KSATIVR & m & 0 \\ \hline
PKT1 & P-term of KT1 & --- & 0 \\ \hline
PLII & P-term of LII & --- & 0 \\ \hline
PLINT & P-term of LINT & m$^{2}$ & 0 \\ \hline
PLPE0 & P-term of LPE0 & m$^{2}$ & 0 \\ \hline
PMEXP & P-term of MEXP & m & 0 \\ \hline
PMEXPR & P-term of MEXPR & m & 0 \\ \hline
PMFQ1NOM & P-term of MFQ1NOM & m & 0 \\ \hline
PMFQ2NOM & P-term of MFQ2NOM & m & 0 \\ \hline
PMFQ3NOM & P-term of MFQ3NOM & m & 0 \\ \hline
PMPOWER & P-term for MPOWER & m & 0 \\ \hline
PNBODY & P-term of NBODY & --- & 0 \\ \hline
PNGATE & P-term of NGATE & --- & 0 \\ \hline
PNIGBACC & P-term of NIGBACC & m & 0 \\ \hline
PNIGBINV & P-term of NIGBINV & m & 0 \\ \hline
PNOIA2 & P-term for NOIA2 & --- & 0 \\ \hline
PNTGEN & P-term of NTGEN & m & 0 \\ \hline
PNTOX & P-term of NTOX & m & 0 \\ \hline
POXEDGE & Factor for the gate edge Tox & --- & 1 \\ \hline
PPCLM & P-term of PCLM & m & 0 \\ \hline
PPCLMCV & P-term of PCLMCV & m & 0 \\ \hline
PPCLMG & P-term of PCLMG & m/V & 0 \\ \hline
PPCLMR & P-term of PCLMR & m & 0 \\ \hline
PPDIBL1 & P-term of PDIBL1 & m & 0 \\ \hline
PPDIBL1R & P-term of PDIBL1R & m & 0 \\ \hline
PPDIBL2 & P-term of PDIBL2 & m & 0 \\ \hline
PPDIBL2R & P-term of PDIBL2R & m & 0 \\ \hline
PPGIDL & P-term of PGIDL & m & 0 \\ \hline
PPGIDLB & P-term of PGIDLB & m & 0 \\ \hline
PPGISL & P-term of PGISL & m & 0 \\ \hline
PPGISLB & P-term of PGISLB & m & 0 \\ \hline
PPHIBE & P-term of PHIBE & --- & 0 \\ \hline
PPHIG & P-term of PHIG & --- & 0 \\ \hline
PPHIN & P-term of PHIN & --- & 0 \\ \hline
PPIGCD & P-term of PIGCD & m & 0 \\ \hline
PPOXEDGE & P-term of POXEDGE & m & 0 \\ \hline
PPRT & P-term of PRT & --- & 0 \\ \hline
PPRT1 & P-term of PRT1 & --- & 0 \\ \hline
PPRWGD & P-term of PRWGD & m/V & 0 \\ \hline
PPRWGS & P-term of PRWGS & m/V & 0 \\ \hline
PPSAT & P-term of PSAT & m & 0 \\ \hline
PPSATCV & P-term of PSATCV & m & 0 \\ \hline
PPTWG & P-term of PTWG & --- & 0 \\ \hline
PPTWGR & P-term of PTWGR & --- & 0 \\ \hline
PPTWGT & P-term of PTWGT & --- & 0 \\ \hline
PPVAG & P-term of PVAG & m & 0 \\ \hline
PQM & Slope of normalized Tcen in inversion & --- & 0.66 \\ \hline
PQMACC & Slope of normalized Tcen in accumulation & --- & 0.66 \\ \hline
PQMFACTOR & P-term of QMFACTOR & m & 0 \\ \hline
PQMTCENCV & P-term of QMTCENCV & m & 0 \\ \hline
PQMTCENCVA & P-term of QMTCENCVA & m & 0 \\ \hline
PQSREF & P-term for QSREF & m & 0 \\ \hline
PRDDR & Drain-side quasi-saturation parameter & --- & 0 \\ \hline
PRDSW & P-term of RDSW & --- & 0 \\ \hline
PRDW & P-term of RDW & --- & 0 \\ \hline
PRSDEND & Extra silicon/silicide interface perimeter at the two ends of the finFET & m & 0 \\ \hline
PRSDR & Source-side quasi-saturation parameter & --- & 1 \\ \hline
PRSW & P-term of RSW & --- & 0 \\ \hline
PRT & Series resistance temperature coefficient & --- & 0.001 \\ \hline
PRT1 & CRYOMOD != 0 series resistance temperature coefficient at low temperatures & --- & 0.0004 \\ \hline
PRTVSRSD & Temperature coefficient of resistance in S/D velocity saturation model & --- & 0 \\ \hline
PRWGD & Gate bias dependence of drain extension resistance & V$^{-1}$ & 0 \\ \hline
PRWGS & Gate bias dependence of source extension resistance & V$^{-1}$ & 0 \\ \hline
PSAT & Velocity saturation exponent, after binning should be from [2.0 : inf) & --- & 2 \\ \hline
PSATCV & Velocity saturation exponent for C-V & --- & 0 \\ \hline
PSATXVSRSD & Fine tuning of PTWGVSRSD effect & V & 60 \\ \hline
PSEJ & Source-to-substrate PN junction perimeter (BULKMOD = 1 or 2) & m & 0 \\ \hline
PSEO & Perimeter of source-to-substrate overlap region through oxide & m & 0 \\ \hline
PSII0 & P-term of SII0 & m/V & 0 \\ \hline
PSII1 & P-term of SII1 & m & 0 \\ \hline
PSII2 & P-term of SII2 & --- & 0 \\ \hline
PSIID & P-term of SIID & --- & 0 \\ \hline
PSPRT & P-term of SPRT & m & 0 \\ \hline
PSSP1 & P-term of SSP1 & m & 0 \\ \hline
PSSP2 & P-term of SSP2 & m & 0 \\ \hline
PSSP3 & P-term of SSP3 & m & 0 \\ \hline
PTGIDL & P-term of TGIDL & --- & 0 \\ \hline
PTII & P-term of TII & m & 0 \\ \hline
PTR0 & P-term of TR0 & --- & 0 \\ \hline
PTSS & P-term of TSS & --- & 0 \\ \hline
PTWG & Gmsat degradation parameter - forward mode & --- & 0 \\ \hline
PTWG1VSRSD & VSATRSD variation with gate bias & V & 0 \\ \hline
PTWGR & Gmsat degradation parameter - reverse mode & --- & 0 \\ \hline
PTWGT & PTWG temperature coefficient & --- & 0.004 \\ \hline
PTWGVSRSD & VSATRSD variation with gate bias & --- & 0 \\ \hline
PU0 & P-term of U0 & --- & 0 \\ \hline
PU0CV & P-term of U0CV & --- & 0 \\ \hline
PU0R & P-term of U0R & --- & 0 \\ \hline
PUA & P-term of UA & --- & 0 \\ \hline
PUA1 & P-term of UA1 & m & 0 \\ \hline
PUA1CV & P-term of UA1CV & m & 0 \\ \hline
PUA1R & P-term of UA1R & m & 0 \\ \hline
PUA2 & P-term of UA2 & m & 0 \\ \hline
PUA2CV & P-term of UA2CV & m & 0 \\ \hline
PUACV & P-term of UACV & --- & 0 \\ \hline
PUAR & P-term of UAR & --- & 0 \\ \hline
PUC & P-term of UC & --- & 0 \\ \hline
PUC1 & P-term of UC1 & m & 0 \\ \hline
PUC1CV & P-term of UC1CV & m & 0 \\ \hline
PUC1R & P-term of UC1R & m & 0 \\ \hline
PUCCV & P-term of UCCV & --- & 0 \\ \hline
PUCR & P-term of UCR & --- & 0 \\ \hline
PUCS & P-term of UCS & m & 0 \\ \hline
PUCSTE & P-term of UCSTE & m & 0 \\ \hline
PUCSTE1 & P-term of UCSTE1 & m & 0 \\ \hline
PUD & P-term of UD & --- & 0 \\ \hline
PUD1 & P-term of UD1 & m & 0 \\ \hline
PUD1CV & P-term of UD1CV & m & 0 \\ \hline
PUD1R & P-term of UD1R & m & 0 \\ \hline
PUD2 & P-term of UD2 & m & 0 \\ \hline
PUD2CV & P-term of UD2CV & m & 0 \\ \hline
PUDCV & P-term of UDCV & --- & 0 \\ \hline
PUDD & P-term of UDD & m & 0 \\ \hline
PUDD1 & P-term of UDD1 & m & 0 \\ \hline
PUDR & P-term of UDR & --- & 0 \\ \hline
PUDS & P-term of UDS & m & 0 \\ \hline
PUDS1 & P-term of UDS1 & m & 0 \\ \hline
PUP & P-term of UP & --- & 0 \\ \hline
PUPR & P-term of UPR & --- & 0 \\ \hline
PUTE & P-term of UTE & m & 0 \\ \hline
PUTE1 & P-term of UTE1 & m & 0 \\ \hline
PUTE1CV & P-term of UTE1CV & m & 0 \\ \hline
PUTECV & P-term of UTECV & m & 0 \\ \hline
PUTER & P-term of UTER & m & 0 \\ \hline
PUTL & P-term of UTL & m & 0 \\ \hline
PUTLCV & P-term of UTLCV & m & 0 \\ \hline
PUTLR & P-term of UTLR & m & 0 \\ \hline
PVAG & Vgs dependence on early voltage & --- & 1 \\ \hline
PVSAT & P-term of VSAT & --- & 0 \\ \hline
PVSAT1 & P-term of VSAT1 & --- & 0 \\ \hline
PVSAT1R & P-term of VSAT1R & --- & 0 \\ \hline
PVSATCV & P-term of VSATCV & --- & 0 \\ \hline
PVSATR & P-term of VSATR & --- & 0 \\ \hline
PWR & P-term of WR & m & 0 \\ \hline
PXL & P-term of XL & m$^{2}$ & 0 \\ \hline
PXRCRG1 & P-term of XRCRG1 & m & 0 \\ \hline
PXRCRG2 & P-term of XRCRG2 & m & 0 \\ \hline
PXW & P-term of XW & m$^{2}$ & 0 \\ \hline
QM0 & Knee-point for Tcen in inversion (Charge normalized to Cox) & V & 0.001 \\ \hline
QM0ACC & Knee-point for Tcen in accumulation (Charge normalized to Cox) & V & 0.001 \\ \hline
QMFACTOR & Prefactor + switch for QM Vth correction & --- & 0 \\ \hline
QMFACTORCV & Charge dependence taking QM effects into account & --- & 0 \\ \hline
QMTCENCV & Prefactor + switch for QM Width and Toxeff correction for CV & --- & 0 \\ \hline
QMTCENCVA & Prefactor + switch for QM Width and Toxeff correction for CV (accumulation region) & --- & 0 \\ \hline
QSREF & Charge at threshold condition & --- & 0.05 \\ \hline
RDDR & Drain-side drift resistance parameter - forward mode & --- & 0 \\ \hline
RDDRR & Drain-side drift resistance parameter - reverse mode & --- & 0 \\ \hline
RDLCW & 'R'esistance of the 'D'rain region at 'L'ow 'C'urrent & --- & 0 \\ \hline
RDSMOD & 0: Bias-dependent S/D resistances internal and bias-independent S/D resistances external, 1: Both bias-dependent and independent of S/D resistances external, 2: Both bias-dependent and independent of S/D resistances internal & --- & 0 \\ \hline
RDSW & RDSMOD = 0 zero bias S/D extension resistance per unit width & --- & 100 \\ \hline
RDSWMIN & RDSMOD = 0 S/D extension resistance per unit width at high Vgs & --- & 0 \\ \hline
RDVDS & Parameter for Isat\_rd variation with drain voltage & V & 8 \\ \hline
RDW & RDSMOD = 1 zero bias drain extension resistance per unit width & --- & 50 \\ \hline
RDWMIN & RDSMOD = 1 drain extension resistance per unit width at high Vgs & --- & 0 \\ \hline
RGATEMOD & 0: Turn off gate electrode resistance (without ge node); 1: Turn on gate electrode resistance (with ge node) & --- & 0 \\ \hline
RGEOA & Fitting parameter for RGEOMOD = 1 & --- & 1 \\ \hline
RGEOB & Fitting parameter for RGEOMOD = 1 & --- & 0 \\ \hline
RGEOC & Fitting parameter for RGEOMOD = 1 & --- & 0 \\ \hline
RGEOD & Fitting parameter for RGEOMOD = 1 & --- & 0 \\ \hline
RGEOE & Fitting parameter for RGEOMOD = 1 & --- & 0 \\ \hline
RGEOMOD & Geometry-dependent source/drain resistance; 0: RSH-based; 1: Holistic & --- & 0 \\ \hline
RGEXT & Effective gate electrode external resistance & --- & 0 \\ \hline
RGFIN & Effective gate electrode per finger per fin resistance & --- & 0.001 \\ \hline
RHOC & Contact resistivity at the silicon/silicide interface & --- & 1e-12 \\ \hline
RHORSD & If non-zero, average resistivity of silicon in the raised source/drain region & --- & 1 \\ \hline
RNOIA & Empirical parameter for Sid level & --- & 0.5774 \\ \hline
RNOIB & Empirical parameter for Sig level & --- & 0.3652 \\ \hline
RNOIC & Empirical parameter for correlation coefficient & --- & 0.3953 \\ \hline
RNOIK & Empirical parameter for Sid level at low Ids & --- & 0 \\ \hline
RSDR & Source-side drift resistance parameter - forward mode & --- & 0 \\ \hline
RSDRR & Source-side drift resistance parameter - reverse mode & --- & 0 \\ \hline
RSHD & Drain-side sheet resistance & --- & 0 \\ \hline
RSHS & Source-side sheet resistance & --- & 0 \\ \hline
RSLCW & 'R'esistance of the 'S'ource region at 'L'ow 'C'urrent & --- & 0 \\ \hline
RSW & RDSMOD = 1 zero bias source extension resistance per unit width & --- & 50 \\ \hline
RSWMIN & RDSMOD = 1 source extension resistance per unit width at high Vgs & --- & 0 \\ \hline
RTH0 & Thermal resistance & --- & 0.01 \\ \hline
SDTERM & Indicator of whether the source/drain are terminated with silicide & --- & 0 \\ \hline
SH\_WARN & 0: Disable self-heating warnings; 1: Enable self-heating warnings & --- & 0 \\ \hline
SHMOD & 0: Turn off self-heating; 1: Turn on self-heating & --- & 0 \\ \hline
SII0 & Vgs dependence parameter of Iii & V$^{-1}$ & 0.5 \\ \hline
SII1 & 1st Vgs dependence parameter of Iii & --- & 0.1 \\ \hline
SII2 & 2nd Vgs dependence parameter of Iii & V & 0 \\ \hline
SIID & 3rd Vds dependence parameter of Iii & V & 0 \\ \hline
SJD & Constant for drain-side two-step second junction & --- & 0 \\ \hline
SJS & Constant for source-side two-step second junction & --- & 0 \\ \hline
SJSWD & Constant for drain-side sidewall two-step second junction & --- & 0 \\ \hline
SJSWGD & Constant for source-side gate sidewall two-step second junction & --- & 0 \\ \hline
SJSWGS & Constant for source-side gate sidewall two-step second junction & --- & 0 \\ \hline
SJSWS & Constant for source-side sidewall two-step second junction & --- & 0 \\ \hline
SMOOTH & Smoothing Parameter & --- & 2 \\ \hline
SPRT & CRYOMOD != 0 parameter for corner temperature smoothing in dual-slope temperature model of series resistance & --- & 0.01 \\ \hline
SSP1 & Subband smoothing parameter for 1st subband (WGAA<WSSP0) & --- & 14 \\ \hline
SSP2 & Subband smoothing parameter for 2nd subband (WGAA<WSSP0) & --- & 24 \\ \hline
SSP3 & Subband smoothing parameter for 3rd subband (WGAA<WSSP0) & --- & 24 \\ \hline
SUBBANDMOD & Switch for GAAFET quantum subband model (0: off; 1: on) & --- & 0 \\ \hline
TBGASUB & Bandgap temperature coefficient & --- & 0.000702 \\ \hline
TBGBSUB & Bandgap temperature coefficient & K & 1108 \\ \hline
TCJ & Temperature coefficient for CJS/CJD & --- & 0 \\ \hline
TCJSW & Temperature coefficient for CJSWS/CJSWD & --- & 0 \\ \hline
TCJSWG & Temperature coefficient for CJSWGS/CJSWGD & --- & 0 \\ \hline
TDWSE2 & TGAA dependence of WGAA scaling for 2nd subband energy & --- & 1 \\ \hline
TDWSE3 & TGAA dependence of WGAA scaling for 3rd subband energy & --- & 0.23 \\ \hline
TDWSQ1 & TGAA dependence of WGAA scaling for 1st subband charge & --- & 2.4 \\ \hline
TDWSQ2 & TGAA dependence of WGAA scaling for 2nd subband charge & --- & 2 \\ \hline
TDWSQ3 & TGAA dependence of WGAA scaling for 3rd subband charge & --- & 2.4 \\ \hline
TEMPMOD & 1: Change temperature dependence of specific parameters & --- & 0 \\ \hline
TETA0 & Temperature dependence of DIBL coefficient & --- & 0 \\ \hline
TETA0CV & CVMOD = 1 temperature dependence of DIBL coefficient & --- & 0 \\ \hline
TETA0R & Temperature dependence of reverse-mode DIBL coefficient & --- & 0 \\ \hline
TFIN & Fin thickness & m & 1.5e-08 \\ \hline
TFIN\_BASE & If non-zero, base fin thickness for trapezoidal triple gate & m & 0 \\ \hline
TFIN\_TOP & If non-zero, top fin thickness for trapezoidal triple gate & m & 0 \\ \hline
TGAA & Thickness of individual GAA bodies & m & 5e-09 \\ \hline
TGATE & Gate height on top of the hard mask & m & 3e-08 \\ \hline
TGIDL & GIDL/GISL temperature dependence & --- & -0.003 \\ \hline
THETADIBL & User-designated DIBL length dependence. & --- & 0 \\ \hline
THETASCE & User-designated Vth roll-off length dependence. & --- & 0 \\ \hline
THETASW & User-designated subthreshold swing length dependence. & --- & 0 \\ \hline
TII & Impact ionization temperature dependence for IIMOD = 2 & --- & 0 \\ \hline
TLOW & CRYOMOD != 0 transition temperature of subthreshold swing at low temperatures & K & 50 \\ \hline
TLOW1 & CRYOMOD != 0 transition temperature of subthreshold swing at low temperatures & K & 0 \\ \hline
TMASK & Height of hard mask on top of the fin & m & 3e-08 \\ \hline
TMEXP & Temperature coefficient for Vdseff smoothing & --- & 0 \\ \hline
TMEXP2 & CRYOMOD != 0 temperature coefficient for Vdseff smoothing & --- & -4e-06 \\ \hline
TMEXPR & Reverse-mode temperature coefficient for Vdseff smoothing & --- & 0 \\ \hline
TNJTS & Temperature coefficient for NJTS & --- & 0 \\ \hline
TNJTSD & Temperature coefficient for NJTSD & --- & 0 \\ \hline
TNJTSSW & Temperature coefficient for NJTSSW & --- & 0 \\ \hline
TNJTSSWD & Temperature coefficient for NJTSSWD & --- & 0 \\ \hline
TNJTSSWG & Temperature coefficient for NJTSSWG & --- & 0 \\ \hline
TNJTSSWGD & Temperature coefficient for NJTSSWGD & --- & 0 \\ \hline
TNOIA & Empirical parameter for Leff trend of Sid & --- & 0 \\ \hline
TNOIB & Empirical parameter for Leff trend of Sig & --- & 0 \\ \hline
TNOIC & Empirical parameter for Leff trend of correlation coefficient & --- & 0 \\ \hline
TNOIK & Empirical parameter for Leff trend of Sid at low Ids & --- & 0 \\ \hline
TNOIK2 & Empirical parameter for sensitivity of RNOIK & --- & 0.1 \\ \hline
TNOIMOD & 0: Charge-based, 1: Correlated thermal noise model & --- & 0 \\ \hline
TNOM & Temperature at which the model is extracted & $^\circ$C & 27 \\ \hline
TOXG & Oxide thickness for gate current model & m & 0 \\ \hline
TOXP & Physical oxide thickness & m & 1.2e-09 \\ \hline
TOXREF & Target tox value & m & 1.2e-09 \\ \hline
TPB & Temperature coefficient for PBS/PBD & --- & 0 \\ \hline
TPBSW & Temperature coefficient for PBSWS/PBSWD & --- & 0 \\ \hline
TPBSWG & Temperature coefficient for PBSWGS/PBSWGD & --- & 0 \\ \hline
TR0 & CRYOMOD != 0 corner temperature in dual-slope temperature model of series resistance & K & 170 \\ \hline
TRDDR & Drain-side drift resistance temperature coefficient & --- & 0 \\ \hline
TRSDR & Source-side drift resistance temperature coefficient & --- & 0 \\ \hline
TSILI & Thickness of the silicide on top of the raised source/drain & m & 1e-08 \\ \hline
TSRE2 & TGAA scaling for 2nd subband energy & --- & 1.8 \\ \hline
TSRE3 & TGAA scaling for 3rd subband energy  & --- & 0.67 \\ \hline
TSRQ1 & TGAA scaling for 1st subband charge & --- & 1.1 \\ \hline
TSRQ2 & TGAA scaling for 2nd subband charge & --- & 2 \\ \hline
TSRQ3 & TGAA scaling for 3rd subband charge & --- & 6 \\ \hline
TSS & Swing temperature coefficient & --- & 0 \\ \hline
TSUS & Separation between GAA bodies & m & 2e-09 \\ \hline
TVTH & CRYOMOD != 0 transition temperature for Vth temperature model & K & 40 \\ \hline
TYPE & 1: NMOS; -1: PMOS & --- & 1 \\ \hline
U0 & Low-field mobility & --- & 0.03 \\ \hline
U0CV & CVMOD = 1 low-field mobility & --- & 0 \\ \hline
U0EMSM1 & Parameter for effective mass scaling & --- & 26.6 \\ \hline
U0EMSM2 & Parameter for effective mass scaling & --- & 4 \\ \hline
U0ETAWSC & Ratio of primary carrier low-field mobilities: U0\_[110]/U0\_[100] & --- & 1.5 \\ \hline
U0LT & Coupled NFIN and length dependence of U0 & --- & 0 \\ \hline
U0LTCV & Coupled NFIN and length dependence of U0CV & --- & 0 \\ \hline
U0MULT & Variability in carrier mobility & --- & 1 \\ \hline
U0N1 & NFIN dependence of U0 & --- & 0 \\ \hline
U0N1CV & CVMOD = 1 NFIN dependence of U0CV & --- & 0 \\ \hline
U0N1R & Reverse-mode NFIN dependence of U0 & --- & 0 \\ \hline
U0N2 & NFIN dependence of U0 & --- & 100000 \\ \hline
U0N2CV & CVMOD=1 NFIN dependence of U0CV & --- & 0 \\ \hline
U0N2R & Reverse-mode NFIN dependence of U0 & --- & 0 \\ \hline
U0R & Reverse-mode low-field mobility & --- & 0 \\ \hline
UA & Phonon/surface roughness scattering parameter & --- & 0.3 \\ \hline
UA1 & Mobility temperature coefficient for UA & --- & 0.001032 \\ \hline
UA1CV & CVMOD = 1 mobility temperature coefficient for UA & --- & 0 \\ \hline
UA1R & Reverse-mode mobility temperature coefficient for UA & --- & 0 \\ \hline
UA2 & CRYOMOD != 0 mobility temperature coefficient for UA & --- & -0.04 \\ \hline
UA2CV & CRYOMOD != 0 and CVMOD = 1 mobility temperature coefficient for UA & --- & 0 \\ \hline
UACV & CVMOD = 1 phonon/surface roughness scattering parameter & --- & 0 \\ \hline
UAIR & Ideality parameter & --- & 0.2 \\ \hline
UAR & Reverse-mode phonon/surface roughness scattering parameter & --- & 0 \\ \hline
UARTSC & Rate of UA decay with TGAA scaling & --- & 0.09 \\ \hline
UATHIN & Phonon/surface roughness scattering parameter for thin GAA bodies & --- & 0 \\ \hline
UATNI & Critical TGAA for non-ideality & m & 6.4e-09 \\ \hline
UATSAT & Critical TGAA for UA saturation & m & 9e-09 \\ \hline
UC & Body effect for mobility degradation parameter - BULKMOD = 1 or 2 & --- & 0 \\ \hline
UC1 & Mobility temperature coefficient for UC & --- & 5.6e-11 \\ \hline
UC1CV & CVMOD = 1 mobility temperature coefficient for UC & --- & 0 \\ \hline
UC1R & Reverse-mode mobility temperature coefficient for UC & --- & 0 \\ \hline
UCCV & CVMOD = 1 body effect for mobility degradation parameter (BULKMOD = 1 or 2) & --- & 0 \\ \hline
UCR & Reverse-mode body effect for mobility degradation parameter - BULKMOD = 1 or 2 & --- & 0 \\ \hline
UCS & Coulomb scattering parameter & --- & 1 \\ \hline
UCSTE & Mobility temperature coefficient & --- & -0.004775 \\ \hline
UCSTE1 & CRYOMOD != 0 mobility temperature coefficient for UCS & --- & -0.04 \\ \hline
UD & Coulomb scattering parameter & --- & 0 \\ \hline
UD1 & Mobility temperature coefficient for UD & --- & 0 \\ \hline
UD1CV & CVMOD = 1 mobility temperature coefficient for UD & --- & 0 \\ \hline
UD1R & Reverse-mode mobility temperature coefficient for UD & --- & 0 \\ \hline
UD2 & CRYOMOD != 0 mobility temperature coefficient for UD & --- & -0.04 \\ \hline
UD2CV & CRYOMOD != 0 and CVMOD = 1 mobility temperature coefficient for UD & --- & 0 \\ \hline
UDCV & CVMOD = 1 coulomb scattering parameter & --- & 0 \\ \hline
UDD & CRYOMOD != 0 weight factor correction for drain side inversion charge density in Coulomb scattering model & --- & -2e-05 \\ \hline
UDD1 & CRYOMOD != 0 temperature coefficient for UDD & --- & -10 \\ \hline
UDPTSC & Exponent for UD power law & --- & 1.3 \\ \hline
UDR & Reverse-mode Coulomb scattering parameter & --- & 0 \\ \hline
UDS & CRYOMOD != 0 weight factor correction for source side inversion charge density in Coulomb scattering model & --- & 2e-05 \\ \hline
UDS1 & CRYOMOD != 0 temperature coefficient for UDS & --- & -10 \\ \hline
UDTHIN & Coulomb scattering parameter for thin GAA bodies & --- & 0 \\ \hline
UDTSAT & Crtitcal TGAA for UD saturation & m & 8.1e-09 \\ \hline
UP & Mobility L coefficient & --- & 0 \\ \hline
UPR & Reverse-mode mobility L coefficient & --- & 0 \\ \hline
UTE & Mobility temperature coefficient & --- & 0 \\ \hline
UTE1 & CRYOMOD != 0 mobility temperature coefficient & --- & -0.4 \\ \hline
UTE1CV & CRYOMOD != 0 and CVMOD = 1 mobility temperature coefficient & --- & 0 \\ \hline
UTECV & CVMOD = 1 mobility temperature coefficient & --- & 0 \\ \hline
UTER & Reverse-mode for mobility temperature coefficient & --- & 0 \\ \hline
UTL & Mobility temperature coefficient & --- & -0.0015 \\ \hline
UTLCV & CVMOD = 1 mobility temperature coefficient & --- & 0 \\ \hline
UTLR & Reverse-mode for mobility temperature coefficient & --- & 0 \\ \hline
VFBSD & User-designated flatband voltage for S/D region & V & 0 \\ \hline
VFBSDCV & User-designated flatband voltage for S/D region for C-V calculations & V & 0 \\ \hline
VSAT & Saturation velocity for the saturation region & --- & 85000 \\ \hline
VSAT1 & Velocity saturation parameter for Ion degradation - forward mode & --- & 0 \\ \hline
VSAT1N1 & NFIN dependence of VSAT1 & --- & 0 \\ \hline
VSAT1N2 & NFIN dependence of VSAT1 & --- & 0 \\ \hline
VSAT1R & Velocity saturation parameter for Ion degradation - reverse mode & --- & 0 \\ \hline
VSAT1RN1 & NFIN dependence of VSAT1R & --- & 0 \\ \hline
VSAT1RN2 & NFIN dependence of VSAT1R & --- & 0 \\ \hline
VSATCV & Velocity saturation parameter for CV & --- & 0 \\ \hline
VSATN1 & NFIN dependence of VSAT & --- & 0 \\ \hline
VSATN2 & NFIN dependence of VSAT & --- & 100000 \\ \hline
VSATR & Saturation velocity for the saturation region in the reverse mode & --- & 0 \\ \hline
VSATRSD & Saturation velocity in S/D region & --- & 100000 \\ \hline
VSRDFACTOR & Parameter for delta\_vsrd tuning & --- & 0.001 \\ \hline
VSRSFACTOR & Parameter for delta\_vsrs tuning & --- & 0.001 \\ \hline
VTSD & Bottom drain junction trap-assisted current voltage dependent parameter & V & 0 \\ \hline
VTSS & Bottom source junction trap-assisted current voltage dependent parameter & V & 10 \\ \hline
VTSSWD & Unit length trap-assisted current voltage dependent parameter for sidewall drain junction & V & 0 \\ \hline
VTSSWGD & Unit length trap-assisted current voltage dependent parameter for gate-edge sidewall drain junction & V & 0 \\ \hline
VTSSWGS & Unit length trap-assisted current voltage dependent parameter for gate-edge sidewall source junction & V & 10 \\ \hline
VTSSWS & Unit length trap-assisted current voltage dependent parameter for sidewall source junction & V & 10 \\ \hline
W\_UFCM & Effective channel width for the unified model & m & 1 \\ \hline
WA1 & W-term of A1 & --- & 0 \\ \hline
WA11 & W-term of A11 & --- & 0 \\ \hline
WA2 & W-term of A2 & m/V & 0 \\ \hline
WA21 & W-term of A21 & --- & 0 \\ \hline
WAGIDL & W-term of AGIDL & --- & 0 \\ \hline
WAGIDLB & W-term of AGIDLB & --- & 0 \\ \hline
WAGISL & W-term of AGISL & --- & 0 \\ \hline
WAGISLB & W-term of AGISLB & --- & 0 \\ \hline
WAIGBACC & W-term of AIGBACC & --- & 0 \\ \hline
WAIGBACC1 & W-term of AIGBACC1 & --- & 0 \\ \hline
WAIGBINV & W-term of AIGBINV & --- & 0 \\ \hline
WAIGBINV1 & W-term of AIGBINV1 & --- & 0 \\ \hline
WAIGC & W-term of AIGC & --- & 0 \\ \hline
WAIGC1 & W-term of AIGC1 & --- & 0 \\ \hline
WAIGD & W-term of AIGD & --- & 0 \\ \hline
WAIGD1 & W-term of AIGD1 & --- & 0 \\ \hline
WAIGEN & W-term of AIGEN & --- & 0 \\ \hline
WAIGS & W-term of AIGS & --- & 0 \\ \hline
WAIGS1 & W-term of AIGS1 & --- & 0 \\ \hline
WALPHA0 & W-term of ALPHA0 & --- & 0 \\ \hline
WALPHA1 & W-term of ALPHA1 & m/V & 0 \\ \hline
WALPHAII0 & W-term of ALPHAII0 & --- & 0 \\ \hline
WALPHAII1 & W-term of ALPHAII1 & m/V & 0 \\ \hline
WAT & W-term of AT & --- & 0 \\ \hline
WATCV & W-term of ATCV & --- & 0 \\ \hline
WATR & W-term of ATR & --- & 0 \\ \hline
WBETA0 & W-term of BETA0 & m/V & 0 \\ \hline
WBETAII0 & W-term of BETAII0 & m/V & 0 \\ \hline
WBETAII1 & W-term of BETAII1 & m & 0 \\ \hline
WBETAII2 & W-term of BETAII2 & --- & 0 \\ \hline
WBGIDL & W-term of BGIDL & V & 0 \\ \hline
WBGIDLB & W-term of BGIDLB & V & 0 \\ \hline
WBGISL & W-term of BGISL & V & 0 \\ \hline
WBGISLB & W-term of BGISLB & V & 0 \\ \hline
WBIGBACC & W-term of BIGBACC & --- & 0 \\ \hline
WBIGBINV & W-term of BIGBINV & --- & 0 \\ \hline
WBIGC & W-term of BIGC & --- & 0 \\ \hline
WBIGD & W-term of BIGD & --- & 0 \\ \hline
WBIGEN & W-term of BIGEN & --- & 0 \\ \hline
WBIGS & W-term of BIGS & --- & 0 \\ \hline
WCDSC & W-term of CDSC & --- & 0 \\ \hline
WCDSCD & W-term of CDSCD & --- & 0 \\ \hline
WCDSCDR & W-term of CDSCDR & --- & 0 \\ \hline
WCFD & W-term of CFD & F & 0 \\ \hline
WCFS & W-term of CFS & F & 0 \\ \hline
WCGBL & W-term of CGBL & F & 0 \\ \hline
WCGDL & W-term of CGDL & F & 0 \\ \hline
WCGIDL & W-term of CGIDL & --- & 0 \\ \hline
WCGIDLB & W-term of CGIDLB & --- & 0 \\ \hline
WCGISL & W-term of CGISL & --- & 0 \\ \hline
WCGISLB & W-term of CGISLB & --- & 0 \\ \hline
WCGSL & W-term of CGSL & F & 0 \\ \hline
WCIGBACC & W-term of CIGBACC & m/V & 0 \\ \hline
WCIGBINV & W-term of CIGBINV & m/V & 0 \\ \hline
WCIGC & W-term of CIGC & m/V & 0 \\ \hline
WCIGD & W-term of CIGD & m/V & 0 \\ \hline
WCIGS & W-term of CIGS & m/V & 0 \\ \hline
WCIT & W-term of CIT & --- & 0 \\ \hline
WCITR & W-term of CITR & --- & 0 \\ \hline
WCKAPPAB & W-term of CKAPPAB & --- & 0 \\ \hline
WCKAPPAD & W-term of CKAPPAD & --- & 0 \\ \hline
WCKAPPAS & W-term of CKAPPAS & --- & 0 \\ \hline
WCOVD & W-term of COVD & F & 0 \\ \hline
WCOVS & W-term of COVS & F & 0 \\ \hline
WDELTAVSAT & W-term of DELTAVSAT & m & 0 \\ \hline
WDELTAVSATCV & W-term of DELTAVSATCV & m & 0 \\ \hline
WDIM0 & WGAA at which dimension change happens & m & 9.5e-09 \\ \hline
WDIMENSION1 & W-term of DIMENSION1 & m & 0 \\ \hline
WDIMENSION2 & W-term of DIMENSION2 & m & 0 \\ \hline
WDIMENSION3 & W-term of DIMENSION3 & m & 0 \\ \hline
WDIMR & Rate of dimension change with WGAA scaling & --- & 0.1 \\ \hline
WDROUT & W-term of DROUT & m & 0 \\ \hline
WDSUB & W-term of DSUB & m & 0 \\ \hline
WDVT0 & W-term of DVT0 & m & 0 \\ \hline
WDVT1 & W-term of DVT1 & m & 0 \\ \hline
WDVT1SS & W-term of DVT1SS & m & 0 \\ \hline
WDVTP0 & W-term of DVTP0 & m & 0 \\ \hline
WDVTP1 & W-term of DVTP1 & m & 0 \\ \hline
WDVTSHIFT & W-term of DVTSHIFT & --- & 0 \\ \hline
WDVTSHIFTR & W-term of DVTSHIFTR & --- & 0 \\ \hline
WDWBIN & W-term of DWBIN & m$^{2}$ & 0 \\ \hline
WE2NOM & W-term of E2NOM & --- & 0 \\ \hline
WE3NOM & W-term of E3NOM & --- & 0 \\ \hline
WEGIDL & W-term of EGIDL & --- & 0 \\ \hline
WEGIDLB & W-term of EGIDLB & --- & 0 \\ \hline
WEGISL & W-term of EGISL & --- & 0 \\ \hline
WEGISLB & W-term of EGISLB & --- & 0 \\ \hline
WEIGBINV & W-term of EIGBINV & --- & 0 \\ \hline
WEMOBT & W-term of EMOBT & m & 0 \\ \hline
WESATII & W-term of ESATII & V & 0 \\ \hline
WETA0 & W-term of ETA0 & m & 0 \\ \hline
WETA0CV & W-term of ETA0CV & m & 0 \\ \hline
WETA0R & W-term of ETA0R & m & 0 \\ \hline
WETAMOB & W-term of ETAMOB & m & 0 \\ \hline
WEU & W-term of EU & --- & 0 \\ \hline
WEU1 & W-term of EU1 & m & 0 \\ \hline
WEUR & W-term of EUR & --- & 0 \\ \hline
WGAA & GAA body width & m & 6e-09 \\ \hline
WGAANOM & Nominal WGAA & m$^{2}$ & 8e-09 \\ \hline
WIGT & W-term of IGT & m & 0 \\ \hline
WIIT & W-term of IIT & m & 0 \\ \hline
WK0 & W-term of K0 & --- & 0 \\ \hline
WK01 & W-term of K01 & --- & 0 \\ \hline
WK0SI & W-term of K0SI & m & 0 \\ \hline
WK0SI1 & W-term of K0SI1 & --- & 0 \\ \hline
WK0SISAT & W-term of K0SISAT & m & 0 \\ \hline
WK0SISAT1 & W-term of K0SISAT1 & m & 0 \\ \hline
WK1 & W-term of K1 & --- & 0 \\ \hline
WK11 & W-term of K11 & --- & 0 \\ \hline
WK1RSCE & W-term of K1RSCE & --- & 0 \\ \hline
WK2 & W-term of K2 & m & 0 \\ \hline
WK21 & W-term of K21 & m & 0 \\ \hline
WK2SAT & W-term of K2SAT & m & 0 \\ \hline
WK2SAT1 & W-term of K2SAT1 & m & 0 \\ \hline
WK2SI & W-term of K2SI & m & 0 \\ \hline
WK2SI1 & W-term of K2SI1 & --- & 0 \\ \hline
WK2SISAT & W-term of K2SISAT & m & 0 \\ \hline
WK2SISAT1 & W-term of K2SISAT1 & m & 0 \\ \hline
WKSATIV & W-term of KSATIV & m & 0 \\ \hline
WKSATIVR & W-term of KSATIVR & m & 0 \\ \hline
WKT1 & W-term of KT1 & --- & 0 \\ \hline
WLII & W-term of LII & --- & 0 \\ \hline
WLPE0 & W-term of LPE0 & m$^{2}$ & 0 \\ \hline
WMAX & Maximum width for which this model should be used & m & 100 \\ \hline
WMEXP & W-term of MEXP & m & 0 \\ \hline
WMEXPR & W-term of MEXPR & m & 0 \\ \hline
WMFQ1NOM & W-term of MFQ1NOM & m & 0 \\ \hline
WMFQ2NOM & W-term of MFQ2NOM & m & 0 \\ \hline
WMFQ3NOM & W-term of MFQ3NOM & m & 0 \\ \hline
WMIN & Minimum width for which this model should be used & m & 0 \\ \hline
WMPOWER & W-term for MPOWER & m & 0 \\ \hline
WNGATE & W-term of NGATE & --- & 0 \\ \hline
WNIGBACC & W-term of NIGBACC & m & 0 \\ \hline
WNIGBINV & W-term of NIGBINV & m & 0 \\ \hline
WNOIA2 & W-term for NOIA2 & --- & 0 \\ \hline
WNTGEN & W-term of NTGEN & m & 0 \\ \hline
WNTOX & W-term of NTOX & m & 0 \\ \hline
WPCLM & W-term of PCLM & m & 0 \\ \hline
WPCLMCV & W-term of PCLMCV & m & 0 \\ \hline
WPCLMG & W-term of PCLMG & m/V & 0 \\ \hline
WPCLMR & W-term of PCLMR & m & 0 \\ \hline
WPDIBL1 & W-term of PDIBL1 & m & 0 \\ \hline
WPDIBL1R & W-term of PDIBL1R & m & 0 \\ \hline
WPDIBL2 & W-term of PDIBL2 & m & 0 \\ \hline
WPDIBL2R & W-term of PDIBL2R & m & 0 \\ \hline
WPGIDL & W-term of PGIDL & m & 0 \\ \hline
WPGIDLB & W-term of PGIDLB & m & 0 \\ \hline
WPGISL & W-term of PGISL & m & 0 \\ \hline
WPGISLB & W-term of PGISLB & m & 0 \\ \hline
WPHIBE & W-term of PHIBE & --- & 0 \\ \hline
WPHIG & W-term of PHIG & --- & 0 \\ \hline
WPHIN & W-term of PHIN & --- & 0 \\ \hline
WPIGCD & W-term of PIGCD & m & 0 \\ \hline
WPOXEDGE & W-term of POXEDGE & m & 0 \\ \hline
WPRT & W-term of PRT & --- & 0 \\ \hline
WPRT1 & W-term of PRT1 & --- & 0 \\ \hline
WPRWGD & W-term of PRWGD & m/V & 0 \\ \hline
WPRWGS & W-term of PRWGS & m/V & 0 \\ \hline
WPSAT & W-term of PSAT & m & 0 \\ \hline
WPSATCV & W-term of PSATCV & m & 0 \\ \hline
WPTWG & W-term of PTWG & --- & 0 \\ \hline
WPTWGR & W-term of PTWGR & --- & 0 \\ \hline
WPTWGT & W-term of PTWGT & --- & 0 \\ \hline
WPVAG & W-term of PVAG & m & 0 \\ \hline
WQMFACTOR & W-term of QMFACTOR & m & 0 \\ \hline
WQMTCENCV & W-term of QMTCENCV & m & 0 \\ \hline
WQMTCENCVA & W-term of QMTCENCVA & m & 0 \\ \hline
WQSREF & W-term for QSREF & m & 0 \\ \hline
WR & W dependence parameter of S/D extension resistance & --- & 1 \\ \hline
WRDSW & W-term of RDSW & --- & 0 \\ \hline
WRDW & W-term of RDW & --- & 0 \\ \hline
WRSW & W-term of RSW & --- & 0 \\ \hline
WSFE2 & WGAA scaling factor for 2nd subband energy & --- & 1 \\ \hline
WSFE3 & WGAA scaling factor for 3rd subband energy & --- & 1 \\ \hline
WSFQ1 & WGAA scaling factor for 1st subband charge & --- & 1 \\ \hline
WSFQ2 & WGAA scaling factor for 2nd subband charge & --- & 1 \\ \hline
WSFQ3 & WGAA scaling factor for 3rd subband charge & --- & 1 \\ \hline
WSII0 & W-term of SII0 & m/V & 0 \\ \hline
WSII1 & W-term of SII1 & m & 0 \\ \hline
WSII2 & W-term of SII2 & --- & 0 \\ \hline
WSIID & W-term of SIID & --- & 0 \\ \hline
WSPRT & W-term of SPRT & m & 0 \\ \hline
WSSP0 & WGAA around which SSP change happens & m & 0 \\ \hline
WSSP1 & W-term of SSP1 & m & 0 \\ \hline
WSSP2 & W-term of SSP2 & m & 0 \\ \hline
WSSP3 & W-term of SSP3 & m & 0 \\ \hline
WSSPR & Rate of SSP change with WGAA scaling & --- & 0 \\ \hline
WTGIDL & W-term of TGIDL & --- & 0 \\ \hline
WTH0 & Width dependence coefficient for Rth and Cth & m & 0 \\ \hline
WTII & W-term of TII & m & 0 \\ \hline
WTR0 & W-term of TR0 & --- & 0 \\ \hline
WTSS & W-term of TSS & --- & 0 \\ \hline
WU0 & W-term of U0 & --- & 0 \\ \hline
WU0CV & W-term of U0CV & --- & 0 \\ \hline
WU0R & W-term of U0R & --- & 0 \\ \hline
WUA & W-term of UA & --- & 0 \\ \hline
WUA1 & W-term of UA1 & m & 0 \\ \hline
WUA1CV & W-term of UA1CV & m & 0 \\ \hline
WUA1R & W-term of UA1R & m & 0 \\ \hline
WUA2 & W-term of UA2 & m & 0 \\ \hline
WUA2CV & W-term of UA2CV & m & 0 \\ \hline
WUACV & W-term of UACV & --- & 0 \\ \hline
WUAR & W-term of UAR & --- & 0 \\ \hline
WUC & W-term of UC & --- & 0 \\ \hline
WUC1 & W-term of UC1 & m & 0 \\ \hline
WUC1CV & W-term of UC1CV & m & 0 \\ \hline
WUC1R & W-term of UC1R & m & 0 \\ \hline
WUCCV & W-term of UCCV & --- & 0 \\ \hline
WUCR & W-term of UCR & --- & 0 \\ \hline
WUCS & W-term of UCS & m & 0 \\ \hline
WUCSTE & W-term of UCSTE & m & 0 \\ \hline
WUCSTE1 & W-term of UCSTE1 & m & 0 \\ \hline
WUD & W-term of UD & --- & 0 \\ \hline
WUD1 & W-term of UD1 & m & 0 \\ \hline
WUD1CV & W-term of UD1CV & m & 0 \\ \hline
WUD1R & W-term of UD1R & m & 0 \\ \hline
WUD2 & W-term of UD2 & m & 0 \\ \hline
WUD2CV & W-term of UD2CV & m & 0 \\ \hline
WUDCV & W-term of UDCV & --- & 0 \\ \hline
WUDD & W-term of UDD & m & 0 \\ \hline
WUDD1 & W-term of UDD1 & m & 0 \\ \hline
WUDR & W-term of UDR & --- & 0 \\ \hline
WUDS & W-term of UDS & m & 0 \\ \hline
WUDS1 & W-term of UDS1 & m & 0 \\ \hline
WUP & W-term of UP & --- & 0 \\ \hline
WUPR & W-term of UPR & --- & 0 \\ \hline
WUTE & W-term of UTE & m & 0 \\ \hline
WUTE1 & W-term of UTE1 & m & 0 \\ \hline
WUTE1CV & W-term of UTE1CV & m & 0 \\ \hline
WUTECV & W-term of UTECV & m & 0 \\ \hline
WUTER & W-term of UTER & m & 0 \\ \hline
WUTL & W-term of UTL & m & 0 \\ \hline
WUTLCV & W-term of UTLCV & m & 0 \\ \hline
WUTLR & W-term of UTLR & m & 0 \\ \hline
WVSAT & W-term of VSAT & --- & 0 \\ \hline
WVSAT1 & W-term of VSAT1 & --- & 0 \\ \hline
WVSAT1R & W-term of VSAT1R & --- & 0 \\ \hline
WVSATCV & W-term of VSATCV & --- & 0 \\ \hline
WVSATR & W-term of VSATR & --- & 0 \\ \hline
WWR & W-term of WR & m & 0 \\ \hline
WXRCRG1 & W-term of XRCRG1 & m & 0 \\ \hline
WXRCRG2 & W-term of XRCRG2 & m & 0 \\ \hline
WXW & W-term of XW & m$^{2}$ & 0 \\ \hline
XJBVD & Fitting parameter for drain diode breakdown current & --- & 0 \\ \hline
XJBVS & Fitting parameter for source diode breakdown current & --- & 1 \\ \hline
XRCRG1 & Parameter for non-quasistatic gate resistance NQSMOD = 1, 2 & --- & 12 \\ \hline
XRCRG2 & Parameter for non-quasistatic gate resistance NQSMOD = 1, 2 & --- & 1 \\ \hline
XTID & Drain junction current temperature exponent & --- & 0 \\ \hline
XTIS & Source junction current temperature exponent & --- & 3 \\ \hline
XTSD & Power dependence of JTSD on temperature & --- & 0 \\ \hline
XTSS & Power dependence of JTSS on temperature & --- & 0.02 \\ \hline
XTSSWD & Power dependence of JTSSWD on temperature & --- & 0 \\ \hline
XTSSWGD & Power dependence of JTSSWGD on temperature & --- & 0 \\ \hline
XTSSWGS & Power dependence of JTSSWGS on temperature & --- & 0.02 \\ \hline
XTSSWS & Power dependence of JTSSWS on temperature & --- & 0.02 \\ \hline
\end{DeviceParamTableGenerated}

%table generated from Verilog-A input
\index{MOSFET level 111!device output variables}
\begin{DeviceParamTableGenerated}{MOSFET level 111 Output Variables}{M_111_OutputVars}
WEFF & Effective width for I-V &   m & none \\ \hline
LEFF & Effective length for I-V &   m & none \\ \hline
WEFFCV & Effective width for C-V &   m & none \\ \hline
LEFFCV & Effective length for C-V &   m & none \\ \hline
IDS & Intrinsic drain current &   A & none \\ \hline
IDEFF & Total drain current &   A & none \\ \hline
ISEFF & Total source current &   A & none \\ \hline
IGTOT & Total gate current &   A & none \\ \hline
IDSGEN & Generation-recombination current &   A & none \\ \hline
III & Impact ionization current &   A & none \\ \hline
IGIDL & GIDL current &   A & none \\ \hline
IGISL & GISL current &   A & none \\ \hline
IJSB & Source-to-substrate junction current &   A & none \\ \hline
IJDB & Drain-to-substrate junction current &   A & none \\ \hline
ISUB & Substrate current &   A & none \\ \hline
BETA & Drain current prefactor per fin per finger &   A/V$^{2}$ & none \\ \hline
VDSSAT & Drain-to-source saturation voltage &   V & none \\ \hline
VDSEFF & Effective drain-to-source voltage &   V & none \\ \hline
VFB & Flatband voltage &   V & none \\ \hline
VTH & Threshold voltage &   V & none \\ \hline
GM & Transconductance &   A/V & none \\ \hline
GDS & Output conductance &   A/V & none \\ \hline
GMBS & Substrate conductance &   A/V & none \\ \hline
QGI & Intrinsic gate charge &   C & none \\ \hline
QDI & Intrinsic drain charge &   C & none \\ \hline
QSI & Intrinsic source charge &   C & none \\ \hline
QBI & Intrinsic substrate charge &   C & none \\ \hline
QG & Total gate charge &   C & none \\ \hline
QD & Total drain charge &   C & none \\ \hline
QS & Total source charge &   C & none \\ \hline
QB & Total substrate charge &   C & none \\ \hline
CGGI & Intrinsic gate capacitance &   F & none \\ \hline
CGSI & Intrinsic gate-to-source capacitance &   F & none \\ \hline
CGDI & Intrinsic gate-to-drain capacitance &   F & none \\ \hline
CGEI & Intrinsic gate-to-substrate capacitance &   F & none \\ \hline
CDGI & Intrinsic drain-to-gate capacitance &   F & none \\ \hline
CDDI & Intrinsic drain capacitance &   F & none \\ \hline
CDSI & Intrinsic drain-to-source capacitance &   F & none \\ \hline
CDEI & Intrinsic drain-to-substrate capacitance &   F & none \\ \hline
CSGI & Intrinsic source-to-gate capacitance &   F & none \\ \hline
CSDI & Intrinsic source-to-drain capacitance &   F & none \\ \hline
CSSI & Intrinsic source capacitance &   F & none \\ \hline
CSEI & Intrinsic source-to-substrate capacitance &   F & none \\ \hline
CEGI & Intrinsic substrate-to-gate capacitance &   F & none \\ \hline
CEDI & Intrinsic substrate-to-drain capacitance &   F & none \\ \hline
CESI & Intrinsic substrate-to-source capacitance &   F & none \\ \hline
CEEI & Intrinsic substrate capacitance &   F & none \\ \hline
CGG & Total gate capacitance &   F & none \\ \hline
CGS & Total gate-to-source capacitance &   F & none \\ \hline
CGD & Total gate-to-drain capacitance &   F & none \\ \hline
CGE & Total gate-to-substrate capacitance &   F & none \\ \hline
CDG & Total drain-to-gate capacitance &   F & none \\ \hline
CDD & Total drain capacitance &   F & none \\ \hline
CDS & Total drain-to-source capacitance &   F & none \\ \hline
CDE & Total drain-to-substrate capacitance &   F & none \\ \hline
CSG & Total source-to-gate capacitance &   F & none \\ \hline
CSD & Total source-to-drain capacitance &   F & none \\ \hline
CSS & Total source capacitance &   F & none \\ \hline
CSE & Total source-to-substrate capacitance &   F & none \\ \hline
CEG & Total substrate-to-gate capacitance &   F & none \\ \hline
CED & Total substrate-to-drain capacitance &   F & none \\ \hline
CES & Total substrate-to-source capacitance &   F & none \\ \hline
CEE & Total substrate capacitance &   F & none \\ \hline
CGSEXT & External gate-to-source capacitance &   F & none \\ \hline
CGDEXT & External gate-to-drain capacitance &   F & none \\ \hline
CGBOV & Gate-to-substrate overlap capacitance &   F & none \\ \hline
CJST & Total junction and source-to-substrate capacitance &   F & none \\ \hline
CJDT & Total junction and drain-to-substrate capacitance &   F & none \\ \hline
RSGEO & External bias-independent source resistance &   Ohm & none \\ \hline
RDGEO & External bias-independent drain resistance &   Ohm & none \\ \hline
CFGEO & Geometric parasitic capacitance &   F & none \\ \hline
T\_TOTAL\_K & Device temperature in Kelvin &   K & none \\ \hline
T\_TOTAL\_C & Device temperature in Celsius &   degC & none \\ \hline
T\_DELTA\_SH & Delta temperature by self-heating &   K or degC & none \\ \hline
IGS & Gate-to-source tunneling current &   A & none \\ \hline
IGD & Gate-to-drain tunneling current &   A & none \\ \hline
IGCS & Gate-to-channel tunneling current at source &   A & none \\ \hline
IGCD & Gate-to-channel tunneling current at drain &   A & none \\ \hline
IGBS & Gate-to-substrate tunneling current at source &   A & none \\ \hline
IGBD & Gate-to-substrate tunneling current at drain &   A & none \\ \hline
IGBACC & Gate-to-substrate tunneling current in accumulation &   A & none \\ \hline
IGBINV & Gate-to-substrate tunneling current in inversion &   A & none \\ \hline
DIDSDVG & dIds / dVg &   A/V & none \\ \hline
DIDSDVS & dIds / dVs &   A/V & none \\ \hline
DIDSDVD & dIds / dVd &   A/V & none \\ \hline
DIGSDVG & dIgs / dVg &   A/V & none \\ \hline
DIGSDVS & dIgs / dVs &   A/V & none \\ \hline
DIGSDVD & dIgs / dVd &   A/V & none \\ \hline
DIGDDVG & dIgd / dVg &   A/V & none \\ \hline
DIGDDVS & dIgd / dVs &   A/V & none \\ \hline
DIGDDVD & dIgd / dVd &   A/V & none \\ \hline
DIIIDVG & dIii / dVg &   A/V & none \\ \hline
DIIIDVS & dIii / dVs &   A/V & none \\ \hline
DIIIDVD & dIii / dVd &   A/V & none \\ \hline
DIGIDLDVG & dIgidl / dVg &   A/V & none \\ \hline
DIGIDLDVS & dIgidl / dVs &   A/V & none \\ \hline
DIGIDLDVD & dIgidl / dVd &   A/V & none \\ \hline
DIGISLDVG & dIgisl / dVg &   A/V & none \\ \hline
DIGISLDVS & dIgisl / dVs &   A/V & none \\ \hline
DIGISLDVD & dIgisl / dVd &   A/V & none \\ \hline
CGT & dQg / dTemp &   C/K & none \\ \hline
CST & dQs / dTemp &   C/K & none \\ \hline
CDT & dQd / dTemp &   C/K & none \\ \hline
DIDSDVTH & dIds / dTemp &   A/K & none \\ \hline
DIGSDVTH & dIgs / dTemp &   A/K & none \\ \hline
DIGDDVTH & dIgd / dTemp &   A/K & none \\ \hline
DIIIDVTH & dIii / dTemp &   A/K & none \\ \hline
DIGIDLDVTH & dIgidl / dTemp &   A/K & none \\ \hline
DIGISLDVTH & dIgisl / dTemp &   A/K & none \\ \hline
ITH & Device power &   A*V & none \\ \hline
DITHDVTH & dPower / dTemp &   A*V/K & none \\ \hline
DITHDVG & dPower / dVg &   A & none \\ \hline
DITHDVS & dPower / dVs &   A & none \\ \hline
DITHDVD & dPower / dVd &   A & none \\ \hline
VDSAT & Synonym for VDSSAT &   V & none \\ \hline
VDS & Drain-source voltage &   V & none \\ \hline
VGS & Gate-source voltage &   V & none \\ \hline
VBS & Bulk-source voltage &   V & none \\ \hline
\end{DeviceParamTableGenerated}



\subsubsection{Level 110 MOSFET Tables (BSIM CMG version 110.0.0)}

\Xyce{} includes the legacy BSIM CMG Common Multi-gate model versions 110. 
This model has been superceded by the level 111 version, but has been retained for 
backward compatibility with previous versions of Xyce and older model cards and PDKs.
The code in \Xyce{} was generated from the BSIM group's Verilog-A
input using the default ``ifdef'' lines provided, and therefore
supports only the subset of BSIM CMG features those defaults enable.
Instance and model parameters for the BSIM CMG model are given in
tables~\ref{M_110_Device_Instance_Params} and
\ref{M_110_Device_Model_Params}.  Details of the model are documented
in the BSIM-CMG technical report\cite{BSIMCMG:Manual}, available from
the BSIM web site at
\url{http://bsim.berkeley.edu/models/bsimcmg/}.

The BSIM CMG devices support output of the internal variables in
tables~\ref{M_107_OutputVars}, \ref{M_108_OutputVars}, and  \ref{M_110_OutputVars} on the \texttt{.PRINT} line of a netlist.
To access them from a print line, use the syntax
\texttt{N(<instance>:<variable>)} where ``\texttt{<instance>}'' refers to the
name of the specific level 107 or 108 M device in your netlist.

\input{M_110_Device_Instance_Params}
\input{M_110_Device_Model_Params}
\input{M_110_OutputVars}

\subsubsection{Level 107  and 108 MOSFET Tables (BSIM CMG versions 107.0.0 and 108.0.0)}
\Xyce{} includes the legacy BSIM CMG Common Multi-gate model versions 107 and 108.
These models have been superceded by the level 110 version, but has been
retained for backward compatibility with previous versions of Xyce and
older model cards and PDKs.  The code in \Xyce{} was generated from the BSIM
group's Verilog-A input using the default ``ifdef'' lines provided,
and therefore supports only the subset of BSIM CMG features those
defaults enable.  Instance and model parameters for the BSIM CMG model
are given in tables~\ref{M_107_Device_Instance_Params},
\ref{M_107_Device_Model_Params}, \ref{M_108_Device_Instance_Params},
and~\ref{M_108_Device_Model_Params}.  Details of the model are documented
in the BSIM-CMG technical report\cite{BSIMCMG:Manual}, available from
the BSIM web site at \url{http://bsim.berkeley.edu/models/bsimcmg/}.

Note that the TNOIMOD=1 option of BSIM-CMG 108 is not supported in
Xyce, as it uses features of Verilog-A that are not supported in our
Verilog-A compiler.  This noise model was added in version 108 and
removed in version 109.  The TNOIMOD=2 option of BSIM-CMG 108 is the
same as the TNOIMOD=1 option of BSIM-CMG 110.

\input{M_107_Device_Instance_Params}
\input{M_107_Device_Model_Params}
\input{M_107_OutputVars}

\input{M_108_Device_Instance_Params}
\input{M_108_Device_Model_Params}
\input{M_108_OutputVars}


\clearpage
\subsubsection{Levels 2000 and 2001 MOSFET Tables (MVS version 2.0.0)}
\Xyce{} includes the MIT Virtual Source (MVS) MOSFET model version
2.0.0 in both ETSOI and HEMT variants.  The code in \Xyce{} was
generated from the MIT Verilog-A input.  Model parameters for the MVS
model are given in \ref{M_2000_Device_Model_Params} and
\ref{M_2001_Device_Model_Params}.  The MVS model does not have
instance parameters.  Details of the model are documented MVS
Nanotransistor Model 2.0.0 manual, available from the NEEDS web site
at \url{https://nanohub.org/publications/74/1}.

{\bf NOTE: } Unlike all other MOSFET models in Xyce, the MVS model
takes only 3 nodes, the drain, gate and source.  It takes no substrate
node.

\input{M_2000_Device_Model_Params}
\input{M_2001_Device_Model_Params}

\clearpage

\subsubsection{Level 2002 MOSFET Tables (MVSG\_CMC version 1.1.0)}
\Xyce{} includes the MIT Virtual Source GaN HEMT High-Voltage
(MVSG\_CMC) MOSFET model version 1.1.0.  The code in \Xyce{} was
generated from the MIT Verilog-A input.  Model parameters for the MVS
model are given in \ref{M_2002_Device_Instance_Params} and
\ref{M_2002_Device_Model_Params}, and its output variables in
\ref{M_2002_OutputVars}.  More information about this model may be
obtained from the CMC standard models page at
\url{https://si2.org/standard-models}.

\input{M_2002_Device_Instance_Params}
\input{M_2002_Device_Model_Params}
\input{M_2002_OutputVars}


\clearpage
\subsubsection{Level 260 MOSFET Tables (EKV version 2.6)}

\Xyce{} includes the EKV MOSFET model, version 2.6 as the level 260
MOSFET device.

Official documentation of this model may be found at \url{https://www.epfl.ch/labs/iclab/wp-content/uploads/2019/02/ekv_v262.pdf}.

We have implemented EKV 2.6 directly from the Verilog-A source
published by its authors at \url{https://github.com/ekv26/model}.
While it is a faithful implementation of the model provided there, we
have had anecdotal evidence that other simulators have different
implementations that contain additional parameters and possibly a
different extrinsic model.  Model cards containing parameters
extracted from other simulators may not result in \Xyce{} simulations
that match those other simulators.  Watch carefully for any warnings
from \Xyce{} regarding unrecognized model parameters, as these are a
strong indication that the model card is not extracted using the exact
version of EKV provided by \Xyce{}.

Tables of EKV MOSFET 2.6 parameters are in
tables~\ref{M_260_Device_Instance_Params} and
\ref{M_260_Device_Model_Params}.

Note that in the tables the device claims that the default
\texttt{TNOM} and \texttt{TEMP} parameter values is 1e21.  This is
merely an artifact of an unusual way the authors have defined those
parameters in the Verilog-A source.  In fact, if not given
\texttt{TNOM} defaults to 25 $^\circ$ C, and if not given \texttt{TEMP}
defaults to the ambient temperature of the simulation.

% This table was generated by Xyce:
%   Xyce -doc M 260
%
\index{ekv mosfet version 2.6!device instance parameters}
\begin{DeviceParamTableGenerated}{EKV MOSFET version 2.6 Device Instance Parameters}{M_260_Device_Instance_Params}
AD &  & -- & 0 \\ \hline
AS &  & -- & 0 \\ \hline
L &  & -- & 1e-05 \\ \hline
M &  & -- & 1 \\ \hline
NS &  & -- & 1 \\ \hline
PD &  & -- & 0 \\ \hline
PS &  & -- & 0 \\ \hline
W &  & -- & 1e-05 \\ \hline
\end{DeviceParamTableGenerated}

% This table was generated by Xyce:
%   Xyce -doc M 260
%
\index{ekv mosfet version 2.6!device model parameters}
\begin{DeviceParamTableGenerated}{EKV MOSFET version 2.6 Device Model Parameters}{M_260_Device_Model_Params}
AD &  & -- & 0 \\ \hline
AF &  & -- & 1 \\ \hline
AGAMMA &  & -- & 1e-06 \\ \hline
AKP &  & -- & 1e-06 \\ \hline
AS &  & -- & 0 \\ \hline
AVTO &  & -- & 1e-06 \\ \hline
BEX &  & -- & -1.5 \\ \hline
COX &  & -- & 0.002 \\ \hline
DL &  & -- & -1e-08 \\ \hline
DW &  & -- & -1e-08 \\ \hline
E0 &  & -- & 1e+08 \\ \hline
GAMMA &  & -- & 0.7 \\ \hline
HDIF &  & -- & 5e-07 \\ \hline
IBA &  & -- & 5e+08 \\ \hline
IBB &  & -- & 4e+08 \\ \hline
IBBT &  & -- & 0.0009 \\ \hline
IBN &  & -- & 1 \\ \hline
KF &  & -- & 0 \\ \hline
KP &  & -- & 0.00015 \\ \hline
L &  & -- & 1e-05 \\ \hline
LAMBDA &  & -- & 0.8 \\ \hline
LETA &  & -- & 0.3 \\ \hline
LK &  & -- & 4e-07 \\ \hline
LMAX &  & -- & 100 \\ \hline
LMIN &  & -- & 0 \\ \hline
M &  & -- & 1 \\ \hline
NOISE &  & -- & 1 \\ \hline
NS &  & -- & 1 \\ \hline
PD &  & -- & 0 \\ \hline
PHI &  & -- & 0.5 \\ \hline
PS &  & -- & 0 \\ \hline
Q0 &  & -- & 0.00023 \\ \hline
RSH &  & -- & 0 \\ \hline
TCV &  & -- & 0.001 \\ \hline
TEMP &  & -- & 1e+21 \\ \hline
THETA &  & -- & 0 \\ \hline
TNOM &  & -- & 1e+21 \\ \hline
TP\_CJ &  & -- & 0 \\ \hline
TP\_CJSW &  & -- & 0 \\ \hline
TP\_CJSWG &  & -- & 0 \\ \hline
TP\_NJTS &  & -- & 0 \\ \hline
TP\_NJTSSW &  & -- & 0 \\ \hline
TP\_NJTSSWG &  & -- & 0 \\ \hline
TP\_PB &  & -- & 0 \\ \hline
TP\_PBSW &  & -- & 0 \\ \hline
TP\_PBSWG &  & -- & 0 \\ \hline
TP\_XTI &  & -- & 3 \\ \hline
TRISE &  & -- & 0 \\ \hline
TYPE &  & -- & 1 \\ \hline
UCEX &  & -- & 0.8 \\ \hline
UCRIT &  & -- & 2e+06 \\ \hline
VTO &  & -- & 0.5 \\ \hline
W &  & -- & 1e-05 \\ \hline
WETA &  & -- & 0.2 \\ \hline
WMAX &  & -- & 100 \\ \hline
WMIN &  & -- & 0 \\ \hline
XD\_BV &  & -- & 10 \\ \hline
XD\_CJ &  & -- & 1e-09 \\ \hline
XD\_CJSW &  & -- & 1e-12 \\ \hline
XD\_CJSWG &  & -- & 1e-12 \\ \hline
XD\_GMIN &  & -- & 0 \\ \hline
XD\_JS &  & -- & 1e-09 \\ \hline
XD\_JSW &  & -- & 1e-12 \\ \hline
XD\_JSWG &  & -- & 1e-12 \\ \hline
XD\_MJ &  & -- & 0.9 \\ \hline
XD\_MJSW &  & -- & 0.7 \\ \hline
XD\_MJSWG &  & -- & 0.7 \\ \hline
XD\_N &  & -- & 1 \\ \hline
XD\_NJTS &  & -- & 1 \\ \hline
XD\_NJTSSW &  & -- & 1 \\ \hline
XD\_NJTSSWG &  & -- & 1 \\ \hline
XD\_PB &  & -- & 0.8 \\ \hline
XD\_PBSW &  & -- & 0.6 \\ \hline
XD\_PBSWG &  & -- & 0.6 \\ \hline
XD\_VTS &  & -- & 0 \\ \hline
XD\_VTSSW &  & -- & 0 \\ \hline
XD\_VTSSWG &  & -- & 0 \\ \hline
XD\_XJBV &  & -- & 0 \\ \hline
XJ &  & -- & 3e-07 \\ \hline
\end{DeviceParamTableGenerated}


\clearpage
\subsubsection{Level 301 MOSFET Tables (EKV version 3.0.1)}
\Xyce{} includes the EKV MOSFET model, version
3.0.1~\cite{BLETK:1997}\cite{EKV:2006}\cite{EKV:2007}.  Full
documentation for the EKV3 model is available on the \Xyce{} internal web site;
the documentation for the EKV3 model may be freely redistributed.  Instance and
model parameters for the EKV model are given in
tables~\ref{M_301_Device_Instance_Params} and \ref{M_301_Device_Model_Params}.

The EKV3 model is developed by the EKV Team of the Electronics Laboratory-TUC
(Technical University of Crete). It is included in \Xyce{} under license from
Technical University of Crete.  The official web site of the EKV model is
\url{http://ekv.epfl.ch/}.

\textbf{Due to licensing restrictions, the EKV3 MOSFET is not available in
     open-source versions of \Xyce{}.  The license for EKV3 authorizes Sandia
     National Laboratories to distribute EKV3 only in binary versions of code.}


\input{M_301_Device_Instance_Params}
\input{M_301_Device_Model_Params}

\clearpage
\subsubsection{Level 10240 MOSFET Tables (L\_UTSOI Version 102.4.0)}
Select \Xyce{} binaries include the L\_UTSOI MOSFET model as the level
10240 MOSFET.  This model's parameters and output variables are listed in tables~\ref{M_10240_Device_Instance_Params}, \ref{M_10240_Device_Model_Params}, and \ref{M_10240_OutputVars}

\input{M_10240_Device_Instance_Params}
\input{M_10240_Device_Model_Params}
%table generated from Verilog-A input
\index{MOSFET level 10240!device output variables}
\begin{DeviceParamTableGenerated}{MOSFET level 10240 Output Variables}{M_10240_OutputVars}
type & Flag for channel type &    & none \\ \hline
vds & Internal drain-source DC voltage (NMOS convention) &   V & none \\ \hline
vsb & Internal source-bulk DC voltage (NMOS convention) &   V & none \\ \hline
vgs & Internal gate-source DC voltage (NMOS convention) &   V & none \\ \hline
vth & Threshold voltage &   V & none \\ \hline
vth\_drive & Effective gate drive voltage, including back bias, drain bias effects and self-heating &   V & none \\ \hline
vdsat & Drain saturation voltage at the given bias &   V & none \\ \hline
vdsat\_marg & Vds voltage margin &   V & none \\ \hline
id & Total DC drain current flowing into drain terminal &   A & none \\ \hline
ig & Total DC gate current flowing into gate terminal &   A & none \\ \hline
is & Total DC source current flowing into source terminal &   A & none \\ \hline
ib & Total DC bulk current flowing into bulk terminal &   A & none \\ \hline
ids & DC channel current, excluding tunnel, GISL and GIDL currents &   A & none \\ \hline
igidl & DC Gate Induced Drain Leakage current &   A & none \\ \hline
igisl & DC Gate Induced Source Leakage current &   A & none \\ \hline
igs & DC gate-source leakage current &   A & none \\ \hline
igd & DC gate-drain leakage current &   A & none \\ \hline
idb & DC drain-bulk current &   A & none \\ \hline
isb & DC source-bulk current &   A & none \\ \hline
gm & Internal DC transconductance &   A/V & none \\ \hline
gmb & Internal DC bulk transconductance &   A/V & none \\ \hline
gds & Internal DC output conductance &   A/V & none \\ \hline
cgg & Internal AC gate capacitance, including overlap capacitances &   F & none \\ \hline
cgd & Internal AC gate-drain transcapacitance, including overlap capacitances &   F & none \\ \hline
cgs & Internal AC gate-source transcapacitance, including overlap capacitances &   F & none \\ \hline
cgb & Internal AC gate-bulk transcapacitance &   F & none \\ \hline
cdd & Internal AC drain capacitance &   F & none \\ \hline
cdg & Internal AC drain-gate transcapacitance &   F & none \\ \hline
cds & Internal AC drain-source transcapacitance &   F & none \\ \hline
cdb & Internal AC drain-bulk transcapacitance &   F & none \\ \hline
cbb & Internal AC bulk capacitance &   F & none \\ \hline
cbg & Internal AC bulk-gate transcapacitance &   F & none \\ \hline
cbs & Internal AC bulk-source transcapacitance &   F & none \\ \hline
cbd & Internal AC bulk-drain transcapacitance &   F & none \\ \hline
css & Internal AC source capacitance &   F & none \\ \hline
csg & Internal AC source-gate transcapacitance &   F & none \\ \hline
csb & Internal AC source-bulk transcapacitance &   F & none \\ \hline
csd & Internal AC source-drain transcapacitance &   F & none \\ \hline
tk & MOSFET device temperature &   K & none \\ \hline
dtsh & MOSFET device temperature increase due to self-heating &   K & none \\ \hline
self\_gain & Internal L-UTSOI model self gain &    & none \\ \hline
rout & AC output resistance &   Ohm & none \\ \hline
beff & Gain factor in saturation &   A/V$^{2}$ & none \\ \hline
ft & Unity gain frequency at the given bias &   Hz & none \\ \hline
rgate & MOS gate resistance (intrinsic input resistance) &   Ohm & none \\ \hline
gmoverid & Gm over Id &   1/V & none \\ \hline
vearly & Equivalent Early voltage &   V & none \\ \hline
\end{DeviceParamTableGenerated}


