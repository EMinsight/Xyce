% Sandia National Laboratories is a multimission laboratory managed and
% operated by National Technology & Engineering Solutions of Sandia, LLC, a
% wholly owned subsidiary of Honeywell International Inc., for the U.S.
% Department of Energy’s National Nuclear Security Administration under
% contract DE-NA0003525.

% Copyright 2002-2024 National Technology & Engineering Solutions of Sandia,
% LLC (NTESS).


\index{solvers!hb!options}
\begin{OptionTable}{Options for HB.}
\label{hbPKG}

NUMFREQ & Number of harmonics to be calculated for each tone. It must have the same number of entries as .HB
statement & 10\\ \hline

STARTUPPERIODS & Number of periods to integrate through before calculating the initial conditions.  This option is only used when TAHB=1.& 0\\ \hline

SAVEICDATA & Write out the initial conditions to a file. & 0\\ \hline

TAHB &  This flag sets transient assisted HB. When TAHB=0, transient analysis is not performed to get an initial guess. When TAHB=1, it uses transient analysis to get an initial guess. For multi-tone HB simulation, the initial guess is generated by a single tone transient simulation. The first tone following \verb|.HB| is used to determine the period for the transient simulation.
For multi-tone HB simulation, it should be set to the frequency that produces the most nonlinear response 
by the circuit. When tahb = 2, the DC op is used as an initial guess & 1  \\ \hline

VOLTLIM &  This flag sets voltage limiting for HB. During the initial guess calculation, which normally uses transient simulation, the voltage limiting flag is determined by .options device voltlim. During the HB phase, the voltage limiting flag is determined by .options hbint voltlim. & 1 \\ \hline

INTMODMAX & The maximum intermodulation product order used in the spectrum. & 
the largest value in the NUMFREQ list. \\ \hline

NUMTPTS & Number of time points in the output & The total number of frequencies (positive, negative and DC). \\ \hline 

SELECTHARMS & The truncation method used in multi-tone HB to select harmonics. Box, diamond and hybrid truncation methods are     supported &  hybrid  \\ \hline
\end{OptionTable}

%%% Local Variables:
%%% mode: latex
%%% End:
