% Sandia National Laboratories is a multimission laboratory managed and
% operated by National Technology & Engineering Solutions of Sandia, LLC, a
% wholly owned subsidiary of Honeywell International Inc., for the U.S.
% Department of Energy’s National Nuclear Security Administration under
% contract DE-NA0003525.

% Copyright 2002-2024 National Technology & Engineering Solutions of Sandia,
% LLC (NTESS).


%%
%% Nonlinear Solver Options Table
%%
\index{solvers!nonlinear!options}
\begin{OptionTable4}{Options for Nonlinear Solver Package.}
\label{NonlinPKG}
NOX & Use NOX nonlinear solver. & 1 (TRUE) & 0 (FALSE) \\ \hline

NLSTRATEGY & Nonlinear solution strategy.  Supported Strategies:
\begin{XyceItemize}
\item 0 (Newton)
\item 1 (Gradient)
\item 2 (Trust Region)
\end{XyceItemize} &
0 (Newton) &
0 (Newton) \\ \hline

SEARCHMETHOD &
Line-search method used by the nonlinear solver.  Supported
line-search methods:
\begin{XyceItemize}
\item 0 (Full Newton - no line search)
\item 1 (Interval Halving)
\item 2 (Quadratic Interpolation)
\item 3 (Cubic Interpolation)
\item 4 (More'-Thuente)
\end{XyceItemize} &
0 (Full Newton) & 
0 (Full Newton) \\ \hline

CONTINUATION & Enables the use of Homotopy/Continuation algorithms for the nonlinear solve.  Options are:
\begin{XyceItemize}
\item 0 (Standard nonlinear solve)
\item 1 (Natural parameter homotopy.  See LOCA options list)
\item 2/mos (Specialized dual parameter homotopy for MOSFET circuits)
\item 3/gmin (GMIN stepping, similar to that of SPICE)
\item 34/sourcestep (Simultaneous source stepping)
\item 35/sourcestep2 (Sequential source stepping)
\end{XyceItemize} & 
0 (Standard nonlinear solve) &
0 (Standard nonlinear solve) \\ \hline

ABSTOL\index{\texttt{ABSTOL}} & Absolute residual vector tolerance &
1.0E-12 & 1.0E-06 \\ \hline

RELTOL\index{\texttt{RELTOL}} & Relative residual vector tolerance &
1.0E-03 & 1.0E-02 \\ \hline

DELTAXTOL & Weighted nonlinear-solution update norm convergence
tolerance & 1.0 & 0.33 \\ \hline

RHSTOL & Residual convergence tolerance (unweighted 2-norm) &
1.0E-06 & 1.0E-02 \\ \hline

SMALLUPDATETOL & Minimum acceptable norm for weighted nonlinear-solution update & 
1.0E-06 & 1.0E-06 \\ \hline

MAXSTEP\index{\texttt{MAXSTEP}} & Maximum number of Newton steps & 200 & 20 \\ \hline

MAXSEARCHSTEP & Maximum number of line-search steps & 2 & 2 \\ \hline

IN\_FORCING & Inexact Newton-Krylov forcing flag & 
0 (FALSE) &
0 (FALSE) \\ \hline

AZ\_TOL &  Sets the minimum allowed linear solver tolerance. Valid only if \texttt{IN\_FORCING}=1.  & 
1.0E-12 &
1.0E-12 \\ \hline

RECOVERYSTEPTYPE &  If using a line search, this option determines the type of step to take if the line search fails. Supported strategies:
\begin{XyceItemize}
\item 0 (Take the last computed step size in the line search algorithm)
\item 1 (Take a constant step size set by \texttt{RECOVERYSTEP})
\end{XyceItemize} & 
0 &
0 \\ \hline

RECOVERYSTEP & Value of the recovery step if a constant step length is selected & 
1.0 &
1.0 \\ \hline

\debug{DEBUGLEVEL} & \debug{The higher this number, the more info is output} & 
\debug{1} &
\debug{1} 
\\ \hline

\debug{DEBUGMINTIMESTEP} & \debug{First time-step debug information is output} & 
\debug{0} &
\debug{0}
\\ \hline

\debug{DEBUGMAXTIMESTEP} & \debug{Last time-step of debug output} &
\debug{99999999} &
\debug{99999999} \\
\hline

\debug{DEBUGMINTIME} & \debug{Same as \texttt{DEBUGMINTIMESTEP} except controlled by
time (sec.) instead of step number} & 
\debug{0.0}  &
\debug{0.0} 
\\ \hline

\debug{DEBUGMAXTIME} & \debug{Same as \texttt{DEBUGMAXTIMESTEP} except controlled by
time (sec.) instead of step number} & 
\debug{1.0E+99} &
\debug{1.0E+99} 
\\ \hline

\end{OptionTable4}
