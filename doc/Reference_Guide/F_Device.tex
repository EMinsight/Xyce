% Sandia National Laboratories is a multimission laboratory managed and
% operated by National Technology & Engineering Solutions of Sandia, LLC, a
% wholly owned subsidiary of Honeywell International Inc., for the U.S.
% Department of Energy’s National Nuclear Security Administration under
% contract DE-NA0003525.

% Copyright 2002-2024 National Technology & Engineering Solutions of Sandia,
% LLC (NTESS).


\begin{Device}

\symbol
{\includegraphics{cccsSymbol}}

\device
\begin{alltt}
F<name> <(+) node> <(-) node>
+ <controlling V device name> <gain>
F<name> <(+) node> <(-) node> POLY(<value>)
+ <controlling V device name>*
+ < <polynomial coefficient value> >*
\end{alltt}

\examples
\begin{alltt}
FSENSE 1 2 VSENSE 10.0
FAMP 13 0 POLY(1) VIN 0 500
FNONLIN 100 101 POLY(2) VCNTRL1 VCINTRL2 0.0 13.6 0.2 0.005
\end{alltt}

\parameters

\begin{Parameters}

\param{\vbox{\hbox{(+) node\hfil}\hbox{(-) node}}}
Output nodes. Positive current flows from the \texttt{(+)} node through
the source to the \texttt{(-)} node.

\param{controlling V device}
The controlling voltage source which must be an independent voltage source
(V device).

\end{Parameters}

\comments

In the first form, a specified current through a controlling device is
multiplied by the gain to determine this device's output current.  The
gain may be expressed either as a number, a parameter, or an arbitrary
brace-delimited ABM expression.

The second form using the \texttt{POLY} keyword is used in analog behavioral
modeling.

Both forms are automatically converted within \Xyce{} to its principal
ABM device, the \texttt{B} nonlinear dependent source device. See the B-source
section (\ref{B_Source_Device}) and the \Xyce{} User's Guide for more guidance
on analog behavioral modeling.  For details concerning the use of the
\texttt{POLY} format, see section~\ref{PspicePoly}.

The power supplied or dissipated by this source device is calculated 
with $I \cdot \Delta V$ where the voltage drop is calculated as $(V_+ - V_-)$ 
and positive current flows from $V_+$ to $V_-$.  Dissipated power has a
positive sign, while supplied power has a negative sign.

F-sources were originally developed primarily to support DC and transient analysis.  
As such, their support for frequency domain analysis (AC and HB) has some limitations.  
The main limitation to be aware of is that time-dependent sources will not work with AC or HB analysis.  
These are sources in which the variable \texttt{TIME} is used in the \texttt{VALUE=} expression. 
However, this time-dependent usage is not common.  The most 
common use case is one in which the F-source is purely dependent (depends only 
on other solution variables), and this use case will work with AC and HB.  

\end{Device}
