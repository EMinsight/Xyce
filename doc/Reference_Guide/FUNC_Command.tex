% Sandia National Laboratories is a multimission laboratory managed and
% operated by National Technology & Engineering Solutions of Sandia, LLC, a
% wholly owned subsidiary of Honeywell International Inc., for the U.S.
% Department of Energy’s National Nuclear Security Administration under
% contract DE-NA0003525.

% Copyright 2002-2024 National Technology & Engineering Solutions of Sandia,
% LLC (NTESS).


User defined functions that can be used in expressions appearing later 
in the same scope as the \texttt{.FUNC} statement (or \texttt{.PARAM} statement)

\begin{Command}
\format
.FUNC <name>([arg]*) [=] \{<body>\} \\
.PARAM <name>([arg]*) = \{<body>\}

\examples
\begin{alltt}
.FUNC E(x) \{exp(x)\}
.FUNC DECAY(CNST) \{E(-CNST*TIME)\}
.FUNC TRIWAV(x) \{ACOS(COS(x))/3.14159\}
.FUNC MIN3(A,B,C) \{MIN(A,MIN(B,C))\}
.FUNC SUM(A,B,C)=\{A+B+C\}
.PARAM SUM(A,B,C)=\{A+B+C\}
\end{alltt}

\arguments

\begin{Arguments}
\argument{name}

Function name.  Functions cannot be redefined and the function name must 
not be the same as any of the
predefined functions (e.g., \texttt{SIN} and \texttt{SQRT}).  

\argument{arg}

The arguments to the function.  \texttt{.FUNC} arguments cannot be node names.
The number of arguments in the use of a function must agree with the number 
in the definition. Parameters, TIME, FREQ, and other functions are allowed in 
the body of function definitions.  \index{constants (\texttt{EXP},\texttt{PI})}
Two constants \texttt{EXP} and \texttt{PI} cannot
be used a argument names.  These constants are equal to $e$ and $\pi$, respectively,
and cannot be redefined.

\argument{body}

May refer to other (previously defined) functions; the second example, 
DECAY, uses the first example, E. 

\end{Arguments}

\comments

The \texttt{<body>} of a defined function is handled in the same way as any 
math expression; it must be enclosed in curly braces (\{\}) or single quotes (').

The original keyword for user-defined functions is \texttt{.FUNC}.  However, 
for simulator compatibility functions can also be specified using \texttt{.PARAM}.
If using \texttt{.FUNC}, each function must get their own netlist line and the equals sign is optional.  
If using \texttt{.PARAM}, multiple functions can be specified on the same line, and the equals sign is required.

\index{\texttt{.FUNC}!subcircuit scoping}The scoping rules for functions are:
\begin{XyceItemize}
\item If a \texttt{.FUNC}, statement is included in the main circuit 
netlist, then it is accessible from the main circuit and all subcircuits. 
\item \texttt{.FUNC} statements defined within a subcircuit are scoped 
to that subciruit definition.  So, their functions are only accessible within 
that subcircuit definition, as well as within ``nested subcircuits'' also 
defined within that subcircuit definition.
\end{XyceItemize}

Additional illustative examples of scoping are given in the
``Working with Subcircuits and Models'' section of the \Xyce{} Users'
Guide\UsersGuide.

\index{\texttt{.FUNC}!allowed names}Rules for function names are as follows:
\begin{XyceItemize}
\item They should start with a letter or the underscore (\verb|_|) character,
for maximal compatibilty with other Spice-like simulators.  The hash (\verb|#|)
at (\verb|@|) and backtick (\verb|`|) symbols also work, but they are not
reserved characters.
\item These arithmetic operators \verb|%| \verb|^| \verb|&| \verb|~|
\verb|*| \verb|-| \verb|+| \verb|<| \verb|>| \verb|/| \verb+|+ should not
be used anywhere in function names, as they cause problems with expression
parsing.
\item Parentheses (``('' or ``)''), braces (``\{'' or ``\}''), commas,
semi-colons, double quotes and single quotes are also not allowed.
\end{XyceItemize}

\end{Command}

