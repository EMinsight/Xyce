% Sandia National Laboratories is a multimission laboratory managed and
% operated by National Technology & Engineering Solutions of Sandia, LLC, a
% wholly owned subsidiary of Honeywell International Inc., for the U.S.
% Department of Energy’s National Nuclear Security Administration under
% contract DE-NA0003525.

% Copyright 2002-2023 National Technology & Engineering Solutions of Sandia,
% LLC (NTESS).

AC Analysis generates two output files, the primary output is in the
frequency domain and the initial conditions output is in the time domain.

Note that when using the \texttt{.PRINT AC} to create the variable list
for DC type output, usage of frequency domain functions like \texttt{VDB} can result
in -Inf output being written to the output file.  This is easily solved
by defining a \texttt{.PRINT AC\_IC} command to specify the correct
output for initial condition data.

Homotopy output can also be generated.

{
\begin{PrintCommandTable}{Print AC Analysis Type}
.PRINT AC & \emph{circuit-file}.FD.prn & INDEX FREQ \\ \hline
.PRINT AC FORMAT=GNUPLOT & \emph{circuit-file}.FD.prn & INDEX FREQ \\ \hline
.PRINT AC FORMAT=SPLOT & \emph{circuit-file}.FD.prn & INDEX FREQ \\ \hline
.PRINT AC FORMAT=NOINDEX & \emph{circuit-file}.FD.prn & FREQ \\ \hline
.PRINT AC FORMAT=CSV & \emph{circuit-file}.FD.csv & FREQ \\ \hline
.PRINT AC FORMAT=RAW & \emph{circuit-file}.raw & FREQ \\ \hline
Xyce -a \newline .PRINT AC FORMAT=RAW & \emph{circuit-file}.raw & FREQ \\ \hline
.PRINT AC FORMAT=TECPLOT & \emph{circuit-file}.FD.dat & FREQ \\ \hline
.PRINT AC FORMAT=PROBE & \emph{circuit-file}.csd & -- \\ \hline


\multicolumn{3}{c}{\smallskip\color{XyceDarkBlue}\em\bfseries Add \texttt{.OP} To Netlist To Enable AC\_IC Output} \\ \hline
.PRINT AC\_IC & \emph{circuit-file}.TD.prn & INDEX TIME \\ \hline
.PRINT AC\_IC FORMAT=GNUPLOT & \emph{circuit-file}.TD.prn & INDEX TIME \\ \hline
.PRINT AC\_IC FORMAT=SPLOT & \emph{circuit-file}.TD.prn & INDEX TIME \\ \hline
.PRINT AC\_IC FORMAT=NOINDEX & \emph{circuit-file}.TD.prn & TIME \\ \hline
.PRINT AC\_IC FORMAT=CSV & \emph{circuit-file}.TD.csv & TIME \\ \hline
.PRINT AC\_IC FORMAT=RAW & \emph{circuit-file}.raw & TIME \\ \hline
Xyce -a \newline .PRINT AC\_IC FORMAT=RAW & \emph{circuit-file}.raw & TIME \\ \hline
.PRINT AC\_IC FORMAT=TECPLOT & \emph{circuit-file}.TD.dat & TIME \\ \hline
.PRINT AC\_IC FORMAT=PROBE & \emph{circuit-file}.TD.csd & -- \\ \hline

\multicolumn{3}{c}{\smallskip\color{XyceDarkBlue}\em\bfseries Command Line Raw Override Output} \\ \hline
Xyce -r raw-file-name & \emph{raw-file-name} & All circuit variables printed \\ \hline
Xyce -r raw-file-name -a & \emph{raw-file-name} & All circuit variables printed \\ \hline

\multicolumn{3}{c}{\smallskip\color{XyceDarkBlue}\em\bfseries Additional Output Available} \\ \hline
.OP & \emph{log file} & Operating point data \\ \hline
.SENS \newline .PRINT SENS & \multicolumn{2}{c}{see~\nameref{Print_Sensitivity}} \\ \hline
.OPTIONS NONLIN CONTINUATION=<method> \newline .PRINT HOMOTOPY & \multicolumn{2}{c}{see~\nameref{Print_Homotopy}} \\ \hline
\end{PrintCommandTable}
}

