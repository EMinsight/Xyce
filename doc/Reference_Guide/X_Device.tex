% Sandia National Laboratories is a multimission laboratory managed and
% operated by National Technology & Engineering Solutions of Sandia, LLC, a
% wholly owned subsidiary of Honeywell International Inc., for the U.S.
% Department of Energy’s National Nuclear Security Administration under
% contract DE-NA0003525.

% Copyright 2002-2023 National Technology & Engineering Solutions of Sandia,
% LLC (NTESS).


\label{XDevicesection}
A subcircuit can be introduced into the circuit netlist
using the specified nodes to substitute for the argument nodes in the
definition.  It provides a building block of circuitry to be defined a single
time and subsequently used multiple times in the overall circuit netlists.
See Section~\ref{SUBCKTsection} for more information about subcircuits.

\begin{Device}

\device
X<name> [node]* <subcircuit name> [PARAMS: [<name> = <value>]*]

\examples
\begin{alltt}
  X12 100 101 200 201 DIFFAMP
  XBUFF 13 15 UNITAMP
  XFOLLOW IN OUT VCC VEE OUT OPAMP
  XFELT 1 2 FILTER PARAMS: CENTER=200kHz
  XNANDI 25 28 7 MYPWR MYGND PARAMS: IO\_LEVEL=2
\end{alltt}

\parameters

\begin{Parameters}

\param{subcircuit name}
The name of the subcircuit's definition.

\param{PARAMS:}

Passed into subcircuits as arguments and into expressions inside the
subcircuit.

\end{Parameters}

\comments

There must be an equal number of nodes in the subcircuit call and in its
definition.

Subcircuit references may be nested to any level. However, the nesting
cannot be circular. For example, if subcircuit \texttt{A}'s definition
includes a call to subcircuit \texttt{B}, then subcircuit \texttt{B}'s
definition cannot include a call to subcircuit \texttt{A}.

\end{Device}
