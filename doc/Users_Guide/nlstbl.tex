% Sandia National Laboratories is a multimission laboratory managed and
% operated by National Technology & Engineering Solutions of Sandia, LLC, a
% wholly owned subsidiary of Honeywell International Inc., for the U.S.
% Department of Energy’s National Nuclear Security Administration under
% contract DE-NA0003525.

% Copyright 2002-2023 National Technology & Engineering Solutions of Sandia,
% LLC (NTESS).

%%
%% Nonlinear Solver Options Table
%%
\small

\caption[Options for Nonlinear Solver Package.] {Options for Nonlinear Solver Package.\label{NonlinPKG}}
\index{solvers!nonlinear!options}
\begin{longtable}[htbp]{|>{\setlength{\hsize}{.8\hsize}}Y|
>{\setlength{\hsize}{1.5\hsize}}Y|
>{\setlength{\hsize}{.7\hsize}}Y|} \hline

\rowcolor{XyceDarkBlue}\color{white}\textbf{Nonlinear Solver (PKG =
NONLIN) Tag} & \color{white}\bf Description
& \color{white}\bf Default \endhead

\texttt{NLSTRATEGY} & Nonlinear solution strategy.  Supported Strategies:
\begin{XyceItemize}
\item 0 (Newton)
\item 1 (Gradient)
\item 2 (Trust Region)
%\item 3 (Modified Newton - testing)
%\item 4 (BFGS - testing)
%\item 5 (Broyden - testing)
%\item 6 (Tensor - testing)
%\item 7 (Fast Newton - testing)
%\item 8 (Steepest Descent/Newton Combo - testing)
\end{XyceItemize} &
0 (Newton) \\ \hline

\texttt{SEARCHMETHOD} &
Line-search method used by the nonlinear solver.  Supported
line-search methods:
\begin{XyceItemize}
\item 0 (Full Newton - no line search)
\item 1 (Interval Halving)
\item 2 (Quadratic Interpolation)
\item 3 (Cubic Interpolation)
\item 4 (More'-Thuente)
\end{XyceItemize} &
0 (Full Newton) \\ \hline

\texttt{ABSTOL}\index{\texttt{ABSTOL}} & Absolute residual vector tolerance &
1.0E-12 \\ \hline

\texttt{RELTOL}\index{\texttt{RELTOL}} & Relative residual vector tolerance &
1.0E-03 \\ \hline

\texttt{DELTAXTOL} & Weighted nonlinear-solution update norm convergence
tolerance & 1.0 \\ \hline

\texttt{RHSTOL} & Residual convergence tolerance (unweighted 2-norm) &
1.0E-06 \\ \hline

\texttt{MAXSTEP}\index{\texttt{MAXSTEP}} & Maximum number of Newton steps & 200
\\ \hline

\texttt{MAXSEARCHSTEP} & Maximum number of line-search steps & 9 \\ \hline

\texttt{IN\_FORCING} & Inexact Newton-Krylov forcing flag & 0 (FALSE) \\ \hline

\texttt{AZ\_TOL} &  Sets the minimum allowed linear solver tolerance. Valid only if \texttt{IN\_FORCING}=1.  & 1.0E-12 \\ \hline

\texttt{DEBUGLEVEL} & The higher this number, the more info is output & 1
\\ \hline

\texttt{DEBUGMINTIMESTEP} & First time-step debug information is output & 0
\\ \hline

\texttt{DEBUGMAXTIMESTEP} & Last time-step of debug output & 99999999 \\
\hline

\texttt{DEBUGMINTIME} & Same as \texttt{DEBUGMINTIMESTEP} except controlled by
time (sec.) instead of step number & 0.0 \\ \hline

\texttt{DEBUGMAXTIME} & Same as \texttt{DEBUGMAXTIMESTEP} except controlled by
time (sec.) instead of step number & 1.0E+99 \\ \hline

\texttt{RECOVERYSTEPTYPE} &  If using a line search, this option determines the type of step to take if the line search fails. Supported strategies:
\begin{XyceItemize}
\item 0 (Take the last computed step size in the line search algorithm)
\item 1 (Take a constant step size set by \texttt{RECOVERYSTEP})
\end{XyceItemize} & 0 \\ \hline

\texttt{RECOVERYSTEP} & Value of the recovery step if a constant step length is selected & 1.0 \\ \hline

\texttt{CONTINUATION} & Enables the use of Homotopy/Continuation algorithms for the nonlinear solve.  Options are:
\begin{XyceItemize}
\item 0 (Standard nonlinear solve)
\item 1 (Natural parameter homotopy.  See LOCA options list)
\item 2 (Specialized dual parameter homotopy for MOSFET circuits)
%\item 3 (Artificial parameter homotopy - unsupported)
\end{XyceItemize} & 0 (Standard nonlinear solve) \\ \hline


\end{longtable}
