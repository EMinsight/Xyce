% Sandia National Laboratories is a multimission laboratory managed and
% operated by National Technology & Engineering Solutions of Sandia, LLC, a
% wholly owned subsidiary of Honeywell International Inc., for the U.S.
% Department of Energy’s National Nuclear Security Administration under
% contract DE-NA0003525.

% Copyright 2002-2023 National Technology & Engineering Solutions of Sandia,
% LLC (NTESS).

%%-------------------------------------------------------------------------
%% Purpose        : Main LaTeX Xyce Users' Guide
%% Special Notes  : Graphic files (pdf format) work with pdflatex.  To use
%%                  LaTeX, we need to use postcript versions.  Not sure why.
%% Creator        : Scott A. Hutchinson, Computational Sciences, SNL
%% Creation Date  : {05/23/2002}
%%
%%-------------------------------------------------------------------------

\chapter{Working with Subcircuits and Models}
\label{Models}

\chapteroverview{Chapter Overview}
{
This chapter provides model examples and summarizes ways to create and
modify models.  Sections include:
\begin{XyceItemize}
\item Section~\ref{Model_Def}, {\em Model Definition}
\item Section~\ref{Subcircuit_Sect}, {\em Subcircuit Creation}
\item Section~\ref{Model_Organization}, {\em Model Organization}
\item Section~\ref{Model_Interpolation}, {\em Model Interpolation}
\item Section~\ref{subcircuit_multiplier}, {\em Subcircuit Multiplier \texttt{M}}
\end{XyceItemize}
}

\section{Model Definitions}
\label{Model_Def}
\index{model!definition}

A model describes the electrical performance of a {\em part}, such as
a specific vendor's version of a 2N2222 transistor.  To simulate a
part requires specification of {\em simulation properties}.  
These properties define the model of the part.

Depending on the given device type and the requirements of the circuit
design, a model is specified using a model parameter set, a subcircuit
netlist, or both.

In general, {\em model parameter sets} define the parameters used in ideal
models of specific device types, while {\em subcircuit netlists} allow the user
to combine ideal device models to simulate more complex effects.  For example,
one could simulate a bipolar transistor using the \Xyce{} BJT device by specifying
model parameters extracted to fit the simulation behavior to the behavior of
the part used. One could also develop a subcircuit macro-model of a capacitor
that adds effects such as lead inductance and resistance to the basic capacitor
device.

Both methods of defining a model use a netlist format, with precise
syntax rules.  In this section we give an overview of how to define
model parameter sets in \Xyce{}.  A subsequent subsection will provide a
similar overview of how to define subcircuit models.  For full
details, consult the \Xyce{} Reference Guide\ReferenceGuide.

\paragraph{Defining models using model parameters}

Although \Xyce{} has no built-in part models,  models can be defined for a
device by changing some or all of the {\em model parameters} from their
defaults via the \texttt{.MODEL} statement. For example:

\begin{vquote}
 \texttt{M5 3 2 1 0 MLOAD1}
 \texttt{.MODEL MLOAD1 NMOS (LEVEL=3 VTO=0.5 CJ=0.025pF)}
\end{vquote}

This example defines a MOSFET device \texttt{M5} that is an instance of a part
described by the model parameter set \texttt{MLOAD1}.  The \texttt{MLOAD1}
parameter set is defined in the \texttt{.MODEL} statement.

Most device types in \Xyce{} support some form of model parameters.  Consult
the \Xyce{} Reference Guide\ReferenceGuide{} for the model parameters supported
by each device type.

\paragraph{Defining models using subcircuit netlists}

In \Xyce{}, models may also be defined using the
\texttt{.SUBCKT}/\texttt{.ENDS} subcircuit syntax. This syntax allows the
creation of {\em Netlists}, which define the configuration and function of the
part, and the use of {\em Variable input parameters}, which can be used to
create device-specific implementations of the model.  The \texttt{.SUBCKT}
syntax, and an example of how to use \texttt{.SUBCKT} to implement a model, is
given in Section~\ref{Subcircuit_Sect}.

\clearpage
\section{Subcircuit Creation}
\label{Subcircuit_Sect}
\index{\texttt{.SUBCKT}}
\index{subcircuits}

A subcircuit can be created within \Xyce{} using the \texttt{.SUBCKT} keyword.
The \texttt{.ENDS} keyword is used to mark the end of the subcircut. All the
lines between the two keywords are considered to be part of the subcurcuit.  
Figure~\ref{Subcircuit_Example} provides an example of how a subcircuit is
defined and used.

\begin{figure}[H]
\begin{centering}
\shadowbox{
\begin{minipage}{0.8\textwidth}
\begin{vquote}
****other devices
X5 5  6  7  8 l3dsc1 PARAMS: ScaleFac=2.0
X6 9 10 11 12 l3dsc1
****more netlist commands

*** SUBCIRCUIT: l3dsc1
*** Parasitic Model: microstrip
*** Only one segment
.SUBCKT l3dsc1 1 3 2 4 PARAMS: ScaleFac=1.0
C01 1 0 4.540e-12
RG01 1 0 7.816e+03
L1 1 5 3.718e-08
R1 5 2 4.300e-01
C1 2 0 4.540e-12
RG1 2 0 7.816e+03
C02 3 0 4.540e-12
RG02 3 0 7.816e+03
L2 3 6 3.668e-08
R2 6 4 4.184e-01
C2 4 0 4.540e-12
RG2 4 0 7.816e+03
CM012 1 3 5.288e-13
KM12 L1 L2 2.229e-01
CM12 2 4 \{5.288e-13*ScaleFac\}
.ENDS
\end{vquote}
\end{minipage}
}
\caption{Example subcircuit model.\label{Subcircuit_Example}}
\index{Example!subcircuit definition}
\end{centering}
\end{figure}

In this example, a subcircuit model named \texttt{l3dsc1}, which implements one
part of a microstrip transmission line, is defined between the
\texttt{.SUBCKT}/\texttt{.ENDS} lines; and two different instances of the
subcircuit are used in the \texttt{X} lines.  This somewhat artificial example
shows how input parameters are used, where the last capacitor in the subcircuit
is scaled by the input parameter \texttt{ScaleFac}.  If input parameters are
not specified on the \texttt{X} line (as in the case of device \texttt{X6}),
then the default values specified on the \texttt{.SUBCKT} line are used.
Non-default values are specified on the \texttt{X} line using the
\texttt{PARAMS:} keyword.  Consult the \Xyce{} Reference Guide\ReferenceGuide{}
for precise syntax.

In addition to devices, a subcircuit may contain definitions, such as models
via the \texttt{.MODEL} statement, parameters via the \texttt{.PARAM}
statement, and functions via the \texttt{.FUNC} statement.  \Xyce{} also
supports the definition of one or more subcircuits within another subcircuit.
\index{subcircuits!hierarchy} Subcircuits can be nested to an arbitrary extent,
where one subcircuit can contain another subcircuit, which can contain yet
another subcircuit, and so on.

The creation of nested subcircuits requires an understanding of ``scope,''
\index{subcircuits!scope} such that each subcircuit defines the scope for the
definitions it contains.  That is, {\em the definitions contained within a
subcircuit can be used within that subcircuit and within any subcircuit it
contains, but not at any higher level.}  Definitions occurring in the main
circuit have global scope and can be used anywhere in the circuit.  A name,
such as a model, parameter, function, or subcircuit name, occurring in a
definition at one level of a circuit hierarchy can be redefined at any lower
level contained directly by that subcircuit.  In this case, the new definition
applies at the given level and those below.

\subsection{Examples of Scoping for Parameters, Models and Functions}
\index{parameter!scope}\index{model!scope}\index{subcircuit!scope}
The idea of ``scope'' may be best illustrated by examples.  This section gives
example for parameters and models.  However, this discussion also applies to
functions (defined with \texttt{.FUNC} statements). 

In the netlist provided
in Figure~\ref{Subcircuit_Example_2}, the model named \texttt{MOD1} can be used
in subcircuits \texttt{SUB1} and \texttt{SUB2}, but not in the subcircuit
\texttt{SUB3}. The parameter \texttt{P1} has a value of $10$ in subcircuit
\texttt{SUB1} and a value of $20$ in subcircuit \texttt{SUB2}. In subcircuit
\texttt{SUB3}, \texttt{P1} has no meaning.  In addition, \texttt{MOD1}
and \texttt{P1} would have no meaning at the main circuit level.

\begin{figure}[H]
\begin{centering}
\shadowbox{
\begin{minipage}{0.8\textwidth}
\begin{vquote}
.SUBCKT SUB1 1 2 3 4
.MODEL MOD1 NMOS(LEVEL=2)
.PARAM P1=10
*
* subcircuit devices omitted for brevity
*
.SUBCKT SUB2 1 3 2 4
.PARAM P1=20
*
* subcircuit devices omitted for brevity
*
.ENDS
.ENDS

.SUBCKT SUB3 1 2 3 4
*
* subcircuit devices omitted for brevity
*
.ENDS
\end{vquote}
\end{minipage}
}
\caption{Example subcircuit hierarchy, with scoping.\label{Subcircuit_Example_2}}
\index{Example!subcircuit model hierarchy with model and parameter scoping}
\end{centering}
\end{figure}

In the netlist provided in Figure~\ref{Subcircuit_Example_3}, the parameters
\texttt{P2} and \texttt{P3} are defined in the main circuit.  So, \texttt{P2}
is accessible in subcircuit \texttt{SUB4}, and has a value of 5 there. The
parameter \texttt{P3} was redefined in the context of subcircuit \texttt{SUB4}.
So, \texttt{P3} has a value of 10 in the main circuit, and a value of 15
in the context of subcircuit \texttt{SUB4} and any subcircuit subsequently 
defined within subcircuit \texttt{SUB4}.  
\begin{figure}[H]
\begin{centering}
\shadowbox{
\begin{minipage}{0.8\textwidth}
\begin{vquote}
* parameters defined in main circuit
.PARAM P2=5
.PARAM P3=10

.SUBCKT SUB4 1 2 3 4
.PARAM P3=15
*
* subcircuit devices omitted for brevity
*
.ENDS
\end{vquote}
\end{minipage}
}
\caption{Example subcircuit, with parameter definition override.\label{Subcircuit_Example_3}}
\index{Example!subcircuit, with parameter definition override}
\end{centering}
\end{figure}

In the netlist provided in Figure~\ref{Subcircuit_Example_4}, the 
definition of subcircuit \texttt{SUB5} defines the argument \texttt{A1}.
The subcircuit instance \texttt{X1} would use the default value of 5
for \texttt{A1}.  So the resistance value of device \texttt{X1:R1}
would be 5.  The subcircuit instance \texttt{X2} uses the specified
\texttt{A1} value of 10. So the resistance value of device \texttt{X2:R1}
would be 10.  Another key point about ``scope'' is that node, device, and
model names are scoped to the subcircuit in which they are defined. 
So, It is allowable to use names in a subcircuit that has been previously
used in either the main circuit netlist or in other subcircuit definitions.   
When the subcircuits are flattened (expanded to become part of the main 
netlist), all of their names are given prefixes via their subcircuit 
instance names. For example, device \texttt{R1} in subcircuit \texttt{X1} 
becomes the unique device name X1:R1 after expansion. 
\begin{figure}[H]
\begin{centering}
\shadowbox{
\begin{minipage}{0.8\textwidth}
\begin{vquote}
* subcircuit instance lines
X1 a b  SUB5 
X2 e f SUB5 PARAMS: A1=10
R1 1 2 2

* Note use of \{\} around A1 and FSIN parameters in
* the body of the subcircuit definition.
.SUBCKT SUB5 1 2 PARAMS: A1=5 FSIN=1
VSIN 1 g SIN(0 1V \{FSIN\} 0)
R1 g 2 \{A1\}
.ENDS
\end{vquote}
\end{minipage}
}
\caption{Example subcircuit, with PARAMS arguments.\label{Subcircuit_Example_4}}
\index{Example!subcircuit, with PARAMS arguments}
\end{centering}
\end{figure}

\clearpage
\section{Model Organization}
\label{Model_Organization}
\index{model!model organization}

While it is always possible to make a self-contained netlist in which
all models for all parts are included along with the circuit
definition, \Xyce{} provides a simple mechanism to conveniently organize
frequently used models into separate model libraries.  Models are simply
collected into model library files, and then accessed by netlists as needed by
inserting an \texttt{.INCLUDE}\index{\texttt{.INCLUDE}} directive.  This
section describes that process in detail.

\subsection{Model Libraries}

Device model and subcircuit definitions may be organized into model
libraries as text files (similar to netlist files) with one or more model
definitions. Many users choose to name model library files ending with
\texttt{.lib}, but they may be named using any convention.

In general, most users create model libraries files that include similar model
types.  In these files, the {\em header comments} describe the models therein.  

\subsection{Model Library Configuration using \texttt{.INCLUDE}}

\Xyce{} uses model libraries by inserting an \texttt{.INCLUDE}
statement into a netlist.  Once a file is included, its contents become
available to the netlist just as if the entire contents had been
inserted directly into the netlist.

As an example, one might create the following model library file
called \texttt{bjtmodels.lib}, containing \texttt{.MODEL} statements for
common types of bipolar junction transistors:

\begin{vquote}
*bjtmodels.lib
* Bipolar transistor models
.MODEL Q2N2222 NPN (Is=14.34f Xti=3 Eg=1.11 Vaf=74.03 Bf=5 Ne=1.307
+  Ise=14.34f Ikf=.2847 Xtb=1.5 Br=6.092 Nc=2 Isc=0 Ikr=0 Rc=1
+  Cjc=7.306p Mjc=.3416 Vjc=.75 Fc=.5 Cje=22.01p Mje=.377 Vje=.75
+  Tr=46.91n Tf=411.1p Itf=.6 Vtf=1.7 Xtf=3 Rb=10)

.MODEL 2N3700 NPN (IS=17.2E-15 BF=100)

.MODEL 2N2907A PNP (IS=1.E-12 BF=100)
\end{vquote}

The models \texttt{Q2N2222}, \texttt{2N3700} and \texttt{2N2907A} could then be
used in a netlist by including the \texttt{bjtmodels.lib} file.

\begin{vquote}
.INCLUDE "bjtmodels.lib"
Q1 1 2 3 Q2N2222
Q2 5 6 7 2N3700
Q3 8 9 10 2N2907A
*other netlist entries
.END
\end{vquote}

Because the contents of an included file are simply inserted into the netlist
at the point where the \texttt{.INCLUDE} statement appears, the scoping rules
for \texttt{.INCLUDE} statements are the same as for other types of definitions
as outlined in the preceding subsections. 

NOTE:	The path to the library file is assumed to be relative to the execution
directory, but absolute pathnames are permissible.  The entire file name,
including its ``extension'' must be specified.  There is no assumed default
extension.

\subsection{Model Library Configuration using \texttt{.LIB}}

An alternative technique for organizing model libraries employs the
\texttt{.LIB} command.  With \texttt{.LIB}, a library file can contain
multiple versions of a model and specific versions may be selected at
the top level using a keyword on the \texttt{.LIB} line.  

There are two different uses for the \texttt{.LIB} command.  In the
main netlist, \texttt{.LIB} functions in a similar manner to
\texttt{.INCLUDE}: it reads in a file.  Inside that file,
\texttt{.LIB} and \texttt{.ENDL} are used to specify blocks of model
code that may be included independently of other parts of the same
file.

As an example, if you had two different 2N2222 transistor models extracted 
at different \textrmb{TNOM} values, you could define them in a model library
inside \texttt{.LIB}/\texttt{.ENDL} pairs:

\begin{vquote}
* transistors.lib file
.lib roomtemp
.MODEL Q2N2222 NPN (TNOM=27 Is=14.34f Xti=3 Eg=1.11 Vaf=74.03 Bf=5 Ne=1.307
+  Ise=14.34f Ikf=.2847 Xtb=1.5 Br=6.092 Nc=2 Isc=0 Ikr=0 Rc=1
+  Cjc=7.306p Mjc=.3416 Vjc=.75 Fc=.5 Cje=22.01p Mje=.377 Vje=.75
+  Tr=46.91n Tf=411.1p Itf=.6 Vtf=1.7 Xtf=3 Rb=10)
.endl

.lib hightemp
.MODEL Q2N2222 NPN (TNOM=55 [...parameters omitted for brevity...])
.endl
\end{vquote}

Note that both models are given identical names, but are enclosed
within \texttt{.LIB}/\texttt{.ENDL} pairs with different names.  When
this file is used in a netlist, a specific model can be used by
specifying it on the \texttt{.LIB} line in the main netlist.

\begin{vquote}
*This netlist uses only the high temperature model from the library
.lib transistors.lib hightemp
Q1 collector base emitter Q2N2222
[...]
\end{vquote}

The exact format and usage of the \texttt{.LIB} command is documented in the
\Xyce{} Reference Guide\ReferenceGuide{}.

\section{Model Interpolation}
\label{Model_Interpolation}
\index{model!model interpolation}
\index{model!tempmodel}

Traditionally, SPICE simulators handle thermal effects by coding
the temperature dependence of model parameters into each device.  These
expressions modify the nominal device parameters given in the
\texttt{.MODEL} card when the ambient temperature is not equal to
\textrmb{TNOM}, the temperature at which the nominal device parameters
were extracted.

These temperature correction equations may be reasonable at
temperatures close to \textrmb{TNOM}, but Sandia users of \Xyce{} have
found them inadequate when simulations must be performed over a wide
range of temperatures.  To address this inadequacy, \Xyce{} implements
a model interpolation option that allows the user to specify multiple
\texttt{.MODEL} cards, each extracted from real device measurements at
a different \textrmb{TNOM}.  From these model cards, \Xyce{} will
interpolate parameters based on the ambient temperature using either
piecewise linear or quadratic interpolation.

Interpolation of models is accessed through the model parameter
\textrmb{TEMPMODEL} in the models that support this capability.  In the netlist,
a base model is specified, and is followed by multiple models at other
temperatures.  

Interpolation of model cards in this fashion is implemented in the BJT
level 1, JFET, MESFET, and MOSFETS levels 1-6, 10, and 18.

The use of model interpolation is best shown by example:

\begin{vquote}
Jtest 1a 2a 3 SA2108 TEMP= 40
*
.MODEL SA2108 PJF ( TEMPMODEL=QUADRATIC TNOM = 27
+ LEVEL=2 BETA= 0.003130 VTO = -1.9966 PB = 1.046
+ LAMBDA = 0.00401 DELTA = 0.578; THETA = 0;
+ IS = 1.393E-10          RS = 1e-3)
*
.MODEL SA2108 PJF ( TEMPMODEL=QUADRATIC TNOM = -55
+ LEVEL=2 BETA = 0.00365 VTO = -1.9360 PB = 0.304
+ LAMBDA = 0.00286 DELTA = 0.2540 THETA = 0.0
+ IS = 1.393E-10 RD = 0.0 RS = 1e-3)
* 
.MODEL SA2108 PJF ( TEMPMODEL=QUADRATIC TNOM = 90
+ LEVEL=2 BETA = 0.002770 VTO = -2.0350 PB = 1.507
+ LAMBDA = 0.00528 DELTA = 0.630 THETA = 0.0
+ IS = 1.393E-10          RS = 5.66)
\end{vquote}

Note that the model names are all identical for the three \texttt{.MODEL} lines,
and that they all specify \texttt{TEMPMODEL=QUADRATIC}, but with different
\textrmb{TNOM}.  For parameters that appear in all three \texttt{.MODEL} lines,
the value of the parameter will be interpolated using the \texttt{TEMP=} value
in the device line in the first line, which is 40$^\circ$C in this example.
For parameters that are not interpolated, such as \textrmb{RD}, it is not
necessary to include these in the second and third \texttt{.MODEL} lines.

The only valid arguments for \textrmb{TEMPMODEL} are \textrmb{QUADRATIC} and
\textrmb{PWL} (piecewise linear).  The quadratic method includes a limiting
feature that prevents the parameter value from exceeding the range of values
specified in the \texttt{.MODEL} lines.  For example, the \textrmb{RS} value in
the example would take on negative values for most of the interval between -55
and 27, as the value at 90 is very high.  This truncation is necessary as
parameters can easily take on values (such as the negative resistance of
\textrmb{RS} in this example) that will cause a \Xyce{} failure.

With the BJT parameters \textrmb{IS} and \textrmb{ISE}, interpolation is done
not on the parameter itself, but on the the log of the parameter.  This
provides excellent interpolation of these two parameters, that vary over
many orders of magnitude, for quadratic and piecewise linear
temperature dependence.

The interpolation scheme used for model interpolation bases the interpolation
on the difference between the ambient temperature and the \textrmb{TNOM} value
of the first model card in the netlist, which can sometimes lead to poorly
conditioned interpolation.  Thus it is often best that the first model card in
the netlist be the one that has the ``middle'' \textrmb{TNOM}, as in the example
above.  This assures that no matter where in the range of temperature values
the ambient temperature lies, it is a minimal distance from the base point of
the interpolation.

\section{Subcircuit Multiplier \texttt{M}}
\label{subcircuit_multiplier}
\index{multiplier!subcircuit}
\index{subcircuits!multiplier}

Similar to many device models, subcircuits support a scalar multiplier, \texttt{M} on the instance line.
The usage syntax is similar to that of subcircuit parameters, except that it is 
implicitly applied to all devices (see table~\ref{Devices_with_multipliers}) in 
the subcircuit instance which support multipliers.
If the devices in a subcircuit already have multiplier parameters specified, 
the value is multiplied by the subcircuit instance value.  This is applied recursively 
for nested subcircuits.

Note that it is not necessary to specify it as a parameter on the \texttt{.subckt} line.  
Also, the multiplier is \emph{not} considered to be a normal user-defined parameter that can be 
used in expressions.

A very simple example of a subcircuit instance multiplier usage is given in figure~\ref{Subcircuit_Multiplier_Example_1}.  In this example, a multiplier, \texttt{M=5}, is given on the subcircuit line, \texttt{Xtest}.   This multiplier is automatically applied to the resistor \texttt{R1} inside the \texttt{Xtest} instance, so the effective multiplier on \texttt{Xtest:R1} is \texttt{M=5}.
\begin{figure}[H]
\begin{centering}
\shadowbox{
\begin{minipage}{0.8\textwidth}
\begin{vquote}
Xtest 2 3 testA \color{blue}M=5\color{black}
.subckt testB A B
R1 A B 0.5  
.ends
\end{vquote}
\end{minipage}
}
\caption{Example subcircuit using a multiplier.\label{Subcircuit_Multiplier_Example_1}}
\index{Example!subcircuit multiplier}
\end{centering}
\end{figure}

An example of nested subcircuits using instance multipliers is given in 
figure~\ref{Subcircuit_Multiplier_Example_2}.  In this example, multipliers are
specified at each level of the hierarchy, including on the resistor, \texttt{R1}.  
As each multiplier is applied multiplicatively, the effective multiplier on 
\texttt{Xtest1:Xtest:R1} is \texttt{M=4*3*5=60}.
\begin{figure}[H]
\begin{centering}
\shadowbox{
\begin{minipage}{0.8\textwidth}
\begin{vquote}

Xtest1 2 3 testA \color{blue}M=5\color{black}

.subckt testA A B
Xtest A B testB \color{blue}M=3\color{black}
.ends

.subckt testB A B
R1 A B 0.5  \color{blue}M=4\color{black}
.ends

\end{vquote}
\end{minipage}
}
\caption{Example nested subcircuits using a multiplier.\label{Subcircuit_Multiplier_Example_2}}
\index{Example!subcircuit multiplier}
\end{centering}
\end{figure}

A slightly more complicated example of nested subcircuits, which also makes use of expressions, 
is given in figure~\ref{Subcircuit_Multiplier_Example_3}.  In this example the effective multiplier
on \texttt{Xtest1:Xtest:R1} will be \texttt{M=4*0.5*2*15=60}.
\begin{figure}[H]
\begin{centering}
\shadowbox{
\begin{minipage}{0.8\textwidth}
\begin{vquote}
Xtest1 2 3 testA  \color{blue}M=15\color{black}

.subckt testA A B
.param fred=2
Xtest A B testB \color{blue}M='0.5*fred'\color{black}
.ends

.subckt testB A B
R1 A B 0.5 \color{blue}M=4\color{black}
.ends

\end{vquote}
\end{minipage}
}
\caption{Example nested subcircuits using a multiplier, where a multiplier is set with an expression.\label{Subcircuit_Multiplier_Example_3}}
\index{Example!subcircuit multiplier}
\end{centering}
\end{figure}

As noted above, the \texttt{M} subcircuit instance multiplier is not considered to be 
a parameter available to expressions.  So, attempting to use it in an expression will
not work, resulting in a fatal error.
However, it is allowed to declare a parameter named \texttt{M}, and use it in combination with multlipliers.
For example of such usage, see figure~\ref{Subcircuit_Multiplier_Example_4}.  
\begin{figure}[H]
\begin{centering}
\shadowbox{
\begin{minipage}{0.8\textwidth}
\begin{vquote}
Xtest1 2 3 testA  M=15

.subckt testA A B 
Xtest A B testB M=2
.ends

.subckt testB A B M=3
R1 A B 0.5 M='4*M'
.ends

\end{vquote}
\end{minipage}
}
\caption{Example of \texttt{M} being used as a subcircuit multiplier and a user-defined parameter.\label{Subcircuit_Multiplier_Example_4}}
\index{Example!subcircuit multiplier}
\end{centering}
\end{figure}
In this case, \texttt{M} is being declared as a user-defined parameter of value \texttt{M=3} inside the subcircuit \texttt{testB}, and this parameter is used inside the right-hand-side expression of \texttt{M='4*M'}.   The expression \texttt{4*M} will evaluate to \texttt{12}.  The other \texttt{M} keywords in this netlist are multipliers on the subcircuit instance lines, and they are applied in the usual way.  As a result, the final multiplier on \texttt{Xtest1:Xtest:R1} is \texttt{M=12*2*15=360}.

%%% Local Variables:
%%% mode: latex
%%% End:

%%% END of Xyce_UG_ch05.tex ************
