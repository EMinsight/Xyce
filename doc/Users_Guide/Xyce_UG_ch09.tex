% Sandia National Laboratories is a multimission laboratory managed and
% operated by National Technology & Engineering Solutions of Sandia, LLC, a
% wholly owned subsidiary of Honeywell International Inc., for the U.S.
% Department of Energy’s National Nuclear Security Administration under
% contract DE-NA0003525.

% Copyright 2002-2023 National Technology & Engineering Solutions of Sandia,
% LLC (NTESS).

%%-------------------------------------------------------------------------
%% Purpose        : Main LaTeX Xyce Users' Guide
%% Special Notes  : Graphic files (pdf format) work with pdflatex.  To use
%%                  LaTeX, we need to use postcript versions.  Not sure why.
%% Creator        : Scott A. Hutchinson, Computational Sciences, SNL
%% Creation Date  : {05/23/2002}
%%
%%-------------------------------------------------------------------------

\chapter{Analysis Types}
\label{Analysis_Chap}

\chapteroverview{Chapter Overview}
{
This chapter describes the different analysis types
available in \Xyce{}.  It includes the following sections:
\begin{XyceItemize}
\item Section~\ref{analysis_intro},     {\em Introduction}
\item Section~\ref{DC_Analysis},        {\em DC Analysis}
\item Section~\ref{Transient_Analysis}, {\em Transient Analysis}
\item Section~\ref{STEP_Analysis},      {\em STEP Parametric Analysis}
\item Section~\ref{SAMPLING_Analysis},  {\em Sampling Analysis}
\item Section~\ref{PCE_Analysis},      {\em Polynomial Chaos Expansion Analysis}
\item Section~\ref{HB_Analysis},        {\em Harmonic Balance Analysis}
\item Section~\ref{AC_Analysis},        {\em AC Analysis}
\item Section~\ref{NOISE_Analysis},     {\em Noise Analysis}
\item Section~\ref{SENS_Analysis},      {\em Sensitivity Analysis}
\item Section~\ref{SP_Analysis},        {\em S-parameter Analysis}
\end{XyceItemize}
}

\section{Introduction}
\label{analysis_intro}

\Xyce{} supports several simulation analysis options, including DC
bias point (\texttt{.DC}, section~\ref{DC_Analysis}), transient
(\texttt{.TRAN}, section~\ref{Transient_Analysis}), AC (\texttt{.AC},
section~\ref{AC_Analysis}), Noise (\texttt{.NOISE},
section~\ref{NOISE_Analysis}), harmonic balance (\texttt{.HB}, section
~\ref{HB_Analysis}), and sensitivity (\texttt{.SENS},
section~\ref{SENS_Analysis}) analysis.
%, and multitime partial differential equation (PDE) (\texttt{.MPDE}, section~\ref{MPDE_Analysis}) analysis. 

Using \texttt{.STEP}(section~\ref{STEP_Analysis}), \Xyce{} can also
apply an outer parametric loop to any type of analysis. This allows
one (for example) to sweep a model parameter and perform a transient
simulation for each parameter value.

Using \texttt{.SAMPLING} or \texttt{.EMBEDDEDSAMPLING} 
(section~\ref{SAMPLING_Analysis}), \Xyce{} can
apply random sampling loop to any type of analysis.  This will compute
statistical moments for various circuit outputs.

Alternatively, one can use several types of Polynomial Chaos Expansion (PCE) 
methods in \Xyce{} to propagate uncertainty thru a calculation (section~\ref{PCE_Analysis}).
Similar to the sampling methods, these methods can also be used to compute
statistical moments for various circuit outputs.

There are some analysis types typically found in SPICE-style
simulators that are still a work in progress for \Xyce{}. Operating
point analysis (\texttt{.OP}, section ~\ref{OP_Analysis}) is partially
supported in \Xyce{}.

% -------------------------------------------------------------------------
% DC Analysis Section ---------------------------------------------------
% -------------------------------------------------------------------------
\section{Steady-State (.DC) Analysis}
\label{DC_Analysis}
\label{DC_Sweep_Overview}
\index{analysis!DC} \index{DC analysis}
\index{DC Sweep} \index{analysis!DC sweep}

The DC sweep analysis capability in \Xyce{} computes the DC bias point
of a circuit for a range of values of input sources.  DC sweep is
supported for a source or device parameter, through a range of
specified values.  As the sweep proceeds, \Xyce{} computes the bias
point\index{bias point} for each value in the specified range of the
sweep.

If the variable to be swept is a voltage or current source, a DC
source must be used and its value set in the netlist (see \Xyce{}
Reference Guide\ReferenceGuide{}). In simulating the DC response of an
analog circuit, \Xyce{} eliminates time dependence from the circuit by
treating capacitor elements as open circuits and inductor elements as
short circuits, while using only the DC values of voltage and current
sources.

\subsection{.DC Statement}

To specify a \verb|.DC| analysis, include a \verb|.DC| line in the netlist.  Some examples of typical \verb|.DC| lines are:

\Example{\\
\texttt{ .DC V1  7m 5m -1m } \\
\texttt{ .DC I1  5u 10u 1u } \\
\texttt{ .DC M1:L  7u 5u -1u } \\
\texttt{ .DC OCT V0 0.125 64 2 } \\
\texttt{ .DC DEC R1 100 10000 3 } \\
\texttt{ .DC TEMP LIST 10.0 15.0 18.0 27.0 33.0 }\\
\texttt{ .DC data=table } \\
\texttt{ .param init=7m, final=5m, step=-1m } \\
\texttt{ .DC V1  \{init\} \{final\} \{step\} } 
}

The examples include several types of sweep --- linear, octave,
decade, list and data.  They also demonstrate sweeping over voltage
and current sources as well as device parameters.  The \Xyce{}
Reference Guide\ReferenceGuide{} provides a complete description of
each.

\subsection{Setting Up and Running a DC Sweep}
\label{Running_DC_Sweep}
\index{DC sweep!running}

Following the example given in section~\ref{DC_Sweep},
figure~\ref{Clipper_Netlist3} shows the diode clipper circuit netlist
with a DC sweep analysis specified.  Here, the voltage source
\texttt{Vin} is swept from -10 to 15 in 1-volt increments, resulting
in 26 DC operating point calculations.

NOTE: \Xyce{} ignores the default setting for \texttt{Vin} during
these calculations.  All other source values use the specified values
(in this case, \texttt{VCC = 5V}).

Running \Xyce{} on this netlist produces an output results file named
\verb|clipper.cir.prn|.  Obtaining this file requires specifying the
\verb|.PRINT DC| line. Plotting this data produces the graph shown in
figure~\ref{Clipper_DCSweep2}.

\begin{figure}[htbp]
\begin{centering}
\shadowbox{
\begin{minipage}{0.8\textwidth}
\begin{vquote}

Diode Clipper Circuit
** Voltage Sources
VCC 1 0 5V 
VIN 3 0 0V
* Analysis Command
\color{XyceRed}.DC VIN -10 15 1\color{black}
* Output
.PRINT DC V(3) V(2) V(4)
* Diodes
D1 2 1 D1N3940 D2 0 2 D1N3940
* Resistors
R1 2 3 1K
R2 1 2 3.3K
R3 2 0 3.3K
R4 4 0 5.6K
* Capacitor
C1 2 4 0.47u
.MODEL D1N3940 D(
+ IS=4E-10 RS=.105 N=1.48 TT=8E-7
+ CJO=1.95E-11 VJ=.4 M=.38 EG=1.36
+ XTI=-8 KF=0 AF=1 FC=.9
+ BV=600 IBV=1E-4)
.END

\end{vquote}
\end{minipage}
}
\caption{Diode clipper circuit netlist for DC sweep analysis.\label{Clipper_Netlist3}}

\end{centering}
\end{figure}

\begin{figure}[htbp]
\begin{centering}
  \shadowbox{ \includegraphics[width=4.5in]{clipper-dcsweep} }
\caption{DC sweep voltages at Vin, node 2 and Vout.\label{Clipper_DCSweep2}}
\end{centering}
\end{figure}

\subsection{OP Analysis}
\label{OP_Analysis}
\index{\texttt{.OP}}
\index{DC sweep!OP Analysis}
\index{OP analysis}

\Xyce{} also supports \texttt{.OP} analysis statements.  In \Xyce{},
consider \texttt{.OP} as a shorthand for a single-step DC sweep, in
which all the default operating point values are used.  One may also
consider \texttt{.OP} analysis to be the operating point calculation
that would occur as the initial step to a transient calculation,
without the subsequent time steps.

This capability was mainly added to enable the code to handle legacy
netlists using this analysis statement type.  In most versions of
SPICE, using \texttt{.OP} results in extra output not available from a
DC sweep.  \Xyce{} will also output some of this extra information about
devices, but the capability is not fully implemented.

\subsection{Output}
\label{DC_Output}\index{\texttt{.PRINT}!\texttt{DC}}

During analysis a number of output files may be generated.  The
selection of which files are created depends on a variety of factors,
most obvious of which is the \texttt{.PRINT} command.
Table~\ref{DC_Output_table} lists the format options and files created.
The column labeled ``Additional Columns'' lists the additional data that
is written, though not specified on the \texttt{.PRINT} line.

\begin{table}[htbp]
  \caption{Output generated for DC analysis \label{DC_Output_table}}
  \begin{tabularx}{\linewidth}{|p{2.75in}|Y|Y|}
    \rowcolor{XyceDarkBlue} \color{white}\textbf{Command} & \color{white}\textbf{Files} & \color{white}\textbf{Additional Columns} \\ \hline
\texttt{.PRINT DC} & \emph{circuit-file}.prn & INDEX TIME \\ \hline
\texttt{.PRINT DC FORMAT=NOINDEX} & \emph{circuit-file}.prn & TIME \\ \hline
\texttt{.PRINT DC FORMAT=CSV} & \emph{circuit-file}.csv & TIME \\ \hline
\texttt{.PRINT DC FORMAT=RAW} & \emph{circuit-file}.raw & TIME \\ \hline
\texttt{.PRINT DC FORMAT=TECPLOT} & \emph{circuit-file}.dat & TIME \\ \hline
\texttt{.PRINT DC FORMAT=PROBE} & \emph{circuit-file}.csd & TIME \\ \hline

\texttt{\emph{Xyce} -r raw-file-name} & \emph{raw-file-name} & All circuit variables printed \\ \hline
\texttt{\emph{Xyce} -r raw-file-name -a} & \emph{raw-file-name} & All circuit variables printed \\ \hline

\texttt{.OP} & \emph{log-file} & Operating point information \\ \hline

  \end{tabularx}
%% \index{sources!time-dependent}
\end{table}





%%%%%%
% -------------------------------------------------------------------------
% Transient Analysis Section ----------------------------------------------
% -------------------------------------------------------------------------
\clearpage
\section{Transient Analysis}
\label{Trans_Overview}
\label{Transient_Analysis}
\index{analysis!transient} \index{transient analysis}
\index{\texttt{.TRAN}}

The transient response analysis simulates the response of the circuit
from \texttt{TIME=0} to a specified time.  Throughout a transient
analysis, any or all of the independent sources may have
time-dependent values.

In \Xyce{} (and most other circuit simulators), the transient analysis
begins by performing its own bias point\index{bias point} calculation
at the beginning of the run, using the same method as used for DC
sweep. This is required to set the initial conditions for the
transient solution as the initial values of the sources may differ
from their DC values.

\subsection{.TRAN Statement}

To run a transient simulation, the circuit netlist file must contain a
\verb|.TRAN| command.

\Example{\\
\texttt{ .TRAN 100us 300ms } \\
\texttt{ .TRAN 100p 12.05u 9.95u }
}
\Example{\\
\texttt{.param tstep=100us, final=300ms } \\
\texttt{.TRAN \{tstep\} \{final\} } 
}

The \Xyce{} Reference Guide\ReferenceGuide{} provides a detailed
explanation of the \verb|.TRAN| statement. The netlist must also
contain one of the following:

\begin{XyceItemize}
\item Independent, transient source (see table~\ref{Time_Sources}),
\item Initial condition on a reactive element, or
\item Time-dependent analog behavioral modeling source (see chapter~\ref{Behavioral_Modeling})
\end{XyceItemize}

\subsection{Defining a Time-Dependent (transient) Source}
\label{Defining_Source}
\index{sources!defining time-dependent}

\subsubsection{Overview of Source Elements}

Source\index{sources} elements, either voltage or current, are entered
in the netlist\index{netlist!sources} file as described in the \Xyce{}
Reference Guide\ReferenceGuide{}.  Table~\ref{Time_Sources} lists the
time-dependent sources available in \Xyce{} for either voltage or
current.  For voltage sources, the name is preceded by \texttt{V} while
current sources are preceded by \texttt{I}.

\begin{table}[htbp]
  \caption{Summary of \Xyce{}-supported time-dependent sources \label{Time_Sources}}
  \begin{tabularx}{\linewidth}{|Y|Y|}
    \rowcolor{XyceDarkBlue} \color{white}\bf Source Element Name &
    \color{white}\bf Description \\ \hline

    EXP & Exponential Waveform \\ \hline
    PAT & Pattern Waveform \\ \hline
    PULSE & Pulse Waveform \\ \hline
    PWL & Piecewise Linear Waveform \\ \hline
    SFFM & Frequency-modulated Waveform \\ \hline
    SIN & Sinusoidal Waveform \\ \hline

  \end{tabularx}
\index{sources!time-dependent}
\end{table}

To use time-dependent or transient sources, place the source element
line in the netlist and characterize the transient behavior using the
appropriate parameters.  Each transient source element has a separate
set of parameters dependent on its transient behavior.  In this way,
the user can create analog sources that produce sine wave, square
pulse, exponential pulse, single-frequency FM, and piecewise linear
(PWL) waveforms.

\subsubsection{Defining Transient Sources}

To define a transient source, select one of the supported sources:
independent voltage or current, choose a transient source type from
table~\ref{Time_Sources}, and provide the transient parameters (refer
to the \Xyce{} Reference Guide\ReferenceGuide{} to fully define the
source).

The following example of an independent sinusoidal voltage source in a
circuit netlist creates a voltage source between nodes 1 and 5
that oscillates sinusoidally between -5V and +5V with a frequency of
50 KHz.  The arguments specify an offset of -5V, a 10V amplitude, and
a 50KHz frequency, in that order.

\Example{\texttt{Vexample 1 5 SIN(-5V 10V 50K)}}


\subsection{Transient Time Steps}
\label{Time_Steps}
\index{solvers!transient} \index{time step!size}

During the simulation, \Xyce{} uses a calculated time step that is
continuously adjusted for accuracy and efficiency
(see~\cite{WKHH:2000} and ~\cite{Petzold:1996}).  Calculation time
step increases during periods of circuit idleness, and decreases
during dynamic portions of the waveform.  \index{time step!maximum
  size} Users may control the maximum internal step size by specifying
the step's ceiling value in the \verb|.TRAN| command (see the \Xyce{}
Reference Guide\ReferenceGuide{}).

The internal calculation time steps used might not be consistent with
the user-requested \index{output!time values} output time steps.  By
default, \Xyce{} outputs solution results at every time step it
calculates.  If the user selects output timesteps via the
\verb|.OPTIONS OUTPUT| statement (see chapter~\ref{Output}), then
\Xyce{} will output results at the interval requested, interpolating
solution variables to desired output times if necessary.


\subsection{Time Integration Methods}
\label{TransientControls}
\index{solvers!transient} \index{time step!size}

For a transient analysis, several time integration methods can be
selected to solve the circuit model's differential algebraic
equations.  The following algorithms are available:

\begin{XyceItemize}
\item Variable order Trapezoidal (combines Trapezoidal and Backward Euler) 
\item Gear method, orders 1-2.
\end{XyceItemize}




You can set the \verb|method|, \verb|maxord| and \verb|minord|
parameters to select the time integration methods via a .OPTIONS
line. The following table shows the possible settings for those three
parameters. (Note: Consult the \Xyce{} Reference Guide\ReferenceGuide
for the exact syntax of the .OPTIONS line for each time integration
method.) The default time integration method in \Xyce{} is Trap, which
is the same as SPICE, PSpice and HSPICE.

\begin{table}[htbp]
  \caption{Summary of \Xyce{}-supported time integration methods \label{Time_integration}}
  \begin{tabularx}{\linewidth}{|Y|Y|Y|}
    \rowcolor{XyceDarkBlue} \color{white}\bf  Integration Methods &
    \color{white}\bf  Option Settings & \color{white}\bf Comments \\ \hline
    Backward-Euler & method=trap maxord=1 & Backward-Euler only \\ \hline
    Trap & method=trap & combines Trapezoidal and Backward Euler (default) \\ \hline
    Trap only & method=trap minord=2 & Trapezoidal only \\ \hline
    Gear &  method=gear &  combines Backward Euler and 2nd order Gear \\ \hline
    Gear2 only &  method=gear minord=2 & 2nd order Gear only  \\ \hline
  \end{tabularx}
\index{Time integration!integration method}
\index{\texttt{.OPTIONS}!\texttt{TIMEINT}!\texttt{METHOD}} \index{\texttt{.OPTIONS}!\texttt{TIMEINT}!\texttt{MINORD}} 
\end{table}



The Trapezoidal method is often the preferred method because it is
accurate and fast.  However, this method can exhibit artificial
point-to-point ringing, which can be controlled by using tighter
tolerances. If a circuit fails to converge with the Trapezoidal method
then you can re-run the transient analysis using the Gear method.

The Gear method may help convergence for some circuits. The 2nd order
Gear method is typically more accurate than the Backward-Euler method
(1st order Gear). However, both of these methods are overly stable
methods, and they can damp the actual circuit behavior when simulating
high-Q resonators such as oscillators.  The Backward-Euler method has
more damping effect than the 2nd order Gear method.  This effect can
be alleviated by using tighter tolerances in the simulations. However,
it is suggested to use the pure Trapezoidal method for oscillators.

\subsection{Error Controls}
\label{Time_Step_Selection}
\index{solvers!transient} \index{time step!size} \index{time step!how to select}

There are two basic time-step error control methods in \Xyce{} ---
Local Truncation Error (LTE) based and non-LTE based.


\subsubsection{Local Truncation Error (LTE) Strategy}
\index{\texttt{.OPTIONS}!\texttt{TIMEINT}!\texttt{RELTOL}} \index{\texttt{.OPTIONS}!\texttt{TIMEINT}!\texttt{ABSTOL}}

All time integration methods use the LTE-based strategy by
default. The accuracy of the simulation can be controlled by
specifying appropriate relative and absolute error tolerances (RELTOL
and ABSTOL).

\Example{\\
\texttt{ .OPTIONS TIMEINT RELTOL=1e-4 ABSTOL=1e-8 } \\
}

The total tolerance of LTE is
{\\
\texttt{ $Tol_{LTE}$  = abstol +  reltol*ref}  
}

The parameter \verb|ref| is the reference value that the relative
error is compared to. It can be controlled by setting \verb|newlte|
option.

\index{\texttt{.OPTIONS}!\texttt{TIMEINT}!\texttt{NEWLTE}}
\Example{\\
\texttt{ .OPTIONS TIMEINT NEWLTE=1 } \\
}

The  choices for \verb|newlte| option are:

\begin{XyceItemize} 
\item 0. The reference value is the current value on each node. This is the most conservative and least used.
\item 1. The reference value is the maximum of all the signals at the current time. This is the default value.
\item 2. The reference value is the maximum of all the signals over all past time. This is the loosest criterion. It normally produces the best performance and should be used if the overall size of the signals is roughly the same on all nodes.
\item 3. The reference value is the maximum value on each signal over all past time. This should be used if the scale of signals varies widely in a system.
\end{XyceItemize}

The Trapezoid integrator algorithm introduces no numerical
dissipation.  So, a strong ringing (artificially introduced by the
numerical algorithm) will occur when sources or models introduce
discontinuities.  This can result in a large local truncation error
estimate, ultimately leading to a ``time-step too small'' error.  In
this case, using the Gear method or a non-LTE strategy may help.

\subsubsection{Non-LTE Strategy}
The non-LTE strategy used in \Xyce{} is based on success of the
nonlinear solve, and is enabled by setting ERROPTION=1.  Since the
step-size selection is based only upon nonlinear iteration statistics
rather than accuracy, it is highly suggested that DELMAX be specified,
in a circuit-specific manner, for all three time integrators.  The
purpose of DELMAX is to limit the largest time step taken.

\Example{\\
\texttt{ .OPTIONS TIMEINT ERROPTION=1 DELMAX=1.0e-4 } \\
}

For the Trapezoid and Gear integrators, the options are slightly more
refined.  If the number of nonlinear iterations is below NLMIN, then
the step size is doubled. If the number of nonlinear iterations is
above NLMAX then the step size is cut by one eighth.  In between, the
step size is not changed.  An example using Trap (METHOD=7) is given
below.  \index{\texttt{.OPTIONS}!\texttt{TIMEINT}!\texttt{NLMIN}}
\index{\texttt{.OPTIONS}!\texttt{TIMEINT}!\texttt{NLMAX}}
\index{\texttt{.OPTIONS}!\texttt{TIMEINT}!\texttt{DELMAX}}

\Example{\\
\texttt{ .OPTIONS TIMEINT METHOD=7 ERROPTION=1 NLMIN=3 NLMAX=8 DELMAX=1.0e-4 } \\
}

If the number of Newton iterations is bigger than NLmax and
TIMESTEPSREVERSAL is not set, then \Xyce{} will cut the next step.  If
the number of Newton iterations is bigger than NLmax and
TIMESTEPSREVERSAL is set, then \Xyce{} will reject the current step
and also cut the current step.

\Example{\\
\texttt{ .OPTIONS TIMEINT METHOD=7 ERROPTION=1 DELMAX=1.0e-4 TIMESTEPSREVERSAL=1 } \\
}

\subsection{Breakpoints}

\index{\texttt{.OPTIONS}!\texttt{TIMEINT}!\texttt{BREAKPOINTS}} 

It is often necessary or desirable for the time integrator to be
forced to land on a specific time point and restart integration from
that point.  The most common scenario for which this is necessary is
when an device (such as a \texttt{PULSE} or a \texttt{PWL} source)
produces a discontinuity.  If a discontinuity is present, and no
breakpoint is set, then the LTE analysis will force the time
integrator to take really small time steps, and this can possibly
impact the robustness of the calculation.

Fortunately, \Xyce{} handles most device-based discontinuities
automatically, so it is not necessary for the user to worry about
them.  However, it can be desirable for the user to manually set
breakpoints from the netlist.  In a \Xyce{} netlist, these are
specified using the BREAKPOINTS parameter with a comma-separated list
in the following \texttt{.OPTIONS TIMEINT} statement.

\Example{\\
\texttt{ .OPTIONS TIMEINT BREAKPOINTS=1ms,2ms,3ms} \\
}

\subsection{Checkpointing and Restarting}
\label{Restart}
\index{checkpoint} \index{restart}

\Xyce{} was designed to simulate large, complex circuits over long
simulation runs.  Because complex simulations can take many hours (or
even days) to complete, it can sometimes be helpful to use
``checkpointing.''  When checkpointing is used, \Xyce{} periodically
saves its complete simulation state.  The saved state can be used to
restart \Xyce{} from one of these ``checkpoints.''  In the event of a
computer crash, power outage, or should the simulation need to be
interrupted for some other reason, checkpointing allows the user to
restart a long simulation in the middle of a run without having to
start over.

\Xyce{} uses the \index{netlist!restart} \index{\texttt{.OPTIONS}!\texttt{RESTART}} \verb|.OPTIONS RESTART|
netlist command to control all checkpoint output and restarting.  

\subsubsection{Checkpointing Command Format}
\index{checkpoint!format} \index{restart!format} \index{\texttt{.OPTIONS}!\texttt{RESTART}}

\begin{XyceItemize}
\item \verb+.OPTIONS RESTART [PACK=<0|1>] JOB=<job name> INITIAL_INTERVAL=<interval>+ \\
  \verb+[[<t0> <i0> [<t1> <i1>...]]]+

  \texttt{PACK=<0|1>} indicates whether restart data files will contain byte-packed (binary) data(\texttt{PACK=1}, the default) or unpacked (ASCII)(\texttt{PACK=0}).
  \texttt{JOB=<job name>} identifies the prefix for restart files.  The actual restart files will be the job name appended with the current simulation time (e.g., \texttt{name1e-05} for \texttt{JOB=name} and simulation time 1e-05 seconds).  Furthermore, the\\ \texttt{INITIAL\_INTERVAL=<interval>} identifies the initial interval time used for restart output; this parameter must be given.  The \texttt{<tx ix>} intervals identify times (\texttt{tx}) at which the output interval (\texttt{ix}) will change.  This functionality is identical to that described for the \index{\texttt{.OPTIONS}!\texttt{OUTPUT}}
  \texttt{.OPTIONS OUTPUT} command (section~\ref{Output_Control}).
\item Example --- Generate checkpoints every 0.1 $\mu s$:
\begin{vquote}
.OPTIONS RESTART JOB=checkpt INITIAL_INTERVAL=0.1us
\end{vquote}
\item Example --- Generate unpacked checkpoints every 0.1 $\mu s$:
\begin{vquote}
.OPTIONS RESTART PACK=0 JOB=checkpt INITIAL_INTERVAL=0.1us
\end{vquote}
\item Example --- Initial interval of 0.1 $\mu s$, at 1 $\mu s$ in the
  simulation, change to interval of 0.5 $\mu s$, and at 10 $\mu s$ change to an
  interval of 0.1 $\mu s$:
\begin{vquote}
.OPTIONS RESTART JOB=checkpt INITIAL_INTERVAL=0.1us 1us 0.5us
+ 10us 0.1us
\end{vquote}
\end{XyceItemize}

\subsubsection{Restarting Command Format}
\index{restart!format}

\begin{XyceItemize}
\item \index{\texttt{.OPTIONS}!\texttt{RESTART}}
\verb+.OPTIONS RESTART <FILE=<filename> | JOB=<job name> START_TIME=<time>>+ \\
\verb|+ [INITIAL_INTERVAL=<interval> [<t0> <i0> [<t1> <i1> ...]]]|
\end{XyceItemize}
To restart\index{restart} from an existing restart file, specify the file by using either the\\%
\texttt{FILE=<filename>} parameter to explicitly request a file or\\%
\texttt{JOB=<job name> START\_TIME=<time>} to specify a file prefix and a
specific time.  The time must exactly match an output file time for the
simulator to correctly load the file.  

To continue checkpointing the simulation in a restarted run, append
\texttt{INITIAL\_INTERVAL=<interval>} and the following intervals to
the command in the same format as previously described.  Without these
additional parameters, the simulation will restart as requested, but
will not generate further checkpoint files.

\begin{XyceItemize}
\item Example\index{Example!restarting} --- Restart from checkpoint file at 0.133
  $\mu s$:
\begin{vquote}
.OPTIONS RESTART JOB=checkpt START_TIME=0.133us
\end{vquote}
\item Example --- Restart from checkpoint file at 0.133 $\mu s$ :
\begin{vquote}
.OPTIONS RESTART FILE=checkpt0.000000133
\end{vquote}
\item Example --- Restart from 0.133 $\mu s$ and continue checkpointing at 0.1
      $\mu s$ intervals:
\begin{vquote}
.OPTIONS RESTART FILE=checkpt0.000000133 JOB=checkpt_again
+ INITIAL_INTERVAL=0.1us
\end{vquote}
\end{XyceItemize}
\subsection{Output}
\label{Transient_Output}\index{\texttt{.PRINT}!\texttt{TRAN}}

During analysis a number of output files may be generated.  The
selection of which files are created depends on a variety of factors,
most obvious of which is the \texttt{.PRINT} command.
Table~\ref{Tran_Output_table} lists the format options and files
created.  The column labeled ``Additional Columns'' lists the additional
data that is written, though not specified on the \texttt{.PRINT} line.

\begin{table}[htbp]
  \caption{Output generated for Transient analysis \label{Tran_Output_table}}
  \begin{tabularx}{\linewidth}{|p{2.75in}|Y|Y|}
    \rowcolor{XyceDarkBlue} \color{white}\textbf{Command} & \color{white}\textbf{Files} & \color{white}\textbf{Additional Columns} \\ \hline
\texttt{.PRINT TRAN} & \emph{circuit-file}.prn & INDEX TIME \\ \hline
\texttt{.PRINT TRAN FORMAT=NOINDEX} & \emph{circuit-file}.prn & TIME \\ \hline
\texttt{.PRINT TRAN FORMAT=CSV} & \emph{circuit-file}.csv & TIME \\ \hline
\texttt{.PRINT TRAN FORMAT=RAW} & \emph{circuit-file}.raw & TIME \\ \hline
\texttt{.PRINT TRAN FORMAT=TECPLOT} & \emph{circuit-file}.dat & TIME \\ \hline
\texttt{.PRINT TRAN FORMAT=PROBE} & \emph{circuit-file}.csd & TIME \\ \hline

\texttt{\emph{Xyce} -r raw-file-name} & \emph{raw-file-name} & All circuit variables printed \\ \hline
\texttt{\emph{Xyce} -r raw-file-name -a} & \emph{raw-file-name} & All circuit variables printed \\ \hline

\texttt{.OP} & \emph{log-file} & Operating point information \\ \hline

  \end{tabularx}
%% \index{sources!time-dependent}
\end{table}

% -------------------------------------------------------------------------
% STEP Analysis Section ---------------------------------------------------
% -------------------------------------------------------------------------
\clearpage
\section{STEP Parametric Analysis}
\label{STEP_Analysis}
\label{step_Overview}
\index{analysis!STEP} \index{STEP parametric analysis}
\index{\texttt{.STEP}}

The \verb|.STEP| command performs a parametric sweep for all the
analyses of the circuit.  When the \verb|.STEP| command is invoked,
typical analyses, such as \verb|.DC|, \verb|.AC|, and \verb|.TRAN| are
performed for each value of the stepped parameter.

This capability is very similar, but not identical, to the \verb|STEP|
capability in PSpice~\cite{PSpiceUG:1998}.  \Xyce{} can use
\verb|.STEP| to sweep over any device instance or device model
parameter, as well as the circuit temperature.    It can also be used 
to sweep over any user-defined parameter (either \texttt{.GLOBAL\_PARAM} 
or \texttt{.PARAM} ) , as long as that parameter is  defined in the 
global scope.

\subsection{.STEP Statement}

A \verb|.STEP| analysis may be specified by simply adding a
\verb|.STEP| line to a netlist.  Unlike \verb|.DC|, \verb|.STEP| by
itself is not an adequate analysis specification, as it merely
specifies an outer loop around the normal analysis.  A standard
analysis line, either specifying \verb|.TRAN|, \verb|.AC| and
\verb|.DC| analysis, is still required.

Some examples of typical \verb|.STEP| lines are:

\Example{\\
\texttt{ .STEP M1:L  7u 5u -1u } \\
\texttt{ .STEP OCT V0 0.125 64 2 } \\
\texttt{ .STEP DEC R1 100 10000 3 } \\
\texttt{ .STEP TEMP LIST 10.0 15.0 18.0 27.0 33.0 } \\
\texttt{ .STEP data=table }
}

\verb|.STEP| has a format similar to that of the \verb|.DC| format
specification.  In the first example, \verb|M1:L| is the name of the
parameter (in this instance, the length parameter of the MOSFET
\texttt{M1}), \verb|7u| is the initial value of the parameter,
\verb|5u| is the final value of the parameter, and \verb|-1u| is the
step size.  Like \verb|.DC|, \verb|.STEP| in \Xyce{} can also handle
sweeps by decade, octave, a specified lists of values, or multivariate parameter values from a table.  Consult the
\Xyce{} Reference Guide\ReferenceGuide{} for complete explanations of each sweep type.

\subsection{Sweeping over a Device Instance Parameter}
\label{step_InstanceParam}

The first example uses \verb|M1:L| as the parameter, but it could have
used any model or instance parameter existing in the
circuit. Internally, \Xyce{} handles the parameters for all device
models and device instances in the same way.  Users can uniquely
identify any parameter by specifying the device instance name,
followed by a colon (:), followed by the specific parameter name.  For
example, all the MOSFET models have an instance parameter for the
channel length, L.  For a MOSFET instance specified in a netlist,
named M1, the full name for the M1 channel length parameter is M1:L.

Figure~\ref{Step_Netlist_1} provides a simple application of
\verb|.STEP| to a device instance.  This is the same diode clipper
circuit as was used in the transient analysis chapter, except that a
single line (in red) has been added.  This \verb|.STEP| line will
cause \Xyce{} to sweep the resistance of the resistor, R4, from 3.0
KOhms to 15.0 KOhms, in 2.0 KOhms increments, meaning seven transient
simulations will be performed, each one with a different value for R4.

As the circuit is executed multiple times, the resulting output file
is a little more sophisticated.  The \verb|.PRINT| statement is still
used in much the same way as before.  However, the .prn output file
contains the concatenated output of each \verb|.STEP| increment.  The
end of this section provides more details of how \texttt{.STEP}
changes output files.

\begin{figure}[htbp]
\begin{centering}
\shadowbox{
\begin{minipage}{0.8\textwidth}
\begin{vquote}

Transient Diode Clipper Circuit with Step Analysis
* Voltage Sources
VCC 1 0 5V
VIN 3 0 SIN(0V 10V 1kHz)\color{black}
* Analysis Command
.TRAN 2ns 2ms\color{black}
* Output
.PRINT TRAN V(3) V(2) V(4)\color{black}
\color{XyceRed} * Step statement \color{black}
\color{XyceRed}.STEP R4:R 3.0K 15.0K 2.0K \color{black}
* Diodes
D1 2 1 D1N3940
D2 0 2 D1N3940
* Resistors
R1 2 3 1K
R2 1 2 3.3K
R3 2 0 3.3K
R4 4 0 5.6K
* Capacitor
C1 2 4 0.47u
.MODEL D1N3940 D(
+ IS=4E-10 RS=.105 N=1.48 TT=8E-7
+ CJO=1.95E-11 VJ=.4 M=.38 EG=1.36
+ XTI=-8 KF=0 AF=1 FC=.9
+ BV=600 IBV=1E-4)
.END
\end{vquote}
\end{minipage}
}
\caption{Diode clipper circuit netlist for step transient
analysis\label{Step_Netlist_1}}
\end{centering}
\end{figure}

\subsection{Sweeping over a Device Model Parameter}
\label{step_ModelParam}

Sweeping a model parameter can be done in an identical manner to an
instance parameter.  Figure~\ref{Step_Netlist_2} contains the same
circuit as in figure~\ref{Step_Netlist_1}, but with additional
\verb|.STEP| line referring to a model parameter, \verb|D1N3940:IS|.

NOTE: \verb|.STEP| line syntax differs from \verb|.DC| line syntax in
that multiple parameters require separate \verb|.STEP| lines. Each
parameter needs a separate line.

\begin{figure}[htbp]
\begin{centering}
\shadowbox{
\begin{minipage}{0.8\textwidth}
\begin{vquote}

Transient Diode Clipper Circuit with Step Analysis
* Voltage Sources
VCC 1 0 5V
VIN 3 0 SIN(0V 10V 1kHz)\color{black}
* Analysis Command
.TRAN 2ns 2ms\color{black}
* Output
.PRINT TRAN V(3) V(2) V(4)\color{black}
\color{XyceRed} * Step statements \color{black}
\color{XyceRed}.STEP R4:R 3.0K 15.0K 2.0K \color{black}
\color{XyceRed}.STEP D1N3940:IS 2.0e-10 6.0e-10 2.0e-10 \color{black}
* Diodes
D1 2 1 D1N3940
D2 0 2 D1N3940
* Resistors
R1 2 3 1K
R2 1 2 3.3K
R3 2 0 3.3K
R4 4 0 5.6K
* Capacitor
C1 2 4 0.47u
.MODEL D1N3940 D(
+ IS=4E-10 RS=.105 N=1.48 TT=8E-7
+ CJO=1.95E-11 VJ=.4 M=.38 EG=1.36
+ XTI=-8 KF=0 AF=1 FC=.9
+ BV=600 IBV=1E-4)
.END
\end{vquote}
\end{minipage}
}
\caption{Diode clipper circuit netlist for 2-step transient
analysis\label{Step_Netlist_2}}
\end{centering}
\end{figure}

\subsection{Sweeping over Temperature}
\label{step_Temperature}

It is also possible to sweep over temperature.  To do so, simply
specify \verb|temp| as the parameter name.  It will work in the same
manner as \verb|.STEP| when applied to model and instance parameters.

\subsection{Special cases: Sweeping Independent Sources, Resistors, Capacitors and Inductors}
\label{step_SpecialCases}

For some devices, there is generally only one parameter that one would
want to sweep.  For example, a linear resistor's only parameter of
interest is resistance, R.  Similarly, for a DC voltage or current
source, one is usually only interested in the magnitude of the source,
DCV0.  Finally, linear capacitors generally only have capacitance, C,
as a parameter of interest, while inductors generally only have
inductance, L, as a parameter of interest.

For these simple devices, it is not necessary to specify both the
parameter and device on the \texttt{.STEP} line: only the device name
is strictly required, as these devices have default parameters that
are assumed if no parameter name is given explicitly.

Examples of usage are given below.  The first two lines are equivalent
--- in the first line, the resistance parameter of \texttt{R4} is
named explicitly, and in the second line the resistance parameter is
implicit. A similar example is then shown for the DC voltage of the
voltage source \texttt{VCC}.  In the remaining lines, parameter names
are all implicit, and the default parameters of the associated devices
are used.

\Example{ \\
\begin{tabular}{lllll}
  \texttt{.STEP R4:R} &\texttt{3.0K}  &\texttt{15.0K}  &\texttt{2.0K} \\ 
  \texttt{.STEP R4} &\texttt{3.0K}  &\texttt{15.0K}  &\texttt{2.0K} \\
  \texttt{.STEP VCC:DCV0}&\texttt{4.0 }  &\texttt{ 6.0 }  &\texttt{1.0 } \\  
  \texttt{.STEP VCC}&\texttt{4.0 }  &\texttt{ 6.0 }  &\texttt{1.0 } \\ 
  \texttt{.STEP ICC}&\texttt{4.0 }  &\texttt{ 6.0 }  &\texttt{1.0 } \\ 
  \texttt{.STEP C1} &\texttt{0.45u} &\texttt{ 0.50u} &\texttt{0.1u} \\
  \texttt{.STEP L1} &\texttt{0.5m} &\texttt{ 1.0m} &\texttt{0.5m} \\
\end{tabular}
        }

Independent sources require further explanation.  Their default
parameter, DCV0, only applies to \texttt{.DC} analyses.  They do not have
default parameters for their transient forms, such as \texttt{SIN}
or \texttt{PULSE}.

\subsection{Output files}
\label{step_output_files}
\index{output!\texttt{.STEP}}

Users can think of \texttt{.STEP} simulations as several distinct
executions of the same circuit netlist.  The output data, as specified
by a \texttt{.PRINT} line, however, goes to a single (\texttt{*.prn})
file.  For convenience, \Xyce{} also creates a second auxilliary file
with the \texttt{*.res} suffix.

Figure~\ref{Step_Netlist_1} shows an example file named
\verb+clip.cir+, which when run will produce files \verb+clip.cir.res+
and \verb+clip.cir.prn+.  The file \verb+clip.cir.res+ contains one
line for each step, showing what parameter value was used on that
step.  \verb+clip.cir.prn+ is the familiar output format, but the
\verb+INDEX+ field recycles to zero each time a new step begins.  As
the default behavior distinguishes each step's output only by
recycling the \verb+INDEX+ field to zero, it can be beneficial to add
the sweep parameters to the \verb+.PRINT+ line.  For the default file
format (\texttt{format=std}), \Xyce{} will not automatically include
these sweep parameters, so for plotting it is usually best to specify
them by hand.

If using the default \texttt{.prn} file format (\texttt{format=std}),
the resulting \texttt{.STEP} simulation output file will be a simple
concatenation of each step's underlying analysis output.  If using
\texttt{format=probe}, the data for each execution of the circuit will
be in distinct sections of the file, and it should be easy to plot the
results using PROBE.  If using \texttt{format=tecplot}, the results of
each \texttt{.STEP} simulation will be in a distinct Tecplot
zone. Finally, format=raw will place the results for each
\texttt{.STEP} simulation in a distinct ``plot'' region.

% Sandia National Laboratories is a multimission laboratory managed and
% operated by National Technology & Engineering Solutions of Sandia, LLC, a
% wholly owned subsidiary of Honeywell International Inc., for the U.S.
% Department of Energy’s National Nuclear Security Administration under
% contract DE-NA0003525.

% Copyright 2002-2023 National Technology & Engineering Solutions of Sandia,
% LLC (NTESS).

% -------------------------------------------------------------------------
% Sampling Analysis Section ---------------------------------------------------
% -------------------------------------------------------------------------
\clearpage
\section{Random Sampling Analysis}
\label{SAMPLING_Analysis}
\label{sampling_Overview}
\index{analysis!Sampling} \index{Sampling analysis} \index{Random Sampling}
\index{\texttt{.SAMPLING}}
\index{\texttt{.EMBEDDEDSAMPLING}}
\Xyce{} supports two different styles of random sampling analysis.  One is specified 
with the \verb|.SAMPLING| 
command, and the other is specified with the \verb|.EMBEDDEDSAMPLING| commmand.  
These are relatively new capabilities and are still under development.  

The \verb|.SAMPLING| command causes \Xyce{} to execute the primary analysis (\verb|.DC|,
\verb|.AC|, \verb|.TRAN|, etc.) inside a loop over randomly distributed parameters.  
The user specifies a list of random parameters, their distributions, distribution 
parameters (such as mean and standard deviation), and the total number of 
sample points.  The primary analysis is then executed sequentially for each sample point.
The \verb|.SAMPLING| capability uses a lot of the same code base as \verb|.STEP|, 
so most of the guidance for \verb|.STEP|, including legal parameters, output 
file formats, supported analyses types, etc., also applies to \verb|.SAMPLING|.

The \verb|.EMBEDDEDSAMPLING| command also causes \Xyce{} to create a loop over 
randomly distributed parameters, but unlike \verb|.SAMPLING|, this loop is 
implemented inside the solver loop rather than outside of it.  As such, it 
creates a block linear system, in which each matrix block has the structure 
of the original matrix, and the number of blocks is equal to the number of sample points.
As such, all the parameters are propagated simultaneously, rather than sequentially.
Unlike \verb|.SAMPLING|, which can be applied to all types of \Xyce{} analysis, 
\verb|.EMBEDDEDSAMPLING| can only be applied to \verb|.DC| and \verb|.TRAN|.

Both types of sampling methods can be applied to a variety of inputs and outputs.
For inputs,
\Xyce{}  can use \verb|.SAMPLING| or \verb|.EMBEDDEDSAMPLING| to modify 
any device instance or device
model parameter, as well as the circuit temperature.  It can also be used 
to sweep over any user-defined parameter (either \texttt{.GLOBAL\_PARAM} 
or \texttt{.PARAM} ), as long as that parameter is  defined in the 
global scope.

For outputs, both \verb|.SAMPLING| and \verb|.EMBEDDEDSAMPLING| 
can apply statistical analysis (computing moments such as mean and standard deviation) 
to most outputs that appear on the \verb|.PRINT| line.
However, the two analysis types have some differences.    \verb|.SAMPLING| will only 
apply statistical analysis at the end of the calculation, while \verb|.EMBEDDEDSAMPLING| 
can apply statistical analysis at each time step.  Relatedly, 
\verb|.SAMPLING| can also be applied to any \verb|.MEASURE|, but 
\verb|.EMBEDDEDSAMPLING| lacks this capability. 

\subsection{.SAMPLING and .EMBEDDEDSAMPLING Statements}
\label{sampling_statement}
Sampling analysis may be specified by simply adding a \verb|.SAMPLING| 
or a \verb|.EMBEDDEDSAMPLING| line to a netlist.    The format for the two commands is nearly 
identical.

Sampling analysis such as \verb|.SAMPLING| or \verb|.EMBEDDEDSAMPLING| is not considered a ``primary'' analysis type.
This means that such a command by itself is not an 
adequate analysis specification, as it merely specifies an additional parametric loop to augment the
primary analysis.  A standard analysis line, specifying \verb|.TRAN|, \verb|.AC|, \verb|.HB|, 
and \verb|.DC| analysis, is still required.

Examples of typical \verb|.SAMPLING| lines:

\Example{\\
\texttt{\\
.SAMPLING  \\
+ param=M1:L, M1:W \\
+ type=uniform,uniform \\
+ lower\_bounds=5u,5u \\
+ upper\_bounds=7u,7u \\
}
\texttt{ \\
.SAMPLING \\
+ param=R1,c1 \\
+ type=normal,normal \\
+ means=3K,1uF \\
+ std\_deviations=1K,0.2uF \\
}

}

In the first example, \verb|M1:L| and \verb|M1:W| 
are the names of the
parameters (in this instance, the length and width parameters of the MOSFET
\texttt{M1}), \verb|5u| is the minimum value of both parameters and
\verb|7u| is the maximum value of both parameters.  For both parameters, 
the distribution (specified by \texttt{type}) is uniform.
In the second example, the two parameters are \verb|R1| and \verb|c1|.   In this 
case, it isn't necessary to specify the specific parameter of \verb|R1| or \verb|c1| 
as resistors and capacitors have simple default parameters (see section~\ref{sampling_SpecialCases}).  
The distributions of both parameters are normal, and the means and standard deviations 
are specified for both in comma-separated lists.

For any list of parameters, all the required fields must be provided in comma-separated 
lists.  \Xyce{} will throw an error if the length of any of the lists is different 
than that of the parameter list.

\subsection{.options SAMPLES and EMBEDDEDSAMPLES Statements}
\label{options_samples_statement}

To specify the number of samples, use the netlist command \texttt{.options 
SAMPLES numsamples=<value>}.   For sampling analysis to work correctly, this parameter is required.
Other sampling details, including the sampling type, the covariance matrix, 
the statistical outputs and the random seed can be specified using the 
same \verb|.options SAMPLES| netlist command.  Some examples of this command are:

\Example{\\
\texttt{ \\
.options samples numsamples=1000 \\
}
\texttt{ \\
.options samples numsamples=30 \\
+ covmatrix=1e6,1.0e-3,1.0e-3,4e-14 \\
+ OUTPUTS={R1:R},{C1:C} \\
+ MEASURES=maxSine \\
+ SAMPLE\_TYPE=LHS \\
+ SEED=743190581 \\
}
}
The random seed can be set either in the \texttt{.options SAMPLES} statement, or 
it can be set from the command line using the \texttt{-randseed} option.  If it 
is not set either way, then the random seed is generated internally.

In the example there are two different types of outputs specified.  One is using 
the keyword \texttt{OUTPUTS}, and this is used to specify a comma-separated list 
of outputs corresponding to typical \texttt{.PRINT} line outputs.  Solution 
variables, user defined parameters, and expressions depending on both are typical.  

The other type of output is specified using the keyword \texttt{MEASURES}, and 
this is used to specify a comma-separated list of netlist \texttt{.MEASURE} commands.
This type of output is only available to \texttt{.SAMPLING} and not to \texttt{.EMBEDDEDSAMPLING}.

The example sets the \texttt{SAMPLE\_TYPE=LHS}, for Latin Hypercube Sampling~\cite{HELTON200323}.  
The other options is \texttt{SAMPLE\_TYPE=MC} for Monte Carlo~\cite{Fishman1996}.  
LHS is the default option for this parameter, and will generally be a better choice.

\subsection{Specifying Uncertain Inputs}

\subsubsection{Sampling a Device Instance Parameter}
\label{sampling_InstanceParam}
Figure~\ref{Sampling_Netlist_1} provides a simple application of \verb|.SAMPLING| 
to a device instance parameter.  The circuit is a simple voltage divider driven 
by a oscillating voltage source.  The randomly sampled parameter is the resistance 
of \texttt{R1}, which is given a normal distribution with a specified mean and 
standard deviation.

As the circuit is executed multiple times, an output file is generated based on 
the \verb|.PRINT| command.  The \verb|.PRINT| statement is still used in the same 
way that it would be in a non-sampling analysis.  However, the output file 
contains the concatenated output of each \verb|.SAMPLING| step.  
Section~\ref{sampling_output_files} provides more details of how \texttt{.SAMPLING} 
changes \texttt{.PRINT} output files.

\Xyce{} computes statistical outputs for quantities specified by \texttt{OUTPUTS} 
and \texttt{MEASURES} on the \texttt{.options SAMPLES} line.  For these outputs,
moments such as mean and variance are computed and sent to standard output.  More 
details of this type of output, as well as examples, are given in section~\ref{statistical_outputs}.
\begin{figure}[htbp]
  \fontsize{9pt}{10pt}
\begin{centering}
\shadowbox{
\begin{minipage}{0.8\textwidth}
\begin{vquote}
\color{blue}Voltage Divider with sampling\color{black}
R2  1  0  7k
R1 1 2 3k
VS1  2  0  SIN(0 1.0e3 1KHZ 0 0)

.tran 0.1ms 1ms
.meas tran maxSine max V(1) 

\color{XyceRed}.SAMPLING param=R1:R 
+ type=normal means=1K std\_deviations=0.1K\color{black}

\color{XyceRed}.options SAMPLES numsamples=1000 OUTPUTS=\{R1:R\} 
+ MEASURES=maxSine SAMPLE\_TYPE=MC stdoutput=true\color{black} 

\end{vquote}
\end{minipage}
}
\caption{Voltage divider circuit netlist with sampling of a device instance parameter.
Sampling commands are in \color{XyceRed}red\color{black}.
\label{Sampling_Netlist_1}}
\end{centering}
\end{figure}

\subsubsection{Sampling a Device Model Parameter}
\label{sampling_ModelParam}

Sampling a model parameter can be done in an identical manner to an instance parameter.  
Figure~\ref{Sampling_Netlist_2} provides a simple application of \verb|.SAMPLING| to a 
device model and a device instance.  This is the same diode clipper circuit as was used 
in the Transient Analysis section (see section \ref{Transient_Analysis_Sec}), except that 
several lines of sampling commands (in red) have been added.  

The sampling commands will cause \Xyce{} to sample the model parameter, \verb|D1N3940:IS| as well as the 
resistance of the resistor, R4. Both of the parameters are randomly sampled using uniform distributions 
with specified lower and upper bounds.
\begin{figure}[htbp]
  \fontsize{9pt}{10pt}
\begin{centering}
\shadowbox{
\begin{minipage}{0.8\textwidth}
\begin{vquote}
\color{blue}Transient Diode Clipper with Sampling Analysis\color{black}

\color{blue}* Primary Analysis Command\color{black}
.tran 2ns 0.5ms
\color{blue}* Output\color{black}
.print tran format=tecplot V(3) V(2) V(4)
.meas tran maxSine max V(4) 

\color{XyceRed}* Sampling statements 
.SAMPLING
+ param=R4:R,D1N3940:IS
+ type=uniform,uniform
+ lower\_bounds=3.0k,2.0e-10
+ upper\_bounds=15.0k,6.0e-10

.options SAMPLES numsamples=10 SAMPLE\_TYPE=LHS
+ OUTPUTS={R4:R},{D1N3940:IS} MEASURES=maxSine stdoutput=true\color{black}
\color{blue}* Voltage Sources\color{black}
VCC 1 0 5V
VIN 3 0 SIN(0V 10V 1kHz)
\color{blue}* Diodes\color{black}
D1 2 1 D1N3940
D2 0 2 D1N3940
\color{blue}* Resistors\color{black}
R1 2 3 1K
R2 1 2 3.3K
R3 2 0 3.3K
R4 4 0 5.6K
\color{blue}* Capacitor\color{black}
C1 2 4 0.47u
.MODEL D1N3940 D(
+ IS=4E-10 RS=.105 N=1.48 TT=8E-7
+ CJO=1.95E-11 VJ=.4 M=.38 EG=1.36
+ XTI=-8 KF=0 AF=1 FC=.9 BV=600 IBV=1E-4)
.END
\end{vquote}
\end{minipage}
}
\caption{Diode clipper circuit netlist with sampling analysis.  This example uses a model parameter as one of the sampling parameters.
Sampling commands are in \color{XyceRed}red\color{black}.
  \label{Sampling_Netlist_2}}
\end{centering}
\end{figure}

\subsubsection{Sampling a User-Defined Parameter}
\label{sampling_GlobalParam}

Sampling can be applied to user-defined parameters, as long as they are declared in the top level netlist (global scope).  
It cannot be applied to user-defined parameters which are defined in a subcircuit.  An example of such usage, with a global parameter, is given 
in figure~\ref{Sampling_Netlist_3}. This circuit identical to the circuit in 
figure~\ref{Sampling_Netlist_1}, except that it uses global parameters.  
If the global parameters in this netlist are changed to normal parameters (\texttt{.PARAM}), this circuit will also work.
\begin{figure}[htbp]
  \fontsize{9pt}{10pt}
\begin{centering}
\shadowbox{
\begin{minipage}{0.8\textwidth}
\begin{vquote}
\color{blue}Voltage Divider, sampling global params\color{black}
.global\_param Vmax=1000.0
.global\_param testNorm=1.5k
.global\_param R1value=\{testNorm*2.0\}

R2  1  0  7k
R1 1 2 \{R1value\}
VS1  2  0  SIN(0 \{Vmax\} 1KHZ 0 0)
.tran 0.1ms 1ms
.meas tran maxSine max V(1) 

\color{XyceRed}.SAMPLING param=testNorm
+ type=normal means=0.5K std\_deviations=0.05K\color{black}

\color{XyceRed}.options SAMPLES numsamples=1000 SAMPLE\_TYPE=MC
+ OUTPUTS=\{R1:R\} MEASURES=maxSine\color{black}

\end{vquote}
\end{minipage}
}
\caption{Voltage divider circuit netlist with sampling of a global parameter.
  Sampling commands are in \color{XyceRed}red\color{black}.  This circuit will also work if the \texttt{.global\_param} statements are changed to \texttt{.param}.
\label{Sampling_Netlist_3}}
\end{centering}
\end{figure}

\subsubsection{Sampling over Temperature}
\label{sampling_Temperature}

It is also possible to sample temperature.  To do so, simply 
specify \verb|temp| as the parameter name.  It will work in the 
same manner as \verb|.SAMPLING| when applied to model and instance 
parameters.

\subsubsection{Special cases: Sampling Independent Sources, Resistors, Capacitors and Inductors}
\label{sampling_SpecialCases}

For some devices, there is generally only one parameter that one would
want to sample.  For example, a linear resistor's only parameter of
interest is resistance, R.  Similarly, for a DC voltage or current
source, one is usually only interested in the magnitude of the source, DCV0.
Finally, linear capacitors generally only have capacitance, C, as a 
parameter of interest, while inductors generally only have inductance, L,
as a parameter of interest.  

For these simple devices, it is not necessary to specify both the
parameter and device on the \texttt{.SAMPLING} line: only the device name
is strictly required, as these devices have default
parameters that are assumed if no parameter name is given explicitly.

Examples of usage are given below.  The first two lines are equivalent
--- in the first line, the resistance parameter of \texttt{R4} is
named explicitly, and in the second line the resistance parameter is
implicit. A similar example is then shown for the DC voltage of the 
voltage source \texttt{VCC}.  In the remaining lines, parameter names are all 
implicit, and the default parameters of the associated devices are used.

\Example{ \\
          \fontsize{10pt}{11pt}
\begin{tabular}{lllll}
  \texttt{.SAMPLING param=R4:R,C1:C} &\texttt{type=normal,normal}  &\texttt{means=3k,1u}  &\texttt{std\_deviations=1k,0.1u} \\ 
  \texttt{.SAMPLING param=R4,C1} &\texttt{type=normal,normal}  &\texttt{means=3k,1u}  &\texttt{std\_deviations=1k,0.1u} \\ 
  \texttt{.SAMPLING VCC:DCV0}&\texttt{type=uniform}  &\texttt{means=6.0}  &\texttt{std\_deviations=1.0 } \\  
  \texttt{.SAMPLING VCC}&\texttt{type=uniform}  &\texttt{means=6.0}  &\texttt{std\_deviations=1.0 } \\ 
  \texttt{.SAMPLING param=C1} &\texttt{type=normal}  &\texttt{means=0.5u}  &\texttt{std\_deviations=0.05u} \\ 
  \texttt{.SAMPLING param=L1} &\texttt{type=normal}  &\texttt{means=0.5m}  &\texttt{std\_deviations=0.01m} \\ 
\end{tabular}
        }

Independent sources require further explanation.  Their default
parameter, DCV0, only applies to \texttt{.DC} analyses.  They do not have
default parameters for their transient forms, such as \texttt{SIN}
or \texttt{PULSE}.

\subsubsection{Sampling using random operators from expressions}

The \Xyce{} expression library supports random operators in expressions, 
including \texttt{AGAUSS}, \texttt{GAUSS}, \texttt{AUNIF}, \texttt{UNIF} and \texttt{RAND}.
\Xyce{} \texttt{.SAMPLING} and \texttt{.EMBEDDEDSAMPLING} analyses are able to use these, but they are not
connected by default.  In order to use random expression operators with \texttt{.SAMPLING}, 
the following commend must be present in the netlist:

\Example{\\
\texttt{\\
.SAMPLING  useExpr=true
}
}
This capability is potentially very powerful, as the random operators can 
act as operators within complex expressions.  
As expressions can be applied to \texttt{.param}, \texttt{.global\_param}, 
device instance parameters  and device model parameters, this capability 
naturally can be applied to any of these types of input parameters.
A simple example of this feature, applied to several instance parameters, 
can be seen in figure~\ref{Sampling_Netlist_4}.  Other examples of this capability can be found 
in the netlists given in figures~\ref{regressionPCE_Netlist1},~\ref{NISP_netlist1} 
and~\ref{NISP_gilbert_netlist}.

This feature cannot be used in the same netlist as the 
specification described in section~\ref{sampling_statement}.  If 
\texttt{useExpr=true}, \Xyce{} will ignore the comma-separated lists such as 
\texttt{param}.  Likewise, if \texttt{useExpr=false} (the default), then 
\Xyce{} will not randomly sample any random operators present in the circuit.  
It will only sample the parameters listed by comma-separated lists.

\begin{figure}[htbp]
  \fontsize{9pt}{10pt}
\begin{centering}
\shadowbox{
\begin{minipage}{0.8\textwidth}
\begin{vquote}
\color{blue}sampling example with random expression operators \color{black}
R4 b 0 \{agauss(4k,0.4k,1)\}
R3 a b \{agauss(1k,0.1k,1)\}
R2 1 a \{agauss(2k,0.2k,1)\}
R1 1 2 \{agauss(3k,0.1k,1)\}
v1 2 0 10V

.dc v1 10 10 1
.print dc v(1)

.SAMPLING 
+ useExpr=true

.options SAMPLES numsamples=10
+ outputs=\{v(1)\}
+ sample\_type=lhs
+ stdoutput=true
.end

\end{vquote}
\end{minipage}
}
\caption{Voltage divider circuit netlist using random expression operators.
\label{Sampling_Netlist_4}}
\end{centering}
\end{figure}


\subsubsection{Local and global variation}

Similar to other simulators, \Xyce{} uses the following convention for 
local and global variations sampling and other uncertainty quantification 
methods.  This is based on how many layers of \texttt{.param} indirection 
sit between a device parameter and the random operator.  If there is only a single 
layer, then the random operator is treated as a local variation, and every
affected device parameter will receive a unique random number at each sample point.
If there is more than one layer, then it is instead treated as global, meaning 
that each affected device parameter will receive the same random number from the operator.

For example, the following netlist fragment will result in local variation, 
where the resistors $r1$, $r2$ and $r3$ each get unique random numbers.
\Example{\\
\texttt{\\
.param resval=aunif(1000,400)  \\
r1 1 2 resval \\
r2 2 3 resval \\
r3 3 4 resval 
}
}
In contrast, the following netlist fragment will result in global variation, 
where the resistors will all get the same random number.
\Example{\\
\texttt{\\
.param globalval=aunif(1000,400)  \\
.param resval=globalval \\
r1 1 2 resval \\
r2 2 3 resval \\
r3 3 4 resval 
}
}

\subsection{Output files}
\label{sampling_output_files}

The two types of sampling produce different types of output files.  This is 
related to the fact that for \texttt{.SAMPLING} the calculations for each sample 
point are performed sequentially, while for \texttt{.EMBEDDEDSAMPLING} the 
calculations for each sample point are performed simultaneously.  As such, 
for \texttt{.SAMPLING} statistical moments can only be computed at the end 
of the calculation, once all the sample points are avaible.  In contrast, 
for \texttt{.EMBEDDEDSAMPLING}, statistical moments can be computed throughout 
the calculation, as all the sample points are available for each step.

\subsubsection{\texttt{.SAMPLING} output files}
\index{output!\texttt{.SAMPLING}}
Users can think of \texttt{.SAMPLING} simulations as being very similar 
to \texttt{.STEP} and thus comprised of a sequential set of distinct executions
of the same circuit netlist. The main difference is the source of the specific sample points.

Similar to \texttt{.STEP}, the output data can be requested by a standard \texttt{.PRINT} command, for the underlying primary analsis.
For example, if applying \texttt{.SAMPLING} to a transient analysis, then the appropriate \texttt{.PRINT} command will be \texttt{.PRINT TRAN}.
The output data will go to a single file, with each sample point in a different section of the output file.
If using the default \texttt{.prn} file format (\texttt{format=std}), the 
resulting \texttt{.SAMPLING} simulation output file will be a simple concatenation of
each step's underlying analysis output.
If using \texttt{format=probe}, the data for each execution of the circuit
will be in distinct sections of the file, and it should be easy to 
plot the results using PROBE.  If using \texttt{format=tecplot}, 
the results of each \texttt{.SAMPLING} simulation will be in a distinct
Tecplot zone. Finally, format=raw will place the results for each \texttt{.SAMPLING} 
simulation in a distinct ``plot'' region. 

Another similarity to \texttt{.STEP} is that \Xyce{} creates a second auxilliary file with the \texttt{*.res} 
suffix, which contains the parameter values.    This can be useful for seeing the exact sampling parameter values in a compact file.

Statistical outputs can also be produced (see section~\ref{statistical_outputs}) 
for \texttt{.SAMPLING} in the form of moments, but these can only be computed 
at the end of the calculation once all the samples are available.  These statistical outputs will not
be present in the \texttt{.PRINT} specified output file, since it is output on-the-fly.

Figure~\ref{Sampling_Netlist_2} shows an example file named \verb+clip.cir+, which when run will produce files
\verb+clip.cir.res+ and \verb+clip.cir.prn+.  The file \verb+clip.cir.res+
contains one line for each step, showing what parameter value was used
on that step.  \verb+clip.cir.prn+ is the familiar output format, but
the \verb+INDEX+ field recycles to zero each time a new step begins.
As the default behavior distinguishes each step's output only by recycling 
the \verb+INDEX+ field to zero, it can be beneficial to add the sampling
parameters to the \verb+.PRINT+ line.   For the default file format 
(\texttt{format=std}), \Xyce{} will not automatically include these sampling parameters,
so for plotting it is usually best to specify them by hand.


Note that for sampling calculations involving a really large number of sample 
points, the single output file can become prohibitively large.  Be careful when 
using \verb|.PRINT| if the number of samples is large.   If the number is really 
large (thousands) consider excluding any \verb|.PRINT| output statements and 
just rely on statistical outputs, described next in section~\ref{statistical_outputs}.
Similarly, the generation of the measure output can be suppressed with 
\texttt{.OPTIONS MEASURE MEASPRINT}.

\subsubsection{\texttt{.EMBEDDEDSAMPLING} output files}
\index{output!\texttt{.EMBEDDEDSAMPLING}}

As \texttt{.EMBEDDEDSAMPLING} computes all the sample points simultaneously, outputs for all 
sample points are also available simulataneously.  For transient simulations, this means that 
all samples are available at every time step, and all statistical outputs are available 
at every time step.  This is very useful if the interest is in seeing statistical moments 
with respect to time   

The output command used to produce this kind of output is \texttt{.PRINT ES}, where \texttt{ES} 
stands for  ``embedded sampling''.
The \texttt{.PRINT ES} command will automatically output several moments of the 
specified PCE outputs without any additional arguments on the \texttt{.PRINT ES} line.  
Also, the \texttt{.PRINT ES} statement will result in a file that has ``ES'' as part of the file name suffix.
Otherwise, \texttt{.PRINT ES} behaves very similarly to any other type of \texttt{.PRINT} 
command, and supports all of the same file formats and options.  For a full description see the
\Xyce{} Reference Guide.

\subsection{Statistical Outputs}
\label{statistical_outputs}
When performing uncertainty quantification analysis, the outputs of interest are 
often statistical moments of different circuit metrics.   \Xyce{} will compute these at the
end of a sampling calculation upon request.  The requested statistical outputs are 
specified on the \texttt{.options SAMPLING} line in the netlist.  For an example 
see the \texttt{OUTPUTS} and \texttt{MEASURES} fields in section~\ref{options_samples_statement}.  
Both \texttt{OUTPUTS} and \texttt{MEASURES} are specified as comma-separated 
lists.  The list of \texttt{OUTPUTS} must contain valid expressions that can be 
processed by the \Xyce{} expression library.  The list of \texttt{MEASURES} must 
consist valid \texttt{.MEASURE} statement names, which are present elsewhere in 
the netlist.

The example netlist given by figure~\ref{Sampling_Netlist_1} will produce the result 
given by figure~\ref{Sampling_Netlist_1_output}.  The example netlist given 
by figure~\ref{Sampling_Netlist_2} will produce the result given by 
figure~\ref{Sampling_Netlist_2_output}.  In both examples, the number of samples 
is relatively small, so the exact quantities computed will vary somewhat from run to run.
\begin{figure}[htbp]
  \fontsize{9pt}{10pt}
\begin{centering}
\shadowbox{
\begin{minipage}{0.8\textwidth}
\begin{vquote}
MC sampling mean of {R1:R} = 1009.58
MC sampling stddev of {R1:R} = 100.131
MC sampling variance of {R1:R} = 10026.3
MC sampling skew of {R1:R} = -0.0208347
MC sampling kurtosis of {R1:R} = 2.79565
MC sampling max of {R1:R} = 1308.03
MC sampling min of {R1:R} = 656.039

MC sampling mean of maxSine = 874.063
MC sampling stddev of maxSine = 10.9115
MC sampling variance of maxSine = 119.061
MC sampling skew of maxSine = 0.08628
MC sampling kurtosis of maxSine = 2.83029
MC sampling max of maxSine = 914.274
MC sampling min of maxSine = 842.547
\end{vquote}
\end{minipage}
}
\caption{Typical statistical outputs for figure~\ref{Sampling_Netlist_1}}
\label{Sampling_Netlist_1_output}
\end{centering}
\end{figure}

\begin{figure}[htbp]
  \fontsize{9pt}{10pt}
\begin{centering}
\shadowbox{
\begin{minipage}{0.8\textwidth}
\begin{vquote}
LHS sampling mean of {R4:R} = 9198
LHS sampling stddev of {R4:R} = 4293.89
LHS sampling variance of {R4:R} = 1.84375e+07
LHS sampling skew of {R4:R} = -0.0939244
LHS sampling kurtosis of {R4:R} = 1.39595
LHS sampling max of {R4:R} = 14996
LHS sampling min of {R4:R} = 3315.6

LHS sampling mean of {D1N3940:IS} = 4.2571e-10
LHS sampling stddev of {D1N3940:IS} = 9.10744e-11
LHS sampling variance of {D1N3940:IS} = 8.29455e-21
LHS sampling skew of {D1N3940:IS} = 0.0486504
LHS sampling kurtosis of {D1N3940:IS} = 2.05108
LHS sampling max of {D1N3940:IS} = 5.78543e-10
LHS sampling min of {D1N3940:IS} = 2.85753e-10

LHS sampling mean of maxSine = 4.46494
LHS sampling stddev of maxSine = 0.0901329
LHS sampling variance of maxSine = 0.00812394
LHS sampling skew of maxSine = -0.682188
LHS sampling kurtosis of maxSine = 2.07339
LHS sampling max of maxSine = 4.5536
LHS sampling min of maxSine = 4.28396
\end{vquote}
\end{minipage}
}
\caption{Typical statistical outputs for figure~\ref{Sampling_Netlist_2}}
\label{Sampling_Netlist_2_output}
\end{centering}
\end{figure}

\subsection{Random Sampling and Running in Parallel}

It is well known that sampling is a good opportunity to apply parallelism, as zero communication is required between sample points.
However, it has only been applied to a limited degree to random sampling in \Xyce{}.  
This is mostly because sampling methods 
are a relatively new feature to \Xyce{}. To get the full benefit of parallel sampling, where each 
sample can be launched and executed on a different unique processor, it is
recommended that \Xyce{} users instead use the Dakota library~\cite{DakotaTheoMan},~\cite{DakotaUsersMan}.
However, despite the limitations of \Xyce{} sampling in parallel, note that:
\begin{XyceItemize}
\item Sampling in \Xyce{} does have \emph{some} available parallelism, just not applied on a sample basis.
\item The cost of sampling can be significantly mitigated using non-intrusive Polynomial Chaos 
  Expansion (PCE) methods, described in section~\ref{PCE_Analysis}
\item \Xyce{} sampling is performed entirely internal to the \Xyce{} binary.  So unlike the black-box form of Dakota, it does not involve any file I/O. Also it does not require any scripting, which can be error-prone.    Finally, the netlist parsing and setup only happens once, rather than for every sample point, and this mitigates some of the computational cost.
\end{XyceItemize}

\subsubsection{\texttt{.SAMPLING} in Parallel}
The \texttt{.SAMPLING} analysis will execute the underlying analysis sequentially, once 
for each sample point.    If using parallel \Xyce{}, then each of these sequential 
calculations will have the same parallel methods applied to them as would be 
applied to a single forward calculation.    For example, if running a calculation
of 100 samples on a large parallel transient calculation, \Xyce{} would perform 
the first transient simulation in parallel, followed by the second transient 
simulation in parallel, etc, until all 100 samples were complete.
This means that if the underlying analysis is 
large enough to get a benefit from parallel, then the \texttt{.SAMPLING} 
analysis will get this same benefit.  The parallel scaling for the underlying 
analysis is unlikely to be perfect, as there are communication costs involved 
in the forward calculation.  However, one of the main parallel bottlenecks, 
the parsing and setup phase, is substantially reduced as it only happens one time.

\subsubsection{\texttt{.EMBEDDEDSAMPLING} in Parallel}
The \texttt{.EMBEDDEDSAMPLING} algorithm sets up a large block matrix system, 
where each block represents a different sample point.  This means that for each 
step of the calcalculation, all the samples are computed simultaneously.  This block system 
can be solved in parallel.   However, as this method is  less mature in \Xyce{}, 
more refinements to the algorithm are needed for this
\texttt{.EMBEDDEDSAMPLING} to fully benefit from running in parallel.

\clearpage



% Sandia National Laboratories is a multimission laboratory managed and
% operated by National Technology & Engineering Solutions of Sandia, LLC, a
% wholly owned subsidiary of Honeywell International Inc., for the U.S.
% Department of Energy’s National Nuclear Security Administration under
% contract DE-NA0003525.

% Copyright 2002-2023 National Technology & Engineering Solutions of Sandia,
% LLC (NTESS).

% -------------------------------------------------------------------------
% Sampling Analysis Section ---------------------------------------------------
% -------------------------------------------------------------------------

\clearpage
\section{Polynomial Chaos Expansion (PCE) methods}
\label{PCE_Analysis}
\label{pce_Overview}
\index{analysis!PCE} \index{PCE analysis} 
\index{\texttt{.SAMPLING}}

\Xyce{} supports several styles of Polynomial Chaos expansion (PCE) methods, which are 
stochastic expansion methods that approximate the functional dependence
of a simulation response on uncertain model parameters by expansion
in a polynomial basis~\cite{XiuKarn02,Xiu2010}.   
In \Xyce{} the Stokhos library~\cite{stokhos} has been used to implement 
generalized PCE methods using the Wiener-Askey scheme\cite{XiuKarn02}.   

To propagate input uncertainty through a model using PCE, \Xyce{}
performs the following steps: (1) input uncertainties are transformed
to a set of uncorrelated random variables, (2) a basis such
as Hermite polynomials is selected, and (3) the parameters of the
functional approximation are determined.  The general PCE 
for a response $O$ has the form:
\begin{equation}
  O(p) \approx \hat{O}(p) = \sum_{i=0}^{P} \alpha^i \varPsi^i(p),
  \label{eq:genPCEresponse}
\end{equation}
where each multivariate basis polynomial $\varPsi^i(p)$ involves products of univariate
polynomials that are tailored to the individual random variables. 
More details about these methods and their \Xyce{} implementation can be found in
reference~\cite{xyceAdvancedUQ} and the references therein.

\Xyce{} supports three versions of polynomial chaos.  They mainly differ in how the coefficients $\alpha$ in equation~\ref{eq:genPCEresponse} are computed.  These three methods are:
\begin{itemize}
  \item Non-intrusive regression polynomial chaos (section~\ref{regressionPCE})
  \item Non-intrusive spectral projection  (NISP) (section~\ref{nisp})
  \item Fully intrusive spectral projection (section~\ref{intrusivePCE})
\end{itemize}
The first two versions are non-intrusive, meaning that they don't directly modify the forward problem being solved.  
Instead, they are augmentations to the \texttt{.SAMPLING} and \texttt{.EMBEDDEDSAMPLING} analysis methods.
The fully intrusive method is a separate analysis type, and is invoked using the \texttt{.PCE} command.
\clearpage
\subsection{Regression-based Polynomial Chaos}
\label{regressionPCE}

Regression-based PCE is briefly described in this section.
Regression-based PCE approaches aim to find the best fit for the linear system:
\begin{equation}
\boldsymbol{\varPsi} \boldsymbol{\alpha} \approx \boldsymbol{R} \label{eq:regression}
\end{equation}
for a set of PCE coefficients $\boldsymbol{\alpha}$ that best
reproduce a set of response values $\boldsymbol{R}$.   The orthogonal polynomials used in the  
PCE approximation are represented by the matrix $\varPsi$.  This matrix is usually a rectangular
matrix.

The regression approach finds a set of PCE coefficients $\alpha^i$ which best fit a set of response
values obtained from a sampling study on the density function of the uncertain 
parameters~\cite{pt_colloc1}.  One advantage of regression-based methods is that
they are very flexible with respect to the exact sample points used to evaluate the responses, 
and can readily be used with standard sampling methods such as Monte Carlo and Latin Hypercube Sampling.
The method that has been implemented in \Xyce{} for solving Eq.
\eqref{eq:regression} is least squares regression.

Regression-based PCE can be applied as a post-process step 
to \texttt{.SAMPLING} and \texttt{.EMBEDDEDSAMPLING}.  It is invoked by adding 
\texttt{regression\_pce=true} to the \texttt{.options SAMPLES} 
or \texttt{.options EMBEDDEDSAMPLES} command lines.  

Regression PCE will use the samples selected by the sampling algorithm (Monte Carlo or Latin Hypercube).  
The regression PCE algorithm will \emph{not} estimate the optimal number of samples to use, and so this must 
be set by the user.   For regression PCE, the number of samples needed for an accurate answer
can be smaller than the number that would have been required by conventional sampling.
For regression problems, to get an accurate answer it is considered good practice to oversample 
the basis size by at least a factor of 2. 

An example netlist, which uses regression PCE with \texttt{.SAMPLING} is given in figure~\ref{regressionPCE_Netlist1}.  
Results from this calculation, which are sent to terminal output are given in figure~\ref{regressionPCE_Result1}.
In this calculation, the polynomial order has been set to \texttt{5}, the output which is subject to the PCE analysis 
is \texttt{v(1)}, the type of sampling used is Latin Hypercube Sampling (LHS), and the results of the analysis are 
sent to terminal output.    For this example, the uncertain parameter is \texttt{testNorm}, which is
set by the \texttt{.param testNorm=\{aunif(2k,1k)\}} statement.  On the \texttt{.SAMPLING} line, 
the command \texttt{useExpr=true} instructs the sampling method to rely on expression-based 
random expressions.  Without this line, the sampling algorithm would ignore \texttt{testNorm}.

The parameter \texttt{resample=true} is a PCE-specific command.   Once the PCE approximation 
has been completed, the resulting polynomial can be thought of as a surrogate model, and this
surrogate can then be sampled, much like the original forward problem.  When the surrogate is 
sampled, it uses 1000 samples.    Since this calculation is sampling a polynomial, it can do 
a really large number of evaluations without incurring much computational expense.
The result of this resampling is (in the example) also sent to the console, as can be 
seen at the bottom of figure~\ref{regressionPCE_Result1}.

\begin{figure}[htbp]
\fontsize{9pt}{10pt}
\begin{centering}
\shadowbox{
\begin{minipage}{0.8\textwidth}
\begin{vquote}
\color{blue}Regression PCE example \color{black}
.param testNorm=\{\color{red}aunif(2k,1k)\color{black}\}
.param R1value=\{testNorm*2.0\}
R2 1 0 6K
R1 1 2 \{R1value\}
v1 2 0 1000V

.dc v1 1000 1000 1
.print dc format=tecplot v(1)
*
\color{red}.SAMPLING useExpr=true

.options SAMPLES numsamples=12
+ regression\_pce=true
+ order=5
+ outputs=\{v(1)\}
+ sample\_type=lhs
+ stdoutput=true
+ resample=true\color{black}
.end
\end{vquote}
\end{minipage}
}
\caption{Voltage divider regression PCE netlist.  The sampling-related commands are in \color{red}red\color{black}.
\label{regressionPCE_Netlist1}}
\end{centering}
\end{figure}

\begin{figure}[htbp]
\fontsize{9pt}{10pt}
\begin{centering}
\shadowbox{
\begin{minipage}{0.8\textwidth}
\begin{vquote}
***** Beginning Latin Hypercube Sampling simulation....

***** Regression PCE enabled.  Number of sample points = 12

***** PCE Basis size = 6

Seeding random number generator with 2412823737

LHS sampling mean of {V(1)} = 614.666
LHS sampling stddev of {V(1)} = 71.4563
LHS sampling variance of {V(1)} = 5106
LHS sampling skew of {V(1)} = 0.299928
LHS sampling kurtosis of {V(1)} = 2.07869
LHS sampling max of {V(1)} = 733.936
LHS sampling min of {V(1)} = 503.315

(traditional sampling) regression PCE mean of {V(1)} = 608.197
(traditional sampling) regression PCE stddev of {V(1)} = 71.3831

Statistics from re-sampling regression PCE approximation of {V(1)}:
mean     = 608.071
stddev   = 70.8532
variance = 5020.18
skew     = 0.298245
kurtosis = 1.93151
max      = 749.652
min      = 500.054
\end{vquote}
\end{minipage}
}
\caption{Voltage divider regression PCE result.  There are 3 sets of statistics in this ouptut.  
  The first set is computed from the 12 LHS sample points.  As this is a PCE calculation, the number of points is small and so the statistics based on them are probably not very accurate.
  The second set is computed from the PCE approximation, which are purely analytical.
  The third set is from running 1000 samples on the PCE approximation.  This output is enabled by \texttt{resample=true}.
\label{regressionPCE_Result1}}
\end{centering}
\end{figure}

\clearpage
\subsection{Non-Intrusive Spectral Projection (NISP)}
\label{nisp}
The spectral projection PCE approach projects the response
against each basis function $\Psi_j(p)$ using inner
products and employs the polynomial orthogonality properties to
extract each coefficient. Each inner product involves a
multidimensional integral over the support range of the weighting
function, which can be evaluated numerically using sampling,
tensor-product quadrature, Smolyak sparse grid \cite{Smolyak_63}, or
cubature \cite{stroud} approaches.   For the \Xyce{} implementation 
this means evaluating the following equation for each coefficient using 
quadrature rules:
\begin{equation}
\alpha^{i} = \frac{\langle O\varPsi ^{i}\rangle} {\langle (\varPsi ^{i})^{2}\rangle} = \frac{1} {\langle (\varPsi^{i})^{2}\rangle}\int _{\varGamma}O(p ;y)\varPsi ^{i}(y)\rho (y)dy = 0,\quad i = 0,\ldots,P.
  \label{nispProjection}
\end{equation}
In this equation $\rho$ is the joint density function, with respect to which the polynomials $\varPsi^i$ are orthogonal.
Evaluating this equation requires solving the integral on the right hand side.  
While many methods for solving this integral could be used, for the NISP algorithm 
the integral is solved using Gaussian Quadrature and related methods.  For this 
reason, these methods are sometimes referred to as ``quadrature PCE'' methods.
Stokhos supports several quadrature methods, and supplies the necessary quadrature points 
to \Xyce{} for the specified details.  

To solve the integral using quadrature, \Xyce{} has to evaluate the forward problem at 
the various quadrature points. In this reguard, this is similar to the
regression methods described in section~\ref{regressionPCE}, which has to do 
forward evaluations at randomly sampled points.
Since NISP uses quadrature points, the points are completely deterministic, 
and there are no random numbers involved.
The number of quadrature points can increase exponentially for larger numbers of uncertain 
parameters, but this can be mitigated by using sparse grid techniques.  

Non-intrusive spectral projection PCE can be applied as a post-process step 
to \texttt{.SAMPLING} and \texttt{.EMBEDDEDSAMPLING}.  It is invoked by adding 
\texttt{projection\_pce=true} to the \texttt{.options SAMPLES} 
or \texttt{.options EMBEDDEDSAMPLES} command lines.

As noted, when using NISP it is not necessary to specify the number of sample points, as they will
be set by the quadrature algorithm.  To enable quadrature on a sparse grid, set the 
option \texttt{sparse\_grid=true} on the \texttt{.options EMBEDDEDSAMPLES} command line.

An example netlist, which uses NISP with \texttt{.EMBEDDEDSAMPLING} is given in figure~\ref{NISP_netlist1}.  
Results from this calculation, which are output using the \texttt{.PRINT ES} command 
are given in figure~\ref{NISP_result1}.  The \texttt{.PRINT ES} command 
will automatically output several moments of the specified PCE outputs (in this 
case \texttt{V(drain)} and \texttt{V(gate)}) without any additional arguments 
on the \texttt{.PRINT ES} line.  In the example, additional information is 
requested using the \texttt{OUTPUT\_PCE\_COEFFS=TRUE} command, which will result 
in all of the PCE coefficients (the various $\alpha^{i}$ given by equation~\ref{nispProjection}).

A \texttt{.SAMPLING} version of the netlist given in figure~\ref{NISP_netlist1} can also be 
constructed, by simply replacing \texttt{.EMBEDDEDSAMPLING} with \texttt{.SAMPLING} and 
replacing \texttt{.options EMBEDDEDSAMPLES} with \texttt{.SAMPLES}.    However, the behavior
of \texttt{.SAMPLING} analysis will be a bit different.  As \texttt{.SAMPLING} 
performs the calculation for each quadrature point sequentially, rather than 
simultaneously, there is no equivalent of \texttt{.PRINT ES} for 
conventional sampling.  All of the \texttt{.SAMPLING} output is handled 
by the \texttt{.PRINT TRAN} command.  

Additionally, since the PCE analysis cannot be performed until the end of a 
\texttt{.SAMPLING} analysis, it is more  useful to have the analysis applied 
to \texttt{.MEASURE} commands which produce metrics for the entire simulation.  
Applying the PCE analysis to voltage node values in this case is less useful, 
as it will  simply use the final values for each voltage node (which will be 
mostly invariant in this example).

\begin{figure}[htbp]
\fontsize{9pt}{10pt}
\begin{centering}
\shadowbox{
\begin{minipage}{0.95\textwidth}
  \begin{vquote}\color{blue}
* embedded sampling with NISP, CMOS inverter\color{black}
.param inFreq=1e6, inPer=\{1/inFreq\}
.param td=\{inPer/2-inPer/10\}
.param tr=\{inPer/10\}, tf=\{inPer/10\}
.param pw=\{inPer/2-inPer/10\}, per=\{inPer\}

VS\_VDD vdd 0 DC 1.0
Vin gate 0 pulse( 0.0 1.0 \{td\} \{tr\} \{tf\} \{pw\} \{per\} )
M1 drain gate 0 0 NMOS w=0.18e-6 l=0.18e-6
M2 drain gate vdd vdd PMOS  w =0.54e-6 

.model nmos nmos (level=9 tnom=27 nch=2.3549e17 \color{red}vth0=\{aunif(0.55,0.15)\}\color{black} 
+ uc=5.105195e-11 ags=0.4044483 keta=-2.730673e-3 rdsw=105 wr=1 
+ dwg=-3.773529e-9 dwb=5.239518e-9 cit=0 cdscb=0 dsub=0.0167866 
+ pdiblc2=3.38377e-3 pscbe1=8e10 delta=0.01 prt=0 kt1l=0 ub1=-7.61e-18 
+ wl=0 wwn=1 lln=1 lwl=0 cgdo=7.9e-10 cj=9.539798e-4 
+ cjsw=2.53972e-10 cjswg=3.3e-10 cf=0 pk2=-4.42044e-4 pu0=10.8203648 
+ pvsat=1.388017e3 tox=4.1e-9 k2=1.63871e-3 w0=1e-7 dvt1w=0 
+ dvt1=0.3683763 ua=-1.504141e-9 vsat=9.896282e4 b0=-4.706134e-8 
+ a1=5.916677e-4 prwg=0.5 wint=0 voff=-0.0883818 cdsc=2.4e-4 
+ eta0=2.470369e-3 pclm=0.7326932 pdiblcb=-0.1 pscbe2=1.254966e-9 
+ rsh=6.5 ute=-1.5 kt2=0.022 uc1=-5.6e-11 wln=1 wwl=0 lw=0 
+ capmod=2 cgso=7.9e-10 pb=0.8 pbsw=0.8 pbswg=0.8 pvth0=7.505753e-4 
+ wketa=3.100384e-3 pua=2.896652e-11 peta0=8.758549e-5 toxm=4.1e-9 
+ xj=1e-7 k1=0.6064385 k3b=2.763267 dvt0w=0 dvt0=1.3330881 u0=258.9066683 
+ nlx=1.71872e-7 dvt2w=0 dvt2=0.0540199 ub=2.428646e-18 a0=1.8904342 
+ k3=1e-3 b1=1.294942e-6 a2=0.9069159 prwb=-0.2 lint=1.69494e-8 
+ nfactor=2.1821266 cdscd=0 etab=1.047744e-5 pdiblc1=0.1823102 
+ drout=0.7469045 pvag=0 mobmod=1 kt1=-0.11 ua1=4.31e-9 at=3.3e4 
+ ww=0 ll=0 lwn=1 xpart=0.5 cgbo=1e-12 mj=0.380768 mjsw=0.1061193 
+ mjswg=0.1061193 prdsw=-2.7650517 lketa=-0.0104103 
+ pub=1.684125e-23 pketa=1.549791e-3)

.model pmos pmos (level=9 tnom=27 nch=4.1589e17 \color{red}vth0=\{-aunif(0.55,0.15)\}\color{black} 
+ k2=0.0258663 w0=1e-6 dvt1w=0 dvt0=0.6117215 u0=106.5280265 uc=-1e-10 
+ ags=0.3667554 keta=0.0237092 rdsw=304.9893888 wr=1 dwg=-2.44019e-8 
+ dwb=-9.06003e-10 cit=0 cdscb=0 dsub=1.0998181 pdiblc2=0.0420477 
+ pscbe1=1.073111e10 delta=0.01 prt=0 kt1l=0 ub1=-7.61e-18 
+ wl=0 wwn=1 lln=1 lwl=0 cgdo=6.41e-10 cj=1.200422e-3 
+ cjsw=2.001802e-10 cjswg=4.22e-10 cf=0 pk2=1.799383e-3 pu0=-1.3399122 
+ pvsat=-50 l=0.18e-6 tox=4.1e-9 k3=0 nlx=1.20187e-7 
+ dvt2w=0 dvt1=0.2286816 ua=1.125454e-9 vsat=1.593712e5 b0=5.263128e-7 
+ a1=0.2276342 prwg=0.5 wint=0 voff=-0.0878287 cdsc=2.4e-4 eta0=0.1672562 
+ pclm=2.2249148 pdiblcb=-1e-3 pscbe2=3.099395e-9 rsh=7.4 ute=-1.5 
+ kt2=0.022 uc1=-5.6e-11 wln=1 wwl=0 lw=0 capmod=2 cgso=6.41e-10 
+ pb=0.8478616 pbsw=0.8483594 pbswg=0.8483594 pvth0=2.098588e-3 
+ wketa=0.0295614 pua=-5.27759e-11 peta0=1.003159e-4 toxm=4.1e-9 
+ xj=1e-7 k1=0.59111 k3b=7.9143108 dvt0w=0 dvt2=0.1 ub=1e-21 a0=1.6904754 
+ b1=1.496707e-6 a2=0.6915706 prwb=0.2553725 lint=3.217673e-8 
+ nfactor=1.8560303 cdscd=0 etab=-0.1249603 pdiblc1=8.275696e-4 
+ drout=0 pvag=15 mobmod=1 kt1=-0.11 ua1=4.31e-9 at=3.3e4 ww=0 ll=0 
+ lwn=1 xpart=0.5 cgbo=1e-12 mj=0.4105254 mjsw=0.3400571 
+ mjswg=0.3400571 prdsw=4.4771801 lketa=-1.935751e-3 
+ pub=1e-21 pketa=-3.434535e-3)

\color{red}.EMBEDDEDSAMPLING useExpr=true
.options EMBEDDEDSAMPLES outputs=\{v(drain)\},\{V(gate)\} projection\_pce=true
.PRINT ES OUTPUT\_PCE\_COEFFS=TRUE FORMAT=tecplot \color{black}
.tran 1ns 0.75e-6
.PRINT TRAN v(drain) V(gate)
.END
\end{vquote}
\end{minipage}
}
\caption{CMOS Inverter Non-intrusive Spectral Projection (NISP) netlist. 
 The NISP-related commands are in \color{red}red\color{black}.
\label{NISP_netlist1}}
\end{centering}
\end{figure}

\begin{figure}[hbt]
\centering
\includegraphics[width=5in]{cmos_inverter_result.pdf}
\caption
  [CMOS Inverter NISP Output] {CMOS Inverter NISP Output}
\label{NISP_result1}
\end{figure}

\clearpage
\subsection{Fully Intrusive Spectral Projection}
\label{intrusivePCE}
The fully intrusive form of PCE involves solving a system of equations which is constructed by 
applying the orthogonal polynomial expansion defined by 
Eq. \eqref{eq:genPCEresponse} to all the terms of the differential-algebraic 
system of equations solved by \Xyce{}.  The details of this construction are given in 
reference~\cite{xyceAdvancedUQ}.  This method is primarily experimental at this
point, and is inherently more expensive,
so it is not the best choice of PCE method in \Xyce{}.   Most \texttt{.PCE} use cases of
interest can be computed much more cheaply and robustly using \texttt{.EMBEDDEDSAMPLING} 
combined with one of the non-intrusive PCE methods (regression PCE or NISP).
Similar to \texttt{.EMBEDDEDSAMPLING} based calculations, \texttt{.PCE} computes
PCE coefficients at every stage of the calculation.   

An example netlist, for a diode clipper circuit which uses intrusive PCE, is given in figure~\ref{Fully_Intrusive_PCE_netlist1}.
For this type of analysis, it is not necessary to specify \texttt{projection\_pce=true}.  The result for this circuit 
is given in figure~\ref{Fully_Intrusive_PCE_result1}.  
\begin{figure}[htbp]
\fontsize{9pt}{10pt}
\begin{centering}
\shadowbox{
\begin{minipage}{0.95\textwidth}
  \begin{vquote}\color{blue}Transient Diode Clipper with intrusive PCE analysis\color{black}

.tran 2ns 2.0ms

\color{red}.PCE useExpr=true
.options PCES 
+ OUTPUTS=R4:R,D1N3940:IS outputs=\{V(4)\}
.print pce format=tecplot \color{black}

VCC 1 0 5V
VIN 3 0 SIN(0V 10V 1kHz)
D1 2 1 D1N3940
D2 0 2 D1N3940
R1 2 3 1K
R2 1 2 3.3K
R3 2 0 3.3K
R4 4 0 \color{red} \{aunif(9.0k,6.0k)\} \color{black}
C1 2 4 0.47u

.MODEL D1N3940 D(
+ \color{red} IS=\{aunif(4.0e-10,2.0e-10)\} \color{black}
+ RS=.105 N=1.48 TT=8E-7
+ CJO=1.95E-11 VJ=.4 M=.38 EG=1.36
+ XTI=-8 KF=0 AF=1 FC=.9 BV=600 IBV=1E-4)
.END 
\end{vquote}
\end{minipage}
}
\caption{Diode clipper fully intrusive PCE netlist.
 The PCE-related commands are in \color{red}red\color{black}.
\label{Fully_Intrusive_PCE_netlist1}}
\end{centering}
\end{figure}

\begin{figure}[hbt]
\centering
\includegraphics[width=5in]{clipperPce2.pdf}
\caption
  [Diode clipper fully intrusive PCE Output] {Diode clipper fully intrusive PCE Output}
\label{Fully_Intrusive_PCE_result1}
\end{figure}

\clearpage
%% -----------------------------------------------------------------------------------
\subsection{Comparing PCE methods: Gilbert Cell Mixer PCE example}
\label{xyceGilbertCell}
A Gilbert cell mixer circuit is given in figure~\ref{gilbert}~\cite{1049925}.  A corresponding netlist 
is given in Fig.~\ref{NISP_gilbert_netlist}.
In this circuit, the LO voltage is given by \texttt{v(6,2)} and \texttt{v(15,10)} is the input. 
The output is the difference between collector voltages of Q4/Q6 and Q3/Q5 (\texttt{V(5,3)}), 
and should be the approximate product of the LO and input voltages.
In this example, all eight resistors are given uncertain resistances,
distributed with the following normal distributions:  R1 and R2 are N(10,1), R5
and R6 are N(100,5), and R3, R4, R7, and R8 are N(1500,50).  These large
standard deviations of the resistances are not realistic, but are chosen for
demonstration purposes.
\begin{figure}[hbt]
\centering
\resizebox{.9\linewidth}{!}{ \input cktFigs/gilbert}
  \caption[Gilbert cell mixer schematic]
  {Gilbert cell mixer schematic.}
\label{gilbert}
\end{figure}

Ensemble and PCE results for the Gilbert cell mixer are shown in figure~\ref{gilbert_compare}.   
As this calculation was concerned with the transient behavior, only \texttt{.EMBEDDEDSAMPLING} 
and \texttt{.PCE} are used in this example.  For \texttt{.EMBEDDEDSAMPLING}, both types of 
non-intrusive PCE are used.
The light grey lines in Fig.~\ref{gilbert_compare} are for all the sample points 
of an embedded LHS study with 1000 samples.    The red lines (which are obscured 
by other results) show the mean and the mean +/- twice the standard deviation for the 
1000 sample study.  This can be considered the ``gold'' standard for this simulation.
The other solid lines show the moments from the various PCE calculations, and they 
lie on top of the 1000 LHS sample moments.  

This version of the netlistgiven in Fig.~\ref{NISP_gilbert_netlist}
is using non-intrusive projection PCE as an augmentation to \texttt{.EMBEDDEDSAMPLING}.  
To convert this netlist to 
use the fully intrusive method, change \texttt{.EMBEDDEDSAMPLING} to \texttt{.PCE}, and 
change \texttt{.options EMBEDDEDSAMPLES} to \texttt{.options PCES}.   Otherwise, the commands are completely the same.
As the number of uncertain parameters is a little bit larger (8), for the projection-based methods a 
Smolyak sparse grid~\cite{Smolyak_63} method is used.  This is set with the  \texttt{sparse\_grid=true} parameter. 
Using the sparse grid, the number of 
required quadrature points for this problem is 129.  Using the non-sparse Gaussian quadrature 
would have required 6561 quadrature points. This would have rendered the method much 
less efficient than sampling, and basically not usable, especially for the \texttt{.PCE} method.

To change the netlist to use non-intrusive regression PCE, change \texttt{projection\_pce=true} 
to \texttt{regression\_pce=true}.  Delete the \texttt{sparse\_grid=true} parameter, as 
it isn't relevant to regression PCE.  

As noted, for regression PCE the number of parameters needs to be set.  Unlike projection-based methods, 
regression PCE does not have a rigid requirement for the number of samples.  For this 
problem the size of the polynomial basis is 45, and for regression problems it 
is considered good practice to oversample by at least a factor of 2, which implies that
a good choice for this circuit is 90 sample points.   
To set this, add \texttt{numsamples=90} on the \texttt{.options EMBEDDEDSAMPLES} line.  
It appears that 90 sampling points was 
adequate for the regression form of PCE, as the all the statistical moments 
match well throughout the transient.

\begin{figure}[htbp]
\fontsize{10pt}{11pt}
\begin{centering}
\shadowbox{
\begin{minipage}{0.95\textwidth}
  \begin{vquote}\color{blue}
* Gilbert cell mixer\color{black}
R5 1 2 \color{red}\{agauss(100,5,1)\}\color{black}
Q3 3 2 4 QB2T2222
Q4 5 6 4 QB2T2222
Q5 3 6 8 QB2T2222
Q6 5 2 8 QB2T2222
R6 2 1 \color{red}\{agauss(100,5,1)\}\color{black}
*the local oscillator
VLO 6 2 DC 0 SIN(0 .05V 4e7 0 0)
Q1 4 10 11 QB2T2222
R1 11 12 \color{red}\{agauss(10,1,1)\}\color{black}
R2 12 13 \color{red}\{agauss(10,1,1)\}\color{black}
* input bias current
I1 12 0 DC 1.8mA
Q2 8 15 13 QB2T2222
R4 15 16 \color{red}\{agauss(1500,50,1)\}\color{black}
R3 16 10 \color{red}\{agauss(1500,50,1)\}\color{black}
V1 16 0 DC 1.8V
*the input voltage to be mixed with the LO
V5 15 10 DC 0 sin(0 .05V 3e6 0 0)
R7 5 17 \color{red}\{agauss(1500,50,1)\}\color{black}
R8 3 17 \color{red}\{agauss(1500,50,1)\}\color{black}
V3 17 0 DC 8V
V2 1 0 DC 6V

.MODEL QB2T2222  NPN (
+ IS=3.136905E-14 BF=189 NF=0.9977664 VAF=29.7280913
+ IKF=0.7405619 ISE=8.49314E-15 NE=1.3186316 BR=38.9531224
+ NR=0.9833179 VAR=28.5119855 IKR=4.632395E-3 ISC=4.624043E-14
+ NC=1.225899 RB=2.0158781 IRB=0.0227681 RBM=0.9730261
+ RE=0.0501513 RC=0.6333 CJE=2.369655E-11 VJE=0.6884357
+ MJE=0.3054743 TF=5.3E-10 XTF=48.5321578 VTF=5.3020062
+ ITF=1.1 PTF=14.6099207 CJC=8.602583E-12 VJC=0.4708887
+ MJC=0.3063885 XCJC=1 TR=74E-9 CJS=1e-12 VJS=.75
+ MJS=0 XTB=0.87435 EG=1.11 XTI=5.825
+ KF=0 AF=1 FC=0.5)

.TRAN 1ns 3e-7
.PRINT TRAN format=tecplot v(6,2) v(15,10) v(5,3)
.options timeint reltol=1.0e-5 
    \color{red}
.EMBEDDEDSAMPLING useExpr=true
.options EMBEDDEDSAMPLES projection\_pce=true 
+ outputs={v(5,3)} order=2 sparse\_grid=true \color{black}
.end
\color{blue}\end{vquote}
\end{minipage}
}
\caption{Gilbert Cell Non-Instrusive Spectral Project (NISP) netlist.  
  UQ commands are in \color{red}red\color{black}, and NISP is specified using the \texttt{projection\_pce} parameter.
\label{NISP_gilbert_netlist}}
\end{centering}
\end{figure}

NISP is used in conjunction with embedded sampling, 
and as noted requires 129 sample points if using sparse grid quadrature.  Fully 
intrusive PCE, which uses all the same machinery as NISP also requires 129 
quadrature points.  



\begin{figure}[hbt]
\centering
\includegraphics[width=5in]{gilbert_compare2.pdf}
\caption
  [Gilbert Cell Mixer Output for several \Xyce{} embedded UQ methods]
  {Gilbert Cell Mixer Output for several \Xyce{} embedded UQ methods.  The mean and standard deviations for all the methods (Embedded LHS sampling, NISP, Regression PCE and fully intrusive PCE) all match very well.}
\label{gilbert_compare}
\end{figure}

\clearpage


\section{Harmonic Balance Analysis}
\label{HB_Analysis}
\label{HB_Overview}
\index{analysis!HB} \index{Harmonic Balance Analysis}
\index{\texttt{.HB}}

Harmonic balance (HB) is a technique that solves for the steady state
solution of nonlinear circuits in the frequency domain. In harmonic
balance simulation, voltages and currents in a nonlinear circuit are
represented by truncated Fourier series. HB directly computes the
frequency spectrum of voltages and currents at the steady state
solution. This can be more efficient than transient analysis in
applications where a transient analysis may take a long time to reach
the steady state solution.  In particular, HB is well suited for
simulating analog RF and microwave circuits.

HB supports an unlimited number of independent input tones for driven
circuits in both serial and parallel builds of \Xyce{}. The output of
HB analysis is the real and imaginary components of voltages and
currents in the frequency domain. \Xyce{} also provides time domain
responses of a circuit. By default, the numbers of samples in both
time and frequency domain outputs are the same.  However, the number
of time domain samples can be much larger than the number of frequency
domain samples by setting \verb|numtpts| in
\verb|.options hbint|. This enables the users to use a small number of
harmonics for each tone and produce well-resolved results in time
domain.

For HB analysis, an initial guess of the solution is required. Often,
a good initial guess is necessary for HB to converge. By default,
\Xyce{} uses transient analysis to determine the initial guess, often
called ``Transient Assisted HB''.  This option is controlled by the
parameter, \verb|TAHB|, which is enabled if \verb|TAHB=1| and disabled
if \verb|TAHB=0| in \verb|.options hbint|. If enabled, the starting
time of the transient analysis can be modified by specifying the
parameter \verb|STARTUPPERIODS| in \verb|.options hbint|. The \Xyce{}
Reference Guide\ReferenceGuide{} provides a detailed explanation of
the HB options.

\subsection{.HB Statement}

To run a HB simulation, the circuit netlist file must contain a \verb|.HB| command.

\Example{\\
\texttt{.HB 1e4} \\
\texttt{.HB 1e4 2e2}
}

The parameters following \verb|.HB| are the fundamental frequencies and {\bf must} be
specified by the user. The \Xyce{} Reference Guide\ReferenceGuide{} provides a detailed 
explanation of the \verb|.HB| statement.


\subsection{HB Options}

Key parameters for \verb|.HB| simulation can be specified by
\verb|.options hbint|.
 
\Example{\\

\texttt{.options hbint numfreq=5,3 STARTUPPERIODS=2 tahb=1 intmodmax=3  numtpts=100} \\
\texttt{.HB 1e4 2e2} 
} 

As shown in the example, \texttt{numfreq} specifies the number of
harmonics to be calculated for each tone. It must have the same number
of entries as the fundamental frequencies in the \verb|.HB| statement.
This example shows the options for a two tone HB analysis.  The
\verb|.HB| statement is \texttt{ .HB $f_1$ $f_2$ }. For the first tone
$f_1$, 5 harmonics are used and for the second tone $f_2$, 3 harmonics
are used.

The \texttt{intmodmax} option is the maximum intermodulation product
order used in the spectrum.

As stated above, \texttt{TAHB} is the Transient Assisted HB option. In
the case of multi-tone HB analysis, the initial guess is generated by
a single tone transient simulation.  The single tone that is used is
the first tone in the \verb|.HB| statement.  So, when ordering the
fundamental frequencies in the \verb|.HB| statement, the first tone
should be the frequency that produces the most nonlinear response by
the circuit.

As stated above, the \texttt{STARTUPPERIODS} option specifies the
number of time periods that \Xyce{} will integrate through using
normal transient analysis \textbf{before} generating the initial
conditions for HB analysis.  For this example, \Xyce{} will integrate
through two periods before computing the initial conditions, which
requires an additional period.  Thus, \Xyce{} will integrate through a
total of three periods to compute the initial guess for HB analysis.

The \texttt{numtpts} option specifies the number of time points in the
output. The number of samples in the time domain output is an odd
number. If an even number is specified, the number of time points will
be \texttt{numtpts} plus one. For this example, the time domain output
has $101$ samples.

\Example{\\
\texttt{.options hbint numfreq=20 STARTUPPERIODS=2}
}

For the single tone example given above, this statement will return
$20$ negative and $20$ positive harmonics plus the DC component.

The \Xyce{} Reference Guide\ReferenceGuide{} provides a detailed
explanation of the \verb|.options hbint|.

\subsubsection{Nonlinear Solver Options}

The HB analysis uses a different set of default nonlinear solver
parameters than that of transient and DC analysis. Nonlinear solver
parameters for HB analysis can be specified by using
\verb|.options nonlin-hb|.

\Example{\\
\texttt{.options nonlin-hb abstol=1e-6}
}

The \Xyce{} Reference Guide\ReferenceGuide{} provides a detailed
explanation of the \verb|.options nonlin-hb| statement.

\subsubsection{Linear Solver Options}

The HB analysis provided by \Xyce{} can employ both iterative and
direct solvers.  If iterative solvers are used, then the HB Jacobian
matrix is not assembled and only HB-specific preconditioners can be
used for this matrix-free approach.  If direct solvers are used, then
the HB Jacobian is assembled and the complex-valued matrix is solved
using the selected method.  Direct solvers are memory intensive and
computationally expensive, so it is suggested that they are only used
when iterative methods fail during HB analysis.  You can set iterative
solver and preconditioner options using the \verb|.options linsol-hb|
statement.

\Example{\\
\texttt{.options linsol-hb type=aztecoo prec\_type=block\_jacobi AZ\_tol=1e-9}
}

In this example, \texttt{type} specifies the iterative solver to use
in the HB analysis and \texttt{AZ\_tol} is the relative tolerance for
the iterative solver.  Any of the iterative solver options in
section~\ref{LinearSolver_Options} are valid.  However,
\texttt{prec\_type} specifies which HB-specific preconditioner to
use. The choices for this option are \texttt{none} and
\texttt{block\_jacobi} (default).

\subsection{Output}
\label{HB_Output}\index{\texttt{.PRINT}!\texttt{HB}}

HB analysis can generate a variety of output files.  The selection of
which files are generated most often depends on what is specified for
the \texttt{.PRINT HB} command.  Table~\ref{HB_Output_table} lists the
format options and files created.  The column labeled ``Additional
Columns'' lists the additional data that is written, though not
specified on the \texttt{.PRINT HB} line.

\begin{table}[htbp]
  \caption{Output generated for HB analysis \label{HB_Output_table}}
  \begin{tabularx}{\linewidth}{|p{2.75in}|Y|Y|}
    \rowcolor{XyceDarkBlue} \color{white}\textbf{Command} & \color{white}\textbf{Files} & \color{white}\textbf{Additional Columns} \\ \hline
\texttt{.PRINT HB} & \emph{circuit-file}.HB.TD.prn \newline \emph{circuit-file}.HB.FD.prn & INDEX TIME \newline INDEX FREQUENCY \\ \hline
\texttt{.PRINT HB\_TD FORMAT=NOINDEX} \newline \texttt{.PRINT HB\_FD FORMAT=NOINDEX} & \emph{circuit-file}.HB.TD.prn \newline \emph{circuit-file}.HB.FD.prn & TIME \newline FREQUENCY \\ \hline
\texttt{.PRINT HB\_TD FORMAT=CSV} \newline \texttt{.PRINT HB\_FD FORMAT=CSV} & \emph{circuit-file}.HB.TD.csv \newline \emph{circuit-file}.HB.FD.csv & TIME \newline FREQUENCY \\ \hline
\texttt{.PRINT HB\_TD FORMAT=TECPLOT} \newline \texttt{.PRINT HB\_FD FORMAT=TECPLOT} & \emph{circuit-file}.HB.TD.dat \newline \emph{circuit-file}.HB.FD.dat & TIME \newline FREQUENCY \\ \hline
\texttt{.options hbint STARTUPPERIODS=n} & \multicolumn{2}{c|}{Falls back to .PRINT HB variables} \\ \hline
\texttt{.PRINT HB\_STARTUP \newline .options hbint STARTUPPERIODS=n} & \emph{circuit-file}.startup.prn & INDEX TIME \\ \hline
\texttt{.PRINT HB\_STARTUP FORMAT=CSV \newline .options hbint STARTUPPERIODS=n} & \emph{circuit-file}.startup.csv & TIME \\ \hline
\texttt{.PRINT HB\_STARTUP FORMAT=TECPLOT \newline .options hbint STARTUPPERIODS=n} & \emph{circuit-file}.startup.dat & TIME \\ \hline

\texttt{.options hbint SAVEICDATA=1} & \multicolumn{2}{c|}{Falls back to .PRINT HB variables} \\ \hline
\texttt{.PRINT HB\_IC \newline .options hbint SAVEICDATA=1} & \emph{circuit-file}.hb\_ic.prn  & INDEX TIME \\ \hline
\texttt{.PRINT HB\_IC FORMAT=CSV \newline .options hbint SAVEICDATA=1} & \emph{circuit-file}.hb\_ic.csv & TIME \\ \hline
\texttt{.PRINT HB\_IC FORMAT=TECPLOT \newline .options hbint SAVEICDATA=1} & \emph{circuit-file}.hb\_ic.dat & TIME \\ \hline

\texttt{.OP} & \emph{log-file} & Operating point information \\ \hline

  \end{tabularx}
%% \index{sources!time-dependent}
\end{table}

\subsection{User Guidance}

One of the most common errors in the HB simulation setup is the use of
too few harmonic frequencies (i.e., \texttt{numfreq} is too
small). One way to determine the optimum number of harmonic
frequencies is to first simulate the circuit with a small
\texttt{numfreq}, then increase the \texttt{numfreq} until the
solution stops changing within a significant bound.  Requesting too
many harmonic frequencies is wasteful of memory and simulation time,
so it is not practical to request a very high \texttt{numfreq} either.

\section{AC Analysis}
\label{AC_Analysis}
\label{AC_Sweep_Overview}
\index{analysis!AC} \index{AC analysis} \index{\texttt{.AC}}
\index{AC sweep} \index{analysis!AC sweep}

The AC small-signal analysis of \Xyce{} computes AC output variables
as a function of frequency. The program first computes the DC
operating point of the circuit and linearizes the circuit. The
resultant linear circuit is then analyzed over a user-specified range
of frequencies. The desired output of an AC small-signal analysis is
usually a transfer function (voltage gain, transimpedance, etc). If
the circuit has one AC input, it is convenient to set that input to
unity and zero phase so output variables have the same value as the
transfer function of the output variable with respect to input.

\subsection{.AC Statement}

One may specify \verb|.AC| analyses by adding a \verb|.AC| line in the
netlist.  Some examples of typical \verb|.AC| lines include:

\Example{\\ 
\texttt{.AC DEC 10 1K 100MEG}  \\
\texttt{.AC DEC 10 1 10K}  \\
\texttt{.AC LIN 100 1 100HZ } \\
\texttt{.AC DATA=table } \\
\texttt{.param init=10, final=1K, step=100MEG } \\
\texttt{.AC DEC \{init\} \{final\} \{step\} } 
}

These examples include some types of sweep (linear, decade and data).
The \Xyce{} Reference Guide\ReferenceGuide{} provides a complete
description of all types of sweep.  Note that \texttt{DATA} is a
special case, which will allow sweeping of frequency, magnitude, phase
and linear model parameters simultaneously.

\subsection{AC Voltage and Current Sources}
\label{AC_Sources}
\index{AC sweep!sources}

\Xyce{} assumes the AC source to be a cosine waveform at a specified
phase angle. Its frequency must be defined in a separate ``.AC''
command defining the frequency for all the sources in the circuit. The
unique information for the individual source is the name (which must
start with ``V'' or ``I''), the node numbers, the magnitude of the
source, and its phase angle. Some examples are as follows:

\Example{\\   
%\texttt{*name nodelist type value  phase(deg)}  \\
\texttt{Vac   4   1    AC   120V    30}  \\
\texttt{Vin 1 0 1.44 ac .1  }  \\
\texttt{Iin 1 0 1.44e-5 ac 0.1e-5 sin(0 1 1e+5 0 0)} 
}

NOTE: The type, AC, must be specified because the default is DC. If
not specified, \Xyce{} assumes the phase angle to be zero degrees.
The units of the phase angle are in degrees.

%%%%%
\subsection{Example}
An example AC analysis netlist is given in figure~\ref{acExample}.
This particular example uses a \texttt{.data} statement to specify a
more complicated sweep.  Each line of the data table corresponds to a
different sweep point, so there are 3 points evaluated, and there is a
different magnitude, phase, frequency and resistance for each point.
\begin{figure}[htbp]
  \begin{centering}
    \shadowbox{
      \begin{minipage}{0.9\textwidth}
        \begin{vquote}
\color{blue}* AC Analysis example\color{black}
.param mag=1
.param phase=0.1
.param cmplxParam1=\{log(-1)\}  ; example complex parameter
.param cmplxParam2=\{2.0+3.0J\} ; example complex parameter

Isrc 1 0 AC \{mag\} \{phase\} 
R1 1 0 1e3
C1 1 0 2e-6

.data table
+  mag  phase  freq   r1
+  1.0   0.1  1.0e0  1e3
+  2.0   0.2  1.0e1  2e3
+  3.0   0.3  1.0e2  3e3
.enddata

\color{blue}* AC commands\color{black}
.AC data=table

\color{blue}* AC file output\color{black}
.print ac \{mag\} \{phase\} \{Isrc:acmag\} \{Isrc:acphase\} 
+ \{r1:r\} v(1) 
+ \{cmplxParam1 * V(1)\}
+ \{cmplxParam2 * V(1)\}

.end
\end{vquote}
\end{minipage}
}
\caption[AC Example Netlist]
{AC Example Netlist.  This example uses a .data statement to simultaneously set frequency, magnitude, phase and the resistance of r1. \label{acExample} }
\end{centering}
\end{figure}

%%%%%
\subsubsection{Linear Solver Options}

The AC analysis provided by \Xyce{} can employ both iterative and
direct solvers. You can set direct, iterative
solver, and preconditioner options using the \verb|.options linsol-ac|
statement.

\Example{\\
\texttt{.options linsol-ac type=ksparse}
}

In this example, \texttt{type} specifies a non-default direct solver
to use in the AC analysis.  Any of the direct and iterative solver options in
section~\ref{LinearSolver_Options} are valid. 

%%%%%
\subsection{Output}
\label{AC_Output}\index{\texttt{.PRINT}!\texttt{AC}}

During analysis a number of output files may be generated.  The
selection of which files are created depends on a variety of factors,
most obvious of which is the \texttt{.PRINT} command.
Table~\ref{AC_Output_table} lists the format options and files created.
The column labeled ``Additional Columns'' lists the additional data that
is written, though not specified on the \texttt{.PRINT} line.

\begin{table}[htbp]
  \caption{Output generated for AC analysis \label{AC_Output_table}}
  \begin{tabularx}{\linewidth}{|p{2.75in}|Y|Y|}
    \rowcolor{XyceDarkBlue} \color{white}\textbf{Command} & \color{white}\textbf{Files} & \color{white}\textbf{Additional Columns} \\ \hline
\texttt{.PRINT AC} & \emph{circuit-file}.FD.prn & INDEX FREQUENCY \\ \hline
\texttt{.PRINT AC FORMAT=NOINDEX} & \emph{circuit-file}.FD.prn & INDEX FREQUENCY \\ \hline
\texttt{.PRINT AC FORMAT=CSV} & \emph{circuit-file}.FD.csv & FREQUENCY \\ \hline
\texttt{.PRINT AC FORMAT=RAW} & \emph{circuit-file}.raw & FREQUENCY \\ \hline
\texttt{.PRINT AC FORMAT=TECPLOT} & \emph{circuit-file}.FD.dat & TIME \\ \hline
\texttt{.PRINT AC FORMAT=PROBE} & \emph{circuit-file}.csd & TIME \\ \hline

\texttt{\emph{Xyce} -r raw-file-name} & \emph{raw-file-name} & All circuit variables printed \\ \hline
\texttt{\emph{Xyce} -r raw-file-name -a} & \emph{raw-file-name} & All circuit variables printed \\ \hline

\texttt{.OP}  & falls back to .PRINT AC & Operating point information \\ \hline
\texttt{.PRINT AC\_IC FORMAT=NOINDEX} & \emph{circuit-file}.TD.prn & TIME \\ \hline
\texttt{.PRINT AC\_IC FORMAT=CSV} & \emph{circuit-file}.TD.csv & TIME \\ \hline
\texttt{.PRINT AC\_IC FORMAT=RAW} & \emph{circuit-file}.raw & TIME \\ \hline
\texttt{.PRINT AC\_IC FORMAT=TECPLOT} & \emph{circuit-file}.TD.dat & TIME \\ \hline
\texttt{.PRINT AC\_IC FORMAT=PROBE} & \emph{circuit-file}.TD.csd & TIME \\ \hline

\texttt{.OPTIONS NONLIN CONTINUATION=<method>\ldots} & falls back to AC & \\ \hline
\texttt{.PRINT HOMOTOPY FORMAT=NOINDEX} & \emph{circuit-file}.HOMOTOPY.prn & INDEX TIME \\ \hline

  \end{tabularx}
%% \index{sources!time-dependent}
\end{table}


%\subsection{Using the .PRINT AC Command}
\label{AC_print}
\index{AC sweep!print}

Running \Xyce{} on AC analysis produces an output results file named
\verb|.cir.FD.prn|. Obtaining this file requires that the
\verb|.PRINT AC| line be specified.

\Xyce{} supports printing the real and imaginary parts of phasor
values (complex numbers) for AC analysis output as voltages or
currents. For instance, specify ``V(1)'' to print the real part and
imaginary part of a voltage at node 1. Additional output variable
formats are also available:
\begin{XyceItemize}
\item \texttt{VR(<circuit node>)} output the real component of the voltage response
\item \texttt{VI(<circuit node>)} output the imaginary component of voltage response
\item \texttt{VM(<circuit node>)} output the magnitude of voltage response
\item \texttt{VDB(<circuit node>)} output the magnitude of voltage response in decibels
\item \texttt{VP(<circuit node>)} output the phase of voltage response, in degrees
\end{XyceItemize}

Also, \Xyce{} supports printing the real and imaginary parts of 
complex expressions.  The user can specify the real or imaginary 
parts of an expression using the \texttt{Re()} and 
\texttt{Img()} operators within the expression.  However, it usually 
isn't necessary to do this, as \Xyce{} will automatically output 
both the real and imaginary parts, if the expression produces a 
complex result.  If an expression produces a purely real result 
(even if the inputs are complex), then \Xyce{} will only output 
the real part.

Some examples are as follows: 
\Example{\\ 
\texttt{.print AC V(3) }\\
\texttt{.print AC VI(15)}\\
\texttt{.print AC VP(OUTPUT)} \\
\texttt{.print AC \{Re(V(OUTPUT))\} } \\
\texttt{.print AC \{Img(V(OUTPUT))\} } \\
\texttt{.print AC \{ V(OUTPUT)*2.0  \} }
}

As shown in the examples, the expression library can handle both real and complex valued solution variables.
See the \Xyce{} Reference Guide\ReferenceGuide{} for more details.  


%%%%%%
% -------------------------------------------------------------------------
% Noise Analysis Section --------------------------------------------------
% -------------------------------------------------------------------------
\section{NOISE Analysis}
\label{NOISE_Analysis}
\label{NOISE_Sweep_Overview}
\index{analysis!NOISE} \index{NOISE analysis}\index{\texttt{.NOISE}}
\index{NOISE sweep} \index{analysis!NOISE sweep}

\Xyce{} supports small-signal noise analysis, which is closely related
to AC analysis. For noise analysis, input and output noise is computed
for a specified output node, relative to an input source, as a
function of frequency.

The program first computes the DC operating point of the circuit and
linearizes the circuit.  Next, at each frequency, an AC calculation is
performed to obtain the AC gain.  This gain is used later to compute
the equivalent input noise from the output noise.  After the AC gain
is computed, the noise calculation is performed at the current
frequency using the adjoint method.  This involves solving the
transpose of the AC matrix.


If the circuit has one NOISE input, it is convenient to set that input
to unity and zero phase so output variables have the same value as the
transfer function of the output variable with respect to the input.

\subsection{.NOISE Statement}

One may specify \verb|.NOISE| analyses by adding a \verb|.NOISE| line in the netlist.  
Some examples of typical \verb|.NOISE| lines include:

\Example{\\ 
\texttt{.NOISE V(5) VIN LIN 101 100Hz 200Hz} \\
\texttt{.NOISE V(5,3) V1 OCT 10 1kHz 16kHz} \\
\texttt{.NOISE V(4) V2 DEC 20 1MEG 100MEG} \\
\texttt{.NOISE V(4) V2 DATA=table}
}

The .NOISE command can specify a linear sweep, decade logarithmic
sweep, octave logarithmic sweep, or a data table of multivariate
values. See the \Xyce{} Reference Guide\ReferenceGuide{} for details.

\subsection{NOISE Voltage and Current Sources}
\label{NOISE_Sources}
\index{NOISE sweep!sources}

Noise analysis requires a reference AC source.  \Xyce{} assumes the AC
source to be a cosine waveform at a specified phase angle. Its
frequency must be defined in a separate ``.NOISE'' command defining
the frequency for all the sources in the circuit. The unique
information for the individual source is the name (which must start
with ``V'' or ``I''), the node numbers, the magnitude of the source,
and its phase angle. Some examples are as follows:

\Example{\\   
%\texttt{*name nodelist type value  phase(deg)}  \\
\texttt{Vac   4   1  AC   120V    30}  \\
\texttt{Vin 1 0 1.44 ac .1  }  \\
\texttt{Iin 1 0 1.44e-5 ac 0.1e-5 sin(0 1 1e+5 0 0)} 
}

NOTE: The type, AC, must be specified because the default is DC. If
not specified, \Xyce{} assumes the phase angle to be zero degrees.
The units of the phase angle are in degrees.

\subsection{Examples}

Example noise analysis netlists are given in
figures~\ref{noiseExample} and \ref{noiseExampleWData}.  In those two
examples, the only noise source is the resistor, \texttt{rlp1}.  The
input source is \texttt{v1} and the output node is \texttt{v(4)}.  A
decade sweep is used in the first example, while a \texttt{.DATA}
statement is used in the second example.
\begin{figure}[htbp]
  \begin{centering}
    \shadowbox{
      \begin{minipage}{0.9\textwidth}
        \begin{vquote}
\color{blue}* Noise Analysis example\color{black}
v1  1 0 DC 5.0 AC  1.0   
r1  1 2 100K
r2  2 0 100K

eamp  3 0 2 0 1
rlp1  3 4 100
clp1  4 0 1.59nf

\color{blue}* Noise commands\color{black}
.noise v(4) v1 dec 5 100 100meg 1

\color{blue}* Noise file output\color{black}
.print noise inoise onoise

.end
\end{vquote}
\end{minipage}
}
\caption[Noise Example Netlist With Decade Sweep]
{Noise Example Netlist With Decade Sweep\label{noiseExample} }
\end{centering}
\end{figure}

\begin{figure}[htbp]
  \begin{centering}
    \shadowbox{
      \begin{minipage}{0.9\textwidth}
        \begin{vquote}
\color{blue}* Noise Analysis example\color{black}
.global\_param mag=1
.global\_param phase=0.1

v1  1 0 DC 5.0 AC \{mag\} \{phase\}
r1  1 2 100K
r2  2 0 100K

eamp  3 0 2 0 1
rlp1  3 4 100
clp1  4 0 1.59nf

.data table
+  mag  phase  freq  rlp1
+  1.0   0.1  1.0e6  1e2
+  2.0   0.2  1.0e7  2e2
+  3.0   0.3  1.0e8  3e2

\color{blue}* Noise commands\color{black}
.noise v(4) v1 data=table

\color{blue}* Noise file output\color{black}
.print noise inoise onoise

.end
\end{vquote}
\end{minipage}
}
\caption[Noise Example Netlist With .Data Statement]
{Noise Example Netlist.  This example uses a .data statement to simultaneously set
frequency, magnitude, phase and the resistance of rlp1.\label{noiseExampleWData} }
\end{centering}
\end{figure}

\subsection{Output}
\label{NOISE_Output}\index{\texttt{.PRINT}!\texttt{NOISE}}

During analysis a number of output files may be generated.  The
selection of which files are created depends on a variety of factors,
most obvious of which is the \texttt{.PRINT} command.
Table~\ref{NOISE_Output_table} lists the format options and files
created.  The column labeled ``Additional Columns'' lists the
additional data that is written, though not specified on the
\texttt{.PRINT} line.

\begin{table}[htbp]
  \caption{Output generated for NOISE analysis \label{NOISE_Output_table}}
  \begin{tabularx}{\linewidth}{|p{2.75in}|Y|Y|}
    \rowcolor{XyceDarkBlue} \color{white}\textbf{Command} & \color{white}\textbf{Files} & \color{white}\textbf{Additional Columns} \\ \hline
\texttt{.PRINT NOISE} & \emph{circuit-file}.NOISE.prn & INDEX FREQUENCY \\ \hline
\texttt{.PRINT NOISE FORMAT=NOINDEX} & \emph{circuit-file}.NOISE.prn & FREQUENCY \\ \hline
\texttt{.PRINT NOISE FORMAT=CSV} & \emph{circuit-file}.NOISE.csv & FREQUENCY \\ \hline
\texttt{.PRINT NOISE FORMAT=TECPLOT} & \emph{circuit-file}.NOISE.dat & FREQUENCY \\ \hline
  \end{tabularx}
\end{table}

%\subsection{Using the .PRINT NOISE Command}
\label{NOISE_print}
\index{NOISE sweep!print}

Running \Xyce{} NOISE analysis produces an output results file named
\verb|.cir.NOISE.prn|. Obtaining this file requires that the
\verb|.PRINT NOISE| line be specified. Using the \texttt{.PRINT} line,
\Xyce{} supports printing the total noise spectral density curves.
The output units for spectral densities are $V^2$/Hz and $A^2$/Hz.
Some examples are as follows:

\Example{\\ 
\texttt{.print noise inoise onoise} \\
\texttt{.print noise {log(inoise)} {log(onoise)}} \\
\texttt{.print noise inoise onoise V(4) VR(1)}  
}

\Xyce{} will also compute the total integrated output noise, and the
total integrated input noise for the specified output node.  This
information is sent to standard output, or to the user-specified log
file.  A typical output (which in this case was generated by the
netlist given in figure~\ref{noiseExample}) will look like this:

\Example{\\ 
\texttt{Total Output Noise = 1.28592e-09} \\
\texttt{Total Input Noise = 5.22876e-07} 
}

If \texttt{.STEP} is used with a noise analysis then the \Xyce{}
standard output will contain the values of the stepped parameters,
along with the values for the total integrated output noise and the
total integrated input noise for the specified output node.

Nodal voltages may be requested with a \verb|.PRINT NOISE| line, and
the values are the AC solution for the circuit.  All of the voltage
operators in Table ~\ref{Print_Complex} are supported for this case.

The DC values of voltage sources (e.g., \texttt{V1}) that are changed
with a \verb|.STEP| line during a NOISE analysis can be printed out
via the following syntax. Alternately, if a \verb|.OP| line is
included in the netlist then the output from the \verb+-r+ command
line option will contain the DC Operating Point (DCOP) values for all
of the nodal voltages.  (Note: Without the \verb|.OP| line, only the
AC solution will be included in the \verb+-r+ output for a NOISE
analysis.)

\Example{\\ 
\texttt{.print noise V1:DCV0} \\ 
}


\subsection{Device Model Support}

Note that stationary noise is not supported in every \Xyce{} device.
A list of \Xyce{} devices, and which of them support stationary noise,
is given in the \Xyce{} Reference Guide\ReferenceGuide{}.

The \texttt{DNI} and \texttt{DNO} operators can print out the individual
input and output noise contributions for each noise source within a device
via \verb|.PRINT NOISE| lines.  Figure ~\ref{DniDnoExample} provides an
example.

The \texttt{-noise\_names\_file} command line option can generate a listing
of the noise source names for each device in a netlist.  The output of that
command for the netlist given in figure ~\ref{DniDnoExample} is shown in
figure ~\ref{NoiseNamesFile}.  In this example, \texttt{DNI(M1)} would
output the input noise contribution for all noise sources in MOSFET M1, while
\texttt{DNI(Q1,FN)} would only output the input flicker-noise contribution
from MOSFET M1.

\begin{figure}[htbp]
  \begin{centering}
    \shadowbox{
      \begin{minipage}{0.9\textwidth}
        \begin{vquote}
\color{blue}* Noise Analysis: DNI and DNO Usage
* A Simple MOSFET Gain Stage.\color{black}

M1 3 2 0 0 nmos w=4u l=1u
.model nmos nmos level=1 tox=1e-7

Rsource 1 2 100k
Rload 3 vdd 25k

Vdd1 vdd 0 5
Vin 1 0 1.44 ac .1 sin(0 1 1e+5 0 0)

\color{blue}* Noise commands\color{black}
.noise v(3) Vin dec 10 100 1000Meg  1

\color{blue}* Noise file output\color{black}
.print noise inoise onoise
+ DNI(rsource) DNI(rload) DNI(M1)
+ DNI(M1,RD) DNI(M1,RS) DNI(M1,ID) DNI(M1,FN)
+ DNO(rsource) DNO(rload) DNO(M1)
+ DNO(M1,RD) DNO(M1,RS) DNO(M1,ID) DNO(M1,FN)

.end
\end{vquote}
\end{minipage}
}
\caption[Noise Example: Use of DNI and DNO Operators]
{Noise Example: Use of DNI and DNO Operators\label{DniDnoExample} }
\end{centering}
\end{figure}

\begin{figure}[htbp]
  \begin{centering}
    \shadowbox{
      \begin{minipage}{0.9\textwidth}
        \begin{vquote}
Valid DNO() and DNI() operators for this netlist.
Format is DNO(deviceName) for the total output noise for
a device or DNO(deviceName, noiseSource) for an individual
output noise contribution.  If a device (e.g., R1) only
has one noise source then use DNO(R1) to get the total
output noise.

deviceName       noiseType
RLOAD
M1
M1      fn
M1      id
M1      rd
M1      rs
RSOURCE
\end{vquote}
\end{minipage}
}
\caption[Example Output for -noise\_names\_file Command Line Option]
{Example Output for -noise\_names\_file Command Line Option\label{NoiseNamesFile} }
\end{centering}
\end{figure}

%%%%%%
% -------------------------------------------------------------------------
% Sensitivity Analysis Section --------------------------------------------
% -------------------------------------------------------------------------
\clearpage
\section{Sensitivity Analysis}
\label{SENS_Analysis}
\label{sensitivity_Overview}

The \texttt{.SENS} \index{\texttt{.SENS}} command instructs \Xyce{} to
calculate the sensitivities of an output expression with respect to a
specified list of circuit parameters.  This capability works for
steady-state (\texttt{.DC}), transient (\texttt{.TRAN}) and 
small-signal (\texttt{.AC}) analysis.

\Format{\par \texttt{.SENS objfunc=\{<output expression(s)>\}
    param=<circuit parameter(s)> \newline .options SENSITIVITY
    [direct=<1 or 0>] [adjoint=<1 or 0>]}}

At least one objective function parameter, \texttt{objfunc}, is
required.  It is possible to specify a comma-separated list of
objective functions.  The list of circuit parameters must have at
least one entry as well.  If there is more than one, the parameters
are specified as a comma-separated list.  \Xyce{} will not assume any
particular parameter list.  Unlike the Spice version of
\texttt{.SENS}, \Xyce{} will not automatically compute the sensitivity
of every parameter; only the parameters the user requested.

\subsection{Steady-State (DC) sensitivities}

For DC analysis, \Xyce{} can compute a direct sensitivity, an adjoint
sensitivity, or both.  In general, adjoint sensitivities are much more
efficient when computing the derivatives with respect to really large
numbers of parameters, but a small number of objective functions.
Direct sensitivities are most efficient for small numbers of
parameters, but large numbers of objective functions.  \Xyce{} allows
the user to specify multiple objective functions, as well as multiple
parameters, so either scenario could apply.

An example of using \texttt{.SENS} on a DC problem is shown in the
following netlist:
\begin{figure}[htbp]
  \begin{centering}
    \shadowbox{
      \begin{minipage}{0.9\textwidth}
        \begin{vquote}
Example Circuit using sensitivity analysis
R1 A B 10.0
R2 B 0 10.0
Va A 0 5

.dc Va 5 5 1
.print dc v(A) v(B)

\color{red}.print sens 
.SENS objfunc=\{0.5*(V(B)-3.0)**2.0\} param=R1:R,R2:R
.options SENSITIVITY direct=1 adjoint=1 stdoutput=1\color{black}
.END
\end{vquote}
\end{minipage}
}
\caption[Steady-State Sensitivity Example Netlist]
{Steady-State Sensitivity Example Netlist \label{DC_Sensitivity_Netlist} }
\end{centering}
\end{figure}
The output to stdout, which was requested by setting \texttt{stdoutput=1}, for this example is:
\begin{verbatim}
Direct Sensitivities of objective function:{0.5*(V(B)-3.0)**2.0}
Name	        Value	  Sensitivity	   Normalized
R1:R	   1.0000e+01	   6.2500e-02	   6.2500e-03
R2:R	   1.0000e+01	  -6.2500e-02	  -6.2500e-03

Adjoint Sensitivities of objective function:{0.5*(V(B)-3.0)**2.0}
Name	        Value	  Sensitivity	   Normalized
R1:R	   1.0000e+01	   6.2500e-02	   6.2500e-03
R2:R	   1.0000e+01	  -6.2500e-02	  -6.2500e-03
\end{verbatim}
In the above example, both adjoint and direct sensitivities are
computed.  This is a linear problem, so it is easy to compare them to
the analytic solution.
%\clearpage
\subsection{Transient sensitivities}
Sensitivities can also be computed for transient analysis.  As with
steady-state (DC) analysis, both direct and adjoint sensitivities are
supported.

\subsubsection{Transient Direct Sensitivities}
Transient direct sensitivities are a good choice if the number of
parameters is modest.  Also, if the output of interest is a full
waveform sensitivity, they will usually outperform adjoint
sensitivities, as from the perspective of adjoint calculations, each
time point constitutes a separate objective function, requiring a
separate integration.

Set \texttt{direct=1 adjoint=0} to specify transient direct
simulations.  The transient direct forumulation in \Xyce{} is based on
the one described by Hocevar~\cite{Hocevar1985}.  An example netlist
for a transient sensitivity calculation is given in
figure~\ref{Tran_Sensitivity_Netlist}.  This is a simple linear
problem (RC decay), so it has an analytic solution.
\begin{figure}[htbp]
  \begin{centering}
    \shadowbox{
      \begin{minipage}{0.99\textwidth}
        \begin{vquote}
          \fontsize{10pt}{11pt}\selectfont
\color{blue}* Transient sensitivity example\color{black}
.param cap=1u
.param res=1K
c1 1 0 {cap} 
R1 1 0 {res}
.IC V(1)=1.0
\color{blue}* Transient commands\color{black}
.tran 0 5ms uic
.options timeint reltol=1e-6 abstol=1e-6

\color{blue}* Conventional file output\color{black}
.print tran format=tecplot v(1) 
\color{blue}* Sensitivity file output.  \color{black}
.print sens format=tecplot
+ { exp(-time/(res*cap)) }; analytic solution (V(1))
+ \{(time/(res*res*cap))*exp(-time/(res*cap))\}; analytic dV(1)/dR
+ \{(time/(res*cap*cap))*exp(-time/(res*cap))\}; analytic dV(1)/dC
\color{red}* Sensitivity commands
.SENS objfunc=\{V(1)\} param=R1:R,C1:C
.options SENSITIVITY direct=1 adjoint=0\color{black}
.end
\end{vquote}
\end{minipage}
}
\caption[Direct Transient Sensitivity Example Netlist]
{Direct Transient Sensitivity Example Netlist \label{Tran_Sensitivity_Netlist} }
\end{centering}
\end{figure}
\begin{figure}[ht]
  \centering
  \scalebox{0.5}
  {\includegraphics[]{sens}}
  \caption[Transient direct sensitivity result]
  {Transient direct sensitivity result.
\label{transientSensitivityResult}}
\end{figure}
Results for the transient example are given in figure~\ref{transientSensitivityResult}.
The analytic sensitivity solution is given by the solid line and the computed numerical
sensitivity is given by the dashed line.  The results match very well in this case,
with the two lines right on top of each other.

\clearpage
\subsubsection{Transient Adjoint Sensitivities}
Transient adjoint sensitivities are a good choice for really large
numbers of parameters, and when the number of objective functions is
modest.  For transient calculations, each time point is considered a
separate objective function, so it is best to use adjoints when the
sensitivity of interest concerns only one or a handful of time points.

Set \texttt{direct=0 adjoint=1} to specify transient adjoint
simulations.  The transient adjoint forumulation in \Xyce{} has
similarities to the ones described by Liu~\cite{Liu2014} and
Meir~\cite{BLAST2012}.  An example netlist for a transient adjoint
sensitivity calculation is given in
figure~\ref{Tran_Adjoint_Sensitivity_Netlist}.  This is a simple
linear problem (RC driven by a sinewave), so it has an analytic
solution.
\begin{figure}[htbp]
  \begin{centering}
    \shadowbox{
      \begin{minipage}{0.9\textwidth}
        \begin{vquote}
\color{blue}*Test of transient adjoint sensitivities \color{black}
.param cap=1e-6
.param res=1e3

V1 1 0 0.0 sin (0.0 1.0 200 0.0 0.0 0.0 )
R1 1 2 {res}
C1 2 0 {cap}

.tran 1.0e-6 0.5e-2
.print tran format=tecplot V(1) V(2) 
.print TRANADJOINT format=tecplot

.options timeint method=gear reltol=1.0e-6 abstol=1e-6

\color{red}* Sensitivity commands
.SENS objfunc={V(2)} param=R1:R,C1:C
.options SENSITIVITY direct=0 adjoint=1\color{black}
.end
\end{vquote}
\end{minipage}
}
\caption[Adjoint Transient Sensitivity Example Netlist]
{Adjoint Transient Sensitivity Example Netlist \label{Tran_Adjoint_Sensitivity_Netlist} }
\end{centering}
\end{figure}
\begin{figure}[ht]
  \centering
  \scalebox{0.5}
  {\includegraphics[]{sensCapGearAdj}}
  \caption[Transient adjoint sensitivity result]
  {Transient adjoint sensitivity result.
\label{transientAdjointSensitivityResult}}
\end{figure}
Results for the transient adjoint example are given in 
figure~\ref{transientAdjointSensitivityResult}.  The
analytic sensitivity solution is given by the dashed line and the
computed numerical sensitivity is given by the solid line.  The
results match very well in this case, with the two lines right on top
of each other.

\paragraph{Guidance for running transient adjoint analysis}
Transient adjoint sensitivities need to perform a backward time integration, 
after the original transient forward problem has completed.    So, generally, 
when running a transient adjoint netlist, the user may notice a pause in the 
calculation at the very end.  This is normal.

It is also important to understand that if using a point objective, the 
transient adjoint algorithm will consider every time step to be a unique, 
separate objective, and will need to perform a backward integration for each one.  
So, it will usually be important to restrict this to a limited number of time 
steps, or ideally a single time point.

To perform the adjoint calculation on a range of times, this can be set using 
the \texttt{.OPTIONS SENSITIVITY} commands \texttt{ADJOINTBEGINTIME} and 
\texttt{ADJOINTFINALTIME}.  Each of these must be set equal to a floating point
number in units of seconds.  To perform the adjoint calculation on a discrete 
set of time points (or a single time point), this can be set using using the 
\texttt{.OPTIONS SENSITIVITY} command \texttt{ADJOINTTIMEPOINTS} followed by a 
comma-separated list of times in units of seconds.  An example netlist, which 
uses this capability is shown in figure~\ref{Tran_Adjoint_Sensitivity_Netlist2}.  
It is identical to the netlist shown in figure~\ref{Tran_Adjoint_Sensitivity_Netlist} 
except that this command has been added.
\begin{figure}[htbp]
  \begin{centering}
    \shadowbox{
      \begin{minipage}{0.9\textwidth}
        \begin{vquote}
\color{blue}*Test of transient adjoint sensitivities \color{black}
.param cap=1e-6
.param res=1e3

V1 1 0 0.0 sin (0.0 1.0 200 0.0 0.0 0.0 )
R1 1 2 {res}
C1 2 0 {cap}

.tran 1.0e-6 0.5e-2
.print tran format=tecplot V(1) V(2) 
.print TRANADJOINT format=tecplot

.options timeint method=gear reltol=1.0e-6 abstol=1e-6

* Sensitivity commands
.SENS objfunc={V(2)} param=R1:R,C1:C
.options SENSITIVITY direct=0 adjoint=1
\color{red}+ adjointTimePoints=0.1e-2, 0.2e-2, 0.3e-2, 0.4e-2 \color{black}
.end
\end{vquote}
\end{minipage}
}
\caption[Adjoint Transient Sensitivity Example Netlist for list of time points]
{Adjoint Transient Sensitivity Example Netlist for list of time points\label{Tran_Adjoint_Sensitivity_Netlist2} }
\end{centering}
\end{figure}

\clearpage
\subsection{AC Sensitivities}
Sensitivities can also be computed for small-signal AC analysis.  As
with steady-state (DC) and transient analysis, both direct and adjoint
sensitivities are supported.  

\Format{\par \texttt{.SENS acobjfunc=\{<output expression(s)>\} 
    \newline objvars=<voltage node name(s)>
    \newline param=<circuit parameter(s)> \newline .options SENSITIVITY
    [direct=<1 or 0>] [adjoint=<1 or 0>]}}

Two example netlists which exercise the
capability is shown in figures~\ref{AC_Sensitivity_Netlist} 
and~\ref{AC_Sensitivity_Netlist2}.
The netlist commands for AC sensitivities are very similar to those
for DC, but with some small exceptions.   

\paragraph{Setting the AC sensitivity ojbective function}
For AC sensitivities, objective functions can be set in two ways.  One option is to 
specify outputs using \texttt{objvars} followed by a list of node names.      
An example which  uses this method is given in figure~\ref{AC_Sensitivity_Netlist}.  
The other option is to specify the objective function as an expression, with 
the \texttt{acobjfunc} parameter.  
An example which  uses this method is given in figure~\ref{AC_Sensitivity_Netlist2}.  
It is also possible to specify both types of objectives in the same netlist.

\paragraph{AC Sensitivities have four objectives per output}
For each specified output, there are four separate objective functions
for which sensitivities will be computed; magnitude and phase for the
polar representation, as well as real and imaginary for the cartesian
representation of the solution.   These will be computed with respect to 
the every parameter in the list of sensitivity parameters.   

Example console output results are shown for the netlists in figures~\ref{AC_Sensitivity_Netlist} 
and ~\ref{AC_Sensitivity_Netlist2} in figures~\ref{AC_Sensitivity_Result} 
and~\ref{AC_Sensitivity_Result2}, respectively.  The console output shown is what one gets 
if \texttt{stdoutput=1} is set in the netlist on the \texttt{.options sensitivity} line.
For file-based output, see section~\ref{SENS_Output}.  

\paragraph{Output of $\Delta P$}
For AC sensitivities, it is likely that a numerical derivative will be needed
as part of the calculation.  This is because, unlike DC and transient 
sensitivities, the AC calculation often requires a matrix derivative.  
The Xyce source code has not been set up to compute analytical matrix 
derivatives, so these must always be computed numerically.
While the numerical derivatives are accurate, some information about 
them has been added to the AC sensitivity console output.

In the console outputs shown in figures~\ref{AC_Sensitivity_Result} 
and~\ref{AC_Sensitivity_Result2}, the two right-most columns
pertain to numerical finite differences.  The ``Delta P'' column 
gives the delta used by the sensitivity calculation if a numerical derivative was needed.   
The ``FD used'' or ``FD not used'' column indicates if finite differences were used 
in part of the calculation.    

\begin{figure}[htbp]
  \begin{centering}
    \shadowbox{
      \begin{minipage}{0.99\textwidth}
        \begin{vquote}
\color{blue}*Lowpass filter test for AC sensitivities\color{black}
v1 1 0 ac 10.0
r1 1 2 4.7k
c1 2 0 47n

.ac dec 10 1 10k
.print ac  vm(1) vp(1) vm(2) vp(2)   
.print sens 

\color{red}* Sensitivity commands
.sens objvars=2 param=r1:r,c1:c
.options sensitivity direct=1 adjoint=0  stdoutput=1\color{black}
.end 
\end{vquote}
\end{minipage}
}
\caption[AC Sensitivity Example Netlist]
{AC Sensitivity Example Netlist \label{AC_Sensitivity_Netlist} }
\end{centering}
\end{figure}



\begin{figure}[htbp]
  \begin{centering}
    \shadowbox{
      \begin{minipage}{0.99\textwidth}
        \begin{vquote}
          \fontsize{8pt}{10pt}\selectfont
Direct Sensitivities for node V(2):
VR(2) = 1.0000e+01  VI(2) = -1.3880e-02
VM(2) = 1.0000e+01  VP(2) = -1.3880e-03

Name        Value    Sens\_Re    Sens\_Im    Sens\_Mag   Sens\_Phase  Delta P
R1:R     4.7000e+03 -8.1975e-09 -2.9531e-06 -4.0988e-09 -2.9531e-07  7.0035e-05 FD not used
C1:C     4.7000e-08 -8.1975e+02 -2.9531e+05 -4.0988e+02 -2.9531e+04  7.0035e-16 FD not used
V1:ACMAG 1.0000e+01  1.0000e+00 -1.3880e-03  1.0000e+00 -2.7105e-20  1.4901e-07 FD not used
\end{vquote}
\end{minipage}
}
\caption[AC Sensitivity Example Result]
    {AC Sensitivity Example Result.  This result corresponds to the first objective in the netlist given by figure~\ref{AC_Sensitivity_Netlist}.
    \label{AC_Sensitivity_Result} }
\end{centering}
\end{figure}

\begin{figure}[htbp]
  \begin{centering}
    \shadowbox{
      \begin{minipage}{0.99\textwidth}
        \begin{vquote}
\color{blue}* Operational Amplifier\color{black}
.param m6w=2u
.param m6l=1u
*Operational Amplifier
M1 bias1 1 cm cm nmos w=10u l=1u
M2 bias2 in2 cm cm  nmos w=10u l=1u
M3 vdd bias1 bias1 vdd pmos w=2u l=1u
M4 bias2 bias1 vdd vdd pmos w=2u l=1u
m5 cm bias vss vss nmos w=2u l=1u
mbias bias bias vss vss nmos w=2u l=1u
rbias 0 bias 195k
m6 8 bias vss vss  nmos w=\{m6w\} l=\{m6l\}
m7 8 bias2 vdd out nmos w=2u l=1u
Cfb bias2 8 2p
Vid 1 c 0 ac 0.1
eid in2 c 1 c -1
vic c 0 dc 0
vss1 vss 0 -5
Vdd1 vdd 0 5 

*AC analysis
.ac dec 10 100 100Meg 
.print ac v(8)
\color{red}* Sensitivity commands
.sens acobjfunc=\{v(8)\} param=m6:w,m6:l
.options sensitivity direct=1 adjoint=0  stdoutput=1   
.print sens \color{black}

.model nmos nmos level=9 version=3.2.2
.model pmos pmos level=9 version=3.2.2
.end
\end{vquote}
\end{minipage}
}
\caption[AC Sensitivity Example Netlist, using expression-based objective function]
{AC Sensitivity Example Netlist, using expression-based objective function\label{AC_Sensitivity_Netlist2} }
\end{centering}
\end{figure}

\begin{figure}[htbp]
  \begin{centering}
    \shadowbox{
      \begin{minipage}{0.99\textwidth}
        \begin{vquote}
          \fontsize{8pt}{10pt}\selectfont
Direct Sensitivities for \{v(8)\}
 Re(\{v(8)\}) = 1.7365e+01  Img(\{v(8)\}) = 3.5597e-04
  M(\{v(8)\}) = 1.7365e+01   Ph(\{v(8)\}) = 6.7295e-02
Name    Value      Sens\_Re    Sens\_Im    Sens\_Mag   Sens\_Phase  Delta P
M6:W  2.0000e-06 -2.2615e+05  -1.4915e+02 -2.2615e+05 -4.7680e+02  2.9802e-14 FD used
M6:L  1.0000e-06  8.6598e+05   1.9292e+02  8.6598e+05  5.7798e+02  1.4901e-14 FD used
\end{vquote}
\end{minipage}
}
\caption[AC Sensitivity Example Result]
    {AC Sensitivity Example Result.  This result is the first sensitivity output produced for the first objective in the netlist given by figure~\ref{AC_Sensitivity_Netlist2}.  
    \label{AC_Sensitivity_Result2} }
\end{centering}
\end{figure}

\subsection{Notes about .SENS accuracy and formulation}

The formulation for \Xyce{} sensitivity analysis is briefly described.  For more details, see
references~\cite{KeiterGradEnhnacedPCE2016,KeiterUQ2019}.
The sensitivity calculation in \Xyce{} is based on a chain rule
calculation, which will produce the sensitivity $dO/dp$, where
$O$ is a user-specified scalar objective function, and $p$ is a
user-specified scalar parameter.  $dO/dp$ is equal to:
\begin{equation}
  \frac{dO}{dp} = \frac{\partial O}{\partial x}\frac{\partial x}{\partial p} + \frac{\partial O}{\partial p}
  \label{objectiveDerivative}
\end{equation}
\noindent where $x$ is a solution vector and $\partial x/\partial p$
is the sensitivity of that solution vector with respect to the
parameter.  
For DC and Transient, the system of equations solved by \Xyce{} is given as a differential algebraic system, given by:
\begin{equation}
  F(x,t,p) = \dot{q}(x(t,p),p) + j(x(t,p),p) - b(t,p) = 0,
  \label{residual}
\end{equation}
where the vector $F$ is the \Xyce{} residual. The derivative of $F$ with 
respect to the solution $x$ ($\partial F/\partial x$) is the Jacobian matrix.
The terms $q$, $j$ and $b$ vectors are all standard components of a DAE, and are provided by
the various device models.  $q$ represents quantities that must be
differentiated with respect to time (such as capacitor charge), and
$j$ represents algebraic terms that depend on the solution $x$ (such
as DC currents), and $b$ are independent sources that only depend on
time.  Equation~\ref{residual} is solved to obtain the solution, $x$.

Equation~\ref{objectiveDerivative} can then be expanded as:
\begin{equation}
  \frac{dO}{dp} = -\frac{\partial O}{\partial x} \left(\frac{\partial F}{\partial x} \right)^{-1} \frac{\partial F}{\partial p} + \frac{\partial O}{\partial p}, {}  
  \label{objectiveDerivative2}
\end{equation}
All of the terms in equation~\ref{objectiveDerivative2} are available in Xyce, which makes it possible to compute $dO/dp$.   
Also, most terms in~\ref{objectiveDerivative2} purely analytical, which helps ensure accuracy. 
The order in which the right-hand-side of equation~\ref{objectiveDerivative2} is evaluated 
(left-to-right or right-to-left) determines if the method is a direct sensitivty or an 
adjoint sensitivity.

\subsubsection{Direct Sensivity}
To obtain direct sensitivities, one must differentiate equation~\ref{residual} with
respect to a parameter, $p$.
\begin{equation}
  \frac{dF(x,t,p)}{dp} = \frac{d}{dp}\left[ \frac{d q(x(t),p)}{dt} + j(x(t),p) - b(t,p) \right] = 0 
  \label{dfdp}
\end{equation}
In steady-state (DC), equation~\ref{dfdp} simplifies to:
\begin{equation}
  \frac{dF(x,t,p)}{dp} = \frac{d}{dp}\left[ j(x(t),p) - b(t,p) \right] = 0 
  \label{steady_dfdp}
\end{equation}
\noindent As $j$ is dependent upon $x$, the $dj/dp$ term  
be expanded via chain rule as $\frac{dj}{dp} = \frac{\partial j}{\partial x}\frac{\partial x}{\partial p} + \frac{\partial j}{\partial p}$, where $\frac{\partial j}{\partial x}$ is the DC Jacobian matrix,
and the resulting equation must be re-arranged to set up a matrix equation that can be solved to obtain
$\frac{\partial x}{\partial p}$:
\begin{equation}
  \frac{\partial j}{\partial x}\frac{\partial x}{\partial p} = - \frac{\partial j}{\partial p} + \frac{db}{dp}.
  \label{finalSens}
\end{equation}
This linear equation is evaluated by the linear solver at the end of the steady-state solve, using the same LU factors as were used by the final Newton step.

For transient, a similar linear system is solved, which depends on the specific time integration
method used.  For Backward-Euler the linear system is:
\begin{equation}
  \left[ \frac{1}{h} \frac{\partial q}{\partial x} 
  + \frac{\partial j}{\partial x} \right] \frac{\partial x}{\partial p}_{n+1} 
 =
  -\frac{1}{h} \left[ \frac{\partial q}{\partial p}_{n+1} - \frac{\partial q}{\partial p}_n \right] 
 - \frac{\partial j}{\partial p} 
 + \frac{db}{dp} 
 + \frac{1}{h} \left[ \frac{\partial q}{\partial x} \right] \frac{\partial x}{\partial p}_n 
 \label{finalTransientSens}
\end{equation}
\noindent Where $h$ is the time step size.  As with ~\ref{finalSens},
the left hand side of the equation contains the original Jacobian
matrix.  Similar to the steady-state case, this equation is solved at the end of each converged time step, using the matrix solver with the same LU factors as were used for the final Newton step.
For both steady state and transient, the final objective sensitivity is obtained by 
taking the dot product of $\partial O/\partial x$ and $\partial x/\partial p$.

\subsubsection{Adjoint Sensivity}
The description of adjoint sensitivies is beyond the scope of this document, but the issues affecting accuracy are largely the same.  For a description of the formulation, see references~\cite{KeiterGradEnhnacedPCE2016,KeiterUQ2019}.

\subsubsection{Analytical vs Numerical derivatives}
As mentioned above, most of the terms in the sensitivity equations are analytical.  
The possible exception is with the 
terms $\partial q/\partial p$, $\partial j/\partial p$ and $\partial b/\partial p$, which 
depend on device model implementation.  Most devices in \Xyce{} are instrumented to provide 
these sensitivities analytically, either with hand-written derivatives or via automatic differentiation.
But there are exceptions to this, so some devices do not.  For these devices, a numerical 
derivative is computed instead.  A list of \Xyce{}
devices, and which of them support analytic sensitivities, is given in
the \Xyce{} Reference Guide\ReferenceGuide{}.  When numerical derivatives are used, 
\Xyce{} automatically selects a small delta just large enough to 
avoid machine precision, following the  procedure suggested by Dennis and 
Schnabel~\cite{doi:10.1137/1.9781611971200.ch5}.

\subsubsection{Time integration error}
For transient, note that the transient direct calculation uses the
same time steps that are used for the original circuit analysis.  It
does not impose any additional error control that is specific to the
accuracy of $\partial x/\partial p$.  The adjoint calculation 
(not described here), is a discrete form of transient adjoints, and thus 
uses the same time steps as the forward calculation.

\subsubsection{AC Sensitivities}
While many of the accuracy issues pertaining to AC sensitivities are the same as for DC sensitivities, 
the formulation is different enough to introduce an additional issue.  AC sensitivities will nearly always
use some numerical derivatives in their evaluation.  AC sensitivies are based on small-signal analysis, 
so the analysis solves for the response of a circuit to a small sinusoidal signal.  
It first solves the DC operating point, and then approximates the AC response 
using a Taylor series expansion.  This exansion approximates a small change in the solution as a 
function of a small change in the AC stimulus.  The original AC equation is given by:
\begin{equation}
J \cdot \Delta x = \Delta b
  \label{ACeqn1}
\end{equation}
where $J$ is the AC Jacobian matrix, $\Delta x$ is the AC change to the solution 
vector and $\Delta b$ is the AC change to the source vector.  Equation~\ref{ACeqn1} 
assumes a phasor representation, so $\Delta x = X e^{i\omega t \pm \phi}$ 
and $\Delta b = B e^{i\omega t \pm \phi}$, where $\omega$ is the frequency 
and $\phi$ is the phase.  The AC Jacobian is determined from the DC Jacobian matrices for $f$ and $q$ as $J = (\partial f/\partial x + i\omega \partial q/\partial x)$.

For AC sensitivities, we want to compute $\Delta x' = \partial (\Delta x)/\partial p$, where $p$ is the sensitivity parameter.  To obtain  this, one takes derivative of equation~\ref{ACeqn1} with respect to a parameter, $p$. This gives the direct sensitivity equation 
\begin{equation}
J \cdot \Delta x' = \Delta b' - J' \cdot \Delta x,
  \label{ACsenseqn1}
\end{equation}
where the apostrophe indicates a parameter derivative.  The linear system in 
equation~\ref{ACsenseqn1} must be solved to  obtain $ \Delta x'$, the sensitivity 
of $\Delta x$ with respect to a parameter.   This is the equation for the direct 
method, but a related equation can be obtained for the adjoint method.  To solve 
equation~\ref{ACsenseqn1}, the terms on the right hand side must be provided by 
\Xyce{}.  The $\Delta x$ term is the AC solution, which must be computed prior to sensitivity analysis. 
The term $\Delta b'$ is trivial to provide analytically, as it is generally from ideal independent source models. The matrix derivative $J'$, however, is more complicated so it usually 
contains numerical derivatives.  To see why, consider that when $J'$ is fully 
expanded it is given as:
\begin{equation}
J' = \frac{\partial J}{\partial p} + \frac{\partial J}{\partial x} \cdot \frac{\partial x}{\partial p}.
  \label{Jderiv1}
\end{equation}
The right hand side terms $\partial J/\partial p$ and $\partial J/\partial x$ of~\ref{Jderiv1} 
are matrix derivatives, and 
\Xyce{} is not instrumented to provide these using the usual method of automatic differentiation.
For complex device models, implementing matrix derivatives manually is prohibitive.  Therefore, 
with the exception of simple linear devices, the terms in equation~\ref{Jderiv1} are
evaluated numerically.
%%%%
\clearpage
\subsection{Output}
\label{SENS_Output}\index{\texttt{.PRINT}!\texttt{.SENS}}
For a full list and explanation of options related to \texttt{.SENS}
output see the \Xyce{} Reference Guide\ReferenceGuide{}.

Sensitivity output can be sent to standard output, or to a
user-specified log file.  The format for this output is similar to
that generated by the circuit given in
figure~\ref{DC_Sensitivity_Netlist}.  This feature is mainly made
available in \Xyce{} so as to be similar to the sensitivity analysis
of older circuit simulators.  However, for most uses it isn't the most
practical output, so it is disabled by default.  To enable standard
output, one should set the following:
\begin{verbatim}
.options SENSITIVITY STDOUTPUT=1
\end{verbatim}

In addition to the screen output, \Xyce{} can also produce a plottable
file containing all the requested sensitivities.  This file can be
requested by adding either a \texttt{.PRINT SENS} or a \texttt{.PRINT
  TRANADJOINT} command to the input file.  Steady-state sensitivities
(adjoint or direct) and transient direct sensitivities will be handled
by the \texttt{.PRINT SENS} command.  Transient adjoint, on the other
hand, is handled by the \texttt{.PRINT TRANADJOINT} command.

Unlike the traditional \texttt{.PRINT} line, both \texttt{.PRINT SENS}
and \texttt{.PRINT TRANADJOINT} will assume that the user wants all
the sensitivities specified on the \texttt{.SENS} line.  As such it is
not necessary (or possible) to specify specific sensitivities on the
\texttt{.PRINT SENS} line.  If the line exists, that is sufficient to
produce the file, and it will contain a column for every objective
function and every derivative.  The file name is the same as the one
produced with \texttt{.PRINT}, but with ``\texttt{.SENS}'' included
just before the suffix.  Similarly, for transient adjoint, the output
file name has the string ``\texttt{.TRADJ}'' included before the
suffix.  Note also, that most of the same output formats
(\texttt{std}, \texttt{tecplot}, etc.)  are available for
\texttt{.PRINT SENS} and\texttt{.PRINT TRANADJOINT} as they are for
conventional \texttt{.PRINT}.  The available formats are listed in
table~\ref{SENS_Output_table},~\ref{SENS_AC_Output_table}and~\ref{TRANADJOINT_Output_table}.
\begin{table}[htbp]
  \caption{Output generated for SENS analysis for .TRAN\label{SENS_Output_table}}
  \begin{tabularx}{\linewidth}{|p{2.75in}|Y|Y|}
    \rowcolor{XyceDarkBlue} \color{white}\textbf{Command} & \color{white}\textbf{Files} & \color{white}\textbf{Additional Columns} \\ \hline
\texttt{.PRINT SENS} & \emph{circuit-file}.SENS.prn & INDEX TIME \\ \hline
\texttt{.PRINT SENS FORMAT=GNUPLOT} & \emph{circuit-file}.SENS.prn & INDEX TIME \\ \hline
\texttt{.PRINT SENS FORMAT=SPLOT} & \emph{circuit-file}.SENS.prn & INDEX TIME \\ \hline
\texttt{.PRINT SENS FORMAT=NOINDEX} & \emph{circuit-file}.SENS.prn & INDEX TIME \\ \hline
\texttt{.PRINT SENS FORMAT=CSV} & \emph{circuit-file}.SENS.csv & TIME \\ \hline
\texttt{.PRINT SENS FORMAT=TECPLOT} & \emph{circuit-file}.SENS.dat & TIME \\ \hline
  \end{tabularx}
\end{table}
\begin{table}[htbp]
  \caption{Output generated for SENS analysis for .AC\label{SENS_AC_Output_table}}
  \begin{tabularx}{\linewidth}{|p{2.75in}|Y|Y|}
    \rowcolor{XyceDarkBlue} \color{white}\textbf{Command} & \color{white}\textbf{Files} & \color{white}\textbf{Additional Columns} \\ \hline
\texttt{.PRINT SENS} & \emph{circuit-file}.FD.SENS.prn & INDEX FREQ \\ \hline
\texttt{.PRINT SENS FORMAT=GNUPLOT} & \emph{circuit-file}.FD.SENS.prn & INDEX FREQ \\ \hline
\texttt{.PRINT SENS FORMAT=SPLOT} & \emph{circuit-file}.FD.SENS.prn & INDEX FREQ \\ \hline
\texttt{.PRINT SENS FORMAT=NOINDEX} & \emph{circuit-file}.FD.SENS.prn & INDEX FREQ \\ \hline
\texttt{.PRINT SENS FORMAT=CSV} & \emph{circuit-file}.FD.SENS.csv & FREQ \\ \hline
\texttt{.PRINT SENS FORMAT=TECPLOT} & \emph{circuit-file}.FD.SENS.dat & FREQ \\ \hline
  \end{tabularx}
\end{table}
As noted, transient adjoints are specified separately from
\texttt{.PRINT SENS}.  This is because the transient adjoint
calculation is performed as a post-process, after the original forward
calculation has been completed, and \Xyce{}'s forward outputters are
no longer active.  To specify transient adjoint output, one must use
\texttt{.PRINT TRANADJOINT} instead.  As it is possible to perform
both a transient direct and a transient adjoint calculation as part of
the same computation, and most of \Xyce{}'s output files are in column
format, there wasn't an easy way to have them use the same outputter.
\begin{table}[htbp]
  \caption{Output generated for transient adjoint SENS analysis \label{TRANADJOINT_Output_table}}
  \begin{tabularx}{\linewidth}{|p{2.75in}|Y|Y|}
    \rowcolor{XyceDarkBlue} \color{white}\textbf{Command} & \color{white}\textbf{Files} & \color{white}\textbf{Additional Columns} \\ \hline
\texttt{.PRINT TRANADJOINT} & \emph{circuit-file}.TRADJ.prn & INDEX TIME \\ \hline
\texttt{.PRINT TRANADJOINT FORMAT=NOINDEX} & \emph{circuit-file}.TRADJ.prn & INDEX TIME \\ \hline
\texttt{.PRINT TRANADJOINT FORMAT=CSV} & \emph{circuit-file}.TRADJ.csv & TIME \\ \hline
\texttt{.PRINT TRANADJOINT FORMAT=TECPLOT} & \emph{circuit-file}.TRADJ.dat & TIME \\ \hline
  \end{tabularx}
\end{table}

\clearpage

%%%%%%
% -------------------------------------------------------------------------
% S-parameter (.LIN) Analysis Section -------------------------------------
% -------------------------------------------------------------------------

\section{S-parameter Analysis}
\label{SP_Analysis}
\label{SP_Sweep_Overview}
\index{analysis!S-parameter} \index{S-parameter analysis} \index{\texttt{.LIN}}
\index{AC sweep} \index{analysis!AC sweep}

The S-parameter small-signal analysis of \Xyce{} computes linear
transfer parameters as a function of frequency for a general
multi-port network. The program first computes the DC operating point
of the circuit and then linearizes the circuit. The resultant linear
circuit is then analyzed over a user-specified range of
frequencies. The output of a S-parameter analysis is multi-port
scattering (S) parameters. The S-parameters represent the ratio of
incident and scattered normalized voltage waves.  The analysis results
can also be output as Y-parameter or Z-parameter values.

\subsection{.LIN Statement}

One may specify S-parameter analyses by using a \verb|.LIN| command with a \verb|.AC| command in the netlist.
Here is an example of typical \verb|.AC| and \verb|.LIN| lines:

\Example{\\
\texttt{.AC DEC 10 1K 100MEG}  \\
\texttt{.LIN sparcalc=1 format=touchstone}  \\
}

The \verb|.LIN| analysis is similar to a basic small signal \verb|.AC|
analysis, but it also calculates small signal transfer parameters
between terminals identified using port (P) devices. The \Xyce{}
Reference Guide\ReferenceGuide{} provides a complete description of
\verb|.LIN| analysis and the P device.  To output Y-parameter or
Z-parameter values instead the \texttt{LINTYPE=Y} or
\texttt{LINTYPE=Z} argument can be used on the \texttt{.LIN} line.

\subsection{Port Devices}
\label{SP_Port}
\index{S-parameter analysis!port}

The \verb|.LIN| analysis computes the S-parameters based on the
location of the port (P) devices and the values of their reference
impedances. The port devices identify the ports used in \verb|.LIN|
analysis. The \Xyce{} Reference Guide\ReferenceGuide{} provides a
complete description of the port devices. Some examples are as
follows:

\Example{\\   
%\texttt{*name nodelist type value  phase(deg)}  \\
\texttt{P1 1 0 port = 1} \\
\texttt{P2 12 0  port= 2  z0=100 }
}

NOTE: Each port requires a unique port number. If a circuit has N
ports, the netlist must contain the sequential set of port numbers, 1
to N.

\subsection{Example}
An example S-parameter analysis netlist is given in
figure~\ref{spExample}.  This example uses 2 ports.  The S parameters
are output to a file in Touchstone 1 format
\cite{touchstone2_std_2009}.  Touchstone 2 format is also supported.
Note that \texttt{SPARCALC=1} is specified on the \texttt{.LIN} line.
If \texttt{SPARCALC=0} is specified instead then a \texttt{.AC}
analysis will be done.

\begin{figure}[htbp]
  \begin{centering}
    \shadowbox{
      \begin{minipage}{0.9\textwidth}
        \begin{vquote}
\color{blue}* S-parameter Analysis example\color{black}
P1 1 0 port= 1
~
C1 2 0 1e-2
Rgs 1 2 0.02

.subckt RCBlock IN OUT GND
R1 IN OUT 20
C1 IN OUT 1p
Cg1 OUT GND 1p
.ends

X1 2 3 0 RCBlock
X2 3 4 0 RCBlock
X3 4 5 0 RCBlock
X4 5 6 0 RCBlock
X5 6 7 0 RCBlock
X6 7 8 0 RCBlock
X7 8 9 0 RCBlock
X8 9 10 0 RCBlock
X9 10 11 0 RCBlock
X10 11 12 0 RCBlock

.AC DEC 10 10  1e5 

.LIN FORMAT=TOUCHSTONE  sparcalc=1

P2 12 0  port=2 z0=100

.END

\end{vquote}
\end{minipage}
}
\caption[S-parameter Example Netlist]
{S-parameter Example Netlist.  \label{spExample} }
\end{centering}
\end{figure}

\subsection{Output}
\label{LIN_Output}
For S-parameter analysis, the output is controlled by the \texttt{.LIN}
command when \texttt{SPARCALC=1} on that line.  Any information on
\texttt{.PRINT AC} lines will be ignored in that case.
Table~\ref{LIN_Output_table} lists the format options and files created.
If \texttt{SPARCALC=0} then a \texttt{.AC} analysis is done instead.

All three data formats defined in the Touchstone standard, which are
real-imaginary (RI), magnitude-angle(MA) and magnitude(db)-angle (DB),
are supported.  The default is RI.  Other options can be selected via
the inclusion of the \texttt{DATAFORMAT=<val>} argument on the
\texttt{.LIN} line. Per the Touchstone standard, all angle values are
output in degrees.

The default filename for both Touchstone formats is
\texttt{<netlistName>.sNp} where N is the number of ``ports''
(\texttt{P} devices) specified in the netlist.  The output can be
redirected to another file with the \texttt{-o} command line option or
by using a \texttt{FILE=<filename>} argument on the \texttt{.LIN}
line.

The default output is S-parameters.  The \texttt{LINTYPE=<S|Y|Z>}
argument can be used on the \texttt{.LIN} line to select Y-parameter
or Z-parameter output instead.  (Note: The default filename is still
\texttt{<netlistName>.sNp} even if the output file contains
Y-parameter or Z-parameter output rather than S-parameter output.)
The \Xyce{} Reference Guide\ReferenceGuide{} has a complete listing of
which arguments, typically used on a \texttt{.PRINT} line also work on
a \texttt{.LIN} line.

\begin{table}[htbp]
  \caption{Output generated for .LIN analysis \label{LIN_Output_table}}
  \begin{tabularx}{\linewidth}{|p{2.75in}|Y|Y|}
    \rowcolor{XyceDarkBlue} \color{white}\textbf{Command} & \color{white}\textbf{Files} & \color{white}\textbf{Format} \\ \hline
\texttt{.LIN} & \emph{circuit-file}.sNp & S-parameter data in Touchstone 2 format. Data format is RI. \\ \hline
\texttt{.LIN FORMAT=TOUCHSTONE} & \emph{circuit-file}.sNp & S-parameter data in Touchstone 1 format. Data format is RI. \\ \hline
\texttt{.LIN FORMAT=TOUCHSTONE2} & \emph{circuit-file}.sNp & S-parameter data in Touchstone 2 format. Data format is RI. \\ \hline
\texttt{.LIN DATAFORMAT=MA} & \emph{circuit-file}.sNp & S-parameter data in Touchstone 2 format. Data format is MA. \\ \hline
\texttt{.LIN DATAFORMAT=DB} & \emph{circuit-file}.sNp & S-parameter data in Touchstone 2 format. Data format is DB. \\ \hline
\texttt{.LIN LINTYPE=Y} & \emph{circuit-file}.sNp & Y-parameter data in Touchstone 2 format. Data format is RI. \\ \hline
\texttt{.LIN LINTYPE=Z} & \emph{circuit-file}.sNp & Z-parameter data in Touchstone 2 format. Data format is RI. \\ \hline

\texttt{\emph{Xyce} -r raw-file-name} & NA & This will produce a parsing error. \\ \hline
\texttt{\emph{Xyce} -r raw-file-name -a} & NA & This will produce a parsing error. \\ \hline

\end{tabularx}
\end{table}

When a \texttt{.LIN} analysis is done then additional output variable
formats are available via the \texttt{.PRINT AC} line, where
\texttt{<index1>} and \texttt{<index2>} must both be greater than 0
and also both less than or equal to the number of ports in the
netlist:
\begin{XyceItemize}
\item \texttt{SR(<index1>,<index2>)} to output the real component of an S-parameter
\item \texttt{SI(<index1>,<index2>)} to output the imaginary component of an S-parameter
\item \texttt{SM(<index1>,<index2>)} to output the magnitude of an S-parameter
\item \texttt{SP(<index1>,<index2>)} to output the phase of an S-parameter in degrees
\item \texttt{SDB(<index1>,<index2>)} to output the magnitude of an S-parameter in decibels.
\item \texttt{YR(<index1>,<index2>)} to output the real component of a Y-parameter
\item \texttt{YI(<index1>,<index2>)} to output the imaginary component of a Y-parameter
\item \texttt{YM(<index1>,<index2>)} to output the magnitude of a Y-parameter
\item \texttt{YP(<index1>,<index2>)} to output the phase of a Y-parameter in degrees
\item \texttt{YDB(<index1>,<index2>)} to output the magnitude of a Y-parameter in decibels.
\item \texttt{ZR(<index1>,<index2>)} to output the real component of a Z-parameter
\item \texttt{ZI(<index1>,<index2>)} to output the imaginary component of a Z-parameter
\item \texttt{ZM(<index1>,<index2>)} to output the magnitude of a Z-parameter
\item \texttt{ZP(<index1>,<index2>)} to output the phase of a Z-parameter in degrees
\item \texttt{ZDB(<index1>,<index2>)} to output the magnitude of a Z-parameter in decibels.
\end{XyceItemize}

\Example{\\
\texttt{.print AC SR(1,1) YI(1,2) ZM(2,1)}
}

%%% Local Variables:
%%% mode: latex
%%% End:
% END of Xyce_UG_ch09.tex ************
