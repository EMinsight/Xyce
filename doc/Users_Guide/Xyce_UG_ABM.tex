% Sandia National Laboratories is a multimission laboratory managed and
% operated by National Technology & Engineering Solutions of Sandia, LLC, a
% wholly owned subsidiary of Honeywell International Inc., for the U.S.
% Department of Energy’s National Nuclear Security Administration under
% contract DE-NA0003525.

% Copyright 2002-2023 National Technology & Engineering Solutions of Sandia,
% LLC (NTESS).

%%-------------------------------------------------------------------------
%% Purpose        : Main LaTeX Xyce Users' Guide
%% Special Notes  : Graphic files (pdf format) work with pdflatex.  To use
%%                  LaTeX, we need to use postcript versions.  Not sure why.
%% Creator        : Scott A. Hutchinson, Computational Sciences, SNL
%% Creation Date  : {05/23/2002}
%%
%%-------------------------------------------------------------------------

\chapter{Analog Behavioral Modeling}
\index{behavioral model!analog behavioral modeling (ABM)}
\index{analog behavioral modeling (ABM)}
\label{Behavioral_Modeling}

\chapteroverview{Chapter Overview}
{
This chapter describes analog behavioral
modeling in \Xyce{}.  Sections include:
\begin{XyceItemize}
\item Section~\ref{Behavioral_Overview},{\em Overview of Analog Behavioral Modeling}
\item Section~\ref{ABM_Devices}, {\em Specifying ABM Devices}
\item Section~\ref{ABM_Guidance}, {\em Guidance for ABM Use}
\end{XyceItemize}
}

\section{Overview of Analog Behavioral Modeling (ABM)}
\label{Behavioral_Overview}

The analog behavioral modeling
\index{behavioral model!analog behavioral modeling (ABM)} 
\index{analog behavioral modeling (ABM)} 
capability of \Xyce{} provides for flexible descriptions of
electronic components or subsystems in terms of a transfer function or lookup table. In other
words, a mathematical relationship is used to model a circuit segment thereby removing
the need for component-by-component design information for those components or subsystems.

The \verb+B+\index{device!\verb+B+ (nonlinear dependent) source}
\index{device!behavioral} device, or nonlinear dependent source, is
the primary device used for analog behavioral modeling in \Xyce{}.  A
\verb+B+ device can serve as either a voltage or current source, and by using
expressions dependent on voltages and currents elsewhere in the
circuit the user can produce a wide range of behaviors.

\section{Specifying ABM Devices}
\label{ABM_Devices}
\index{device!specifying ABM devices}

ABM devices\index{device!\verb+B+ (nonlinear dependent) source} (\verb+B+
devices) are specified in a netlist the same way as other devices.  Customizing
the operational behavior of the device is achieved by defining an ABM
expression describing how inputs are transformed into outputs.

For example, the following pair of lines would provide exactly the same
behavior as a 10K resistor between nodes 1 and 2, and is
written to be a current source with current specified using Ohm's
law and the constant resistance value of 10K $\Omega$.

\begin{vquote}
.PARAM Res1=10K
Blinearres 1 2 I=\{(V(2)-V(1))/Res1\}
\end{vquote}

A nonlinear resistor could be specified similarly:
\begin{vquote}
.PARAM R1=0.15
.PARAM R2=6
.PARAM E2 = \{2*E1\}
.PARAM delr = \{R1-R0\}
.PARAM k1 = \{1/E1**2\}
.PARAM r2 = \{R0+sqrt(2)*delr\}

.FUNC Rreg1(a,b,c,d) \{a +(b-a)*c/d\}
.Func Rreg2(a,b,c,d,f) \{a+sqrt(2-b*(2*c-d)**2)*f\}

Bnlr 4 2 V = \{I(Vmon) * IF(
+ V(101) < E1, Rreg1(R0,R1,V(101),E1),
+ IF(
+ V(101) < E2, Rreg2(R0,k1,E1,V(101),delr), R2
+ )
+ )\}
\end{vquote}

In this example, \verb+Bnlr+ provides a voltage between nodes 4 and 2,
determined using Ohm's law with a resistance that is a function of the
voltage on node 101 and a number of parameters.  These two examples
demonstrate how the \verb+B+ source can be used either as a voltage
source (by specifying \verb+V={expression}+) or as a current source
(with \verb+I={expression}+).

NOTE: Unlike expressions used in parameters or function declarations,
expressions in the nonlinear dependent source may contain voltages and
currents from other parts of the circuit, or even explicit
time-dependent functions.  \Xyce{} evaluates these expressions when the
current or voltage through the ABM source is needed.  In contrast, expressions
used in parameters or function declarations are evaluated only once, prior
to the start of the circuit simulation.

\subsection{Additional constructs for use in ABM expressions}

ABM expressions follow the same rules as other expressions in a
netlist, with the additional ability to specify signals (node voltages
and voltage source currents) and explicitly time-dependent functions
in the expression.  In ABM expressions, refer to signals by
name. \Xyce{} recognizes the following constructs in ABM expressions:
\index{expressions!additional constructs for ABM modeling}
\index{expressions!time-dependent}
\begin{XyceItemize}
\item \texttt{V(<node name>)}
\item \texttt{V(<node name>,<node name>)} (the voltage difference between the first and second nodes)
\item \texttt{I(<voltage source name>)}
\item Reserved simulator variables, such as \texttt{TIME}, \texttt{TEMP}, \texttt{VT}, \texttt{FREQ} and \texttt{GMIN}
\item The constants, \texttt{PI} and \texttt{EXP}, which equal $\pi$ and $e$, respectively.
\item Lookup tables, which may be a function of time and other inputs
\item User-defined functions via \texttt{.func}
\item User-defined \texttt{.param} and \texttt{.global\_param} parameters.
\item Random operators such as \texttt{AGAUSS}.
\end{XyceItemize}

In a hierarchical circuit (a circuit with possibly nested levels of
subcircuits), voltage source names in an ABM expression must be the
name of a voltage source in the same subcircuit as the ABM device, or in a
subcircuit instantiated by that subcircuit.  Similarly, node names in an ABM 
expression must be the node names of one or more devices in the same subcircuit
as the ABM device, or in a subcircuit instantiated by that subcircuit. 

\subsection{Examples of Analog Behavioral Modeling}
\index{behavioral model!examples}

A variety of examples of legal usage of analog behavioral modeling is
probably the most effective means of demonstrating what is
allowed. The following netlist fragment shows the range of 
simple items allowed in ABM  expressions:

\begin{vquote}
\color{blue}* Current through B1 given as expression of voltage drop between 
* nodes 2 and 3 plus current through voltage source Vr4mon\color{black}
B1  1  0  I=\{V(2,3) + I(Vr4mon)\}
R4  2  0  10K
Vr4mon  2a 2  0V
\color{blue}* Voltage across device Em given as time-dependent expression \color{black}
Em  3  2a  VALUE=\{PAR3+1000*time\}
\color{blue}* Voltage across device B2 set to  current through device Em\color{black}
B2  2a  0  V=\{I(Em)\}
M3  Drain  6  0  NMOD
VdrainM3 DrainPrime Drain 0v
\color{blue}* Voltage across B3 is function of voltage on node two and 
* current through device VdrainM3\color{black}
B3  6  4  V=\{I(VdrainM3)+V(2)\}
\color{blue}* Voltage across device B4 is function of an internal node 
* named "5" of subcircuit instance X1\color{black}
X1 1 3 mysubcircuit
B4 4 5 V=\{V(X1:5)\}
\color{blue}* Current through device B5 taken from current through 
* internal device V4 of subcircuit instance X1\color{black}
B5 4 5 I=\{I(X1:V4)\}
\end{vquote}

The range of items that can be used in the current and voltage
parameters of a B (or E, F, G, or H) source is far greater than what
is allowed for expressions in other contexts. In particular, the use
of solution values (V(*), V(*,*), I(V*)) are prohibited in all other
expressions because they lead to unstable behavior if used
elsewhere beside ABM. Time-dependent expressions are allowed for some device
parameters, but this feature should be used with caution, as the
behavior of a non-ABM device cannot be guaranteed to be correct when its 
device parameters are not constant throughout a simulation run.
Frequency-dependent expressions are not supported for the current and voltage
parameters of a B (or E, F, G, or H) source.

\subsubsection{Behavioral modeling with tables}
\index{behavioral model!lookup table}
\index{expressions!lookup table}



Lookup tables and splines provide a means of
specifying a interpolated data in an expression.  A table expression is
specified with the keyword \verb+TABLE+ followed by an expression that
is evaluated as the independent variable of the function, followed by a
list of pairs of independent variable/dependent variable values.  For
example:

\Example{B1 1 0 V=\{TABLE \{time\} = (0, 0) (1, 2) (2, 4) (3, 6)\}}

An equivalent example uses the table function, which has a simpler syntax, but
 may be hard to read for long tables:

\Example{B1 1 0 V=\{TABLE(time, 0, 0, 1, 2, 2, 4, 3, 6)\}}

The previous two examples will produce a voltage source (B1) whose voltage
is a simple linear function of time.  At $t=0$ the voltage is $0$ volts and 
at time $t=1s$ the voltage is $2$ volts.  Similarly, the voltage will
be $4$ volts at $t=2s$ and $6$ volts at $t=3s$. Linear interpolation is
used at times in between those tabulated values.

It is also possible to create ABM sources from files of time-value pairs by providing the name of the file containing the pairs between quotation marks \texttt{(")}:

\Example{Bfile 1 0 V=tablefile("myfile")}
\Example{Bfile 1 0 V=\{tablefile("myfile")\}}
\Example{Bfile 1 0 V=\{table("myfile")\}}

The file provided must have one time-voltage pair per line, separated
by spaces.  Comma-separated files are not supported, and will not be
parsed correctly.  If the file ``myfile'' contains the following data:
\begin{verbatim}
0 0
1 2
2 4
3 6
\end{verbatim}
then the ``Bfile'' example above will be identical to either of the
``B1'' examples given above.  The quoted-file sytax is in fact converted
internally to precisely the same TABLE format as the first B1 example,
with an independent variable of TIME and the given time-value pairs
inserted.\footnote{The use of a B source in this manner is similar to using the \texttt{FILE} option to the piecewise linear (\texttt{PWL}) voltage source as documented in the \Xyce{} Reference Guide\ReferenceGuide, but unlike the PWL source, the file-based table function does {\em NOT\/} support reading comma-separated files.}

The independent variable of the table source does not have to be a
simple expression.  In the example shown below, the independent variable is
a function of voltages and currents throughout the circuit.

\Example{Bcomplicated 1 0 V=\{TABLE \{V(5)-V(3)/4+I(V6)*Res1\} = (0, 0) (1, 2) (2, 4) (3, 6)\}}

There is also a version of the table capability which can run significantly faster 
for really large tables.  That capability is the \texttt{fasttable} capability, 
which is available in the \Xyce{} expression library.
It has the same format as \texttt{table}, and behaves very similarly.  However, 
whereas \texttt{table} will apply time integration breakpoints at every entry in 
the table, the \texttt{fasttable} capability will only apply breakpoints at 
the first and last time point in the table.

\subsubsection{Behavioral modeling with splines}
\index{behavioral model!spline}
\index{expressions!spline}
\index{behavioral model!interpolation}
\index{expressions!interpolation}
\index{behavioral model!BLI}
\index{expressions!BLI}
\index{behavioral model!Barycentric Lagrange Interpolation}
\index{expressions!Barycentric Lagrange Interpolation}

An alternative to \texttt{table} is to use one of the more advanced
interpolator functions that are available in the \Xyce{} expression library.   
All of these methods use higher-order 
interpolation methods and (in theory) will produce a smoother result.  As they 
tend to be very smooth, the interpolators do not produce any time integration 
breakpoints, except for the first and last time point.  In this regard 
(breakpoints) they behave the same as \texttt{fasttable}.
The spline options include \texttt{spline}, \texttt{cubic}, 
\texttt{akima}~\cite{10.1145/321607.321609}, and \texttt{wodicka}~\cite{Engeln1996}. 
Another high-order interpolator is \texttt{BLI}, which is for Barycentric Lagrange 
Interpolation~\cite{Berrut_barycentriclagrange}.  All of these interpolators use the same 
syntax as the \texttt{table} function in the expression library.  For example:

\Example{Bfile 1 0 V=\{spline("myfile")\}}
\Example{Bfile 1 0 V=\{bli("myfile")\}}

Note, this is useful for data pulses that resemble smooth functions.   
If the intended data has a lot of sharp discontinuities (such as a square wave), 
higher orders of interpolation (and the removal of breakpoitns) are not a good choice.

\subsubsection{Automatically sparsifying tabular data}
\index{behavioral model!downsampling}
\index{behavioral model!sparsifying}

Interpolated ABM models, based on tabular data can optionally have the size of their 
data set sparsified or reduced.  To enable this feature, the \texttt{TABLE} or \texttt{SPLINE} operator requires an additional argument specifying the number of points.

\Example{Bfile 1 0 V=\{table("myfile",100)\}}

In the above example, the number of points used internally for the 
\texttt{Bfile} device is 
automatically reduced to 100.  This reduction is a gradient-based reduction, 
so the resulting dataset will cluster larger numbers of points in regions of 
high gradients and small numbers of points where the dataset is nearly flat.
The reduction requires an internal inteprolation, and this internal interpolation 
is performed using splines, even if the interpolation used by the expression 
is \texttt{TABLE}, which is uses piecewise linear interpolation.

The motivation for using this feature is to improve simulation performance.  If a source is based on a really large dataset with very finely spaced points, the \Xyce{} time integrator must enforce breakpoints at every point.  This forces \Xyce{} to take a large number of timesteps and this can slow the calculations down considerably.  By sparsifying the data, the number of breakpoints can be dramatically reduced.

Note, this main useful for data pulses that resemble smooth functions.   
If the intended data has a lot of sharp discontinuities (such as a square wave), 
this feature is not recommended.

\subsection{Alternate behavioral modeling sources}

In addition to the primary nonlinear dependent source, the \verb+B+
source, \Xyce{} also supports the PSpice extensions to the standard
Spice voltage- and current-controlled sources, the \verb+E+, \verb+F+,
\verb+G+, and \verb+H+ sources.  \Xyce{} provides these sources for
PSpice compatibility, and converts them internally into equivalent
\verb+B+ sources.  The \Xyce{} Reference Guide\ReferenceGuide{} netlist
reference chapter provides the syntax of these compatibility devices.

\section{Guidance for ABM Use}
\label{ABM_Guidance}

\subsection{ABM devices add equations to the system of equations used by the solver}

As \Xyce{} solves a complex nonlinear set of equations at each time
step, it is important to remember this system of equations is solved
iteratively to obtain a converged solution. Specifying an ABM device
in a \Xyce{} netlist adds one or more equations to the nonlinear
problem that \Xyce{} must solve.

When the nonlinear solver has converged, the expression given in the
ABM device will be satisfied to within a solver tolerance. However,
during the course of the iterative solve, the unconverged values of
nodal voltages and currents, which are often inputs and outputs of ABM
devices, are not guaranteed to be solutions to the system of
equations.

During this preconverged phase, solution variables are not guaranteed
to have physically reasonable values. They could, for example,
temporarily have the wrong sign. Only at the end of a successful
nonlinear iterative solve are the solution variables consistent, legal
values. This convergence behavior motivates the caveats on ABM usage given 
in the next subsection.   

\subsection{Expressions used in ABM devices must be valid for any possible input}

While ABM devices look temptingly like calculators, it is potentially
dangerous to use them as such. The previous subsection stated that
during the nonlinear solution of each timestep equations, nodal
voltages and currents are usually not solutions to the full set of
equations, and often violate Kirchhoff's laws. Only at the end of the
nonlinear solution are all the constraints on voltages and currents
satisfied. This has some important consequences to the user of ABM
devices.

All expressions involving nodal voltages and currents used in ABM
devices should be valid for any possible value they might see --- even
those that appear to be physically meaningless and those that a
knowledgeable user might never expect to see in the real circuit. This
is particularly important when using square roots or exponentiating to
a fractional power. For example, consider the following netlist
fragment:


\begin{vquote}
*...other parts of more complex circuit deleted...
\color{blue}* potentially bad usage of ABM device \color{black}
Vexample 1 0 5V
d1 1 0 diode_model
\color{red}B1 2 0 V=\{sqrt(v(1))\}\color{black}
r1 2 0 10k
*...other parts of more complex circuit deleted...
\end{vquote}

This example demonstrates a potentially dangerous usage. It is
assumed, because node 1 is connected to a 5V DC source, that the argument
of the square root function is always positive. However, it could be
the case that during the nonlinear solution of the full circuit that an
unconverged value of node 1 might be negative. Tracking down mistakes
such as this can be difficult, as on most
platforms B1 will result in a ``Not a Number'' value for the nodal voltage
of node 2 but the program will not crash. This frequently results in
inexplicable ``timestep too small'' errors.

Although such things can be avoided by protecting the arguments of
functions with a limited domain, care must be taken when doing this. One
obvious way to protect the example circuit fragment would be to take
the absolute value of V(1) before calling the square root (sqrt)
function:

\begin{vquote}
*...other parts of more complex circuit deleted...
\color{blue}* safer usage of ABM device \color{black}
Vexample 1 0 5V
d1 1 0 diode_model
B1 2 0 V=\{sqrt(abs(v(1)))\}
r1 2 0 10k
*...other parts of more complex circuit deleted...
\end{vquote}

There are many other ways to protect the square root function from
negative arguments, such as by using the maximum of zero and V(1).
Some alternatives might be more appropriate than others in different
contexts.

Note, though, that it would be a mistake to attempt to generate the absolute 
value as shown here: 
\begin{vquote}
*...other parts of more complex circuit deleted...
\color{blue}* really bad misuse of ABM device \color{black}
Vexample 1 0 5V
d1 1 0 diode_model
\color{red}B2 3 0 V=\{abs(v(1))\} ; watch out!
B1 2 0 V=\{sqrt(v(3))\}; just as bad as first example!\color{black}
r1 2 0 10k
*...other parts of more complex circuit deleted...
\end{vquote}

There are two things wrong with this example --- first, node 3 is floating
and this alone could lead to convergence problems. Second, by adding
the second ABM device one has merely created an equation whose
solution is that node 3 contains the absolute value of the voltage on
node 1. However, until convergence is reached there is no guarantee
node 3 will be precisely the absolute value of V(3), nor is it
guaranteed that node 3 will have a positive voltage. To re-iterate, nodes have 
values that are solutions to the set of equations created by the netlist only at
convergence.

\subsection{Infinite slope transitions can cause convergence problems}

It is possible for a user to specify expressions that could have
infinite-slope transitions with B-, E-, F-, G- and H-sources.  This
can lead to ``timestep too small'' errors when \Xyce{} reaches the
transition point.  For example, inclusion of the following B-source
expression in a circuit can cause the simulation to fail when
\texttt{V(IN)=3.5}:
\begin{alltt} Bcrtl OUTA 0 V=\{ IF( (V(IN) > 3.5), 5, 0 ) \} \end{alltt}

Infinite-slope transitions in expressions dependent only on the
\texttt{time} variable are a special case, because \Xyce{} can detect
that they are going to happen, and set a ``breakpoint'' to capture
them.  Infinite-slope transitions depending on other solution
variables, however, cannot be predicted in advance, and will cause the
time integrator to scale back the timestep repeatedly in an attempt to
capture the feature until the timestep is too small to continue. One
solution to this problem is to modify the expression to allow a
continuous transition. For example, the above expression could be
modified to:
\begin{alltt} Bcrtl OUTA 0 V=\{IF( (V(IN) > 3.4), IF( (V(IN) > 3.5), 5, 50*(V(IN)-3.4) ), 0 )\}\end{alltt}

However, this can become complicated with multiple inputs. The other
solution is to specify device options or instance parameters to allow
smooth transitions. The parameter \texttt{smoothbsrc} enables the
smooth transitions. This is done by adding a RC network to the output
of B sources. For example,

\begin{alltt} Bcrtl OUTA 0 V=\{ IF( (V(IN) > 3.5), 5, 0 ) \} smoothbsrc=1 \end{alltt}

\begin{alltt} .options device  smoothbsrc=1 \end{alltt}

The smoothness of the transition can be controlled by specifying the
rc constant of the RC network. For example,

\begin{alltt} Bcrtl OUTA 0 V=\{ IF( (V(IN) > 3.5), 5, 0 ) \} smoothbsrc=1   
 + rcconst = 1e-10 \end{alltt}

     
\begin{alltt} .options device  smoothbsrc=1 rcconst = 1e-10 \end{alltt}

Note that this smoothed ABM only applies to voltage sources, such as voltage
B sources and E sources. The voltage behavioral source supports two instance
parameters \texttt{smoothbsrc} and \texttt{rcconst}. Parameters may be provided
as space  separated \texttt{<parameter>=<value>} specifications as needed. The
default value for \texttt{smoothbsrc} is 0 and the default for \texttt{rcconst}
is 1e-9.

If the device options are specified, they apply to all voltage ABM devices in
the circuit. If the device instance parameters are specified, they apply to that
device. When both the device options and instance parameters are specified, the
device instance parameters take precedence over the device options.


\subsection{ABM devices should not be used purely for output postprocessing}

Users sometimes use ABM devices to provide output postprocessing. For
example, if a user was interested in the absolute value, or the log of
an output voltage, then that user might create an ABM circuit element to
calculate the desired output value.

Using ABM sources in this manner is bad practice though. By creating a
circuit element whose only purpose is postprocessing, \Xyce{} is forced
to include it and the corresponding nonlinear solve in the circuit,
which can cause unnecessary solver problems. If postprocessing is the
goal, it is much better to use expressions directly on the
\verb|.PRINT| line.

An example of a ``bad use'' of ABM sources can be found in the
following code fragment:

\begin{vquote}
\color{blue}* Bad example
\color{XyceRed}B1 test1 0 V = \{(abs(I(VMON)))*1.0e-10\} \color{black}
VIN 1 0 DC 5V
R1 1 2 2K
D1 3 0 DMOD
VMON 2 3 0
.MODEL DMOD D (IS=100FA)
.DC VIN 5 5 1
.PRINT DC I(VMON) V(3) V(test1)
\end{vquote}

Although the source \texttt{B1} provides a 
postprocessing output, it doesn't play a functional role in the 
circuit and \Xyce{} would still be forced to include \texttt{B1} in 
the problem it is attempting to solve.  

A better solution to the previous problem is given here, where the
post-processing is done on a \verb|.PRINT| line:

\begin{vquote}
\color{blue}* Good example \color{black}
VIN 1 0 DC 5V
R1 1 2 2K
D1 3 0 DMOD
VMON 2 3 0
.MODEL DMOD D (IS=100FA)
.DC VIN 5 5 1
.PRINT DC I(VMON) V(3) \color{XyceRed} \{(abs(I(VMON)))*1.0e-10\} \color{black}
\end{vquote}

Section~\ref{Output_Control} and the \Xyce{} Reference
Guide\ReferenceGuide provide a more detailed explanation of how to use
expressions in the \texttt{.PRINT} line.

\subsection{ABM devices in frequency domain analysis}

ABM devices can be used in both types of frequency domain analysis, AC and HB.  
However, there is one important limitation.   They can only be used in this type of analysis 
if the source is not an independent source.  In other words, they cannot be time-dependent.
ABM sources that only depend on circuit solution variables (such as voltage node values) 
will work in both AC and HB.

%%% Local Variables:
%%% mode: latex
%%% End:

%%% END of Xyce_UG_ABM.tex ************
