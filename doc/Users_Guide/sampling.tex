% Sandia National Laboratories is a multimission laboratory managed and
% operated by National Technology & Engineering Solutions of Sandia, LLC, a
% wholly owned subsidiary of Honeywell International Inc., for the U.S.
% Department of Energy’s National Nuclear Security Administration under
% contract DE-NA0003525.

% Copyright 2002-2023 National Technology & Engineering Solutions of Sandia,
% LLC (NTESS).

% -------------------------------------------------------------------------
% Sampling Analysis Section ---------------------------------------------------
% -------------------------------------------------------------------------
\clearpage
\section{Random Sampling Analysis}
\label{SAMPLING_Analysis}
\label{sampling_Overview}
\index{analysis!Sampling} \index{Sampling analysis} \index{Random Sampling}
\index{\texttt{.SAMPLING}}
\index{\texttt{.EMBEDDEDSAMPLING}}
\Xyce{} supports two different styles of random sampling analysis.  One is specified 
with the \verb|.SAMPLING| 
command, and the other is specified with the \verb|.EMBEDDEDSAMPLING| commmand.  
These are relatively new capabilities and are still under development.  

The \verb|.SAMPLING| command causes \Xyce{} to execute the primary analysis (\verb|.DC|,
\verb|.AC|, \verb|.TRAN|, etc.) inside a loop over randomly distributed parameters.  
The user specifies a list of random parameters, their distributions, distribution 
parameters (such as mean and standard deviation), and the total number of 
sample points.  The primary analysis is then executed sequentially for each sample point.
The \verb|.SAMPLING| capability uses a lot of the same code base as \verb|.STEP|, 
so most of the guidance for \verb|.STEP|, including legal parameters, output 
file formats, supported analyses types, etc., also applies to \verb|.SAMPLING|.

The \verb|.EMBEDDEDSAMPLING| command also causes \Xyce{} to create a loop over 
randomly distributed parameters, but unlike \verb|.SAMPLING|, this loop is 
implemented inside the solver loop rather than outside of it.  As such, it 
creates a block linear system, in which each matrix block has the structure 
of the original matrix, and the number of blocks is equal to the number of sample points.
As such, all the parameters are propagated simultaneously, rather than sequentially.
Unlike \verb|.SAMPLING|, which can be applied to all types of \Xyce{} analysis, 
\verb|.EMBEDDEDSAMPLING| can only be applied to \verb|.DC| and \verb|.TRAN|.

Both types of sampling methods can be applied to a variety of inputs and outputs.
For inputs,
\Xyce{}  can use \verb|.SAMPLING| or \verb|.EMBEDDEDSAMPLING| to modify 
any device instance or device
model parameter, as well as the circuit temperature.  It can also be used 
to sweep over any user-defined parameter (either \texttt{.GLOBAL\_PARAM} 
or \texttt{.PARAM} ), as long as that parameter is  defined in the 
global scope.

For outputs, both \verb|.SAMPLING| and \verb|.EMBEDDEDSAMPLING| 
can apply statistical analysis (computing moments such as mean and standard deviation) 
to most outputs that appear on the \verb|.PRINT| line.
However, the two analysis types have some differences.    \verb|.SAMPLING| will only 
apply statistical analysis at the end of the calculation, while \verb|.EMBEDDEDSAMPLING| 
can apply statistical analysis at each time step.  Relatedly, 
\verb|.SAMPLING| can also be applied to any \verb|.MEASURE|, but 
\verb|.EMBEDDEDSAMPLING| lacks this capability. 

\subsection{.SAMPLING and .EMBEDDEDSAMPLING Statements}
\label{sampling_statement}
Sampling analysis may be specified by simply adding a \verb|.SAMPLING| 
or a \verb|.EMBEDDEDSAMPLING| line to a netlist.    The format for the two commands is nearly 
identical.

Sampling analysis such as \verb|.SAMPLING| or \verb|.EMBEDDEDSAMPLING| is not considered a ``primary'' analysis type.
This means that such a command by itself is not an 
adequate analysis specification, as it merely specifies an additional parametric loop to augment the
primary analysis.  A standard analysis line, specifying \verb|.TRAN|, \verb|.AC|, \verb|.HB|, 
and \verb|.DC| analysis, is still required.

Examples of typical \verb|.SAMPLING| lines:

\Example{\\
\texttt{\\
.SAMPLING  \\
+ param=M1:L, M1:W \\
+ type=uniform,uniform \\
+ lower\_bounds=5u,5u \\
+ upper\_bounds=7u,7u \\
}
\texttt{ \\
.SAMPLING \\
+ param=R1,c1 \\
+ type=normal,normal \\
+ means=3K,1uF \\
+ std\_deviations=1K,0.2uF \\
}

}

In the first example, \verb|M1:L| and \verb|M1:W| 
are the names of the
parameters (in this instance, the length and width parameters of the MOSFET
\texttt{M1}), \verb|5u| is the minimum value of both parameters and
\verb|7u| is the maximum value of both parameters.  For both parameters, 
the distribution (specified by \texttt{type}) is uniform.
In the second example, the two parameters are \verb|R1| and \verb|c1|.   In this 
case, it isn't necessary to specify the specific parameter of \verb|R1| or \verb|c1| 
as resistors and capacitors have simple default parameters (see section~\ref{sampling_SpecialCases}).  
The distributions of both parameters are normal, and the means and standard deviations 
are specified for both in comma-separated lists.

For any list of parameters, all the required fields must be provided in comma-separated 
lists.  \Xyce{} will throw an error if the length of any of the lists is different 
than that of the parameter list.

\subsection{.options SAMPLES and EMBEDDEDSAMPLES Statements}
\label{options_samples_statement}

To specify the number of samples, use the netlist command \texttt{.options 
SAMPLES numsamples=<value>}.   For sampling analysis to work correctly, this parameter is required.
Other sampling details, including the sampling type, the covariance matrix, 
the statistical outputs and the random seed can be specified using the 
same \verb|.options SAMPLES| netlist command.  Some examples of this command are:

\Example{\\
\texttt{ \\
.options samples numsamples=1000 \\
}
\texttt{ \\
.options samples numsamples=30 \\
+ covmatrix=1e6,1.0e-3,1.0e-3,4e-14 \\
+ OUTPUTS={R1:R},{C1:C} \\
+ MEASURES=maxSine \\
+ SAMPLE\_TYPE=LHS \\
+ SEED=743190581 \\
}
}
The random seed can be set either in the \texttt{.options SAMPLES} statement, or 
it can be set from the command line using the \texttt{-randseed} option.  If it 
is not set either way, then the random seed is generated internally.

In the example there are two different types of outputs specified.  One is using 
the keyword \texttt{OUTPUTS}, and this is used to specify a comma-separated list 
of outputs corresponding to typical \texttt{.PRINT} line outputs.  Solution 
variables, user defined parameters, and expressions depending on both are typical.  

The other type of output is specified using the keyword \texttt{MEASURES}, and 
this is used to specify a comma-separated list of netlist \texttt{.MEASURE} commands.
This type of output is only available to \texttt{.SAMPLING} and not to \texttt{.EMBEDDEDSAMPLING}.

The example sets the \texttt{SAMPLE\_TYPE=LHS}, for Latin Hypercube Sampling~\cite{HELTON200323}.  
The other options is \texttt{SAMPLE\_TYPE=MC} for Monte Carlo~\cite{Fishman1996}.  
LHS is the default option for this parameter, and will generally be a better choice.

\subsection{Specifying Uncertain Inputs}

\subsubsection{Sampling a Device Instance Parameter}
\label{sampling_InstanceParam}
Figure~\ref{Sampling_Netlist_1} provides a simple application of \verb|.SAMPLING| 
to a device instance parameter.  The circuit is a simple voltage divider driven 
by a oscillating voltage source.  The randomly sampled parameter is the resistance 
of \texttt{R1}, which is given a normal distribution with a specified mean and 
standard deviation.

As the circuit is executed multiple times, an output file is generated based on 
the \verb|.PRINT| command.  The \verb|.PRINT| statement is still used in the same 
way that it would be in a non-sampling analysis.  However, the output file 
contains the concatenated output of each \verb|.SAMPLING| step.  
Section~\ref{sampling_output_files} provides more details of how \texttt{.SAMPLING} 
changes \texttt{.PRINT} output files.

\Xyce{} computes statistical outputs for quantities specified by \texttt{OUTPUTS} 
and \texttt{MEASURES} on the \texttt{.options SAMPLES} line.  For these outputs,
moments such as mean and variance are computed and sent to standard output.  More 
details of this type of output, as well as examples, are given in section~\ref{statistical_outputs}.
\begin{figure}[htbp]
  \fontsize{9pt}{10pt}
\begin{centering}
\shadowbox{
\begin{minipage}{0.8\textwidth}
\begin{vquote}
\color{blue}Voltage Divider with sampling\color{black}
R2  1  0  7k
R1 1 2 3k
VS1  2  0  SIN(0 1.0e3 1KHZ 0 0)

.tran 0.1ms 1ms
.meas tran maxSine max V(1) 

\color{XyceRed}.SAMPLING param=R1:R 
+ type=normal means=1K std\_deviations=0.1K\color{black}

\color{XyceRed}.options SAMPLES numsamples=1000 OUTPUTS=\{R1:R\} 
+ MEASURES=maxSine SAMPLE\_TYPE=MC stdoutput=true\color{black} 

\end{vquote}
\end{minipage}
}
\caption{Voltage divider circuit netlist with sampling of a device instance parameter.
Sampling commands are in \color{XyceRed}red\color{black}.
\label{Sampling_Netlist_1}}
\end{centering}
\end{figure}

\subsubsection{Sampling a Device Model Parameter}
\label{sampling_ModelParam}

Sampling a model parameter can be done in an identical manner to an instance parameter.  
Figure~\ref{Sampling_Netlist_2} provides a simple application of \verb|.SAMPLING| to a 
device model and a device instance.  This is the same diode clipper circuit as was used 
in the Transient Analysis section (see section \ref{Transient_Analysis_Sec}), except that 
several lines of sampling commands (in red) have been added.  

The sampling commands will cause \Xyce{} to sample the model parameter, \verb|D1N3940:IS| as well as the 
resistance of the resistor, R4. Both of the parameters are randomly sampled using uniform distributions 
with specified lower and upper bounds.
\begin{figure}[htbp]
  \fontsize{9pt}{10pt}
\begin{centering}
\shadowbox{
\begin{minipage}{0.8\textwidth}
\begin{vquote}
\color{blue}Transient Diode Clipper with Sampling Analysis\color{black}

\color{blue}* Primary Analysis Command\color{black}
.tran 2ns 0.5ms
\color{blue}* Output\color{black}
.print tran format=tecplot V(3) V(2) V(4)
.meas tran maxSine max V(4) 

\color{XyceRed}* Sampling statements 
.SAMPLING
+ param=R4:R,D1N3940:IS
+ type=uniform,uniform
+ lower\_bounds=3.0k,2.0e-10
+ upper\_bounds=15.0k,6.0e-10

.options SAMPLES numsamples=10 SAMPLE\_TYPE=LHS
+ OUTPUTS={R4:R},{D1N3940:IS} MEASURES=maxSine stdoutput=true\color{black}
\color{blue}* Voltage Sources\color{black}
VCC 1 0 5V
VIN 3 0 SIN(0V 10V 1kHz)
\color{blue}* Diodes\color{black}
D1 2 1 D1N3940
D2 0 2 D1N3940
\color{blue}* Resistors\color{black}
R1 2 3 1K
R2 1 2 3.3K
R3 2 0 3.3K
R4 4 0 5.6K
\color{blue}* Capacitor\color{black}
C1 2 4 0.47u
.MODEL D1N3940 D(
+ IS=4E-10 RS=.105 N=1.48 TT=8E-7
+ CJO=1.95E-11 VJ=.4 M=.38 EG=1.36
+ XTI=-8 KF=0 AF=1 FC=.9 BV=600 IBV=1E-4)
.END
\end{vquote}
\end{minipage}
}
\caption{Diode clipper circuit netlist with sampling analysis.  This example uses a model parameter as one of the sampling parameters.
Sampling commands are in \color{XyceRed}red\color{black}.
  \label{Sampling_Netlist_2}}
\end{centering}
\end{figure}

\subsubsection{Sampling a User-Defined Parameter}
\label{sampling_GlobalParam}

Sampling can be applied to user-defined parameters, as long as they are declared in the top level netlist (global scope).  
It cannot be applied to user-defined parameters which are defined in a subcircuit.  An example of such usage, with a global parameter, is given 
in figure~\ref{Sampling_Netlist_3}. This circuit identical to the circuit in 
figure~\ref{Sampling_Netlist_1}, except that it uses global parameters.  
If the global parameters in this netlist are changed to normal parameters (\texttt{.PARAM}), this circuit will also work.
\begin{figure}[htbp]
  \fontsize{9pt}{10pt}
\begin{centering}
\shadowbox{
\begin{minipage}{0.8\textwidth}
\begin{vquote}
\color{blue}Voltage Divider, sampling global params\color{black}
.global\_param Vmax=1000.0
.global\_param testNorm=1.5k
.global\_param R1value=\{testNorm*2.0\}

R2  1  0  7k
R1 1 2 \{R1value\}
VS1  2  0  SIN(0 \{Vmax\} 1KHZ 0 0)
.tran 0.1ms 1ms
.meas tran maxSine max V(1) 

\color{XyceRed}.SAMPLING param=testNorm
+ type=normal means=0.5K std\_deviations=0.05K\color{black}

\color{XyceRed}.options SAMPLES numsamples=1000 SAMPLE\_TYPE=MC
+ OUTPUTS=\{R1:R\} MEASURES=maxSine\color{black}

\end{vquote}
\end{minipage}
}
\caption{Voltage divider circuit netlist with sampling of a global parameter.
  Sampling commands are in \color{XyceRed}red\color{black}.  This circuit will also work if the \texttt{.global\_param} statements are changed to \texttt{.param}.
\label{Sampling_Netlist_3}}
\end{centering}
\end{figure}

\subsubsection{Sampling over Temperature}
\label{sampling_Temperature}

It is also possible to sample temperature.  To do so, simply 
specify \verb|temp| as the parameter name.  It will work in the 
same manner as \verb|.SAMPLING| when applied to model and instance 
parameters.

\subsubsection{Special cases: Sampling Independent Sources, Resistors, Capacitors and Inductors}
\label{sampling_SpecialCases}

For some devices, there is generally only one parameter that one would
want to sample.  For example, a linear resistor's only parameter of
interest is resistance, R.  Similarly, for a DC voltage or current
source, one is usually only interested in the magnitude of the source, DCV0.
Finally, linear capacitors generally only have capacitance, C, as a 
parameter of interest, while inductors generally only have inductance, L,
as a parameter of interest.  

For these simple devices, it is not necessary to specify both the
parameter and device on the \texttt{.SAMPLING} line: only the device name
is strictly required, as these devices have default
parameters that are assumed if no parameter name is given explicitly.

Examples of usage are given below.  The first two lines are equivalent
--- in the first line, the resistance parameter of \texttt{R4} is
named explicitly, and in the second line the resistance parameter is
implicit. A similar example is then shown for the DC voltage of the 
voltage source \texttt{VCC}.  In the remaining lines, parameter names are all 
implicit, and the default parameters of the associated devices are used.

\Example{ \\
          \fontsize{10pt}{11pt}
\begin{tabular}{lllll}
  \texttt{.SAMPLING param=R4:R,C1:C} &\texttt{type=normal,normal}  &\texttt{means=3k,1u}  &\texttt{std\_deviations=1k,0.1u} \\ 
  \texttt{.SAMPLING param=R4,C1} &\texttt{type=normal,normal}  &\texttt{means=3k,1u}  &\texttt{std\_deviations=1k,0.1u} \\ 
  \texttt{.SAMPLING VCC:DCV0}&\texttt{type=uniform}  &\texttt{means=6.0}  &\texttt{std\_deviations=1.0 } \\  
  \texttt{.SAMPLING VCC}&\texttt{type=uniform}  &\texttt{means=6.0}  &\texttt{std\_deviations=1.0 } \\ 
  \texttt{.SAMPLING param=C1} &\texttt{type=normal}  &\texttt{means=0.5u}  &\texttt{std\_deviations=0.05u} \\ 
  \texttt{.SAMPLING param=L1} &\texttt{type=normal}  &\texttt{means=0.5m}  &\texttt{std\_deviations=0.01m} \\ 
\end{tabular}
        }

Independent sources require further explanation.  Their default
parameter, DCV0, only applies to \texttt{.DC} analyses.  They do not have
default parameters for their transient forms, such as \texttt{SIN}
or \texttt{PULSE}.

\subsubsection{Sampling using random operators from expressions}

The \Xyce{} expression library supports random operators in expressions, 
including \texttt{AGAUSS}, \texttt{GAUSS}, \texttt{AUNIF}, \texttt{UNIF} and \texttt{RAND}.
\Xyce{} \texttt{.SAMPLING} and \texttt{.EMBEDDEDSAMPLING} analyses are able to use these, but they are not
connected by default.  In order to use random expression operators with \texttt{.SAMPLING}, 
the following commend must be present in the netlist:

\Example{\\
\texttt{\\
.SAMPLING  useExpr=true
}
}
This capability is potentially very powerful, as the random operators can 
act as operators within complex expressions.  
As expressions can be applied to \texttt{.param}, \texttt{.global\_param}, 
device instance parameters  and device model parameters, this capability 
naturally can be applied to any of these types of input parameters.
A simple example of this feature, applied to several instance parameters, 
can be seen in figure~\ref{Sampling_Netlist_4}.  Other examples of this capability can be found 
in the netlists given in figures~\ref{regressionPCE_Netlist1},~\ref{NISP_netlist1} 
and~\ref{NISP_gilbert_netlist}.

This feature cannot be used in the same netlist as the 
specification described in section~\ref{sampling_statement}.  If 
\texttt{useExpr=true}, \Xyce{} will ignore the comma-separated lists such as 
\texttt{param}.  Likewise, if \texttt{useExpr=false} (the default), then 
\Xyce{} will not randomly sample any random operators present in the circuit.  
It will only sample the parameters listed by comma-separated lists.

\begin{figure}[htbp]
  \fontsize{9pt}{10pt}
\begin{centering}
\shadowbox{
\begin{minipage}{0.8\textwidth}
\begin{vquote}
\color{blue}sampling example with random expression operators \color{black}
R4 b 0 \{agauss(4k,0.4k,1)\}
R3 a b \{agauss(1k,0.1k,1)\}
R2 1 a \{agauss(2k,0.2k,1)\}
R1 1 2 \{agauss(3k,0.1k,1)\}
v1 2 0 10V

.dc v1 10 10 1
.print dc v(1)

.SAMPLING 
+ useExpr=true

.options SAMPLES numsamples=10
+ outputs=\{v(1)\}
+ sample\_type=lhs
+ stdoutput=true
.end

\end{vquote}
\end{minipage}
}
\caption{Voltage divider circuit netlist using random expression operators.
\label{Sampling_Netlist_4}}
\end{centering}
\end{figure}


\subsubsection{Local and global variation}

Similar to other simulators, \Xyce{} uses the following convention for 
local and global variations sampling and other uncertainty quantification 
methods.  This is based on how many layers of \texttt{.param} indirection 
sit between a device parameter and the random operator.  If there is only a single 
layer, then the random operator is treated as a local variation, and every
affected device parameter will receive a unique random number at each sample point.
If there is more than one layer, then it is instead treated as global, meaning 
that each affected device parameter will receive the same random number from the operator.

For example, the following netlist fragment will result in local variation, 
where the resistors $r1$, $r2$ and $r3$ each get unique random numbers.
\Example{\\
\texttt{\\
.param resval=aunif(1000,400)  \\
r1 1 2 resval \\
r2 2 3 resval \\
r3 3 4 resval 
}
}
In contrast, the following netlist fragment will result in global variation, 
where the resistors will all get the same random number.
\Example{\\
\texttt{\\
.param globalval=aunif(1000,400)  \\
.param resval=globalval \\
r1 1 2 resval \\
r2 2 3 resval \\
r3 3 4 resval 
}
}

\subsection{Output files}
\label{sampling_output_files}

The two types of sampling produce different types of output files.  This is 
related to the fact that for \texttt{.SAMPLING} the calculations for each sample 
point are performed sequentially, while for \texttt{.EMBEDDEDSAMPLING} the 
calculations for each sample point are performed simultaneously.  As such, 
for \texttt{.SAMPLING} statistical moments can only be computed at the end 
of the calculation, once all the sample points are avaible.  In contrast, 
for \texttt{.EMBEDDEDSAMPLING}, statistical moments can be computed throughout 
the calculation, as all the sample points are available for each step.

\subsubsection{\texttt{.SAMPLING} output files}
\index{output!\texttt{.SAMPLING}}
Users can think of \texttt{.SAMPLING} simulations as being very similar 
to \texttt{.STEP} and thus comprised of a sequential set of distinct executions
of the same circuit netlist. The main difference is the source of the specific sample points.

Similar to \texttt{.STEP}, the output data can be requested by a standard \texttt{.PRINT} command, for the underlying primary analsis.
For example, if applying \texttt{.SAMPLING} to a transient analysis, then the appropriate \texttt{.PRINT} command will be \texttt{.PRINT TRAN}.
The output data will go to a single file, with each sample point in a different section of the output file.
If using the default \texttt{.prn} file format (\texttt{format=std}), the 
resulting \texttt{.SAMPLING} simulation output file will be a simple concatenation of
each step's underlying analysis output.
If using \texttt{format=probe}, the data for each execution of the circuit
will be in distinct sections of the file, and it should be easy to 
plot the results using PROBE.  If using \texttt{format=tecplot}, 
the results of each \texttt{.SAMPLING} simulation will be in a distinct
Tecplot zone. Finally, format=raw will place the results for each \texttt{.SAMPLING} 
simulation in a distinct ``plot'' region. 

Another similarity to \texttt{.STEP} is that \Xyce{} creates a second auxilliary file with the \texttt{*.res} 
suffix, which contains the parameter values.    This can be useful for seeing the exact sampling parameter values in a compact file.

Statistical outputs can also be produced (see section~\ref{statistical_outputs}) 
for \texttt{.SAMPLING} in the form of moments, but these can only be computed 
at the end of the calculation once all the samples are available.  These statistical outputs will not
be present in the \texttt{.PRINT} specified output file, since it is output on-the-fly.

Figure~\ref{Sampling_Netlist_2} shows an example file named \verb+clip.cir+, which when run will produce files
\verb+clip.cir.res+ and \verb+clip.cir.prn+.  The file \verb+clip.cir.res+
contains one line for each step, showing what parameter value was used
on that step.  \verb+clip.cir.prn+ is the familiar output format, but
the \verb+INDEX+ field recycles to zero each time a new step begins.
As the default behavior distinguishes each step's output only by recycling 
the \verb+INDEX+ field to zero, it can be beneficial to add the sampling
parameters to the \verb+.PRINT+ line.   For the default file format 
(\texttt{format=std}), \Xyce{} will not automatically include these sampling parameters,
so for plotting it is usually best to specify them by hand.


Note that for sampling calculations involving a really large number of sample 
points, the single output file can become prohibitively large.  Be careful when 
using \verb|.PRINT| if the number of samples is large.   If the number is really 
large (thousands) consider excluding any \verb|.PRINT| output statements and 
just rely on statistical outputs, described next in section~\ref{statistical_outputs}.
Similarly, the generation of the measure output can be suppressed with 
\texttt{.OPTIONS MEASURE MEASPRINT}.

\subsubsection{\texttt{.EMBEDDEDSAMPLING} output files}
\index{output!\texttt{.EMBEDDEDSAMPLING}}

As \texttt{.EMBEDDEDSAMPLING} computes all the sample points simultaneously, outputs for all 
sample points are also available simulataneously.  For transient simulations, this means that 
all samples are available at every time step, and all statistical outputs are available 
at every time step.  This is very useful if the interest is in seeing statistical moments 
with respect to time   

The output command used to produce this kind of output is \texttt{.PRINT ES}, where \texttt{ES} 
stands for  ``embedded sampling''.
The \texttt{.PRINT ES} command will automatically output several moments of the 
specified PCE outputs without any additional arguments on the \texttt{.PRINT ES} line.  
Also, the \texttt{.PRINT ES} statement will result in a file that has ``ES'' as part of the file name suffix.
Otherwise, \texttt{.PRINT ES} behaves very similarly to any other type of \texttt{.PRINT} 
command, and supports all of the same file formats and options.  For a full description see the
\Xyce{} Reference Guide.

\subsection{Statistical Outputs}
\label{statistical_outputs}
When performing uncertainty quantification analysis, the outputs of interest are 
often statistical moments of different circuit metrics.   \Xyce{} will compute these at the
end of a sampling calculation upon request.  The requested statistical outputs are 
specified on the \texttt{.options SAMPLING} line in the netlist.  For an example 
see the \texttt{OUTPUTS} and \texttt{MEASURES} fields in section~\ref{options_samples_statement}.  
Both \texttt{OUTPUTS} and \texttt{MEASURES} are specified as comma-separated 
lists.  The list of \texttt{OUTPUTS} must contain valid expressions that can be 
processed by the \Xyce{} expression library.  The list of \texttt{MEASURES} must 
consist valid \texttt{.MEASURE} statement names, which are present elsewhere in 
the netlist.

The example netlist given by figure~\ref{Sampling_Netlist_1} will produce the result 
given by figure~\ref{Sampling_Netlist_1_output}.  The example netlist given 
by figure~\ref{Sampling_Netlist_2} will produce the result given by 
figure~\ref{Sampling_Netlist_2_output}.  In both examples, the number of samples 
is relatively small, so the exact quantities computed will vary somewhat from run to run.
\begin{figure}[htbp]
  \fontsize{9pt}{10pt}
\begin{centering}
\shadowbox{
\begin{minipage}{0.8\textwidth}
\begin{vquote}
MC sampling mean of {R1:R} = 1009.58
MC sampling stddev of {R1:R} = 100.131
MC sampling variance of {R1:R} = 10026.3
MC sampling skew of {R1:R} = -0.0208347
MC sampling kurtosis of {R1:R} = 2.79565
MC sampling max of {R1:R} = 1308.03
MC sampling min of {R1:R} = 656.039

MC sampling mean of maxSine = 874.063
MC sampling stddev of maxSine = 10.9115
MC sampling variance of maxSine = 119.061
MC sampling skew of maxSine = 0.08628
MC sampling kurtosis of maxSine = 2.83029
MC sampling max of maxSine = 914.274
MC sampling min of maxSine = 842.547
\end{vquote}
\end{minipage}
}
\caption{Typical statistical outputs for figure~\ref{Sampling_Netlist_1}}
\label{Sampling_Netlist_1_output}
\end{centering}
\end{figure}

\begin{figure}[htbp]
  \fontsize{9pt}{10pt}
\begin{centering}
\shadowbox{
\begin{minipage}{0.8\textwidth}
\begin{vquote}
LHS sampling mean of {R4:R} = 9198
LHS sampling stddev of {R4:R} = 4293.89
LHS sampling variance of {R4:R} = 1.84375e+07
LHS sampling skew of {R4:R} = -0.0939244
LHS sampling kurtosis of {R4:R} = 1.39595
LHS sampling max of {R4:R} = 14996
LHS sampling min of {R4:R} = 3315.6

LHS sampling mean of {D1N3940:IS} = 4.2571e-10
LHS sampling stddev of {D1N3940:IS} = 9.10744e-11
LHS sampling variance of {D1N3940:IS} = 8.29455e-21
LHS sampling skew of {D1N3940:IS} = 0.0486504
LHS sampling kurtosis of {D1N3940:IS} = 2.05108
LHS sampling max of {D1N3940:IS} = 5.78543e-10
LHS sampling min of {D1N3940:IS} = 2.85753e-10

LHS sampling mean of maxSine = 4.46494
LHS sampling stddev of maxSine = 0.0901329
LHS sampling variance of maxSine = 0.00812394
LHS sampling skew of maxSine = -0.682188
LHS sampling kurtosis of maxSine = 2.07339
LHS sampling max of maxSine = 4.5536
LHS sampling min of maxSine = 4.28396
\end{vquote}
\end{minipage}
}
\caption{Typical statistical outputs for figure~\ref{Sampling_Netlist_2}}
\label{Sampling_Netlist_2_output}
\end{centering}
\end{figure}

\subsection{Random Sampling and Running in Parallel}

It is well known that sampling is a good opportunity to apply parallelism, as zero communication is required between sample points.
However, it has only been applied to a limited degree to random sampling in \Xyce{}.  
This is mostly because sampling methods 
are a relatively new feature to \Xyce{}. To get the full benefit of parallel sampling, where each 
sample can be launched and executed on a different unique processor, it is
recommended that \Xyce{} users instead use the Dakota library~\cite{DakotaTheoMan},~\cite{DakotaUsersMan}.
However, despite the limitations of \Xyce{} sampling in parallel, note that:
\begin{XyceItemize}
\item Sampling in \Xyce{} does have \emph{some} available parallelism, just not applied on a sample basis.
\item The cost of sampling can be significantly mitigated using non-intrusive Polynomial Chaos 
  Expansion (PCE) methods, described in section~\ref{PCE_Analysis}
\item \Xyce{} sampling is performed entirely internal to the \Xyce{} binary.  So unlike the black-box form of Dakota, it does not involve any file I/O. Also it does not require any scripting, which can be error-prone.    Finally, the netlist parsing and setup only happens once, rather than for every sample point, and this mitigates some of the computational cost.
\end{XyceItemize}

\subsubsection{\texttt{.SAMPLING} in Parallel}
The \texttt{.SAMPLING} analysis will execute the underlying analysis sequentially, once 
for each sample point.    If using parallel \Xyce{}, then each of these sequential 
calculations will have the same parallel methods applied to them as would be 
applied to a single forward calculation.    For example, if running a calculation
of 100 samples on a large parallel transient calculation, \Xyce{} would perform 
the first transient simulation in parallel, followed by the second transient 
simulation in parallel, etc, until all 100 samples were complete.
This means that if the underlying analysis is 
large enough to get a benefit from parallel, then the \texttt{.SAMPLING} 
analysis will get this same benefit.  The parallel scaling for the underlying 
analysis is unlikely to be perfect, as there are communication costs involved 
in the forward calculation.  However, one of the main parallel bottlenecks, 
the parsing and setup phase, is substantially reduced as it only happens one time.

\subsubsection{\texttt{.EMBEDDEDSAMPLING} in Parallel}
The \texttt{.EMBEDDEDSAMPLING} algorithm sets up a large block matrix system, 
where each block represents a different sample point.  This means that for each 
step of the calcalculation, all the samples are computed simultaneously.  This block system 
can be solved in parallel.   However, as this method is  less mature in \Xyce{}, 
more refinements to the algorithm are needed for this
\texttt{.EMBEDDEDSAMPLING} to fully benefit from running in parallel.

\clearpage

